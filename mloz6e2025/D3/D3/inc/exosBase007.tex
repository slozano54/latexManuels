\begin{exercice*}
   Pour réaliser 30 crêpes, il faut \Masse[g]{500} de farine, 6 œufs, \mbox{1 litre} de lait et \Masse[g]{50} de beurre.
   \begin{enumerate}
      \item Quelles sont les quantités nécessaires pour réaliser 15 crêpes ?
      \item Même question pour réaliser 75 crêpes.
      \item Combien de crêpes, au maximum, peut-on réaliser avec \Masse[g]{400} de farine, 4 œufs, \Capa[mL]{400} de lait et \Masse[g]{40} de beurre ?
   \end{enumerate}
\end{exercice*}
\begin{corrige}
   \begin{enumerate}
      \item Pour 15 crêpes, il faut \Masse[g]{250} de farine, 3 œufs, \Capa[mL]{500} de lait et \Masse[g]{25} de beurre.
      \item Pour 75 crêpes, il faut \Masse[g]{1\,250} de farine, 15 œufs, \Capa[L]{2,5} de lait et \Masse[g]{125} de beurre.
      \item On peut réaliser au maximum 24 crêpes avec \Masse[g]{400} de farine, 4 œufs, \Capa[mL]{400} de lait et \Masse[g]{40} de beurre.
   \end{enumerate}
\end{corrige}
