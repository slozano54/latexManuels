\begin{exercice*}
   Pour \Prix{4,25}, j'ai acheté cinq baguettes de pain. Pour \Prix{5,95}, j'aurais eu sept baguettes. Le prix payé est proportionnel au nombre de baguettes. \\
Sans calculer le prix d'une baguette, calculer le prix de :
   \begin{multicols}{2}
   \begin{enumerate}
      \item douze baguettes.
      \item deux baguettes.
      \item trois baguettes.
      \item quinze baguettes.
   \end{enumerate}
   \end{multicols}
\end{exercice*}
\begin{corrige}
   \begin{enumerate}
      \item Le prix de douze baguettes est \Prix{10,20}.
      \item Le prix de deux baguettes est \Prix{1,70}.
      \item Le prix de trois baguettes est \Prix{2,55}.
      \item Le prix de quinze baguettes est \Prix{12,75}.
   \end{enumerate}
\end{corrige}
