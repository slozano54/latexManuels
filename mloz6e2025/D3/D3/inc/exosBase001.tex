\begin{exercice*}
   Résoudre des problèmes quand c'est possible.
   \begin{enumerate}
      \item Une moto consomme en moyenne 4 litres d'essence pour 100 kilomètres. \\
         Quelle est sa consommation pour 350 kilomètres ?
      \item Jane a 11 ans et son père 35 ans. \\
         Quand Jane aura 33 ans, quel sera l'âge de son père ?
      \item Théo pèse \Masse[kg]{32} à 10 ans.
         Combien pèsera-t-il à 20 ans ?
      \item Le prix d'un kilogramme de pommes est \Prix{1,50}. \\
         Quel est le prix de 5 kilogrammes de pommes ?
      \item Un robinet remplit 8 seaux de 10 litres en deux minutes. \\
         Quelle est la quantité d'eau écoulée en une heure ?
      \item Un ticket de bus coûte \Prix{1,20} et un carnet de 10 tickets vaut \Prix{11}.
         Quel est le prix minimum pour acheter exactement 32 tickets ?
   \end{enumerate}
\end{exercice*}
\begin{corrige}
   \begin{enumerate}
      \item Pour 350 kilomètres, la moto consomme 14 litres d'essence.
      \item Quand Jane aura 33 ans, son père aura 57 ans.
      \item À 20 ans, Théo pèsera \Masse[kg]{64}.
      \item Le prix de 5 kilogrammes de pommes est \Prix{7,50}.
      \item En une heure, le robinet remplit 2400 litres d'eau.
      \item Le prix minimum pour acheter exactement 32 tickets est \Prix{35,20}.
   \end{enumerate}
\end{corrige}