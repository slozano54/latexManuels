\vspace*{-5mm}
\begin{changemargin}{-10mm}{-10mm}
%pre-001
\begin{prerequis}[Connaisances \emoji{red-heart} et compétences \emoji{diamond-suit} du cycle 3]    
   \begin{itemize}        
       \item[\emoji{red-heart}] Vocabulaire associé à ces objets et à leurs propriétés : côté, sommet, angle, hauteur.
       \columnbreak
       \item[\emoji{diamond-suit}] Reconnaître, nommer, décrire des triangles, dont les triangles particuliers (triangle rectangle, triangle isocèle, triangle équilatéral).       
   \end{itemize}
\end{prerequis}
\begin{debat}[Proportions et feuilles de papier] 
   Les formats de papier : A0, A1, A2, A3, A4, A5\dots{} ne sont pas dûs au hasard : ils ont été instaurés de manière à ce que les proportions de la feuille soient conservées lorsqu'on la coupe en deux. Le rapport entre la longueur est la largeur est de $\sqrt2$ (racine de 2) soit environ 1,414. \\
   C'est pourquoi une feuille de papier au format A4 a des dimensions de \ucm{21} par \ucm{29,7} car $21\times\sqrt2 \approx29,7$.
   \begin{center}
      \begin{Geometrie}[CoinHD={(15u,5u)}]
         pair A[];
         pair legendes[];
         numeric Longueur,Largeur;
         Longueur:=5.94;
         Largeur:=4.2;
         % format A0
         A0=u*(5,0.5);
         A1-A0=u*(Longueur,0);
         A2-A1=u*(0,Largeur);
         A3-A2=u*(-Longueur,0);         
         % format A1
         A4=iso(A2,A3);
         A5=iso(A0,A1);         
         % format A2
         A6=iso(A4,A5);
         A7=iso(A1,A2);
         % format A3
         A8=iso(A6,A7);
         A9=iso(A2,A4);
         %format A4
         A10=iso(A7,A2);
         A11=iso(A8,A9);
         %formatA5
         A12=iso(A10,A11);
         A13=iso(A9,A2);
         legendes[0]=iso(A0,A3);
         label.lft(TEX("\textbf{A0}"),legendes0 shifted (-u,0));
         drawarrow legendes0 shifted (-u,0) -- legendes0 shifted (-0.1u,0);
         % rempli polygone
         fill polygone(A0,A5,A4,A3) withcolor Crimson;
         fill polygone(A5,A1,A7,A6) withcolor 0.8Crimson;
         fill polygone(A8,A9,A4,A6) withcolor 0.6Crimson;
         fill polygone(A8,A7,A10,A11) withcolor LightCoral;
         fill polygone(A11,A12,A13,A9) withcolor 0.8LightCoral;
         label(TEX("\textbf{A1}"),iso(A3,A5)) withcolor white;
         label(TEX("\textbf{A2}"),iso(A6,A1)) withcolor white;
         label(TEX("\textbf{A3}"),iso(A4,A8)) withcolor white;
         label(TEX("\textbf{A4}"),iso(A11,A7)) withcolor white;
         label(TEX("\textbf{A5}"),iso(A9,A12)) withcolor white;
         trace polygone(A0,A1,A2,A3) withpen pencircle scaled 1.2bp;
         draw A4--A5 withpen pencircle scaled 1.2bp;
         draw A6--A7 withpen pencircle scaled 1.2bp;
         draw A8--A9 withpen pencircle scaled 1.2bp;
         draw A10--A11 withpen pencircle scaled 1.2bp;
         draw A12--A13 withpen pencircle scaled 1.2bp;
      \end{Geometrie}
   \end{center}
   \bigskip
   \begin{cadre}[B2][F4]
      \begin{center}
         \hrefVideo{https://www.youtube.com/watch?v=VsvplYbMEzc}{\bf Dimensions idéales d'un terrain de foot}, chaîne {\it Micmaths} de {\it Mickaël Launay}
      \end{center}
   \end{cadre}   
\end{debat}
\end{changemargin}