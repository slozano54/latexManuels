\begin{changemargin}{-10mm}{-15mm}
    \section{Se déplacer}
        \begin{methode}[Langages de déplacement]
            Pour se déplacer dans le plan, il existe principalement deux langages de déplacement :
            \begin{itemize}
                \item le langage {\bf absolu} composé des mots de vocabulaire du type : \og haut \fg{}, \og bas \fg{}, \og droite \fg{} et \og gauche \fg. Le déplacement se fait comme si on se plaçait en vue du dessus ;
                \item le langage {\bf relatif} composé des mots de vocabulaire du type : \og avancer \fg{}, \og tourner à droite \fg{} et \og tourner à gauche \fg. C'est ici le point de vue de l'observateur qui est adopté.
            \end{itemize}
            \exercice
            \begin{center}
            \psset{unit=0.7}
            \begin{pspicture}(0,-1)(5,4)
                \psgrid[subgriddiv=1,gridlabels=0mm](0,-1)(5,4)
                \psset{linecolor=A1,arrowsize=3mm,linewidth=0.5mm}
                \psdots(0.5,0.5)(3.5,2.5)     
                \psline{->}(0.5,0.5)(2.5,0.5)(2.5,2.5)(3.5,2.5)
            \end{pspicture}
            \end{center}
            \correction
            \begin{minipage}{4cm}
                Avec le langage absolu : \\
                \og droite \\
                droite \\
                haut \\
                haut \\
                droite \fg
            \end{minipage}
            \qquad
            \begin{minipage}{4cm}   
                Avec le langage relatif : \\
                \og avancer \\
                avancer \\
                tourner à gauche \\
                avancer \\
                avancer \\
                tourner à droite \\
                avancer \fg
            \end{minipage}
        \end{methode}
\end{changemargin}
