\begin{changemargin}{-15mm}{-15mm}
    \begin{activite}[Un robot nommé Hercule]        
        {\bf Objectifs :} se repérer, décrire et exécuter des déplacements sur un plan.
        \partie[texte n\degre3 des 12 travaux d'Hercule. Capturer le sanglier d'Erymanthe]
            \ \\
            \hspace*{1cm}
            \begin{minipage}{14cm}
                Le sanglier d’Erymanthe était une bête sauvage que personne n’osait approcher et qui faisait d’énormes ravages dans les cultures des paysans. Ramener ce sanglier vivant était une épreuve impossible pour un être humain, mais pas pour un demi-dieu\dots \\
                
                Les paysans du mont Erymanthe implorèrent Eurysthée de faire quelque chose contre l’énorme sanglier qui dévastait leurs cultures, dévorait leurs troupeaux et anéantissait leurs récoltes. Eurysthée ordonna à Hercule de capturer ce sanglier et de le lui amener vivant. Il espérait rendre ainsi Hercule ridicule aux yeux du peuple qui l’admirait un peu trop au goût du souverain. \\
                
                Il fallut plusieurs jours à Hercule pour rejoindre la rivière Erymanthe. Il en remonta le cours jusqu’au sommet de la montagne où les paysans s’agenouillèrent devant Hercule en l’implorant de les délivrer de ce monstre qui ravage leurs cultures. Il fut facile pour Hercule de pister le sanglier car il était si lourd que ses traces étaient fortement marquées sur le sol. De plus les arbres étaient lacérés et la terre retournée partout où la bête était passée. \\
                
                Il finit par se trouver nez à nez avec l’animal. Mais ne pouvant utiliser ses armes de peur de blesser ou de tuer l’animal, Hercule le vit s’enfuir. Il passa ensuite plusieurs jours à explorer le mont Erymanthe et à étudier les habitudes du sanglier. Après plusieurs tentatives infructueuses pour le capturer, Hercule changea sa stratégie. Il passa plusieurs semaines à creuser des fossés, bouger des pierres, créer des impasses et encore agrandir des sentiers sur le mont Erymanthe. \\
                
                Un matin d’hiver, la neige se mit à tomber recouvrant le mont. Hercule repéra aisément les traces du sanglier dans la neige et se mit à le poursuivre. À la fin de la journée, comme Hercule l’espérait, le sanglier s’engagea sur un sentier qu’il avait modifié débouchant sur un petit ravin. Le sanglier ne pouvait plus lui échapper. \\
                
                \begin{minipage}{8cm}
                La bête tomba dans le ravin mais sa chute fut amortie par la neige. Il fut juste assommé. Hercule n’eut plus qu’à ficeler les pattes de l’animal et à le hisser sur son dos afin de l’amener à son cousin Eurysthée.
                \end{minipage}
                \qquad
                \begin{minipage}{5cm}
                \includegraphics[width=5cm]{\currentpath/images/capture_sanglier} \\ [2mm]
                \end{minipage}
            \end{minipage}

            \hfill{\footnotesize \href{https://www.les-12-travaux-hercule.fr/les-12-travaux/le-sanglier-derymanthe/}{https://www.les-12-travaux-hercule.fr/les-12-travaux/le-sanglier-derymanthe/}}
       \clearpage       
    \rotatebox{90}{%    
        \begin{minipage}{1.25\linewidth}
            Nom/Prénom/Classe : \dotfill\\    
            
            \vspace*{10mm}
            \begin{minipage}{0.06\linewidth}
                \includegraphics[width=1.2cm]{\currentpath/images/Hercule}
            \end{minipage}
            \begin{minipage}{0.84\linewidth}
            \begin{center}
            {\huge Un robot nommé Hercule} \\ [5mm]
                Imagine un code (des signes) permettant de créer un programme pour déplacer Hercule vers le sanglier d’Erymanthe.
            \end{center}
            \end{minipage}
            \begin{minipage}{0.06\linewidth}
                \includegraphics[width=2cm]{\currentpath/images/sanglier}
            \end{minipage}
            
            \vspace*{10mm}
            
            \begin{tabular}{|p{0.08\linewidth}|*{5}{p{0.16\linewidth}|}}
                \hline
                Instruction & & & & & \\
                \hline
                \vspace*{0.7cm}
                Code \newline
                \vspace*{1cm}
                & & & & & \\
                \hline
            \end{tabular}
            
            \vspace*{20mm}
            
            \begin{center}
                Le code retenu collectivement.
            \end{center}
            
            \vspace*{10mm}
            
            \begin{tabular}{|p{0.08\linewidth}|*{5}{>{\centering\arraybackslash}p{0.16\linewidth}|}}
                \hline
                Instruction & & & & & \\
                \hline
                \vspace*{0.7cm}
                Code \newline
                \vspace*{1cm}
                & & & & & \\
                \hline
            \end{tabular}
        \end{minipage}
    }
    \end{activite}
\end{changemargin}     