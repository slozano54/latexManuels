\begin{exercice*}
    La fourmi de Langton\footnote{du nom de son inventeur le scientifique américain Christopher Langton. Ce système a été inventé vers la fin des années 1980.} est un automate qui se déplace dans un quadrillage suivant les règles suivantes :
    \begin{itemize}
       \item au départ, toutes les cases sont de la même couleur, ici blanches ;
       \item si la fourmi est sur une case blanche, elle tourne de \ang{90} vers la droite, change la couleur de la case en noir et avance d'une case ;
       \item si la fourmi est sur une case noire, elle tourne de \ang{90} vers la gauche, change la couleur de la case en blanc et avance d'une case.
    \end{itemize}
    
    Compléter dans les quadrillages ci-dessous les 15 premières étapes du déplacement de la fourni. \\
    Quelques étapes intermédiaires sont données afin de pouvoir vérifier que l'algorithme est correctement suivi.
    
    \begin{center}
    \psset{unit=0.4,subgriddiv=1,gridlabels=0mm,gridcolor=gray}
    \small
    \newcommand{\fourmi}[3]{\rput{#3}(#1,#2){\psdot[linecolor=red,dotstyle=triangle*,linewidth=1mm](0,0)}}
    \begin{pspicture}(0,-0.7)(7,7)
       \psgrid(0,0)(7,7)
       \fourmi{3.5}{3.5}{0}
       \rput(3.5,-0.5){étape 0}
    \end{pspicture}
    \quad
    \begin{pspicture}(0,-0.7)(7,7)
       \psframe[fillstyle=solid,fillcolor=darkgray](3,3)(4,4)
       \psgrid(0,0)(7,7)
       \fourmi{4.5}{3.5}{-90}
       \rput(3.5,-0.5){étape 1}
    \end{pspicture}
    \quad
    \begin{pspicture}(0,-0.7)(7,7)  
       \psframe[fillstyle=solid,fillcolor=darkgray](3,3)(5,4)
       \psgrid(0,0)(7,7)
       \fourmi{4.5}{2.5}{180}
       \rput(3.5,-0.5){étape 2}
    \end{pspicture}
    \quad
    \begin{pspicture}(0,-0.7)(7,7)
       \psgrid(0,0)(7,7)
       \rput(3.5,-0.5){étape 3}
    \end{pspicture}
    
    \bigskip
    
    \begin{pspicture}(0,-0.7)(7,7)
       \psframe[fillstyle=solid,fillcolor=darkgray](3,2)(5,4)
       \psgrid(0,0)(7,7)
       \fourmi{3.5}{3.5}{0}
       \rput(3.5,-0.5){étape 4}
    \end{pspicture}
    \quad
    \begin{pspicture}(0,-0.7)(7,7)
       \psgrid(0,0)(7,7)
       \rput(3.5,-0.5){étape 5}
    \end{pspicture}
    \quad
    \begin{pspicture}(0,-0.7)(7,7)
       \psgrid(0,0)(7,7)
       \rput(3.5,-0.5){étape 6}
    \end{pspicture}
    \quad
    \begin{pspicture}(0,-0.7)(7,7)
       \psframe[fillstyle=solid,fillcolor=darkgray](2,3)(3,5)
       \pspolygon[fillstyle=solid,fillcolor=darkgray](3,2)(5,2)(5,4)(4,4)(4,3)(3,3)
       \psgrid(0,0)(7,7)
       \fourmi{3.5}{4.5}{-90}
       \rput(3.5,-0.5){étape 7}
    \end{pspicture}
    
    \bigskip
    
    \begin{pspicture}(0,-0.7)(7,7)
       \psgrid(0,0)(7,7)
       \rput(3.5,-0.5){étape 8}
    \end{pspicture}
    \quad
    \begin{pspicture}(0,-0.7)(7,7)
       \psgrid(0,0)(7,7)
       \rput(3.5,-0.5){étape 9}
    \end{pspicture}
    \quad
    \begin{pspicture}(0,-0.7)(7,7)
       \psgrid(0,0)(7,7)
       \rput(3.5,-0.5){étape 10}
    \end{pspicture}
    \quad
    \begin{pspicture}(0,-0.7)(7,7)
       \pspolygon[fillstyle=solid,fillcolor=darkgray](2,2)(5,2)(5,4)(4,4)(4,5)(2,5)(2,4)(3,4)(3,3)(2,3)
       \psgrid(0,0)(7,7)
       \rput(3.5,-0.5){étape 11}
       \fourmi{1.5}{2.5}{90}
    \end{pspicture}
    
    \bigskip
    
    \begin{pspicture}(0,0)(7,7)
       \psgrid(0,0)(7,7)
       \rput(3.5,-0.5){étape 12}
    \end{pspicture}
    \quad
    \begin{pspicture}(0,0)(7,7)
       \psgrid(0,0)(7,7)
       \rput(3.5,-0.5){étape 13}
    \end{pspicture}
    \quad
    \begin{pspicture}(0,0)(7,7)
       \psgrid(0,0)(7,7)
       \rput(3.5,-0.5){étape 14}
    \end{pspicture}
    \quad
    \begin{pspicture}(0,0)(7,7)
       \psgrid(0,0)(7,7)
       \rput(3.5,-0.5){étape 15}
    \end{pspicture} 
    \end{center}
\end{exercice*}
\begin{corrige}
    Pas de correction
\end{corrige} 