\begin{exercice*} %1
   On considère les catégories de solides suivants que l'on peut trouver dans la vie courante :
   \begin{enumerate}
      \item[\textbf{A.~}] Différentes boites parallélépipédiques et cubiques.
      \item[\textbf{B.~}] Des prismes droits à bases triangulaires (Toblerone) ou octogonales.
      \item[\textbf{C.~}] Une pyramide à base carrée tronquée (boite de fromage de chèvre).
      \item[\textbf{D.~}] Des cylindres (boîtes de conserve diverses, boîte de camembert, rouleau de papier d'aluminium).
      \item[\textbf{E.~}] Des cônes.
      \item[\textbf{F.~}] Des boules (balles, ballons).
      \item[\textbf{G.~}] D'autres emballages (formes ovales, anneaux, boîtes en forme de c\oe ur\dots).
   \end{enumerate}
   \begin{enumerate}
      \item Classer ces sept catégories selon leur caractère polyédrique ou non
      \item Pourquoi, en cours de maths, faut-il s'intéresser uniquement aux formes des boites d'emballages ?
      \item Quelle est la particularité du rouleau de papier d'aluminium ?
      \item Des classes ont proposé les classements suivants : \smallskip
      \begin{center}
         \renewcommand{\arraystretch}{1.2}
         \begin{ltableau}{0.7\linewidth}{2}
            \hline
            5\up{e}A & 5\up{e}B \\
            \hline
            A, B et C & A, B et D \\
            F & C et E \\
            G & F et G \\
            \hline
         \end{ltableau}
      \end{center}
      \medskip
      Quels ont pu être leurs critères de classement ?
      \item On propose la définition suivante pour déterminer un polyèdre : \og solide qui ne roule pas \fg{} et pour un non polyèdre : \og solide qui roule \fg. \\
      Ces définition vous paraissent-elles pertinentes ?
   \end{enumerate}
\end{exercice*}
  
\begin{corrige}
   \begin{enumerate}
      \item On obtient le classement suivant :
      \begin{itemize}
         \item \textcolor{red}{ les polyèdres : A, B, C} ;
         \item \textcolor{red}{ les non polyèdres : D, E, F, G}. 
      \end{itemize}
      \item On pourrait proposer un classement du type \og sucré, salé \fg{}, ou \og ça se mange - ça ne se mange pas \fg{} ou encore proposer un classement par couleur, par taille\dots{} ce qui n'est pas l'objectif attendu.
      \item Le rouleau de papier d'aluminium est un \textcolor{red}{ solide non fermé}. 
      \item 
      \begin{itemize}
         \item Pour les 5\up{e}A, il semble que les élèves ont \og repéré \fg{} les polyèdres A, B et C. \\
           Les boules forment une deuxième catégorie. \\
           Les autres emballages une troisième. \\
           On remarque que les cylindres et les cônes n'apparaissent pas dans le classement, sûrement en raison de leur forme mélangeant des faces planes comme pour les polyèdres A, B, C et des surfaces non planes comme pour les boules F.
         \item Pour les 5\up{e}B, les élèves ont classé ensemble les prismes et les cylindres, c'est-à-dire les solides ayant des faces opposées parallèles. \\
         Puis ils ont mis ensemble les solides \og pointus \fg{} comme les pyramides et les cônes. \\
         Enfin, les boules et autres emballages, constituent la catégorie des formes \og arrondies \fg{}, les cylindres et les cônes ayant déjà été classés. 
      \end{itemize}
      \item Cette définition n'est pas pertinente : par exemple un cône peut rouler ou pas selon si on le pose sur sa base ou sur le côté. \\
      Un polyèdre régulier avec de multiples faces donne l'impression qu'il roule lorsqu'on le lance.
   \end{enumerate}
\end{corrige}



