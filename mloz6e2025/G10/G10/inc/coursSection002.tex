\begin{changemargin}{-10mm}{-10mm}
    \section{Représenter et construire des solides}

    \subsection{Perspective cavalière}
    \vspace*{-3mm}
    \begin{definition}
       Un \textbf{polyèdre} est un solide de l'espace délimité par un nombre fini de polygones, appelés les faces du polyèdre.
    \end{definition}
    \begin{remarque}    
        Pour représenter un solide de l'espace, on utilise généralement la {\bf perspective cavalière} : technique de dessin permettant de représenter un solide sur une surface à deux dimensions en respectant le parallélisme.
    \end{remarque}
    \vspace*{-6mm}   
    \begin{exemple}
       Représentation d'un parallélépipède rectangle en perspective cavalière :
       \begin{enumerate}
          \item on trace en vraie grandeur la face de devant ;
          \item on trace les arêtes visibles des faces latérales parallèles et de même longueur : ce sont les fuyantes, plus courtes que leur mesure réelle ;
          \item on trace les arêtes visibles de la face arrière ;
          \item on trace les arêtes cachées en pointillés.
       \end{enumerate}
       \correction
          {\psset{unit=0.6}
          \begin{pspicture}(-3,-0.5)(5,3)
             \pspolygon(0,0)(4,0)(5,1)(5,3)(1,3)(0,2)
             \psline(0,2)(4,2)(4,0)
             \psline(4,2)(5,3)
             \psline[linestyle=dashed](0,0)(1,1)(5,1)
             \psline[linestyle=dashed](1,1)(1,3)
             \rput(0,-0.3){A}
             \rput(4,-0.3){B}
             \rput(5.1,0.7){C}
             \rput(1.2,0.7){D}
             \rput(0,2.3){E}
             \rput(4,2.3){F}
             \rput(5,3.3){G}
             \rput(1,3.3){H}
          \end{pspicture}} \\
          Ce parallélépipède rectangle possède :
          \begin{itemize}
             \item $8$ \textbf{sommets} : $A$, $B$, $C$, $D$, $E$, $F$, $G$ et $H$ ;
             \item $12$ \textbf{arêtes} : $[AB]$, $[BC]$, $[AE]$, $[BF]$, $[EF]$, $[FG]$, $[CG]$, $[EH]$ et $[GH]$ apparentes, et $[AD]$, $[DC]$ et $[DH]$ cachées ;
             \item 6 \textbf{faces} : $ABFE$ est la face de devant, $CDHG$ celle de derrière, $ABCD$ la face du dessous, $EFGH$ celle du dessus, $BCGF$ la face de droite, et $ADHE$ celle de gauche.
          \end{itemize}
    \end{exemple}
    
    \subsection{Patron}
    \vspace*{-3mm}
    \begin{definition}
       Le \textbf{patron} d'un solide est une surface plane d'un seul tenant qui, par pliage, permet de reconstituer le solide sans recouvrement de ses faces.
    \end{definition}    
    On \og déplie \fg{} le parallélépipède rectangle pour en obtenir un de ses patrons.
    \vspace*{-8mm}
    \begin{exemple}        
       \qquad \includegraphics[scale=0.4]{\currentpath/images/Cours_pave_deplie}\correction
    {\psset{unit=0.7}
    \begin{pspicture}(-2,-0.5)(6,4.5)
       \psline(1,0)(1,1)(0,1)(0,3.5)(1,3.5)(1,4.5)(3,4.5)(3,3.5)(4,3.5)(6,3.5)(6,1)(4,1)(3,1)(3,0)(1,0)
       \psframe(1,1)(3,3.5)
       \psline(4,3.5)(4,1)
       \rput(0.5,1){\textcolor{B2}{x}}
       \rput(1,0.5){\textcolor{B2}{x}}
       \rput(0.5,3.5){\textcolor{B2}{x}}
       \rput(1,4){\textcolor{B2}{x}}
       \rput(3.5,1){\textcolor{B2}{x}}
       \rput(3,0.5){\textcolor{B2}{x}}
       \rput(3.5,3.5){\textcolor{B2}{x}}
       \rput(3,4){\textcolor{B2}{x}}
       \rput(0,2.25){\textcolor{A1}{o}}
       \rput(1,2.25){\textcolor{A1}{o}}
       \rput(3,2.25){\textcolor{A1}{o}}
       \rput(4,2.25){\textcolor{A1}{o}}
       \rput(6,2.25){\textcolor{A1}{o}}
       \rput(2,0){\textcolor{J1}{|\!\!|}}
       \rput(2,1){\textcolor{J1}{|\!\!|}}
       \rput(2,3.5){\textcolor{J1}{|\!\!|}}
       \rput(2,4.5){\textcolor{J1}{|\!\!|}}
       \rput(5,1){\textcolor{J1}{|\!\!|}}
       \rput(5,3.5){\textcolor{J1}{|\!\!|}}
    \end{pspicture}}
    \end{exemple}
    
    \begin{remarque}    
        Le patron d'un polyèdre n'est pas unique, il dépend de la manière dont on le déplie. \\
        On ne parle de patron que pour un polyèdre. On parle de développement pour un cylindre ou un cône.
    \end{remarque}    
\end{changemargin}
 
