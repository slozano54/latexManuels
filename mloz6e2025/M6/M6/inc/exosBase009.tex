\begin{exercice*}
   Yanis, un bébé de \Masse[kg]{8}, a de la fièvre. Pour la faire baisser, ses parents hésitent entre deux médicaments.
   Chaque médicament est donné à l'aide d'une pipette graduée selon le poids de l'enfant en \Masse[kg]{} :
   \begin{itemize}
      \item Paracétamol : 1 graduation de \Masse[kg]{1} correspond à \Capa[mL]{0.625} de sirop buvable.
      \item Ibuprofène : 1 graduation de  \Masse[kg]{1} correspond à \Capa[mL]{0.375} de sirop buvable.
   \end{itemize}
   La posologie préconisée  est de 4 prises par jour.
   \begin{enumerate}
      \item Déterminer le nombre de millilitres de sirop que contient une prise de paracétamol pour Yanis.
      \item Déterminer le nombre de millilitres de sirop que contient une prise d'ibuprofène pour Yanis.
      \item Déterminer la quantité de paracétamol nécessaire pour soigner Yanis pendant 4 jours.
      \item Déterminer la quantité d'ibuprofène nécessaire pour soigner Yanis pendant 4 jours.
      \item Déterminer le nombre de jours que l'on peut traiter Yanis au paracétamol avec un flacon plein de \Capa[mL]{100}.
   \end{enumerate}
\end{exercice*}