\begin{exercice*}
   Deux échelles la température sont principalement utilisées : l'échelle Celsius et l'échelle Fahrenheit. \\
   La valeur en \Temp[F]{} d'une température s'obtient en multipliant par \num{1.8} la température en \Temp{} et en ajoutant 32.
   \begin{enumerate}
      \item La température de la glace fondante correspond à \Temp{0}, déterminer la température en \Temp[F]{}.
      \item La température d'ébullition de l'eau correspond à \Temp{100}, déterminer la température en \Temp[F]{}.
      \item La météo prévoit une température de \Temp{25} pour demain, déterminer la température en \Temp[F]{}.
      \item Il fait actuellement \Temp[F]{40}, déterminer la température en \Temp{}.
   \end{enumerate}
\end{exercice*}