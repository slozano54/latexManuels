\begin{exercice*}
   L'infographie suivante donne le nombre de fois que des sortilèges de magie apparaissent dans les sept livres de la série {\it Harry Potter}.
   \begin{center}
      \includegraphics[width=7.5cm]{\currentpath/images/infographie_HP}

      \hfill {\footnotesize\it Harry Potter, les nombres d'or de la saga, Le Figaro.fr, 2017}
   \end{center}
   \begin{enumerate}
      \item Comment sont représentées les données dans cette infographie ?
      \item Combien y a-t-il eu de sortilèges donnés au total dans les sept livres ?
      \item Représenter les données dans un tableau en notant uniquement les sortilèges cités plus de 20 fois.
      \item Construire un diagramme en bâtons pour les valeurs de ce tableau.
   \end{enumerate}   
\end{exercice*}
\begin{corrige}
   \begin{enumerate}
      \item Les données sont représentées par des cercles de surface proportionnelle au nombre de sortilèges.
      \item \num{1023} sortilèges ont été donnés au total dans les sept livres (somme des valeurs des cercles).
      \item \par\hfill
         \begin{tabular}{|c|c|}
            \hline
            Sortilège & Nombre d'apparitions \\
            \hline
            Expelliarmus & 56 \\
            Stupefix & 45 \\
            Lumos & 36 \\
            Wingardium Leviosa & 30 \\
            Protego & 28 \\
            Accio & 25 \\
            Expecto Patronum & 22 \\
            \hline
         \end{tabular}
         \par\medskip
      \item [Diagramme en bâtons à construire par l'élève.]
   \end{enumerate}
\end{corrige}