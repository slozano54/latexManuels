
\begin{exercice*}
   Le tableau ci-dessous représente la répartition des médailles françaises aux Jeux olympiques (JO) d'été de 1896 à 2016 pour les dix sports ayant eu le plus de médailles.
   \begin{enumerate}
      \item Compléter le tableau.
      \item Comment est établi le classement des sports aux JO ?
   \end{enumerate}   
   \begin{center}
      \includegraphics[scale=0.7]{\currentpath/images/JO_trous}

      \hfill {\footnotesize\it Source : France aux Jeux olympiques, Wikipedia, 2019}
   \end{center}
\end{exercice*}
\begin{corrige}
   \begin{enumerate}
      \item \par\hfill
         \begin{tabular}{|c|c|c|c|c|}
            \hline
            Sport & Or & Argent & Bronze & Total \\
            \hline
            Athlétisme & 41 & 29 & 31 & 101 \\
            Cyclisme & 28 & 26 & 31 & 85 \\
            Escrime & 26 & 25 & 31 & 82 \\
            Judo & 20 & 14 & 22 & 56 \\
            Natation & 18 & 25 & 31 & 74 \\
            Haltérophilie & 17 & 10 & 7 & 34 \\
            Lutte & 15 & 19 & 21 & 55 \\
            Tir sportif & 14 & 14 & 11 & 39 \\
            Canoë-kayak & 13 & 15 & 16 & 44 \\
            Football & 12 & 4 & 1 & 17 \\
            \hline
         \end{tabular}
      \item Le classement des sports aux JO est établi en fonction du nombre de médailles d'or obtenues. En cas d'égalité, on regarde le nombre de médailles d'argent, puis de bronze.
   \end{enumerate}
\end{corrige}