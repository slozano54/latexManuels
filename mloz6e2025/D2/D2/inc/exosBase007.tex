
\begin{exercice*}
   Le tableau suivant donne l'évolution de la population de Montpellier depuis un siècle.
   \begin{center}
   {\small
   \renewcommand{\arraystretch}{1.6}
      \begin{tabular}{|p{1.5cm}|*{5}{>{\centering\arraybackslash}m{0.1\linewidth}|}}
      \hline
      Année & 1921 & 1931 & 1946 & 1962 & 1975 \\
      \hline
      Population & 81\,548 & 86\,924 & 93\,102 & 118\;864 & 191\,354 \\
      \hline
      Année & 1982 & 1999 & 2006 & 2011 & 2016 \\
      \hline
      Population & 197\,231 & 225\,392 & 251\,634 & 264\,538 & 281\,613 \\
      \hline
   \end{tabular}

   \hfill {\scriptsize\it Source : Montpellier, Wikipedia, 2019}}   
   \end{center}
   \begin{enumerate}
      \item Construire un diagramme cartésien représentant ce tableau. On prendra \ucm{1} pour 5 années en abscisses et \ucm{1} pour 20\,000 habitants en ordonnées.
      \item Que peut-on dire de cette évolution ?
   \end{enumerate}
   \vspace*{-100mm}   
\end{exercice*}
\begin{corrige}
   \begin{enumerate}
      \item [Diagramme cartésien à construire par l'élève.]
      \item La population de Montpellier a augmenté de façon continue depuis 1921. Cette augmentation est particulièrement marquée entre 1962 et 1975.
   \end{enumerate}
\end{corrige}