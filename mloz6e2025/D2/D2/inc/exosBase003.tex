
\begin{exercice*}
   Le diagramme ci-dessous représente la répartition des hommes et des femmes suivant leur indice de masse corporelle (I.M.C.).
   \begin{center}
      \includegraphics[scale=0.2]{\currentpath/images/surpoids}

      \hfill {\scriptsize\it Source : Santé publique française, Bulletin épidémiologique, 2016}
   \end{center}
   \begin{enumerate}
      \item Qu'est-ce que l'I.M.C. ?
      \item En moyenne, qui des hommes ou des femmes sont le plus affectés par le surpoids ?
      \item Quel est le pourcentage de femmes en obésité ?
      \item Quel est le pourcentage d'hommes en insuffisance pondérale ?
      \item Quelle est la catégorie la plus représentée chez les hommes ?
   \end{enumerate}
\end{exercice*}
\begin{corrige}
   \begin{enumerate}
      \item L'I.M.C. est un indicateur permettant d'évaluer la corpulence d'une personne en fonction de son poids et de sa taille.
      \item En moyenne, les hommes sont plus affectés par le surpoids que les femmes (35,3 % contre 25,6 %).
      \item Le pourcentage de femmes en obésité est de 16,9 %.
      \item Le pourcentage d'hommes en insuffisance pondérale est de 2,9 %.
      \item La catégorie la plus représentée chez les hommes est le surpoids avec 35,3 %.
   \end{enumerate}
\end{corrige}
