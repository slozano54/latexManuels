
\begin{exercice*}
   La Mascareignes est un ultra-trail se déroulant sur l'île de la Réunion.
   \begin{center}
      \includegraphics[width=8.5cm,height=4.5cm]{\currentpath/images/Mascareigne}
      
      \hfill {\footnotesize\it Source : La Réunion lé la, overblog, 2016}
   \end{center}
   D'après le diagramme cartésien ci-dessus représentant le dénivelé en fonction du lieu, répondre aux questions suivantes :
   \begin{enumerate}
      \item Que signifie \og PK14 \fg ?
      \item Quelle est la longueur de la course ?
      \item Où se situe l'altitude la plus élevée ? Quelle est cette altitude ?
      \item Combien y-a-t-il de km entre Grand-Sable et la Grande Chaloupe ?
      \item Quel est le dénivelé entre le début du Sentier Scout et Deux-bras ?
      \item À peu près combien de km parcourt-on au delà de 1\,000 m d'altitude ?
   \end{enumerate}
\end{exercice*}
\begin{corrige}
   \begin{enumerate}
      \item \og PK14 \fg{} signifie \og Point Kilométrique 14 \fg{}, c'est-à-dire que l'on se situe à 14 km du départ de la course.
      \item La longueur de la course est de 65 km.
      \item L'altitude la plus élevée se situe au Maïdo, à environ 2\,200 m.
      \item Il y a environ 15 km entre Grand-Sable et la Grande Chaloupe (de PK35 à PK50).
      \item Le dénivelé entre le début du Sentier Scout (environ 500 m) et Deux-bras (environ 1\,800 m) est d'environ 1\,300 m.
      \item On parcourt environ 20 km au delà de 1\,000 m d'altitude (de PK25 à PK45).
   \end{enumerate}
\end{corrige}
