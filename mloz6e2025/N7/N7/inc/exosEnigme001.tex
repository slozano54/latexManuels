% Les enigmes ne sont pas numérotées par défaut donc il faut ajouter manuellement la numérotation
% si on veut mettre plusieurs enigmes
% \refstepcounter{exercice}
% \numeroteEnigme
\begin{changemargin}{-10mm}{-10mm}
\begin{enigme}[Tableaux en fils tendus]
    Reproduire les deux tableaux 1 et 2 suivants de sorte que chaque figure ait une dimension de \ucm{10} par \ucm{10}.
    \begin{center}
       {\psset{unit=0.3}      
       \begin{pspicture}(-10,-10)(11,13)
          %\psgrid[gridlabels=0,subgriddiv=0,gridcolor=lightgray](-10,-10)(10,10)
          \rput(0,12){Tableau 1}
          \multido{\n=0+1,\i=10+-1}{11}{\psline(-10,\n)(-\i,10) %NOe
                                                          \psline(10,\n)(\i,10) %NEe
                                                          \psline(-10,-\n)(-\i,-10) %SOe
                                                          \psline(10,-\n)(\i,-10) %SEe
                                                          \psline(-\i,0)(0,\n) %NOi
                                                          \psline(\i,0)(0,\n) %NEi
                                                          \psline(\i,0)(0,-\n) %SEi
                                                          \psline(-\i,0)(0,-\n) %SEi
                                                         }
       \end{pspicture}}
       {\psset{unit=0.85,linecolor=PartieStatistique}
       \begin{pspicture}(-11,0)(0,10)
          %\psgrid[gridlabels=0,subgriddiv=0,gridcolor=lightgray](-10,-10)(10,10)
          \rput(-5,5){\textcolor{PartieStatistique}{détail du tableau 1}}
          \multido{\n=0+1,\i=10+-1}{11}{\psline(-10,\n)(-\i,10) %NOe
                                                          \psline(-\i,0)(0,\n) %NOi
                                                         }
       \end{pspicture}} \\ [10mm]
        {\psset{unit=0.55}
        \begin{pspicture}(-10,-9)(10,10)
        \psframe(-10,-10)(10,10)
           \rput(0,-8){Tableau 2}
           \multido{\n=0+1,\i=10+-1}{11}{\psline(-\i,-\i)(-\n,\n) %O
                                                       \psline(\i,\i)(\n,-\n) %E
                                                       \psline(-\i,\i)(\n,\n) %N
                                                       \psline(-\i,-\i)(\n,-\n) %S
                                                      }
       \end{pspicture}}
    \end{center}
\end{enigme}
\end{changemargin}
% Pour le corrigé, il faut décrémenter le compteur, sinon il est incrémenté deux fois
% \addtocounter{exercice}{-1}
% \begin{corrige}
%     \ldots
% \end{corrige}