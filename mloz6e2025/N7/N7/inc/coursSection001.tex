\section{Pourcentages}

\begin{definition}
   Le {\bf pourcentage} d'une quantité est le nombre qui aurait été proportionnellement obtenu si la quantité avait été de 100.
\end{definition}

\begin{propriete}
   Pour calculer le pourcentage $p\,\%$ d'une quantité, on multiplie la quantité par $\dfrac{p}{100}$.
\end{propriete}

\begin{exemple}
   Un jeu à 39 \euro{} est en promotion à $-$20\,\%. 
   \begin{itemize}
      \item Quel est le montant de la remise ?
      \item Quel est le prix du jeu après remise ?
   \end{itemize}
   \correction
   \ \\ [-8mm]
   \begin{itemize}
      \item Calcul de la remise : $\dfrac{20}{100}\times\ueuro{39}=\ueuro{7}$. \medskip
      \item Calcul du nouveau prix : $\ueuro{39}-\ueuro{7,8} =\ueuro{31,2}$.
   \end{itemize}
\end{exemple}

\bigskip

Pourcentages simples : \\
$\bullet$ 10\,\% d'une quantité correspond à un dixième de cette quantité. \\
$\bullet$ 25\,\% d'une quantité correspond à un quart de cette quantité. \\
$\bullet$ 50\,\% d'une quantité correspond à la moitié de cette quantité. \\
$\bullet$ 75\,\% d'une quantité correspond aux trois quarts de cette quantité. \\
$\bullet$ 100\,\% d'une quantité correspond à la totalité de la quantité. \\

\begin{center}
   \begin{pspicture}(-1,-1)(2,1)
      \pscircle(0,0){1}
      \pswedge[fillstyle=solid,fillcolor=B1](0,0){1}{0}{90}
      \rput(0.45,0.4){\white 25\,\%}
   \end{pspicture}
   \begin{pspicture}(-1,-1)(2,1)
      \pscircle(0,0){1}
      \pswedge[fillstyle=solid,fillcolor=B1!80](0,0){1}{0}{180}
      \rput(0,0.6){\white 50\,\%}
   \end{pspicture}
   \begin{pspicture}(-1,-1)(2,1)
      \pscircle(0,0){1}
      \pswedge[fillstyle=solid,fillcolor=B1!60](0,0){1}{0}{-90}
      \rput(-0.3,0.3){\white 75\,\%}
   \end{pspicture}
   \begin{pspicture}(-1,-1)(2,1)
      \pscircle[fillstyle=solid,fillcolor=B1!40](0,0){1}
      \rput(0,0){\white 100\,\%}
   \end{pspicture}
\end{center}