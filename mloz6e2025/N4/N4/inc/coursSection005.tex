\section{Premières équations}
\begin{exemple*1}
   Un(e) élève de la classe" pense à un nombre et y ajoute 17. Il(elle) trouve 63.
   À quel nombre "cet(te) élève" a-t-il(elle) pensé?
   \correction
   Faire un schéma avec des longueurs.
   \begin{center}
      \begin{pspicture}(0,0)(7,3.5)
         \psline[linewidth=1pt,linecolor=black]{-}(0,1.5)(7,1.5)
         \psline[linewidth=1pt,linecolor=black,showpoints=true]{-}(1,1.5)(4,1.5)(6.5,1.5)
         \psline[linewidth=1pt,linecolor=red]{<->}(1,0.5)(4,0.5)
         \psline[linewidth=1pt,linecolor=mygreen]{<->}(4,0.5)(6.5,0.5)
         \psline[linewidth=1pt,linecolor=blue]{<->}(1,2.5)(6.5,2.5)
         \rput(1,1){R}
         \rput(4,1){S}
         \rput(6.5,1){T}
         \rput(2.5,0){?}
         \rput(5.25,0){17}
         \rput(3.75,3){63}
      \end{pspicture}
   \end{center} 
   Pour trouver ce nombre il faut effectuer $63-17$
\end{exemple*1}

\begin{propriete}
   Si on veut trouver un terme inconnu dans une somme alors il faut effectuer une soustraction.
\end{propriete}

\begin{remarque}
   En mathématique pour désigner les nombres inconnus on utilise des lettres.
\end{remarque}

\begin{exemple*1}
   Un champ rectangulaire a un périmètre de \Lg[m]{18}. Sachant que sa longueur mesure 5m, déterminer sa largeur.
   \correction
   On peut écrire que $2\times \text{largeur} + 2\times 5 = 18$

   Donc $2\times \text{largeur} + 10 = 18$ d'où $2\times \text{largeur} = 18 - 10$ c'est à dire $2\times \text{largeur} = 8$

   Le double de la largeur vaut \Lg[m]{8} donc la largeur vaut \Lg[m]{4}.
\end{exemple*1}