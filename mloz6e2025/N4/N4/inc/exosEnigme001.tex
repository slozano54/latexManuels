% Les enigmes ne sont pas numérotées par défaut donc il faut ajouter manuellement la numérotation
% si on veut mettre plusieurs enigmes
% \refstepcounter{exercice}
% \numeroteEnigme
\begin{enigme}[Addition à l'abaque romain]
\partie[introduction]
   Pour additionner à l'abaque romain, on inscrit les deux nombres à l'abaque l'un au-dessus de l'autre puis, pour chaque rang, on dénombre les jetons en faisant éventuellement des échanges. \\ [3mm]
   {\bf Exemple :} calcul de $63+52$. Expliquer sous chaque tableau ce qui a été fait dans l'abaque.
   \begin{center}
      \begin{tabular}{>{\centering\arraybackslash}p{5cm}>{\centering\arraybackslash}p{5cm}>{\centering\arraybackslash}p{5cm}}
         \begin{pspicture}(0,0)(3,4)
            \multido{\n=0+1}{4}{\psline(\n,0)(\n,4)}
            \multido{\n=3+1}{2}{\psline(0,\n)(3,\n)}
            \rput(0.5,3.5){C}
            \rput(1.5,3.5){X}
            \rput(2.5,3.5){I}
            \psdots[linecolor=A1](1.3,2.7)(1.7,2.7)(1.3,2.4)(1.7,2.4)(1.3,2.1)(1.7,2.1) %X1
            \psdots[linecolor=A1](2.3,2.7)(2.5,2.4)(2.7,2.1) %I1
            \psdots[linecolor=B1](1.3,1.3)(1.7,1.3)(1.5,1)(1.3,0.7)(1.7,0.7) %X2
            \psdots[linecolor=B1](2.4,1.15)(2.6,0.85) %I2
         \end{pspicture}
         &
         \begin{pspicture}(0,0)(3,4)
            \multido{\n=0+1}{4}{\psline(\n,0)(\n,4)}
            \multido{\n=3+1}{2}{\psline(0,\n)(3,\n)}
            \rput(0.5,3.5){C}
            \rput(1.5,3.5){X}
            \rput(2.5,3.5){I}
            \psdots(1.3,2.7)(1.7,2.7)(1.3,2.4)(1.7,2.4)(1.3,2.1)(1.7,2.1) %X1
            \psdots(2.3,2.7)(2.5,2.4)(2.7,2.1) %I1
            \psdots(1.3,1.8)(1.7,1.8)(1.5,1.5)(1.3,1.2)(1.7,1.2) %X2
            \psdots(2.4,1.65)(2.6,1.35) %I2
            \pspolygon[linecolor=J1](1.1,1)(1.1,2.9)(1.9,2.9)(1.9,1.5)(1.5,1.3)(1.5,1)
         \end{pspicture}
         &
         \begin{pspicture}(0,0)(3,4)
            \multido{\n=0+1}{4}{\psline(\n,0)(\n,4)}
            \multido{\n=3+1}{2}{\psline(0,\n)(3,\n)}
            \rput(0.5,3.5){C}
            \rput(1.5,3.5){X}
           \rput(2.5,3.5){I}
            \psdot[linecolor=J1](0.5,2.4) %C1
            \psdot(1.5,2.4) %X1
            \psdots(2.3,2.7)(2.3,2.1)(2.5,2.4)(2.7,2.1)(2.7,2.7) %I1
         \end{pspicture} \\ [3mm]
        \dotfill & \dotfill & \dotfill \\ [3mm]
        \dotfill & \dotfill & \dotfill \\ [3mm]
        \dotfill & \dotfill & \dotfill \\ [3mm]
      \end{tabular}
  \end{center}

\partie[à vous de jouer]
   Effectuer à l'abaque les additions suivantes (représenter dans les tableaux le calcul effectué).
   \begin{center}
      \begin{tabular}{>{\centering\arraybackslash}p{3cm}>{\centering\arraybackslash}p{3cm}>{\centering\arraybackslash}p{4cm}>{\centering\arraybackslash}p{5cm}}
         \begin{pspicture}(0,-1)(3,4)
            \multido{\n=0+1}{4}{\psline(\n,-1)(\n,4)}
            \multido{\n=3+1}{2}{\psline(0,\n)(3,\n)}
            \rput(0.5,3.5){C}
            \rput(1.5,3.5){X}
            \rput(2.5,3.5){I}
         \end{pspicture}
         &
         \begin{pspicture}(0,-1)(3,4)
            \multido{\n=0+1}{4}{\psline(\n,-1)(\n,4)}
            \multido{\n=3+1}{2}{\psline(0,\n)(3,\n)}
            \rput(0.5,3.5){C}
            \rput(1.5,3.5){X}
            \rput(2.5,3.5){I}
         \end{pspicture}
         &
         \begin{pspicture}(0,-1)(4,4)
            \multido{\n=0+1}{5}{\psline(\n,-1)(\n,4)}
            \multido{\n=3+1}{2}{\psline(0,\n)(4,\n)}
            \rput(0.5,3.5){X}
            \rput(1.5,3.5){I}
            \rput(2.5,3.5){$\frac{1}{10}$}
            \rput(3.5,3.5){$\frac{1}{100}$}
         \end{pspicture}
         &
         \begin{pspicture}(0,-1)(5,4)
            \multido{\n=0+1}{6}{\psline(\n,-1)(\n,4)}
            \multido{\n=3+1}{2}{\psline(0,\n)(5,\n)}
            \rput(0.5,3.5){X}
            \rput(1.5,3.5){I}
            \rput(2.5,3.5){$\frac{1}{10}$}
            \rput(3.5,3.5){$\frac{1}{100}$}
            \rput(4.5,3.5){$\frac{1}{1000}$}
         \end{pspicture} \\ [1mm]
         $136+321$ & $89+23$ & $23,45+1,3$ & $34,891+59,129$ \\
         = \dotfill & = \dotfill & = \dotfill & = \dotfill \\ [5mm]
      \end{tabular}
   \end{center}

\partie[le défi de la soustraction]
   Obélix souhaite acheter un sanglier à 234 sesterces. Il dispose d'un menhir à 1622 sesterces. Imaginer une méthode permettant de trouver le résultat sur l'abaque. \\
   Et s'il voulait acheter un deuxième sanglier ?
\end{enigme}

% Pour le corrigé, il faut décrémenter le compteur, sinon il est incrémenté deux fois
% \addtocounter{exercice}{-1}
% \begin{corrige}
%     \ldots
% \end{corrige}