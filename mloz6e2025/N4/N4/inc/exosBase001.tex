\begin{exercice*}        
    \begin{enumerate}
        \item Quelle est la différence entre $8$ et $2$ ?
        \item Dans l'expression « $6 + 2 = 8$ », comment s'appelle le nombre $8$ ?
        \item Dans l'expression « $8 - 7$ », comment s'appellent les nombres $8$ et $7$ ?
        \item Quelle est la somme de $4$ et $3$ ?
        \item Dans l'expression « $8 + 5$ », comment s'appellent les nombres $8$ et $5$ ?
        \item Dans l'expression « $7 - 3 = 4$ », comment s'appelle le nombre $4$ ?
    \end{enumerate}    
\end{exercice*}
\begin{corrige}
    %\setcounter{partie}{0} % Pour s'assurer que le compteur de \partie est à zéro dans les corrigés
    \begin{enumerate}
        \item Quelle est la différence entre $8$ et $2$ ?
        
        {\red La différence entre $8$ et $2$ est $6$.}
        \item Dans l'expression « $6 + 2 = 8$ », comment s'appelle le nombre $8$ ?
        
        {\red Dans l'expression « $6 + 2 = 8$ », $8$ s'appelle la somme de $6$ et $2$.}
        \item Dans l'expression « $8 - 7$ », comment s'appellent les nombres $8$ et $7$ ?
        
        {\red Dans l'expression « $8 - 7$ », $8$ et $7$ s'appellent des termes.}
        \item Quelle est la somme de $4$ et $3$ ?
        
        {\red La somme de $4$ et $3$ est $7$.}
        \item Dans l'expression « $8 + 5$ », comment s'appellent les nombres $8$ et $5$ ?
        
        {\red Dans l'expression « $8 + 5$ », $8$ et $5$ s'appellent des termes.}
        \item Dans l'expression « $7 - 3 = 4$ », comment s'appelle le nombre $4$ ?
        
        {\red Dans l'expression « $7 - 3 = 4$ », $4$ s'appelle la différence entre $7$ et $3$.}
    \end{enumerate}
\end{corrige}

