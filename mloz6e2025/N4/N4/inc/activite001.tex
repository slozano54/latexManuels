\begin{activite}[Le labyrinthe]
    {\bf Objectifs :} effectuer une addition ou une soustraction mentalement ; manipuler les nombres entiers et décimaux.
 
    \partie[règle du jeu]
        Pour chaque labyrinthe, l'objectif est de trouver le chemin qui mène à la sortie en effectuant l'une des
        opérations proposées et en passant d'une case à l'autre horizontalement ou verticalement uniquement. \\
        Le point de départ est le nombre entouré dans la première ligne et le point d'arrivée est le nombre
        entouré dans la dernière ligne.
        
    \partie[à vous de jouer !]
    \vspace*{-8mm}
    \begin{center}
        \tikzset{/csteps/inner ysep=16pt}
        \tikzset{/csteps/inner xsep=8pt}
        \pgfkeys{/csteps/inner color=red}
        \pgfkeys{/csteps/outer color=red}
        \pgfkeys{/csteps/fill color=gray!30}
        {\renewcommand{\arraystretch}{2.2}
        \begin{tabular}{|*{5}{>{\centering\arraybackslash}p{0.7cm}|}}
        \multicolumn{5}{c}{$+6$ ou $-6$} \\
        \hline
        28 & \Circled{42} & 49 & 21 & 27 \\
        \hline
        32 & 36 & 30 & 24 & 30 \\
        \hline
        60 & 54 & 18 & 42 & 36 \\
        \hline
        66 & 48 & 54 & 48 & 56 \\
        \hline
        \Circled{72} & 42 & 30 & 36 & 48 \\
        \hline
        \end{tabular}
        \hspace*{1cm}
        \begin{tabular}{|*{5}{>{\centering\arraybackslash}p{0.7cm}|}}
        \multicolumn{5}{c}{$+9$ ou $-9$} \\
        \hline
        45 & 37 & 29 & \Circled{38} & 47 \\
        \hline
        56 & 65 & 56 & 47 & 29 \\
        \hline
        47 & 38 & 65 & 60 & 45 \\
        \hline
        36 & 29 & 20 & 11 & 28 \\
        \hline
        42 & 35 & 42 & \Circled{2} & 37 \\
        \hline
        \end{tabular}
        \bigskip
        \begin{tabular}{|*{5}{>{\centering\arraybackslash}p{0.7cm}|}}
        \multicolumn{5}{c}{$+0,5$ ou $-0,5$} \\
        \hline
        5,5 & 5 & \Circled{3,5} & 4 & 3,5 \\
        \hline
        6 & 4,5 & 4 & 3,5 & 5,5 \\
        \hline
        5,5 & 7 & 5,5 & 6 & 6,5 \\
        \hline
        5 & 4,5 & 5 & 7,5 & 7 \\
        \hline
        6,5 & \Circled{6} & 6,5 & 7 & 2,5 \\
        \hline
        \end{tabular}
        \hspace*{1cm}
        \begin{tabular}{|*{5}{>{\centering\arraybackslash}p{0.7cm}|}}
        \multicolumn{5}{c}{$+0,08$ ou $-0,08$} \\
        \hline
        1,04 & 1,14 & \Circled{1,22} & 10,2 & 0,99 \\
        \hline
        0,98 & 1,06 & 1,1 & 0,95 & 0,82 \\
        \hline
        1,06 & 0,98 & 1,07 & 0,9 & 0,86 \\
        \hline
        1,14 & 1,22 & 1,3 & 1,38 & 1,46 \\
        \hline
        0,86 & 0,74 & 0,82 & 0,9 & \Circled{1,38} \\
        \hline
        \end{tabular}
        }
    \end{center}
    \vspace*{-10mm}    
    \begin{flushright}
       {\it\footnotesize Source : inspiré de 123 jeux de nombres, 8 à 13 ans, Accès Édition, 2007.}
    \end{flushright}
 \end{activite}
 