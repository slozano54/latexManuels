\begin{changemargin}{-10mm}{-10mm}
%pre-001
\begin{prerequis}[Connaisances \emoji{red-heart} et compétences \emoji{diamond-suit} du cycle 3]    
   \begin{itemize}        
       \item[\emoji{red-heart}] Vocabulaire associé à ces objets et à leurs propriétés : côté, sommet, angle, hauteur.
       \columnbreak
       \item[\emoji{diamond-suit}] Reconnaître, nommer, décrire des triangles, dont les triangles particuliers (triangle rectangle, triangle isocèle, triangle équilatéral).       
   \end{itemize}
\end{prerequis}
\vspace*{-3mm}
\begin{debat}[levez vos ardoises !] 
   
   {\bf Claude Martin} (1735-1800) est un soldat français. À sa mort, il lègue une grande partie de sa fortune à la création d'écoles \og La Martinière \fg{} à Lyon, Lucknow et Calcutta. À Lyon, on invente une technique d’utilisation de l’ardoise portant son nom : il s'agit de la méthode \og La Martinière \fg{} dont voici une description tirée du manuel de CM2 de 1969 :
   
   \begin{minipage}{0.6\linewidth}
      \begin{itemize}
         \item les enfants ont devant eux leur ardoise et un morceau de craie ;
         \item le maître pose la question et la répète une fois ;
         \item le maître laisse les élèves réfléchir quelques instants ;
         \item au signal (coup de règle), les enfants écrivent la réponse ;
         \item au second coup de règle, les élèves doivent lever l’ardoise ;
         \item le maître contrôle les résultats et on fait la correction. \\ [-15mm]
      \end{itemize}
   \end{minipage}
   \begin{minipage}{0.35\linewidth}   
      \begin{center}      
         \begin{pspicture}(0,-0.5)(3.5,3.8)
            \psset{fillstyle=solid}
            \psframe[fillcolor=brown!50,framearc=.1](0,0)(3.5,2.5)
            \psframe[fillcolor=black!90,framearc=.1](0.3,0.3)(3.2,2.2)
            \rput(1.75,1.25){\textcolor{white}{\large $1+1=2$}}
         \end{pspicture}
      \end{center}
   \end{minipage}
   \begin{cadre}[B2][F4]
      \begin{center}
         \hrefVideo{https://www.dailymotion.com/video/x2j0wh1}{\bf Plickers : méthode la Martinière \og moderne \fg}, académie d'{\it Orléans Tours}.
      \end{center}
   \end{cadre}
\end{debat}
\end{changemargin}