\begin{exercice*}
   \ \\ [-5mm]
   \begin{enumerate}
      \item Reproduire deux fois cette figure sur le cahier.
         \begin{center}
            % \begin{pspicture}(-1,0)(6,8)
            %    {\psset{fillstyle=solid,fillcolor=A2}
            %    \pstGeonode[CurveType=polygon,PointName=none,PointSymbol=none](1,4){A}(2.5,4){B}(2.5,6){C}
            %    \pstGeonode[CurveType=polygon,PointName=none,PointSymbol=none,fillcolor=B2](1,4){A}(2.5,6){C}(0.5,7.5){D}
            %    \pstGeonode[PointName=none,PointSymbol=none](0,1){E}(5,4){F}
            %    \pstOrtSym[CodeFig=false,PointSymbol=none,PointName=none,CurveType=polygon]{E}{F}{A,B,C}[A',B',C']}
            % \end{pspicture}
            \begin{Geometrie}
               pair A[],B[],C[],D,E,F;
               A0=u*(1,4);
               B0=u*(2.5,4);
               C0=u*(2.5,6);
               D=u*(0.5,7.5);
               E=u*(0,1);
               F=u*(5,4);
               path triangleRouge,triangleBleu;
               triangleRouge=polygone(A0,C0,D);
               triangleBleu=polygone(A0,B0,C0);               
               fill triangleRouge withcolor Crimson;
               fill triangleBleu withcolor LightSkyBlue;
               A1=symetrie(A0,E,F);
               B1=symetrie(B0,E,F);
               C1=symetrie(C0,E,F);
               fill polygone(A1,B1,C1) withcolor LightSkyBlue;
            \end{Geometrie}
         \end{center}
      \item On souhaite compléter la figure de telle sorte qu'elle ait un axe de symétrie. Proposer une méthode avec la règle non graduée et le compas.
      \item Proposer une autre méthode avec uniquement une règle non graduée.
   \end{enumerate}
\end{exercice*}
\begin{corrige}
   Pas de corrigé.
\end{corrige}