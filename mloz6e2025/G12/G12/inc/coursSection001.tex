\begin{changemargin}{0mm}{-15mm}
    \section{Propriétés de conservation}
    \subsection{Énoncés}
    \begin{propriete}[Conservations \admise]
    Dans une \textbf{symétrie par rapport à un axe}, une figure se transforme en une \textbf{reproduction} de celle-ci; La figure symétrique possède donc les mêmes propriétés géométriques que la figure d'origine :
        \begin{itemize}
            \item  toutes les longueurs sont conservées.
            \item  toutes les mesures d'angles sont conservées.
            \item  l'alignement des points est conservé.
            \item  Le parallélisme est conservé.
        \end{itemize}
    \end{propriete}   
    
    Par exemple cela permet d'écrire les propriétés suivantes :

    \begin{propriete}[conservation des angles \admise]
        Si deux angles sont symétriques par rapport à un axe alors ils ont la même mesure.
    \end{propriete}   
    \begin{propriete}[conservation de l'alignement \admise]
        Si trois points sont alignés alors leurs symétriques par rapport à un axe le sont aussi.
    \end{propriete}   
    \begin{propriete}[conservation des aires \admise]
        Si deux figures sont symétriques par rapport à un axe alors elles ont la même aire.
    \end{propriete}   

    En écrire d'autres...
    \subsection{Illustrations}
    \begin{Geometrie}
        pair A[],B[],H[],axe[];
        axe0=u*(3,0);
        axe1-axe0=u*(1,5);
        trace segment(axe0,axe1) withcolor red dashed dashpattern(on6bp off3bp on1.5bp off3bp);
        A0-axe0=u*(-1,4);
        A1=symetrie(A0,axe0,axe1);
        H0=projection(A0,axe0,axe1);
        B0-axe0=u*(-2,2);
        B1=symetrie(B0,axe0,axe1);
        H1=projection(B0,axe0,axe1);
        trace segment(A0,B0);
        trace segment(A1,B1);
        trace segment(A0,A1) dashed evenly;
        trace segment(B0,B1) dashed evenly;
        trace codeperp(A0,H0,axe0,5);
        trace codeperp(B0,H1,axe0,5);
        marque_s:=0.5*marque_s;
        trace marquesegment(A0,B0);
        trace marquesegment(A1,B1);        
        marque_s:=0.5*marque_s;
        trace Codelongueur(A0,B0,B1,A1,2);
        trace Codelongueur(A0,H0,H0,A1,3);
        trace Codelongueur(B0,H1,H1,B1,4);
        label.ulft(TEX("$A$"),A0);
        label.ulft(TEX("$B$"),B0);
        label.urt(TEX("$A'$"),A1);
        label.urt(TEX("$B'$"),B1);
    \end{Geometrie}
    \hfill
    \begin{Geometrie}
        pair A[],B[],H[],axe[];
        axe0=u*(5,0);
        axe1-axe0=u*(-1,5);
        trace segment(axe0,axe1) withcolor red dashed dashpattern(on6bp off3bp on1.5bp off3bp);
        A0-axe0=u*(-2,3);
        A1=symetrie(A0,axe0,axe1);
        H0=projection(A0,axe0,axe1);        
        trace segment(A0,A1) dashed evenly;        
        trace cercles(A0,u);
        trace cercles(A1,u);
        trace codeperp(A0,H0,axe0,5);
        marque_s:=0.3*marque_s;
        trace Codelongueur(A0,H0,H0,A1,2);
        label.ulft(TEX("$O$"),A0);
        label.urt(TEX("$O'$"),A1);
        u:=u*0.5;
        marque_p:="croix";
        pointe(A0,A1);
        u:=u*2;
    \end{Geometrie}
    \hfill
    \begin{Geometrie}
        pair A[],B[],C[],H[],axe[];
        axe0=u*(0,4);
        axe1-axe0=u*(7,0.5);
        trace segment(axe0,axe1) withcolor red dashed dashpattern(on6bp off3bp on1.5bp off3bp);
        A0-axe0=u*(1,0.75);
        A1=symetrie(A0,axe0,axe1);
        H0=projection(A0,axe0,axe1);
        B0-axe0=u*(1.5,2.5);
        B1=symetrie(B0,axe0,axe1);
        H1=projection(B0,axe0,axe1);
        C0-axe0=u*(5.5,1.5);
        C1=symetrie(C0,axe0,axe1);
        H2=projection(C0,axe0,axe1);
        trace polygone(A0,B0,C0);
        trace polygone(A1,B1,C1);
        trace segment(A0,A1) dashed evenly;
        trace segment(B0,B1) dashed evenly;
        trace segment(C0,C1) dashed evenly;
        trace codeperp(A0,H0,axe0,5);
        trace codeperp(B0,H1,axe0,5);
        trace codeperp(C0,H2,axe0,5);
        marque_s:=0.5*marque_s;
        trace marquesegment(A0,B0);        
        %
        trace marqueangle(B0,C0,A0,0);
        trace marqueangle(A1,C1,B1,0);
        marque_a:=1.1*marque_a;
        trace marqueangle(B0,C0,A0,0);
        trace marqueangle(A1,C1,B1,0);
        marque_a:=1.1*marque_a;        
        trace marqueangle(B0,C0,A0,0);
        trace marqueangle(A1,C1,B1,0);
        %
        marque_a:=marque_a/1.21;
        marque_a:=marque_a*0.5;
        trace marqueangle(C0,A0,B0,0);
        trace marqueangle(B1,A1,C1,0);
        marque_a:=marque_a*1.2;
        trace marqueangle(C0,A0,B0,0);
        trace marqueangle(B1,A1,C1,0);
        marque_s:=0.5*marque_s;
        trace Codelongueur(A0,H0,H0,A1,3);
        trace Codelongueur(B0,H1,H1,B1,4);
        trace Codelongueur(C0,H2,H2,C1,2);
        label.ulft(TEX("$A$"),A0);
        label.ulft(TEX("$B$"),B0);
        label.urt(TEX("$C$"),C0);
        label.llft(TEX("$A'$"),A1);
        label.llft(TEX("$B'$"),B1);
        label.lrt(TEX("$C'$"),C1);
    \end{Geometrie}

    \begin{center}
        \begin{Geometrie}[CoinHD={(5u,5u)}]
            pair axe[],A[],B[];
            axe0=u*(5,0);
            axe1=u*(0,5);        
            trace segment(axe0,axe1) withcolor red withpen pencircle scaled 1bp;
            A0=u*(0.5,0.5);
            A1-A0=u*(1,0);
            A2-A0=u*(1,0.5);
            A3-A0=u*(1.5,1);
            A4-A0=u*(0.5,1);
            A5-A0=u*(0.5,2);
            A6-A0=u*(0,2);
            B0=symetrie(A0,axe0,axe1);
            B1=symetrie(A1,axe0,axe1);
            B2=symetrie(A2,axe0,axe1);
            B3=symetrie(A3,axe0,axe1);
            B4=symetrie(A4,axe0,axe1);
            B5=symetrie(A5,axe0,axe1);
            B6=symetrie(A6,axe0,axe1);
            fill polygone(A0,A1,A2,A3,A4,A5,A6) withcolor LightSkyBlue;
            fill polygone(B0,B1,B2,B3,B4,B5,B6) withcolor LightPink;
            trace grille(0.5);
            trace polygone(A0,A1,A2,A3,A4,A5,A6) withcolor blue withpen pencircle scaled 1bp;
            trace polygone(B0,B1,B2,B3,B4,B5,B6) withcolor Crimson withpen pencircle scaled 1bp;
        \end{Geometrie}
    \end{center}
\end{changemargin}
 