% \vspace*{-5mm}
\begin{changemargin}{-5mm}{-5mm}
    %pre-001
    \begin{prerequis}[Connaisances \emoji{red-heart} et compétences \emoji{diamond-suit} du cycle 3]    
   \begin{itemize}        
       \item[\emoji{red-heart}] Vocabulaire associé à ces objets et à leurs propriétés : côté, sommet, angle, hauteur.
       \columnbreak
       \item[\emoji{diamond-suit}] Reconnaître, nommer, décrire des triangles, dont les triangles particuliers (triangle rectangle, triangle isocèle, triangle équilatéral).       
   \end{itemize}
\end{prerequis}
    % \end{changemargin}
    % \begin{changemargin}{-7mm}{-7mm}
    \begin{debat}[À la découverte du hasard]
        Avez-vous déjà joué à pile ou face ? Ou lancé un dé en espérant obtenir un six ? Dans la vie de tous les jours, on est souvent confrontés à des situations où le résultat dépend du hasard. En mathématiques, on appelle cela des \textbf{expériences aléatoires}.
        Dans ce chapitre, vous allez apprendre à :
        \begin{itemize}
            \item \textbf{Lister toutes les issues possibles} d’une expérience (par exemple, les résultats possibles d’un lancer de dé).
            \item \textbf{Calculer la probabilité} qu’un événement se produise, c’est-à-dire la chance qu’il a d’arriver.
            \item \textbf{Exprimer cette probabilité} sous forme de fraction, de nombre décimal ou de pourcentage.
        \end{itemize}

        Vous découvrirez aussi comment organiser vos résultats à l’aide d'\textbf{arbres} ou de \textbf{tableaux}, pour mieux comprendre et prévoir ce que le hasard peut nous réserver.

        Prêts à devenir des experts du hasard ? C’est parti !
        \begin{cadre}[B2][F4]
            \begin{center}
                \hrefVideo{https://player.vimeo.com/video/1079300481}{\bf Peut-on prévoir le hasard ?}, {\it Stéphane et Adrien}.
            \end{center}
        \end{cadre}
        \vspace*{-10mm}
    \end{debat}
\end{changemargin}



