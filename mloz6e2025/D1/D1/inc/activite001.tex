\begin{changemargin}{-10mm}{-5mm}
\begin{activite}[Notion de probabilité][]
    \vspace*{-9mm}
    \subsection*{Matériel nécessaire}
    \vspace*{-9mm}
    \begin{itemize}
        \item Une pièce de monnaie (1€ ou 2€).
        \item Un dé à 6 faces (ou un dé virtuel sur tablette/téléphone).
        \item Une feuille et un stylo.
    \end{itemize}
    \vspace*{-9mm}
    \subsection*{Lancer de pièce}
    \vspace*{-9mm}
    \begin{enumerate}
        \item Lancez une pièce de monnaie 20 fois et notez les résultats dans le tableau ci-dessous :
              \begin{center}
                  \begin{tabular}{|*{21}{c|}}
                      \hline
                      \textbf{Lancer}   & \textbf{1} & \textbf{2} & \textbf{3} & \textbf{4} & \textbf{5} & \textbf{6} & \textbf{7} & \textbf{8} & \textbf{9} & \textbf{10} & \textbf{11} & \textbf{12} & \textbf{13} & \textbf{14} & \textbf{15} & \textbf{16} & \textbf{17} & \textbf{18} & \textbf{19} & \textbf{20} \\
                      \hline
                      \textbf{Résultat} &            &            &            &            &            &            &            &            &            &             &             &             &             &             &             &             &             &             &             &             \\
                      \hline
                  \end{tabular}
              \end{center}
              \medskip
        \item Comptez le nombre de fois où vous avez obtenu \textbf{Pile} et le nombre de fois où vous avez obtenu \textbf{Face}.
        \item Comparez vos résultats avec ceux de votre voisin(e). Que constatez-vous ?
    \end{enumerate}
    \vspace*{-9mm}
    \subsection*{Lancer de dé}
    \vspace*{-9mm}
    \begin{enumerate}
        \item Lancez un dé à 6 faces 20 fois et notez les résultats dans le tableau ci-dessous :
              \begin{center}
                  \begin{tabular}{|*{22}{c|}}
                      \hline
                      \textbf{Lancer}   & \textbf{1} & \textbf{2} & \textbf{3} & \textbf{4} & \textbf{5} & \textbf{6} & \textbf{7} & \textbf{8} & \textbf{9} & \textbf{10} & \textbf{11} & \textbf{12} & \textbf{13} & \textbf{14} & \textbf{15} & \textbf{16} & \textbf{17} & \textbf{18} & \textbf{19} & \textbf{20} & \textbf{Total} \\
                      \hline
                      \textbf{Face 1} &            &            &            &            &            &            &            &            &            &             &             &             &             &             &             &             &             &             &             &             &             \\
                      \hline
                      \textbf{Face 2} &            &            &            &            &            &            &            &            &            &             &             &             &             &             &             &             &             &             &             &             &             \\
                      \hline
                      \textbf{Face 3} &            &            &            &            &            &            &            &            &            &             &             &             &             &             &             &             &             &             &             &             &             \\
                      \hline
                      \textbf{Face 4} &            &            &            &            &            &            &            &            &            &             &             &             &             &             &             &             &             &             &             &             &             \\
                      \hline
                      \textbf{Face 5} &            &            &            &            &            &            &            &            &            &             &             &             &             &             &             &             &             &             &             &             &             \\
                      \hline
                      \textbf{Face 6} &            &            &            &            &            &            &            &            &            &             &             &             &             &             &             &             &             &             &             &             &             \\
                      \hline
                  \end{tabular}
              \end{center}
              \medskip
        \item Comptez le nombre de fois où chaque face est apparue.
        \item Quelle face est apparue le plus souvent ? La moins souvent ?
        \item Est-il possible d'obtenir un 7 avec ce dé ? Pourquoi ?
    \end{enumerate}
    \vspace*{-9mm}
    \subsection*{Vocabulaire}
    \vspace*{-9mm}
    \begin{itemize}
        \item \textbf{Expérience aléatoire} : Une expérience dont on ne peut pas prévoir le résultat à l'avance (exemple : lancer une pièce ou un dé).
        \item \textbf{Événement} : Un résultat possible d'une expérience aléatoire (exemple : obtenir "Pile").
        \item \textbf{Événement certain} : Un événement qui se produira forcément (exemple : obtenir un nombre entre 1 et 6 avec un dé).
        \item \textbf{Événement impossible} : Un événement qui ne se produira jamais (exemple : obtenir un 7 avec un dé à 6 faces).
    \end{itemize}
    \vspace*{-9mm}
    \subsection*{Questions de réflexion}
    \vspace*{-9mm}
    \begin{enumerate}
        \item Pourquoi dit-on que lancer une pièce ou un dé est une \textbf{expérience aléatoire} ?
        \item Citez un autre exemple d'expérience aléatoire.
        \item Peut-on prévoir avec certitude le résultat d'une expérience aléatoire ? Pourquoi ?
    \end{enumerate}
\end{activite}
\end{changemargin}