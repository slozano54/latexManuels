\section{Vocabulaire et notations}
\begin{definition}
On appelle \textbf{expérience aléatoire}, une expérience qui vérifie les trois conditions suivantes :
    \begin{itemize}
        \item les résultats possibles sont parfaitement identifiables.
        \item On ne sait pas, a priori, quelle résultat va se produire lorsqu'on réalise l'expérience.
        \item L'expérience doit être reproductible dans les mêmes conditions, dans la mesure du possible!
    \end{itemize}
\end{definition}

\begin{exemple*1}
    \titreExemple{Expérience (A)}

    \textit{"On lance une pièce de monnaie et on regarde sur quelle face elle tombe"}

    Cette expérience est bien aléatoire car :
    \begin{itemize}
        \item il y a deux issues possibles : \og pile\fg{} ou \og face\fg{}.
        \item quand on lance la pièce, on ne peut pas savoir, a priori, sur quelle face elle va tomber.
    \end{itemize}
\end{exemple*1}

\begin{vocabulaire}
   \begin{itemize}
      \item Chaque résultat possible et prévisible d'une expérience aléatoire est appelé une \textbf{issue}.
      \item L'ensemble formé par les issues est appelé \textbf{univers}, souvent noté $\Omega$.
      \item Un \textbf{événement} de l'expérience aléatoire est une partie quelconque de l'univers, c'est donc un ensemble d'issues de l'expérience.
   \end{itemize}
\end{vocabulaire}

\begin{exemples*1}
   \begin{enumerate}
      \item \titreExemple{Expérience (A)} : {\it \og Lancer d'une pièce de monnaie\fg{}} : $\Omega =\{$~pile~;~face~$\}$.
      \item \titreExemple{Expérience (B)} : \textit{\og Lancer un dé à 6 faces numérotées de 1 à 6.\fg{}} : $\Omega =\{~1~;~2~;~3~;~4~;~5~;~6~\}$.
      \begin{itemize}
        \item \og Obtenir $1$\fg{}, \og Obtenir $2$\fg{}, \og Obtenir $3$\fg{} etc... sont des \textbf{événements élémentaires}, ils sont réalisés par une seule issue.
        \item \og Obtenir un numéro pair \fg{} est un événement que l'on peut noter $A=\{~2~;~4~;~6~\}$.
        \item \og Sortir un $7$\fg{} est un \textbf{événement impossible}.
        \item \og Sortir un nombre compris entre $0$ et $7$\fg{} est un \textbf{événement certain} car tous les lancers donneront un nombre compris entre $0$ et $7$.
      \end{itemize}
   \end{enumerate}
\end{exemples*1}
