% Les enigmes ne sont pas numérotées par défaut donc il faut ajouter manuellement la numérotation
% si on veut mettre plusieurs enigmes
\refstepcounter{exercice}
% \numeroteEnigme
\begin{enigme}[Quelques activités de fin de chapitre sur les probabilités]
    \vspace*{-5mm}
    \subsection*{Règles du jeu}
    \vspace*{-9mm}
    \begin{itemize}
        \item Chaque élève reçoit cette fiche et un crayon.
        \item Le jeu se compose de 3 défis à réaliser \textbf{en dessinant ou en cochant des cases}.
        \item Pour chaque défi, suivez les instructions et répondez aux questions.
        \item À la fin, comparez vos résultats avec ceux de vos camarades.
    \end{itemize}
    \vspace*{-5mm}
    \subsection*{Défi 1 : Simulation de lancer de pièce}
    \vspace*{-9mm}
    \begin{enumerate}
        \item \textbf{Simulez 20 lancers de pièce} en utilisant la méthode suivante :
              \begin{itemize}
                  \item Fermez les yeux et pointez au hasard sur le tableau ci-dessous.
                  \item Si vous pointez sur "Pile", notez un "P" dans le tableau des résultats.
                  \item Si vous pointez sur "Face", notez un "F".
              \end{itemize}

              \begin{center}
                  \begin{tabular}{|c|c|}
                      \hline
                      \textbf{\Huge Pile} & \textbf{\Huge Face} \\
                      \hline
                  \end{tabular}
              \end{center}

        \item \textbf{Tableau des résultats} :
              \begin{center}
                  \begin{tabular}{|*{21}{c|}}
                      \hline
                      \textbf{Lancer}   & \textbf{1} & \textbf{2} & \textbf{3} & \textbf{4} & \textbf{5} & \textbf{6} & \textbf{7} & \textbf{8} & \textbf{9} & \textbf{10} & \textbf{11} & \textbf{12} & \textbf{13} & \textbf{14} & \textbf{15} & \textbf{16} & \textbf{17} & \textbf{18} & \textbf{19} & \textbf{20} \\
                      \hline
                      \textbf{Résultat} &            &            &            &            &            &            &            &            &            &             &             &             &             &             &             &             &             &             &             &             \\
                      \hline
                  \end{tabular}
              \end{center}
              \medskip
        \item Comptez le nombre de "P" et de "F". Que constatez-vous ?
    \end{enumerate}
    \vspace*{-5mm}
    \subsection*{Défi 2 : Simulation de lancer de dé}
    \vspace*{-9mm}
    \begin{enumerate}
        \item \textbf{Simulez 15 lancers de dé} en utilisant la méthode suivante :
              \begin{itemize}
                  \item Dessinez un dé à 6 faces sur votre feuille.
                  \item Fermez les yeux et pointez au hasard sur une face.
                  \item Notez le numéro pointé dans le tableau ci-dessous.
              \end{itemize}

        \item \textbf{Tableau des résultats} :
              \begin{center}
                  \begin{tabular}{|*{16}{c|}}
                      \hline
                      \textbf{Lancer}   & 1 & 2 & 3 & 4 & 5 & 6 & 7 & 8 & 9 & 10 & 11 & 12 & 13 & 14 & 15 \\
                      \hline
                      \textbf{Résultat} &   &   &   &   &   &   &   &   &   &    &    &    &    &    &    \\
                      \hline
                  \end{tabular}
              \end{center}
              \medskip
        \item Quel numéro est sorti le plus souvent ? Pourquoi ?
    \end{enumerate}
    \vspace*{-5mm}
    \subsection*{Défi 3 : Événements certains, impossibles ou aléatoires}
    \vspace*{-9mm}
    \begin{enumerate}
        \item Pour chaque situation, indiquez si l'événement est \textbf{certain}, \textbf{impossible} ou \textbf{aléatoire} :
              \begin{itemize}
                  \item Obtenir un 7 en lançant un dé à 6 faces : \pointilles[20mm]
                  \item Obtenir un nombre entre 1 et 6 en lançant un dé à 6 faces : \pointilles[20mm]
                  \item Obtenir "Pile" en lançant une pièce : \pointilles[20mm]
                  \item Obtenir un 3 en lançant un dé à 4 faces : \pointilles[20mm]
              \end{itemize}
    \end{enumerate}
    \vspace*{-5mm}
    \subsection*{Bilan}
    \vspace*{-5mm}
    Répondez aux questions suivantes :
    \begin{itemize}
        \item Pourquoi dit-on que ces expériences sont aléatoires ?
        \item Avez-vous obtenu les mêmes résultats que vos camarades ? Pourquoi ?
    \end{itemize}
\end{enigme}

% Pour le corrigé, il faut décrémenter le compteur, sinon il est incrémenté deux fois
\addtocounter{exercice}{-1}
\begin{corrige}
    \setcounter{subsection}{0}
    \subsection*{Défi 1}
    \begin{enumerate}
        \item \textbf{Tableau des résultats} : Les résultats varient selon les élèves, mais on s’attend à obtenir environ 10 "P" et 10 "F" sur 20 lancers (car la probabilité de "Pile" ou "Face" est de 1/2).
        \item \textbf{Question} : On constate que les résultats sont proches de 10 "P" et 10 "F", mais pas exactement égaux. Cela illustre la variabilité des expériences aléatoires.
    \end{enumerate}
    \Coupe
    \subsection*{Défi 2}
    \begin{enumerate}
        \item \textbf{Tableau des résultats} : Les résultats varient, mais chaque face (1 à 6) devrait apparaître environ 2 ou 3 fois sur 15 lancers (car la probabilité de chaque face est de 1/6).
        \item \textbf{Question} : Le numéro qui sort le plus souvent dépend du hasard. Sur un grand nombre de lancers, chaque face devrait sortir à peu près le même nombre de fois.
    \end{enumerate}

    \subsection*{Défi 3}
    \begin{itemize}
        \item Obtenir un 7 en lançant un dé à 6 faces : \textbf{impossible} (un dé à 6 faces ne peut pas donner 7).
        \item Obtenir un nombre entre 1 et 6 en lançant un dé à 6 faces : \textbf{certain} (toutes les faces du dé sont comprises entre 1 et 6).
        \item Obtenir "Pile" en lançant une pièce : \textbf{aléatoire} (il y a une chance sur deux d'obtenir "Pile").
        \item Obtenir un 3 en lançant un dé à 4 faces : \textbf{aléatoire} (le dé a 4 faces, donc il y a une chance sur quatre d'obtenir 3).
    \end{itemize}

    \subsection*{Bilan}
    \begin{itemize}
        \item \textbf{Pourquoi dit-on que ces expériences sont aléatoires ?}
              \\ On ne peut pas prévoir le résultat à l'avance. Chaque expérience (lancer de pièce ou de dé) a plusieurs résultats possibles, et le résultat obtenu dépend du hasard.

        \item \textbf{Avez-vous obtenu les mêmes résultats que vos camarades ? Pourquoi ?}
              \\ Non, les résultats varient d’un élève à l’autre car les expériences sont aléatoires. Cependant, si on répète un grand nombre de fois la même expérience, les résultats se rapprochent des probabilités théoriques (par exemple, environ 1/2 de "Pile" pour une pièce).
    \end{itemize}
\end{corrige}