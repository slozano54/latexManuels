\begin{exercice*}
    Rayan tire une carte dans un jeu ordinaire de cinquante-deux cartes.
    \begin{enumerate}
       \item Donner les probabilités qu'il obtienne les événements suivants : \og Obtenir un carreau \fg ; \og Obtenir un valet \fg{} et \og Obtenir un valet de carreau \fg.
       \item Calculer la probabilité de ne pas obtenir de carreau.
    \end{enumerate}
\end{exercice*}
\begin{corrige}
    %\setcounter{partie}{0} % Pour s'assurer que le compteur de \partie est à zéro dans les corrigés
    %\phantom{rrr}
    \ \\ [-5mm]
    \begin{enumerate}
       \item Obtenir un carreau : ${\red P =\dfrac{13}{52}}$. \\
          Obtenir un valet : ${\red P =\dfrac{4}{52}}$. \\
          Obtenir un valet de carreau  : ${\red P =\red \dfrac{1}{52}}$. \smallskip
       \item La probabilité de ne pas obtenir de carreau s'obtient en calculant la probabilité d'obtenir un c\oe ur, un pique ou un trèfle, ce qui fait au total $3\times13$ carte, soit 39 cartes. {\red $P =\dfrac{39}{52}$}.
    \end{enumerate} 
\end{corrige}    