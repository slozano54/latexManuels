\vspace*{-7mm}
\begin{changemargin}{-10mm}{-10mm}
%pre-001
\begin{prerequis}[Connaisances \emoji{red-heart} et compétences \emoji{diamond-suit} du cycle 3]    
   \begin{itemize}        
       \item[\emoji{red-heart}] Vocabulaire associé à ces objets et à leurs propriétés : côté, sommet, angle, hauteur.
       \columnbreak
       \item[\emoji{diamond-suit}] Reconnaître, nommer, décrire des triangles, dont les triangles particuliers (triangle rectangle, triangle isocèle, triangle équilatéral).       
   \end{itemize}
\end{prerequis}
\end{changemargin}
\vspace*{-13mm}
\begin{debat}[L'\oe il d'Oudjat, ou \oe il d'Horus] 
   Dans la mythologie égyptienne, on trouve une histoire liée aux fractions : {\it Seth}, dieu de la violence et incarnation du mal, arrache l’{\bf œil} à {\bf Horus}. {\it Seth} le partage en six morceaux et les répand à travers l’Egypte. {\it Thot}, dieu magicien, reconstitue l’œil, symbole du bien contre le mal. Chacune de ses parties symbolise une fraction de numérateur 1 et de dénominateurs 2, 4, 8, 16, 32 et 64. Il accordera le 64\up{e} manquant à tout scribe recherchant et acceptant sa protection.
   \begin{center}
      \includegraphics[width=6cm]{\currentpath/images/Horus}
   \end{center}
   \bigskip
   \begin{cadre}[B2][F4]
      \begin{center}
         \hrefVideo{https://www.yout-ube.com/watch?v=yHft4m1Gi7k}{\bf La légende de l'\oe il d'Horus}, chaîne YouTube {\it Ma deuxième école}.
      \end{center}
   \end{cadre}
\end{debat}