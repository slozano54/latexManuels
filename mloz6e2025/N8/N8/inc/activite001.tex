\begin{changemargin}{-10mm}{-15mm}
\begin{activite}[Décomposer une fraction]    
    {\bf Objectifs :} représenter des fractions ; écrire une fraction sous la forme de la somme d'un entier et d'une fraction inférieure à 1.
       \partie[des toasts en entrée]
          Pour faire des toasts à ses amis, Mattis coupe des tranches de pain de mie en quatre, avant de les garnir. \\
          Samy mange onze de ces petits toasts. Quelle fraction de grande tranche de pain de mie Samy a-t-il mangée ?
          \begin{center}
             {\psset{xunit=0.9,yunit=0.8}
             \begin{pspicture}(0,0)(3,1.5)
                \psframe[fillstyle=solid,fillcolor=B3](0,0)(2,1.5)
                \psline(1,0)(1,1.5)
                \psline(0,0.75)(2,0.75)
             \end{pspicture}
             \begin{pspicture}[fillstyle=solid,fillcolor=B3](0,0)(3,1.5)
                \psframe(0,0)(2,1.5)
                \psline(1,0)(1,1.5)
                \psline(0,0.75)(2,0.75)
             \end{pspicture}
             \begin{pspicture}(0,0)(3,1.5)
                \psframe(0,0)(2,1.5)
                \pspolygon[fillstyle=solid,fillcolor=B3](0,0)(1,0)(1,0.75)(2,0.75)(2,1.5)(0,1.5)
                \psline(1,0)(1,1.5)
                \psline(0,0.75)(2,0.75)
             \end{pspicture}
             \begin{pspicture}(0,0)(3,1.5)
                \psframe(0,0)(2,1.5)
                \psline(1,0)(1,1.5)
                \psline(0,0.75)(2,0.75)
             \end{pspicture}
             \begin{pspicture}(0,0)(3,1.5)
                \psframe(0,0)(2,1.5)
                \psline(1,0)(1,1.5)
                \psline(0,0.75)(2,0.75)
             \end{pspicture}}
          \end{center}
          Il a mangé 11 fois $\dfrac14$ de grande tranche, donc $\dfrac{11}4$ de grande tranche, ce qui fait 2 tranches et $\dfrac34$ de tranche. \\ [2mm]
 On peut écrire : \hspace{40mm} $\dfrac{11}4 =2+\dfrac34$. \\ [2mm]
          Meryem, elle, a mangé 17 petits toasts. Colorier ce que cela représente ci-dessous.
          \begin{center}
             {\psset{xunit=0.9,yunit=0.8}
             \begin{pspicture}(0,0)(3,1.5)
                \psframe(0,0)(2,1.5)
                \psline(1,0)(1,1.5)
                \psline(0,0.75)(2,0.75)
             \end{pspicture}
             \begin{pspicture}(0,0)(3,1.5)
                \psframe(0,0)(2,1.5)
                \psline(1,0)(1,1.5)
                \psline(0,0.75)(2,0.75)
             \end{pspicture}
             \begin{pspicture}(0,0)(3,1.5)
                \psframe(0,0)(2,1.5)
                \pspolygon(0,0)(1,0)(1,0.75)(2,0.75)(2,1.5)(0,1.5)
                \psline(1,0)(1,1.5)
                \psline(0,0.75)(2,0.75)
             \end{pspicture}
             \begin{pspicture}(0,0)(3,1.5)
                \psframe(0,0)(2,1.5)
                \psline(1,0)(1,1.5)
                \psline(0,0.75)(2,0.75)
             \end{pspicture}
             \begin{pspicture}(0,0)(3,1.5)
                \psframe(0,0)(2,1.5)
                \psline(1,0)(1,1.5)
                \psline(0,0.75)(2,0.75)
             \end{pspicture}}
          \end{center} \medskip
          Compléter l'égalité  : \hspace{30mm} $\dfrac{17}{4} = \pointilles[1cm] + \dfrac{\pointilles[1cm]}{\pointilles[1cm]}$ \\
 
       \partie[des \og flam \fg{} en plat principal]
          Pour poursuivre, Mattis propose à ses amis de manger des flammekueches (flam). \\
          \begin{minipage}{7.75cm}
             \begin{pspicture}(-1.2,-1)(1,1)
                \pscircle[fillstyle=solid,fillcolor=J2](0,0){0.7}
                \multido{\n=0+72}{5}{\psline(0,0)(0.7;\n)}
             \end{pspicture}
             \begin{pspicture}(-0.7,-1)(1,1)
                \pscircle(0,0){0.7}
                \pswedge[fillstyle=solid,fillcolor=J2](0,0){0.7}{72}{288}
                \multido{\n=0+72}{5}{\psline(0,0)(0.7;\n)}
             \end{pspicture}
             \begin{pspicture}(-0.7,-1)(1,1)
                \pscircle(0,0){0.7}
                \multido{\n=0+72}{5}{\psline(0,0)(0.7;\n)}
             \end{pspicture}
             \begin{pspicture}(-0.7,-1)(1,1)
                \pscircle(0,0){0.7}
                \multido{\n=0+72}{5}{\psline(0,0)(0.7;\n)}
             \end{pspicture} \\
             Une part de flam représente $\dfrac{\pointilles[1cm]}{\pointilles[1cm]}$ de flam. \\ [3mm]
             $\dfrac{\pointilles[1cm]}{\pointilles[1cm]}$ de flam est colorié. \\ [3mm]
             Donc, $\dfrac{\pointilles[1cm]}{\pointilles[1cm]} ={\pointilles[1cm]} + \dfrac{\pointilles[1cm]}{\pointilles[1cm]}$ \\ [2mm]
          \end{minipage}
          \qquad
          \begin{minipage}{7.75cm}
             \begin{pspicture}(-1.2,-1)(1,1)
                \pscircle(0,0){0.7}
                \multido{\n=0+45}{8}{\psline(0,0)(0.7;\n)}
                \end{pspicture}
                \begin{pspicture}(-0.7,-1)(1,0.8)
                \pscircle(0,0){0.7}
                \multido{\n=0+45}{8}{\psline(0,0)(0.7;\n)}
                \end{pspicture}
                \begin{pspicture}(-0.7,-1)(1,1)
                \pscircle(0,0){0.7}
                \multido{\n=0+45}{8}{\psline(0,0)(0.7;\n)}
                \end{pspicture}
                \begin{pspicture}(-0.7,-1)(1,1)
                \pscircle(0,0){0.7}
                \multido{\n=0+45}{8}{\psline(0,0)(0.7;\n)}
                \end{pspicture} \\
                Une part de flam représente $\dfrac{\pointilles[1cm]}{\pointilles[1cm]}$ de flam. \\
                Colorier $\dfrac{13}{8}$ de flam. \\ [1mm]
                Donc, $\dfrac{13}{8} ={\pointilles[1cm]}+\dfrac{\pointilles[1cm]}{\pointilles[1cm]}$ \\ [2mm]
          \end{minipage}
 
       \partie[Des éclairs en dessert]
          Julie a mangé $\dfrac{73}9$ de mini-éclairs au chocolat, trouver une manière de décomposer cette fraction en la somme \\ [1mm]
          d'un nombre entier et d'une fraction inférieure à 1. \\ [3mm]
        \phantom{rrr}\hfill{\it\footnotesize{Inspiré de \href{http://ww2.ac-poitiers.fr/dsden86-pedagogie/sites/dsden86-pedagogie/IMG/pdf/groupe4_c13.pdf}{Écrire une fraction sous la forme d'un entier et d'une fraction inférieure à 1}, académie de Poitiers.}}
\end{activite}
\end{changemargin}
