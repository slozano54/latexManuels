% Les enigmes ne sont pas numérotées par défaut donc il faut ajouter manuellement la numérotation
% si on veut mettre plusieurs enigmes
% \refstepcounter{exercice}
% \numeroteEnigme
\begin{enigme}[Sudofractions]
    \begin{changemargin}{-10mm}{-10mm}
        \vspace*{-10mm}
        \begin{minipage}{7.5cm}
            \begin{itemize}
               \item Pour chaque fraction ou nombre décimal, déterminer le nombre entier immédiatement inférieur.
               \item Placer ce nombre entier dans la grille vierge ci-contre, exactement au même endroit que sur la grille de calculs.
               \item Compléter la grille selon les règles du Sudoku.
            \end{itemize}
            {\it Par exemple, $\dfrac{5}{2} =2+\dfrac{1}{2} (=2,5)$ donc, \\ [1mm]
               la troisième case de la ligne du haut comporte un 2}.
         \end{minipage}
         \qquad
         \begin{minipage}{7cm}
        {\renewcommand{\arraystretch}{1.8}
            \begin{tabular}{|*{3}{>{\centering\arraybackslash}p{0.5cm}|}|*{3}{>{\centering\arraybackslash}p{0.5cm}|}|*{3}{>{\centering\arraybackslash}p{0.5cm}|}}
               \hline
               & & {\bf 2} & & & & & & \\
               \hline
               & & & & & & & & \\
               \hline
               & & & & & & & & \\
               \hline
               \hline
               & & & & & & & & \\
               \hline
               & & & & & & & & \\
               \hline
               & & & & & & & & \\
               \hline
               \hline
               & & & & & & & & \\
               \hline
               & & & & & & & & \\
               \hline
               & & & & & & & & \\
               \hline
            \end{tabular}}
         \end{minipage}
        \vfill
         \begin{center}
         {\renewcommand{\arraystretch}{3.44}
         \footnotesize
            \begin{tabular}{|*{3}{>{\centering\arraybackslash}p{0.9cm}|}|*{3}{>{\centering\arraybackslash}p{0.9cm}|}|*{3}{>{\centering\arraybackslash}p{0.9cm}|}}
               \hline
                \!$\dfrac{13}{2}$\! & $\dfrac{27}{5}$ & $\dfrac{5}{2}$ & & 8,73 & 7,99 & $\dfrac{39}{4}$ & & \!$\dfrac{43}{10}\!$ \\
               \hline
                $\dfrac{77}{10}$ & $\dfrac{7}{6}$ & & 4,5 & & & $\dfrac{25}{7}$ & & \\
               \hline
                & & $\dfrac{17}{2}$ & $1,001$ & & & & & $\dfrac{38}{5}$\\
                \hline
                \hline
                8,25 & 2,89 & & & $\dfrac{33}{10}$ & & & & $6,7$ \\
                \hline
                5,55 & & & $\dfrac{21}{9}$ & & & & 3,199 & \\
               \hline
                & & & $\dfrac{80}{9}$ & & & 5,356 & 2,1 & \\
               \hline
               \hline
                & $\dfrac{845}{100}$ & $\dfrac{60}{8}$ & 5,2 & 2,8 & & $\dfrac{13}{2}$ & $\dfrac{25}{6}$ & \\
               \hline
                \!$\dfrac{30}{11}$\! & $\dfrac{25}{8}$ & & 6,001 & & 8,81 & & $\dfrac{65}{7}$ & \\
               \hline
                $\dfrac{29}{3}$ & & & & 4,909 & & & & \\
               \hline
            \end{tabular}}
         \end{center}
    \end{changemargin}
\end{enigme}
% Pour le corrigé, il faut décrémenter le compteur, sinon il est incrémenté deux fois
% \addtocounter{exercice}{-1}
\begin{corrige}
    Correction du binz.
\end{corrige}
 