\vspace*{-8mm}
%pre-001
\begin{prerequis}[Connaisances \emoji{red-heart} et compétences \emoji{diamond-suit} du cycle 3]    
   \begin{itemize}        
       \item[\emoji{red-heart}] Vocabulaire associé à ces objets et à leurs propriétés : côté, sommet, angle, hauteur.
       \columnbreak
       \item[\emoji{diamond-suit}] Reconnaître, nommer, décrire des triangles, dont les triangles particuliers (triangle rectangle, triangle isocèle, triangle équilatéral).       
   \end{itemize}
\end{prerequis}
\vspace*{-5mm}
\begin{debat}[Angles et perspective]
    La {\bf perspective} est un ensemble de techniques destinées à représenter un objet ou une image en trois dimensions sur une surface plane. Il existe plusieurs sortes de perspectives, comme par exemple la perspective cavalière, la perspective isométrique, la perspective à point de fuite\dots{} Selon la perspective utilisée, on utilise des directions suivant des angles différents.
    \begin{center}
       \begin{pspicture}(0,-1)(4,3.5)
          \psframe(0,0)(1.5,1.5)
          \psline(0,1.5)(0.8,2.3)(2.3,2.3)(2.3,0.8)(1.5,0)
          \psline(1.5,1.5)(2.3,2.3)
          \psset{linestyle=dashed,linecolor=A1}
          \psline(0.8,2.3)(1.6,3.1)
          \psline(2.3,2.3)(3.1,3.1)
          \psline(2.3,0.8)(3.1,1.6)
          \rput(0.75,-0.75){\it cavalière}
       \end{pspicture}
       \begin{pspicture}(-2,-2.5)(2,2)
          \pspolygon(1.3;-30)(1.3;30)(1.3;90)(1.3;150)(1.3;-150)(1.3;-90)
          \psline(1.3;150)(0,0)(1.3;30)
          \psline(0,0)(1.3;-90)
         \psset{linestyle=dashed,linecolor=A1}
          \psline(1.3;150)(2.3;150)
          \psline(1.3;30)(2.3;30)
          \psline(1.3;-90)(2;-90)
          \rput(0,-2.25){\it isométrique}
       \end{pspicture}
       \begin{pspicture}(-1,-1)(3,3.5)
          \psframe(0,0)(1.5,1.5)
          \psline(0,1.5)(0.8,1.9)(1.9,1.9)(1.9,0.8)(1.5,0)
          \psline(1.5,1.5)(1.9,1.9)
          \psdot(3,3)
          \psset{linestyle=dashed,linecolor=A1}
          \psline(0.8,1.9)(3,3)
          \psline(1.9,1.9)(3,3)
          \psline(1.9,0.8)(3,3)
          \rput(0.75,-0.75){\it un point de fuite}
       \end{pspicture}
    \end{center}
    \bigskip
    \begin{cadre}[B2][F4]
       \begin{center}
         \hrefVideo{https://www.yout-ube.com/watch?v=zCIxdOCQiZg}{\bf Dessiner des illusions d'optique 3D}

         \vspace*{5mm}          
          Chaîne YouTube {\it Simple drawing tutorial}.
       \end{center}
    \end{cadre}
 \end{debat}