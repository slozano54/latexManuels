\vspace*{-17mm}
\begin{activite}[Une nouvelle unité de mesure]    
    \vspace*{-5mm}
    \begin{changemargin}{-10mm}{-15mm}
    {\bf Objectif :} comparer des angles sans recours à la mesure ; découvrir le degré ; mesurer un angle donné.
    \partie[fabrication d'angles - {\it Partie à faire en binôme.}]
        \begin{enumerate}
            \item 
            \begin{enumerate}
            \item Sur une feuille, construire deux \textbf{carrés} dont les mesures des côtés sont respectivement \Lg[cm]{4} et \Lg[cm]{7}.
            \item Découper ces carrés puis les plier en deux suivant une \textbf{diagonale} : on obtient pour chaque carré deux triangles identiques à découper.
            \begin{center}
                {\psset{unit=0.6}
                \begin{pspicture}(-0.5,-0.3)(4,3)
                    \psframe(0,0)(3,3)
                    \psline[linestyle=dashed](0,0)(3,3)
                    \rput(4,1.5){$\Longrightarrow$}
                \end{pspicture}
                \begin{pspicture}(0,-0.3)(4,3)
                    \pswedge[fillstyle=solid,fillcolor=G1,linecolor=G1](0,0){0.8}{0}{45}
                    \pswedge[fillstyle=solid,fillcolor=G1,linecolor=G1](3,3){0.8}{225}{-90}
                    \psframe[fillstyle=solid,fillcolor=B1,linecolor=B1](3,0)(2.5,0.5)
                    \pspolygon(0,0)(3,0)(3,3)
                \end{pspicture}}
            \end{center}
            \end{enumerate}
            \item
            \begin{enumerate}
            \item Construire deux \textbf{triangles équilatéraux} dont les mesures des côtés sont respectivement de \Lg[cm]{5} et \Lg[cm]{7}.
            \item Découper ces triangles puis les plier en deux suivant une \textbf{hauteur} : on obtient pour chaque triangle deux triangles identiques à découper.
            \end{enumerate}
            \begin{center}
            {\psset{unit=0.6}
            \begin{pspicture}(0,0.3)(4.5,2.5)
                \pspolygon(0,0)(3.5,0)(1.75,3.03)
                \psline[linestyle=dashed](1.75,0)(1.75,3.03)
                \rput(4,1.5){$\Longrightarrow$}
            \end{pspicture}
            \begin{pspicture}(-0.7,0.3)(2,2.5)
                \pswedge[fillstyle=solid,fillcolor=J1,linecolor=J1](0,0){0.6}{0}{60}
                \pswedge[fillstyle=solid,fillcolor=A1,linecolor=A1](1.75,3.03){0.7}{240}{-90}
                \psframe[fillstyle=solid,fillcolor=B1,linecolor=B1](1.75,0)(1.25,0.5)
                \pspolygon(0,0)(1.75,0)(1.75,3.03)
            \end{pspicture}}
            \end{center}
        \item Au total on obtient huit triangles, deux à deux identiques. Prendre chacun un triangle de chaque sorte. %\bigskip
        \end{enumerate}

    \partie[classement des angles]
        \begin{enumerate}
            \setcounter{enumi}{3}
            \item Colorier les trois angles de chaque triangle : combien d'angles différents obtient-on ? \pointilles\par\smallskip
            \item La taille des angles dépend-elle de la longueur des côtés ? \pointilles
            \item Classer ces angles du plus grand au plus petit en les codant A, B, C et D (le plus grand est A) : 
            
            \par\smallskip\pointilles            
            \par\smallskip\pointilles

            \textit{Ces différents angles sont appelés des \textbf{gabarits}.} %\bigskip
        \end{enumerate}

    \partie[le degré]
        Le {\bf degré} est une unité de mesure d'angle. L'angle droit mesure \ang{90}.
        \begin{enumerate}
            \setcounter{enumi}{6}
            \item À partir de l'assemblage de quels angles peut-on former l'angle A (deux possibilités) ? 
            
            \par\smallskip\pointilles\par\smallskip
            \item À partir de quels angles peut-on former l'angle B ? \pointilles
            \item Sachant que l'angle A mesure \ang{90}, retrouver la mesure des angles B, C et D.
            \par\smallskip\pointilles
            \par\smallskip\pointilles
            \par\smallskip\pointilles
        \end{enumerate}
    \end{changemargin}
    \vspace*{-20mm}
 \end{activite} 
 