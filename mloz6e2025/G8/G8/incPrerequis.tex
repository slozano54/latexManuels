\vspace*{-7mm}
\begin{changemargin}{-10mm}{-10mm}
%pre-001
\begin{prerequis}[Connaisances \emoji{red-heart} et compétences \emoji{diamond-suit} du cycle 3]    
   \begin{itemize}        
       \item[\emoji{red-heart}] Vocabulaire associé à ces objets et à leurs propriétés : côté, sommet, angle, hauteur.
       \columnbreak
       \item[\emoji{diamond-suit}] Reconnaître, nommer, décrire des triangles, dont les triangles particuliers (triangle rectangle, triangle isocèle, triangle équilatéral).       
   \end{itemize}
\end{prerequis}
\end{changemargin}
\vspace*{-13mm}

 \begin{debat}[Mot-valise]
   Un {\bf mot-valise} est un mot formé par l'accolement du début d'un mot et la fin d'un autre mot. À l'heure actuelle, on invente régulièrement des mots-valise : {\it Brexit} pour Britain et exit, {\it Twictée} pour Twitter et dictée, {\it pourriel} pour poubelle et courriel\dots \\
   Les maths n'échappent par à la règle et le mot {\it médiatrice} est un mot-valise qui vient de médiane (dans un triangle, droite joignant un sommet au milieu du côté opposé) et bissectrice (droite coupant un angle en deux angles égaux). Il a été formé en 1923, donc très récemment.
\begin{center}
    \begin{pspicture}(0,0)(3,3)
       \psline[linearc=0.2,linewidth=2mm,linecolor=gray](1,2)(1,2.4)(2,2.4)(2,2)
       \psframe[fillstyle=solid,fillcolor=B1,framearc=0.3](0,0)(3,2)
       \psframe[fillstyle=solid,fillcolor=A1](0.55,0)(0.85,2)
       \psarc(0.7,0.8){0.15}{180}{360}
       \psdot(0.7,0.85)
       \psframe[fillstyle=solid,fillcolor=A1](2.45,0)(2.15,2)
       \psarc(2.3,0.9){0.15}{180}{360}
       \psdot(2.3,0.95)
       \psset{linecolor=gray,linewidth=1.5mm}
       \psarc(0,0){0.3}{2}{88}
       \psarc(3,0){0.3}{92}{178}
       \psarc(3,2){0.3}{182}{268}
       \psarc(0,2){0.3}{272}{358}
    \end{pspicture}
\end{center}
\bigskip
\begin{cadre}[B2][F4]
   \begin{center}
      \hrefVideo{https://www.yout-ube.com/watch?v=yq9AhiPg7IY}{\bf Les mots-valises}, chaîne YouTube {\it Solidarité Laïque}.
   \end{center}
\end{cadre}  
\end{debat}