\begin{changemargin}{0mm}{-15mm}
\section{Axes de symétrie du segment}
\begin{definition}
    La \textbf{médiatrice} d'un segment est \textbf{l'axe de symétrie} de ce segment qui ne le porte pas.
\end{definition}

\begin{propriete}[\admise]
    \textbf{La médiatrice} d'un segment est la droite perpendiculaire à ce segment et qui passe par son milieu.
\end{propriete}

\begin{minipage}{0.7\linewidth}
    \begin{propriete}[\admise]
        Si on place un point sur la médiatrice d'un segment $[AB]$ alors ce point est à égale distance de $A$ et $B$.
    \end{propriete}
\end{minipage}
\begin{minipage}{0.25\linewidth}
    \begin{center}
        \begin{Geometrie}[CoinHD={(5u,4u)}]    
            pair A,B,O,I,J,K;
            path ci,cj;
            path d;
            A=u*(1,3);
            B=u*(4,1);
            O=milieu(A,B);
            trace segment(A,B);
            marque_s:=marque_s/3;
            trace marquesegment(A,B);
            trace mediatrice(A,B);
            ci=cercles(O,u);
            cj=cercles(O,2u);
            d=mediatrice(A,B);
            I= d intersectionpoint ci;
            J= d intersectionpoint cj;
            K= symetrie(J,O);
            trace codeperp(A,O,I,5);
            trace chemin(A,I,B);
            trace chemin(A,J,B);
            trace chemin(A,K,B);
            trace codesegments(A,I,I,B,2);
            trace codesegments(A,J,J,B,3);
            trace codesegments(A,K,K,B,4);
            trace codesegments(A,O,O,B,5);
            label.ulft(btex $A$ etex,A);
            label.lrt (btex $B$ etex,B);
        \end{Geometrie}
    \end{center}
\end{minipage}

\begin{minipage}{0.7\linewidth}
    \begin{propriete}[\admise]
        Si on sait qu'un point est à égale distance des extrémités d'un segment$[AB]$ alors il est situé sur la médiatrice de ce segment.
    \end{propriete}
\end{minipage}
\begin{minipage}{0.25\linewidth}
    \begin{center}
        \begin{Geometrie}[CoinHD={(5u,5u)}]
            pair E,F,G,O;
            path ce;
            path d;
            F=u*(1,3);
            G=u*(4,1);
            O=milieu(F,G);
            trace segment(F,G);
            marque_s:=marque_s/3;
            trace marquesegment(F,G);
            trace mediatrice(F,G);
            ce=cercles(O,2.5u);
            d=mediatrice(F,G);
            E= d intersectionpoint ce;
            trace codeperp(F,O,E,5);
            trace chemin(F,E,G);
            trace codesegments(F,E,E,G,3);
            trace codesegments(F,O,O,G,5);
            label.ulft(btex $F$ etex, F);
            label.lrt (btex $G$ etex, G);
            label.top (btex $E$ etex, E);
        \end{Geometrie}
    \end{center}
\end{minipage}

\begin{remarque}

    \begin{minipage}{0.4\linewidth}
    \begin{center}
        \begin{Geometrie}[CoinHD={(6u,5u)}]    
            pair A,B,O,I,J,K;
            path cj;
            path d;
            A=u*(2.5,3.5);
            B=u*(3.5,1.5);
            O=milieu(A,B);
            trace segment(A,B);
            marque_s:=marque_s/3;
            trace marquesegment(A,B);
            trace mediatrice(A,B) withcolor red withpen pencircle scaled 1.2bp;
            cj=cercles(O,3u);
            d=mediatrice(A,B);
            J= d intersectionpoint cj;
            K= symetrie(J,O);
            trace codeperp(A,O,J,5);
            trace chemin(A,J,B);
            trace chemin(A,K,B);
            trace codesegments(A,J,J,B,3);
            trace codesegments(A,K,K,B,3);
            trace codesegments(A,O,O,B,5);
            path ca,cb;
            ca = cercles(A,K);
            cb = cercles(B,K);
            draw arccercle(pointarc(cb,0),pointarc(cb,200),B) dashed evenly;
            draw arccercle(pointarc(ca,200),pointarc(ca,20),A) dashed evenly;
            label.ulft(btex $A$ etex,A);
            label.lrt(btex $B$ etex,B);
            label.top(btex $J$ etex,J);
            label.bot(btex $K$ etex,K);
        \end{Geometrie}
    \end{center}
    \end{minipage}
    \begin{minipage}{0.6\linewidth}
    Les triangles isocèles ABH et ABG peuvent être différents.\\
    C'est à dire que l'on peut avoir $(AH=BH) \neq (AG=GB)$.\\
    Cela revient à prendre deux rayons différents pour le tracé de ces deux triangles isocèles.
    \end{minipage}
\end{remarque}
\end{changemargin}

\begin{center}
    \begin{myBox}{\infoComplementsNumeriques{pluriel}}
        \begin{flushleft}        
            \hrefConstruction{http://lozano.maths.free.fr/iep_local/figures_html/scr_iep_145.html}{Construction à la règle et l'équerre}

            \smallskip
            \hrefConstruction{http://lozano.maths.free.fr/iep_local/figures_html/scr_iep_146.html}{Construction à la règle et au compas}
        \end{flushleft}
    
        \creditInstrumentPoche
    \end{myBox}
\end{center}
