\begin{activite}[Patchwork]
    \begin{changemargin}{-10mm}{-17mm}
    {\bf Objectifs :} reconnaître un axe de symétrie ; suivre un programme de construction ; conjecturer une propriété.

    Un patchwork est une technique de couture qui consiste à assembler plusieurs morceaux de tissus de tailles, formes et couleurs différentes. \medskip
       \partie[construction de la figure]
          Construire la figure ci-dessous en suivant ce programme de construction (au crayon à papier) :
          \begin{itemize}
             \item Tracer un segment vertical $[AB]$ de longueur \Lg[cm]{8} à peu près au milieu de la feuille.
             \item Placer le milieu $O$ de ce segment.
             \item Tracer la droite $(d)$ perpendiculaire à $[AB]$ passant par $O$.
             \item Placer sur la droite $(d)$ les points $I, J, K, L$ et $M$ distants de \Lg[cm]{1,5}; \Lg[cm]{3}; \Lg[cm]{4,5}; \Lg[cm]{6} et \Lg[cm]{7,5} du point O.
             \item Joindre ces points aux extrémités du segment $[AB]$.
             \item Terminer la construction par symétrie par rapport à l’axe $(AB)$.
          \end{itemize}
          \begin{center}
             \begin{pspicture}(-4,-1.2)(4,1.2)
                \psset{unit=0.8}
                \pspolygon(-3.75,0)(0,2)(3.75,0)(0,-2)
                \psset{fillstyle=solid,fillcolor=black}
                \pspolygon(2.25,0)(3,0)(0,2)
                \pspolygon(0.75,0)(1.5,0)(0,2)
                \pspolygon(0,0)(-0.75,0)(0,2)
                \pspolygon(-1.5,0)(-2.25,0)(0,2)
                \pspolygon(-3,0)(-3.75,0)(0,2)
                \pspolygon(-2.25,0)(-3,0)(0,-2)
                \pspolygon(-0.75,0)(-1.5,0)(0,-2)
                \pspolygon(0,0)(0.75,0)(0,-2)
                \pspolygon(1.5,0)(2.25,0)(0,-2)
                \pspolygon(3,0)(3.75,0)(0,-2)
             \end{pspicture}
          \end{center}
    
       \partie[analyse de la figure]
          \begin{enumerate}
             \item Que représente la droite $(d)$ pour la figure ? 
             
                \medskip
                \pointilles
             \item Mesurer à la règle au \Lg[mm]{} près les longueurs suivantes sur la figure :
             \smallskip
                \begin{center}
                   {\renewcommand{\arraystretch}{1.5}
                   \begin{tabular}{|*{5}{p{2.3cm}|}}
                      \hline
                      $IA=$ & $JA=$ & $KA=$ & $LA=$ & $MA=$ \\
                      \hline
                      $IB=$ & $JB=$ & $KB=$ & $LB=$ & $MB=$ \\
                      \hline
                   \end{tabular}}
                \end{center} 
                \smallskip
             \item Que remarque-t-on ? 
             
                \medskip             
                \pointilles 

                \bigskip
                \pointilles
         \end{enumerate}
 
       \partie[décoration de la figure]
          Repasser les traits au stylo puis colorier la figure à votre guise. \bigskip
 
    \hfill{\it\footnotesize Source : \href{http://mathsavesnes.etab.ac-lille.fr/pdf/5eme/tg2_activite_introduction_mediatrice.pdf}{mathsavesnes}, académie de Lille.}.
    \end{changemargin}
\end{activite}