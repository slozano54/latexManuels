\begin{exercice*}
   On considère le cerf-volant ci-dessous :
   \begin{center}
   \psset{unit=0.5}
      \begin{pspicture}(-1,0)(7,3.5)
         \pstGeonode[PointSymbol=none,PosAngle={180,90,0,-45}](0,0){N}(4,3){O}(6,2){U}(5,0){R}
         \pstSegmentMark{N}{O}
         \pstSegmentMark{N}{R}
         \pstSegmentMark[SegmentSymbol=MarkCros]{U}{O}
         \pstSegmentMark[SegmentSymbol=MarkCros]{U}{R}
      \end{pspicture}
      \vspace*{-5mm}
   \end{center}
   \begin{enumerate}
      \item Justifier pourquoi le point $U$ appartient à la médiatrice du segment $[OR]$.
      \item Que peut-on dire des longueurs $NO$ et $NR$ ? Justifier.
      \item En déduire que les droites $(NU)$ et $(OR)$ sont perpendiculaires.
   \end{enumerate}
\end{exercice*}
\hfill {\it\footnotesize Source : Les cahiers Sesamath 6\up{e}. Magnard-Sésamath 2017.}
