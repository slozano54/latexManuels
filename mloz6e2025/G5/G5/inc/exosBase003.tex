\begin{exercice*}
   Construire les figures suivantes :
   \begin{enumerate}
      \item
      \begin{itemize}
         \item Tracer un cercle ($\mathcal{C}$) de centre O et de rayon 4 cm puis un cercle de rayon 4 cm et passant par O.
         \item Où se trouve le centre du deuxième cercle ?
      \end{itemize}
      \item
      \begin{itemize}
         \item Tracer un segment [AB] de longueur 5 cm.
         \item Tracer le cercle ($\mathcal{C}$) de diamètre [AB].
         \item Quel est le rayon du cercle ($\mathcal{C}$) ?
      \end{itemize}
   \end{enumerate}
 \end{exercice*} 
\begin{corrige}
   Construire les figures suivantes :
   \begin{enumerate}
      \item
      \begin{itemize}
         \item Tracer un cercle ($\mathcal{C}$) de centre O et de rayon 4 cm puis un cercle de rayon 4 cm et passant par O.
         \item Où se trouve le centre du deuxième cercle ? {\red Le centre du deuxième cercle est sur le premier cercle.}
      \end{itemize}
      \item
      \begin{itemize}
         \item Tracer un segment [AB] de longueur 5 cm.
         \item Tracer le cercle ($\mathcal{C}$) de diamètre [AB].
         \item Quel est le rayon du cercle ($\mathcal{C}$) ? {\red Le rayon du cercle ($\mathcal{C}$) est de 2,5 cm.}
      \end{itemize}
   \end{enumerate}
 \end{corrige}
 