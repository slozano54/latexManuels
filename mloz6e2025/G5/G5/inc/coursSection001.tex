\section{Le cercle, le disque}
\begin{minipage}{0.7\linewidth}
    Les longueurs $OB$, $OT$, $OM$ et $OS$ mesurent toutes \Lg{4.5}.
    
    \begin{definition}
        Lorsque des points sont à la même distance d'un point donné, on dit qu'ils sont {\bfseries équidistants} de ce point.
    \end{definition}

    \begin{remarque}
        Sur la figure ci-contre, $B$, $T$, $M$ et $S$ sont équidistants de $O$.
    \end{remarque}

    \begin{definition}
        \begin{itemize}
            \item L'ensemble des points qui sont équidistants de $O$ est appelé {\red\bfseries cercle} de {\bfseries centre} $O$ et de {\bfseries rayon} la distance en question.
            \item On peut le noter $(\mathscr C)$.
            \item Un {\bfseries disque} est l'intérieur d'un cercle.
        \end{itemize}
    \end{definition}
\end{minipage}
\begin{minipage}{0.3\linewidth}
    \begin{Geometrie}[CoinHD={(5u,5u)}]
        pair B,S,T,O,M;
        O=u*(2.5,2.5);
        path cc;
        cc=cercles(O,2u);
        trace cc withcolor red;
        B=pointarc(cc,50);
        T=pointarc(cc,200);
        S=pointarc(cc,130);
        M=pointarc(cc,260);
        trace segment(O,T) dashed evenly;
        trace segment(O,M) dashed evenly;
        trace segment(O,B) dashed evenly;
        trace segment(O,S) dashed evenly;
        marque_s:=marque_s/3;
        trace Codelongueur(O,T,O,M,O,B,O,S,2);
        trace appelation(O,B,3mm,btex $\Lg{4.5}$ etex);
        label.rt(btex $B$ etex,B);
        label.lft(btex $T$ etex,T);
        label.ulft(btex $S$ etex,S);
        label.bot(btex $M$ etex,M);
        label.lrt(btex $O$ etex,O);
        label.ulft(btex $(\mathscr C)$ etex,pointarc(cc,160));
    \end{Geometrie} 
\end{minipage}
\begin{remarque}
    Un cercle se dessine en général à l'aide d'un compas.
\end{remarque}

\begin{definition}
    \begin{itemize}
       \item Un {\bf rayon} d'un cercle est un segment d'extrémité le centre du cercle et un point du cercle, c'est aussi la longueur de ce segment.
       \item Une \textbf{corde} est un segment reliant deux points du cercle.
       \item Lorsqu'une corde passe par le centre du cercle, on l'appelle un \textbf{diamètre} du cercle.
       \item Une partie du cercle comprise entre deux points est appelée \textbf{arc} de cercle.
    \end{itemize}
 \end{definition}
 
 \begin{vocabulaire}
    \phantom{rrr}

    \begin{minipage}{0.35\linewidth}    
        \psset{unit=0.9}
        \begin{pspicture}(-0.2,0.2)(4.2,4.4)            
            \psdots(2,2)
            \psarc[linecolor=A1](2,2){2}{180}{240}
            \psarc(2,2){2}{240}{180}
            \rput(2,1.6){$O$}
            \rput(0.4,3.7){$(\mathscr{C})$}
            \psline[linecolor=B1](0,2)(2,2)
            \psdots(0,2)
            \rput(-0.3,2){$A$}
            \psdots(3,3.7)
            \rput(3.2,4){$B$}
            \psdots(1,0.3)
            \rput(0.8,-0.2){$C$}
            \psline[linecolor=D1](1,0.3)(3,3.7)
            \psline[linecolor=J1](0,2)(3,3.7)
            \psdots[dotstyle=+](1,2)(2.5,2.85)(1.52,1.2)
        \end{pspicture}
    \end{minipage} 
    \begin{minipage}{0.65\linewidth}        
            \begin{itemize}
                \item \textcolor{B1}{$[OA]$ est {\bfseries un} rayon du cercle $(\mathscr{C})$.}
                \item La distance \textcolor{B1}{$OA$ est {\bfseries le} rayon du cercle $(\mathscr{C})$.}
                \item \textcolor{D1}{$[BC]$ est {\bfseries un} diamètre du cercle $(\mathscr{C})$.}
                \item La distance \textcolor{D1}{$BC$ est {\bfseries le} diamètre du cercle $(\mathscr{C})$.}
                \item \textcolor{A1}{$\wideparen{AC}$ est {\bfseries un} arc du cercle $(\mathscr{C})$.}
                \item \textcolor{J1}{$[AB]$ est {\bfseries une} corde du cercle $(\mathscr{C})$.}
                \item On a $OA = OB = OC$ et $BC =2\,OC$.
                \item Les points $C$ et $B$ sont diamètralement opposés.
                \item $O$ est le milieu du segment $[BC]$.
            \end{itemize}    
    \end{minipage}
\end{vocabulaire}

\begin{propriete}[\admise]
    Si le point $A$ est sur le cerlce de centre $O$ et de rayon $R$ alors $OA=R$.
\end{propriete}

\begin{propriete}[\admise]
    Si $OA=R$ alors le point $A$ est sur le cerlce de centre $O$ et de rayon $R$.
\end{propriete}

\begin{minipage}{0.75\linewidth}
    \begin{definition}
        \begin{itemize}
            \item Le "petit morceau" de cercle compris entre $A$ et $B$ est un
            
            {\red ARC du cercle $(\mathscr C)$}.
            \item Son centre et son rayon sont les mêmes que ceux du cercle $(\mathscr C)$.
            \item "L'arc de cercle AB" se note $\wideparen{AB}$, son angle $\widehat{AOB}$ vaut ici $60$\degre.
        \end{itemize}
    \end{definition}
\end{minipage}
\begin{minipage}{0.3\linewidth}
    \begin{Geometrie}[CoinHD={(5u,5u)}]
        pair A,B,O;
        O=u*(2.5,2.5);
        path cc;
        cc=cercles(O,2u);
        trace cc;
        A=pointarc(cc,20);
        B=pointarc(cc,80);
        trace segment(O,A) dashed evenly;
        trace segment(O,B) dashed evenly;
        trace arccercle(A,B,O) withcolor red;
        trace Codeangle(A,O,B,0,btex $\ang{60}$ etex);
        label.ulft(btex $O$ etex,O);
        label.rt(  btex $A$ etex,A);
        label.top( btex $B$ etex,B);
        label.ulft(btex $(\mathscr C)$ etex,pointarc(cc,160));
    \end{Geometrie} 
\end{minipage} 
    