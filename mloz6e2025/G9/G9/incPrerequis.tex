\vspace*{-7mm}
\begin{changemargin}{-10mm}{-10mm}
%pre-001
\begin{prerequis}[Connaisances \emoji{red-heart} et compétences \emoji{diamond-suit} du cycle 3]    
   \begin{itemize}        
       \item[\emoji{red-heart}] Vocabulaire associé à ces objets et à leurs propriétés : côté, sommet, angle, hauteur.
       \columnbreak
       \item[\emoji{diamond-suit}] Reconnaître, nommer, décrire des triangles, dont les triangles particuliers (triangle rectangle, triangle isocèle, triangle équilatéral).       
   \end{itemize}
\end{prerequis}
\end{changemargin}
\vspace*{-15mm}
\begin{debat}[Vocabulaire des quadrilatères] 
   \begin{changemargin}{-15mm}{-15mm}
   Le mot {\bf quadrilatère} provient du latin : {\it quatuor}, quatre, et {\it latus}, côté. Il existe un mot équivalent d'origine grecque : {\bf tétrapleure} de {\it tèssera}, quatre, et {\it pleura}, côté ou {\bf tétragone}, de {\it gônia}, angle. \\
   Comme pour les triangles, les quadrilatères peuvent être particuliers selon qu'ils ont certaines propriétés : parmi ceux-ci, on peut trouver par exemple la famille des trapèzes, des parallélogrammes, des rectangles, des losanges, des carrés ou encore des cerfs-volant. \\
   \end{changemargin}
   \begin{center}
      {\psset{unit=0.5}
      \begin{pspicture}(-1,-0.5)(14,7.5)
         \psframe[linecolor=red](7.25,0.25)(9.75,7.5)
         \psframe[linecolor=yellow](0.5,0.5)(9.5,2.5)
         \psframe[linecolor=orange](4.25,0)(10,5)
         \psframe[linecolor=orange!50](0.25,-0.25)(10.25,5.25)
         \psframe[linecolor=red!50](0,-0.5)(10.5,7.75)
         \psframe[linecolor=blue](-0.25,-0.75)(13.75,8)
         \psset{fillstyle=solid}
         \psframe[fillcolor=yellow](8,1)(9,2) %carré
         \psframe[fillcolor=yellow!50](5,1)(7,2) %rectangle
         \pspolygon[fillcolor=yellow!25](1,1)(3,1)(3,2)(1.5,2) %trapèze rectangle
         \pspolygon[fillcolor=orange!25](1,3.5)(3.5,3.5)(2.5,4.5)(1.5,4.5) %trapèze
         \pspolygon[fillcolor=orange!50](4.5,3.5)(6.25,3.5)(6.75,4.5)(5,4.5) %parallélogramme
         \pspolygon[fillcolor=orange](7.5,4)(8.5,3.5)(9.5,4)(8.5,4.5) %losange
         \pspolygon[fillcolor=red!50](3,6.5)(3,7)(5,7.5)(4.5,6) %convexe
         \pspolygon[fillcolor=red](8,6.75)(8.5,7.25)(9,6.75)(8.5,5.5) %cerf-volant
         \pspolygon[fillcolor=cyan!50](11,1.5)(13.5,3)(13,1.5)(11.25,2.5) %croisé
         \pspolygon[fillcolor=cyan](11,5)(13,5)(12.5,7)(12,5.5) %concave
      \end{pspicture}}
   \end{center}
   \begin{changemargin}{-15mm}{-15mm}
   \begin{cadre}[B2][F4]
      \begin{center}
         \hrefVideo{https://www.youtube.com/watch?v=j_seCDgA-lU}{\bf Pourquoi \og mathématiques \fg{} ?}, site Internet {\it m@ths-et-tiques} de {\it Yvan Monka}.
      \end{center}
   \end{cadre}
   \end{changemargin}
 \end{debat}