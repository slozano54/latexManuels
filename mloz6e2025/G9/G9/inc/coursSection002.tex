\begin{changemargin}{-5mm}{-15mm}
    \section{Un nouveau quadrilatère : le parallélogramme}
    \begin{definition}
        \begin{minipage}{0.5\linewidth}
            Un \textbf{parallélogramme} est un quadrilatère dont les côtés sont deux à deux parallèles.

            \bigskip
            \textcolor{red}{$(AB)//(DC)$} et \textcolor{blue}{$(AD)//(BC)$} 
            
            donc $ABCD$ est un parallélogramme.
        \end{minipage}
        \hfill
        \begin{minipage}{0.4\linewidth}
            \begin{Geometrie}[CoinHD={(6u,4u)}]
                pair A,B,C,D;
                A=u*(1,1);
                B-A=u*(3,0.5);
                D-A=u*(1,2);
                C-D=u*(3,0.5);
                trace droite(A,B) withcolor red;
                trace droite(D,C) withcolor red;
                trace droite(A,D) withcolor blue;
                trace droite(B,C) withcolor blue;
                label.ulft(btex $A$ etex,A);
                label.lrt(btex $B$ etex,B);
                label.ulft(btex $C$ etex,C);
                label.ulft(btex $D$ etex,D);
            \end{Geometrie}
        \end{minipage}
    \end{definition}
\end{changemargin}
