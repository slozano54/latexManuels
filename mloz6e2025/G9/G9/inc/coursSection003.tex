\section{Compléments numériques}
    \begin{myBox}{\emoji{triangular-ruler} \emoji{straight-ruler} Animations en ligne}
        \begin{flushleft}        
            \hrefLien{http://lozano.maths.free.fr/iep_local/figures_html/scr_iep_084.html}{Cerfs-volants (2 segments consécutifs égaux)}

            \smallskip
            \hrefLien{http://lozano.maths.free.fr/iep_local/figures_html/scr_iep_095.html}{Trapèze connaissant les deux bases}

            \smallskip
            \hrefLien{http://lozano.maths.free.fr/iep_local/figures_html/scr_iep_085.html}{Losange connaissant le côté}

            \smallskip
            \hrefLien{http://lozano.maths.free.fr/iep_local/figures_html/scr_iep_086.html}{Losange connaissant les diagonales (Réquerre)}

            \smallskip
            \hrefLien{http://lozano.maths.free.fr/iep_local/figures_html/scr_iep_093.html}{Rectangle côtés de 5 cm et 7 cm (avec l'équerre)}

            \smallskip
            \hrefLien{http://lozano.maths.free.fr/iep_local/figures_html/scr_iep_093.html}{Rectangle ABCD diagonales}

            \smallskip
            \hrefLien{http://lozano.maths.free.fr/iep_local/figures_html/scr_iep_094.html}{Rectangle BRCF 4 angles droits}

            \smallskip
            \hrefLien{http://lozano.maths.free.fr/iep_local/figures_html/scr_iep_082.html}{Carré inscrit dans un cercle}

            \smallskip
            \hrefLien{http://lozano.maths.free.fr/iep_local/figures_html/scr_iep_063.html}{Hexagone régulier (règle-compas)}
        \end{flushleft}

        \creditInstrumentPoche
    \end{myBox}     
