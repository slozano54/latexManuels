% Les enigmes ne sont pas numérotées par défaut donc il faut ajouter manuellement la numérotation
% si on veut mettre plusieurs enigmes
% \refstepcounter{exercice}
% \numeroteEnigme
\vspace*{-10mm}
\begin{enigme}[Des quadrilatères dans un octogone]
    \begin{changemargin}{-10mm}{-10mm}
        \begin{center}
        \fbox{
            \begin{minipage}{15cm}
                \parbox{5cm}{
                    Tracer quatre diagonales de l'octogone régulier pour obtenir le quadrilatère demandé comme dans les exemples ci-contre.}
                \parbox{5cm}{
                    \begin{pspicture}(-2.5,-2.2)(2.5,2.2)
                    \pspolygon(2;22.5)(2;67.5)(2;112.5)(2;157.5)(2;-157.5)(2;-112.5)(2;-67.5)(2;-22.5)
                    \psline(2;-22.5)(2;-157.5)
                    \psline(2;157.5)(2;22.5)
                    \psline(2;-112.5)(2;22.5)
                    \psline(2;112.5)(2;-157.5)
                    \pspolygon[fillstyle=solid,fillcolor=lightgray](2;-157.5)(0.32,-0.77)(2;22.5)(-1.21,0.77)
                    \end{pspicture}}
                \parbox{5cm}{
                    \begin{pspicture}(-2.5,-2.2)(2.5,2.2)
                    \pspolygon(2;22.5)(2;67.5)(2;112.5)(2;157.5)(2;-157.5)(2;-112.5)(2;-67.5)(2;-22.5)
                    \psline(2;67.5)(2;157.5)
                    \psline(2;67.5)(2;-112.5)
                    \psline(2;-112.5)(2;157.5)
                    \psline(2;-22.5)(2;-157.5)
                    \pspolygon[fillstyle=solid,fillcolor=lightgray](2;157.5)(2;67.5)(-0.32,-0.77)(-1.215,-0.77)
                    \end{pspicture}}
            \end{minipage}
        }
        

        {\psset{unit=0.95}
        \begin{pspicture}(-3.2,-3)(2.5,2.5)
            \pspolygon(2;22.5)(2;67.5)(2;112.5)(2;157.5)(2;-157.5)(2;-112.5)(2;-67.5)(2;-22.5)
            \rput(0,-2.5){Pour faire des essais}
            \end{pspicture}
            \begin{pspicture}(-3,-3)(2.5,2.5)
            \pspolygon(2;22.5)(2;67.5)(2;112.5)(2;157.5)(2;-157.5)(2;-112.5)(2;-67.5)(2;-22.5)
            \rput(0,-2.5){Un carré}
            \end{pspicture}
        \begin{pspicture}(-3,-3)(2.5,2.5)
            \pspolygon(2;22.5)(2;67.5)(2;112.5)(2;157.5)(2;-157.5)(2;-112.5)(2;-67.5)(2;-22.5)
            \rput(0,-2.5){Un autre carré}
        \end{pspicture}
        
        \begin{pspicture}(-3.2,-3)(2.5,2.3)
            \pspolygon(2;22.5)(2;67.5)(2;112.5)(2;157.5)(2;-157.5)(2;-112.5)(2;-67.5)(2;-22.5)
            \rput(0,-2.5){Pour faire des essais}
        \end{pspicture}
        \begin{pspicture}(-3,-3)(2.5,2.3)
            \pspolygon(2;22.5)(2;67.5)(2;112.5)(2;157.5)(2;-157.5)(2;-112.5)(2;-67.5)(2;-22.5)
            \rput(0,-2.5){Un rectangle}
        \end{pspicture}
        \begin{pspicture}(-3,-3)(2.5,2.3)
            \pspolygon(2;22.5)(2;67.5)(2;112.5)(2;157.5)(2;-157.5)(2;-112.5)(2;-67.5)(2;-22.5)
            \rput(0,-2.5){Un parallélogramme}
        \end{pspicture}

        \begin{pspicture}(-3.2,-3)(2.5,2.3)
            \pspolygon(2;22.5)(2;67.5)(2;112.5)(2;157.5)(2;-157.5)(2;-112.5)(2;-67.5)(2;-22.5)
            \rput(0,-2.5){Pour faire des essais}
        \end{pspicture}
        \begin{pspicture}(-3,-3)(2.5,2.3)
            \pspolygon(2;22.5)(2;67.5)(2;112.5)(2;157.5)(2;-157.5)(2;-112.5)(2;-67.5)(2;-22.5)
            \rput(0,-2.5){Un losange}
        \end{pspicture}
        \begin{pspicture}(-3,-3)(2.5,2.3)
            \pspolygon(2;22.5)(2;67.5)(2;112.5)(2;157.5)(2;-157.5)(2;-112.5)(2;-67.5)(2;-22.5)
            \rput(0,-2.5){Un autre losange}
        \end{pspicture}}
    \end{center}

        \hfill{\footnotesize\it Source : inspiré de \href{http://www-irem.univ-paris13.fr/site_spip/IMG/pdf/activitepapierquad1.pdf}{Angles, parallélogrammes et programmation au cycle 4}, IREM Paris-Nord.}
    \end{changemargin}
\end{enigme}

% Pour le corrigé, il faut décrémenter le compteur, sinon il est incrémenté deux fois
% \addtocounter{exercice}{-1}

% \begin{corrige}
%     Correction du binz.
% \end{corrige}
 