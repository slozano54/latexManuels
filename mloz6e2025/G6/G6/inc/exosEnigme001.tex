% Les enigmes ne sont pas numérotées par défaut donc il faut ajouter manuellement la numérotation
% si on veut mettre plusieurs enigmes
% \refstepcounter{exercice}
% \phantom{\numeroteEnigme}
\begin{changemargin}{-15mm}{-15mm}
\begin{enigme}[Shikaku]
    Le {\bf Shikaku} est un casse-tête japonais. Son nom vient du Japonais et signifie \og diviser en carrés \fg. \\
    Le but de ce jeu est de diviser une grille donnée en plusieurs rectangles. \medskip
    
    \partie[règle du jeu]
       \ \\[-11mm]
       \begin{itemize}
          \item Paver la grille à l'aide de rectangles.
          \item Chaque rectangle doit contenir un nombre et un seul.
          \item Le nombre contenu dans un rectangle indique combien de cases le constituent. \medskip
       \end{itemize}

    \partie[exemple]
       {\psset{unit=0.6}\footnotesize
       \begin{tabular}{*{6}{>{\centering\arraybackslash}p{2.5cm}}}
          \begin{pspicture}(0,0)(4,4)
             \psgrid[subgriddiv=0,gridlabels=0](0,0)(4,4)
             \rput(3.5,3.5){4}
             \rput(3.5,2.5){3}
             \rput(2.5,2.5){6}
             \rput(2.5,0.5){1}
             \rput(0.5,0.5){2}
          \end{pspicture}
          &
          \begin{pspicture}(0,0)(4,4)
             \psgrid[subgriddiv=0,gridlabels=0](0,0)(4,4)
             \rput(3.5,3.5){4}
             \rput(3.5,2.5){3}
             \rput(2.5,2.5){6}
             \rput(2.5,0.5){1}
             \rput(0.5,0.5){2}
            \psset{linewidth=0.8mm,linecolor=PartieStatistique}
             \psframe(2,0)(3,1)
          \end{pspicture}
          &
          \begin{pspicture}(0,0)(4,4)
             \psgrid[subgriddiv=0,gridlabels=0](0,0)(4,4)
             \rput(3.5,3.5){4}
             \rput(3.5,2.5){3}
             \rput(2.5,2.5){6}
             \rput(2.5,0.5){1}
             \rput(0.5,0.5){2}
             \psset{linewidth=0.8mm,linecolor=PartieStatistique}
             \psframe(2,0)(3,1)
             \psframe(0,3)(4,4)
          \end{pspicture}
          &
          \begin{pspicture}(0,0)(4,4)
             \psgrid[subgriddiv=0,gridlabels=0](0,0)(4,4)
             \rput(3.5,3.5){4}
             \rput(3.5,2.5){3}
             \rput(2.5,2.5){6}
             \rput(2.5,0.5){1}
             \rput(0.5,0.5){2}
             \psset{linewidth=0.8mm,linecolor=PartieStatistique}
             \psframe(2,0)(3,1)
             \psframe(0,3)(4,4)
             \psframe(3,0)(4,3)
          \end{pspicture}
          &
          \begin{pspicture}(0,0)(4,4)
             \psgrid[subgriddiv=0,gridlabels=0](0,0)(4,4)
             \rput(3.5,3.5){4}
             \rput(3.5,2.5){3}
             \rput(2.5,2.5){6}
             \rput(2.5,0.5){1}
             \rput(0.5,0.5){2}
             \psset{linewidth=0.8mm,linecolor=PartieStatistique}
             \psframe(2,0)(3,1)
             \psframe(0,3)(4,4)
             \psframe(3,0)(4,3)
             \psframe(0,1)(3,3)
          \end{pspicture}
          &
          \begin{pspicture}(0,0)(4,4)
             \psgrid[subgriddiv=0,gridlabels=0](0,0)(4,4)
             \rput(3.5,3.5){4}
             \rput(3.5,2.5){3}
             \rput(2.5,2.5){6}
             \rput(2.5,0.5){1}
             \rput(0.5,0.5){2}
             \psset{linewidth=0.8mm,linecolor=PartieStatistique}
             \psframe(2,0)(3,1)
             \psframe(0,3)(4,4)
             \psframe(3,0)(4,3)
             \psframe(0,1)(3,3)
             \psframe(0,0)(2,1)
          \end{pspicture}
          \\
          grille d'origine
          &
          un rectangle à 1 case est forcément un carré de côté 1
          &
          un rectangle à 4 cases est un carré de côté 2 ou un rectangle de côtés 1 et 4
          &
          un rectangle à 3 cases est un rectangle de côtés 1 et 3
          &
          un rectangle à 6 cases est un rectangle de côtés 2 et 3 ou de côtés 1 et 6
          &
          un rectangle à 2 cases est un rectangle de côtés 1 et 2 \\
       \end{tabular}} \smallskip

    \partie[let's go !!!]
        {\renewcommand{\arraystretch}{1.49}
        \scalebox{0.9}{
        \begin{tabular}{|*{4}{>{\centering\arraybackslash}p{0.33cm}|}}
          \hline
          2 & 2 & 4 & \\
          \hline
          & & & \\
          \hline
          2 & 3 & & \\
          \hline
          & & & 3 \\
          \hline
       \end{tabular}
        }        
       \hfill
       \scalebox{0.9}{
       \begin{tabular}{|*{5}{>{\centering\arraybackslash}p{0.33cm}|}}
          \hline
          3 & & & 4 & \\
         \hline
          & & 2 & & \\
          \hline
          & & & 4 & \\
          \hline
          & 2 & 2 & 2 & \\
          \hline
          2 & & 4 & & \\
          \hline
       \end{tabular}
       }
       \hfill
       \scalebox{0.9}{
       \begin{tabular}{|*{6}{>{\centering\arraybackslash}p{0.33cm}|}}
          \hline
          4 & 5 & & & 3 & \\
          \hline
          & & 3 & & & \\
          \hline
          & & & 6 & & \\
          \hline
          & & & & 4 & \\
          \hline
          & & & 2 & 2 & \\
          \hline
          2 & & 5 & & & \\
          \hline
       \end{tabular}
        }
        
       \scalebox{0.9}{
       \begin{tabular}{|*{7}{>{\centering\arraybackslash}p{0.33cm}|}}
          \hline
          & & & & & & 2 \\
          \hline
          & & & 6 & 2 & 2 & 2 \\
          \hline
          & 2 & & 4 & & & \\
          \hline
          6 & & 3 & & & & \\
          \hline
          & 3 & & & & 4 &  \\
          \hline
          & & & 4 & & 2 & 2 \\
          \hline
          & 2 & & & 3 & & \\
          \hline
       \end{tabular}
       }
       \hfill
       \scalebox{0.9}{
       \begin{tabular}{|*{9}{>{\centering\arraybackslash}p{0.33cm}|}}
          \hline
          2 & & & & & & & & \\
          \hline
          2 & & 6 & & & 3 & 2 & 4 & \\
          \hline
          & & & 2 & & & & 2 & \\
          \hline
          & & & 2 & & 2 & & & \\
          \hline
          5 & & 6 & & & 2 & 3 & 4 & \\
          \hline
          & 7 & & 3 & & & & & \\
          \hline
          & & & & & & & 6 & \\
          \hline
          & & & & 9 & & & & \\
         \hline
          & 3 & & & 4 & & & 2 & \\
          \hline
       \end{tabular}
        }
    }
\end{enigme}
\end{changemargin}