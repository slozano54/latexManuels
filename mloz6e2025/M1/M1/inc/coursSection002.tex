\section{Unités de longueur}

On peut mesurer une longueur grâce au mètre (m) et à toutes les unités qui en découlent :
   \begin{center}
   \begin{CLtableau}{0.9\linewidth}{8}{p{3cm}}
      \hline
      Préfixe & kilo & hecto & déca & & déci & centi & milli \\
      \hline
      Signification & 1 000 & 100 & 10 & 1 & 1/10 & 1/100 & 1/1 000 \\
      \hline
      Unité de longueur & km & hm & dam & m & dm & cm & mm \\
      \hline
      Exemple & & $9$ & $7$ & $3$ & $2$ & $1$ & \\
      \hline
   \end{CLtableau}
   \end{center}

\begin{exemple*1}
   $\ukm{0,97321} =\uhm{9,7321} =\udam{97,321} =\um{973,21}=\udm{9732,1} =\ucm{97321} =\umm{973210}$. Pour passer d'une unité à une autre, on multiplie/divise par 10, 100, 1 000 \dots
\end{exemple*1}

\begin{definition}
   La {\bf longueur} $AB$ d'un segment [$AB$] est la distance qui sépare ses deux extrémités $A$ et $B$.
\end{definition}