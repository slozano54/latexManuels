\section{Utilisation d'un quadrillage}    
    % utilisation d'un quadrillage
    \begin{methode*1}[Image d'un point à l'aide d'un quadrillage]
      Tracer le symétrique du point $A$ par rapport à un axe à l'aide du quadrillage.
       \exercice
         {\bf Dans un quadrillage :} si l'axe est horizontal ou vertical, il suffit de reporter le nombre de carreaux séparant le point de l'axe de l'autre côté de cet axe. Si le quadrillage est en diagonale : 
         
         \medskip
         \begin{tabular}{>{\centering\arraybackslash}p{50mm}>{\centering\arraybackslash}p{50mm}>{\centering\arraybackslash}p{50mm}}   
         \begin{Geometrie}[CoinHD={(4.21u,3.51u)}]                        
            trace grille(0.7);
            pair axe[],A;
            axe0=u*(0.7,0);
            axe1=u*(4.2,3.5);
            A=u*(0.7,2.8);
            marque_p:="croix";
            pointe(A);
            trace segment(axe0,axe1) withcolor red withpen pencircle scaled 1bp;
            label.llft(TEX("$A$"),A);
            label.ulft(TEX("$(d)$"),(3.5u,2.8u)) withcolor red;
         \end{Geometrie}
            &
         \begin{Geometrie}[CoinHD={(4.21u,3.51u)}]                        
            trace grille(0.7) dashed withdots;
            pair axe[],A,H;
            axe0=u*(0.7,0);
            axe1=u*(4.2,3.5);
            A=u*(0.7,2.8);
            H=projection(A,axe0,axe1);
            marque_p:="croix";
            pointe(A);
            trace segment(axe0,axe1) withcolor red withpen pencircle scaled 1bp;
            label.llft(TEX("$A$"),A);
            label.ulft(TEX("$(d)$"),(3.5u,2.8u)) withcolor red;
            trace segment(A,H) withcolor blue;
            path cc[];
            pair I[];
            I0=iso(A,iso(A,H));
            I1=iso(H,iso(A,H));
            cc0=cercles(I0,u*0.35*sqrt(2));            
            drawarrow pointarc(cc0,135)..pointarc(cc0,45)..pointarc(cc0,-45) withcolor blue;
            cc1=cercles(I1,u*0.35*sqrt(2));            
            drawarrow pointarc(cc1,135)..pointarc(cc1,45)..pointarc(cc1,-45) withcolor blue;
            label.urt(TEX("\num{1}"),I0) withcolor blue;
            label.urt(TEX("\num{2}"),I1) withcolor blue;
         \end{Geometrie}
            &
         \begin{Geometrie}[CoinHD={(4.21u,3.51u)}]                        
            trace grille(0.7) dashed withdots;
            pair axe[],A,H;
            axe0=u*(0.7,0);
            axe1=u*(4.2,3.5);
            A=u*(0.7,2.8);
            H=projection(A,axe0,axe1);
            marque_p:="croix";
            pointe(A);
            trace segment(axe0,axe1) withcolor red withpen pencircle scaled 1bp;
            label.llft(TEX("$A$"),A);
            label.ulft(TEX("$(d)$"),(3.5u,2.8u)) withcolor red;
            trace segment(A,H) withcolor blue;
            path cc[];
            pair I[];
            I0=iso(A,iso(A,H));
            I1=iso(H,iso(A,H));
            cc0=cercles(I0,u*0.35*sqrt(2));            
            drawarrow pointarc(cc0,135)..pointarc(cc0,45)..pointarc(cc0,-45) withcolor blue;
            cc1=cercles(I1,u*0.35*sqrt(2));            
            drawarrow pointarc(cc1,135)..pointarc(cc1,45)..pointarc(cc1,-45) withcolor blue;
            label.urt(TEX("\num{1}"),I0) withcolor blue;
            label.urt(TEX("\num{2}"),I1) withcolor blue;
            I2=symetrie(I1,axe0,axe1);
            I3=symetrie(I0,axe0,axe1);
            pair A';
            A'=symetrie(A,axe0,axe1);
            pointe(A');
            label.urt(TEX("$A'$"),A');
            trace segment(H,A') withcolor DarkGreen;
            cc2=cercles(I2,u*0.35*sqrt(2));            
            drawarrow pointarc(cc2,135)..pointarc(cc2,45)..pointarc(cc2,-45) withcolor DarkGreen;
            cc3=cercles(I3,u*0.35*sqrt(2));            
            drawarrow pointarc(cc3,135)..pointarc(cc3,45)..pointarc(cc3,-45) withcolor DarkGreen;
            label.urt(TEX("\num{1}"),I2) withcolor DarkGreen;
            label.urt(TEX("\num{2}"),I3) withcolor DarkGreen;
         \end{Geometrie}
         \\
            symétrique du point $A$ par rapport à $(d)$ :
            &
            compter le nombre de diagonales entre le point et la droite,
            &
            reporter ce nombre de l'autre côté en diagonal. \\
         \end{tabular}
         \vspace*{-5mm}
   \end{methode*1}