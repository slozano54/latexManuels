\begin{exercice*}
   Compléter le tableau pour donner le volume de chaque solide en unités de volume. Tous les empilements sont suposés pleins, mais selon les cas plusieurs vues du solide sont proposées.

   \begin{center}
   \raisebox{-0.5\totalheight}[0.5\totalheight]{\raisebox{\depth}{    
      \begin{Geometrie}[CoinBG={(-u,0)},CoinHD={(3u,3u)}]
          u:=0.4*u;
          pair A,B,C,D,E,F,G,H;
          A=u*(1,1);
          B=u*(2,1);
          F=u*(2,2);
          E=u*(1,2);
          D=A shifted (0.5u,0.4u);
          C=B shifted (0.5u,0.4u);
          H=E shifted (0.5u,0.4u);
          G=F shifted (0.5u,0.4u);
          remplis polygone(B,C,G,F) withcolor Cornsilk;
          remplis polygone(E,H,G,F) withcolor 0.7Cornsilk+0.3black;
          remplis polygone(A,B,F,E) withcolor 0.9Cornsilk+0.1black;
          trace A--B--F--E--cycle;
          trace polygone(B,C,G,F);
          trace chemin(G,H,E);
          pair O;
          O=u*(1.5,0.5);
          label(btex {\bfseries 1 unité de volume} etex,O);
      \end{Geometrie}
      }}

      {\renewcommand*{\arraystretch}{1.2}
      \begin{tabular}{|>{\centering\arraybackslash}m{0.5\linewidth}|*{4}{>{\centering\arraybackslash}m{0.1\linewidth}|}}%
          \hline
          \rowcolor{gray!20}{\bf Solide} & {\bf n°1} & {\bf n°2}  & {\bf n°3} & {\bf n°4} \\
          \hline
          {Nombre de petits cubes manquant pour former le grand cube}&$0$&&&\\
          \hline
          {Volume en unités de volume}&&&&\\
          \hline
      \end{tabular}
      }
   \end{center}
   \begin{enumerate}
      \item Solide n°1
      
         \begin{Geometrie}[CoinHD={(5u,4u)}]
            u:=0.4*u;
            pair A,B,C,D,E,F,G,H;
            A=u*(1,1);
            B=u*(2,1);
            F=u*(2,2);
            E=u*(1,2);
            pair Vx,Vy,Vz;
            Vx=u*(1,0);
            Vy=u*(0.5,0.4);
            Vz=u*(0,1);
            D=A shifted Vy;
            C=B shifted Vy;
            H=E shifted Vy;
            G=F shifted Vy;
            picture cube;
            cube = image(
               remplis polygone(B,C,G,F) withcolor Cornsilk;
               remplis polygone(E,H,G,F) withcolor 0.7Cornsilk+0.3black;
               remplis polygone(A,B,F,E) withcolor 0.9Cornsilk+0.1black;
               trace A--B--F--E--cycle;
               trace polygone(B,C,G,F);
               trace chemin(G,H,E);
            );        
            trace cube;
            for j=0 upto 2:
               for k=0 upto 2:
                  for i=0 upto 2:
                        trace cube shifted (k*Vx+(2-j)*Vy+i*Vz);
                  endfor;
               endfor;
            endfor;
         \end{Geometrie}
      \item Solide n°2
     
         \VueCubes[%
            Creation,%
            Profondeur=3,%
            Largeur=3,%
            Angle=45,%
            CouleurCube=Cornsilk]{%
               3,3,3,%
               1,3,3,%
               1,2,3%
         }

         \VueCubes[%
            Creation,%
            Profondeur=3,%
            Largeur=3,%
            Angle=-20,%
            CouleurCube=Cornsilk]{%
                  3,3,3,%
                  1,3,3,%
                  1,2,3%
         }
      \item Solide n°3
      
         \VueCubes[%
            Creation,%
            Profondeur=3,%
            Largeur=3,%
            Angle=45,%
            CouleurCube=Cornsilk]{%
                  1,2,3,%
                  1,1,2,%
                  1,2,3%
         }

         \VueCubes[%
            Creation,%
            Profondeur=3,%
            Largeur=3,%
            Angle=-20,%
            CouleurCube=Cornsilk]{%
                  1,2,3,%
                  1,1,2,%
                  1,2,3%
         }
      \item Solide n°4 : ce solide est percé de part en part, au centre de chaque face.
 
         \begin{Geometrie}
            u:=0.4*u;
            pair A,B,C,D,E,F,G,H;
            A=u*(1,1);
            B=u*(2,1);
            F=u*(2,2);
            E=u*(1,2);
            pair Vx,Vy,Vz;
            Vx=u*(1,0);
            Vy=u*(0.5,0.4);
            Vz=u*(0,1);
            D=A shifted Vy;
            C=B shifted Vy;
            H=E shifted Vy;
            G=F shifted Vy;
            picture cube;
            cube = image(
               remplis polygone(B,C,G,F) withcolor Cornsilk;
               remplis polygone(E,H,G,F) withcolor 0.7Cornsilk+0.3black;
               remplis polygone(A,B,F,E) withcolor 0.9Cornsilk+0.1black;
               trace A--B--F--E--cycle;
               trace polygone(B,C,G,F);
               trace chemin(G,H,E);
            );
            picture colonneTypeUn,colonneTypeDeux;
            colonneTypeUn = image(
            for k=0 upto 2:
               trace cube shifted (k*Vz);
            endfor;
            );
            colonneTypeDeux = image(
            trace cube;
            trace cube shifted (2*Vz);
            );
            % tranche Gauche
            trace colonneTypeUn shifted (2*Vy);
            trace colonneTypeDeux shifted (Vy);
            trace colonneTypeUn;
            % tranche Centre                
            trace colonneTypeDeux shifted (Vx+2*Vy);
            trace colonneTypeDeux shifted (Vx);
            % tranche Droite                
            trace colonneTypeUn shifted (2*Vx+2*Vy);
            trace colonneTypeDeux shifted (2*Vx+Vy);                
            trace colonneTypeUn shifted (2*Vx);                
         \end{Geometrie}
   \end{enumerate}
\end{exercice*}
\begin{corrige}
    Pas de correction
\end{corrige} 