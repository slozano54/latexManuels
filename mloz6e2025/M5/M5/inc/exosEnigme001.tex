% Les enigmes ne sont pas numérotées par défaut donc il faut ajouter manuellement la numérotation
% si on veut mettre plusieurs enigmes
%\refstepcounter{exercice}
%\numeroteEnigme
\vspace*{-10mm}
\begin{enigme}[Visualiser pour calculer]
   \partie[Partie 1 - Accès à l'activité]
   
   Aller sur le site Internet de l'IREM de Paris-Nord : Rubricamaths, rubrique \og Visualiser pour calculer (1) \fg. \\
   Voilà son adresse : \url{https://www-irem.univ-paris13.fr/site_spip/spip.php?rubrique121} \\
   
   \begin{tabular}{>{\centering\arraybackslash}p{8cm}|>{\centering\arraybackslash}p{8cm}}
      Choisir l'un des solides proposés dans la partie 2 en cliquant sur {\red\underline{En ligne}} & Pour chacun de ces solides, indiquer le volume en \og cubes \fg. Un résultat juste apparaitra sur fond vert, un résultat faux apparaitra sur fond rouge. \\ 
      \includegraphics[width=0.65\linewidth]{\currentpath/images/rubricamaths1} & \includegraphics[width=0.65\linewidth]{\currentpath/images/rubricamaths2} \\
   \end{tabular}
   
   \partie[Partie 2 - Restitution des résultats]
   Pour chaque solide, écrire le résultat (juste) que vous avez obtenu sur le site Internet dans le tableau suivant. \\
   Vous avez le droit à autant d'essais que vous le souhaitez.
   \begin{center}
      \includegraphics[width=\linewidth]{\currentpath/images/rubricamaths3}
   \end{center}
\end{enigme}  
% % Pour le corrigé, il faut décrémenter le compteur, sinon il est incrémenté deux fois
% \addtocounter{exercice}{-1}
% \begin{corrige}
%     Correction enigme de la fin de la partie cours.  
%     
% \end{corrige}