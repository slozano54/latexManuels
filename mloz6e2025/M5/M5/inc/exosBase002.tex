\begin{exercice*}
   Ci-dessous, de gauche à droite, l'unité de volume, la pyramide de niveau 1, la pyramide de niveau 2.

   \scalebox{0.5}{
      \begin{Geometrie}[CoinHD={(16u,6u)}]
         trace papierisometrique withcolor gris;
         pair A,B,C,D,E,F,G,H;
         A=ppiso(-1,2);
         B=ppiso(0,1);
         C=ppiso(0,2);
         D=ppiso(-1,3);
         E=ppiso(0,0);
         F=ppiso(1,-1);
         G=ppiso(1,0);
         pair legende;
         legende = ppiso(-1.5,4);
         label(btex {\Large\bfseries 1 u.v.} etex,legende);
         picture cube;
         cube = image(
         remplis polygone(A,B,C,D) withcolor 0.7Cornsilk+0.3black;
         remplis polygone(A,B,F,E) withcolor 0.9Cornsilk+0.1black;
         remplis polygone(B,F,G,C) withcolor Cornsilk;
         trace polygone(A,B,C,D);
         trace polygone(A,B,F,E);
         trace polygone(B,F,G,C);
         );
         % Les vecteurs pour les translations
         pair Vx,Vy,Vz;
         Vx=ppiso(1,-1);
         Vy=ppiso(0,0);
         Vz=ppiso(-1,1);
         % Cube unité
         trace cube;
         % Pyramide de niveau 1
         pair legPyrUn;
         legPyrUn = legende shifted (2*Vx+Vy+Vz);
         label(btex {\Large\bfseries Pyramide de niveau 1} etex,legPyrUn);
         trace cube shifted (2*Vx);
         trace cube shifted (3*Vx);
         trace cube shifted (2*Vx+Vy+Vz);
         % Pyramide de niveau 2
         pair legPyrDeux;
         legPyrDeux = legende shifted (5*Vx+2*Vy+2*Vz);
         label(btex {\Large\bfseries Pyramide de niveau 2} etex,legPyrDeux);
         trace cube shifted (5*Vx);
         trace cube shifted (6*Vx);
         trace cube shifted (5*Vx+Vy+Vz);
         trace cube shifted (7*Vx);
         trace cube shifted (6*Vx+Vy+Vz);
         trace cube shifted (5*Vx+2*Vy+2*Vz);
      \end{Geometrie}
   }

   \begin{enumerate}
      \item Construire la pyramide de niveau 3, ci-dessous.

            \scalebox{0.5}{
               \begin{Geometrie}[CoinHD={(16u,8u)}]
                  trace papierisometrique withcolor gris;
               \end{Geometrie}
            }
      \item Déterminer le volume de chaque pyramide, en unités de volume (u.v.).
   \end{enumerate}
\end{exercice*}
\begin{corrige}
   Ci-dessous, de gauche à droite, l'unité de volume, la pyramide de niveau 1, la pyramide de niveau 2.

   \hspace*{-5mm}
   \scalebox{0.5}{
      \begin{Geometrie}[CoinHD={(16u,6u)}]
         trace papierisometrique withcolor gris;
         pair A,B,C,D,E,F,G,H;
         A=ppiso(-1,2);
         B=ppiso(0,1);
         C=ppiso(0,2);
         D=ppiso(-1,3);
         E=ppiso(0,0);
         F=ppiso(1,-1);
         G=ppiso(1,0);
         pair legende;
         legende = ppiso(-1.5,4);
         label(btex {\Large\bfseries 1 u.v.} etex,legende);
         picture cube;
         cube = image(
         remplis polygone(A,B,C,D) withcolor 0.7Cornsilk+0.3black;
         remplis polygone(A,B,F,E) withcolor 0.9Cornsilk+0.1black;
         remplis polygone(B,F,G,C) withcolor Cornsilk;
         trace polygone(A,B,C,D);
         trace polygone(A,B,F,E);
         trace polygone(B,F,G,C);
         );
         % Les vecteurs pour les translations
         pair Vx,Vy,Vz;
         Vx=ppiso(1,-1);
         Vy=ppiso(0,0);
         Vz=ppiso(-1,1);
         % Cube unité
         trace cube;
         % Pyramide de niveau 1
         pair legPyrUn;
         legPyrUn = legende shifted (2*Vx+Vy+Vz);
         label(btex {\Large\bfseries Pyramide de niveau 1} etex,legPyrUn);
         trace cube shifted (2*Vx);
         trace cube shifted (3*Vx);
         trace cube shifted (2*Vx+Vy+Vz);
         % Pyramide de niveau 2
         pair legPyrDeux;
         legPyrDeux = legende shifted (5*Vx+2*Vy+2*Vz);
         label(btex {\Large\bfseries Pyramide de niveau 2} etex,legPyrDeux);
         trace cube shifted (5*Vx);
         trace cube shifted (6*Vx);
         trace cube shifted (5*Vx+Vy+Vz);
         trace cube shifted (7*Vx);
         trace cube shifted (6*Vx+Vy+Vz);
         trace cube shifted (5*Vx+2*Vy+2*Vz);
      \end{Geometrie}
   }

   \begin{enumerate}
      \item Construire la pyramide de niveau 3, ci-dessous.

            \hspace*{-5mm}
            \scalebox{0.5}{
               \begin{Geometrie}[CoinHD={(16u,8u)}]
                  trace papierisometrique withcolor gris;
                  pair A,B,C,D,E,F,G,H;
                  A=ppiso(-1,2);
                  B=ppiso(0,1);
                  C=ppiso(0,2);
                  D=ppiso(-1,3);
                  E=ppiso(0,0);
                  F=ppiso(1,-1);
                  G=ppiso(1,0);
                  picture cube;
                  cube = image(
                  remplis polygone(A,B,C,D) withcolor 0.7Tomato+0.3black;
                  remplis polygone(A,B,F,E) withcolor 0.9Tomato+0.1black;
                  remplis polygone(B,F,G,C) withcolor Tomato;
                  trace polygone(A,B,C,D);
                  trace polygone(A,B,F,E);
                  trace polygone(B,F,G,C);
                  );
                  % Les vecteurs pour les translations
                  pair Vx,Vy,Vz;
                  Vx=ppiso(1,-1);
                  Vy=ppiso(0,0);
                  Vz=ppiso(-1,1);
                  % Pyramide de niveau 3
                  trace cube shifted (2*Vx);
                  trace cube shifted (3*Vx);
                  trace cube shifted (2*Vx+Vy+Vz);
                  trace cube shifted (4*Vx);
                  trace cube shifted (3*Vx+Vy+Vz);
                  trace cube shifted (2*Vx+2*Vy+2*Vz);
                  trace cube shifted (5*Vx);
                  trace cube shifted (4*Vx+Vy+Vz);
                  trace cube shifted (3*Vx+2*Vy+2*Vz);
                  trace cube shifted (2*Vx+3*Vy+3*Vz);
               \end{Geometrie}
            }
      \item Déterminer le volume de chaque pyramide, en unités de volume (u.v.).

               {\red
                  Pyramide de niveau 1 : $1+3=4$;\\
                  Pyramide de niveau 2 : $1+3+6=10$;\\
                  Pyramide de niveau 3 : $1+3+6+10=20$;
               }
   \end{enumerate}
\end{corrige}