\section{Triangles particuliers} %%%%%%%%%

\begin{vocabulaire}

    \begin{minipage}{0.2\linewidth}
        Triangle :
        \begin{itemize}            
            \item 3 angles
            \item 3 côtés
            \item 3 sommets
            \item 3 hauteurs
        \end{itemize}
    \end{minipage}
    \begin{minipage}{0.65\linewidth}
        \begin{pspicture}(-1,-0.5)(10,2.5)
            \pstTriangle[PointSymbol=none,linecolor=A1](0,0){U}(4,0){I}(-1,2){O}
            \pstTriangle[PointSymbol=none,linecolor=A1](6,0){Y}(10,0){S}(7,2){E}
            \rput(2,2){\textcolor{A1}{côté}}
            \psline[linecolor=A1]{->}(1.5,1.8)(1.2,1.2)
            \rput(9,2){sommet}
            \psline{->}(8.2,2)(7.3,2)
            \pstMarkAngle[linecolor=J1,MarkAngleType=double,MarkAngleRadius=1,LabelSep=1.8]{O}{I}{U}{\textcolor{J1}{angle}}
            \psline[linestyle=dashed,dash=.1,linecolor=A1](-1,0)(0,0)
            \psset{linecolor=B1}
            \psline(-1,0)(-1,2)
            \psline(-1,0.25)(-0.75,0.25)(-0.75,0)            
            \psline(7,0)(7,2)
            \psframe(7,0)(7.25,0.25)
            \rput(5,1.5){\textcolor{B1}{hauteur}}
            \psline{->}(5.8,1.3)(6.8,1)
        \end{pspicture}
    \end{minipage}
\end{vocabulaire}

\begin{definition}
    Un \textbf{triangle} est un polygone (plusieurs-angles) à trois côtés.
\end{definition}
 
\begin{vocabulaire}

    \begin{minipage}{0.4\linewidth}
        Triangle $ABC$ :
        \begin{itemize}            
            \item 3 angles $\widehat{CBA}$; $\widehat{ACB}$; $\widehat{BAC}$.
            \item 3 côtés $[AB]$; $[BC]$; $[CA]$.
            \item 3 sommets $A$; $B$; $C$.
            \item sommet opposé au côté $[BC]$ : $A$.
            \item côté opposé au sommet $A$ : $[BC]$.
        \end{itemize}
    \end{minipage}
    \begin{minipage}{0.55\linewidth}
        \begin{pspicture}(-1,-0.5)(10,2.5)            
            \pstTriangle[PointSymbol=none,linecolor=A1](0,0){B}(4,0){C}(1,2){A}            
            \rput(4.5,2){sommet opposé au côté $[BC]$}
            \psline{->}(2.2,2)(1.2,2)
            \rput(4,-0.7){côté opposé au sommet $A$}
            \psline{->}(2.2,-0.5)(2,-0.1)
            \pstMarkAngle[linecolor=J1,MarkAngleType=double,MarkAngleRadius=0.5,LabelSep=0.95]{C}{B}{A}{\textcolor{J1}{$\widehat{CBA}$}}
            \pstMarkAngle[linecolor=J1,MarkAngleType=triple,MarkAngleRadius=0.5,LabelSep=1]{A}{C}{B}{\textcolor{J1}{$\widehat{ACB}$}}
            \pstMarkAngle[linecolor=J1,MarkAngleRadius=0.4,LabelSep=0.65]{B}{A}{C}{\textcolor{J1}{$\widehat{BAC}$}}
        \end{pspicture}
    \end{minipage}
\end{vocabulaire}
\begin{definition}
   \begin{itemize}
        \item Un \textbf{triangle rectangle} est un triangle ayant un angle droit.
        \item Un \textbf{triangle isocèle} est un triangle ayant deux côtés de même longueur ou plus.
        \item Un triangle \textbf{équilatéral} est un triangle dont tous les côtés sont de même longueur.
   \end{itemize}
\end{definition}


{\psset{unit=0.9}
\begin{pspicture}(-2,-1)(4,4)
   \pstTriangle[PointSymbol=none](0,0){E}(4,0){C}(0,2){R}
   \pstRightAngle[linecolor=B1]{R}{E}{C}
   \rput(2,3.5){\bf triangle rectangle}
   \rput(3,1.7){hypoténuse}
   \psline{->}(2.8,1.5)(2,1.1)
\end{pspicture}
\begin{pspicture}(-1.5,-1)(5.5,4)
   \pstTriangle[PointSymbol=none](0,0){I}(4,0){O}(2,2){S}
   \pstSegmentMark[SegmentSymbol=MarkHashh,MarkAngle=90,linecolor=A1]{I}{S} 
   \pstSegmentMark[SegmentSymbol=MarkHashh,MarkAngle=90,linecolor=A1]{S}{O}
   \pstLineAB[linecolor=A1]{I}{O}
   \pstMarkAngle[linecolor=J1,MarkAngleType=double]{S}{O}{I}{}
   \pstMarkAngle[linecolor=J1,MarkAngleType=double]{O}{I}{S}{}
   \rput(2,3.5){\bf triangle isocèle}
   \rput(3.8,2.5){sommet principal}
   \psline{->}(3.2,2.3)(2.1,2)
   \rput(3,-0.7){base}
   \psline{->}(2.8,-0.5)(2,-0.1)
\end{pspicture}
\begin{pspicture}(0,-1)(5,4)
   \pstTriangle[PointSymbol=none](0,0){E}(3,0){I}(1.5,2.55){Q}
   \psset{SegmentSymbol=MarkHash,MarkAngle=90,linecolor=A1}
   \pstSegmentMark{E}{Q} 
   \pstSegmentMark{I}{Q}
   \pstSegmentMark{I}{E}
   \psset{linecolor=A1}
   \pstMarkAngle[linecolor=J1]{Q}{I}{E}{}
   \pstMarkAngle[linecolor=J1]{I}{E}{Q}{}
   \pstMarkAngle[linecolor=J1]{E}{Q}{I}{}
   \rput(1.5,3.5){\bf triangle équilatéral}
\end{pspicture}}

\begin{remarques}
   un triangle équilatéral et un triangle isocèle particulier et un triangle rectangle peut être isocèle, mais pas équilatéral.
\end{remarques}