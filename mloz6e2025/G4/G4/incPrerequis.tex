%pre-001
\begin{prerequis}[Connaisances \emoji{red-heart} et compétences \emoji{diamond-suit} du cycle 3]    
   \begin{itemize}        
       \item[\emoji{red-heart}] Vocabulaire associé à ces objets et à leurs propriétés : côté, sommet, angle, hauteur.
       \columnbreak
       \item[\emoji{diamond-suit}] Reconnaître, nommer, décrire des triangles, dont les triangles particuliers (triangle rectangle, triangle isocèle, triangle équilatéral).       
   \end{itemize}
\end{prerequis}

\vfill

\begin{debat}[Vocabulaire du triangle] 
    Le mot {\bf triangle} est construit à partir du préfixe \og tri \fg, trois et du mot angle, c'est donc un polygone à trois\dots{} angles ! À la renaissance, on hésite entre triangle et trigone (du grec trigônum). Les mathématiciens choisissent {\it triangle} et les astrologues {\it trigone}, qui donnera le mot {\it trigonométrie}, branche des mathématiques qui traite des relations entre distances et angles dans les triangles.
    \begin{center}
       \begin{pspicture}(0,-0.5)(4.3,4.2)
          \pspolygon[fillstyle=solid,fillcolor=A2](0,1.6)(4,0)(4.2,3.8)
          \rput[r](-0.3,1.6){Floride}
          \rput[l](4.4,4){Bermudes}
          \rput[l](4.3,0){Porto Rico}
       \end{pspicture}
    \end{center}
    \begin{cadre}[B2][F4]
       \begin{center}
          \hrefVideo{https://www.youtube.com/watch?v=MUr7dgn8wdo}{\bf Le triangle des Bermudes}, chaîne YouTube {\it SYMPA}.
       \end{center}
    \end{cadre}
 \end{debat}