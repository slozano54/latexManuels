\begin{exercice}
    Dans le nombre $\num{6083472}$ donner :
    \begin{enumerate}
       \item le chiffre des unités ;
       \item le chiffre des dizaines de mille ;
       \item le chiffre des unités de millions ;
       \item le nombre de centaines ;
       \item le nombre de centaines de mille ;
       \item le nombre de millions.
    \end{enumerate}
    \hrefAleaTeX{https://urls.mathslozano.fr/6n12025ex04}
 \end{exercice}

 \begin{corrige}
    Dans le nombre $\num{6083472}$ donner :
    
    \begin{enumerate}
       \item le chiffre des unités ;                {\red $2$}
       \item le chiffre des dizaines de mille ;     {\red $8$}
       \item le chiffre des unités de millions ;    {\red $6$}
       \item le nombre de centaines ;               {\red $\num{6083472} = \num{60834}\times 100 + 72$ donc $\num{60834}$}
       \item le nombre de centaines de mille ;      {\red $\num{6083472} = 60\times \num{100000} + \num{83472}$ donc $\num{83472}$}
       \item le nombre de millions.                 {\red $\num{6083472} = 6\times \num{1000000} + \num{83472}$ donc $6$}
    \end{enumerate}
 \end{corrige}