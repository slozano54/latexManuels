\begin{exercice}
    Voici cinq cartes contenant un nombre :
    \begin{center}
       \fbox{415} \qquad \fbox{2\,103} \qquad \fbox{9} \qquad \fbox{87} \qquad\fbox{13}
    \end{center}
    Placer ces cartes côte à côte pour écrire :
    \begin{enumerate}
       \item le plus petit nombre entier de douze chiffres ;
       \item le plus grand nombre entier.
    \end{enumerate}
    \hrefAleaTeX{https://urls.mathslozano.fr/6n12025ex03}
 \end{exercice}

 \begin{corrige}
    Voici cinq cartes contenant un nombre :
    \begin{center}
       \fbox{$415$} \qquad \fbox{$\num{2103}$} \qquad \fbox{$9$} \qquad \fbox{$87$} \qquad\fbox{$13$}
    \end{center}
    Placer ces cartes côte à côte pour écrire :

    \begin{enumerate}
       \item le plus petit nombre entier de douze chiffres ;
       
       {\red  \fbox{$13$} \qquad \fbox{$\num{2103}$} \qquad \fbox{$415$} \qquad \fbox{$87$}  \qquad\fbox{$9$} 
       
       soit $\num{132103415879}$}
       \item le plus grand nombre entier.
       
       {\red \fbox{$9$} \qquad \fbox{$87$} \qquad \fbox{$415$} \qquad \fbox{$\num{2103}$} \qquad\fbox{$13$} 
       
       soit $\num{987415210313}$}
    \end{enumerate}
 \end{corrige}