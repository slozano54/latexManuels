% Les enigmes ne sont pas numérotées par défaut donc il faut ajouter manuellement la numérotation
% si on veut mettre plusieurs enigmes
% \refstepcounter{exercice}
% phantom{\numeroteEnigme}
\begin{enigme}[Repérage multi-axes]
   Pour découvrir le dessin codé, placer les points A, B, C \dots selon les indications des tableaux ci-dessous.
   
   Quand tous les points seroont placer, les relier en suivant les instructions données.

   \begin{minipage}{0.3\linewidth}
      \begin{tabular}{|*{3}{>{\centering\arraybackslash}p{0.5cm}|}}
         \hline
         \cellcolor{lightgray}{\!\!\!\small Ligne} & \cellcolor{lightgray}{\!\!\!\small Point} & \cellcolor{lightgray}{\!\small Abs.} \\
         \hline
         (1) & A & $4$\\
         \hline
         (3) & B & $0$\\
         \hline
         (4) & C &  $3$\\
         \hline
         (5) & D &  $6$\\
         \hline
         (7) & E &  $2$\\
         \hline
      \end{tabular}
   \end{minipage}
   \begin{minipage}{0.3\linewidth}
      \begin{tabular}{|*{3}{>{\centering\arraybackslash}p{0.5cm}|}}
         \hline
         \cellcolor{lightgray}{\!\!\!\small Ligne} & \cellcolor{lightgray}{\!\!\!\small Point} & \cellcolor{lightgray}{\!\small Abs.} \\
         \hline
         (7) & F &  $8$\\
         \hline
         (8) & G &  $5$\\
         \hline
         (9) & H &  $8$\\
         \hline
         (11) & I & $5$\\
         \hline
      \end{tabular}
   \end{minipage}
   \begin{minipage}{0.3\linewidth}
      Tracer le triangle $ABD$\\
      Tracer le quadrilatère $CEHF$.\\
      Tracer le triangle $EGI$
   \end{minipage}

   \begin{center}
      \DessinGradue[Lignes=11,Debut=0,Fin=12,Pas=12,Echelle=1]{%
      % 1er argument inutile si les "lignes" sont identiques.
      }{%
      % 2eme argument : on place les points. La notation 1/A/4 signifie que sur la ligne 1,
      %on place le point A au repère numéroté 4.
      1/A/4,
      3/B/0,
      4/C/3,
      5/D/6,
      7/E/2,7/F/8,
      8/G/5,
      9/H/8,
      11/I/4%
      }{%
      % 3eme argument : on définit les tracés nécessaires.
      polygone(A,B,D)§polygone(F,C,E,H)§polygone(E,I,G)%
      }
   \end{center}
\end{enigme}
% Pour le corrigé, il faut décrémenter le compteur, sinon il est incrémenté deux fois
% \addtocounter{exercice}{-1}
\begin{corrige}
   \phantom{rrr}
   
   \begin{center}
      \DessinGradue[Lignes=11,Debut=0,Fin=12,Pas=12,Echelle=0.6,Solution]{%
      % 1er argument inutile si les "lignes" sont identiques.
      }{%
      % 2eme argument : on place les points. La notation 1/A/4 signifie que sur la ligne 1,
      %on place le point A au repère numéroté 4.
      1/A/4,
      3/B/0,
      4/C/3,
      5/D/6,
      7/E/2,7/F/8,
      8/G/5,
      9/H/8,
      11/I/4%
      }{%
      % 3eme argument : on définit les tracés nécessaires.
      polygone(A,B,D)§polygone(F,C,E,H)§polygone(E,I,G)%
      }
   \end{center}
\end{corrige}
