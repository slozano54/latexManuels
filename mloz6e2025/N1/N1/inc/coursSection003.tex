\section{Ordonner des nombres entiers}

\begin{definition}
   {\bf Comparer} deux nombres, c'est dire s'ils sont égaux ou si l'un est plus petit (ou plus grand) que l'autre.
\end{definition}

\begin{notation}
   Dans notre sens de lecture, le symbole \fbox{<} signifie \og plus petit que \fg{} et \fbox{>} signifie \og plus grand que \fg.
\end{notation}

\begin{exemple*1}
   $\num{1000000200} > \num{1000000002}$ \qquad ; \qquad $\num{999999}<\num{1000000}$.
\end{exemple*1}

\bigskip

\begin{definition}
   \begin{itemize}
      \item Ranger des nombres dans l'ordre {\bf croissant} signifie les ranger du plus petit au plus grand.
       \item Ranger des nombres dans l'ordre {\bf décroissant} signifie les ranger du plus grand au plus petit.
   \end{itemize}
\end{definition}

\begin{exemple*1}
   \begin{itemize}
      \item $\num{1000045}<\num{1000085}<\num{1000600}<\num{1000607}$ sont rangés dans l'ordre croissant.
      \item $321>312>231>213>132>123$ sont rangés dans l'ordre décroissant.
   \end{itemize}  
\end{exemple*1}

\begin{definition}
 {\bf Encadrer} un nombre, c'est l'entourer par un nombre plus petit et un nombre plus grand.
\end{definition}

\begin{exemple*1}
   On peut encadrer le nombre 8\,199 de différentes façons, par exemple :   
   \begin{colitemize}{3}
     \item $\num{8198}<\num{8199}<\num{8200}$
     \item $\num{8000}<\num{8199}<\num{9000}$
     \item $\num{1000}<\num{8199}<\num{10000}$\dots
   \end{colitemize}
\end{exemple*1}