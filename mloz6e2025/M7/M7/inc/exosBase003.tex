\begin{exercice*}[Papier isometrique]
   \scalebox{0.8}{
      \begin{Geometrie}[CoinHD={(9.5u,5.5u)}]
         % Pour changer la taille de la Grid
         coeff:=0.7;
         uni:=true;
         x.u:=coeff*cm;
         y.u:=coeff*(sqrt(3)/2)*cm;
         %Fig1
         pair A[];
         A0=pptri(1,4);
         A1-A0=pptri(0,2);
         A2-A0=pptri(-2,4);
         A3-A0=pptri(-2,2);
         trace   polygone(A0,A1,A2,A3) withcolor Crimson withpen pencircle scaled 2bp;
         remplis polygone(A0,A1,A2,A3) withcolor Tomato;  
         %Fig2
         pair B[];
         B0=pptri(3,4);
         B1-B0=pptri(2,0);
         B2-B0=pptri(2,2);
         B3-B0=pptri(0,4);
         B4-B0=pptri(-2,4);
         B5-B0=pptri(0,2);
         trace   polygone(B0,B1,B2,B3,B4,B5) withcolor OrangeRed withpen pencircle scaled 2bp;
         remplis polygone(B0,B1,B2,B3,B4,B5) withcolor LightSalmon;      
         %Fig3
         pair C[];
         C0=pptri(8,5);
         C1-C0=pptri(0,3);
         C2-C0=pptri(-3,3);
         trace   polygone(C0,C1,C2) withcolor HotPink withpen pencircle scaled 2bp;
         remplis polygone(C0,C1,C2) withcolor LightPink;       
         %Fig4
         pair D[];
         D0=pptri(4,1);
         D1-D0=pptri(1,0);
         D2-D0=pptri(1,1);
         D3-D0=pptri(0,2);
         D4-D0=pptri(-1,2);
         D5-D0=pptri(-1,1);
         trace   polygone(D0,D1,D2,D3,D4,D5) withcolor Brown withpen pencircle scaled 2bp;
         remplis polygone(D0,D1,D2,D3,D4,D5) withcolor BurlyWood;
         %Fig5
         pair E[];
         E0=pptri(7,1);
         E1-E0=pptri(3,0);
         E2-E0=pptri(3,1);
         E3-E0=pptri(2,2);
         E4-E0=pptri(-1,2);
         E5-E0=pptri(0,1);
         trace   polygone(E0,E1,E2,E3,E4,E5) withcolor DarkGreen withpen pencircle scaled 2bp;
         remplis polygone(E0,E1,E2,E3,E4,E5) withcolor LightGreen;
         %Fig6
         pair F[];
         F0=pptri(11,3);
         F1-F0=pptri(0,1);
         F2-F0=pptri(-1,2);
         F3-F0=pptri(-1,1);
         trace   polygone(F0,F1,F2,F3) withcolor DarkBlue withpen pencircle scaled 2bp;
         remplis polygone(F0,F1,F2,F3) withcolor LightBlue;   
         % Grille et légendes
         trace papiertriangle withcolor DarkGrey;
         label(btex {\bfseries Fig 1} etex,iso(A1,A3) shifted (0,0.25u));
         label(btex {\bfseries Fig 2} etex,iso(B5,B2) shifted (0,0.25u));
         label(btex {\bfseries Fig 3} etex,iso(C1,C2) shifted (0,-0.25u));
         label(btex {\bfseries Fig 4} etex,iso(D0,D3) shifted (0,0.25u));
         label(btex {\bfseries Fig 5} etex,iso(E5,E2) shifted (0,0.25u));
         label(btex {\bfseries Fig 6} etex,iso(F1,F3) shifted (0,-u));
         % Unités
         %% Longueur
         pair UL[];
         UL[0]=pptri(0,1);
         UL1-UL0=pptri(0,2);
         trace segment(UL0,UL1) withcolor red withpen pencircle scaled 2bp;
         %%% légende
         trace appelation(UL0,UL1,3mm,btex {\red\bfseries u.l.} etex);      
         %% Aire
         pair UA[];
         UA0=pptri(1,1);
         UA1-UA0=pptri(2,0);
         UA2-UA0=pptri(0,2);
         trace polygone  (UA0,UA1,UA2) withcolor Grey withpen pencircle scaled 2bp;
         remplis polygone(UA0,UA1,UA2) withcolor LightGrey;
         %%% légende
         label(btex {\bfseries u.a.} etex,iso(UA0,UA1,UA2));
      \end{Geometrie}
   }
   \begin{center}
      \begin{tabular}{|>{\columncolor{gray!20}\bfseries\centering\arraybackslash}m{0.35\linewidth}|*{3}{>{\centering\arraybackslash}m{0.12\linewidth}|}}
          \hline
          \rowcolor{gray!20} Figure&{\bfseries 1}&{\bfseries 2}&{\bfseries 3}\\\hline
          Périmètre exprimé en u.l.&&&\\\hline
          Aire exprimée en u.a.&&&\\\hline
      \end{tabular}

      \smallskip
      \begin{tabular}{|>{\columncolor{gray!20}\bfseries\centering\arraybackslash}m{0.35\linewidth}|*{3}{>{\centering\arraybackslash}m{0.12\linewidth}|}}
          \hline
          \rowcolor{gray!20} Figure&{\bfseries 4}&{\bfseries 5}&{\bfseries 6}\\\hline
          Périmètre exprimé en u.l.&&&\\\hline
          Aire exprimée en u.a.&&&\\\hline
      \end{tabular}
  \end{center}
\end{exercice*}
\begin{corrige}
   \phantom{text}
   \scalebox{0.8}{
      \begin{Geometrie}[CoinHD={(9.5u,5.5u)}]
         % Pour changer la taille de la Grid
         coeff:=0.7;
         uni:=true;
         x.u:=coeff*cm;
         y.u:=coeff*(sqrt(3)/2)*cm;
         %Fig1
         pair A[];
         A0=pptri(1,4);
         A1-A0=pptri(0,2);
         A2-A0=pptri(-2,4);
         A3-A0=pptri(-2,2);
         trace   polygone(A0,A1,A2,A3) withcolor Crimson withpen pencircle scaled 2bp;
         remplis polygone(A0,A1,A2,A3) withcolor Tomato;  
         %Fig2
         pair B[];
         B0=pptri(3,4);
         B1-B0=pptri(2,0);
         B2-B0=pptri(2,2);
         B3-B0=pptri(0,4);
         B4-B0=pptri(-2,4);
         B5-B0=pptri(0,2);
         trace   polygone(B0,B1,B2,B3,B4,B5) withcolor OrangeRed withpen pencircle scaled 2bp;
         remplis polygone(B0,B1,B2,B3,B4,B5) withcolor LightSalmon;      
         %Fig3
         pair C[];
         C0=pptri(8,5);
         C1-C0=pptri(0,3);
         C2-C0=pptri(-3,3);
         trace   polygone(C0,C1,C2) withcolor HotPink withpen pencircle scaled 2bp;
         remplis polygone(C0,C1,C2) withcolor LightPink;       
         %Fig4
         pair D[];
         D0=pptri(4,1);
         D1-D0=pptri(1,0);
         D2-D0=pptri(1,1);
         D3-D0=pptri(0,2);
         D4-D0=pptri(-1,2);
         D5-D0=pptri(-1,1);
         trace   polygone(D0,D1,D2,D3,D4,D5) withcolor Brown withpen pencircle scaled 2bp;
         remplis polygone(D0,D1,D2,D3,D4,D5) withcolor BurlyWood;
         %Fig5
         pair E[];
         E0=pptri(7,1);
         E1-E0=pptri(3,0);
         E2-E0=pptri(3,1);
         E3-E0=pptri(2,2);
         E4-E0=pptri(-1,2);
         E5-E0=pptri(0,1);
         trace   polygone(E0,E1,E2,E3,E4,E5) withcolor DarkGreen withpen pencircle scaled 2bp;
         remplis polygone(E0,E1,E2,E3,E4,E5) withcolor LightGreen;
         %Fig6
         pair F[];
         F0=pptri(11,3);
         F1-F0=pptri(0,1);
         F2-F0=pptri(-1,2);
         F3-F0=pptri(-1,1);
         trace   polygone(F0,F1,F2,F3) withcolor DarkBlue withpen pencircle scaled 2bp;
         remplis polygone(F0,F1,F2,F3) withcolor LightBlue;   
         % Grille et légendes
         trace papiertriangle withcolor DarkGrey;
         label(btex {\bfseries Fig 1} etex,iso(A1,A3) shifted (0,0.25u));
         label(btex {\bfseries Fig 2} etex,iso(B5,B2) shifted (0,0.25u));
         label(btex {\bfseries Fig 3} etex,iso(C1,C2) shifted (0,-0.25u));
         label(btex {\bfseries Fig 4} etex,iso(D0,D3) shifted (0,0.25u));
         label(btex {\bfseries Fig 5} etex,iso(E5,E2) shifted (0,0.25u));
         label(btex {\bfseries Fig 6} etex,iso(F1,F3) shifted (0,-u));
         % Unités
         %% Longueur
         pair UL[];
         UL[0]=pptri(0,1);
         UL1-UL0=pptri(0,2);
         trace segment(UL0,UL1) withcolor red withpen pencircle scaled 2bp;
         %%% légende
         trace appelation(UL0,UL1,3mm,btex {\red\bfseries u.l.} etex);      
         %% Aire
         pair UA[];
         UA0=pptri(1,1);
         UA1-UA0=pptri(2,0);
         UA2-UA0=pptri(0,2);
         trace polygone  (UA0,UA1,UA2) withcolor Grey withpen pencircle scaled 2bp;
         remplis polygone(UA0,UA1,UA2) withcolor LightGrey;
         %%% légende
         label(btex {\bfseries u.a.} etex,iso(UA0,UA1,UA2));
      \end{Geometrie}
   }
   \begin{center}
      \begin{tabular}{|>{\columncolor{gray!20}\bfseries\centering\arraybackslash}m{0.35\linewidth}|*{3}{>{\centering\arraybackslash}m{0.12\linewidth}|}}
          \hline
          \rowcolor{gray!20} Figure&{\bfseries 1}&{\bfseries 2}&{\bfseries 3}\\\hline
          Périmètre exprimé en u.l.&{\red 4}&{\red 6}&{\red \num{4.5}}\\\hline
          Aire exprimée en u.a.&{\red 2}&{\red 4}&{\red \num{2.25}}\\\hline
      \end{tabular}

      \smallskip
      \begin{tabular}{|>{\columncolor{gray!20}\bfseries\centering\arraybackslash}m{0.35\linewidth}|*{3}{>{\centering\arraybackslash}m{0.12\linewidth}|}}
          \hline
          \rowcolor{gray!20} Figure&{\bfseries 4}&{\bfseries 5}&{\bfseries 6}\\\hline
          Périmètre exprimé en u.l.&{\red 3}&{\red 5}&{\red 2}\\\hline
          Aire exprimée en u.a.&{\red \num{1.5}}&{\red 3}&{\red \num{0.5}}\\\hline
      \end{tabular}
  \end{center}
\end{corrige}