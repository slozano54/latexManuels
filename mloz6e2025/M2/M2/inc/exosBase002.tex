\begin{exercice}
    On considère les deux figures A et B suivantes :
    \begin{center}
       {\psset{unit=0.5}
       \begin{pspicture}(-1,-1)(8,3)
          \pspolygon[fillstyle=solid,fillcolor=A2,linecolor=gray](0,0)(3,0)(1,2)(0,2)
          \pspolygon[fillstyle=solid,fillcolor=B2,linecolor=gray](4,0)(7,0)(7,1)(6,1)(6,2)(4,2)
          \psgrid[subgriddiv=1,gridlabels=0,gridcolor=gray](-1,-1)(8,3)
          \rput(1,1){A}
          \rput(5,1){B}
       \end{pspicture}}
    \end{center}
    Dans toute la suite, les sommets des polygones doivent être des noeuds du quadrillage.
    \begin{enumerate}
       \item Construire une figure différente de A mais de même périmètre que A.
       \item Construire une figure différente de B mais de même périmètre que B.
       \item Construire une figure différente de A mais de même aire que la figure A.
       \item Construire une figure différente de B mais de même aire que la figure B.
       \item Construire trois figures différentes qui ont à la fois le même périmètre que la figure A et la même aire que la figure B.
    \end{enumerate}
 \end{exercice}
