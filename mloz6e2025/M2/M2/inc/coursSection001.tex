\section{Surface et aire}

\begin{definition}
   La \textbf{surface} d'une figure est la partie située à l'intérieur de son contour. \\
   Sa mesure s'appelle l'\textbf{aire}, qui est le nombre d'unités d'aire que la figure contient.
\end{definition}

\begin{remarque}
   attention à ne pas confondre avec le périmètre qui est une mesure de longueur !
\end{remarque}

Pour déterminer l'aire d'une surface, on peut découper la figure en figures simples ou utiliser un pavage simple.

\begin{exemple*1}
   \par\hfill

   \begin{Geometrie}[CoinHD={(7.7u,4.9u)}]
      u:=u*0.7;
      trace grille(0.7) withcolor LightGray;      
      for i=0 upto 5 :
         draw (i*u,0)--(i*u+7u,7u) withcolor LightGray;
         draw (i*u,7u)--(i*u+7u,0) withcolor LightGray;
      endfor;
      for i=0 upto 6 :
         draw (0,i*u)--(7u-i*u,7u) withcolor LightGray;
         draw (i*u,0)--(0,i*u) withcolor LightGray;
         draw (4u+i*u,0)--(11u,7u-i*u) withcolor LightGray;
         draw (4u+i*u,7u)--(11u,i*u) withcolor LightGray;
      endfor;
      fill (2u,5u)--(3u,5u)--(3u,6u)--(2u,6u)--cycle withcolor black;
      label(TEX("\Large $u_1$"),(1.5u,5.5u));
      fill (6u,5u)--(6u,6u)--(5.5u,5.5u)--cycle withcolor black;
      label(TEX("\Large $u_2$"),(4.5u,5.5u));
      fill (8u,5u)--(9u,5u)--(8u,6u)--cycle withcolor black;
      label(TEX("\Large $u_1$"),(7.5u,5.5u));
      fill (u,u)--(3u,u)--(3u,4u)--(u,4u)--(u,3u)--(2u,3u)--(2u,2u)--(u,2u)--cycle withcolor Crimson;
      draw (u,u)--(3u,u)--(3u,4u)--(u,4u)--(u,3u)--(2u,3u)--(2u,2u)--(u,2u)--cycle withcolor black withpen pencircle scaled 1.5bp;
      label(TEX("\Large $A$"),(2.5u,2.5u));
      fill (5u,u)--(6u,u)--(6u,3u)--(5u,3u)--(5u,4u)--(4u,4u)--(4u,2u)--cycle withcolor LightBlue;
      draw (5u,u)--(6u,u)--(6u,3u)--(5u,3u)--(5u,4u)--(4u,4u)--(4u,2u)--cycle withcolor black withpen pencircle scaled 1.5bp;
      label(TEX("\Large $B$"),(5u,2.5u));
      fill (8u,u)--(8.5u,1.5u)--(9u,u)--(9u,2u)--(10u,2u)--(9.5u,2.5u)--(10u,3u)--(9u,3u)--(9u,4u)--(8.5u,3.5u)--(8u,4u)--(8u,3u)--(7u,3u)--(7.5u,2.5u)--(7u,2u)--(8u,2u)--cycle withcolor PeachPuff;
      draw (8u,u)--(8.5u,1.5u)--(9u,u)--(9u,2u)--(10u,2u)--(9.5u,2.5u)--(10u,3u)--(9u,3u)--(9u,4u)--(8.5u,3.5u)--(8u,4u)--(8u,3u)--(7u,3u)--(7.5u,2.5u)--(7u,2u)--(8u,2u)--cycle withcolor black withpen pencircle scaled 1.5bp;
      label(TEX("\Large $C$"),(8.5u,2.5u));
   \end{Geometrie}
   \correction   
   Lorsque l'on n'a pas une unité d'aire entière $u_1$, on prend une partie de l'unité : 
   \begin{itemize}
      \item $u_2$ correspond à la moitié d'un carré $=\dfrac12 =0,5$ ;
      \item $u_3$ correspond au quart d'un carré $=\dfrac14 =0,25$. \smallskip
   \end{itemize}
   On peut aussi \og découper \fg{} une partie de la figure afin de la déplacer ailleurs pour former une unité d'aire.
   \smallskip
   \begin{center}
      \begin{cltableau}{0.9\linewidth}{4}
         \hline
         Unité & fig. A & fig. B & fig. C \\
         \hline
            $u_1$ & $5$ & $4,5$ & $4$ \\
         \hline
         $u_2$ & $20$ & $18$ & $16$ \\
         \hline
         $u_3$ & $10$ & $9$ & $8$ \\
         \hline
      \end{cltableau}
   \end{center}
\end{exemple*1}
