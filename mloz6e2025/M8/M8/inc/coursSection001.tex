\section{Qu'est-ce qu'un angle ?}
\begin{definition}
    Un \textbf{angle} est une portion de plan délimitée par deux demi-droites ayant la même origine.\\
    Les deux demi-droites sont appelées \textbf{côtés} de l'angle, alors que leur origine commune est appelée \textbf{sommet} de l'angle.
\end{definition}
 
\tikzset{
    %Define style for boxes
    punkt/.style={
           rectangle,
           rounded corners,
           draw=black!30, very thick,
           text width=6.5em,
           minimum height=2em,
           text centered},
           % Define arrow style
           pil/.style={
                  ->,
                  thick,
                  shorten <=2pt,
                  shorten >=2pt,} 
}
\begin{tikzpicture}    
   \coordinate[label=below:$A$] (A) at (3,-1);
   \coordinate[label=above:$O$] (O) at (0,0);
   \coordinate[label=above:$C$] (C) at (2,1);
   \pic["$\alpha$", draw=orange, -, angle eccentricity=1.2, angle radius=1cm]{angle=A--O--C};
   \tkzDrawPoints[shape=cross out, size=5pt](A,C);
   \tkzDefPointOnLine[pos=1.2](O,C)\tkzGetPoint{C1}
   \tkzDefPointOnLine[pos=1.4](O,C)\tkzGetPoint{C2}
   \tkzLabelPoint[above](C2){$x$};
   \tkzDefPointOnLine[pos=1.2](O,A)\tkzGetPoint{A1}
   \tkzDefPointOnLine[pos=1.4](O,A)\tkzGetPoint{A2}
   \tkzLabelPoint[below](A2){$y$};
   \tkzDrawLine[add=0 and 0.5](O,A);
   \tkzDrawLine[add=0 and 0.5](O,C);
   \node[punkt] (cotes) at (6,0) {Côtés de l'angle};
   \node[left of=cotes] (dummy) {}
     edge[pil, bend right=-30,color=black!30] (C1.south)
     edge[pil, bend right=30,color=black!30] (A1.north);
   \node[punkt] (sommet) at (-3,0) {Sommet de l'angle}
     edge[pil, bend right=45,color=black!30] (O.west);
\end{tikzpicture}

\begin{notations}
   Cet angle peut être noté de différentes façons :
   \begin{itemize}
      \item $\alpha$, \textit{lettre de l'alphabet grec, se prononçant "alpha", équivalent de notre "a"}.
      \item $\widehat{AOC}$ ou encore $\widehat{COA}$, \textit{la lettre désignant le sommet de l'angle doit être placée au milieu}.
      \item $\widehat{xOy}$ ou $\widehat{yOx}$, où $x$ et $y$ représentent des directions.
   \end{itemize}
\end{notations}

 En général, on marque l'angle considéré par un arc de cercle.\\
 Pour nommer un angle, il suffit de connaître son sommet et un point de chacun de ses côtés.
