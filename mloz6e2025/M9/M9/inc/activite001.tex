\begin{changemargin}{-10mm}{-25mm}
    \begin{activite}[Des petits cubes]

        {\bf Objectifs :} déterminer le volume d'un cube, d'un pavé droit par dénombrement ; analyser un solide en trois dimensions.
        \partie[volume d'un pavé]
        On considère ce pavé droit : \quad 
        \begin{Geometrie}[CoinBG={(-u,-0.5u)},CoinHD={(5u,4u)}]
            u:=0.5*u;
            pair A,B,C,D,E,F,G,H;
            A=u*(1,1);
            B=u*(2,1);
            F=u*(2,2);
            E=u*(1,2);
            pair Vx,Vy,Vz;
            Vx=u*(1,0);
            Vy=u*(0.5,0.4);
            Vz=u*(0,1);
            D=A shifted Vy;
            C=B shifted Vy;
            H=E shifted Vy;
            G=F shifted Vy;
            picture cube;
            cube = image(
                remplis polygone(B,C,G,F) withcolor Cornsilk;
                remplis polygone(E,H,G,F) withcolor 0.7Cornsilk+0.3black;
                remplis polygone(A,B,F,E) withcolor 0.9Cornsilk+0.1black;
                trace A--B--F--E--cycle;
                trace polygone(B,C,G,F);
                trace chemin(G,H,E);
            );        
            trace cube;
            for j=0 upto 5:
                for k=0 upto 4:
                    for i=0 upto 2:
                        trace cube shifted (k*Vx+(2-j)*Vy+i*Vz);
                    endfor;
                endfor;
            endfor;
        \end{Geometrie}
        composé de petits cubes identiques \quad 
        \begin{Geometrie}
            u:=0.5*u;
            pair A,B,C,D,E,F,G,H;
            A=u*(1,1);
            B=u*(2,1);
            F=u*(2,2);
            E=u*(1,2);
            D=A shifted (0.5u,0.4u);
            C=B shifted (0.5u,0.4u);
            H=E shifted (0.5u,0.4u);
            G=F shifted (0.5u,0.4u);
            remplis polygone(B,C,G,F) withcolor Cornsilk;
            remplis polygone(E,H,G,F) withcolor 0.7Cornsilk+0.3black;
            remplis polygone(A,B,F,E) withcolor 0.9Cornsilk+0.1black;
            trace A--B--F--E--cycle;
            trace polygone(B,C,G,F);
            trace chemin(G,H,E);
        \end{Geometrie}
        \par\smallskip
        \begin{tabular}{>{\centering\arraybackslash}p{0.3\linewidth}|>{\centering\arraybackslash}p{0.3\linewidth}|>{\centering\arraybackslash}p{0.3\linewidth}}
            Colorier la \og tranche \fg{} de devant de ce pavé droit
            &
            Colorier la \og tranche \fg{} du dessus de ce pavé droit
            &
            Colorier la \og tranche \fg{} de côté de ce pavé droit \\
            \begin{Geometrie}[CoinBG={(-u,-0.5u)},CoinHD={(5u,4u)}]
                u:=0.5*u;
                pair A,B,C,D,E,F,G,H;
                A=u*(1,1);
                B=u*(2,1);
                F=u*(2,2);
                E=u*(1,2);
                pair Vx,Vy,Vz;
                Vx=u*(1,0);
                Vy=u*(0.5,0.4);
                Vz=u*(0,1);
                D=A shifted Vy;
                C=B shifted Vy;
                H=E shifted Vy;
                G=F shifted Vy;
                picture cube;
                cube = image(
                    remplis polygone(B,C,G,F) withcolor Cornsilk;
                    remplis polygone(E,H,G,F) withcolor 0.7Cornsilk+0.3black;
                    remplis polygone(A,B,F,E) withcolor 0.9Cornsilk+0.1black;
                    trace A--B--F--E--cycle;
                    trace polygone(B,C,G,F);
                    trace chemin(G,H,E);
                );        
                trace cube;
                for j=0 upto 5:
                    for k=0 upto 4:
                        for i=0 upto 2:
                            trace cube shifted (k*Vx+(2-j)*Vy+i*Vz);
                        endfor;
                    endfor;
                endfor;
            \end{Geometrie}
            &
            \begin{Geometrie}[CoinBG={(-u,-0.5u)},CoinHD={(5u,4u)}]
                u:=0.5*u;
                pair A,B,C,D,E,F,G,H;
                A=u*(1,1);
                B=u*(2,1);
                F=u*(2,2);
                E=u*(1,2);
                pair Vx,Vy,Vz;
                Vx=u*(1,0);
                Vy=u*(0.5,0.4);
                Vz=u*(0,1);
                D=A shifted Vy;
                C=B shifted Vy;
                H=E shifted Vy;
                G=F shifted Vy;
                picture cube;
                cube = image(
                    remplis polygone(B,C,G,F) withcolor Cornsilk;
                    remplis polygone(E,H,G,F) withcolor 0.7Cornsilk+0.3black;
                    remplis polygone(A,B,F,E) withcolor 0.9Cornsilk+0.1black;
                    trace A--B--F--E--cycle;
                    trace polygone(B,C,G,F);
                    trace chemin(G,H,E);
                );        
                trace cube;
                for j=0 upto 5:
                    for k=0 upto 4:
                        for i=0 upto 2:
                            trace cube shifted (k*Vx+(2-j)*Vy+i*Vz);
                        endfor;
                    endfor;
                endfor;
            \end{Geometrie}
            &
            \begin{Geometrie}[CoinBG={(-u,-0.5u)},CoinHD={(5u,4u)}]
                u:=0.5*u;
                pair A,B,C,D,E,F,G,H;
                A=u*(1,1);
                B=u*(2,1);
                F=u*(2,2);
                E=u*(1,2);
                pair Vx,Vy,Vz;
                Vx=u*(1,0);
                Vy=u*(0.5,0.4);
                Vz=u*(0,1);
                D=A shifted Vy;
                C=B shifted Vy;
                H=E shifted Vy;
                G=F shifted Vy;
                picture cube;
                cube = image(
                    remplis polygone(B,C,G,F) withcolor Cornsilk;
                    remplis polygone(E,H,G,F) withcolor 0.7Cornsilk+0.3black;
                    remplis polygone(A,B,F,E) withcolor 0.9Cornsilk+0.1black;
                    trace A--B--F--E--cycle;
                    trace polygone(B,C,G,F);
                    trace chemin(G,H,E);
                );        
                trace cube;
                for j=0 upto 5:
                    for k=0 upto 4:
                        for i=0 upto 2:
                            trace cube shifted (k*Vx+(2-j)*Vy+i*Vz);
                        endfor;
                    endfor;
                endfor;
            \end{Geometrie}
            \\
            Combien de petits cubes possède cette tranche ?
            &
            Combien de petits cubes possède cette tranche ?
            &
            Combien de petits cubes possède cette tranche ? \\
            &&\\
            \dotfill & \dotfill & \dotfill \\
            Combien de fois cette tranche apparait-elle dans ce pavé droit ?
            &
            Combien de fois cette tranche apparait-elle dans ce pavé droit ?
            &
            Combien de fois cette tranche apparait-elle dans ce pavé droit ? \\
            &&\\
            \dotfill & \dotfill & \dotfill \\
            En déduire le nombre de petits cubes dans ce pavé droit.
            &
            En déduire le nombre de petits cubes dans ce pavé droit.
            &
            En déduire le nombre de petits cubes dans ce pavé droit. \\
            &&\\
            \dotfill & \dotfill & \dotfill \\
        \end{tabular}
        \par
        \partie[volume d'un cube]
        Colorier dans ce pavé un cube de plus grande taille possible : 
        \begin{Geometrie}[CoinBG={(-u,-0.5u)},CoinHD={(5u,4u)}]
            u:=0.5*u;
            pair A,B,C,D,E,F,G,H;
            A=u*(1,1);
            B=u*(2,1);
            F=u*(2,2);
            E=u*(1,2);
            pair Vx,Vy,Vz;
            Vx=u*(1,0);
            Vy=u*(0.5,0.4);
            Vz=u*(0,1);
            D=A shifted Vy;
            C=B shifted Vy;
            H=E shifted Vy;
            G=F shifted Vy;
            picture cube;
            cube = image(
                remplis polygone(B,C,G,F) withcolor Cornsilk;
                remplis polygone(E,H,G,F) withcolor 0.7Cornsilk+0.3black;
                remplis polygone(A,B,F,E) withcolor 0.9Cornsilk+0.1black;
                trace A--B--F--E--cycle;
                trace polygone(B,C,G,F);
                trace chemin(G,H,E);
            );        
            trace cube;
            for j=0 upto 5:
                for k=0 upto 4:
                    for i=0 upto 2:
                        trace cube shifted (k*Vx+(2-j)*Vy+i*Vz);
                    endfor;
                endfor;
            endfor;
        \end{Geometrie}
        \par\medskip
        Combien y a-t-il de petits cubes dans ce cube ? \dotfill
    \end{activite} 
\end{changemargin}