\begin{exercice*} %4
   Calculer le périmètre des figures suivantes.

   \begin{Geometrie}
      pair A[];
      path cc;
      A0=u*(2,2);
      A1-A0=u*(1.5,0);
      cc=cercles(A0,1.5u);
      trace cc;
      trace segment(A0,A1);
      trace appelation(A0,A1,-3mm,TEX("\Lg[km]{15}"));
      label(TEX("Fig. \ding{172}"),(2u,2.5u));
      marque_p:="croix";
      u:=u/2;
      pointe(A0);
   \end{Geometrie}
   \hfill
   \begin{Geometrie}
      pair A[];
      path cc;
      A0=u*(2,2);
      A1-A0=u*(1.5,0);
      A2-A0=u*(-1.5,0);
      cc=cercles(A0,1.5u);
      trace cc;
      trace segment(A2,A1);
      trace appelation(A2,A1,-3mm,TEX("\Lg[m]{5.6}"));
      label(TEX("Fig. \ding{173}"),(2u,2.5u));
      marque_p:="croix";
      u:=u/2;
      pointe(A0);
   \end{Geometrie}

   \begin{Geometrie}
      pair A[];
      path cc;
      A0=u*(2,2);
      A1-A0=u*(1.5,0);
      A2-A0=u*(-1.5,0);
      cc=cercles(A0,1.5u);
      % trace cc;
      trace subpath (0,4) of cc;
      trace segment(A2,A1);
      trace appelation(A2,A1,-3mm,TEX("\Lg[mm]{8.3}"));
      label(TEX("Fig. \ding{174}"),(2u,2.5u));
      marque_p:="croix";
      u:=u/2;
      pointe(A0);
   \end{Geometrie}
   \hfill
      \begin{Geometrie}
      pair A[];
      path cc;
      A0=u*(3,3);
      A1-A0=u*(1,0);
      A2-A0=u*(-1,0);
      A3-A0=u*(-1,-2);
      A4-A0=u*(1,-2);
      cc=cercles(A0,1u);
      trace subpath (0,4) of cc;
      trace polygone(A1,A2,A3,A4);
      trace appelation(A2,A1,3mm,TEX("\Lg[cm]{20}"));
      label(TEX("Fig. \ding{175}"),iso(A1,A2,A3,A4));
      marque_p:="croix";
      u:=u/2;
      pointe(A0,iso(A2,A3),iso(A4,A3),iso(A1,A4));
   \end{Geometrie}
\end{exercice*}
\begin{corrige}
  Pas de correction.
\end{corrige}



