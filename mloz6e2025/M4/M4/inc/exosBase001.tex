\begin{exercice*}
    Calculer le périmètre des figures suivantes.
    
    \begin{Geometrie}
      pair A[];
      A0=u*(0.5,0.5);
      A1-A0=u*(3,0);
      A2-A0=u*(3,2);
      A3-A0=u*(0,2);
      trace polygone(A0,A1,A2,A3);
      trace codeperp(A1,A0,A3,8);
      trace appelation(A0,A1,-3mm,TEX("\Lg[m]{6}"));
      trace appelation(A0,A3,3mm,TEX("\Lg[m]{4}"));
      marque_s:=0.3*marque_s;
      trace Codelongueur(A0,A1,A2,A3,4);
      trace Codelongueur(A1,A2,A3,A0,2);
      label(TEX("Fig. \ding{172}"),iso(A0,A1,A2,A3));
   \end{Geometrie}
   \hfill
   \begin{Geometrie}
      pair A[];
      A0=u*(0.5,0.5);
      A1-A0=u*(3,0);
      A2-A0=u*(3,3);
      A3-A0=u*(0,3);
      trace polygone(A0,A1,A2,A3);
      trace codeperp(A1,A0,A3,8);
      trace appelation(A0,A1,-3mm,TEX("\Lg[cm]{27}"));
      marque_s:=0.3*marque_s;
      trace Codelongueur(A0,A1,A2,A3,3);
      trace Codelongueur(A1,A2,A3,A0,3);
      label(TEX("Fig. \ding{173}"),iso(A0,A1,A2,A3));
   \end{Geometrie}

   \begin{Geometrie}
      pair A[];
      A0=u*(0.5,0.5);
      A1-A0=u*(0.5,0);
      A2-A0=u*(3.5,0);
      A3-A0=u*(0.5,2);
      trace polygone(A0,A2,A3);
      trace codeperp(A2,A1,A3,8);
      trace segment(A1,A3);
      trace cotationmil(A0,A2,-3mm,15,TEX("\Lg[cm]{6.6}"));      
      trace appelation(A0,A3,3mm,TEX("\Lg[cm]{4.6}"));
      trace appelation(A1,A3,-3mm,TEX("\Lg[cm]{4.4}"));
      trace appelation(A3,A2,3mm,TEX("\Lg[cm]{6.9}"));
      label(TEX("Fig. \ding{174}"),iso(A1,A2,A3));
   \end{Geometrie}
   \hfill
   \begin{Geometrie}
      pair A[];
      A0=u*(0.5,0.5);
      A1-A0=u*(3.5,0);
      A2-A0=u*(0,2);
      trace polygone(A0,A1,A2);
      trace codeperp(A1,A0,A2,8);
      trace appelation(A0,A1,-3mm,TEX("\Lg[mm]{83}"));      
      trace appelation(A0,A2,3mm,TEX("\Lg[mm]{25}"));
      trace appelation(A2,A1,3mm,TEX("\Lg[mm]{86.68}"));
      label(TEX("Fig. \ding{175}"),iso(A0,A1,A2));
   \end{Geometrie}
 \end{exercice*}
\begin{corrige}
  Pas de correction.
\end{corrige}

