\begin{exercice*}
   On considère la figure suivante :
   \begin{center}
      \begin{Geometrie}[CoinBG={(0,0)},CoinHD={(5u,3u)}]
         pair A[];
         path cc[];
         A0=u*(1,1.5);
         A1=u*(1.5,1.5);
         A2=u*(3.5,1.5);
         A3=u*(4,1.5);
         cc0=cercles(A0,u);
         cc1=cercles(A1,1.5u);
         cc2=cercles(A2,1.5u);
         cc3=cercles(A3,u);
         trace grille(0.5) withcolor LightGray;
         trace subpath(4,8) of cc0 withpen pencircle scaled 1.2bp;
         trace subpath(4,8) of cc1 withpen pencircle scaled 1.2bp;
         trace subpath(0,4) of cc2 withpen pencircle scaled 1.2bp;
         trace subpath(0,4) of cc3 withpen pencircle scaled 1.2bp;
         trace cotation((4u,u),(4.5u,u),0mm,-2mm,TEX("\Lg[cm]{2}"));
      \end{Geometrie}
   \end{center}
   \begin{enumerate}
      \item De quoi est composée cette figure ?
      \item Déterminer alors son périmètre.
   \end{enumerate}
\end{exercice*}
\begin{corrige}
   Pas de correction.
\end{corrige}



