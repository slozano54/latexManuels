\begin{exercice*}
   On considère la figure suivante :
      \begin{center}
      \begin{Geometrie}[CoinBG={(0,0)},CoinHD={(4.5u,2.5u)}]
         pair A[];
         path cc[];
         A0=u*(2,0.5);
         A1-A0=u*(0.5,0);
         A2-A0=u*(0.5,0.5);         
         A3-A0=u*(0,0.5);
         cc0=cercles(A0,1.5u);
         cc1=cercles(A2,0.5u);
         cc2=cercles(A3,u);
         trace grille(0.5) withcolor LightGray;
         trace subpath(2,4) of cc0 withpen pencircle scaled 1.2bp;
         trace subpath(6,8) of cc1 withpen pencircle scaled 1.2bp;
         trace subpath(0,2) of cc2 withpen pencircle scaled 1.2bp;
         trace segment((0.5u,0.5u),A0) withpen pencircle scaled 1.2bp;
         trace polygone(A0,A1,A2,A3) withpen pencircle scaled 1.2bp;
         trace segment((3u,u),A2) withpen pencircle scaled 1.2bp;
         trace segment((2u,2u),A3) withpen pencircle scaled 1.2bp;
         trace cotation((3.5u,1.5u),(4u,1.5u),0mm,-2mm,TEX("\Lg[cm]{4}"));
      \end{Geometrie}
   \end{center}
   \begin{enumerate}
      \item De quoi est composée cette figure ?
      \item Déterminer alors son périmètre.
   \end{enumerate}
\end{exercice*}
\begin{corrige}
  Pas de correction.
\end{corrige}
