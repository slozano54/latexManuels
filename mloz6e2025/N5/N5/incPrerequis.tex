\vspace*{-7mm}
%pre-001
\begin{prerequis}[Connaisances \emoji{red-heart} et compétences \emoji{diamond-suit} du cycle 3]    
   \begin{itemize}        
       \item[\emoji{red-heart}] Vocabulaire associé à ces objets et à leurs propriétés : côté, sommet, angle, hauteur.
       \columnbreak
       \item[\emoji{diamond-suit}] Reconnaître, nommer, décrire des triangles, dont les triangles particuliers (triangle rectangle, triangle isocèle, triangle équilatéral).       
   \end{itemize}
\end{prerequis}
\vspace*{-3mm}
\begin{debat}[Les fractions, ces nombres rompus !] 
   C'est vers 3000 ans avant J.-C. que l'on trouve les premières représentations des fractions en Mésopotamie. Au {\small XII}\up{e} siècle, le mot {\bf fractiones} est traduit de l'arabe du mot {\it kasr}, qui veut dire {\it rompu}. En effet, à cette époque, les fractions sont considérées comme des nombres rompus : des nombres que l'on aurait cassé en plusieurs morceaux. Au {\small XIV}\up{e} siècle, le mathématicien {\it Nicole Oresme} utilise la notation des fractions avec la barre et définit les termes de numérateur et dénominateur.
   \begin{center}
      \textcolor{B1}{\fontsize{30}{30}\selectfont $\dfrac{a}{b} =a\div b =a\times\dfrac{1}{b}$}
   \end{center}
   \bigskip
   \begin{cadre}[B2][F4]
      \begin{center}
         \hrefVideo{https://leblob.fr/fondamental/les-fractions}{\bf Les fractions}, site Internet {\it Le Blob, l'extra-média}.
      \end{center}
   \end{cadre}
\end{debat}
