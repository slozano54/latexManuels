\begin{exercice*}
    Écrire chaque désignation sous la forme d'une fraction, d'une somme de fractions et d'un produit comme dans l'exemple : $\dfrac24 =\dfrac14+\dfrac14 =2\times\dfrac14$. \smallskip
    \begin{multicols}{2}
        \begin{enumerate}
            \item trois cinquièmes
            \item quatre tiers
            \item cinq septièmes
            \item six centièmes
        \end{enumerate}
    \end{multicols}
 \end{exercice*}
\begin{corrige}
    %\setcounter{partie}{0} % Pour s'assurer que le compteur de \partie est à zéro dans les corrigés
    Écrire chaque désignation sous la forme d'une fraction, d'une somme de fractions et d'un produit comme dans l'exemple : $\dfrac24 =\dfrac14+\dfrac14 =2\times\dfrac14$. \smallskip
    \begin{enumerate}
        \item trois cinquièmes\\
        {\red\small $\dfrac{3}{5}=\dfrac{1}{5}+\dfrac{1}{5}+\dfrac{1}{5}=3\times\dfrac{1}{5}$}
        \item quatre tiers\\
        {\red\small $\dfrac{4}{3}=\dfrac{1}{3}+\dfrac{1}{3}+\dfrac{1}{3}+\dfrac{1}{3}=4\times\dfrac{1}{3}$}
        \item cinq septièmes\\
        {\red\small $\dfrac{5}{7}=\dfrac{1}{7}+\dfrac{1}{7}+\dfrac{1}{7}+\dfrac{1}{7}+\dfrac{1}{7}=5\times\dfrac{1}{7}$}
        \item six centièmes\\
        {\red\small $\dfrac{6}{100}=\dfrac{1}{100}+\dfrac{1}{100}+\dfrac{1}{100}+\dfrac{1}{100}+\dfrac{1}{100}+\dfrac{1}{100}=6\times\dfrac{1}{100}$}
    \end{enumerate}    
\end{corrige}
    
    