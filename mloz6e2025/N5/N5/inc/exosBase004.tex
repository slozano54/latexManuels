% \begin{exercice*}        
%     Sur la figure suivante, on a représenté un réseau de droites parallèles et équidistantes ainsi qu'un segment de longueur $u$ à trois reprises. À l'aide du compas, tracer sur le cahier les segments de longueur suivante :
%     \smallskip
%     \begin{multicols}{3}
%         \begin{spacing}{1.8}
%             \begin{enumerate}
%                 \item $\dfrac16u$
%                 \item $\dfrac15u$
%                 \item $\dfrac13u$
%                 \item $\dfrac56u$
%                 \item $\dfrac23u$
%                 \item $\dfrac35u$
%             \end{enumerate}
%         \end{spacing}
%     \end{multicols}
%     \medskip
%     \psset{xunit=0.7,yunit=0.3}
%     \begin{pspicture*}(0,0)(11,8)
%        \multido{\n=-3+1,\i=1+1}{14}{\psline[linecolor=gray](\n,0)(\i,8)}
%         \psline[linewidth=0.5mm]{|-|}(0.5,3)(5.8,7.6)
%        \psline[linewidth=0.5mm]{|-|}(2,2)(8.4,4.8)
%        \psline[linewidth=0.5mm]{|-|}(3.5,1)(10.3,2.6)
%        \rput(10.6,2.6){$u$}
%        \rput(8.7,4.8){$u$}
%        \rput(6.2,7.6){$u$}
%     \end{pspicture*}
% \end{exercice*}
\begin{exercice*}
    Le même segment de longueur $\ell$ est représenté plusieurs fois sur trois réseaux de droites parallèles et équidistantes.
    En choisissant le réseau approprié, à l'aide du compas, tracer sur le cahier des segments de longueur $\dfrac16\ell$, $\dfrac15\ell$, $\dfrac13\ell$, $\dfrac56\ell$, $\dfrac23\ell$ et $\dfrac35\ell$.
    \medskip
    \GuideAne[NomSegment=\Large$\ell$,Graine=10]{3}
    \smallskip
    \GuideAne[NomSegment=\Large$\ell$,Graine=20]{5}
    \smallskip
    \GuideAne[NomSegment=\Large$\ell$,Graine=30]{6}
\end{exercice*}  
\begin{corrige}
    %\setcounter{partie}{0} % Pour s'assurer que le compteur de \partie est à zéro dans les corrigés
    % \GuideAne[NomSegment=\Large$\ell$,Graine=30Couleur=LightGray,Marque=1,Traces={
    %     trace compas(O,1/6[O,C],1);
    % }]{6}
    Pour chaque segment la méthode est la même : on choisit le réseau de droites parallèles et équidistantes qui permet d'obtenir le nombre de parties égales souhaité, puis on utilise le compas pour reporter la longueur demandée.
    \medskip
    \GuideAne[NomSegment=\Large$\ell$,Graine=30,Repere=1,Couleur=LightGray,Marque=1,Traces={
        trace compas(O,1/6[O,C],1);
    }]{6}
    \medskip
    \GuideAne[NomSegment=\Large$\ell$,Graine=20,Repere=1,Couleur=LightGray,Marque=1,Traces={
        trace compas(O,1/5[O,C],1);
    }]{5}
    \medskip
    \GuideAne[NomSegment=\Large$\ell$,Graine=10,Repere=1,Couleur=LightGray,Marque=1,Traces={
        trace compas(O,1/3[O,C],1);
    }]{3}
    \medskip
    \GuideAne[NomSegment=\Large$\ell$,Graine=30,Repere=5,Couleur=LightGray,Marque=1,Traces={
        trace compas(O,5/6[O,C],1);
    }]{6}
    \Coupe
    \GuideAne[NomSegment=\Large$\ell$,Graine=20,Repere=2,Couleur=LightGray,Marque=1,Traces={
        trace compas(O,2/3[O,C],1);
    }]{3}
    \medskip
    \GuideAne[NomSegment=\Large$\ell$,Graine=10,Repere=3,Couleur=LightGray,Marque=1,Traces={
        trace compas(O,3/5[O,C],1);
    }]{5}
\end{corrige}  
    