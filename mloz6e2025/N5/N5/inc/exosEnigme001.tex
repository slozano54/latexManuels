% Les enigmes ne sont pas numérotées par défaut donc il faut ajouter manuellement la numérotation
% si on veut mettre plusieurs enigmes
% \refstepcounter{exercice}
% \numeroteEnigme
\begin{enigme}[L'atelier des potions]
    \partie[présentation du jeu]
    L’atelier des potions est un jeu innovant, basé sur la manipulation, qui permet un enseignement et un apprentissage ludique et concret des fractions. \\  
Il a été conçu de façon collaborative par des chercheurs et des enseignants, dans le but d'être facilement utilisable en classe, en petits groupes ou en classe entière. \\
    \begin{center}
      \includegraphics[scale=1]{\currentpath/images/potions}
    \end{center}

    \partie[principes du jeu]
    \begin{minipage}{0.7\linewidth}
        Les élèves, apprentis sorciers, réalisent des potions en sélectionnant la bonne fraction d’ingrédient parmi ceux à leur disposition. Les cartes, ordonnées suivant un ordre croissant de difficulté, permettent de travailler les notions suivantes :
        \begin{itemize}
        \item représentation de fractions ;
        \item fractions supérieures à 1 ;
        \item équivalence de fraction ;
        \item somme de fractions ;
        \item décomposition de fractions, etc.
        \end{itemize}
    \end{minipage}
    \hspace*{5mm}
    \begin{minipage}{0.3\linewidth}    
        \begin{center}
           \includegraphics[scale=0.9]{\currentpath/images/carte}
        \end{center}
    \end{minipage}

    \textbf{Sur le plateau de la partie A, colorier les proportions indiquées pour la potion n°27.}
    \vfill
    \begin{flushright}
    \href{https://www.atelier-potions.fr}{{\small https://www.atelier-potions.fr}}
    \end{flushright}
\end{enigme}

% Pour le corrigé, il faut décrémenter le compteur, sinon il est incrémenté deux fois
% \addtocounter{exercice}{-1}
% \begin{corrige}
%     \ldots
% \end{corrige}