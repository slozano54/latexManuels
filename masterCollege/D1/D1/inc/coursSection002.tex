\section{Section 2}
\subsection{Sous-section 2.1}
Quatre affichages prévus pour les méthodes.

\begin{methode}[Titre de la méthode]
    Texte introductif.
    \exercice
    Texte de l’exercice
    \correction
    Texte de la correction sur un minimum de trois lignes pour faire la
    différence entre vis-à-vis et double colonne. C’est l’endroit de la
    coupure qui va différer.
\end{methode}

\begin{methode*1}[Titre de la méthode*1]
    Texte introductif

    \Fraction[Disque,Rayon=13mm,Reponse,Couleur=0.85,white]{1/3}
    \exercice
    Texte de l’exercice
    \correction
    Texte de la correction sur un minimum de trois lignes pour faire la
    différence entre vis-à-vis et double colonne. C’est l’endroit de la
    coupure qui va différer.
\end{methode*1}

\subsection{Sous-section 2.2}
\begin{methode*2}[Titre de la méthode*2]
    Texte introductif
    \exercice
    Texte de l’exercice
    \correction
    Texte de la correction sur un minimum de trois lignes pour faire la
    différence entre vis-à-vis et double colonne. C’est l’endroit de la
    coupure qui va différer.
\end{methode*2}

\begin{methode*2*2}[Dernière méthode  \MethodeRefExercice{exo-exemple1} \MethodeRefExercice{exo-exemple2}]
    \exercice
    \label{methode-exemple}
    Texte du premier exercice
    \correction
    Correction du premier exercice
    \exercice
    Texte du deuxième exercice
    \correction
    Texte de la correction du deuxième exercice sur un minimum de trois
    lignes pour faire la différence entre vis-à-vis et double
    colonne. C’est l’endroit de la coupure qui va différer.
\end{methode*2*2}