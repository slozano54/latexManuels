% ============================================================================================
% ======= Compatibilité avec ProfCollege
% ============================================================================================
\PassOptionsToPackage{table}{xcolor}
\PassOptionsToPackage{svgnames}{xcolor}

% ============================================================================================
% ======= Générer les correction dans un dossier spécifique
% ============================================================================================
\renewcommand\PrefixeCorrection{corrections/}

% ============================================================================================
% ======= Thèmes de base prédéfinis
% \themaG ; \themaF ; \themaS
%
% ======= Thèmes personnalisés
%
% \NewThema{N}{n}{titre}{Titre}{TITRE}{couleur entete et ...}{couleur pied de page et ...}
%
% ============================================================================================
\NewThema{I}{i}{introduction}{Introduction}{INTRODUCTION}{gray}{gray!50}

\NewThema{N}{n}{nombres \&\\~calculs}{Nombres \&\\~calculs}{NOMBRES \&\\~CALCULS}{B1}{B1!50}

% Pour la géométrie on garde le théme d'origine

\NewThema{D}{d}{organisation \&\\~gestion de données}{Organisation \&\\~gestion de données}{ORGANISATION \&\\~GESTION DE DONNÉES}{PartieStatistique}{PartieStatistique!50}

\NewThema{M}{m}{grandeurs \&\\~mesures}{Grandeurs \&\\~mesures}{GRANDEURS \&\\~MESURES}{G1}{G1!50}

\NewThema{A}{a}{algorithmique \&\\~programmation}{Algorithmique \&\\~programmation}{ALGORITHMIQUE \&\\~PROGRAMMATION}{J1}{J1!50}

% ============================================================================================
% ======= Comme il y a des thèmes personnalisés
% ======= Il faut redefinir la commande \ListeMethodesThemes{}
% ============================================================================================

\renewcommand\ListeMethodesThemes{{i}{I},{n}{N},{g}{G},{d}{D},{m}{M},{a}{A}}

% ============================================================================================
% ======= Quelques styles supplémentaires
% ============================================================================================
\fancypagestyle{firstCover}{%   1ere de couverture
    \fancyhf{}%                 On initialise headers and footers à rien du tout !        
    \renewcommand{\headrulewidth}{0pt}%trait horizontal pour l'en-tête
    %\renewcommand{\footrulewidth}{0.4pt}%trait horizontal pour le pied de page    
}
\fancypagestyle{backCover}{%   4ere de couverture
    \fancyhf{}%                 On initialise headers and footers à rien du tout !        
    %\renewcommand{\headrulewidth}{0pt}%trait horizontal pour l'en-tête
    %\renewcommand{\footrulewidth}{0.4pt}%trait horizontal pour le pied de page
}

