\usepackage{sesamanuelTIKZ}
\usepackage{ProfCollege}
% Pour la gestion des fontes maths notamment en mode LuaLaTeX.
\usepackage{unicode-math}
% Par exemple, une fonte sans serif pour les briques Scratch.
\newfontfamily\myfontScratch[]{FreeSans}

% Pour générer du faux texte
\usepackage{lipsum}

% % Pour les hyperliens, notamment des sommaires cliquables
\usepackage{hyperref}

\newcommand\myAuthorName{Sébastien LOZANO \par Nom de l'auteur à modifier dans le fichier 0persoCommandes.tex}

% En prévision de l'application d'un style particulier
% pour les parties dans le sommaire
\newcommand{\myTocFrame}[2]{%couleur, texte
    \addtocontents{toc}{
        \vspace{1cm}
        \begin{cadre}[#1][#1!50]
            \begin{center}
                #2
            \end{center}
        \end{cadre}
    }
}

% Pour pouvoir séparer la numérotation des chapitres en fonction des parties
% Important pour que les liens cliquables du sommaire renvoient au bon endroit
\counterwithin*{chapter}{part}

% Pour pouvoir numéroter les énigmes quand on en met plusieurs
% dans la section \recreation
\newcommand{\numeroteEnigme}{
    \begin{pspicture}(0,0)(\ExerciceNumFrameWidth,\ExerciceNumFrameHeight)
        \psframe*[linewidth=0pt,
                  linecolor=LibreExerciceNumFrameColor]
                 (0,-\ExerciceNumFrameDepth)
                 (\ExerciceNumFrameWidth,\ExerciceNumFrameHeight)
      \rput[B](\dimexpr\ExerciceNumFrameWidth/2,0){%
        \textcolor{LibreExerciceNumColor}{\ExerciceNumFont \theexercice}%
      }
  \end{pspicture}  
}

% On factorise les titres des sections pour le glossaire de propriété
\newcommand{\sectionsGlossaireProprietes}[3]{
    \section{#1 \textcolor{#2}{#3}}
}
\newcommand{\titreSectionGlossairePropUn}{
    \sectionsGlossaireProprietes{Section 1}{A1}{texte en couleur différente}
}
\newcommand{\titreSectionGlossairePropDeux}{
    \sectionsGlossaireProprietes{Section 2}{A1}{texte en couleur différente}
}