% Les enigmes ne sont pas numérotées par défaut donc il faut ajouter manuellement la numérotation
% si on veut mettre plusieurs enigmes
\refstepcounter{exercice}
\numeroteEnigme
\begin{enigme}
    Trouveras-tu un chemin de multiples entre les cases colorées ?

    \LabyNombre[Multiple=10,Longueur=12,Largeur=8,XDepart=2,YDepart=2,XArrivee=10,YArrivee=6]    

    Remarque : ce labyrinthe ne correspondra pas à sa correction
\end{enigme}

% Pour le corrigé, il faut décrémenter le compteur, sinon il est incrémenté deux fois
\addtocounter{exercice}{-1}
\begin{corrige}
    N1 - Correction enigme de la fin de la partie cours.  
    \LabyNombre[Multiple=10,Longueur=12,Largeur=8,XDepart=2,YDepart=2,XArrivee=10,YArrivee=6]    
    
    Problème d'inclusion de corrigé de LabyNombre ici pour l'instant la solution consiste
    à mettre le labyrinthe enoncé dans la correction sauf que dans ce cas cela ne correspond plus au labyrinthe initial.

    \LabyNombre[Multiple=10,Longueur=12,Largeur=8,XDepart=2,YDepart=2,XArrivee=10,YArrivee=6,Solution]    
\end{corrige}