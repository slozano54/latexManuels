\section{Section 1}
\subsection{Sous-section 1.1}
\begin{definition}[Titre optionnel]
    Dans le cours, on utilise assez souvent des cadres du type
    définition (comme ici par exemple).

    Une dizaine c'est dix unités.

    \RepresenterEntier{10}
\end{definition}
\begin{remarque}
    Ceci est une remarque permettant de tirer sur la corde nostalgie.

    \begin{center}
        \multido{\i=1+1}{10}{%
        \fbox{%
        \tiny%
        \setlength{\arraycolsep}{0.25\arraycolsep}%
        \Tables[Seul]{\i}%
        \setlength{\arraycolsep}{4\arraycolsep}%
        }%
        }
    \end{center}


\end{remarque}
\begin{propriete}[Titre optionnel]
  Dans le cours, on utilise assez souvent des cadres du type
  définition, comme ici par exemple pour une propriete.
\end{propriete}
\begin{remarques}
  \begin{itemize}
    \item remarque.
    \item remarque.
  \end{itemize}
\end{remarques}

\subsection{Sous-section 1.2}
\begin{theoreme}[Titre optionnel]
  Dans le cours, on utilise assez souvent des cadres du type
  définition, comme ici par exemple pour un théorème.
\end{theoreme}
\begin{notation}
  notation
\end{notation}
\begin{notations}
  \begin{itemize}
    \item notation.
    \item notation.
  \end{itemize}
\end{notations}
\begin{preuve}
  Ceci est une preuve\par Deuxième ligne de la preuve
\end{preuve}
\begin{exemple}
  Texte de l’exemple
  \correction
  Texte de la correction en vis à vis
\end{exemple}

\begin{exemple*1}
  Texte de l’exemple
  \correction
  Texte de la correction, le tout verticalement affiché
\end{exemple*1}

\begin{exemple}[0.6]
  Texte de l’exemple très long sur une ligne, très très très long.
  On peut modifier la répartition horizontale  à l'aide d'un argument optionnel valant par défaut 0,4, valant ici 0,6.
  \correction
  Texte de la correction en vis à vis
\end{exemple}