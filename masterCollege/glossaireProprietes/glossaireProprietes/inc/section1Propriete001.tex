\begin{pspicture}(0,0.25)(3.5,2.5)
\pnode(0,0.5){A}
\pnode(2.5,0.5){B}
\pnode(3.5,2){C}
\pnode(1,2){D}
\pspolygon(A)(B)(C)(D)
\psline(A)(C)
\psline(B)(D)
\uput[d](A){$A$}
\uput[d](B){$B$}
\uput[u](C){$C$}
\uput[u](D){$D$}
\end{pspicture}
&
\propriete{} Si un quadrilatère est un parallélogramme alors ses
diagonales se coupent en leur milieu. (C’est aussi vrai pour les
losanges, rectangles et carrés qui sont des parallélogrammes
particuliers.)
&
Ici $ABCD$ est un parallélogramme donc ses diagonales $[AC]$ et
$[BD]$ se coupent en leur milieu.
    