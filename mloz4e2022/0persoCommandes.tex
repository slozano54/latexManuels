% ============================================================================================
% ======= 1ere et 4eme de couverture
% ============================================================================================

\newcommand\myAuthorName{Sébastien LOZANO}
\newcommand\myAuthorSchoolName{Collège Jean Lurçat 54390 FROUARD}
\newcounter{postCurrentSchoolYear}
\setcounter{postCurrentSchoolYear}{\the\year}
\addtocounter{postCurrentSchoolYear}{1}
\newcommand\currentSchoolYear{Année  \the\year ~- \thepostCurrentSchoolYear}

\newcommand{\mySite}{\href{https://mathslozano.fr}{https://mathslozano.fr}}
\newcommand{\myManualName}{Master Manuel \LaTeX}

\newcommand{\myMessage}{Nom, Lieu de travail et Année courante, sont à modifier dans le fichier 0persoCommandes.tex \par
Ce message est à supprimer en supprimant l'appel à la commande \textbackslash myMessage !
}

% ============================================================================================
% ======= Sommaire
% ============================================================================================

% Pour pouvoir séparer la numérotation des chapitres en fonction des parties
% Important pour que les liens cliquables du sommaire renvoient au bon endroit
\counterwithin*{chapter}{part}

% En prévision de l'application d'un style particulier
% pour les parties dans le sommaire
\newcommand{\myTocFrame}[2]{%couleur, texte
    \addtocontents{toc}{
        \vspace{1cm}
        \begin{cadre}[#1][#1!50]
            \begin{center}
                #2
            \end{center}
        \end{cadre}
    }
}

% ============================================================================================
% Factorisation de commandes
% ============================================================================================
% Pour pouvoir numéroter les énigmes quand on en met plusieurs
% dans la section \recreation
\newcommand{\numeroteEnigme}{
    \begin{pspicture}(0,0)(\ExerciceNumFrameWidth,\ExerciceNumFrameHeight)
        \psframe*[linewidth=0pt,
                  linecolor=LibreExerciceNumFrameColor]
                 (0,-\ExerciceNumFrameDepth)
                 (\ExerciceNumFrameWidth,\ExerciceNumFrameHeight)
      \rput[B](\dimexpr\ExerciceNumFrameWidth/2,0){%
        \textcolor{LibreExerciceNumColor}{\ExerciceNumFont \theexercice}%
      }
  \end{pspicture}  
}

% ======= Il n'y a qu'un glossaire, est-ce judicieux de regrouper ici ?
% ======= Je pense que oui, question de "séparatisme" !

% On factorise les titres des sections pour le glossaire de propriété
\newcommand{\sectionsGlossaireProprietes}[3]{
    \section{#1 \textcolor{#2}{#3}}
}
\newcommand{\titreSectionGlossairePropUn}{
    \sectionsGlossaireProprietes{Section 1}{A1}{texte en couleur différente}
}
\newcommand{\titreSectionGlossairePropDeux}{
    \sectionsGlossaireProprietes{Section 2}{A1}{texte en couleur différente}
}

% Pour écrire (admise) pour les propriétés
\newcommand{\admise}{(admise)}