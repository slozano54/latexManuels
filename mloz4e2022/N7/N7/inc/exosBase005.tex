\begin{exercice*}
    Déterminer si le nombre \num{4} est solution de chacune de ces équations.
    \begin{enumerate}
        \item $5x-6=3x+2$
        \item $x^2-9=3x-5$
        \item $\dfrac{x-1}{12}=\dfrac{1}{4}$
    \end{enumerate}
\end{exercice*}
\begin{corrige}
    %\setcounter{partie}{0} % Pour s'assurer que le compteur de \partie est à zéro dans les corrigés
    %\phantom{rrr}    
    Déterminer si le nombre \num{4} est solution de chacune de ces équations.

    {\color{red}Comme pour l'exercice précédent.}
    \begin{enumerate}
        \item $5x-6=3x+2$
        
        {\color{red}Pour $x=4$ :
        
        - d'une part \dots
        
        - d'autre part \dots
        
        les résultats sont égaux, donc $4$ est solution de l'équation.
        }
        \item $x^2-9=3x-5$
        
        {\color{red}Pour $x=4$ :
        
        - d'une part \dots
        
        - d'autre part \dots
        
        les résultats sont égaux donc $4$ est solution de cette équation.
        }
        \item $\dfrac{x-1}{12}=\dfrac{1}{4}$
        
        {\color{red}Pour $x=4$ :
        
        - d'une part \dots
        
        - d'autre part \dots
        
        les résultats sont égaux donc $4$ est solution de cette équation.
        }
    \end{enumerate}
\end{corrige}

