\begin{exercice*}
    Cette figure est une vue de la surface au sol du C.D.I. d'un collège qui doit être réaménagé
    en deux parties distinctes : une salle de recherche et une salle de travail.

    $ABCE$ est un trapèze rectangle tel que $AB=\Lg[m]{9}$, $BC=\Lg[m]{8}$ et $DE=\Lg[m]{6}$.
    $M$ est un point du segment $[AB]$. 
    
    On pose $AM=x$ avec $x$ désignant une distance \textbf{exprimée en mètre} : $0\leq x \leq 9$.

    \begin{Geometrie}[CoinHD={(17u,10u)}]
        u:=0.5*u;
        pair A,B,C,D,E,F,M;
        E=u*(1,1);
        D-E=u*(6,0);
        F-D=u*(3,0);
        C-F=u*(6,0);
        B-C=u*(0,8);
        M-B=u*(-6,0);
        A-M=u*(-3,0);
        fill polygone(F,C,B,M) withcolor LightGray;
        trace polygone(E,C,B,A);
        trace segment(A,D) dashed evenly;
        trace segment(M,F);
        trace codeperp(A,D,E,8);
        trace codeperp(C,F,M,8);
        trace codeperp(B,C,F,8);
        trace codeperp(M,B,C,8);
        trace cotation(A,M,3mm,2mm,btex $x$ etex);
        pair st,sr[];
        st=iso(F,C,B,M);
        label(btex \textbf{Salle de travail} etex,st);
        sr1-iso(E,F)=u*(1.5,4);
        sr2-iso(E,F)=u*(1.5,3);
        sr3-iso(E,F)=u*(1.5,2);
        label(btex \textbf{Salle} etex,sr1);
        label(btex \textbf{de} etex,sr2);
        label(btex \textbf{Recherche} etex,sr3);
        label.llft(btex $E$ etex,E);
        label.bot(btex  $D$ etex,D);
        label.bot(btex  $F$ etex,F);
        label.lrt(btex  $C$ etex,C);
        label.urt(btex  $B$ etex,B);
        label.urt(btex  $M$ etex,M);
        label.ulft(btex $A$ etex,A);
    \end{Geometrie}

    \textbf{Rappel :} L'aire d'un trapèze de hauteur $h$ , de bases $b$ et $B$, est donné par $\mathcal{A}=\dfrac{h(b+B)}{2}$.

    La documentaliste souhaite que l'aire de la salle de travail soit égale à celle de la salle de recherche.
    \begin{enumerate}
        \item Dans cette question, uniquement, on suppose que $x=1$. Calculer l'aire du trapèze $AMFE$, celle de la salle de
        recherche, et l'aire du rectangle MBCF, celle de la salle de travail.
        \item Exprimer en fonction de x l'aire du trapèze $AMFE$.
        \item Exprimer en fonction de x l'aire du rectangle $MBCF$.
        \item Conclure.
    \end{enumerate}
\end{exercice*}
\begin{corrige}
    %\setcounter{partie}{0} % Pour s'assurer que le compteur de \partie est à zéro dans les corrigés
    %\phantom{rrr}    
    Cette figure est une vue de la surface au sol du C.D.I. d'un collège qui doit être réaménagé
    en deux parties distinctes : une salle de recherche et une salle de travail.

    $ABCE$ est un trapèze rectangle tel que $AB=\Lg[m]{9}$, $BC=\Lg[m]{8}$ et $DE=\Lg[m]{6}$.
    $M$ est un point du segment $[AB]$. 
    
    On pose $AM=x$ avec $x$ désignant une distance \textbf{exprimée en mètre} : $0\leq x \leq 9$.

    \smallskip
    \hspace*{-5mm}
    \begin{Geometrie}[CoinHD={(17u,10u)}]
        u:=0.5*u;
        pair A,B,C,D,E,F,M;
        E=u*(1,1);
        D-E=u*(6,0);
        F-D=u*(3,0);
        C-F=u*(6,0);
        B-C=u*(0,8);
        M-B=u*(-6,0);
        A-M=u*(-3,0);
        fill polygone(F,C,B,M) withcolor LightGray;
        trace polygone(E,C,B,A);
        trace segment(A,D) dashed evenly;
        trace segment(M,F);
        trace codeperp(A,D,E,8);
        trace codeperp(C,F,M,8);
        trace codeperp(B,C,F,8);
        trace codeperp(M,B,C,8);
        trace cotation(A,M,3mm,2mm,btex $x$ etex);
        pair st,sr[];
        st=iso(F,C,B,M);
        label(btex \textbf{Salle de travail} etex,st);
        sr1-iso(E,F)=u*(1.5,4);
        sr2-iso(E,F)=u*(1.5,3);
        sr3-iso(E,F)=u*(1.5,2);
        label(btex \textbf{Salle} etex,sr1);
        label(btex \textbf{de} etex,sr2);
        label(btex \textbf{Recherche} etex,sr3);
        label.llft(btex $E$ etex,E);
        label.bot(btex  $D$ etex,D);
        label.bot(btex  $F$ etex,F);
        label.lrt(btex  $C$ etex,C);
        label.urt(btex  $B$ etex,B);
        label.urt(btex  $M$ etex,M);
        label.ulft(btex $A$ etex,A);
    \end{Geometrie}

    \textbf{Rappel :} L'aire d'un trapèze de hauteur $h$ , de bases $b$ et $B$, est donné par $\mathcal{A}=\dfrac{h(b+B)}{2}$.

    La documentaliste souhaite que l'aire de la salle de travail soit égale à celle de la salle de recherche.
    \begin{enumerate}
        \item Dans cette question, uniquement, on suppose que $x=1$. Calculer l'aire du trapèze $AMFE$, celle de la salle de
        recherche, et l'aire du rectangle MBCF, celle de la salle de travail.

        {\color{red}$\mathcal{A}_{AMFE}=\dfrac{(7+1)\times 8}{2}=\Aire[m]{32}$
        
        $\mathcal{A}_{MBCF}=8\times 8=\Aire[m]{64}$}
        \item Exprimer en fonction de x l'aire du trapèze $AMFE$.
        
        {\color{red}
        $\mathcal{A}_{AMFE}=\dfrac{(6+x+x)\times 8}{2}$
        
        $\mathcal{A}_{AMFE}=(6+2x)\times4=24+8x$}
        \item Exprimer en fonction de x l'aire du rectangle $MBCF$.
        
        {\color{red}
        $\mathcal{A}_{MBCF}=(9-x)\times 8$
        
        $\mathcal{A}_{MBCF}=72-8x$}
    \end{enumerate}
    \Coupe
    \begin{enumerate}
        \setcounter{enumi}{3}
        \item Conclure.
        
        {\color{red}$\mathcal{A}_{AMFE}=\mathcal{A}_{MBCF}$
        
        d'où on tire l'équation $24+8x=72-8x$
        
        \ResolEquation[Decomposition,Decimal]{8}{24}{-8}{72}
        
        Les deux salles auront donc la même aire à condition que $x=\Lg[m]{3}$}
    \end{enumerate}
\end{corrige}

