\begin{exercice*}[\tableurLogo]
    La colonne B donne les valeurs de l’expression $2x^2-3x-9$ pour les valeurs de $x$ de la colonne A.

    \begin{minipage}{0.45\linewidth}
        \begin{Tableur}[Bandeau=false,LargeurUn=25pt,Largeur=50pt,Colonnes=2]
            $x$&$2x^2-3x-9$\\
            \num{-2.5}  &\num{11}\\
            \num{-2}    &\num{5}\\
            \num{-1.5}  &\num{0}\\
            \num{-1}    &\num{-4}\\
            \num{-0.5}  &\num{-7}\\        
            \num{0}     &\num{-9}\\
            \num{0.5}   &\num{-10}\\
            \num{1}     &\num{-10}\\
        \end{Tableur}
    \end{minipage}
    \hfill
    \begin{minipage}{0.45\linewidth}
        \begin{Tableur}[Bandeau=false,LargeurUn=25pt,Largeur=25pt,Colonnes=2,DebutLignes=10]
            \num{1.5}  &\num{-9}\\
            \num{2}    &\num{-7}\\
            \num{2.5}  &\num{-4}\\
            \num{3}    &\num{0}\\
            \num{3.5}  &\num{5}\\
            \num{4}    &\num{11}\\
            \num{4.5}  &\num{18}\\
            \num{5}    &\num{26}\\
            &\\
        \end{Tableur}
    \end{minipage}
    \begin{enumerate}
        \item Si on tape le nombre 6 dans la cellule A18, déterminer la valeur obtenue dans la cellule B18.
        \item À l’aide du tableur, trouve deux solutions de l’équation : $2x^2-3x-9=0$.
    \end{enumerate}
\end{exercice*}
% \begin{corrige}
%     %\setcounter{partie}{0} % Pour s'assurer que le compteur de \partie est à zéro dans les corrigés
%     %\phantom{rrr}    
%     Pas de correction
% \end{corrige}

