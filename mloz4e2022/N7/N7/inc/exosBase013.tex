\begin{exercice*}
    Un sac de 250 billes rouges et noires contient 18 billes rouges de plus que de billes noires.

    Déterminer le nombre de billes de chaque couleur en utilisant la méthode ci-dessous.

    On désigne par $x$ le nombre de billes noires.
    \begin{enumerate}
        \item Exprimer en fonction de x :
        \begin{itemize}
            \item le nombre de billes rouges,
            \item le nombre total de billes.
        \end{itemize}
            \item Écrire une équation qui correspond à la résolution du problème, puis la résoudre.
            \item Conclure en donnant le nombre de billes de chaque couleur.
    \end{enumerate}
\end{exercice*}
\begin{corrige}
    %\setcounter{partie}{0} % Pour s'assurer que le compteur de \partie est à zéro dans les corrigés
    %\phantom{rrr}    
    Un sac de 250 billes rouges et noires contient 18 billes rouges de plus que de billes noires.

    Déterminer le nombre de billes de chaque couleur en utilisant la méthode ci-dessous.

    On désigne par $x$ le nombre de billes noires.
    \begin{enumerate}
        \item Exprimer en fonction de x : 
        \begin{itemize}
            \item le nombre de billes rouges, \textcolor{red}{$x+18$}
            \item le nombre total de billes.\textcolor{red}{$x+x+18=2x+18$}
        \end{itemize}
            \item Écrire une équation qui correspond à la résolution du problème, puis la résoudre. 
            
            {\color{red}$2x+18=250$

            \ResolEquation[Decomposition]{2}{18}{0}{250}
            }
            \item Conclure en donnant le nombre de billes de chaque couleur.
            
            \textcolor{red}{Il y a donc 116 billes noires et 134 billes rouges dans le sac.}
    \end{enumerate}
\end{corrige}

