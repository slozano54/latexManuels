\begin{exercice*}
    Dans chaque cas, la balance est en équilibre. Écrire une équation exprimant chuaque situation, puis calculer la masse du triangle.
    \begin{enumerate}
        \item \phantom{rrr}
        
        \begin{center}
            \begin{pspicture}(-4,0.5)(4,2)
                \psline[linewidth=0.05cm]{*-*}(-2.25,0.5)(2.25,0.5)
                \psline[linewidth=0.05cm]{*-}(2.25,0.5)(2.25,1)
                \psline[linewidth=0.05cm]{*-}(-2.25,0.5)(-2.25,1)
                \psline[linewidth=0.05cm](-3.75,1.5)(-3.5,1)(-1,1)(-0.75,1.5)
                \psline[linewidth=0.05cm](0.75,1.5)(1,1)(3.5,1)(3.75,1.5)
                \psline[linewidth=0.05cm]{*->}(0,0.5)(0,2)
                \multips(-3,1)(0.5,0){3}{\pspolygon*(0,0)(0.5,0)(0.25,0.5)}                
                \multips(-2.75,1.5)(0.5,0){2}{\pspolygon*(0,0)(0.5,0)(0.25,0.5)}
                \multips(1.05,1)(0.5,0){2}{\pspolygon*(0,0)(0.5,0)(0.25,0.5)}
                % \pspolygon(-1.1,1)(-1.7,1)(-1.7,1.8)(-1.1,1.8)\psellipse(-1.4,1.9)(0.12,0.1)
                % \uput[90](-1.4,1.2){\scriptsize{400}}
                \pspolygon(2.2,1)(2.85,1)(2.85,1.9)(2.2,1.9)\psellipse(2.525,2)(0.12,0.1)
                \uput[90](2.525,1.2){\scriptsize{300}}
                \pspolygon(2.95,1)(3.45,1)(3.45,1.6)(2.95,1.6)\psellipse(3.2,1.7)(0.12,0.1)
                \uput[90](3.2,1.1){\scriptsize{75}}
            \end{pspicture}
        \end{center}
        \item \phantom{rrr}
                
        \begin{center}
            \begin{pspicture}(-4,0.5)(4,2)
                \psline[linewidth=0.05cm]{*-*}(-2.25,0.5)(2.25,0.5)
                \psline[linewidth=0.05cm]{*-}(2.25,0.5)(2.25,1)
                \psline[linewidth=0.05cm]{*-}(-2.25,0.5)(-2.25,1)
                \psline[linewidth=0.05cm](-3.75,1.5)(-3.5,1)(-1,1)(-0.75,1.5)
                \psline[linewidth=0.05cm](0.75,1.5)(1,1)(3.5,1)(3.75,1.5)
                \psline[linewidth=0.05cm]{*->}(0,0.5)(0,2)
                \multips(-3.4,1)(0.5,0){3}{\pspolygon*(0,0)(0.5,0)(0.25,0.5)}
                % \multips(-3.15,1.5)(0.5,0){2}{\pspolygon*(0,0)(0.5,0)(0.25,0.5)}
                \multips(1.25,1)(0.5,0){1}{\pspolygon*(0,0)(0.5,0)(0.25,0.5)}
                \pspolygon(-1.2,1)(-1.8,1)(-1.8,1.7)(-1.2,1.7)\psellipse(-1.5,1.8)(0.12,0.1)
                \uput[90](-1.5,1.2){\scriptsize{70}}
                \pspolygon(2.2,1)(2.85,1)(2.85,1.9)(2.2,1.9)\psellipse(2.525,2)(0.12,0.1)
                \uput[90](2.525,1.2){\scriptsize{200}}
                \pspolygon(2.95,1)(3.45,1)(3.45,1.6)(2.95,1.6)\psellipse(3.2,1.7)(0.12,0.1)
                \uput[90](3.2,1.1){\scriptsize{50}}
            \end{pspicture}
        \end{center}
    \end{enumerate}
\end{exercice*}
\begin{corrige}
    %\setcounter{partie}{0} % Pour s'assurer que le compteur de \partie est à zéro dans les corrigés
    %\phantom{rrr}    
    Dans chaque cas, la balance est en équilibre. Écrire une équation exprimant chuaque situation, puis calculer la masse du triangle.

    \begin{enumerate}
        \item \phantom{rrr}        

            \hspace*{-5mm}\begin{pspicture}(-4,0.5)(4,2)
                \psline[linewidth=0.05cm]{*-*}(-2.25,0.5)(2.25,0.5)
                \psline[linewidth=0.05cm]{*-}(2.25,0.5)(2.25,1)
                \psline[linewidth=0.05cm]{*-}(-2.25,0.5)(-2.25,1)
                \psline[linewidth=0.05cm](-3.75,1.5)(-3.5,1)(-1,1)(-0.75,1.5)
                \psline[linewidth=0.05cm](0.75,1.5)(1,1)(3.5,1)(3.75,1.5)
                \psline[linewidth=0.05cm]{*->}(0,0.5)(0,2)
                \multips(-3,1)(0.5,0){3}{\pspolygon*(0,0)(0.5,0)(0.25,0.5)}                
                \multips(-2.75,1.5)(0.5,0){2}{\pspolygon*(0,0)(0.5,0)(0.25,0.5)}
                \multips(1.05,1)(0.5,0){2}{\pspolygon*(0,0)(0.5,0)(0.25,0.5)}
                % \pspolygon(-1.1,1)(-1.7,1)(-1.7,1.8)(-1.1,1.8)\psellipse(-1.4,1.9)(0.12,0.1)
                % \uput[90](-1.4,1.2){\scriptsize{400}}
                \pspolygon(2.2,1)(2.85,1)(2.85,1.9)(2.2,1.9)\psellipse(2.525,2)(0.12,0.1)
                \uput[90](2.525,1.2){\scriptsize{300}}
                \pspolygon(2.95,1)(3.45,1)(3.45,1.6)(2.95,1.6)\psellipse(3.2,1.7)(0.12,0.1)
                \uput[90](3.2,1.1){\scriptsize{75}}
            \end{pspicture}

        {\color{red} \ResolEquation[Decomposition]{5}{0}{2}{375}}
        \item \phantom{rrr}                

            \hspace*{-5mm}\begin{pspicture}(-4,0.5)(4,2)
                \psline[linewidth=0.05cm]{*-*}(-2.25,0.5)(2.25,0.5)
                \psline[linewidth=0.05cm]{*-}(2.25,0.5)(2.25,1)
                \psline[linewidth=0.05cm]{*-}(-2.25,0.5)(-2.25,1)
                \psline[linewidth=0.05cm](-3.75,1.5)(-3.5,1)(-1,1)(-0.75,1.5)
                \psline[linewidth=0.05cm](0.75,1.5)(1,1)(3.5,1)(3.75,1.5)
                \psline[linewidth=0.05cm]{*->}(0,0.5)(0,2)
                \multips(-3.4,1)(0.5,0){3}{\pspolygon*(0,0)(0.5,0)(0.25,0.5)}
                % \multips(-3.15,1.5)(0.5,0){2}{\pspolygon*(0,0)(0.5,0)(0.25,0.5)}
                \multips(1.25,1)(0.5,0){1}{\pspolygon*(0,0)(0.5,0)(0.25,0.5)}
                \pspolygon(-1.2,1)(-1.8,1)(-1.8,1.7)(-1.2,1.7)\psellipse(-1.5,1.8)(0.12,0.1)
                \uput[90](-1.5,1.2){\scriptsize{70}}
                \pspolygon(2.2,1)(2.85,1)(2.85,1.9)(2.2,1.9)\psellipse(2.525,2)(0.12,0.1)
                \uput[90](2.525,1.2){\scriptsize{200}}
                \pspolygon(2.95,1)(3.45,1)(3.45,1.6)(2.95,1.6)\psellipse(3.2,1.7)(0.12,0.1)
                \uput[90](3.2,1.1){\scriptsize{50}}
            \end{pspicture}

        {\color{red}\ResolEquation[Decomposition]{3}{70}{1}{250}}
    \end{enumerate}
    \vspace*{-20mm}
\end{corrige}

