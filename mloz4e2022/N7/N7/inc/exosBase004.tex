\begin{exercice*}
    Déterminer si le nombre \num{-2} est solution de chacune de ces équations.
    \begin{enumerate}
        \item $7x-3=6x-5$
        \item $4x-7=7x+1$
        \item $\num{-2.7}x+5=\num{3.3}x-\num{6.2}$
    \end{enumerate}
\end{exercice*}
\begin{corrige}
    %\setcounter{partie}{0} % Pour s'assurer que le compteur de \partie est à zéro dans les corrigés
    %\phantom{rrr}    
    Déterminer si le nombre \num{-2} est solution de chacune de ces équations.
    \begin{enumerate}
        \item $7x-3=6x-5$
        
        {\color{red}Pour $x=-2$ :
        
        - d'une part $7\times (-2) -3 = -14 - 3 = -17$
        
        - d'autre part $6\times (-2) - 5 = -12 - 5 = -17$
        
        ces résultats sont égaux donc $-2$ est solution de cette équation.
        }
        \item $4x-7=7x+1$
        
        {\color{red}Pour $x=-2$ :
        
        - d'une part $4\times (-2) -7 = -8 - 7 = -15$
        
        - d'autre part $7\times (-2) +1 = -14 +1 = -13$
        
        ces résultats sont différents donc $-2$ n'est pas solution de cette équation.
        }
        \item $\num{-2.7}x+5=\num{3.3}x-\num{6.2}$
        
        {\color{red}Pour $x=-2$ :
        
        - d'une part $\num{-2.7}\times (-2) +5 = \num{5.4}+5 = \num{10.4}$
        
        - d'autre part $\num{3.3}\times (-2) - \num{6.2} = \num{-6.6} - \num{6.2} = \num{-12.8}$
        
        ces résultats sont différents donc $-2$ n'est pas solution de cette équation.
        }
    \end{enumerate}
\end{corrige}

