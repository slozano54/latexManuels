\begin{exercice*}
    On juxtapose un triangle équilatéral et un carré comme sur le schéma.

    \begin{Geometrie}
        pair t[],c[];
        t1=u*(1,1);
        t2-t1=u*(4,0);
        c1=t2;
        c2-t1=u*(7,0);
        t3=rotation(t2,t1,60);
        c3=rotation(c1,c2,-90);
        c4=rotation(c2,c1,90);
        trace polygone(t1,t2,t3);
        trace polygone(c1,c2,c3,c4);
        trace cotationmil(t1,c2,-5mm,20,btex \Lg[cm]{14} etex);
    \end{Geometrie}

    Justifier s'il est possible ou non que le triangle et le carré aient le même périmètre.
\end{exercice*}
\begin{corrige}
    %\setcounter{partie}{0} % Pour s'assurer que le compteur de \partie est à zéro dans les corrigés
    %\phantom{rrr}    
        On juxtapose un triangle équilatéral et un carré comme sur le schéma.

    \begin{Geometrie}
        pair t[],c[];
        t1=u*(1,1);
        t2-t1=u*(4,0);
        c1=t2;
        c2-t1=u*(7,0);
        t3=rotation(t2,t1,60);
        c3=rotation(c1,c2,-90);
        c4=rotation(c2,c1,90);
        trace polygone(t1,t2,t3);
        trace polygone(c1,c2,c3,c4);
        trace cotationmil(t1,c2,-5mm,20,btex \Lg[cm]{14} etex);
    \end{Geometrie}

    Justifier s'il est possible ou non que le triangle et le carré aient le même périmètre.

    {\color{red} Notons $t$ la longueur du côté du triangle et $c$ celle du côté du carré.

    On a donc les rélations suivantes : $3t=4c$ et $t+c=14$.

    $t+c=14$ est équivalente à $c=14-t$, l'autre égalité s'écrit donc $3t=4\times(14-t)$ qui revient à $3t=56-4t$.

    \ResolEquation[Decomposition,Decimal]{3}{0}{-4}{56}

    Cette équation admet une solution, il est donc possible que le triangle et le carré aient la même aire et cela
    se produit lorsque le côté du triangle mesure \Lg[cm]{8} et le côté du carré mesure \Lg[cm]{6}.
    }
\end{corrige}

