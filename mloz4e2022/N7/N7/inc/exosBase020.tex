\begin{exercice*}[\tableurLogo]
    À l'aide du tableur, complèter la feuille de calcul ci-dessous.

    \begin{minipage}{0.45\linewidth}
        \begin{Tableur}[Bandeau=false,LargeurUn=25pt,Largeur=40pt,Colonnes=2]
            $x$&$x^2+x-2$\\
            \num{-5}    &\\
            \num{-4.5}  &\\
            \num{-4}    &\\
            \num{-3.5}  &\\
            \num{-3}    &\\
            \num{-2.5}  &\\
            \num{-2}    &\\
            \num{-1.5}  &\\
            \num{-1}    &\\
            \num{-0.5}  &\\        
        \end{Tableur}
    \end{minipage}
    \hfill
    \begin{minipage}{0.45\linewidth}
        \begin{Tableur}[Bandeau=false,LargeurUn=25pt,Largeur=40pt,Colonnes=2,DebutLignes=12]
            \num{0}    &\\
            \num{0.5}  &\\
            \num{1}    &\\
            \num{1.5}  &\\
            \num{2}    &\\
            \num{2.5}  &\\
            \num{3}    &\\
            \num{3.5}  &\\
            \num{4}    &\\
            \num{4.5}  &\\
            \num{5}    &\\
        \end{Tableur}
    \end{minipage}

    On souhaite résoudre l'équation d'inconnue $x$ : $x^2+x-2=4$.
    \begin{enumerate}
        \item Margot dit que le nombre 2 est solution.
        
        Déterminer si elle a raison. Justifier.
        \item Léo pense que le nombre 18 est solution.
        
        Déterminer si il a raison. Justifier.
        \item Justifier s'il est possible de trouver une autre solution.
    \end{enumerate}
\end{exercice*}
% \begin{corrige}
%     %\setcounter{partie}{0} % Pour s'assurer que le compteur de \partie est à zéro dans les corrigés
%     %\phantom{rrr}    
%     Pas de correction
% \end{corrige}

