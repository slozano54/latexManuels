\begin{exercice*}
    Déterminer si le nombre 3 est solution de chacune de ces équations.
    \begin{enumerate}
        \item $4x+2=5$
        \item $7-5x=-8$
        \item $\num{1.5}x-\num{4.5}=0$
    \end{enumerate}
\end{exercice*}
\begin{corrige}
    %\setcounter{partie}{0} % Pour s'assurer que le compteur de \partie est à zéro dans les corrigés
    %\phantom{rrr}    
    Déterminer si le nombre 3 est solution de chacune de ces équations.
    \begin{enumerate}
        \item $4x+2=5$
        
        \textcolor{red}{$4\times 3 + 2 = 12 + 2 = 14$ or $14 \neq 5$ donc $3$ n'est pas solution de cette équation.}
        \item $7-5x=-8$
        
        \textcolor{red}{$7-5\times 3 = 7-15 = -8$ donc $3$ est solution de cette équation.}
        \item $\num{1.5}x-\num{4.5}=0$
        
        \textcolor{red}{$\num{1.5}\times 3 - \num{4.5} = \num{4.5} - \num{4.5} = 0$ donc $3$ est solution de cette équation.}
    \end{enumerate}
\end{corrige}

