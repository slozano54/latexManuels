\begin{exercice*}
    La tirelire de Paola contient 200 pièces, les unes de \Prix{0,20} et les autres de \Prix{0,50}, pour un total de \Prix{52,30}.

    Déterminer le nombre de pièces de chaque sorte.
\end{exercice*}
\begin{corrige}
    %\setcounter{partie}{0} % Pour s'assurer que le compteur de \partie est à zéro dans les corrigés
    %\phantom{rrr}    
    La tirelire de Paola contient 200 pièces, les unes de \Prix{0,20} et les autres de \Prix{0,50}, pour un total de \Prix{52,30}.

    Déterminer le nombre de pièces de chaque sorte.

    {\color{red} On désigne par $x$ le nombre de pièces de \Prix{0.20}, il y en a donc $200-x$ de \Prix{0.50}.
    
    
    On obtient donc l'équation $x\times\Prix{0.20} + (200-x)\times\Prix{0.50}=\Prix{52.30}$ soit $\num{-0.3}x+100=\num{52.30}$
    
    \ResolEquation[Decomposition,Decimal]{-0.3}{100}{0}{52.30}

    Paola a 159 pièces de \Prix{0.20} et 41 pièces de \Prix{0.50}.
}
\end{corrige}

