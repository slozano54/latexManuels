\begin{exercice*}
    Medhi et Sarah commencent par taper un même nombre sur leur calculatrice.
    
    Mehdi tape ensuite la suite de touches suivante :
    \begin{center}
        \Calculatrice{/$\times$,/4,/$-$,/7,/$=$}
    \end{center}
    Tandis que Sarah tape celle-ci :
    \begin{center}
        \Calculatrice{/$+$,/3,/$=$,/$\times$,/2,/$=$}
    \end{center}
    Ils constatent qu'ils obtiennent le même résultat.

    Déterminer le nombre qu'ils ont tapé au départ.
\end{exercice*}
\begin{corrige}
    %\setcounter{partie}{0} % Pour s'assurer que le compteur de \partie est à zéro dans les corrigés
    %\phantom{rrr}    
    Medhi et Sarah commencent par taper un même nombre sur leur calculatrice.
    
    Mehdi tape ensuite la suite de touches suivante :

        \Calculatrice{/$\times$,/4,/$-$,/7,/$=$}

    Tandis que Sarah tape celle-ci :

        \Calculatrice{/$+$,/3,/$=$,/$\times$,/2,/$=$}

    Ils constatent qu'ils obtiennent le même résultat.

    Déterminer le nombre qu'ils ont tapé au départ.

    {\color{red} Soit $x$ le nombre tapé au départ, on obtient l'équation : $4x-7=(x+3)\times 2$.

    Elle est équivalente à  $4x-7=2x+6$ d'où :
    }
    \Coupe
    {\color{red}
    \ResolEquation[Decomposition]{4}{-7}{2}{6}

    Ils ont donc tapé \num{6.5} au départ.
     }
\end{corrige}

