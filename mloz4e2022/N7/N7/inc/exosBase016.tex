\begin{exercice*}
    Dans une assemblée de 500 personnes, il y a deux fois plus de Belges que de Luxembourgeois, 
    et 48 Néerlandais de plus que de Luxembourgeois. 
    
    Déterminer la composition de l'assemblée en suivant la méthode ci-dessous. 
    
    On désigne par $x$ le nombre de Luxembourgeois.
    \begin{enumerate}
        \item Exprimer en fonction du nombre $x$ :
        \begin{itemize}
            \item le nombre de Belges ;
            \item le nombre de Néerlandais ;
            \item le nombre total de personnes.
        \end{itemize}
        \item Écrire une équation qui traduit que le nombre total de personnes est 500, puis la résoudre.
        \item Conclure sur la composition de cette assemblée.
    \end{enumerate}
\end{exercice*}
\begin{corrige}
    %\setcounter{partie}{0} % Pour s'assurer que le compteur de \partie est à zéro dans les corrigés
    %\phantom{rrr}    
    Dans une assemblée de 500 personnes, il y a deux fois plus de Belges que de Luxembourgeois, 
    et 48 Néerlandais de plus que de Luxembourgeois. 
    
    Déterminer la composition de l'assemblée en suivant la méthode ci-dessous. 
    
    On désigne par $x$ le nombre de Luxembourgeois.

    \begin{enumerate}
        \item Exprimer en fonction du nombre $x$ :
        \begin{itemize}
            \item le nombre de Belges ; \textcolor{red}{$2x$}
            \item le nombre de Néerlandais ; \textcolor{red}{$x+48$}
            \item le nombre total de personnes. \textcolor{red}{$x+2x+x+48$}
        \end{itemize}
        \item Écrire une équation qui traduit que le nombre total de personnes est 500, puis la résoudre.
        
        {\color{red} $x+2x+x+48=500$ soit $4x+48=500$
        
        \ResolEquation[Decomposition,Decimal]{4}{48}{0}{500}
        }
        \item Conclure sur la composition de cette assemblée.
        
        {\color{red} Dans cette assemblée, il y a donc 113 Luxembourgeois, 226 Belges et 161 Néerlandais.}
    \end{enumerate}
\end{corrige}

