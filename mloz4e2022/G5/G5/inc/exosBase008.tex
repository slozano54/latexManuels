\begin{exercice*}
    Représenter, en perspective cavalière, un cône de révolution de hauteur \num{3.5} \textit{u.l.} et dont le rayon de base vaut \num{2.5} \textit{u.l.}.

    On a déjà placer le centre $O$ de la base du cône.

    En perspective cavalière, la base d'une cône de révolution est représentée par une ellipse.
    \begin{center}
        \begin{Geometrie}[CoinHD={(10u,6u)}]
            trace grille(0.5) withcolor Grey;
            z0=(5,1)*u;
            marque_p:="croix";
            pointe(z0);
            label.lft(btex$O$etex,z0);
            pair R[];
            R1=u*(1,5.5);
            R2=u*(2,5.5);
            trace segment(R1,R2) withpen pencircle scaled 1bp;
            label.bot(btex u.l. etex,iso(R1,R2));
        \end{Geometrie}
    \end{center}
\end{exercice*}
\begin{corrige}
    %\setcounter{partie}{0} % Pour s'assurer que le compteur de \partie est à zéro dans les corrigés
    %\phantom{rrr}    
    Représenter, en perspective cavalière, un cône de révolution de hauteur \num{3.5} \textit{u.l.} et dont le rayon de base vaut \num{2.5} \textit{u.l.}.

    On a déjà placer le centre $O$ de la base du cône.

    En perspective cavalière, la base d'une cône de révolution est représentée par une ellipse.
    \begin{center}
        \scalebox{0.8}{
            \Solide[%
                Nom=cone,
                RayonCone=1.25,
                HauteurCone=1.75,
                ListeSommets={S,O},
                Phi=0,        
                Traces={%
                %horizontal
                for i=-4 upto 10:
                trace segment((0,-2.5,i*0.25),(0,2.5,i*.25)) withcolor Grey;
                endfor;
                %vertical
                for i=-10 upto 10:
                trace segment((0,i*0.25,-1),(0,i*.25,2.5)) withcolor Grey;
                endfor;
                Label.llft(btex O etex,O);
                Label.urt(btex S etex,S);
                color D;
                D-O=rayoncone*(cosd(30),sind(30),0);
                % Label.lrt(btex $D$ etex,D);
                trace codeperp(S,O,D,5) withpen pencircle scaled 1bp;
                trace segment(S,O) withpen pencircle scaled 1bp dashed evenly;
                trace segment(D,O) withpen pencircle scaled 1bp dashed evenly;
                marque_p:="croix";
                u:=u*0.5;
                pointe(O);
                u:=u*2;
                color R[];
                R1=(0,-1.75,2.25);
                R2=(0,-1.25,2.25);
                trace segment(R1,R2) withpen pencircle scaled 1bp;
                Label.bot(btex u.l. etex,iso(R1,R2));
                }
            ]
        }
    \end{center}
\end{corrige}

