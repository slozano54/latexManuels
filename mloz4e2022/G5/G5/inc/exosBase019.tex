\begin{exercice*}
    Soient des cônes de révolution de rayon $r$, de diamètre $D$ et de hauteur $h$.

    Compléter le tableau. Justifier chaque réponse.

    {\renewcommand{\arraystretch}{1.2}
    \begin{tabular}{m{0.07\linewidth}|>{\centering\arraybackslash}m{0.12\linewidth}|>{\centering\arraybackslash}m{0.12\linewidth}|>{\centering\arraybackslash}m{0.12\linewidth}|>{\centering\arraybackslash}m{0.15\linewidth}|>{\centering\arraybackslash}m{0.3\linewidth}|}
        \cline{2-6}
        &\cellcolor{LightGray}$r$&\cellcolor{LightGray}$D$&\cellcolor{LightGray}$h$&\cellcolor{LightGray}Volume exact&\cellcolor{LightGray}Volume au millième près\\
        \cline{2-6}
        \begin{enumerate}\item\phantom{r}\end{enumerate}&\Lg[cm]{5}&&&\num{35}$\pi$~\Vol[cm]{}&\\
        \cline{2-6}
        \begin{enumerate}\setcounter{enumi}{1}\item\phantom{r}\end{enumerate}&&\Lg[cm]{3}&\Lg[cm]{7}&&\\
        \cline{2-6}
        \begin{enumerate}\setcounter{enumi}{2}\item\phantom{r}\end{enumerate}&&&\Lg[cm]{2}&\num{54}$\pi$~\Vol[cm]{}&\\
        \cline{2-6}
    \end{tabular}
    }  
\end{exercice*}
\begin{corrige}
    %\setcounter{partie}{0} % Pour s'assurer que le compteur de \partie est à zéro dans les corrigés
    %\phantom{rrr}    
    Soient des cônes de révolution de rayon $r$, de diamètre $D$ et de hauteur $h$.

    Compléter le tableau. Justifier chaque réponse.

    {\renewcommand{\arraystretch}{1.2}
    \begin{tabular}{m{0.07\linewidth}|>{\centering\arraybackslash}m{0.12\linewidth}|>{\centering\arraybackslash}m{0.12\linewidth}|>{\centering\arraybackslash}m{0.12\linewidth}|>{\centering\arraybackslash}m{0.15\linewidth}|>{\centering\arraybackslash}m{0.3\linewidth}|}
        \cline{2-6}
        &\cellcolor{LightGray}$r$&\cellcolor{LightGray}$D$&\cellcolor{LightGray}$h$&\cellcolor{LightGray}Volume exact&\cellcolor{LightGray}Volume au millième près\\
        \cline{2-6}
        \begin{enumerate}\item\phantom{r}\end{enumerate}&\Lg[cm]{5}&\textcolor{red}{\Lg[cm]{10}}&\textcolor{red}{\Lg[cm]{4.2}}&\num{35}$\pi$~\Vol[cm]{}&\textcolor{red}{\num{109.956}}\\
        \cline{2-6}
        \begin{enumerate}\setcounter{enumi}{1}\item\phantom{r}\end{enumerate}&\textcolor{red}{\Lg[cm]{1.5}}&\Lg[cm]{3}&\Lg[cm]{7}&\textcolor{red}{\num{5.25}$\pi$~\Vol[cm]{}}&\textcolor{red}{\num{16.493}}\\
        \cline{2-6}
        \begin{enumerate}\setcounter{enumi}{2}\item\phantom{r}\end{enumerate}&\textcolor{red}{\Lg[cm]{9}}&\textcolor{red}{\Lg[cm]{18}}&\Lg[cm]{2}&\num{54}$\pi$~\Vol[cm]{}&\textcolor{red}{\num{169.646}}\\
        \cline{2-6}
    \end{tabular}
    }  

    \textcolor{red}{$D=2\times r$}

    \begin{enumerate}        
        \setcounter{enumi}{0}
        \item \textcolor{red}{$25\times h \div 3 = 35$ d'où $h=\Lg[cm]{4.2}$}
        \item \textcolor{red}{Calcul du volume !}
        \item \textcolor{red}{$2r^2\pi \div 3 = \num{54}\pi$ d'où $r^2=81$ soit $r=\Lg[cm]{9}$ et $D=\Lg[cm]{18}$}
    \end{enumerate}
\end{corrige}

