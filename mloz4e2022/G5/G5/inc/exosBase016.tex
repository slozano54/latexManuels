\begin{exercice*}
    Calculer le volume exact de chacun des cônes de révolution. Le résultat sera exprimé à l'aide de $\pi$.

    Les figures sont données à titre indicatif.

    \begin{enumerate}
        \item Méthode détaillée :
        
        \begin{minipage}{0.45\linewidth}
            \begin{center}
                \Solide[%
                    Nom=cone,
                    HauteurCone=1.7,
                    RayonCone=1,                
                    ListeSommets={S,O},
                    Phi=0,                            
                    Traces={%
                    % Label.top(btex $S$ etex,S);
                    % Label.bot(btex $O$ etex,O);
                    trace segment(S,O) dashed evenly;
                    color A,B,D;
                    A-O=rayoncone*(0,-sind(90),0);
                    B-O=rayoncone*(0,sind(90),0);
                    D-O=rayoncone*(cosd(30),sind(30),0);
                    % Label.rt(btex $B$ etex,B);
                    % Label.lft(btex $A$ etex,A);
                    % Label.lrt(btex $E$ etex,E);
                    % Label.lrt(btex $D$ etex,D);
                    % Label.ulft(btex $I$ etex,I);
                    trace cotationmil(O,S,3mm,15,btex \Lg[cm]{5.7} etex);
                    trace cotationmil(S,D,13mm,15,btex \Lg[cm]{6.3} etex);
                    trace cotationmil(A,O,-10mm,15,btex \Lg[cm]{2.8} etex);
                    trace segment(A,B) dashed evenly;
                    trace segment(S,D);                
                    marque_p:="croix";
                    trace codeperp(S,O,B,5);
                    }
                ]
            \end{center}
        \end{minipage}
        \hspace*{5mm}
        \begin{minipage}{0.55\linewidth}
            Aire de la base : 

            \medskip            
            $\mathcal{A}_{\text{de la base}} = \pi\times \pointilles[10mm]{}^2$

            \medskip
            $\mathcal{A}_{\text{de la base}} = \pointilles[10mm]\times\pi{}~\Aire[cm]{}$

            \bigskip
            Volume du cône :

            \medskip
            $\mathcal{V}_{\text{cône}} = \dfrac{\pointilles[10mm]\times \pointilles[10mm]\pi}{3}$

            \medskip
            $\mathcal{V}_{\text{cône}} = \pointilles[15mm]~\Vol[cm]{}$
        \end{minipage}
    \end{enumerate}
    \begin{minipage}{0.45\linewidth}        
        \begin{enumerate}
            \setcounter{enumi}{1}
            \item \phantom{rrr}
            
            \begin{center}
                \Solide[%
                    Nom=cone,
                    HauteurCone=1.7,
                    RayonCone=1,                
                    ListeSommets={S,O},
                    Phi=0,                            
                    Traces={%
                    Label.top(btex $S$ etex,S);
                    Label.bot(btex $O$ etex,O);
                    trace segment(S,O) dashed evenly;
                    color A,B,D;
                    A-O=rayoncone*(0,-sind(90),0);
                    B-O=rayoncone*(0,sind(90),0);
                    D-O=rayoncone*(cosd(30),sind(30),0);
                    % Label.rt(btex $B$ etex,B);
                    % Label.lft(btex $A$ etex,A);
                    % Label.lrt(btex $E$ etex,E);
                    % Label.lrt(btex $D$ etex,D);
                    % Label.ulft(btex $I$ etex,I);
                    trace cotationmil(O,S,3mm,15,btex \Lg[cm]{8.7} etex);
                    trace cotationmil(A,B,-10mm,15,btex \Lg[cm]{6.4} etex);
                    trace segment(A,B) dashed evenly;
                    trace segment(S,D);                
                    marque_p:="croix";
                    trace codeperp(S,O,B,5);
                    }
                ]
            \end{center}
        \end{enumerate}
    \end{minipage}
    \hfill        
    \begin{minipage}{0.45\linewidth}
        \begin{enumerate}
            \setcounter{enumi}{2}
            \item \phantom{rrr}
            
            \begin{center}
                \Solide[%
                    Nom=cone,
                    HauteurCone=1.7,
                    RayonCone=1,                
                    ListeSommets={S,O},
                    Phi=0,                            
                    Traces={%
                    Label.top(btex $S$ etex,S);
                    Label.bot(btex $O$ etex,O);
                    trace segment(S,O) dashed evenly;
                    color A,B,D;
                    A-O=rayoncone*(0,-sind(90),0);
                    B-O=rayoncone*(0,sind(90),0);
                    D-O=rayoncone*(cosd(30),sind(30),0);
                    % Label.rt(btex $B$ etex,B);
                    % Label.lft(btex $A$ etex,A);
                    % Label.lrt(btex $E$ etex,E);
                    % Label.lrt(btex $D$ etex,D);
                    % Label.ulft(btex $I$ etex,I);
                    trace cotationmil(O,S,3mm,15,btex \Lg[cm]{7.3} etex);
                    trace cotationmil(S,D,13mm,15,btex \Lg[cm]{7} etex);
                    trace cotationmil(A,B,-10mm,15,btex \Lg[cm]{9} etex);
                    trace segment(A,B) dashed evenly;
                    trace segment(S,D);                
                    marque_p:="croix";
                    trace codeperp(S,O,B,5);
                    }
                ]
            \end{center}
    \end{enumerate}
    \end{minipage}
\end{exercice*}
\begin{corrige}
    %\setcounter{partie}{0} % Pour s'assurer que le compteur de \partie est à zéro dans les corrigés
    %\phantom{rrr}    
    Calculer le volume exact de chacun des cônes de révolution. Le résultat sera exprimé à l'aide de $\pi$.

    Les figures sont données à titre indicatif.

    \begin{enumerate}
        \item Méthode détaillée :
        
       
        \Solide[%
            Nom=cone,
            HauteurCone=1.7,
            RayonCone=1,                
            ListeSommets={S,O},
            Phi=0,                            
            Traces={%
            % Label.top(btex $S$ etex,S);
            % Label.bot(btex $O$ etex,O);
            trace segment(S,O) dashed evenly;
            color A,B,D;
            A-O=rayoncone*(0,-sind(90),0);
            B-O=rayoncone*(0,sind(90),0);
            D-O=rayoncone*(cosd(30),sind(30),0);
            % Label.rt(btex $B$ etex,B);
            % Label.lft(btex $A$ etex,A);
            % Label.lrt(btex $E$ etex,E);
            % Label.lrt(btex $D$ etex,D);
            % Label.ulft(btex $I$ etex,I);
            trace cotationmil(O,S,3mm,15,btex \Lg[cm]{5.7} etex);
            trace cotationmil(S,D,13mm,15,btex \Lg[cm]{6.3} etex);
            trace cotationmil(A,O,-10mm,15,btex \Lg[cm]{2.8} etex);
            trace segment(A,B) dashed evenly;
            trace segment(S,D);                
            marque_p:="croix";
            trace codeperp(S,O,B,5);
            }
        ]

        Aire de la base : 

        \medskip            
        $\mathcal{A}_{\text{de la base}} = \pi\times \textcolor{red}{\num{2.8}}^2$

        \medskip
        $\mathcal{A}_{\text{de la base}} = \textcolor{red}{\num{7.84}}\times\pi{}~\Aire[cm]{}$

        \bigskip
        Volume du cône :

        \medskip
        $\mathcal{V}_{\text{cône}} = \dfrac{\textcolor{red}{\num{5.7}}\times \textcolor{red}{\num{7.84}}\pi}{3}$

        \medskip
        $\mathcal{V}_{\text{cône}} = \textcolor{red}{\num{14.896}\pi~\Vol[cm]{}}$
        \item \phantom{rrr}
        
        \Solide[%
            Nom=cone,
            HauteurCone=1.7,
            RayonCone=1,                
            ListeSommets={S,O},
            Phi=0,                            
            Traces={%
            Label.top(btex $S$ etex,S);
            Label.bot(btex $O$ etex,O);
            trace segment(S,O) dashed evenly;
            color A,B,D;
            A-O=rayoncone*(0,-sind(90),0);
            B-O=rayoncone*(0,sind(90),0);
            D-O=rayoncone*(cosd(30),sind(30),0);
            % Label.rt(btex $B$ etex,B);
            % Label.lft(btex $A$ etex,A);
            % Label.lrt(btex $E$ etex,E);
            % Label.lrt(btex $D$ etex,D);
            % Label.ulft(btex $I$ etex,I);
            trace cotationmil(O,S,3mm,15,btex \Lg[cm]{8.7} etex);
            trace cotationmil(A,B,-10mm,15,btex \Lg[cm]{6.4} etex);
            trace segment(A,B) dashed evenly;
            trace segment(S,D);                
            marque_p:="croix";
            trace codeperp(S,O,B,5);
            }
        ]

        Aire de la base : 

        \medskip            
        $\mathcal{A}_{\text{de la base}} = \pi\times \textcolor{red}{\num{3.2}}^2$

        \medskip
        $\mathcal{A}_{\text{de la base}} = \textcolor{red}{\num{10.24}}\times\pi{}~\Aire[cm]{}$

        \bigskip
        Volume du cône :

        \medskip
        $\mathcal{V}_{\text{cône}} = \dfrac{\textcolor{red}{\num{8.7}}\times \textcolor{red}{\num{10.24}}\pi}{3}$

        \medskip
        $\mathcal{V}_{\text{cône}} = \textcolor{red}{\num{29.696}\pi~\Vol[cm]{}}$
    \end{enumerate}
    \Coupe
    \begin{enumerate}
        \setcounter{enumi}{2}
        \item \phantom{rrr}
        
        \Solide[%
            Nom=cone,
            HauteurCone=1.7,
            RayonCone=1,                
            ListeSommets={S,O},
            Phi=0,                            
            Traces={%
            Label.top(btex $S$ etex,S);
            Label.bot(btex $O$ etex,O);
            trace segment(S,O) dashed evenly;
            color A,B,D;
            A-O=rayoncone*(0,-sind(90),0);
            B-O=rayoncone*(0,sind(90),0);
            D-O=rayoncone*(cosd(30),sind(30),0);
            % Label.rt(btex $B$ etex,B);
            % Label.lft(btex $A$ etex,A);
            % Label.lrt(btex $E$ etex,E);
            % Label.lrt(btex $D$ etex,D);
            % Label.ulft(btex $I$ etex,I);
            trace cotationmil(O,S,3mm,15,btex \Lg[cm]{7.3} etex);
            trace cotationmil(S,D,13mm,15,btex \Lg[cm]{7} etex);
            trace cotationmil(A,B,-10mm,15,btex \Lg[cm]{9} etex);
            trace segment(A,B) dashed evenly;
            trace segment(S,D);                
            marque_p:="croix";
            trace codeperp(S,O,B,5);
            }
        ]

        Aire de la base : 

        \medskip            
        $\mathcal{A}_{\text{de la base}} = \pi\times \textcolor{red}{\num{4.5}}^2$

        \medskip
        $\mathcal{A}_{\text{de la base}} = \textcolor{red}{\num{20.25}}\times\pi{}~\Aire[cm]{}$

        \bigskip
        Volume du cône :

        \medskip
        $\mathcal{V}_{\text{cône}} = \dfrac{\textcolor{red}{\num{7.3}}\times \textcolor{red}{\num{20.25}}\pi}{3}$

        \medskip
        $\mathcal{V}_{\text{cône}} = \textcolor{red}{\num{49.275}\pi~\Vol[cm]{}}$
    \end{enumerate}
\end{corrige}

