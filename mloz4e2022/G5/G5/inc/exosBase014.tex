\begin{exercice*}
    Calculer les volumes des pyramides. 
    
    On rappelle que $\mathcal{V}_{pyramide}=\dfrac{\text{Aire de la base}\times \text{hauteur}}{3}$.

    \hspace*{-15mm}
    \begin{minipage}{0.45\linewidth}
    \Solide[%
        Nom=pyramide,        
        HauteurPyramide=1.2,
        SommetsPyramide=5,
        Sommets=false,
        Phi=60,
        Theta=22,
        Traces={%        
        pair K,L;
        K=u*(-1,-1);
        label.bot(btex \Aire[cm]{6} etex,K);         
        L=u*(0,-2);
        label.bot(btex Pyramide 1 etex,L);         
        % trace segment(A,PiedHauteur) dashed evenly;
        fill polygone(B,C,D,E) withcolor LightGray;
        trace codeperp(A,PiedHauteur,B,5);
        trace cotationmil(A,PiedHauteur,0mm,20,btex \Lg[cm]{7.2} etex);
        trace segment(A,E);    
        trace segment(A,C) dashed evenly;
        drawarrow K--K shifted (0.5u,u);
        }
    ]
    \end{minipage}
    \begin{minipage}{0.45\linewidth}
    \Solide[%
        Nom=pyramide, 
        Phi=0,
        Theta=22,
        DecalageSommet={(cosd(270),1.7*sind(270),0)},
        HauteurPyramide=1.2,
        SommetsPyramide=6,
        Sommets=false,
        Reguliere,
        Traces={%
        pair K,L;
        K=u*(1,-1);
        label.bot(btex \Aire[cm]{7} etex,K); 
        L=u*(0,-2.5);
        label.bot(btex Pyramide 2 etex,L);                 
        fill polygone(B,C,D,E,F) withcolor LightGray;
        % trace segment(A,PiedHauteur) dashed evenly;
        % trace codeperp(A,PiedHauteur,B,5);
        trace cotationmil(A,PiedHauteur,0mm,20,btex \Lg[cm]{4,8} etex);
        trace segment(A,B);
        trace segment(A,E);
        trace segment(A,F);
        drawarrow K--K shifted (-0.5u,u);
        }
    ]
    \end{minipage}
\end{exercice*}
\begin{corrige}
    %\setcounter{partie}{0} % Pour s'assurer que le compteur de \partie est à zéro dans les corrigés
    %\phantom{rrr}    
    Calculer les volumes des pyramides. 
    
    On rappelle que $\mathcal{V}_{pyramide}=\dfrac{\text{Aire de la base}\times \text{hauteur}}{3}$.

    \Solide[%
        Nom=pyramide,        
        HauteurPyramide=1.2,
        SommetsPyramide=5,
        Sommets=false,
        Phi=60,
        Theta=22,
        Traces={%        
        pair K;
        K=u*(-1,-1);
        label.bot(btex \Aire[cm]{6} etex,K);         
        % trace segment(A,PiedHauteur) dashed evenly;
        fill polygone(B,C,D,E) withcolor LightGray;
        trace codeperp(A,PiedHauteur,B,5);
        trace cotationmil(A,PiedHauteur,0mm,20,btex \Lg[cm]{7.2} etex);
        trace segment(A,E);    
        trace segment(A,C) dashed evenly;
        drawarrow K--K shifted (0.5u,u);
        }
    ]

    \textcolor{red}{$\mathcal{V}_{pyramide}=\dfrac{\num{6}\times \num{7.2}}{3}$}

    \textcolor{red}{$\mathcal{V}_{pyramide}=\Vol[cm]{14.4}$}

    \Solide[%
        Nom=pyramide, 
        Phi=0,
        Theta=22,
        DecalageSommet={(cosd(270),1.7*sind(270),0)},
        HauteurPyramide=1.2,
        SommetsPyramide=6,
        Sommets=false,
        Reguliere,
        Traces={%
        pair K;
        K=u*(1,-1);
        label.bot(btex \Aire[cm]{7} etex,K);         
        fill polygone(B,C,D,E,F) withcolor LightGray;
        % trace segment(A,PiedHauteur) dashed evenly;
        % trace codeperp(A,PiedHauteur,B,5);
        trace cotationmil(A,PiedHauteur,0mm,20,btex \Lg[cm]{4,8} etex);
        trace segment(A,B);
        trace segment(A,E);
        trace segment(A,F);
        drawarrow K--K shifted (-0.5u,u);
        }
    ]

    \textcolor{red}{$\mathcal{V}_{pyramide}=\dfrac{\num{7}\times \num{4.8}}{3}$}

    \textcolor{red}{$\mathcal{V}_{pyramide}=\Vol[cm]{11.2}$}
\end{corrige}

