\begin{exercice*}
    On a représenté ci-dessous un cône de révolution de sommet $S$.
    \begin{enumerate}
        \item La figure ci-dessous n'est pas en vraie grandeur
        \begin{center}
            \Solide[%
                Nom=cone,
                HauteurCone=1.7,
                RayonCone=1,                
                ListeSommets={S,O},
                Phi=0,                            
                Traces={%
                Label.top(btex $S$ etex,S);
                Label.bot(btex $O$ etex,O);
                trace segment(S,O) dashed evenly;
                color A,B,D,E,I;
                A-O=rayoncone*(0,-sind(90),0);
                B-O=rayoncone*(0,sind(90),0);
                E-O=rayoncone*(-cosd(30),-sind(30),0);
                D-O=rayoncone*(cosd(30),sind(30),0);
                I-O=hauteurcone*(0,0,0.65);
                Label.rt(btex $B$ etex,B);
                Label.lft(btex $A$ etex,A);
                Label.lrt(btex $E$ etex,E);
                Label.lrt(btex $D$ etex,D);
                Label.ulft(btex $I$ etex,I);
                trace segment(A,B) dashed evenly;
                trace segment(A,I) dashed evenly;
                trace segment(S,D);                
                marque_p:="croix";
                u:=u/2;
                pointe(E);
                u:=u*2;
                trace codeperp(S,O,B,5);
                }
            ]
        \end{center}
        Nommer : 
        \begin{itemize}
            \item le sommet : \pointilles
            \item le centre de la base : \pointilles
            \item un diamètre de la base : \pointilles
            \item la hauteur : \pointilles
            \item trois génératrices : \pointilles
        \end{itemize}
        \item Déterminer la nature du triangle $SAD$.
        \item Déterminer la nature du triangle $SOD$.
        \item Citer toutes les longueurs égales à $OA$.
    \end{enumerate}
\end{exercice*}
\begin{corrige}
    %\setcounter{partie}{0} % Pour s'assurer que le compteur de \partie est à zéro dans les corrigés
    %\phantom{rrr}    
        On a représenté ci-dessous un cône de révolution de sommet $S$.

    \begin{enumerate}
        \item La figure ci-dessous n'est pas en vraie grandeur
        
        \begin{minipage}{0.45\linewidth}
            \hspace*{-10mm}
            \Solide[%
                Nom=cone,
                HauteurCone=1.7,
                RayonCone=1,                
                ListeSommets={S,O},
                Phi=0,                            
                Traces={%
                Label.top(btex $S$ etex,S);
                Label.bot(btex $O$ etex,O);
                trace segment(S,O) dashed evenly;
                color A,B,D,E,I;
                A-O=rayoncone*(0,-sind(90),0);
                B-O=rayoncone*(0,sind(90),0);
                E-O=rayoncone*(-cosd(30),-sind(30),0);
                D-O=rayoncone*(cosd(30),sind(30),0);
                I-O=hauteurcone*(0,0,0.65);
                Label.rt(btex $B$ etex,B);
                Label.lft(btex $A$ etex,A);
                Label.lrt(btex $E$ etex,E);
                Label.lrt(btex $D$ etex,D);
                Label.ulft(btex $I$ etex,I);
                trace segment(A,B) dashed evenly;
                trace segment(A,I) dashed evenly;
                trace segment(S,D);                
                marque_p:="croix";
                u:=u/2;
                pointe(E);
                u:=u*2;
                trace codeperp(S,O,B,5);
                }
            ]
        \end{minipage}
        \hfill
        \begin{minipage}{0.5\linewidth}
            Nommer : 
            \begin{itemize}
                \item le sommet : \textcolor{red}{$S$}
                \item le centre de la base : \textcolor{red}{$O$}
                \item un diamètre de la base : \textcolor{red}{$[AB]$}
                \item la hauteur : \textcolor{red}{$[SO]$}
            \end{itemize}
        \end{minipage}
        \begin{itemize}
            \item trois génératrices : \textcolor{red}{$[SD]$};\textcolor{red}{$[SB]$};\textcolor{red}{$[SE]$}.
        \end{itemize}
        \item Déterminer la nature du triangle $SAD$.
        \textcolor{red}{$SA=SD$, donc $SAD$ est un triangle isocèle en $S$.}
        \item Déterminer la nature du triangle $SOD$.
        \textcolor{red}{$\widehat{SOD}$ est un angle droit, donc $SOD$ est un triangle rectangle en $O$.}
        \item Citer toutes les longueurs égales à $OA$.
        \textcolor{red}{$OA$ est le rayon de la base, donc tous les points du cercle de base sont à la même distance de $O$ que $A$, d'où $OB=OD=OE=OA$.}
    \end{enumerate}
\end{corrige}

