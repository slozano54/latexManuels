\begin{exercice*}
    On considère une pyramide dont la base a $n$ côtés.

    Exprimer en fonction de $n$ :
    \begin{itemize}
        \item son nombre de faces, \textbf{f} : \pointilles
        \item son nombre de sommets, \textbf{s} : \pointilles
        \item son nombre d' arêtes, \textbf{a} : \pointilles
    \end{itemize}
    Calculer ensuite , \textbf{s+f-a} : \pointilles
\end{exercice*}
\begin{corrige}
    %\setcounter{partie}{0} % Pour s'assurer que le compteur de \partie est à zéro dans les corrigés
    %\phantom{rrr}    
    On considère une pyramide dont la base a $n$ côtés.

    Exprimer en fonction de $n$ :
    \begin{itemize}
        \item son nombre de faces, \textbf{f} :    \textcolor{red}{$n+1$}
        \item son nombre de sommets, \textbf{s} :  \textcolor{red}{$n+1$}
        \item son nombre d' arêtes, \textbf{a} :   \textcolor{red}{$2n$}
    \end{itemize}
    Calculer ensuite , \textbf{s+f-a} : \textcolor{red}{$n+1+n+1-2n=2$}
\end{corrige}

