\begin{exercice*}
    Compléter les représentations en perspective suivantes sachant que ce sont des pyramides de sommet $S$ à base triangulaire.

    \begin{Geometrie}[CoinHD={u*(10,5)}]
        pair S[],A[],B[],C[];
        pair legende[];
        trace grille(0.5) withcolor Grey;
        %
        A1=u*(0.5,1);
        B1-A1=u*(2.5,0);
        C1-A1=u*(1,1.5);
        S1-A1=u*(2.5,2.5);
        legende1=iso(A1,B1) shifted (0,-0.75u);
        label(TEX("Perspective 1"),legende1);
        trace C1--A1--B1--S1 withpen pencircle scaled 1.2bp;
        trace segment(B1,C1) withpen pencircle scaled 1.2bp dashed evenly;
        marque_p:="croix";
        pointe(S1);
        label.top(TEX("S"),S1);
        %
        A2-A1=u*(3.5,1);
        B2-A2=u*(2.5,2);
        C2-A2=u*(0.5,2);
        S2-A2=u*(1.5,-0.5);
        legende2-legende1=(3.5u,0);
        label(TEX("Perspective 2"),legende2);
        trace A2--B2--C2 withpen pencircle scaled 1.2bp;
        trace B2--S2 withpen pencircle scaled 1.2bp;
        marque_p:="croix";        
        pointe(S2);
        label.bot(TEX("S"),S2);
        %
        A3-A1=u*(7,0);
        B3-A3=u*(2,0.5);
        C3-A3=u*(-0.5,2);
        S3-A3=u*(2,2.5);
        legende3-legende2=(3.5u,0);
        label(TEX("Perspective 3"),legende3);
        trace A3--B3--C3 withpen pencircle scaled 1.2bp;        
        marque_p:="croix";        
        pointe(S3);
        label.top(TEX("S"),S3);
    \end{Geometrie}
\end{exercice*}
\begin{corrige}
    %\setcounter{partie}{0} % Pour s'assurer que le compteur de \partie est à zéro dans les corrigés
    %\phantom{rrr}    
    Compléter les représentations en perspective suivantes sachant que ce sont des pyramides de sommet $S$ à base triangulaire.

    \hspace*{-7mm}
    \scalebox{0.8}{
        \begin{Geometrie}[CoinHD={u*(10,5)}]
            pair S[],A[],B[],C[];
            pair legende[];
            trace grille(0.5) withcolor Grey;
            %
            A1=u*(0.5,1);
            B1-A1=u*(2.5,0);
            C1-A1=u*(1,1.5);
            S1-A1=u*(2.5,2.5);
            legende1=iso(A1,B1) shifted (0,-0.75u);
            label(TEX("Perspective 1"),legende1);
            trace C1--A1--B1--S1 withpen pencircle scaled 1.2bp;
            trace segment(B1,C1) withpen pencircle scaled 1.2bp dashed evenly;
            marque_p:="croix";
            pointe(S1);
            label.top(TEX("S"),S1);
            %corr
            drawoptions(withcolor red);
            trace S1--A1 withpen pencircle scaled 1.2bp;
            trace S1--C1 withpen pencircle scaled 1.2bp;
            drawoptions(withcolor black);
            %
            A2-A1=u*(3.5,1);
            B2-A2=u*(2.5,2);
            C2-A2=u*(0.5,2);
            S2-A2=u*(1.5,-0.5);
            legende2-legende1=(3.5u,0);
            label(TEX("Perspective 2"),legende2);
            trace A2--B2--C2 withpen pencircle scaled 1.2bp;
            trace B2--S2 withpen pencircle scaled 1.2bp;
            marque_p:="croix";        
            pointe(S2);
            label.bot(TEX("S"),S2);
            %corr
            drawoptions(withcolor red);
            trace S2--A2 withpen pencircle scaled 1.2bp;
            trace S2--C2 withpen pencircle scaled 1.2bp dashed evenly;
            trace C2--A2 withpen pencircle scaled 1.2bp;
            drawoptions(withcolor black);
            %
            A3-A1=u*(7,0);
            B3-A3=u*(2,0.5);
            C3-A3=u*(-0.5,2);
            S3-A3=u*(2,2.5);
            legende3-legende2=(3.5u,0);
            label(TEX("Perspective 3"),legende3);
            trace A3--B3--C3 withpen pencircle scaled 1.2bp;        
            marque_p:="croix";        
            pointe(S3);
            label.top(TEX("S"),S3);
            %corr
            drawoptions(withcolor red);
            trace S3--A3 withpen pencircle scaled 1.2bp dashed evenly;
            trace S3--B3 withpen pencircle scaled 1.2bp;
            trace S3--C3 withpen pencircle scaled 1.2bp;
            trace C3--A3 withpen pencircle scaled 1.2bp;
            drawoptions(withcolor black);
        \end{Geometrie}
    }
\end{corrige}

