\begin{exercice*}
    $SABCD$ est une pyramide à base rectangulaire dont les faces latérales sont des triangles isocèles.
    \begin{enumerate}
        \item La figure ci-dessous n'est pas en vraie grandeur        
        \begin{center}
            \Solide[%
                Nom=pyramide,
                Reguliere,
                HauteurPyramide=1.7,
                SommetsPyramide=5,
                Sommets=false,
                ListeSommets={S,A,B,C,D},            
                Phi=45,
                Traces={%
                Label.top(btex $S$ etex,S);
                Label.llft(btex $A$ etex,A);
                Label.ulft(btex $D$ etex,B);
                Label.rt(btex $C$ etex,C);
                Label.lrt(btex $B$ etex,D);
                Label.bot(btex $H$ etex,PiedHauteur);            
                trace segment(B,D) dashed evenly;
                trace segment(A,C) dashed evenly;
                trace segment(S,PiedHauteur) dashed evenly;
                trace codeperp(S,PiedHauteur,C,8);
                trace appelation(A,D,-3mm,btex \Lg[m]{8} etex);
                trace appelation(C,D,3mm,btex \Lg[m]{6} etex);
                trace appelation(S,C,3mm,btex \Lg[m]{13} etex);
                trace appelation(PiedHauteur,S,3mm,btex \Lg[m]{12} etex);
                }
            ]
        \end{center}
        Nommer : 
        \begin{itemize}
            \item le sommet : \pointilles
            \item la hauteur : \pointilles
            \item la base : \pointilles
            \item les arêtes latérales : \pointilles
            \item les faces latérales : \pointilles
        \end{itemize}
        \item En déduire les longueurs suivantes :
        
        $AD=\pointilles[2cm]$;\hfill$CD=\pointilles[2cm]$;
        
        $SH=\pointilles[2cm]$;\hfill$SA=\pointilles[2cm]$;
        
        $SB=\pointilles[2cm]$;\hfill$SD=\pointilles[2cm]$;
    \end{enumerate}
\end{exercice*}
\begin{corrige}
    %\setcounter{partie}{0} % Pour s'assurer que le compteur de \partie est à zéro dans les corrigés
    %\phantom{rrr}    
    $SABCD$ est une pyramide à base rectangulaire dont les faces latérales sont des triangles isocèles.
    \begin{enumerate}
        \item La figure ci-dessous n'est pas en vraie grandeur
        
        \begin{minipage}{0.4\linewidth}
            \Solide[%
                Nom=pyramide,
                Reguliere,
                HauteurPyramide=1.7,
                SommetsPyramide=5,
                Sommets=false,
                ListeSommets={S,A,B,C,D},            
                Phi=45,
                Traces={%
                Label.top(btex $S$ etex,S);
                Label.llft(btex $A$ etex,A);
                Label.ulft(btex $D$ etex,B);
                Label.rt(btex $C$ etex,C);
                Label.lrt(btex $B$ etex,D);            
                Label.bot(btex $H$ etex,PiedHauteur);
                trace segment(B,D) dashed evenly;
                trace segment(A,C) dashed evenly;
                trace segment(S,PiedHauteur) dashed evenly;
                trace codeperp(S,PiedHauteur,C,8);
                trace appelation(A,D,-3mm,btex \Lg[m]{8} etex);
                trace appelation(C,D,3mm,btex \Lg[m]{6} etex);
                trace appelation(S,C,3mm,btex \Lg[m]{13} etex);
                trace appelation(PiedHauteur,S,3mm,btex \Lg[m]{12} etex);
                }
            ]
        \end{minipage}
        \hfill
        \begin{minipage}{0.5\linewidth}
            Nommer : 
            \begin{itemize}
                \item le sommet :   \textcolor{red}{$S$}
                \item la hauteur :  \textcolor{red}{$[SH]$}
                \item la base :     \textcolor{red}{$ABCD$}
            \end{itemize}
        \end{minipage}
        \begin{itemize}
            \item les arêtes latérales :    \textcolor{red}{$[SA]$};\textcolor{red}{$[SB]$};\textcolor{red}{$[SC]$};\textcolor{red}{$[SD]$}.
            \item les faces latérales :     \textcolor{red}{$SAB$};\textcolor{red}{$SCB$};\textcolor{red}{$SCD$};\textcolor{red}{$SDA$}.
        \end{itemize}
        \item En déduire les longueurs suivantes :
        
        $AD=\textcolor{red}{\Lg[m]{6}}$;\hfill$CD=\textcolor{red}{\Lg[m]{8}}$;\hfill
        
        $SH=\textcolor{red}{\Lg[m]{12}}$;\hfill$SA=\textcolor{red}{\Lg[m]{13}}$;\hfill
        
        $SB=\textcolor{red}{\Lg[m]{13}}$;\hfill$SD=\textcolor{red}{\Lg[m]{13}}$;\hfill
    \end{enumerate}
\end{corrige}

