\begin{exercice*}
    Pour chacune de ces pyramides, colorier la base en gris et repasser la hauteur en rouge, puis calculer le volume.

    \begin{multicols}{2}
        \begin{enumerate}
            \item \phantom{rrr}
            
            \Solide[%
                Nom=pyramide,
                Reguliere,
                HauteurPyramide=1.6,
                SommetsPyramide=5,
                Sommets=false,
                Phi=70,
                Traces={%
                pair K;
                K=u*(0,-1);
                label.bot(btex Pyramide à base carrée etex,K); 
                trace appelation (A,PiedHauteur,-2mm,btex \Lg[cm]{7} etex);
                trace appelation (B,E,-3mm,btex \Lg[cm]{3.6} etex);
                % fill polygone(B,C,D,E) withcolor LightGray;
                trace segment(B,D) dashed evenly;
                trace segment(C,E) dashed evenly;
                trace segment(A,PiedHauteur) dashed evenly;
                trace codeperp(A,PiedHauteur,B,5);
                trace segment(A,B);
                trace segment(A,C) dashed evenly;
                trace segment(A,D);
                trace segment(A,E);
                marque_s:=marque_s*0.3;
                trace Codelongueur(B,C,C,D,D,E,E,B,2);
                % drawoptions(withcolor red withpen pencircle scaled 1.2bp);
                % trace segment(A,PiedHauteur);            
                }
            ]
            \item \phantom{rrr}
            
            \Solide[%            
                Nom=pyramide, 
                Phi=0,
                Theta=22,
                DecalageSommet={(cosd(270),sind(270),0)},
                HauteurPyramide=1.2,
                SommetsPyramide=5,
                Sommets=false,
                Reguliere,
                Traces={%
                pair K;
                K=u*(0,-1);
                label.bot(btex Pyramide à base rectangulaire etex,K); 
                trace appelation (A,PiedHauteur,-2mm,btex \Lg[cm]{37} etex);
                trace appelation (B,E,-3mm,btex \Lg[cm]{24} etex);
                trace appelation (PiedHauteur,B,-3mm,btex \Lg[cm]{23} etex);
                % fill polygone(B,C,D,E) withcolor LightGray;
                trace segment(A,PiedHauteur) dashed evenly;
                trace codeperp(A,PiedHauteur,B,5);
                trace segment(A,B);
                trace segment(A,C) dashed evenly;
                trace segment(A,D);
                trace segment(A,E);
                marque_s:=marque_s*0.3;
                trace Codelongueur(B,C,D,E,2);
                trace Codelongueur(C,D,E,B,3);
                % drawoptions(withcolor red withpen pencircle scaled 1.2bp);
                % trace segment(A,PiedHauteur);            
                }
            ]
        \end{enumerate}
    \end{multicols}
\end{exercice*}
\begin{corrige}
    %\setcounter{partie}{0} % Pour s'assurer que le compteur de \partie est à zéro dans les corrigés
    %\phantom{rrr}    
    Pour chacune de ces pyramides, colorier la base en gris et repasser la hauteur en rouge, puis calculer le volume.

    \begin{enumerate}
        \item \phantom{rrr}
        
        \Solide[%
            Nom=pyramide,
            Reguliere,
            HauteurPyramide=1.6,
            SommetsPyramide=5,
            Sommets=false,
            Phi=70,
            Traces={%
            pair K;
            K=u*(0,-1);
            label.bot(btex Pyramide à base carrée etex,K); 
            trace appelation (A,PiedHauteur,-2mm,btex \Lg[cm]{7} etex);
            trace appelation (B,E,-3mm,btex \Lg[cm]{3.6} etex);
            fill polygone(B,C,D,E) withcolor LightGray;
            trace segment(B,D) dashed evenly;
            trace segment(C,E) dashed evenly;
            trace segment(A,PiedHauteur) dashed evenly;
            trace codeperp(A,PiedHauteur,B,5);
            trace segment(A,B);
            trace segment(A,C) dashed evenly;
            trace segment(A,D);
            trace segment(A,E);
            marque_s:=marque_s*0.3;
            trace Codelongueur(B,C,C,D,D,E,E,B,2);
            drawoptions(withcolor red withpen pencircle scaled 1.2bp);
            trace segment(A,PiedHauteur);            
            }
        ]

        \textcolor{red}{$\mathcal{V}_{pyramide}=\dfrac{\num{3.6}\times \num{3.6}\times \num{7}}{3}$}

        \textcolor{red}{$\mathcal{V}_{pyramide}=\Vol[cm]{30.24}$}
        \item \phantom{rrr}
        
        \Solide[%            
            Nom=pyramide, 
            Phi=0,
            Theta=22,
            DecalageSommet={(cosd(270),sind(270),0)},
            HauteurPyramide=1.2,
            SommetsPyramide=5,
            Sommets=false,
            Reguliere,
            Traces={%
            pair K;
            K=u*(0,-1);
            label.bot(btex Pyramide à base rectangulaire etex,K); 
            trace appelation (A,PiedHauteur,-2mm,btex \Lg[cm]{37} etex);
            trace appelation (B,E,-3mm,btex \Lg[cm]{24} etex);
            trace appelation (PiedHauteur,B,-3mm,btex \Lg[cm]{23} etex);
            fill polygone(B,C,D,E) withcolor LightGray;
            trace segment(A,PiedHauteur) dashed evenly;
            trace codeperp(A,PiedHauteur,B,5);
            trace segment(A,B);
            trace segment(A,C) dashed evenly;
            trace segment(A,D);
            trace segment(A,E);
            marque_s:=marque_s*0.3;
            trace Codelongueur(B,C,D,E,2);
            trace Codelongueur(C,D,E,B,3);
            drawoptions(withcolor red withpen pencircle scaled 1.2bp);
            trace segment(A,PiedHauteur);            
            }
        ]

        \textcolor{red}{$\mathcal{V}_{pyramide}=\dfrac{\num{23}\times \num{24}\times \num{37}}{3}$}
        
        \textcolor{red}{$\mathcal{V}_{pyramide}=\Vol[cm]{6808}$}
    \end{enumerate}
\end{corrige}

