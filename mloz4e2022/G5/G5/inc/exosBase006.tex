\begin{exercice*}
    Compléter les représentations en perspective suivantes.
    \begin{enumerate}
        \item Perspective d'une pyramide à base triangulaire.
        \begin{center}
            \Solide[%
                Nom=pyramide,
                Reguliere,
                HauteurPyramide=1.2,
                SommetsPyramide=5,
                Sommets=false,
                Aretes=false,
                Phi=70,
                Traces={% 
                trace polygone(A,B,E,D);
                drawoptions(withcolor white);
                pointe(C);
                }
            ]
        \end{center}
        \item Perspective d'une pyramide à base rectangulaire.
        \begin{center}
            \Solide[%
                Nom=pyramide,
                Reguliere,
                HauteurPyramide=1.2,
                SommetsPyramide=5,
                Sommets=false,
                Aretes=false,
                Phi=70,
                Traces={% 
                trace polygone(A,B,E,D);
                drawoptions(withcolor white);
                pointe(C);
                }
            ]
        \end{center}
    \end{enumerate}
\end{exercice*}
\begin{corrige}
    %\setcounter{partie}{0} % Pour s'assurer que le compteur de \partie est à zéro dans les corrigés
    %\phantom{rrr}    
    Compléter les représentations en perspective suivantes.

    \begin{enumerate}
        \item Perspective d'une pyramide à base triangulaire.
        
        \Solide[%
            Nom=pyramide,
            Reguliere,
            HauteurPyramide=1.2,
            SommetsPyramide=5,
            Sommets=false,
            Aretes=false,
            Phi=70,
            Traces={% 
            trace polygone(A,B,E,D);
            drawoptions(withcolor white);
            pointe(C);
            drawoptions(withcolor red);
            trace segment(B,D) dashed evenly;
            trace segment(A,E);
            }
        ]
    \end{enumerate}
    \Coupe
    \begin{enumerate}
        \setcounter{enumi}{1}
        \item Perspective d'une pyramide à base rectangulaire.
        
        \Solide[%
            Nom=pyramide,
            Reguliere,
            HauteurPyramide=1.2,
            SommetsPyramide=5,
            Sommets=false,
            Aretes=false,
            Phi=70,
            Traces={% 
            trace polygone(A,B,E,D);
            drawoptions(withcolor white);
            pointe(C);
            drawoptions(withcolor red);
            trace segment(C,D) dashed evenly;
            trace segment(C,B) dashed evenly;
            trace segment(A,C) dashed evenly;
            trace segment(A,E);
            }
        ]
    \end{enumerate}
\end{corrige}

