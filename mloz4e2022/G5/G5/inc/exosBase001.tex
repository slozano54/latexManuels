\begin{exercice*}
    Pour chacune des pyramides ci-dessous,  colorier :
    \begin{itemize}
        \item en bleu, le sommet ;
        \item en vert, les arêtes latérales ;
        \item en rouge, la hauteur ; 
        \item en jaune, le polygone représentant sa base.
    \end{itemize}

    \Solide[%
        Nom=pyramide,
        Reguliere,
        HauteurPyramide=1.2,
        SommetsPyramide=5,
        Sommets=false,
        Phi=70,
        Traces={%
        pair K;
        K=u*(0,-1);
        label.bot(btex $P_1$ etex,K); 
        trace segment(B,D) dashed evenly;
        trace segment(C,E) dashed evenly;
        trace segment(A,PiedHauteur) dashed evenly;
        trace codeperp(A,PiedHauteur,B,5);
        }
    ]
    \Solide[%
        Nom=pyramide,        
        HauteurPyramide=1.2,
        SommetsPyramide=6,
        Sommets=false,
        Phi=60,
        Theta=22,
        Traces={%        
        pair K;
        K=u*(0,-1);
        label.bot(btex $P_2$ etex,K);         
        trace segment(A,PiedHauteur) dashed evenly;
        trace codeperp(A,PiedHauteur,B,5);
        }
    ]
    \Solide[%
        Nom=pyramide, 
        Phi=0,
        Theta=22,
        DecalageSommet={(cosd(270),sind(270),0)},
        HauteurPyramide=1.2,
        SommetsPyramide=5,
        Sommets=false,
        Reguliere,
        Traces={%
        pair K;
        K=u*(0,-1);
        label.bot(btex $P_3$ etex,K);         
        % trace segment(A,PiedHauteur) dashed evenly;
        trace codeperp(A,PiedHauteur,B,5);
        }
    ]
    \Solide[%
        Nom=pyramide, 
        Phi=70,
        Theta=22,
        Anglex=180,
        DecalageSommet={(cosd(270),sind(270),0)},
        HauteurPyramide=1.2,
        SommetsPyramide=4,
        Sommets=false,
        Reguliere,
        Traces={%
        pair K;
        K=u*(-0.75,-2);
        label.bot(btex $P_4$ etex,K);         
        % trace segment(A,PiedHauteur) dashed evenly;
        trace codeperp(C,D,A,8);
        }
    ]

    Compléter alors le tableau suivant :

    \begin{tabular}{|>{\arraybackslash}m{0.5\linewidth}|*{4}{c|}}
        \hline
        \rowcolor{gray!20}Nom&$P_1$&$P_2$&$P_3$&$P_4$\\\hline
        \cellcolor{gray!20}Nombre de côté de la base&&&&\\\hline
        \cellcolor{gray!20}Nombre de faces : \textbf{f}&&&&\\\hline
        \cellcolor{gray!20}Nombre d'arêtes : \textbf{a}&&&&\\\hline
        \cellcolor{gray!20}Nombre de sommets : \textbf{s}&&&&\\\hline        
        \cellcolor{gray!20}\textbf{s+f-a}&&&&\\\hline        
    \end{tabular}
    
    Faire le calcul de la dernière ligne pour un pavé droit puis faire une remarque.
\end{exercice*}
\begin{corrige}
    %\setcounter{partie}{0} % Pour s'assurer que le compteur de \partie est à zéro dans les corrigés
    %\phantom{rrr}    
    Pour chacune des pyramides ci-dessous,  colorier :
    \begin{itemize}
        \item en bleu, le sommet ;
        \item en vert, les arêtes latérales ;
        \item en rouge, la hauteur ; 
        \item en jaune, le polygone représentant sa base.
    \end{itemize}
    \begin{changemargin}{-7mm}{0mm}
    \Solide[%
        Nom=pyramide,
        Reguliere,
        HauteurPyramide=1.2,
        SommetsPyramide=5,
        Sommets=false,
        Phi=70,
        Traces={%
        pair K;
        K=u*(0,-1);
        label.bot(btex $P_1$ etex,K); 
        fill polygone(B,C,D,E) withcolor jaune;
        trace segment(B,D) dashed evenly;
        trace segment(C,E) dashed evenly;
        trace segment(A,PiedHauteur) dashed evenly;
        trace codeperp(A,PiedHauteur,B,5);
        drawoptions(withcolor blue withpen pencircle scaled 1.2bp);
        pointe(A);        
        drawoptions(withcolor DarkGreen withpen pencircle scaled 1.2bp);
        trace segment(A,B);
        trace segment(A,C) dashed evenly;
        trace segment(A,D);
        trace segment(A,E);
        drawoptions(withcolor red withpen pencircle scaled 1.2bp);
        trace segment(A,PiedHauteur);                
        }
    ]
    \Solide[%
        Nom=pyramide,        
        HauteurPyramide=1.2,
        SommetsPyramide=6,
        Sommets=false,
        Phi=60,
        Theta=22,
        Traces={%        
        pair K;
        K=u*(0,-1);
        label.bot(btex $P_2$ etex,K);         
        fill polygone(B,C,D,E,F) withcolor jaune;
        trace segment(A,PiedHauteur) dashed evenly;
        trace codeperp(A,PiedHauteur,B,5);
        drawoptions(withcolor blue withpen pencircle scaled 1.2bp);
        pointe(A);        
        drawoptions(withcolor DarkGreen withpen pencircle scaled 1.2bp);
        trace segment(A,B);
        trace segment(A,C) dashed evenly;
        trace segment(A,D);
        trace segment(A,E);
        trace segment(A,F);
        drawoptions(withcolor red withpen pencircle scaled 1.2bp);
        trace segment(A,PiedHauteur);
        }
    ]
    \Solide[%
        Nom=pyramide, 
        Phi=0,
        Theta=22,
        DecalageSommet={(cosd(270),sind(270),0)},
        HauteurPyramide=1.2,
        SommetsPyramide=5,
        Sommets=false,
        Reguliere,
        Traces={%
        pair K;
        K=u*(0,-1);
        label.bot(btex $P_3$ etex,K);            
        fill polygone(B,C,D,E) withcolor jaune;     
        trace codeperp(A,PiedHauteur,B,5);
        drawoptions(withcolor blue withpen pencircle scaled 1.2bp);
        pointe(A);        
        drawoptions(withcolor DarkGreen withpen pencircle scaled 1.2bp);
        trace segment(A,B);
        trace segment(A,C) dashed evenly;
        trace segment(A,D);
        trace segment(A,E);
        drawoptions(withcolor red withpen pencircle scaled 1.2bp);
        trace segment(A,PiedHauteur);
        }
    ]
    \Solide[%
        Nom=pyramide, 
        Phi=70,
        Theta=22,
        Anglex=180,
        DecalageSommet={(cosd(270),sind(270),0)},
        HauteurPyramide=1.2,
        SommetsPyramide=4,
        Sommets=true,
        Reguliere,
        Traces={%
        pair K;
        K=u*(-0.75,-2);
        label.bot(btex $P_4$ etex,K);                         
        drawoptions(withcolor blue withpen pencircle scaled 1.2bp);
        pointe(A);        
        fill polygone(B,C,D) withcolor jaune;
        drawoptions(withcolor black);
        trace codeperp(C,D,A,8);
        drawoptions(withcolor DarkGreen withpen pencircle scaled 1.2bp);
        trace segment(A,B);
        trace segment(A,C);
        trace segment(A,D);        
        drawoptions(withcolor red withpen pencircle scaled 1.2bp);
        trace segment(A,D) dashed evenly;        
        drawoptions(withcolor black);
        trace polygone(B,C,D);
        }
    ]
    \end{changemargin}

    Compléter alors le tableau suivant :

    \begin{tabular}{|>{\arraybackslash}m{0.5\linewidth}|*{4}{c|}}
        \hline
        \rowcolor{gray!20}Nom&$P_1$&$P_2$&$P_3$&$P_4$\\\hline
        \cellcolor{gray!20}Nombre de côté de la base        &\textcolor{red}{4}&\textcolor{red}{5} &\textcolor{red}{4}&\textcolor{red}{3}\\\hline
        \cellcolor{gray!20}Nombre de faces : \textbf{f}     &\textcolor{red}{5}&\textcolor{red}{6} &\textcolor{red}{5}&\textcolor{red}{4}\\\hline
        \cellcolor{gray!20}Nombre d'arêtes : \textbf{a}     &\textcolor{red}{8}&\textcolor{red}{10}&\textcolor{red}{8}&\textcolor{red}{6}\\\hline
        \cellcolor{gray!20}Nombre de sommets : \textbf{s}   &\textcolor{red}{5}&\textcolor{red}{6} &\textcolor{red}{5}&\textcolor{red}{4}\\\hline        
        \cellcolor{gray!20}\textbf{s+f-a}                   &\textcolor{red}{2}&\textcolor{red}{2} &\textcolor{red}{2}&\textcolor{red}{2}\\\hline        
    \end{tabular}
    
    \smallskip
    Faire le calcul de la dernière ligne pour un pavé droit puis faire une remarque.
    \textcolor{red}{Pour un pavé droit, il y a 8 sommets, 6 faces et 12 arêtes donc s+f-a vaut également 2.}
\end{corrige}

