\begin{exercice*}
    Parmi les figures suivantes, barrer celles qui ne sont pas des patrons, puis indiquer les noms des solides des patrons restants.

    Les noms sont à piocher dans cette liste : prisme droit, pyramide, cône de révolution et cylindre de révolution.

    \begin{minipage}{0.45\linewidth}
        \begin{enumerate}
            \item \phantom{rrr}
            
            \begin{Geometrie}[CoinHD={u*(4.5,4.5)}]
                trace feuillet;                
                pair A[],B[];
                u:=0.3*u;
                A0=u*(1,4.5);
                A1-A0=u*(2.5,0);
                A2-A0=u*(4.5,0);
                A3-A0=u*(7.5,0);
                A4-A0=u*(9.5,0);
                A5-A0=u*(12,0);
                A6-A0=u*(4.5,-2);
                A7-A0=u*(6,-4);
                A8-A0=u*(7.5,-2);
                B0-A0=u*(0,5);
                B1-A1=u*(0,5);
                B2-A2=u*(0,5);
                B3-A3=u*(0,5);
                B4-A4=u*(0,5);
                B5-A5=u*(0,5);
                B6-A6=u*(0,9);
                B7-A7=u*(0,13);
                B8-A8=u*(0,9);
                trace polygone(A0,A5,B5,B0);
                trace segment(B1,A1);
                trace segment(B2,A2);
                trace segment(B3,A3);
                trace segment(B4,A4);
                trace chemin(A2,A6,A7,A8,A3);
                trace chemin(B2,B6,B7,B8,B3);
            \end{Geometrie}

            \pointilles
            \item \phantom{rrr}
            
            \begin{Geometrie}[CoinHD={u*(4.5,4.5)}]
                trace feuillet;
                pair A,B,C,D,E,F,G,H;
                u:=0.3*u;
                A=u*(5.5,5.5);
                B-A=u*(3,0);
                C-A=u*(3,3);
                D-A=u*(0,3);
                E-A=u*(3,-5);
                F-A=u*(6,3);
                G-A=u*(3,7);
                H-A=u*(-5,3);
                trace chemin(G,C,D,A,B);
                trace polygone(H,A,E,C,F,G,D);                
            \end{Geometrie}

            \pointilles
            \item \phantom{rrr}
            
            \begin{Geometrie}[CoinHD={u*(4.5,4.5)}]
                trace feuillet;
                numeric pi;
                pi = 3.141592653589793;
                pair A[],B[];
                path cA[];
                u:=0.5*u;
                B0=u*(1,3.5);
                B1-B0=u*(2*pi,0);
                B2-B0=u*(2*pi,2);
                B3-B0=u*(0,2);
                A0-B0=u*(2,-1);
                A1-B0=u*(4,3);
                trace polygone(B0,B1,B2,B3);
                trace cercles(A0,u);
                trace cercles(A1,u);
            \end{Geometrie}

            \pointilles
        \end{enumerate}
    \end{minipage}
    \hfill
    \begin{minipage}{0.45\linewidth}        
        \begin{enumerate}
            \setcounter{enumi}{3}
            \item \phantom{rrr}
            
            \begin{Geometrie}[CoinHD={u*(4.5,4.5)}]
                trace feuillet;
                pair A[],S[];
                u:=0.4*u;
                A0=u*(6,7);                
                A1-A0=u*(2,-2);
                A2=rotation(A0,A1,90);
                A3=rotation(A1,A0,-90);
                trace polygone(A0,A1,A2,A3);                
                S1=rotation(A2,A1,60);
                S0=cercles(A0,4u) intersectionpoint cercles(A1,4u);
                S2=cercles(A3,4u) intersectionpoint subpath (4,6) of cercles(A2,4u);
                S3=cercles(A3,4u) intersectionpoint cercles(A0,4u);
                trace chemin(A0,S0,A1,S1,A2,S2,A3,S3,A0);
            \end{Geometrie}
            
            \pointilles
            \item \phantom{rrr}
            
            \begin{Geometrie}[CoinHD={u*(4.5,4.5)}]
                trace feuillet;
                pair O[],A[];
                u:=0.5*u;
                O0=u*(6.5,4.5);
                O1-O0=u*(-4,0);
                trace cercles(O1,u);
                path cc;
                cc=cercles(O0,3u);
                A0=pointarc(cc,-120);
                A1=pointarc(cc,-240);
                trace arccercle(A1,A0,O0);
                trace chemin(A0,O0,A1);
            \end{Geometrie}

            \pointilles
            \item \phantom{rrr}
            
            \begin{Geometrie}[CoinHD={u*(4.5,4.5)}]
                trace feuillet;
                pair A[],O;
                u:=0.5*u;
                A0=u*(1,4.5);
                A1-A0=u*(4.5,-2);
                A2-A0=u*(4.5,2);
                O-A0=u*(5.5,0);
                trace cercles(O,u);
                trace polygone(A0,A1,A2);
            \end{Geometrie}

            \pointilles
        \end{enumerate}
    \end{minipage}
\end{exercice*}
\begin{corrige}
    %\setcounter{partie}{0} % Pour s'assurer que le compteur de \partie est à zéro dans les corrigés
    %\phantom{rrr}    
    Parmi les figures suivantes, barrer celles qui ne sont pas des patrons, puis indiquer les noms des solides des patrons restants.

    Les noms sont à piocher dans cette liste : prisme droit, pyramide, cône de révolution et cylindre de révolution.

    \begin{minipage}{0.45\linewidth}
        \begin{enumerate}
            \item \phantom{rrr}
            
            \scalebox{0.8}{
            \begin{Geometrie}[CoinHD={u*(4.5,4.5)}]
                trace feuillet;                
                pair A[],B[];
                u:=0.3*u;
                A0=u*(1,4.5);
                A1-A0=u*(2.5,0);
                A2-A0=u*(4.5,0);
                A3-A0=u*(7.5,0);
                A4-A0=u*(9.5,0);
                A5-A0=u*(12,0);
                A6-A0=u*(4.5,-2);
                A7-A0=u*(6,-4);
                A8-A0=u*(7.5,-2);
                B0-A0=u*(0,5);
                B1-A1=u*(0,5);
                B2-A2=u*(0,5);
                B3-A3=u*(0,5);
                B4-A4=u*(0,5);
                B5-A5=u*(0,5);
                B6-A6=u*(0,9);
                B7-A7=u*(0,13);
                B8-A8=u*(0,9);
                trace polygone(A0,A5,B5,B0);
                trace segment(B1,A1);
                trace segment(B2,A2);
                trace segment(B3,A3);
                trace segment(B4,A4);
                trace chemin(A2,A6,A7,A8,A3);
                trace chemin(B2,B6,B7,B8,B3);
            \end{Geometrie}
            }

            \textcolor{red}{Prisme droit.}
        \end{enumerate}
    \end{minipage}
    \hfill
    \begin{minipage}{0.45\linewidth}        
        \begin{enumerate}
            \setcounter{enumi}{3}
            \item \phantom{rrr}
            
            \scalebox{0.8}{
            \begin{Geometrie}[CoinHD={u*(4.5,4.5)}]
                trace feuillet;
                pair A[],S[];
                u:=0.4*u;
                A0=u*(6,7);                
                A1-A0=u*(2,-2);
                A2=rotation(A0,A1,90);
                A3=rotation(A1,A0,-90);
                trace polygone(A0,A1,A2,A3);                
                S1=rotation(A2,A1,60);
                S0=cercles(A0,4u) intersectionpoint cercles(A1,4u);
                S2=cercles(A3,4u) intersectionpoint subpath (4,6) of cercles(A2,4u);
                S3=cercles(A3,4u) intersectionpoint cercles(A0,4u);
                trace chemin(A0,S0,A1,S1,A2,S2,A3,S3,A0);
                trace segment((0.5u,0.5u),(11u,11u)) withpen pencircle scaled 1.5bp withcolor red;
            \end{Geometrie}
            }

            \phantom{\textcolor{red}{\dots}}
        \end{enumerate}
    \end{minipage}
    \Coupe
    \begin{minipage}{0.45\linewidth}
        \begin{enumerate}
            \setcounter{enumi}{1}
            \item \phantom{rrr}
            
            \scalebox{0.8}{
            \begin{Geometrie}[CoinHD={u*(4.5,4.5)}]
                trace feuillet;
                pair A,B,C,D,E,F,G,H;
                u:=0.3*u;
                A=u*(5.5,5.5);
                B-A=u*(3,0);
                C-A=u*(3,3);
                D-A=u*(0,3);
                E-A=u*(3,-5);
                F-A=u*(6,3);
                G-A=u*(3,7);
                H-A=u*(-5,3);
                trace chemin(G,C,D,A,B);
                trace polygone(H,A,E,C,F,G,D);                
            \end{Geometrie}
            }

            \textcolor{red}{Pyramide}
            \item \phantom{rrr}
            
            \scalebox{0.8}{
            \begin{Geometrie}[CoinHD={u*(4.5,4.5)}]
                trace feuillet;
                numeric pi;
                pi = 3.141592653589793;
                pair A[],B[];
                path cA[];
                u:=0.5*u;
                B0=u*(1,3.5);
                B1-B0=u*(2*pi,0);
                B2-B0=u*(2*pi,2);
                B3-B0=u*(0,2);
                A0-B0=u*(2,-1);
                A1-B0=u*(4,3);
                trace polygone(B0,B1,B2,B3);
                trace cercles(A0,u);
                trace cercles(A1,u);
            \end{Geometrie}
            }

            \textcolor{red}{Cylindre de révolution}
        \end{enumerate}
    \end{minipage}
    \hfill
    \begin{minipage}{0.45\linewidth}        
        \begin{enumerate}
            \setcounter{enumi}{4}
            \item \phantom{rrr}
            
            \scalebox{0.8}{
            \begin{Geometrie}[CoinHD={u*(4.5,4.5)}]
                trace feuillet;
                pair O[],A[];
                u:=0.5*u;
                O0=u*(6.5,4.5);
                O1-O0=u*(-4,0);
                trace cercles(O1,u);
                path cc;
                cc=cercles(O0,3u);
                A0=pointarc(cc,-120);
                A1=pointarc(cc,-240);
                trace arccercle(A1,A0,O0);
                trace chemin(A0,O0,A1);
            \end{Geometrie}
            }

            \textcolor{red}{Cône de révolution}
            \item \phantom{rrr}
            
            \scalebox{0.8}{
            \begin{Geometrie}[CoinHD={u*(4.5,4.5)}]
                trace feuillet;
                pair A[],O;
                u:=0.5*u;
                A0=u*(1,4.5);
                A1-A0=u*(4.5,-2);
                A2-A0=u*(4.5,2);
                O-A0=u*(5.5,0);
                trace cercles(O,u);
                trace polygone(A0,A1,A2);
                trace segment((0.5u,0.5u),(8.5u,8.5u)) withpen pencircle scaled 1.5bp withcolor red;
            \end{Geometrie}
            }

            \phantom{\textcolor{red}{\dots}}
        \end{enumerate}
    \end{minipage}
\end{corrige}

