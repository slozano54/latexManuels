\begin{exercice*}
    $QRSTUVWX$ est un cube de côté \Lg[cm]{2}. On considère la pyramide $WVXSQ$.
    \begin{center}
        % \Solide[ListeSommets={V,T,U,H,M,N,S,R},PointsSection={A,B,C,D}]
    \Solide[%
        ListeSommets={Q,R,S,T,U,V,W,X},
        Traces={
            drawoptions(withpen pencircle scaled 1.5bp withcolor blue);
            trace segment(Q,S) dashed evenly;
            trace chemin(Q,V,X,S,W,Q);
            trace chemin(V,W,X);
        }
    ]
    \end{center}
    \begin{enumerate}
        \item Nommer la base de cette pyramide.
        \item Donner la nature de cette base.
        \item Indiquer la nature des faces latérales de cette pyramide.
        \item Terminer le patron de la pyramide $WVXSQ$ dans le cadre ci-dessous et finir le codage des longueurs.
        \begin{center}
            \begin{Geometrie}[CoinHD={u*(9,8)}]
                trace feuillet;
                pair W[],V,X,Q,S;
                W1=u*(1,2+sqrt(6)+1);
                V-W1=u*(2,0);
                X-W1=u*(2+2*sqrt(2),0);
                Q-W1=u*(2,-2);
                S-X=u*(0,-2);
                W2-X=u*(2,0);
                W3-V=u*(sqrt(2),sqrt(2));
                W4-W3=u*(0,-sqrt(2)-2-sqrt(6));
                trace chemin(W1,X,S,Q,W1);                
                trace segment(V,Q);
                label.lft(btex $W_1$ etex,W1);
                label.ulft(btex $V$ etex,V);
                label.urt(btex $X$ etex,X);
                label.lrt(btex $S$ etex,S);
                label.llft(btex $Q$ etex,Q);                
                trace codeperp(Q,V,X,5);
                trace codeperp(V,X,S,5);
                trace codeperp(X,S,Q,5);
                trace codeperp(S,Q,V,5);
                marque_s:=marque_s/3;
                trace Codelongueur(W1,V,V,Q,S,X,1);
                trace Codelongueur(V,X,Q,S,W1,Q,2);
                % drawoptions(withcolor red);
                % trace chemin(V,W3,X);
                % trace chemin(S,W2,X);
                % trace chemin(Q,W4,S);
                % label.top(btex $W_3$ etex,W3);
                % label.rt(btex $W_2$ etex,W2);
                % label.bot(btex $W_4$ etex,W4);
                % trace Codelongueur(V,W3,W3,X,X,W2,1);
                % trace Codelongueur(Q,W4,W4,S,S,W2,2);
            \end{Geometrie}
        \end{center}
    \end{enumerate}
\end{exercice*}
\begin{corrige}
    %\setcounter{partie}{0} % Pour s'assurer que le compteur de \partie est à zéro dans les corrigés
    %\phantom{rrr}    
    $QRSTUVWX$ est un cube de côté \Lg[cm]{2}. On considère la pyramide $WVXSQ$.
    \begin{center}
        % \Solide[ListeSommets={V,T,U,H,M,N,S,R},PointsSection={A,B,C,D}]
    \Solide[%
        ListeSommets={Q,R,S,T,U,V,W,X},
        Traces={
            drawoptions(withpen pencircle scaled 1.5bp withcolor blue);
            trace segment(Q,S) dashed evenly;
            trace chemin(Q,V,X,S,W,Q);
            trace chemin(V,W,X);
        }
    ]
    \end{center}
    \begin{enumerate}
        \item Nommer la base de cette pyramide. \textcolor{red}{$VXSQ$.}
        \item Donner la nature de cette base. \textcolor{red}{C'est un rectangle.}
        \item Indiquer la nature des faces latérales de cette pyramide. \textcolor{red}{Il y a deux triangles isocèle-rectangles, un équilatéral et un isocèle.}
        \item Terminer le patron de la pyramide $WVXSQ$ dans le cadre ci-dessous et finir le codage des longueurs.
        \scalebox{0.8}{
            \begin{Geometrie}[CoinHD={u*(9,8)}]
                trace feuillet;
                pair W[],V,X,Q,S;
                W1=u*(1,2+sqrt(6)+1);
                V-W1=u*(2,0);
                X-W1=u*(2+2*sqrt(2),0);
                Q-W1=u*(2,-2);
                S-X=u*(0,-2);
                W2-X=u*(2,0);
                W3-V=u*(sqrt(2),sqrt(2));
                W4-W3=u*(0,-sqrt(2)-2-sqrt(6));
                trace chemin(W1,X,S,Q,W1);                
                trace segment(V,Q);
                label.lft(btex $W_1$ etex,W1);
                label.ulft(btex $V$ etex,V);
                label.urt(btex $X$ etex,X);
                label.lrt(btex $S$ etex,S);
                label.llft(btex $Q$ etex,Q);                
                trace codeperp(Q,V,X,5);
                trace codeperp(V,X,S,5);
                trace codeperp(X,S,Q,5);
                trace codeperp(S,Q,V,5);
                marque_s:=marque_s/3;
                trace Codelongueur(W1,V,V,Q,S,X,1);
                trace Codelongueur(V,X,Q,S,W1,Q,2);
                drawoptions(withcolor red);
                trace chemin(V,W3,X);
                trace chemin(S,W2,X);
                trace chemin(Q,W4,S);
                label.top(btex $W_3$ etex,W3);
                label.rt(btex $W_2$ etex,W2);
                label.bot(btex $W_4$ etex,W4);
                trace Codelongueur(V,W3,W3,X,X,W2,1);
                trace Codelongueur(Q,W4,W4,S,S,W2,2);
            \end{Geometrie}
        }
    \end{enumerate}
\end{corrige}

