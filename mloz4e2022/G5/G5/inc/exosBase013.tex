\begin{exercice*}
    Soit un cône de révolution de génératrice \Lg[cm]{2.5} et dont la base a pour rayon \Lg[cm]{1.5}.
    \begin{enumerate}
        \item Faire, à main levée, une représentation en perspective de ce cône de révolution en y indiquant les mesures connues.
        \begin{center}
            \begin{Geometrie}[CoinBG={(-4u,-2u)},CoinHD={(5u,6u)}]
                trace feuillet;
                u:=1cm;
                % z0=(0,0)*u;
                % path c;
                % c=(fullcircle scaled 4cm)yscaled 0.25;draw  (subpath (0,0.5*(length c)) of c)dashed evenly;
                % draw (subpath(0.5*(length c),length c) of c)  ;z6=(0,3.4u);label.top(btex$S$etex,z6);
                % draw (point 0.5*(length c) of c)--z6;draw point 0 of c--z6;label.lft(btex$O$etex,z0);
                % trace cotationmil(z6,(point 1*(length c) of c),3mm,22,btex \textcolor{red}{\Lg[cm]{2.5}} etex);
                % z4=point 0.85*(length c) of c;
                % drawdblarrow z6--z0 dashed evenly;drawdblarrow z0--z4 dashed evenly; draw codeperp(z4,z0,z6,8);                
                % label.rt(btex\textcolor{red}{$h$}etex,0.5[z0,z6]);                                
                % trace appelation(z0,z4,-3mm,btex \textcolor{red}{\Lg[cm]{1.5}} etex);
            \end{Geometrie}
        \end{center}
        \item Calculer la hauteur de ce cône de révolution.
        \item Ce schéma représente un patron de ce cône. Il n'est pas en vraie grandeur.
        
        Compléter avec les longueurs connues.
        \begin{center}
            \begin{Geometrie}[CoinBG={(-4u,-2u)}]
                % trace feuillet;
                pair O[],A[];
                u:=0.45*u;
                O1=u*(1.7,2);
                O0-O1=u*(8,0);            
                path c[];
                c0=cercles(O0,5.5u);
                A0=pointarc(c0,140);
                A1=pointarc(c0,356);
                trace arccercle(A0,A1,O0);
                trace chemin(A0,O0,A1);
                c1=cercles(O1,2.5u);
                trace c1;
                trace cotationmil(O1,pointarc(c1,-10),-20mm,22,btex \pointilles[5mm] \Lg[cm]{} etex);
                trace cotationmil(pointarc(c0,180),O0,0mm,22,btex \pointilles[10mm] \Lg[cm]{} etex);
                marque_a:=marque_a*0.5;
                trace Codeangle(A0,O0,A1,0,btex $\alpha$ ° etex);
                marque_p:="croix";
                pointe(O1);
                % label.top(btex A0 etex,A0);
                % label.top(btex A1 etex,A1);
                % label.top(btex O0 etex,O0);
                % label.top(btex O1 etex,O1);
            \end{Geometrie}

            \smallskip
            \textbf{Pour constuire ce patron, il faut déterminer la valeur de $\alpha$.}
        \end{center}

        \item Calculer le périmètre du cercle de base de ce cône.
        \item Faire une remarque sur les longueurs du cercle de base et de l'arc de cercle.
        \item On admet que la mesure de l'angle $\alpha$ est proportionnelle à la longueur de l'arc.
        
        Compléter ce tableau et déterminer la mesure de $\alpha$.

        \smallskip
        {\renewcommand{\arraystretch}{1.2}
            \begin{tabular}{|>{\columncolor{LightGray}}m{0.3\linewidth}|>{\centering\arraybackslash}m{0.2\linewidth}|>{\centering\arraybackslash}m{0.3\linewidth}|}
                \hline
                &Longueur&Mesure de l'angle\\
                \hline
                Cercle complet&&\ang{360}\\
                \hline            
                Arc de cercle&&$\alpha$\\
                \hline
            \end{tabular}
        }        

        \smallskip
        \item Constuire le patron de ce cône en vraie grandeur.
    \end{enumerate}
\end{exercice*}
\begin{corrige}
    %\setcounter{partie}{0} % Pour s'assurer que le compteur de \partie est à zéro dans les corrigés
    %\phantom{rrr}    
    Soit un cône de révolution de génératrice \Lg[cm]{2.5} et dont la base a pour rayon \Lg[cm]{1.5}.
    \begin{enumerate}
        \item Faire, à main levée, une représentation en perspective de ce cône de révolution en y indiquant les mesures connues.
        \scalebox{0.7}{
            \begin{Geometrie}[CoinBG={(-4u,-2u)},CoinHD={(5u,6u)}]
                trace feuillet;
                u:=1cm;
                z0=(0,0)*u;
                path c;
                c=(fullcircle scaled 4cm)yscaled 0.25;draw  (subpath (0,0.5*(length c)) of c)dashed evenly;
                draw (subpath(0.5*(length c),length c) of c)  ;z6=(0,3.4u);label.top(btex$S$etex,z6);
                draw (point 0.5*(length c) of c)--z6;draw point 0 of c--z6;label.lft(btex$O$etex,z0);
                trace cotationmil(z6,(point 1*(length c) of c),3mm,22,btex \textcolor{red}{\Lg[cm]{2.5}} etex);
                z4=point 0.85*(length c) of c;
                drawdblarrow z6--z0 dashed evenly;drawdblarrow z0--z4 dashed evenly; draw codeperp(z4,z0,z6,8);                
                label.rt(btex\textcolor{red}{$h$}etex,0.5[z0,z6]);                                
                trace appelation(z0,z4,-3mm,btex \textcolor{red}{\Lg[cm]{1.5}} etex);
            \end{Geometrie}
        }
        \item Calculer la hauteur de ce cône de révolution.
        
        \textcolor{red}{Le théorème de Pythagore permet d'écrire :\\$\num{2.5}^2=h^2+\num{1.5}^2$ d'où $h^2=4$ soit $h=\Lg[cm]{2}$.}
        \item Ce schéma représente un patron de ce cône. Il n'est pas en vraie grandeur.
        
        Compléter avec les longueurs connues.
        \begin{Geometrie}[CoinBG={(-4u,-2u)}]
            % trace feuillet;
            pair O[],A[];
            u:=0.45*u;
            O1=u*(1.7,2);
            O0-O1=u*(8,0);            
            path c[];
            c0=cercles(O0,5.5u);
            A0=pointarc(c0,140);
            A1=pointarc(c0,356);
            trace arccercle(A0,A1,O0);
            trace chemin(A0,O0,A1);
            c1=cercles(O1,2.5u);
            trace c1;
            trace cotationmil(O1,pointarc(c1,-10),-20mm,22,btex \textcolor{red}{\Lg[cm]{1.5}} etex);
            trace cotationmil(pointarc(c0,180),O0,0mm,22,btex \textcolor{red}{\Lg[cm]{2.5}} etex);
            marque_a:=marque_a*0.5;
            trace Codeangle(A0,O0,A1,0,btex $\alpha$ ° etex);
            marque_p:="croix";
            pointe(O1);        
        \end{Geometrie}

        \smallskip
        \textbf{Pour constuire ce patron, il faut déterminer la valeur de $\alpha$.}

        \item Calculer le périmètre du cercle de base de ce cône.
        
        \textcolor{red}{$\mathcal{P}=2\times\pi\times R$, or $R=\Lg[cm]{1.5}$ d'où $\mathcal{P}=3\pi$}
        \item Faire une remarque sur les longueurs du cercle de base et de l'arc de cercle.
        
        \textcolor{red}{Ces deux longueurs sont égales.}
    \end{enumerate}
    \Coupe
    \begin{enumerate}
        \setcounter{enumi}{5}
        \item On admet que la mesure de l'angle $\alpha$ est proportionnelle à la longueur de l'arc.
        
        Compléter ce tableau et déterminer la mesure de $\alpha$.

        {\renewcommand{\arraystretch}{1.2}
            \begin{tabular}{|>{\columncolor{LightGray}}m{0.3\linewidth}|>{\centering\arraybackslash}m{0.2\linewidth}|>{\centering\arraybackslash}m{0.3\linewidth}|}
                \hline
                &Longueur&Mesure de l'angle\\
                \hline
                Cercle complet&\textcolor{red}{$5\pi$}&\ang{360}\\
                \hline            
                Arc de cercle&\textcolor{red}{$3\pi$}&$\alpha$\\
                \hline
            \end{tabular}
        }        

        \smallskip
        \textcolor{red}{d'où $\alpha=\ang{360}\times 3\pi \div 5\pi = \ang{216}$, $\alpha$ mesure donc \ang{216}.}
        \item Constuire le patron de ce cône en vraie grandeur.
        
        \textcolor{red}{\dots}
    \end{enumerate}
\end{corrige}

