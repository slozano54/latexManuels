\begin{exercice*}
    $ROSE$ est une pyramide telle que $RO=\Lg[cm]{3}$ ; $OS=\Lg[cm]{4}$ et $SE=\Lg[cm]{2}$.
    \begin{enumerate}
        \item Reporter les longueurs connues sur cette réprésentation en perspective cavalière.
        
        \begin{Geometrie}
            pair R,O,S,E;
            O=u*(1,2);
            R-O=u*(0,3);
            S-O=u*(4,0);
            E-O=u*(3,-1);            
            label.lft (btex O etex,O);
            label.ulft(btex R etex,R);
            label.urt (btex S etex,S);
            label.rt  (btex E etex,E);
            trace R--S--E--O--cycle;
            trace R--E;
            trace O--S dashed evenly;
            trace codeperp(R,O,S,5); draw codeperp(O,S,E,5);
        \end{Geometrie}
        \item Sur ce patron qui n'est pas en vraie grandeur :
        \begin{itemize}
            \item écrire le nom des sommets de chaque triangle ;
            \item coder les segments de même longueur ;
            \item indiquer les longueurs connues.
        \end{itemize}

        \vspace*{-20mm}
        \hspace*{-7mm}
        \rotatebox{20}{
        \scalebox{0.7}{
        \begin{Geometrie}[CoinBG={u*(-1,-2)},CoinHD={u*(10,5.5)}]
            % trace feuillet;
            pair R[],O,S,E;
            O=u*(0.5,2);            
            S-O=u*(4,0);
            E-O=u*(4,-2);
            R0-O=u*(0,3);
            R1-S=u*(5,0);
            R2=cercles(O,3u) intersectionpoint subpath (4,6) of cercles(E,sqrt(29)*u);
            trace polygone(R0,S,R1,E,R2,O);
            trace polygone(O,S,E);
            trace codeperp(R0,O,S,5);
            trace codeperp(O,S,E,5);
            % label.lft (btex O etex,O);            
            % label.urt (btex S etex,S);
            % label.lrt  (btex E etex,E);
            % label.top(btex R etex,R0);
            % label.rt(btex R etex,R1);
            % label.bot(btex R etex,R2);
        \end{Geometrie}
        }        
        }
        \item sur une feuille non quadrillée, reproduire le patron en vraie grandeur.
        % \begin{Geometrie}[CoinBG={u*(-1,-2)},CoinHD={u*(10.5,5.5)}]
        %     trace feuillet;
        %     pair R[],O,S,E;
        %     O=u*(1,2);            
        %     S-O=u*(4,0);
        %     E-O=u*(4,-2);
        %     R0-O=u*(0,3);
        %     R1-S=u*(5,0);
        %     R2=cercles(O,3u) intersectionpoint subpath (4,6) of cercles(E,sqrt(29)*u);
        %     trace polygone(R0,S,R1,E,R2,O);
        %     trace polygone(O,S,E);
        %     trace codeperp(R0,O,S,5);
        %     trace codeperp(O,S,E,5);
        %     label.lft (btex O etex,O);            
        %     label.urt (btex S etex,S);
        %     label.lrt  (btex E etex,E);
        %     label.top(btex R etex,R0);
        %     label.rt(btex R etex,R1);
        %     label.bot(btex R etex,R2);
        % \end{Geometrie}
    \end{enumerate}
\end{exercice*}
\begin{corrige}
    %\setcounter{partie}{0} % Pour s'assurer que le compteur de \partie est à zéro dans les corrigés
    %\phantom{rrr}    
    $ROSE$ est une pyramide telle que $RO=\Lg[cm]{3}$ ; $OS=\Lg[cm]{4}$ et $SE=\Lg[cm]{2}$.

    \begin{enumerate}
        \item Reporter les longueurs connues sur cette réprésentation en perspective cavalière.
        
        \scalebox{0.7}{
        \begin{Geometrie}
            pair R,O,S,E;
            O=u*(1,2);
            R-O=u*(0,3);
            S-O=u*(4,0);
            E-O=u*(3,-1);            
            label.lft (btex O etex,O);
            label.ulft(btex R etex,R);
            label.urt (btex S etex,S);
            label.rt  (btex E etex,E);
            trace R--S--E--O--cycle;
            trace R--E;
            trace O--S dashed evenly;
            trace codeperp(R,O,S,5); draw codeperp(O,S,E,5);
            trace appelation(O,S,3mm,btex \textcolor{red}{\Lg[cm]{4}} etex);
            trace appelation(O,R,3mm,btex \textcolor{red}{\Lg[cm]{3}} etex);
            trace appelation(E,S,-3mm,btex \textcolor{red}{\Lg[cm]{2}} etex);
        \end{Geometrie}
        }
        \item Sur ce patron qui n'est pas en vraie grandeur :
        \begin{itemize}
            \item écrire le nom des sommets de chaque triangle ;
            \item coder les segments de même longueur ;
            \item indiquer les longueurs connues.
        \end{itemize}

        \vspace*{-15mm}
        \hspace*{-7mm}
        \rotatebox{20}{
        \scalebox{0.7}{
        \begin{Geometrie}[CoinBG={u*(-1,-2)},CoinHD={u*(10,5.5)}]
            % trace feuillet;
            pair R[],O,S,E;
            O=u*(0.5,2);            
            S-O=u*(4,0);
            E-O=u*(4,-2);
            R0-O=u*(0,3);
            R1-S=u*(5,0);
            R2=cercles(O,3u) intersectionpoint subpath (4,6) of cercles(E,sqrt(29)*u);
            trace polygone(R0,S,R1,E,R2,O);
            trace polygone(O,S,E);
            trace codeperp(R0,O,S,5);
            trace codeperp(O,S,E,5);
            label.lft(btex \textcolor{red}{O} etex,O);            
            label.urt(btex \textcolor{red}{S} etex,S);
            label.lrt(btex \textcolor{red}{E} etex,E);
            label.top(btex \textcolor{red}{R} etex,R0);
            label.rt(btex \textcolor{red}{R} etex,R1);
            label.bot(btex \textcolor{red}{R} etex,R2);
            drawoptions(withcolor red);
            marque_s:=marque_s/3;
            trace Codelongueur(R0,O,O,R2,1);
            trace Codelongueur(R0,S,S,R1,2);
            trace Codelongueur(R1,E,E,R2,3);
            trace appelation(O,S,3mm,btex \textcolor{red}{\Lg[cm]{4}} etex);
            trace appelation(O,R0,3mm,btex \textcolor{red}{\Lg[cm]{3}} etex);
            trace appelation(S,E,3mm,btex \textcolor{red}{\Lg[cm]{2}} etex);
        \end{Geometrie}
        }        
        }
    \end{enumerate}
    \Coupe
    \begin{enumerate}
        \setcounter{enumi}{2}
        \item sur une feuille non quadrillée, reproduire le patron en vraie grandeur.
        
        \textcolor{red}{\dots}        
    \end{enumerate}
\end{corrige}

