\begin{exercice*}
    \begin{enumerate}
        \item Poser la division de $1$ par $7$, en donnant six décimales.
        \item Donner les six décimales suivantes sans aucun calcul supplémentaires.
        \begin{myBox}{\emoji{light-bulb} Définition}
            Cette suite de chiffres qui se répètent dans la partie décimale, s'appelle période.
        \end{myBox}
        \item Déterminer la période de :
        \begin{multicols}{2}
            \begin{enumerate}
                \begin{spacing}{1.5}
                    \item $\dfrac{2}{7}$
                    \item $\dfrac{5}{7}$
                    \item $\dfrac{1}{13}$
                    \item $\dfrac{3}{13}$
                \end{spacing}
            \end{enumerate}            
        \end{multicols}
    \end{enumerate}
\end{exercice*}
\begin{corrige}
    %\setcounter{partie}{0} % Pour s'assurer que le compteur de \partie est à zéro dans les corrigés
    % \phantom{rrr}    
    \begin{enumerate}
        \item Poser la division de $1$ par $7$, en donnant six décimales.
        
        % \opset{voperator=bottom,decimalsepsymbol={,}}
        \scalebox{1}{
            {\red \opdiv[displayintermediary=all, shiftdecimalsep=both, period, decimalsepsymbol={,}, voperator=bottom]{1}{7}}
        }
        \item Donner les six décimales suivantes sans aucun calcul supplémentaires.
        
        {\red $142857$}
        \item Déterminer la période de :
        \begin{multicols}{2}            
            \begin{enumerate}
                \begin{spacing}{1.5}
                    \item $\dfrac{2}{7}  {\red \Leftarrow 258714}$
                    \item $\dfrac{5}{7}  {\red \Leftarrow 714285}$
                    \item $\dfrac{1}{13} {\red \Leftarrow 076923}$
                    \item $\dfrac{3}{13} {\red \Leftarrow 230769}$
                \end{spacing}
            \end{enumerate}
        \end{multicols}
    \end{enumerate}
\end{corrige}

