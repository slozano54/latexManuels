\begin{exercice*}
    Un confiseur prépare des sachets de bonbons en mettant un tiers de caramels, deux neuvièmes de bonbons au fruits 
    et pour le reste des bonbons au chocolat.

    \begin{enumerate}
        \item Déterminer la fraction de l'ensemble qui représente la proportion des caramels.
        \item Déterminer la fraction de l'ensemble qui représente la proportion des bonbons aux fruits et au caramel.
        \item Déterminer la fraction de l'ensemble qui représente la proportion des bonbons au chocolat.
    \end{enumerate}
\end{exercice*}
\begin{corrige}
    %\setcounter{partie}{0} % Pour s'assurer que le compteur de \partie est à zéro dans les corrigés
    % \phantom{rrr}    
    Un confiseur prépare des sachets de bonbons en mettant un tiers de caramels, deux neuvièmes de bonbons au fruits 
    et pour le reste des bonbons au chocolat.

    \begin{enumerate}
        \item Déterminer la fraction de l'ensemble qui représente la proportion des caramels.
        
        {\red La fraction qui représente la proportion de caramels vaut $\dfrac{1}{3}$.}
        \item Déterminer la fraction de l'ensemble qui représente la proportion des bonbons aux fruits et au caramel.
        
        {\red La fraction qui représente la proportion des bonbons aux fruits et au caramel vaut $\dfrac{1}{3}+\dfrac{2}{9}=\dfrac{3}{9}+\dfrac{2}{9}=\dfrac{5}{9}$.}
        \item Déterminer la fraction de l'ensemble qui représente la proportion des bonbons au chocolat.
        
        {\red La fraction qui représente la proportion des bonbons au chocolat vaut $\dfrac{9}{9}-\dfrac{5}{9}=\dfrac{4}{9}$.}
    \end{enumerate}

\end{corrige}

