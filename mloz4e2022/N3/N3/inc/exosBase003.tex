\begin{exercice*}
    Parmi les nombres suivants :

    $\dfrac{1}{3}$;\hfill$\dfrac{3}{4}$;\hfill$\dfrac{3}{5}$;\hfill$\dfrac{5}{6}$;\hfill$\dfrac{5}{7}$;\hfill$\dfrac{7}{8}$;\hfill$\dfrac{7}{9}$;\hfill$\dfrac{8}{10}$;\hfill$\dfrac{9}{11}$.\hfill
    \begin{enumerate}
        \item Dresser la liste des nombres décimaux.
        \item Pour chaque fraction de la liste, déterminer les diviseurs premiers du dénominateur.
        \item Écrire chaque fraction de la liste avec une puissance de 10 comme dénominateur.
        \item Faire une conjecture.
    \end{enumerate}
\end{exercice*}
\begin{corrige}
    %\setcounter{partie}{0} % Pour s'assurer que le compteur de \partie est à zéro dans les corrigés
    % \phantom{rrr}    
    Parmi les nombres suivants :

    $\dfrac{1}{3}$;\hfill$\dfrac{3}{4}$;\hfill$\dfrac{3}{5}$;\hfill$\dfrac{5}{6}$;\hfill$\dfrac{5}{7}$;\hfill$\dfrac{7}{8}$;\hfill$\dfrac{7}{9}$;\hfill$\dfrac{8}{10}$;\hfill$\dfrac{9}{11}$.\hfill

    \begin{enumerate}
        \item Dresser la liste des nombres décimaux.
        
        \smallskip
        {\red $\dfrac{3}{4}$;\hfill$\dfrac{3}{5}$;\hfill$\dfrac{7}{8}$;\hfill$\dfrac{8}{10}$.\hfill}
        \smallskip
        \item Pour chaque fraction de la liste, déterminer les diviseurs premiers du dénominateur.
        
        {\red 
            $\dfrac{3}{4} \Leftarrow 2$ ;\hfill
            $\dfrac{3}{5} \Leftarrow 5$ ;\hfill
            $\dfrac{7}{8} \Leftarrow 2$ ;\hfill
            $\dfrac{8}{10}\Leftarrow 2$ et $5$.
        }
        \item Écrire chaque fraction de la liste avec une puissance de 10 comme dénominateur.
        \begin{spacing}{1.5}            
            {\red
                $\dfrac{3}{4}=\dfrac{3\times 25}{4\times 25}=\dfrac{75}{100}$;\hfill
                $\dfrac{3}{5}=\dfrac{3\times 2}{5\times 2}=\dfrac{6}{10}$.

                $\dfrac{7}{8}=\dfrac{7\times 125}{8\times 125}=\dfrac{875}{1000}$;\hfill
                $\dfrac{8}{10}=\dfrac{8}{10}$.
            }
        \end{spacing}

        \item Faire une conjecture.
        
        {\red Si les seuls diviseurs premiers du dénominateur d'une fraction sont $2$ et $5$ alors cette fraction admet une écriture décimale.}
    \end{enumerate}
\end{corrige}

