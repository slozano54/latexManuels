\section{Comparaison}
% \subsubsection{Définitions et vocabulaire}
\begin{definition}
    \begin{itemize}
        \item {\bf L'ordre croissant}, c'est l'ordre du plus petit au plus grand.
        \item {\bf L'ordre décroissant}, c'est l'ordre du plus grand au plus petit.
        \item {\bf Ranger dans l'ordre croissant}, c'est ranger du plus petit au plus grand.
        \item {\bf Ranger dans l'ordre décroissant}, c'est ranger du plus grand au plus petit.
    \end{itemize}
\end{definition}

\begin{exemples*1}
    \begin{itemize}
        \item \num{-0,5}; \num{1}; \num{1.3}; \num{12} sont ranger dans l'ordre croissant.
        \item \num{4}; \num{-2}; \num{-2.7}; \num{-4} sont rangés dans l'ordre décroissant.
    \end{itemize}
\end{exemples*1}

\begin{propriete}[Comparaison lorsque les dénominateurs sont les mêmes]
    Si deux écritures fractionnaires ont le même dénominateur alors elles sont rangées dans l'ordre de leurs numérateurs.
\end{propriete}

\begin{preuve}
    Lorsque deux groupes sont aussi nombreux l'un que l'autre (les dénominateurs sont identiques) et qu'ils ont à se partager
    une somme d'argent (les numérateurs), il vaut mieux faire partie du groupe qui se partage la plus grande somme d'argent (les numérateurs).
\end{preuve}

\begin{exemples*1}
    \begin{enumerate}
        \item Comparer $\dfrac{\num{8.5}}{\num{4.2}}$ et $\dfrac{\num{4.8}}{\num{4.2}}$.
        
        \smallskip
        \item Comparer $\dfrac{\num{48}}{\num{22}}$ et $\dfrac{\num{76}}{\num{22}}$.
    \end{enumerate}
    \correction
    \begin{enumerate}
        \item 
        $\left.
        \begin{array}{l}
        $Les dénominateurs de $\dfrac{\num{8.5}}{\num{4.2}}$ et $\dfrac{\num{4.8}}{\num{4.2}}$ sont les mêmes.$\\
        \num{8.5}$ est plus grand que $\num{4.8}.\\
        \end{array}
        \right\}$
        donc $\dfrac{8,5}{4,2}$ est plus grande que $\dfrac{4,8}{4,2}$.

        \smallskip
        \item
        $\left.
        \begin{array}{l}
        $Les dénominateurs de $\dfrac{\num{48}}{\num{22}}$ et $\dfrac{\num{76}}{\num{22}}$ sont les mêmes.$\\
        48$ est plus petit que $76.\\
        \end{array}
        \right\}$
        donc $\dfrac{48}{22}$ est plus petite que $\dfrac{76}{22}$.
    \end{enumerate}
\end{exemples*1}

\begin{propriete}[Comparaison lorsque les numérateurs sont les mêmes]
    Si deux écritures fractionnaires ont le même numérateur alors elles sont rangées dans l'ordre inverse de leurs dénominateurs.
\end{propriete}

\begin{preuve}
    Lorsque deux groupes se partagent une somme d'argent identique (les numérateurs), il vaut mieux faire partie du groupe le moins nombreux (les dénominateurs).
\end{preuve}

\begin{exemple*1}
    Comparer $\dfrac{\num{8.5}}{\num{4.2}}$ et $\dfrac{\num{8.5}}{\num{3.6}}$.
    \correction
    $\left.
    \begin{array}{l}
    $Les numérateurs de $\dfrac{\num{8.5}}{\num{4.2}}$ et $\dfrac{\num{8.5}}{\num{3.6}}$ sont les mêmes.$\\
    \num{4.2}$ est plus grand que $\num{4.8}.\\
    \end{array}
    \right\}$
    donc $\dfrac{\num{8.5}}{\num{4.2}}$ est plus petite que $\dfrac{\num{8.5}}{\num{3.6}}$.
\end{exemple*1}

% \subsubsection{}
\begin{methode}[Comparaison - dénominateurs ou numérateurs différents]
    Lorsque deux écritures fractionnaires ont des dénominateurs différents ou des numérateurs différents, pour les comparer, on peut :
    \begin{itemize}
        \item les réduire au même numérateur ou au même dénominateur.
        \item puis les comparer avec l'une des propriétés précédentes.
    \end{itemize}
    \exercice
    \begin{spacing}{1.5}
        Comparer les fractions suivantes en :
        \begin{enumerate}
            \item les réduisant au même dénominateur
            \begin{enumerate}
                \item $\dfrac{\num{11 }}{\num{2 }}$ et $\dfrac{\num{11 }}{\num{4 }}$.
                \item $\dfrac{\num{3.5}}{\num{5 }}$ et $\dfrac{\num{4  }}{\num{15}}$.
                \item $\dfrac{\num{5.4}}{\num{28}}$ et $\dfrac{\num{3.6}}{\num{7 }}$.
                \item $\dfrac{\num{5  }}{\num{8 }}$ et $\dfrac{\num{14 }}{\num{10}}$.
            \end{enumerate}
            \item les réduisant au même numérateur
            \begin{enumerate}
                \item $\dfrac{\num{3.5}}{\num{5 }}$ et $\dfrac{\num{7  }}{\num{15}}$.
                \item $\dfrac{\num{5.4}}{\num{25}}$ et $\dfrac{\num{2.7}}{\num{7 }}$.
                \item $\dfrac{\num{5  }}{\num{8 }}$ et $\dfrac{\num{14 }}{\num{10}}$.            
            \end{enumerate}
        \end{enumerate}
    \end{spacing}
    \correction
    \begin{spacing}{1.5}
        \begin{enumerate}
            \item 
            \begin{enumerate}
                \item $\dfrac{11}{2}=\dfrac{11\times {\red 2}}{2\times {\red 2}}=\dfrac{22}{4}$ donc $\dfrac{\num{11 }}{\num{2 }}$ est plus grand que $\dfrac{\num{11 }}{\num{4 }}$.
                \item $\dfrac{3,5}{5}=\dfrac{3,5\times {\red 3}}{5\times {\red 3}}=\dfrac{10,5}{15}$ donc $\dfrac{\num{3.5}}{\num{5 }}$ est plus grand que $\dfrac{\num{4  }}{\num{15}}$.
                \item $\dfrac{3,6}{7}=\dfrac{3,6\times {\red 4}}{7\times {\red 4}}=\dfrac{14,4}{28}$ donc $\dfrac{\num{5.4}}{\num{28}}$ est plus petit que $\dfrac{\num{3.6}}{\num{7 }}$.
                \item $\dfrac{\num{13}}{\num{8}} = \dfrac{13\times{\red 5}}{8\times {\red 5}} = \dfrac{65}{40}$ et $\dfrac{14}{10} = \dfrac{14\times{\red 4}}{10\times {\red 4}}=\dfrac{56}{40}$
                
                donc $\dfrac{\num{13}}{\num{8}}$ est plus grand que $\dfrac{\num{14}}{\num{10}}$.
            \end{enumerate}
            \item les réduisant au même numérateur
            \begin{enumerate}
                \item car $\dfrac{3,5}{5}=\dfrac{3,5\times {\red 2}}{5\times {\red 2}}=\dfrac{7}{10}$ donc $\dfrac{\num{3.5}}{\num{5 }}$ est plus grand que $\dfrac{\num{7  }}{\num{15}}$.
                \item car $\dfrac{2,7}{7}=\dfrac{2,7\times {\red 2}}{7\times {\red 2}}=\dfrac{5,4}{14}$ donc $\dfrac{\num{5.4}}{\num{25}}$ est plus petit que $\dfrac{\num{2.7}}{\num{7 }}$.
                \item $\dfrac{\num{13}}{\num{8}} = \dfrac{13\times{\red 14}}{8\times {\red 14}} = \dfrac{182}{112}$ et $\dfrac{14}{10} = \dfrac{14\times{\red 13}}{10\times {\red 13}}=\dfrac{182}{130}$
                
                donc $\dfrac{\num{13}}{\num{8}}$ est plus grand que $\dfrac{\num{14}}{\num{10}}$.
            \end{enumerate}
        \end{enumerate}
    \end{spacing}
\end{methode}

\begin{remarque}
    \titreRemarque{\emoji{light-bulb} Dis moi pas qu'c'est pas vrai!}

    Si il y a 25 croissants (numérateur) à partager alors :    
    \begin{itemize}
        \item si on est plus de 25 élèves (dénominateur), on aura chacun moins de 1 croissant.
        \item si on est exactement 25 élèves(dénominateur), on aura chacun exactement 1 croissant.
        \item si on est moins de 25 élèves(dénominateur), on aura chacun plus de 1 croissant.
    \end{itemize}

    \smallskip
    Transposées en vocabulaire des fractions, cela donne les propriétés suivantes ! 
\end{remarque}

{\renewcommand{\StringPROPRIETE}{PROPRI\'ET\'ES}
\begin{propriete}[\admises]
    \begin{itemize}
        \item Si le numérateur (nombre de croissants) est \textbf{inférieur} au dénominateur (nombre d'élèves) alors la \textbf{fraction est inférieure à 1.}
        \item Si le numérateur (nombre de croissants) est \textbf{égal} au dénominateur (nombre d'élèves) alors la \textbf{fraction est égale à 1.}
        \item Si le numérateur (nombre de croissants) est \textbf{supérieur} au dénominateur (nombre d'élèves) alors la \textbf{fraction est supéreireure à 1.}
    \end{itemize}
\end{propriete}
}

\begin{exemples*1}
    \begin{itemize}
        \item le numérateur 17 est inférieur au dénominateur 25 donc $\dfrac{17}{25}$est inférieure à $1$
        \item le numérateur 25 est égal au dénominateur 25 donc $\dfrac{25}{25}$est égale à $1$
        \item le numérateur 37 est supérieur au dénominateur 25 donc $\dfrac{37}{25}$ est supérieure à $1$
    \end{itemize}
\end{exemples*1}

\begin{remarque}
    \titreRemarque{non fortuite !}

    Lorsque qu'une première fraction est inférieure à 1 et qu'une seconde est supérieure à 1
    
    alors la première est inférieure à la seconde!
\end{remarque}

\begin{methode}[Comparer à 1 ou à un nombre entier simple]
    Pour comparer deux fractions, on peut les comparer à un entier simple.
% $\dfrac{2}{3}$ est inférieure à 1 et $\dfrac{5}{4}$ est supérieure à 1 donc $\dfrac{2}{3}$ est inférieure $\dfrac{5}{4}$\\
% En effet, dire que $\dfrac{5}{4}$ est supérieure à 1 revient à dire que 1 est inférieur à $\dfrac{5}{4}$\\
% donc $\dfrac{2}{3}$ est inférieure à 1 qui est lui-même inférieur à $\dfrac{5}{4}$ !
% }
\exercice
\begin{enumerate}
    \item Comparer $\dfrac{2}{3}$ et $\dfrac{5}{4}$.
    \item Comprarer $\dfrac{13}{6}$ et $\dfrac{5}{3}$.
\end{enumerate}
\correction
\begin{enumerate}
    \item $\dfrac{2}{3}<1$ et $1<\dfrac{5}{4}$ donc $\dfrac{2}{3}<\dfrac{5}{4}$.
    \item $\dfrac{13}{6}>2$ et $2>\dfrac{5}{3}$ donc $\dfrac{13}{6}>\dfrac{5}{3}$.
\end{enumerate}
\end{methode}