\section{Addition et soustraction}

\begin{propriete}[\admise]
    $a, b$ et $k$ sont des nombres, $k$ étant non nul.
    \begin{itemize}
        \item \textcolor{B2}{\bf Addition} : $\dfrac{a}{\textcolor{red}{k}}+\dfrac{b}{\textcolor{red}{k}} = \dfrac{a+b}{\textcolor{red}{k}}$
        
        \smallskip
        \item \textcolor{B2}{\bf Soustraction} : $\dfrac{a}{\textcolor{red}{k}}-\dfrac{b}{\textcolor{red}{k}} = \dfrac{a-b}{\textcolor{red}{k}}$
    \end{itemize}
\end{propriete}


\begin{methode*1}[Addition/Soustraction avec le même dénominateur]
    Pour additionner ou soustraire deux écritures fractionnaires de même dénominateur :
    \begin{itemize}
        \item on additionne, ou on soustrait les numérateurs.
        \item on écrit le résultat sur le dénominateur commun aux écritures de départ.
    \end{itemize}
    \exercice
    Calculer
    \begin{list}{}{}
        \item $A=\dfrac{5}{3} + \dfrac{7}{3}$
        \item $B=\dfrac{\num{7.5}}{4} - \dfrac{\num{5.2}}{4}$
        \item $C=\dfrac{37}{12} - \dfrac{50}{12} + \dfrac{10}{12}$
    \end{list}
    \correction
    \begin{list}{}{}
        \item $A=\dfrac{5}{{\red 3}}  + \dfrac{7}{    {\red 3}} = \dfrac{5+7}{    {\red 3}} = \psshadowbox{\dfrac{12}{3}}$
        \item $B=\dfrac{\num{7.5}}{{\red 4}}  - \dfrac{\num{5.2}}{{\red 4}} = \dfrac{\num{7.5}+\num{5.2}}{{\red 4}} = \psshadowbox{\dfrac{\num{2.3}}{4}}$
        \item $C=\dfrac{37}{ {\red 12}} - \dfrac{50}{ {\red 12}} + \dfrac{10}{    {\red 12}} = \dfrac{37-50+10}{{\red 12}} = \psshadowbox{\dfrac{-3}{12}}$
    \end{list}
\end{methode*1}

\begin{methode*1}[Addition/Soustraction avec des dénominateurs différents]
    Pour additionner ou soustraire deux fractions qui ont des dénominateurs différents :
    \begin{itemize}
        \item on cherche un dénominateur commun.
        \item on les réduit au même dénominateur.
        \item on ajoute ou on soustrait les fractions obtenues à l'aide de la proprieté précédente.
    \end{itemize}
    \exercice
    Calculer \hfill $A=\dfrac{3}{4}+\dfrac{\num{4.7}}{6}$\hfill $B=\dfrac{25}{49}-\dfrac{2}{7}$\hfill $C=\dfrac{\num{2.1}}{3}-\dfrac{1}{6}+\dfrac{3}{18}$
    \correction
    Pour le calcul de $A$ :
    \begin{itemize}
        \item \textit{on cherche d'abord un nombre non nul dans la table de 4 {\bf et} de 6.}
        
        \smallskip
        $\left.
        \begin{array}{l}
            \text{{\bf Table de 4 : }}0~4~8~\psframebox{12}~16~20~\psframebox{24}~28~32~\psframebox{36}~40~44~\psframebox{48}~52~$\ldots$\\
            \text{{\bf Table de 6 : }}0~6~\psframebox{12}~18~\psframebox{24}~30~\psframebox{36}~42~\psframebox{48}~$\ldots$
        \end{array}
        \right\}$
        \textit{Il en suffit d'un !}

        \smallskip
        \item \textit{on transforme les écritures si nécesaire.}
        $\dfrac{3}{4}=\dfrac{3\times {\red 3}}{4\times {\red 3}}=\dfrac{9}{12}$ et $\dfrac{\num{4.7}}{6}=\dfrac{\num{4.7}\times {\red 2}}{6\times {\red 2}}=\dfrac{\num{9.4}}{12}$ 
    \end{itemize}

    \smallskip
    donc $A=\dfrac{9}{12}+\dfrac{\num{9.4}}{12}=\psshadowbox{\dfrac{\num{18.4}}{12}}$

    \smallskip
    \begin{spacing}{1.5}
        \begin{minipage}{0.48\linewidth}
            $B=\dfrac{25}{49}-\dfrac{2}{7}$\\
            $B=\dfrac{25}{49}-\dfrac{2\times {\red 7}}{7\times {\red 7}}$\\
            $B=\dfrac{25-2\times 7}{49}$\\
            $B=\psshadowbox{\dfrac{11}{49}}$
        \end{minipage}
        \begin{minipage}{0.48\linewidth}
            $C=\dfrac{\num{2.1}}{3}-\dfrac{1}{6}+\dfrac{3}{18}$\\
            $C=\dfrac{2,1\times {\red 6}}{3\times {\red 6}}-\dfrac{1\times {\red 3}}{6\times {\red 3}}+\dfrac{3}{18}$\\
            $C=\dfrac{\num{12.6}}{18}-\dfrac{3}{18}+\dfrac{3}{18}$\\
            $C=\dfrac{\num{12.6}-3+3}{18}$\\
            $C=\psshadowbox{\dfrac{\num{12.6}}{18}}$
        \end{minipage}        
    \end{spacing}
\end{methode*1}

\begin{remarque}
    \titreRemarque{\emoji{bomb} ATTENTION \emoji{bomb} }

    $\dfrac{\pnode(0,0.2em){A}{9}}{\pnode(0,0.2em){C}{2}}+\dfrac{10}{8}=\dfrac{\pnode(0.4,0.2em){D}{9+10}}{\pnode(0.4,0.2em){B}{2+8}}$\ncarc{-}{A}{B} \ncarc{-}{C}{D}.
    ~En effet, $\dfrac{9}{2}=4,5$ et $\dfrac{10}{8}=1,25$ donc $\dfrac{9}{2}+\dfrac{10}{8}=5,75$ et $\dfrac{9+10}{2+8}=\dfrac{19}{10}=1,9$.
\end{remarque}
