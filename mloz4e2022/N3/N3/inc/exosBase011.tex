\begin{exercice*}
    Pour chacun des nombres suivants :
    \begin{enumerate}
        \item déterminer son signe,
        \item puis réécrire avec un dénominateur positif et le minimum de signes moins.        
    \end{enumerate}
    \begin{multicols}{2}
        \begin{itemize}
            \item $\dfrac{3}{-7}$
            \item $-\dfrac{6}{-29}$
            \item $-\dfrac{-5}{9}$
            \item $-\dfrac{-10}{-53}$
        \end{itemize}
    \end{multicols}
    
\end{exercice*}
\begin{corrige}
    %\setcounter{partie}{0} % Pour s'assurer que le compteur de \partie est à zéro dans les corrigés
    % \phantom{rrr}    
    Pour chacun des nombres suivants :

    \begin{enumerate}
        \item déterminer son signe,
        \item puis réécrire avec un dénominateur positif et le minimum de signes moins.        
    \end{enumerate}
    
    \begin{itemize}
        \item $\dfrac{3}{-7}$       
        
        {\red Comme le numérateur et le dénominateur sont de signes différents, cette fraction est négative.
        
        $\dfrac{3}{-7} = -\dfrac{3}{7}$
        }
        \item $-\dfrac{6}{-29}$     
        
        {\red $-\dfrac{6}{-29}$ est l'opposé d'un nombre négatif, donc il est positif.
        
        $-\dfrac{6}{-29} = \dfrac{6}{29}$
        }
    \end{itemize}
    \Coupe
    \begin{itemize}
        \item $-\dfrac{-5}{9}$      
        
        {\red $-\dfrac{-5}{9}$ est l'opposé d'un nombre négatif, donc il est positif.
        
        $-\dfrac{-5}{9} = \dfrac{5}{9}$
        }
        \item $-\dfrac{-10}{-53}$   
        {\red $-\dfrac{-10}{-53}$ est l'opposé d'un nombre positif, donc il est négatif.
        
        $-\dfrac{-10}{-53} = -\dfrac{10}{53}$            
        }
    \end{itemize}
    

\end{corrige}

