\begin{exercice*}
    Simplifier les fractions suivantes. En utilisant les critères de divisibilité ou les tables de multiplication.

    Écrire une étape expliquant la méthode utilisée.    
    \begin{multicols}{3}
        \begin{enumerate}
            \begin{spacing}{1.5}
                \item $\dfrac{77}{56}$
                \item $\dfrac{42}{78}$
                \item $\dfrac{55}{75}$
                \item $\dfrac{24}{27}$
                \item $\dfrac{48}{32}$
                \item $\dfrac{34}{46}$
            \end{spacing}
        \end{enumerate}            
    \end{multicols}    
\end{exercice*}
\begin{corrige}
    %\setcounter{partie}{0} % Pour s'assurer que le compteur de \partie est à zéro dans les corrigés
    % \phantom{rrr}    
    Simplifier les fractions suivantes. En utilisant les critères de divisibilité ou les tables de multiplication.

    Écrire une étape expliquant la méthode utilisée.
    \begin{multicols}{2}
        \begin{enumerate}
            \begin{spacing}{1.5}
                \item $\dfrac{77}{56} {\red ~= \dfrac{7\times11}{7\times8} = \dfrac{11}{8}  }$
                \item $\dfrac{42}{78} {\red ~= \dfrac{6\times7}{6\times13} = \dfrac{6}{13}  }$
                \item $\dfrac{55}{75} {\red ~= \dfrac{5\times11}{5\times15} = \dfrac{11}{15}}$
                \item $\dfrac{24}{27} {\red ~= \dfrac{3\times8}{3\times9} = \dfrac{8}{9}    }$
                \item $\dfrac{48}{32} {\red ~= \dfrac{16\times3}{16\times2} = \dfrac{3}{2}  }$
                \item $\dfrac{34}{46} {\red ~= \dfrac{2\times17}{2\times23} = \dfrac{17}{23}}$
            \end{spacing}
        \end{enumerate}            
    \end{multicols}
\end{corrige}

