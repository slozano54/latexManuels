\begin{exercice*}[Produits en croix]
    Après avoir calculé les produits en croix, indiquer si les nombres suiavnts sont égaux ou différents.    
    \begin{enumerate}
        \begin{spacing}{2}
            \item $\dfrac{45}{60}$ et $\dfrac{75}{100}$.
            \item $\dfrac{-87}{-42}$ et $\dfrac{\num{5.8}}{\num{2.8}}$.
            \item $\dfrac{\num{12.15}}{\num{35.1}}$ et $\dfrac{\num{5.8}}{\num{16.75}}$.        
        \end{spacing}
    \end{enumerate}
\end{exercice*}
\begin{corrige}
    %\setcounter{partie}{0} % Pour s'assurer que le compteur de \partie est à zéro dans les corrigés
    % \phantom{rrr}    
    Après avoir calculé les produits en croix, indiquer si les nombres suiavnts sont égaux ou différents.

    \begin{enumerate}
        \item $\dfrac{45}{60}$ et $\dfrac{75}{100}$.
        
        {\red $45\times 100 = \num{4500}$ et $60\times 75 = \num{4500}$ donc $\dfrac{45}{60}=\dfrac{75}{100}$.}
        \item $\dfrac{-87}{-42}$ et $\dfrac{\num{5.8}}{\num{2.8}}$.
        
        {\red $-87\times\num{2.8}=\num{-243.6}$ et $-42\times\num{5.8}=\num{-243.6}$ 
        
        donc $\dfrac{-87}{-42}=\dfrac{\num{5.8}}{\num{2.8}}$.}
        \item $\dfrac{\num{12.15}}{\num{35.1}}$ et $\dfrac{\num{5.8}}{\num{16.75}}$.        
        
        {\red $\num{12.15}\times\num{16.75}=\num{203.5125}$ et $\num{5.8}\times\num{35.1}=\num{203.58}$ donc $\dfrac{\num{12.15}}{\num{35.1}} \neq \dfrac{\num{5.8}}{\num{16.75}}$.}
    \end{enumerate}
\end{corrige}

