    %\setcounter{partie}{0} % Pour s'assurer que le compteur de \partie est à zéro dans les corrigés
    % \phantom{rrr}
    \begin{enumerate}
        \item Justifier si les nombres suivants sont des multiples communs de $12$ et $15$ ? Justifier.

        \begin{enumerate}
            \item $3$ est un \textbf{diviseur} de $12$ et de $15$ et non pas un multiple.
            \item $30$ est un multiple de $15$ mais pas de $12$.
            \item $120$ est un multiple de $12$, $12\times 10$, c'est aussi un multiple de $15$, $8\times 15$.
        \end{enumerate}
        \setcounter{enumi}{1}
        \item Multiples de $12$ : \sout{$0$} ; $12$ ; $24$ ; $36$ ; $48$ ; $\mathbf{60}$ ; $72$ ; $84$ ; $96$ ; $108$ ; $\mathbf{120}$ ; $132$ ; \dots

        Multiples de $15$ : \sout{$0$} ; $15$ ; $30$ ; $45$ ; $\mathbf{60}$ ; $75$ ; $90$ ; $105$ ; $\mathbf{120}$ ; $135$ ; \dots

        Le plus petit multiple commun non nul de $12$ et $15$ vaut donc $60$.
    \end{enumerate}
