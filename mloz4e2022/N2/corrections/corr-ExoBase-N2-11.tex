    %\setcounter{partie}{0} % Pour s'assurer que le compteur de \partie est à zéro dans les corrigés
    % \phantom{rrr}
    \begin{enumerate}
        \item Dans le tableau ci-dessous :
        \begin{itemize}
            \item Barrer $1$
            \item Barrer tous les multiples de $2$ sauf $2$ et entourer $2$ ;
            \item Barrer tous les multiples de $3$ restant sauf $3$ et entourer $3$ ;
            \item Barrer tous les multiples de $5$ restant sauf $5$ et entourer $5$ ;
            \item Barrer tous les multiples de $7$ restant sauf $7$ et entourer $7$ ;
        \end{itemize}

        \smallskip
        \begin{tabularx}{0.75\linewidth}{|*{10}{>{\centering\arraybackslash}X|}}
            \hline        
            \xout{$1 $}  & \Circled{$2 $}  & \Circled{$3 $}  & \xout{$4 $}  & \Circled{$5 $}  & \xout{$6 $}  & \Circled{$7 $}  & \xout{$8 $}  & \xout{$9 $}  & \xout{$10 $}\\
            \hline
            \Circled{$11$}  & \xout{$12$} & \Circled{$13$}  & \xout{$14$} & \xout{$15$} & \xout{$16$} & \Circled{$17$}  & \xout{$18$} & \Circled{$19$}  & \xout{$20 $}\\
            \hline
            \xout{$21$} & \xout{$22$} & \Circled{$23$}  & \xout{$24$} & \xout{$25$} & \xout{$26$} & \xout{$27$} & \xout{$28$} & \Circled{$29$}  & \xout{$30 $}\\
            \hline
            \Circled{$31$}  & \xout{$32$} & \xout{$33$} & \xout{$34$} & \xout{$35$} & \xout{$36$} & \Circled{$37$}  & \xout{$38$} & \xout{$39$} & \xout{$40 $}\\
            \hline
            \Circled{$41$}  & \xout{$42$} & \Circled{$43$}  & \xout{$44$} & \xout{$45$} & \xout{$46$} & \Circled{$47$}  & \xout{$48$} & \xout{$49$} & \xout{$50 $}\\
            \hline
            \xout{$51$} & \xout{$52$} & \Circled{$53$}  & \xout{$54$} & \xout{$55$} & \xout{$56$} & \xout{$57$} & \xout{$58$} & \Circled{$59$}  & \xout{$60 $}\\
            \hline
            \Circled{$61$}  & \xout{$62$} & \xout{$63$} & \xout{$64$} & \xout{$65$} & \xout{$66$} & \Circled{$67$}  & \xout{$68$} & \xout{$69$} & \xout{$70 $}\\
            \hline
            \Circled{$71$}  & \xout{$72$} & \Circled{$73$}  & \xout{$74$} & \xout{$75$} & \xout{$76$} & \xout{$77$} & \xout{$78$} & \Circled{$79$}  & \xout{$80 $}\\
            \hline
            \xout{$81$} & \xout{$82$} & \Circled{$83$}  & \xout{$84$} & \xout{$85$} & \xout{$86$} & \xout{$87$} & \xout{$88$} & \Circled{$89$}  & \xout{$90 $}\\
            \hline
            \xout{$91$} & \xout{$92$} & \xout{$93$} & \xout{$94$} & \xout{$95$} & \xout{$96$} & \Circled{$97$}  & \xout{$98$} & \xout{$99$} & \xout{$100$}\\
            \hline
        \end{tabularx}
        \smallskip
        \item Après $7$, le nombre premier suivant sera $11$ or tous les multiples de $11$ inférieurs à $100$ sont déjà barrés.

        Il n'est donc pas nécessaire de continuer.
        \item Liste de tous les nombres premiers inférieurs à $100$ :

        $2$, $3$, $5$, $7$, $11$, $13$, $17$, $19$, $23$, $29$, $31$, $37$, $41$, $43$, $47$, $53$, $59$, $61$, $67$, $71$, $73$, $79$, $83$, $89$, $97$.
    \end{enumerate}
