    %\setcounter{partie}{0} % Pour s'assurer que le compteur de \partie est à zéro dans les corrigés
    % \phantom{rrr}
    \begin{enumerate}
        \item $60 = 10\times 6$ et $330 = 55\times 6$, donc il peut faire $6$ lots identiques.
        \item $60 = 5\times 12$ et $330 = 27\times 12 + 6$, donc il ne peut pas faire $12$ lots identiques.
        \item \og Le nombre de lots est un diviseur \textbf{commun} de $\mathbf{60}$ et $\mathbf{330}$ .\fg
        \item Les diviseurs communs de $60$ et $330$ sont : $1$ ; $2$ ; $3$ ; $5$ ; $6$ ; $10$ ; $15$ ; $30$.

        Il peut donc réaliser $1$ ; $2$ ; $3$ ; $5$ ; $6$ ; $10$ ; $15$ ou $30$ lot(s) identique(s).
    \end{enumerate}
