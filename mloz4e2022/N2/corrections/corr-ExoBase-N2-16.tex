    %\setcounter{partie}{0} % Pour s'assurer que le compteur de \partie est à zéro dans les corrigés
    % \phantom{rrr}

    \begin{enumerate}
        \item La décomposition en facteurs premiers de $\num{1815}$ est : $3\times 5\times 11^{2}$
        \item Pour lister les diviseurs on peut faire un tableau :

        % \renewcommand{\arraystretch}{1}
        \smallskip
        $
        \begin{array}{|c|c|c|}
        \hline
        \times & 3^{0} & 3^{1}\\
        \hline
        5^{0}\times11^{0} & 5^{0}\times11^{0}\times3^{0}=\mathbf{{\color[HTML]{f15929}1}} & 5^{0}\times11^{0}\times3^{1}=\mathbf{{\color[HTML]{f15929}3}}\\
        \hline
        5^{0}\times11^{1} & 5^{0}\times11^{1}\times3^{0}=\mathbf{{\color[HTML]{f15929}11}} & 5^{0}\times11^{1}\times3^{1}=\mathbf{{\color[HTML]{f15929}33}}\\
        \hline
        5^{0}\times11^{2} & 5^{0}\times11^{2}\times3^{0}=\mathbf{{\color[HTML]{f15929}121}} & 5^{0}\times11^{2}\times3^{1}=\mathbf{{\color[HTML]{f15929}363}}\\
        \hline
        5^{1}\times11^{0} & 5^{1}\times11^{0}\times3^{0}=\mathbf{{\color[HTML]{f15929}5}} & 5^{1}\times11^{0}\times3^{1}=\mathbf{{\color[HTML]{f15929}15}}\\
        \hline
        5^{1}\times11^{1} & 5^{1}\times11^{1}\times3^{0}=\mathbf{{\color[HTML]{f15929}55}} & 5^{1}\times11^{1}\times3^{1}=\mathbf{{\color[HTML]{f15929}165}}\\
        \hline
        5^{1}\times11^{2} & 5^{1}\times11^{2}\times3^{0}=\mathbf{{\color[HTML]{f15929}605}} & 5^{1}\times11^{2}\times3^{1}=\mathbf{{\color[HTML]{f15929}1\,815}}\\
        \hline
        \end{array}
        $
        % \renewcommand{\arraystretch}{1}$
        \smallskip

        $\num{1815}$ a donc $(1+1)\times(1+1)\times(2+1) = 2\times2\times3 = 12$ diviseurs.

        En effet, dans la décomposition apparaît :
        \begin{itemize}
            \item Le facteur premier $3$ avec la multiplicité $1$, le facteur $3$ apparaît donc sous les formes : $3^{0}$ ou $3^{1}$ d'où le facteur $(1+1)$.
            \item Le facteur premier $5$ avec la multiplicité $1$, le facteur $5$ apparaît donc sous les formes : $5^{0}$ ou $5^{1}$ d'où le facteur $(1+1)$.
            \item Le facteur premier $11$ avec la multiplicité $2$, le facteur $11$ apparaît donc sous les formes : $11^{0}$ ou $11^{1}$ ou $11^{2}$ d'où le facteur $(2+1)$.
        \end{itemize}

        Enfin, voici la liste des $12$ diviseurs de $1\,815$ issus du tableau ci-dessus : $1\text{ ; }3\text{ ; }5\text{ ; }11\text{ ; }15\text{ ; }33\text{ ; }55\text{ ; }121\text{ ; }165\text{ ; }363\text{ ; }605\text{ ; }1\,815.$
    \end{enumerate}

