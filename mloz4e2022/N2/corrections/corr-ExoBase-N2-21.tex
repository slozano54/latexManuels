    %\setcounter{partie}{0} % Pour s'assurer que le compteur de \partie est à zéro dans les corrigés
    % \phantom{rrr}
    \begin{enumerate}
        \item $162=2\times 3\times 3\times 3\times 3$ et $108 = 2\times 2\times 3\times 3\times 3$.
        \item Deux diviseurs communs à $162$ et $108$, plus grands que $10$ : $27$ et $54$.
        \item
        \begin{enumerate}
            \item $108 = 36\times 3$ donc $36$ divise $108$ mais $162 = 4\times 36 + 18$ donc $36$ ne divise pas $162$

            Le cuisinier ne peut donc pas constituer $36$ barquettes.
            \item Le plus grand nombre qui divise à la fois $162$ et $108$ vaut $2\times 3\times 3\times 3 = 54$ donc le nombre maximal de barquettes
            que le cuisinier peut réaliser vaut $54$.
            \item $162 = 3\times 54$ et $108 = 2\times 54$ donc chaque barquette contiendra $3$ nems et $2$ samossas.
        \end{enumerate}
    \end{enumerate}
