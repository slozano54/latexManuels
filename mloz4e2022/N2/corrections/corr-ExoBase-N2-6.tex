    %\setcounter{partie}{0} % Pour s'assurer que le compteur de \partie est à zéro dans les corrigés
    % \phantom{rrr}
    \begin{spacing}{1.5}
        \begin{enumerate}
            \item On liste les multiples de $6$ et $14$ et le plus petit multiples commun non nul vaut $42$.
            \begin{multicols}3
                \item \phantom{rrr}

                $A=\dfrac{1}{6}+\dfrac{-3}{14}$

                $A=\dfrac{1\times 7}{6\times 7}+\dfrac{-3\times 3}{14\times 3}$

                $A=\dfrac{7}{42}+\dfrac{-9}{42}$

                $A=\dfrac{7-9}{42}$

                $A=\dfrac{-2}{42}$

                $A=\dfrac{-1}{21}$
                \columnbreak
                \item
                \begin{enumerate}
                    \item \phantom{rrr}

                    $B=\dfrac{3}{8}+\dfrac{7}{12}$

                    $B=\dfrac{3\times 3}{8\times 3}+\dfrac{7\times 2}{12\times 2}$

                    $B=\dfrac{9}{24}+\dfrac{14}{24}$

                    $B=\dfrac{9+14}{24}$

                    $B=\dfrac{23}{24}$
                    \columnbreak
                    \item \phantom{rrr}

                    $C=\dfrac{12}{21}-\dfrac{9}{14}$

                    $C=\dfrac{12\times 2}{21\times 2}-\dfrac{9\times 3}{14\times 3}$

                    $C=\dfrac{24}{42}-\dfrac{27}{42}$

                    $C=\dfrac{24-27}{42}$

                    $C=\dfrac{-3}{42}$

                    $C=\dfrac{-1}{14}$
                \end{enumerate}
            \end{multicols}
        \end{enumerate}
    \end{spacing}
