    %\setcounter{partie}{0} % Pour s'assurer que le compteur de \partie est à zéro dans les corrigés
    % \phantom{rrr}
    \begin{enumerate}
        \item $20$ est un multipe de $10$ mais pas de $15$. Donc il n'y aura pas un autre \og restau - ciné \fg dans $20$ jours.
        \item Aude va au restaurant à chaque multiple de {\bfseries \color{red}15} jours, au cinéma à chaque multiple de {\bfseries \color{green}10} jours.\\
        elle se fera à nouveau un « restau - ciné » à chaque multiple commun de {\bfseries \color{red}15} et de {\bfseries \color{green}10}.\\
        Pour trouver le nombre de jours avant le prochain « restau - ciné », on cherche le plus petit multiple qu'ils ont en commun.\\
        Un moyen d'y arriver est de décomposer les nombres de jours en produits de facteurs premiers et d'identifier les différences entre les décompositions :\\
{\bfseries \color{red}15} = {\bfseries \color{blue}5} $\times$ {\bfseries \color{red}3} \\
        {\bfseries \color{green}10} = {\bfseries \color{blue}5} $\times$ {\bfseries \color{green}2}\\
On multiplie les facteurs communs aux deux décompositions avec les facteurs spécifiques à chaque décomposition :\\
{\bfseries \color{blue}5} $\times$ {\bfseries \color{red}3} $\times$ {\bfseries \color{green}2} = 30\\
      Ce phénomène se produira à nouveau au bout de 30 jours,
      après {\bfseries \color{green}2 sorties} pour {\bfseries \color{red}aller au restaurant} et après {\bfseries \color{red}3 sorties} pour {\bfseries \color{green}aller au cinéma}.\\
30 est bien un multiple de {\bfseries \color{red}15} car :
         {\bfseries \color{blue}5} $\times$ {\bfseries \color{red}3} $\times$ {\bfseries \color{green}2} =
         ({\bfseries \color{blue}5} $\times$ {\bfseries \color{red}3}) $\times$ {\bfseries \color{green}2} =
         {\bfseries \color{red}15} $\times$ {\bfseries \color{green}2}.\\
        30 est bien un multiple de {\bfseries \color{green}10} car :
         {\bfseries \color{blue}5} $\times$ {\bfseries \color{red}3} $\times$ {\bfseries \color{green}2} =
         {\bfseries \color{blue}5} $\times$ {\bfseries \color{green}2} $\times$ {\bfseries \color{red}3} =
         ({\bfseries \color{blue}5} $\times$ {\bfseries \color{green}2}) $\times$ {\bfseries \color{red}3} =
         {\bfseries \color{green}10} $\times$ {\bfseries \color{red}3}.\\
    \end{enumerate}


