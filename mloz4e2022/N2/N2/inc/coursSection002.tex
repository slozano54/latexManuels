\section{Nombres premiers}
\begin{definition}[Nombre premier]
	Un nombre entier supérieur à 1 est un \textbf{nombre premier} s'il admet \textbf{exactement} deux diviseurs, 1 et lui-même.
\end{definition}

\begin{remarque}
	\begin{itemize}
		\item Quelques nombres premiers : $2$ ; $3$ ; $5$ ; $7$ ; $11$ ; $13$ ; $17$ ; $19$ ; $23$ ; \ldots
		\item $119$ n'est pas divisible par $2$, $3$, $5$ ou $9$ d'après les critères de divisibilité.
		
		Pour autant, il n'est pas premier car $7\times 17 = 119$. 
	\end{itemize}
\end{remarque}	

\begin{methode}[Primalité ou pas]
	Pour savoir si un nombre entier est premier :
	\begin{itemize}
		\item Il est nécessaire de tester sa divisibilité par tous les nombres premiers qui lui sont inférieurs,
		\item Mais il peut être suffisant de tester par beaucoup moins que ça si le nombre entier n'est pas premier, puisque dès lors qu'il a un autre diviseur que 1 et lui-même, il n'est pas premier dans ce cas il a alors au moins trois diviseurs. 
	\end{itemize}
	\exercice
	Pour chaque nombre, déterminer si c'est une nombre premier.
	
	\begin{itemize}
		\item $23$
		\item $\num{18645}$
	\end{itemize}

	\correction
	\begin{itemize}
		\item $23$ n'est divisible par aucun des nombres premiers qui lui sont inférieurs, $2$, $3$, $5$, $7$, $11$, $13$, $17$ et $19$, $23$ est donc premier.
		\item $\num{18645}$ se termine par $5$ donc il est divisible par $5$.
		$\num{18645}$ admet donc au moins trois diviseurs $1$, $5$ et $\num{18645}$, il n'est pas premier.
	\end{itemize}
\end{methode}

\begin{exemple*1}
$29$ est un nombre premier, en effet, On teste sa divisibilité par $2$, $3$, $5$, $7$, $11$, $13$, $17$, $19$ et $23$ :
\begin{itemize}
	\item $29=14\times 2+1$ donc $2$ ne divise pas $29$.
	\item $29=9\times 3+2$ donc $29$ n'est pas divisible par $3$.
	\item $29=5\times 5+4$ donc $29$ n'est pas un multiple de $5$.
	\item $29=4\times 7+1$ donc $7$ n'est pas un diviseur de $29$.
	\item $29=2\times 11+7$ donc $11$ ne divise pas $29$.
	\item $29=2\times 13+3$ donc $29$ n'est pas divisible par $13$.
	\item $29=1\times 17+12$ donc $29$ n'est pas un multiple de $17$.
	\item $29=1\times 19+10$ donc $19$ n'est pas un diviseur de $29$.
	\item $29=1\times 23+6$ donc $23$ n'est pas un diviseur de $29$.
\end{itemize}

$29$ n'est divisible par aucun des nombres premiers qui lui sont iférieurs, il est donc premier.
\end{exemple*1}

\begin{exemple*1}
	$27$ n'est pas un nombre premier, en effet, on teste sa divisibilité au plus par $2$, $3$, $5$, $7$, $11$, $13$, $17$, $19$ et $23$ :
	\begin{itemize}
		\item $27=13\times 2+1$ donc $2$ ne divise pas $27$.
		\item $27=9\times 3+0$ donc 3 est un diviseur de $27$ !
	\end{itemize}
Inutile de continuer, $27$ a au moins trois diviseurs $1$, $3$ et $27$, il n'est donc pas premier.
\end{exemple*1}