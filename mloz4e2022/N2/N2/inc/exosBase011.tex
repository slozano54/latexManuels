\begin{exercice*}
    On désigne sous le nom de crible d'Eratosthène (vers 276 av.J.-C - vers 194 av.J.-C),
    une méthode de recherche des nombres premiers plus petits qu'un entier naturel $N$ donné.

    \smallskip
    Pour ceci, on écrit la liste de tous les nombres jusqu'à $N$, puis on applique l'algorithme suivant.

    \begin{myBox}{Crible d'Eratosthène}
        \begin{itemize}
            \item On élimine $1$, en le barrant par exemple.
            \item On entoure $2$ et on élimine tous les multiples de $2$.
            \item On recommence avec le plus petit nombre non éliminé restant dans la liste : $3$.
            \item On recommence avec le plus petit nombre non entouré et non éliminé : $5$.
            \item \dots
            \item On réitère le procédé jusqu'à la partie entière de la racine carrée de $N$.
        \end{itemize}
    \end{myBox}

    Les nombres non éliminés sont les nombres premiers jusqu'à $N$.    
    \begin{enumerate}
        \item Déterminons tous les nombres premiers inférieurs à $100$ avec cette algorithme. Dans le tableau ci-dessous :
        \begin{itemize}
            \item Barrer $1$
            \item Barrer tous les multiples de $2$ sauf $2$ et entourer $2$ ;
            \item Barrer tous les multiples de $3$ restant sauf $3$ et entourer $3$ ;
            \item Barrer tous les multiples de $5$ restant sauf $5$ et entourer $5$ ;
            \item Barrer tous les multiples de $7$ restant sauf $7$ et entourer $7$ ;
        \end{itemize}

        \smallskip        
        \begin{tabularx}{0.75\linewidth}{|*{10}{>{\centering\arraybackslash}X|}}
            \hline	
            $1 $  & $2 $  & $3 $  & $4 $  & $5 $  & $6 $  & $7 $  & $8 $  & $9 $  & $10 $\\
            \hline
            $11$  & $12$  & $13$  & $14$  & $15$  & $16$  & $17$  & $18$  & $19$  & $20 $\\
            \hline
            $21$  & $22$  & $23$  & $24$  & $25$  & $26$  & $27$  & $28$  & $29$  & $30 $\\
            \hline
            $31$  & $32$  & $33$  & $34$  & $35$  & $36$  & $37$  & $38$  & $39$  & $40 $\\
            \hline
            $41$  & $42$  & $43$  & $44$  & $45$  & $46$  & $47$  & $48$  & $49$  & $50 $\\
            \hline
            $51$  & $52$  & $53$  & $54$  & $55$  & $56$  & $57$  & $58$  & $59$  & $60 $\\
            \hline
            $61$  & $62$  & $63$  & $64$  & $65$  & $66$  & $67$  & $68$  & $69$  & $70 $\\
            \hline
            $71$  & $72$  & $73$  & $74$  & $75$  & $76$  & $77$  & $78$  & $79$  & $80 $\\
            \hline
            $81$  & $82$  & $83$  & $84$  & $85$  & $86$  & $87$  & $88$  & $89$  & $90 $\\
            \hline
            $91$  & $92$  & $93$  & $94$  & $95$  & $96$  & $97$  & $98$  & $99$  & $100$ \\
            \hline
        \end{tabularx}
        \smallskip
        \item Expliquer pourquoi il n'est pas nécessaire de continuer.
        \item Écrire la liste de tous les nombres premiers inférieurs à $100$.
    \end{enumerate}
\end{exercice*}
\begin{corrige}
    %\setcounter{partie}{0} % Pour s'assurer que le compteur de \partie est à zéro dans les corrigés
    % \phantom{rrr}    
    \begin{enumerate}
        \item Dans le tableau ci-dessous :
        \begin{itemize}
            \item Barrer $1$
            \item Barrer tous les multiples de $2$ sauf $2$ et entourer $2$ ;
            \item Barrer tous les multiples de $3$ restant sauf $3$ et entourer $3$ ;
            \item Barrer tous les multiples de $5$ restant sauf $5$ et entourer $5$ ;
            \item Barrer tous les multiples de $7$ restant sauf $7$ et entourer $7$ ;
        \end{itemize}

        \smallskip        
        \begin{tabularx}{0.75\linewidth}{|*{10}{>{\centering\arraybackslash}X|}}
            \hline	
            \xout{$1 $}  & \Circled{$2 $}  & \Circled{$3 $}  & \xout{$4 $}  & \Circled{$5 $}  & \xout{$6 $}  & \Circled{$7 $}  & \xout{$8 $}  & \xout{$9 $}  & \xout{$10 $}\\
            \hline
            \Circled{$11$}  & \xout{$12$} & \Circled{$13$}  & \xout{$14$} & \xout{$15$} & \xout{$16$} & \Circled{$17$}  & \xout{$18$} & \Circled{$19$}  & \xout{$20 $}\\
            \hline
            \xout{$21$} & \xout{$22$} & \Circled{$23$}  & \xout{$24$} & \xout{$25$} & \xout{$26$} & \xout{$27$} & \xout{$28$} & \Circled{$29$}  & \xout{$30 $}\\
            \hline
            \Circled{$31$}  & \xout{$32$} & \xout{$33$} & \xout{$34$} & \xout{$35$} & \xout{$36$} & \Circled{$37$}  & \xout{$38$} & \xout{$39$} & \xout{$40 $}\\
            \hline
            \Circled{$41$}  & \xout{$42$} & \Circled{$43$}  & \xout{$44$} & \xout{$45$} & \xout{$46$} & \Circled{$47$}  & \xout{$48$} & \xout{$49$} & \xout{$50 $}\\
            \hline
            \xout{$51$} & \xout{$52$} & \Circled{$53$}  & \xout{$54$} & \xout{$55$} & \xout{$56$} & \xout{$57$} & \xout{$58$} & \Circled{$59$}  & \xout{$60 $}\\
            \hline
            \Circled{$61$}  & \xout{$62$} & \xout{$63$} & \xout{$64$} & \xout{$65$} & \xout{$66$} & \Circled{$67$}  & \xout{$68$} & \xout{$69$} & \xout{$70 $}\\
            \hline
            \Circled{$71$}  & \xout{$72$} & \Circled{$73$}  & \xout{$74$} & \xout{$75$} & \xout{$76$} & \xout{$77$} & \xout{$78$} & \Circled{$79$}  & \xout{$80 $}\\
            \hline
            \xout{$81$} & \xout{$82$} & \Circled{$83$}  & \xout{$84$} & \xout{$85$} & \xout{$86$} & \xout{$87$} & \xout{$88$} & \Circled{$89$}  & \xout{$90 $}\\
            \hline
            \xout{$91$} & \xout{$92$} & \xout{$93$} & \xout{$94$} & \xout{$95$} & \xout{$96$} & \Circled{$97$}  & \xout{$98$} & \xout{$99$} & \xout{$100$}\\
            \hline
        \end{tabularx}
        \smallskip
        \item Après $7$, le nombre premier suivant sera $11$ or tous les multiples de $11$ inférieurs à $100$ sont déjà barrés.
        
        Il n'est donc pas nécessaire de continuer.
        \item Liste de tous les nombres premiers inférieurs à $100$ :
        
        $2$, $3$, $5$, $7$, $11$, $13$, $17$, $19$, $23$, $29$, $31$, $37$, $41$, $43$, $47$, $53$, $59$, $61$, $67$, $71$, $73$, $79$, $83$, $89$, $97$.
    \end{enumerate}
\end{corrige}

