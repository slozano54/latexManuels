\begin{exercice*}
    Dire si les assertions suivantes sont vraies ou fausses. Justifier.
    \begin{enumerate}
        \item $111$ est un nombre premier.
        \item Aucun nombre pair n'est premier.
        \item Tous les nombres impairs sont premiers.
        \item $1$ n'est pas un nombre premier.
    \end{enumerate}

    \hrefMathalea{https://coopmaths.fr/mathalea.html?ex=4A10,s=1,s2=false,i=0&v=l}
\end{exercice*}
\begin{corrige}
    %\setcounter{partie}{0} % Pour s'assurer que le compteur de \partie est à zéro dans les corrigés
    % \phantom{rrr}    
    \begin{enumerate}
        \item $1+1+1 = 3$ donc $111$ est divisible par $3$ donc $111$ n'est pas premier donc \textbf{l'assertion est fausse}.
        \item $2$ est pair et premier. \textbf{L'assertion est fausse}.
        \item $9$ est impair et divisible par $3$ donc $9$ n'est pas premier. \textbf{L'assertion est fausse}.
        \item $1$ n'a qu'un seul diviseur, lui-même, or un nombre premier doit en avoir exactement deux.
        
        \textbf{L'assertion est vraie}.
    \end{enumerate}
\end{corrige}

