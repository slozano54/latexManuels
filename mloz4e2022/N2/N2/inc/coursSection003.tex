\section{Décomposition en facteurs premiers}

\begin{propriete}[\admise]
Tout nombre entier $n$ supérieur à 1 admet une unique \textbf{décomposition} en 

produit de \textbf{facteurs premiers}.
$$n=p_1^{a_1}\times...\times p_k^{a_k}$$
\end{propriete}

\begin{methode}[Décomposition en facteurs premiers]
    Soit N un nombre entier quelconque, dont on cherche la décomposition en facteurs premiers :
    \begin{itemize}
        \item On teste la divisibilité de N par les nombres premiers en commençant par 2
        \item Si N est divisible par 2, on teste à nouveau la divisibilité par 2 du quotient de N par 2
        \item Tant que c'est possible on continue avec 2
        \item Lorsque le quotient n'est plus divisible par 2, on teste avec 3
        \item Ainsi de suite avec les nombres premiers 
        \item L'algorithme se termine au plus tard avec le nombre premier qui précède N.
    \end{itemize}
    \exercice
    Décomposer $396$ en produit de facteurs premiers.
    \correction
    $396 = 2 \times  198$\\
    $396 = 2 \times  2 \times  99$\\
    $396 = 2 \times  2 \times  3 \times  33$\\
    $396 = 2 \times  2 \times  3 \times  3 \times  11$\\
    {\bfseries Donc la décomposition en produit de facteurs premiers de $\mathbf{396}$ vaut $\mathbf{2 \times  2 \times  3 \times  3 \times  11 = 2^2\times 3^2\times 11}$}

\end{methode}

\begin{exemple}
    Décomposer $\num{3626}$ en produit de facteurs premiers.
    \correction
    $\num{3626}=\colorbox{red!30}{2}\times \num{1813}$\\
    $\num{3626}=\colorbox{red!30}{2}\times\colorbox{blue!30}{7}\times 259$\\
    $\num{3626}=\colorbox{red!30}{2}\times\colorbox{blue!30}{7}\times\colorbox{blue!30}{7}\times \colorbox{orange}{37}$

    \medskip
    d'où $\num{3626}=\colorbox{red!30}{2}\times $\colorbox{blue!30}{$7^2$}$\times $ \colorbox{orange}{$37$}
\end{exemple}
\begin{exemple}
    Décomposer $\num{504}$ en produit de facteurs premiers.
    \correction
    $504=\colorbox{red!30}{2}\times 252$ car 504 est pair donc divisible par 2!\\
    $504=\colorbox{red!30}{2}\times\colorbox{red!30}{2}\times 126$ car 252 est pair!\\
    $504=\colorbox{red!30}{2}\times\colorbox{red!30}{2}\times\colorbox{red!30}{2}\times 63$ car 126 est pair!\\
    $504=\colorbox{red!30}{2}\times\colorbox{red!30}{2}\times\colorbox{red!30}{2}\times\colorbox{blue!30}{3}\times 21$ car 63 est multiple de 3\\
    $504=\colorbox{red!30}{2}\times\colorbox{red!30}{2}\times\colorbox{red!30}{2}\times\colorbox{blue!30}{3}\times\colorbox{blue!30}{3}\times \colorbox{green!30}{7}$
    
    \medskip
    Il ne reste qu'à écrire la décomposition en produit de facteurs premiers de $504$

    \medskip
    $504=\colorbox{red!30}{2}\times \colorbox{red!30}{2}\times \colorbox{red!30}{2}\times \colorbox{blue!30}{3}\times \colorbox{blue!30}{3}\times \colorbox{green!30}{7}$

    \medskip
    Puis on utilise les notations puissances : $504=$\colorbox{red!30}{$2^{3}$}$\times $\colorbox{blue!30}{$3^2$}$\times \colorbox{green!30}{7}$
\end{exemple}
\begin{exemple}
    Décomposer $\num{4680}$ en produit de facteurs premiers.
    \correction
    $\num{4680}=\colorbox{red!30}{2}\times \num{2340}$\\
    $\num{4680}=\colorbox{red!30}{2}\times\colorbox{red!30}{2}\times \num{1170}$\\
    $\num{4680}=\colorbox{red!30}{2}\times\colorbox{red!30}{2}\times\colorbox{red!30}{2}\times 585$\\
    $\num{4680}=\colorbox{red!30}{2}\times\colorbox{red!30}{2}\times\colorbox{red!30}{2}\times\colorbox{blue!30}{3}\times 195$\\
    $\num{4680}=\colorbox{red!30}{2}\times\colorbox{red!30}{2}\times\colorbox{red!30}{2}\times\colorbox{blue!30}{3}\times\colorbox{blue!30}{3}\times 65$\\
    $\num{4680}=\colorbox{red!30}{2}\times\colorbox{red!30}{2}\times\colorbox{red!30}{2}\times\colorbox{blue!30}{3}\times\colorbox{blue!30}{3}\times\colorbox{orange}{5}\times \colorbox{green!30}{13}$\\

    \medskip
    d'où $\num{4680} =$\colorbox{red!30}{$2^3$}$\times$\colorbox{blue!30}{$3^2$}$\times$\colorbox{orange}{$5$}$\times \colorbox{green!30}{13}$
\end{exemple}