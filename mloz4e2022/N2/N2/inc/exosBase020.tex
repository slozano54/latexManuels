\begin{exercice*}
    Un fleuriste dispose de $93$ iris et de $217$ roses.

    Il veut, en utilisant toutes ses fleurs, réaliser un maximum de bouquets contenant tous le même nombre d'iris et le même nombre de roses. 
    \begin{enumerate}
        \item Quel est le nombre maximal de bouquets ?
        \item Quel est le nombre d'iris dans chaque bouquet ?
        \item Quel est le nombre de roses dans chaque bouquet ?
    \end{enumerate}
\end{exercice*}
\begin{corrige}
    %\setcounter{partie}{0} % Pour s'assurer que le compteur de \partie est à zéro dans les corrigés
    % \phantom{rrr} 
    \begin{enumerate}
        \item 
        \begin{itemize}
            \item Les diviseurs de $93$ sont : $1$, $3$, $31$, $93$.
            \item Les diviseurs de $217$ sont : $1$, $7$, $31$, $217$.
            \item $31$ est le plus grand nombre qui divise à la fois $93$ et $217$.
        \end{itemize}
        Le nombre maximal de bouquets est donc : $31$.

        \item $93 \div 31 = 3$
     
        Le nombre d'iris dans chaque bouquet est : $3$
        \item $217 \div 31 = 7$
    
        Le nombre de roses dans chaque bouquet est : $7$
    \end{enumerate}
       
\end{corrige}

