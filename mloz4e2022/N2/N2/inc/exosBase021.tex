\begin{exercice*}[Chinoiseries]
    \begin{enumerate}
        \item Décomposer les nombres $162$ et $108$ en produits de facteurs premiers.
        \item Déterminer deux diviseurs communs à $162$ et $108$, plus grands que $10$.
        \item Un snack vend des barquettes composées de nems et de samossas. Le cuisinier a préparé $162$ nems et $108$ samossas.
        
        Dans chaque barquette :
        \begin{list}{$\bullet$}{}
            \item Le nombre de nems doit être le même.
            \item Le nombre de samossas doit être le même.
        \end{list}
        Tous les nems et tous les samossas doivent être utilisés.
        \begin{enumerate}
            \item Déterminer si le cuisinier peut constituer $36$ barquettes. Justifier.
            \item Déterminer le nombre maximal de barquettes qu'il peut réaliser.
            \item Déterminer alors le nombre de nems et de samossas de chaque barquette.
        \end{enumerate}        
    \end{enumerate}
\end{exercice*}
\begin{corrige}
    %\setcounter{partie}{0} % Pour s'assurer que le compteur de \partie est à zéro dans les corrigés
    % \phantom{rrr} 
    \begin{enumerate}
        \item $162=2\times 3\times 3\times 3\times 3$ et $108 = 2\times 2\times 3\times 3\times 3$.
        \item Deux diviseurs communs à $162$ et $108$, plus grands que $10$ : $27$ et $54$.
        \item 
        \begin{enumerate}
            \item $108 = 36\times 3$ donc $36$ divise $108$ mais $162 = 4\times 36 + 18$ donc $36$ ne divise pas $162$
            
            Le cuisinier ne peut donc pas constituer $36$ barquettes.
            \item Le plus grand nombre qui divise à la fois $162$ et $108$ vaut $2\times 3\times 3\times 3 = 54$ donc le nombre maximal de barquettes
            que le cuisinier peut réaliser vaut $54$.
            \item $162 = 3\times 54$ et $108 = 2\times 54$ donc chaque barquette contiendra $3$ nems et $2$ samossas.
        \end{enumerate}        
    \end{enumerate}     
\end{corrige}

