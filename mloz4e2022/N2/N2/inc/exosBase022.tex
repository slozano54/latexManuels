\begin{exercice*}[Mosaïque]
    Carole souhaite réaliser une mosaïque sur un mur de sa maison. La surface à recouvrir est rectangulaire.
    Elle fait \Lg{126} de largeur et \Lg{225} de hauteur et doit être entièrement recouverte par la faïence.
    Les carreaux de faîence sont des carrés qui tous la même dimension et il ne doivent pas être découper à la pose.    
    \begin{enumerate}
        \item Déterminer si Carole peut utiliser des carreaux de \Lg{3} de côté.
        \item Déterminer si Carole peut utiliser des carreaux de \Lg{6} de côté.
        \item Déterminer la dimension maxiamle des carreaux que Carole peut utiliser.
        \item Déterminer alors combien il lui faudra de carreaux.
    \end{enumerate}
    \hrefMathalea{https://coopmaths.fr/mathalea.html?ex=3A12-1,s=,n=3,i=0&v=l}
\end{exercice*}
\begin{corrige}
    %\setcounter{partie}{0} % Pour s'assurer que le compteur de \partie est à zéro dans les corrigés
    % \phantom{rrr} 
    Carole souhaite réaliser une mosaïque sur un mur de sa maison. La surface à recouvrir est rectangulaire.
    Elle fait \Lg{126} de largeur et \Lg{225} de hauteur et doit être entièrement recouverte par la faïence.
    Les carreaux de faîence sont des carrés qui tous la même dimension et il ne doivent pas être découper à la pose.

    \begin{enumerate}
        \item Déterminer si Carole peut utiliser des carreaux de \Lg{3} de côté.
        \item Déterminer si Carole peut utiliser des carreaux de \Lg{6} de côté.
        \item Déterminer la dimension maxiamle des carreaux que Carole peut utiliser.
        \item Déterminer alors combien il lui faudra de carreaux.
    \end{enumerate}
\end{corrige}

