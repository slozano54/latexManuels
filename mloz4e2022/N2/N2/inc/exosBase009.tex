\begin{exercice*}
    Ce matin, Valère a récolté $60$ laitues et $330$ carottes. Afin de les vendre au marché, il veut
constituer des lots identiques en utilisant toutes ces denrées.
    \begin{enumerate}
        \item Justifier que l'on peut réaliser $6$ lots.
        \item Justifier que l'on ne peut pas réaliser $12$ lots.
        \item Recopier et compléter la phrase suivante : \og Le nombre de lots doit être un diviseur \dots\dots~de \dots et \dots .\fg
        \item Trouver tous les nombres de lots qu'il peut réaliser.
    \end{enumerate}

    \hrefMathalea[\emoji{star-struck} \emoji{link} - Recherche de diviseurs communs]{https://coopmaths.fr/mathalea.html?ex=4A12,s=1,n=1,cd=1,i=0&v=l}
\end{exercice*}
\begin{corrige}
    %\setcounter{partie}{0} % Pour s'assurer que le compteur de \partie est à zéro dans les corrigés
    % \phantom{rrr}    
    \begin{enumerate}
        \item $60 = 10\times 6$ et $330 = 55\times 6$, donc il peut faire $6$ lots identiques.
        \item $60 = 5\times 12$ et $330 = 27\times 12 + 6$, donc il ne peut pas faire $12$ lots identiques.        
        \item \og Le nombre de lots est un diviseur \textbf{commun} de $\mathbf{60}$ et $\mathbf{330}$ .\fg
        \item Les diviseurs communs de $60$ et $330$ sont : $1$ ; $2$ ; $3$ ; $5$ ; $6$ ; $10$ ; $15$ ; $30$.
        
        Il peut donc réaliser $1$ ; $2$ ; $3$ ; $5$ ; $6$ ; $10$ ; $15$ ou $30$ lot(s) identique(s).
    \end{enumerate}
\end{corrige}

