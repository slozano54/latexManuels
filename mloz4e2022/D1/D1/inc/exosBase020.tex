\begin{exercice*}
    Lisa et Aymeric ont chacun un scooter. Ils doivent rejoindre leurs copains à la piscine qui est à \Lg[km]{8} de chez eux.
    \begin{enumerate}
        \item Lisa roule en moyenne à \Vitesse[kmh]{40}. Déterminer le temps, en minutes, qu'elle mettra pour aller à la piscine.
        \item Aymeric est plus pressé, il roule en moyenne à \Vitesse[kmh]{48}. Calculer, en minutes, le temps qu’il mettra pour retrouver ses copains à la piscine.
        \item Déterminer le qu'Aymeric a gagné par rapport à Lisa.
    \end{enumerate}
\end{exercice*}
\begin{corrige}
    %\setcounter{partie}{0} % Pour s'assurer que le compteur de \partie est à zéro dans les corrigés
    %\phantom{rrr}    
    Lisa et Aymeric ont chacun un scooter. Ils doivent rejoindre leurs copains à la piscine qui est à \Lg[km]{8} de chez eux.
    \begin{enumerate}
        \item Lisa roule en moyenne à \Vitesse[kmh]{40}. Déterminer le temps, en minutes, qu'elle mettra pour aller à la piscine.
        \par\textcolor{red}{À \Vitesse[kmh]{40} Lisa parcourt \Lg[km]{40} en 1 h soit 60 min, \Lg[km]{8} sont $\dfrac{1}{5}$ de \Lg[km]{40}, donc il faudra $\dfrac{1}{5}$ d'1h soit 12 min.}
        \item Aymeric est plus pressé, il roule en moyenne à \Vitesse[kmh]{48}. Calculer, en minutes, le temps qu’il mettra pour retrouver ses copains à la piscine.
        \par\textcolor{red}{Avec le même raisonnement \Lg[km]{8} sont $\dfrac{1}{6}$ de \Lg[km]{48} donc il faudra $\dfrac{1}{6}$ d'heure soit 10 min.}
        \item Déterminer le qu'Aymeric a gagné par rapport à Lisa.
        \par\textcolor{red}{12 min -10min = 2 min donc Aymeric a gagné 2 min.}
    \end{enumerate}
\end{corrige}

