\begin{exercice*}
    Voici les valeurs (en \Lg[m]{}) des lancers de poids réalisés par les 11 finalistes aux JO de 2008 :\par\smallskip
    \num{20.06}\hfill\num{20.53}\hfill\num{21.09}\hfill\num{19.67}\hfill\num{20.98}\hfill\num{20.42}\par\smallskip
    \num{21.51}\hfill\num{21.04}\hfill\num{20.41}\hfill\num{20.63}\hfill\num{21.05}\par\bigskip
    Calculer le pourcentage des lanceurs qui ont lancé au delà des \Lg[m]{21}.
\end{exercice*}
\begin{corrige}
    %\setcounter{partie}{0} % Pour s'assurer que le compteur de \partie est à zéro dans les corrigés
    %\phantom{rrr}    
    Voici les valeurs (en \Lg[m]{}) des lancers de poids réalisés par les 11 finalistes aux JO de 2008 :\par\smallskip
    \num{20.06}\hfill\num{20.53}\hfill\num{21.09}\hfill\num{19.67}\hfill\num{20.98}\hfill\num{20.42}\par\smallskip
    \num{21.51}\hfill\num{21.04}\hfill\num{20.41}\hfill\num{20.63}\hfill\num{21.05}\par\bigskip
    Calculer le pourcentage des lanceurs qui ont franchi les \Lg[m]{21}.
    \par\medskip
    \textcolor{red}{%
    4 lanceurs sur 11 ont lancé au delà de \Lg[m]{21}.\\
    $4\div 11 \simeq \num{0.36}$\\
    36\% de lanceur ont donc franchi \Lg[m]{21}.%
    }
\end{corrige}

