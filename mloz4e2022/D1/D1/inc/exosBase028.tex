\begin{exercice*}
    F11ex2
    
    Un commerçant vend des tee-shirts à \Prix{5} l'unité. Les cinq derniers jours du mois de juillet, il lance une
    promotion de fin de saison : il vend ces tee-shirts par 3, au prix de \Prix{12} le lot.
    \begin{enumerate}
    \item Complète le tableau suivant.
    \par\smallskip
    \begin{tabular}{|>{\centering\arraybackslash\columncolor{LightGray}}m{0.3\linewidth}|*{7}{c|}}
        \hline
        Nombre de tee-shirts&\num{1}&\num{2}&\num{3}&\num{4}&\num{5}&\num{6}&\num{7}\\\hline
        Prix normal&&&&&&&\\\hline
        Prix soldé&&&&&&&\\\hline
    \end{tabular}
    \par\smallskip
    \item Sur le papier millimétré ci-dessous, trace un repère dans lequel 1 cm en abscisse représente un tee-shirt,
    et 1 cm en ordonnée représente \Prix{5}.
    \par\smallskip
    \begin{tikzpicture}[scale=1]
        \draw (0,0) rectangle (9,7);
        % Fond
        \papierMillimetre
    \end{tikzpicture}
    \item Place, en bleu, les points correspondant à la situation normale et, en vert, les points correspondant à la situation des soldes.
    \item Que remarques-tu ?
\end{enumerate}
\end{exercice*}
\begin{corrige}
    %\setcounter{partie}{0} % Pour s'assurer que le compteur de \partie est à zéro dans les corrigés
    %\phantom{rrr}    
    \dots
\end{corrige}

