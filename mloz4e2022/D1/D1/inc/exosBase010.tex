\begin{exercice*}
    Un drôle d'épicier utilise le graphique ci-dessous pour indiquer le prix des oranges à ses clients.\par
    Déterminer le prix d'un kilogramme d'oranges.
    \begin{center}
    \scalebox{0.5}{
    \Fonction[%
        Calcul=8*x/5,%
        Epaisseur=1.5,%
        Trace,%
        CouleurTrace=bleu,%
        Xmin=0,Xmax=5,Xstep=2,%
        Ymin=0,Ymax=5,Ystep=4,%
        Origine={(0,0)},%
        Grille,PasGrilleX=0.5,PasGrilleY=0.5,%
        Graduations,PasGradY=4,PasGradX=2,%
        Bornea=0,Borneb=11,%
        % Pour placer les labels sur les axes
        Traces={
            label.rt(btex \Masse[kg]{} etex,placepoint(11,0));
            label.top(btex \Prix{} etex,placepoint(0,22));
        },%
    ]{}
    }
    \end{center}
    \smallskip
    \hrefMathalea{https://coopmaths.fr/alea/?uuid=c668a&id=4P10-1&alea=Yqk1&v=eleve&es=021100}    
\end{exercice*}
\begin{corrige}
    %\setcounter{partie}{0} % Pour s'assurer que le compteur de \partie est à zéro dans les corrigés
    %\phantom{rrr}    
    Un drôle d'épicier utilise le graphique ci-dessous pour indiquer le prix des oranges à ses clients.\par
    Déterminer le prix d'un kilogramme d'oranges.    \begin{center}
    \scalebox{0.5}{
    \Fonction[%
        Calcul=8*x/5,%
        Epaisseur=1.5,%
        Trace,%
        CouleurTrace=bleu,%
        Xmin=0,Xmax=5,Xstep=2,%
        Ymin=0,Ymax=5,Ystep=4,%
        Origine={(0,0)},%
        Grille,PasGrilleX=0.5,PasGrilleY=0.5,%
        Graduations,PasGradY=4,PasGradX=2,%
        Bornea=0,Borneb=11,%
        % Pour placer les labels sur les axes
        Traces={
            label.rt(btex \Masse[kg]{} etex,placepoint(11,0));
            label.top(btex \Prix{} etex,placepoint(0,22));
        },%
    ]{}
    }
    \end{center}
    \smallskip
    {\color{red} \Masse[kg]{5} d'oranges coutent \Prix{8} donc \Masse[kg]{1} coute $8\div 5= \Prix{1.60}$.}
\end{corrige}

