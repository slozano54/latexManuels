\begin{exercice*}
    Un bouquet de cinq jonquilles coute \Prix{4.50}.\par
    Calculer le prix d'un bouquet de sept jonquilles en utilisant le tableau ci-dessous.
    \begin{center}
        \Propor[Math,Stretch=1.2,GrandeurA=Nombre de jonquilles,GrandeurB=Prix en \Prix{},CouleurTab=LightGray]{5/\num{4.50},7/\textcolor{OrangeRed}{$p$ ?}}        
    \end{center}
\end{exercice*}
\begin{corrige}
    %\setcounter{partie}{0} % Pour s'assurer que le compteur de \partie est à zéro dans les corrigés
    %\phantom{rrr}    
    Un bouquet de cinq jonquilles coute \Prix{4.50}.\par
    Calculer le prix d'un bouquet de sept jonquilles en utilisant le tableau ci-dessous.
    \begin{center}
        \Propor[Math,Stretch=1.2,GrandeurA=Nombre de jonquilles,GrandeurB=Prix en \Prix{},CouleurTab=LightGray]{5/\num{4.50},7/\textcolor{OrangeRed}{$p$ ?}}
        \FlechePCB{2}{1}
    \end{center}
    {\color{red}
        On détermine la quatrième proportionnelle à l'aide des produits en croix.
    }
    \Coupe
    {\color{red}
        \begin{align*}
            \dfrac{5}{\num{4.50}}                       &=\dfrac{7}{\color{OrangeRed} p}&\\
            5\times {\color{OrangeRed} p}               &= \num{4.50}\times 7           &\text{\tiny\bfseries\color{red}Les produits en croix sont égaux.}\\
            \dfrac{5\times {\color{OrangeRed} p}}{5}    &= \dfrac{\num{4.50}\times 7}{5}&\text{\tiny\bfseries\color{red}On divise les deux membres par 5.}\\
            {\color{OrangeRed} p}                       &=\num{6.30}                    &\text{\tiny\bfseries\color{red}On simplifie et on calcule.}
        \end{align*}
        \par\smallskip
        Sept jonquilles coûtent donc \Prix{6.30}.
    }
\end{corrige}

