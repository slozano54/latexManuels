\begin{exercice*}
    En 2010, l’UNESCO\footnote{UNESCO (United Nations Educational, Scientific and Cultural Organization) : Organisation des Nations Unies pour l’Éducation, la Science et la Culture}
    a dressé un inventaire des langues en danger dans le monde. Il vise à susciter une prise de conscience sur la nécessité de préserver une diversité linguistique mondiale.
    Voici un tableau récapitulatif du nombre de langues en voie de disparition ou déjà éteintes.
    \par\smallskip
    \begin{tabular}{|>{\centering\arraybackslash\columncolor{LightGray}}m{0.2\linewidth}|*{3}{>{\centering\arraybackslash}m{0.2\linewidth}|}}
        \hline
        Niveau de vitalité & En voie de disparition & Déjà éteintes & \cellcolor{LightGray}Total\\\hline
        Nombre de langues&&231&\num{2580}\\\hline
    \end{tabular}
    \begin{enumerate}
        \item Sur \num{6 000} langues répertoriées, 43 \% sont soit en voie de disparition, soit déjà éteintes.
        Montrer, par un calcul, que cela représente un total de \num{2 580} langues.
        \item En déduire le nombre de langues qui sont en voie de disparition.
        \item Calculer le pourcentage de langues qui sont déjà éteintes sur les \num{6 000} langues répertoriées dans le monde.
    \end{enumerate}
\end{exercice*}
\begin{corrige}
    %\setcounter{partie}{0} % Pour s'assurer que le compteur de \partie est à zéro dans les corrigés
    %\phantom{rrr}    
    En 2010, l’UNESCO\footnote{UNESCO (United Nations Educational, Scientific and Cultural Organization) : Organisation des Nations Unies pour l’Éducation, la Science et la Culture}
    a dressé un inventaire des langues en danger dans le monde. Il vise à susciter une prise de conscience sur la nécessité de préserver une diversité linguistique mondiale.
    Voici un tableau récapitulatif du nombre de langues en voie de disparition ou déjà éteintes.
    \par\smallskip
    \begin{tabular}{|>{\centering\arraybackslash\columncolor{LightGray}}m{0.2\linewidth}|*{3}{>{\centering\arraybackslash}m{0.2\linewidth}|}}
        \hline
        Niveau de vitalité & En voie de disparition & Déjà éteintes & \cellcolor{LightGray}Total\\\hline
        Nombre de langues&\textcolor{red}{\num{2349}}&231&\num{2580}\\\hline
    \end{tabular}
    \Coupe
    \begin{enumerate}
        \item Sur \num{6 000} langues répertoriées, 43 \% sont soit en voie de disparition, soit déjà éteintes.
        Montrer, par un calcul, que cela représente un total de \num{2 580} langues.
        \par\textcolor{red}{$\num{0.43}\times\num{6000}=\num{2580}$, cela représente donc bien un total de \num{2580} langues.
        }
    % \end{enumerate}
    
    % \begin{enumerate}
    %     \setcounter{enumi}{1}
        \item En déduire le nombre de langues qui sont en voie de disparition.
        \par\textcolor{red}{$\num{2580}-231=\num{2349}$ donc \num{2349} langues sont en voie de disparition.}
        \item Calculer le pourcentage de langues qui sont déjà éteintes sur les \num{6 000} langues répertoriées dans le monde.
        \par\textcolor{red}{$231\div \num{6000} = \num{0.0385} = \num{3.85} \%$ donc \num{3.85} \% des langues répertoriées dans le monde sont éteintes.}
    \end{enumerate}
\end{corrige}

