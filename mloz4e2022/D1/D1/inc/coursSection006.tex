\section{Grandeurs composées}
% \subsection{Vitesse moyenne}
% \proprNum{(admise)}{l'on a parcouru une distance $d$ pendant un temps $t$}{sur ce parcours, {\bf la vitesse moyenne} $v$ est le quotient de la distance parcourue $d$ par le temps $t$ du trajet.
% $$v=\frac{d}{t}$$
% }

% \Remarques[Remarque]{
% L'unit\'e de vitesse d\'epend des unit\'es choisies pour la distance et le temps.\\
% Par exemple, si on exprime la distance en {\bf kilom\`{e}tres} et le temps en {\bf heures} on obtiendra une vitesse en {\bf kilom\`{e}tre par heure}.\\
% Citer d'autres exemples \ldots
% }

% \Exemples[Application]{}{
% En parcourant $120\,km$ en 2 heures, la vitesse moyenne est $\dfrac{120}2=60\,km/h$ ou $60\,km.h^{-1}$.

% En parcourant $30\,m$ en 5 secondes, la vitesse moyenne est $\dfrac{30}5=6\,m/s$ ou $6\,m.s^{-1}$.
% }

% \subsection{Débit}

% \subsection{Masse volumique}