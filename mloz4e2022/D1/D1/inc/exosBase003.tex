\begin{exercice*}
    Dire si les tableaux suivants sont de tableaux de proportionnalité. Justifier.
    \begin{enumerate}
        \item $\begin{array}{|*{3}{>{\centering\arraybackslash}m{0.15\linewidth}|}}
                    \hline
                    5 & 7 & 8\\\hline
                    1 & 3 & 4\\\hline
                \end{array}$\smallskip    
        \item $\begin{array}{|*{3}{>{\centering\arraybackslash}m{0.15\linewidth}|}}
                    \hline
                    56 & 72 & 48\\\hline
                    7 & 9 & 6\\\hline
                \end{array}$\smallskip    
        \item $\begin{array}{|*{3}{>{\centering\arraybackslash}m{0.15\linewidth}|}}
                    \hline
                    \num{9,5} & 6 & 1\\\hline
                    \num{17,5} & 14 & 9\\\hline
                \end{array}$\smallskip    
        \item $\begin{array}{|*{3}{>{\centering\arraybackslash}m{0.15\linewidth}|}}
                    \hline
                    8 & 5 & 9\\\hline
                    40 & 25 & 45\\\hline
                \end{array}$    
    \end{enumerate}
    \hrefMathalea{https://coopmaths.fr/alea/?uuid=aa997&id=5P10&alea=pJPg&v=eleve&es=021100}    
\end{exercice*}
\begin{corrige}
    %\setcounter{partie}{0} % Pour s'assurer que le compteur de \partie est à zéro dans les corrigés
    %\phantom{rrr}
    Dire si les tableaux suivants sont de tableaux de proportionnalité. Justifier.\par
    {\red Pour déterminer si c'est un tableau de proportionnalité, il suffit de comparer les quotients d'un nombre de la première ligne par le nombre correspondant de la seconde ligne ou inversement.}\par    
    \begin{enumerate}
        \item $\begin{array}{|*{3}{>{\centering\arraybackslash}m{0.15\linewidth}|}}
            \hline
            5 & 7 & 8\\\hline
            1 & 3 & 4\\\hline
        \end{array}$\smallskip    
        \\ Soit $\dfrac{\textcolor{blue}{5}}{\textcolor{red}{1}}\neq\dfrac{\textcolor{blue}{7}}{\textcolor{red}{3}}\neq\dfrac{\textcolor{blue}{8}}{\textcolor{red}{4}}$, on constate qu'ils sont différents.
        \\Ou bien $\dfrac{\textcolor{red}{1}}{\textcolor{blue}{5}}\neq\dfrac{\textcolor{red}{3}}{\textcolor{blue}{7}}\neq\dfrac{\textcolor{red}{4}}{\textcolor{blue}{8}}$, on constate aussi qu'ils sont différents.
        \\{\bfseries \color{red}Ce n'est donc pas un tableau de proportionnalité.}
        \item $\begin{array}{|*{3}{>{\centering\arraybackslash}m{0.15\linewidth}|}}
            \hline
            56 & 72 & 48\\\hline
            7 & 9 & 6\\\hline
        \end{array}$\smallskip 
        \\ Soit $\dfrac{\textcolor{blue}{56}}{\textcolor{red}{7}} = \dfrac{\textcolor{blue}{72}}{\textcolor{red}{9}} = \dfrac{\textcolor{blue}{48}}{\textcolor{red}{6}}$, on constate qu'ils sont égaux.
        \\Ou bien $\dfrac{\textcolor{red}{7}}{\textcolor{blue}{56}} = \dfrac{\textcolor{red}{9}}{\textcolor{blue}{72}} = \dfrac{\textcolor{red}{6}}{\textcolor{blue}{48}}$, on constate aussi qu'ils sont égaux.
        \\{\bfseries \color{red}C'est donc un tableau de proportionnalité.}
        \item $\begin{array}{|*{3}{>{\centering\arraybackslash}m{0.15\linewidth}|}}
            \hline
            \num{9,5} & 6 & 1\\\hline
            \num{17,5} & 14 & 9\\\hline
        \end{array}$\smallskip    
        \\ Soit $\dfrac{\textcolor{blue}{9.5}}{\textcolor{red}{17.5}}\neq\dfrac{\textcolor{blue}{6}}{\textcolor{red}{14}}\neq\dfrac{\textcolor{blue}{1}}{\textcolor{red}{9}}$, on constate qu'ils sont différents.
        \\Ou bien $\dfrac{\textcolor{red}{9.5}}{\textcolor{blue}{17.5}}\neq\dfrac{\textcolor{red}{6}}{\textcolor{blue}{14}}\neq\dfrac{\textcolor{red}{1}}{\textcolor{blue}{9}}$, on constate aussi qu'ils sont différents.
        \\{\bfseries \color{red}Ce n'est donc pas un tableau de proportionnalité.}
    \end{enumerate}
    \Coupe
    \begin{enumerate}
        \setcounter{enumi}{3}
        \item $\begin{array}{|*{3}{>{\centering\arraybackslash}m{0.15\linewidth}|}}
            \hline
            8 & 5 & 9\\\hline
            40 & 25 & 45\\\hline
        \end{array}$\smallskip
        \\ Soit $\dfrac{\textcolor{blue}{8}}{\textcolor{red}{40}} = \dfrac{\textcolor{blue}{5}}{\textcolor{red}{25}} = \dfrac{\textcolor{blue}{9}}{\textcolor{red}{45}}$, on constate qu'ils sont égaux.
        \\Ou bien $\dfrac{\textcolor{red}{40}}{\textcolor{blue}{8}} = \dfrac{\textcolor{red}{25}}{\textcolor{blue}{5}} = \dfrac{\textcolor{red}{45}}{\textcolor{blue}{9}}$, on constate aussi qu'ils sont égaux.
        \\{\bfseries \color{red}C'est donc un tableau de proportionnalité.}
    \end{enumerate}
\end{corrige}

