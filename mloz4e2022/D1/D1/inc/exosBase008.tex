\begin{exercice*}
    Avec \num{75} bouteilles en plastique, on fabrique trois pulls en maille polaire.\par
    En complétant le tableau suivant, calculer le nombre de pulls fabriqués avec 825 bouteilles plastiques.
    \begin{center}
        \Propor[Math,Stretch=1.2,GrandeurA=Nombre de bouteilles,GrandeurB=Nombre de pulls,CouleurTab=LightGray]{/,/\textcolor{OrangeRed}{$n$ ?}}        
    \end{center}
\end{exercice*}
\begin{corrige}
    %\setcounter{partie}{0} % Pour s'assurer que le compteur de \partie est à zéro dans les corrigés
    %\phantom{rrr}    
    Avec \num{75} bouteilles en plastique, on fabrique trois pulls en maille polaire.\par
    En complétant le tableau suivant, calculer le nombre de pulls fabriqués avec 825 bouteilles plastiques.
    \begin{center}
        \Propor[Math,Stretch=1.2,GrandeurA=Nombre de bouteilles,GrandeurB=Nombre de pulls,CouleurTab=LightGray]{\textcolor{red}{$75$}/\textcolor{red}{$3$},\textcolor{red}{$825$}/\textcolor{OrangeRed}{$n$ ?}}        
        \FlechePCB{2}{1}
    \end{center}
    {\color{red}
    On détermine la quatrième proportionnelle à l'aide des produits en croix.
        \begin{align*}
            \dfrac{75}{3}                             &=\dfrac{825}{\color{OrangeRed} n}&\\
            75\times {\color{OrangeRed} n}            &= 3\times 825                    &\text{\tiny\bfseries\color{red}Les produits en croix sont égaux.}\\
            \dfrac{75\times {\color{OrangeRed} n}}{75}&= \dfrac{3\times 825}{75}        &\text{\tiny\bfseries\color{red}On divise les deux membres par 75.}\\
            {\color{OrangeRed} n}                     &=33                              &\text{\tiny\bfseries\color{red}On simplifie et on calcule.}
        \end{align*}
        \par\smallskip
        Avec 825 bouteilles, on peut donc faire 33 pulls.
    }
\end{corrige}

