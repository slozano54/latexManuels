\begin{exercice*}
    Parmi ces graphiques :
    \begin{multicols}{2}
        \begin{center}
            \scalebox{0.5}{
                \Fonction[%
                    Trace,% Pour avoir le repère et pouvoir y placer des points
                    Xmin=0,Xmax=5,Xstep=2,%
                    Ymin=0,Ymax=5,Ystep=2,%
                    Origine={(0,0)},%
                    Grille,PasGrilleX=0.5,PasGrilleY=0.5,%
                    Bornea=0,Borneb=0,% Pour ne pas voir la courbe mais quand même placer des points
                    % Pour placer des points
                    Traces={
                            marque_p:="croix";
                            pointe(placepoint(0,0),placepoint(1,1),placepoint(4,4),placepoint(9,9),);
                    },%
                ]{}
            }            
            {\sc Graphique n°1}
            \par\medskip\par
            \scalebox{0.5}{
                \Fonction[%
                    Trace,% Pour avoir le repère et pouvoir y placer des points
                    Xmin=0,Xmax=5,Xstep=2,%
                    Ymin=0,Ymax=5,Ystep=2,%
                    Origine={(0,0)},%
                    Grille,PasGrilleX=0.5,PasGrilleY=0.5,%
                    Bornea=0,Borneb=0,% Pour ne pas voir la courbe mais quand même placer des points
                    % Pour placer des points
                    Traces={
                            marque_p:="croix";
                            pointe(placepoint(0,3),placepoint(2,4),placepoint(6,6),placepoint(10,8),);
                    },%
                ]{}
            }
            {\sc Graphique n°2}
            \par\medskip\par
            \scalebox{0.5}{
                \Fonction[%
                    Trace,% Pour avoir le repère et pouvoir y placer des points
                    Xmin=0,Xmax=5,Xstep=2,%
                    Ymin=0,Ymax=5,Ystep=2,%
                    Origine={(0,0)},%
                    Grille,PasGrilleX=0.5,PasGrilleY=0.5,%
                    Bornea=0,Borneb=0,% Pour ne pas voir la courbe mais quand même placer des points
                    % Pour placer des points
                    Traces={
                            marque_p:="croix";
                            pointe(placepoint(0,0),placepoint(2,1),placepoint(4,4),placepoint(6,9),);
                    },%
                ]{}
            }
            {\sc Graphique n°3}
            \par\medskip\par
            \scalebox{0.5}{
                \Fonction[%
                    Trace,% Pour avoir le repère et pouvoir y placer des points
                    Xmin=0,Xmax=5,Xstep=2,%
                    Ymin=0,Ymax=5,Ystep=2,%
                    Origine={(0,0)},%
                    Grille,PasGrilleX=0.5,PasGrilleY=0.5,%
                    Bornea=0,Borneb=0,% Pour ne pas voir la courbe mais quand même placer des points
                    % Pour placer des points
                    Traces={
                            marque_p:="croix";
                            pointe(placepoint(2,1.5),placepoint(4,3),placepoint(6,4.5),placepoint(10,7.5),);
                    },%
                ]{}
            }
            {\sc Graphique n°4}
        \end{center}
    \end{multicols}
    \begin{enumerate}
        \item Lister ceux qui semblent représenter une situation de proportionnalité. Justifier.
        \item Lister ceux qui ne peuvent pas représenter une situation de proportionnalité. Justifier.
    \end{enumerate}
\end{exercice*}
\begin{corrige}
    %\setcounter{partie}{0} % Pour s'assurer que le compteur de \partie est à zéro dans les corrigés
    %\phantom{rrr}    
    Parmi ces graphiques :
    \begin{multicols}{2}
        \begin{center}
            \scalebox{0.5}{
                \Fonction[%
                    Trace,% Pour avoir le repère et pouvoir y placer des points
                    Xmin=0,Xmax=5,Xstep=2,%
                    Ymin=0,Ymax=5,Ystep=2,%
                    Origine={(0,0)},%
                    Grille,PasGrilleX=0.5,PasGrilleY=0.5,%
                    Bornea=0,Borneb=0,% Pour ne pas voir la courbe mais quand même placer des points
                    % Pour placer des points
                    Traces={
                            marque_p:="croix";
                            pointe(placepoint(0,0),placepoint(1,1),placepoint(4,4),placepoint(9,9),);
                    },%
                ]{}
            }            
            {\sc Graphique n°1}
            \par\medskip\par
            \scalebox{0.5}{
                \Fonction[%
                    Trace,% Pour avoir le repère et pouvoir y placer des points
                    Xmin=0,Xmax=5,Xstep=2,%
                    Ymin=0,Ymax=5,Ystep=2,%
                    Origine={(0,0)},%
                    Grille,PasGrilleX=0.5,PasGrilleY=0.5,%
                    Bornea=0,Borneb=0,% Pour ne pas voir la courbe mais quand même placer des points
                    % Pour placer des points
                    Traces={
                            marque_p:="croix";
                            pointe(placepoint(0,3),placepoint(2,4),placepoint(6,6),placepoint(10,8),);
                    },%
                ]{}
            }
            {\sc Graphique n°2}
            \par\medskip\par
            \scalebox{0.5}{
                \Fonction[%
                    Trace,% Pour avoir le repère et pouvoir y placer des points
                    Xmin=0,Xmax=5,Xstep=2,%
                    Ymin=0,Ymax=5,Ystep=2,%
                    Origine={(0,0)},%
                    Grille,PasGrilleX=0.5,PasGrilleY=0.5,%
                    Bornea=0,Borneb=0,% Pour ne pas voir la courbe mais quand même placer des points
                    % Pour placer des points
                    Traces={
                            marque_p:="croix";
                            pointe(placepoint(0,0),placepoint(2,1),placepoint(4,4),placepoint(6,9),);
                    },%
                ]{}
            }
            {\sc Graphique n°3}
            \par\medskip\par
            \scalebox{0.5}{
                \Fonction[%
                    Trace,% Pour avoir le repère et pouvoir y placer des points
                    Xmin=0,Xmax=5,Xstep=2,%
                    Ymin=0,Ymax=5,Ystep=2,%
                    Origine={(0,0)},%
                    Grille,PasGrilleX=0.5,PasGrilleY=0.5,%
                    Bornea=0,Borneb=0,% Pour ne pas voir la courbe mais quand même placer des points
                    % Pour placer des points
                    Traces={
                            marque_p:="croix";
                            pointe(placepoint(2,1.5),placepoint(4,3),placepoint(6,4.5),placepoint(10,7.5),);
                    },%
                ]{}
            }
            {\sc Graphique n°4}
        \end{center}
    \end{multicols}
    \Coupe
    \begin{enumerate}
        \item Lister ceux qui semblent représenter une situation de proportionnalité. Justifier.
        \par\textcolor{red}{Sur les grapiques 1 et 4, les points semblent alignés avec l'origine du repère, ce sont donc a priori des représentations de la proportionnalité.}
        \item Lister ceux qui ne peuvent pas représenter une situation de proportionnalité. Justifier.
        \par\textcolor{red}{Sur le grapiques 2, les points sont alignés mais pas avec l'origine, ce ne peut donc pas être une représentation de la proportionnalité.}
        \par\textcolor{red}{Sur les grapiques 3, les points ne sont pas alignés, ce ne peut donc pas être une représentation de la proportionnalité.}
    \end{enumerate}
\end{corrige}

