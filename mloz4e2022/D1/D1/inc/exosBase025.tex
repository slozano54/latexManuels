\begin{exercice*}
    \begin{enumerate}
        \item Déterminer le temps nécessaire au remplissage d'une baignoire de \Capa[L]{300} avec un robinet dont le débit 
        est de \Capa[L]{17} par minute. Arrondir à la seconde.
        \item Déterminer le débit de ce robinet pour remplir la baignoire en 8 minutes.
        \item Avec un robinet dont le débit est de \Capa[L]{25} par minute, déterminer le pourcentage du volume d'une piscine de \Vol[m]{12} que l'on peut remplir en 3 h.
    \end{enumerate}
\end{exercice*}
\begin{corrige}
    %\setcounter{partie}{0} % Pour s'assurer que le compteur de \partie est à zéro dans les corrigés
    %\phantom{rrr}    
    \begin{enumerate}
        \item Déterminer le temps nécessaire au remplissage d'une baignoire de \Capa[L]{300} avec un robinet dont le débit 
        est de \Capa[L]{17} par minute. Arrondir à la seconde.
        \par\textcolor{red}{$300\div 17\simeq\num{17.65}\text{ min}$ soit \Temps{;;;;17;39}, il faut donc environ \Temps{;;;;17;39} pour remplir la baignoire.}
        \item Déterminer le débit de ce robinet pour remplir la baignoire en 8 minutes.
        \par\textcolor{red}{$300\div 8 = \num{37.5}$ il faudrait donc un débit de \Capa[L]{37.5} par minute.}
        \item Avec un robinet dont le débit est de \Capa[L]{25} par minute, déterminer le pourcentage du volume d'une piscine de \Vol[m]{12} que l'on peut remplir en 3 h.
        \par\textcolor{red}{$25\times 180 = \Capa[L]{4500} = \Vol[m]{4.5}$ et $\num{4.5}\div 12 = \num{0.375}=\num{37.5}\%$, en 3 h on peut donc remplir \num{37.5}\% du volume de la piscine.}
    \end{enumerate}
\end{corrige}

