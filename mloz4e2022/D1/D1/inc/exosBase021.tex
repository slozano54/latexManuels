\begin{exercice*}
    Une moto roule à la vitesse de \Vitesse[kmh]{90}.
    \begin{enumerate}
        \item Déterminer la distance parcourue :
        \begin{enumerate}
            \item en 2h;
            \item en 4h 30;
        \end{enumerate}
        \item Déterminer le temps nécessaire pour parcourir :
        \begin{enumerate}
            \item \Lg[km]{450};
            \item \Lg[km]{600};
        \end{enumerate}
        \item Convertir en m/s la vitesse de cette moto.
    \end{enumerate}
\end{exercice*}
\begin{corrige}
    %\setcounter{partie}{0} % Pour s'assurer que le compteur de \partie est à zéro dans les corrigés
    %\phantom{rrr}    
    Une moto roule à la vitesse de \Vitesse[kmh]{90}.
    \begin{enumerate}
        \item Déterminer la distance parcourue :\\
        \begin{enumerate}
            \item en 2h; \par\textcolor{red}{En 2h, le double de distance soit \Lg[km]{180}.}
            \item en 4h 30; \par\textcolor{red}{En 4h30, c'est \num{4.5} fois ladistance en 1h soit \Lg[km]{405}.}
        \end{enumerate}
        \setcounter{enumi}{1}
        \item Déterminer le temps nécessaire pour parcourir :\\
        \begin{enumerate}
            \item \Lg[km]{450};\par\textcolor{red}{\Lg[km]{90} représente $\dfrac15$ de \Lg[km]{450} donc il faudra 5h.}
            \item \Lg[km]{600};\par\textcolor{red}{\Lg[km]{600} représente $\dfrac{600}{90}=\dfrac{20}{3}$ de \Lg[km]{90} donc il faudra $\dfrac{20}{3}$ d'heure soit 6h40.}
        \end{enumerate}
        \setcounter{enumi}{2}
        \item Convertir en m/s la vitesse de cette moto.\par\textcolor{red}{$\dfrac{90\text{ km}}{1 \text{ h}}=\dfrac{\num{90000}\text{ m}}{\num{3600} \text{ s}}$, donc \Vitesse[kmh]{90}=\Vitesse[ms]{25}. La moto roule à \Vitesse[ms]{25}.}
    \end{enumerate}
\end{corrige}

