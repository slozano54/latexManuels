\begin{exercice*}
    \phantom{rrr}\par
    \begin{minipage}{0.8\linewidth}
    Un bac à sable a \\
    une forme triangulaire,\\
    similaire à la figure ci-contre.\\\smallskip
    Les dimensions réelles du terrain sont :\\ 
    $DE = \Lg[m]{12}$, $EF = \Lg[m]{9}$, $DF = \Lg[m]{15}$.\\\smallskip
    On veut construire ce triangle à l'échelle $\frac{1}{200}$. Compléter le tableau ci-dessous.
    \end{minipage}
    \begin{minipage}{0.2\linewidth}
        \hspace*{-20mm}
        \begin{Geometrie}
            pair D,E,F;
            D=u*(1,4);
            E-D=u*(4,0);
            F-E=u*(0,-3);
            draw D--E--F--cycle;
            label.ulft(btex $D$ etex,D);
            label.urt(btex $E$ etex,E);
            label.lrt(btex $F$ etex,F);
        \end{Geometrie}
    \end{minipage}
    {\renewcommand{\arraystretch}{1.4}
    \begin{tabular}{|l|*{3}{>{\centering\arraybackslash}m{0.1\linewidth}|}l}
        \cline{2-4}
        \multicolumn{1}{l|}{}&\cellcolor{LightGray}$DE$&\cellcolor{LightGray}$EF$&\cellcolor{LightGray}$DF$&\\\cline{1-4}
        \cellcolor{LightGray}Dimensions réelles   & \Lg[m]{12} & \Lg[m]{9} & \Lg[m]{15} & \tikz[remember picture,overlay]{\coordinate[name=C,xshift=-\tabcolsep-\arrayrulewidth,yshift=\getstrut\dp];}\\\cline{1-4}
        \cellcolor{LightGray}Dimensions du dessin & \Lg[cm]{6} &           &            &\tikz[remember picture,overlay]{\coordinate[name=D,xshift=-\tabcolsep-\arrayrulewidth,yshift=\getstrut\dp];}\\\cline{1-4}
    \end{tabular}
    \par\medskip
    \tikzstyle{FlechePropor}=[-{Stealth[width=2mm]}]
    \begin{tikzpicture}[remember picture,overlay]%
        \draw[FlechePropor,out=-30,in=30] (C) to node[transform canvas={xshift=1pt},inner sep=0pt, inner xsep=1pt,fill=white, pos=0.5,right]{$\times\dots$}(D);%
        \coordinate[shift={(20mm,2mm)}] (n) at (D.east);
        \coordinate[shift={(0mm,-2mm)}] (d) at (D.south);
        \coordinate[shift={(0mm,2mm)}] (c) at (C.north);
        \draw (d) to node[transform canvas={xshift=40pt},inner sep=0pt, inner xsep=1pt,fill=white, pos=0.5,right]{$\times\dots$}(c);%
        \tikzmath{  
            \rx= 2.9;  % x radius of ellipse
            \ry= 0.6;  % y radius of ellipse
        };
        \draw[FlechePropor] (d) arc[start angle=-60, end angle=60, x radius=\rx, y radius=\ry];
    \end{tikzpicture}%
    }
\end{exercice*}
\begin{corrige}
    %\setcounter{partie}{0} % Pour s'assurer que le compteur de \partie est à zéro dans les corrigés
    %\phantom{rrr}    
    \phantom{rrr}\par
    \begin{minipage}{0.8\linewidth}
    Un bac à sable a \\
    une forme triangulaire,\\
    similaire à la figure ci-contre.\\\smallskip
    Les dimensions réelles du terrain sont :\\ 
    $DE = \Lg[m]{12}$, $EF = \Lg[m]{9}$, $DF = \Lg[m]{15}$.\\\smallskip
    On veut construire ce triangle à l'échelle $\frac{1}{200}$. Compléter le tableau ci-dessous.
    \end{minipage}
    \begin{minipage}{0.1\linewidth}
        \hspace*{-20mm}
        \begin{Geometrie}
            pair D,E,F;
            D=u*(1,4);
            E-D=u*(4,0);
            F-E=u*(0,-3);
            draw D--E--F--cycle;
            label.ulft(btex $D$ etex,D);
            label.urt(btex $E$ etex,E);
            label.lrt(btex $F$ etex,F);
        \end{Geometrie}
    \end{minipage}
    {\renewcommand{\arraystretch}{1.4}
    \begin{tabular}{|l|*{3}{>{\centering\arraybackslash}m{0.1\linewidth}|}l}
        \cline{2-4}
        \multicolumn{1}{l|}{}&\cellcolor{LightGray}$DE$&\cellcolor{LightGray}$EF$&\cellcolor{LightGray}$DF$&\\\cline{1-4}
        \cellcolor{LightGray}Dimensions réelles   & \Lg[m]{12} & \Lg[m]{9} & \Lg[m]{15} & \tikz[remember picture,overlay]{\coordinate[name=C,xshift=-\tabcolsep-\arrayrulewidth,yshift=\getstrut\dp];}\\\cline{1-4}
        \cellcolor{LightGray}Dimensions du dessin & \Lg[cm]{6} & {\color{red} \Lg[cm]{4.5}}          &  {\color{red} \Lg[cm]{7.5}}         &\tikz[remember picture,overlay]{\coordinate[name=D,xshift=-\tabcolsep-\arrayrulewidth,yshift=\getstrut\dp];}\\\cline{1-4}
    \end{tabular}
    \par\medskip
    \tikzstyle{FlechePropor}=[-{Stealth[width=2mm]}]
    \begin{tikzpicture}[remember picture,overlay]%
        \draw[FlechePropor,out=-30,in=30] (C) to node[transform canvas={xshift=1pt},inner sep=0pt, inner xsep=1pt,fill=white, pos=0.5,right]{$\times{\color{red}{\num{0.005}}}$}(D);%
        \coordinate[shift={(20mm,2mm)}] (n) at (D.east);
        \coordinate[shift={(0mm,-2mm)}] (d) at (D.south);
        \coordinate[shift={(0mm,2mm)}] (c) at (C.north);
        \draw (d) to node[transform canvas={xshift=40pt},inner sep=0pt, inner xsep=1pt,fill=white, pos=0.5,right]{$\times{\color{red}{\num{200}}}$}(c);%
        \tikzmath{  
            \rx= 2.9;  % x radius of ellipse
            \ry= 0.6;  % y radius of ellipse
        };
        \draw[FlechePropor] (d) arc[start angle=-60, end angle=60, x radius=\rx, y radius=\ry];
    \end{tikzpicture}%
    }
    {\color{red}{{\bfseries Remarque : } $\dfrac{1}{200}=\num{0.005}$}}
\end{corrige}

