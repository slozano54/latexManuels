\section{Quatrième proportionnelle}
\begin{definition}
    Connaissant trois valeurs $a$, $b$ et $c$, le nombre $x$ qui fait du tableau suivant, un tableau de proportionnalité est appelé \mbox{\textbf{une quatrième proportionnelle}.}\par\smallskip
    {\renewcommand{\arraystretch}{1.2}
        \begin{tabular}{|>{\columncolor{LightGray}}m{0.15\linewidth}|>{\centering\arraybackslash}m{0.1\linewidth}|>{\centering\arraybackslash}m{0.1\linewidth}|}
            \hline
            Grandeur n° 1&$a$&$c$\\
            \hline
            Grandeur n° 2&$b$&{\bfseries\color{OrangeRed}$x$ ?}\\
            \hline            
        \end{tabular}
    }
\end{definition}

\begin{remarque}
    Les quatre tableaux suivants sont équivalents.\par\smallskip
    {\renewcommand{\arraystretch}{1.2}
        \begin{tabular}{|>{\columncolor{LightGray}}m{0.15\linewidth}|>{\centering\arraybackslash}m{0.1\linewidth}|>{\centering\arraybackslash}m{0.1\linewidth}|}
            \hline
            Grandeur n° 1&$a$&$c$\\
            \hline
            Grandeur n° 2&$b$&{\bfseries\color{OrangeRed}$x$ ?}\\
            \hline            
        \end{tabular}
        \hspace*{7mm}
        \begin{tabular}{|>{\columncolor{LightGray}}m{0.2\linewidth}|>{\centering\arraybackslash}m{0.1\linewidth}|>{\centering\arraybackslash}m{0.1\linewidth}|}
            \hline
            Grandeur n° 2&$b$&{\bfseries\color{OrangeRed}$x$ ?}\\
            \hline
            Grandeur n° 1&$a$&$c$\\            
            \hline            
        \end{tabular}
        \par\smallskip
        \begin{tabular}{|>{\columncolor{LightGray}}m{0.15\linewidth}|>{\centering\arraybackslash}m{0.1\linewidth}|>{\centering\arraybackslash}m{0.1\linewidth}|}
            \hline
            Grandeur n° 1&$c$&$a$\\
            \hline
            Grandeur n° 2&{\bfseries\color{OrangeRed}$x$ ?}&$b$\\
            \hline            
        \end{tabular}
        \hspace*{7mm}
        \begin{tabular}{|>{\columncolor{LightGray}}m{0.2\linewidth}|>{\centering\arraybackslash}m{0.1\linewidth}|>{\centering\arraybackslash}m{0.1\linewidth}|}
            \hline
            Grandeur n° 2&{\bfseries\color{OrangeRed}$x$ ?}&$b$\\
            \hline
            Grandeur n° 1&$c$&$a$\\            
            \hline            
        \end{tabular}
    }
\end{remarque}

\begin{remarque}
    \textbf{La quatrième proportionnelle} $\color{OrangeRed} x$ est donc aussi le nombre qui rend vraie l'une des égaliés suivantes.

    $\dfrac{a}{b}=\dfrac{c}{\color{OrangeRed} x}$\hfill$\dfrac{b}{a}=\dfrac{\color{OrangeRed} x}{c}$\hfill$\dfrac{c}{\color{OrangeRed} x}=\dfrac{a}{b}$\hfill$\dfrac{\color{OrangeRed} x}{c}=\dfrac{b}{a}$
\end{remarque}

\begin{exemple*1}
    Lisa lit sur sa recette de mousse au chocolat qu'il faut \Masse[g]{108} de beurre pour six personnes .
    Déterminer la masse de beurre nécessaire pour cinq personnes.
    \correction    
    \begin{itemize}
        \item On regroupe les données dans un tableau de proportionnalité.\par\smallskip
        {\renewcommand{\arraystretch}{1.2}
            \begin{tabular}{|>{\columncolor{LightGray}}m{0.25\linewidth}|>{\centering\arraybackslash}m{0.1\linewidth}|>{\centering\arraybackslash}m{0.1\linewidth}|}
                \hline
                Nombre de personnes &$6$&$5$\\
                \hline
                Masse de beurre en \Masse[g]{}&$108$&{\bfseries\color{OrangeRed}$x$ ?}\\
                \hline            
            \end{tabular}
        }\par\smallskip
        \item On détermine la quatrième proportionnelle à l'aide des produits en croix.
        \begin{align*}
            \dfrac{6}{108}                              &=\dfrac{5}{\color{OrangeRed} x}&\\
            6\times {\color{OrangeRed} x}               &= 108\times 5                  &\text{\bfseries \color{OrangeRed}Les produits en croix sont égaux.}\\
            \dfrac{6\times {\color{OrangeRed} x}}{6}    &= \dfrac{108\times 5}{6}       &\text{\bfseries \color{OrangeRed}On divise les deux membres par 6.}\\
            {\color{OrangeRed} x}                       &=\num{90}                      &\text{\bfseries \color{OrangeRed}On simplifie et on calcule.}
            \end{align*}
        \item Il faut donc \Masse[g]{90} de beurre pour cinq personnes.
    \end{itemize}
\end{exemple*1}

\begin{exemple*1}
    Un fichier de \Octet[Mo]{225} est téléchargé en \Temps{;;;;54}.\par\smallskip
    Déterminer le temps nécessaire au téléchargement d'un fichier de \Octet[Mo]{850} dans les mêmes conditions.
    \correction
    Pour être dans une situation de proportionnalité, il convient de supposer que le débit de la connexion est constant.
    \begin{itemize}
        \item On regroupe les données dans un tableau de proportionnalité.\par\smallskip
        {\renewcommand{\arraystretch}{1.2}
            \begin{tabular}{|>{\columncolor{LightGray}}m{0.45\linewidth}|>{\centering\arraybackslash}m{0.1\linewidth}|>{\centering\arraybackslash}m{0.1\linewidth}|}
                \hline
                Durée de téléchargement en secondes &$54$&{\bfseries\color{OrangeRed}$x$ ?}\\
                \hline
                Taille du fichier en \Octet[Mo]{}&$225$&$850$\\
                \hline            
            \end{tabular}
        }\par\smallskip
        \item On détermine la quatrième proportionnelle à l'aide des produits en croix.
        \begin{align*}
            \dfrac{54}{225}                             &=\dfrac{\color{OrangeRed} x}{850}&\\
            225\times {\color{OrangeRed} x}             &= 54\times 850                   &\text{\bfseries \color{OrangeRed}Les produits en croix sont égaux.}\\
            \dfrac{225\times {\color{OrangeRed} x}}{225}&= \dfrac{54\times 850}{225}      &\text{\bfseries \color{OrangeRed}On divise les deux membres par 6.}\\
            {\color{OrangeRed} x}                       &=\num{204}                       &\text{\bfseries \color{OrangeRed}On simplifie et on calcule.}
            \end{align*}
        \item Il faut donc \Temps{;;;;204} soit \Temps{;;;;3;24} pour télécharger un fichier de \Octet[Mo]{850}.
    \end{itemize}
\end{exemple*1}

\begin{remarque}
    Pour le premier exemple, le calcul de la quantité de beurre pour une personne est possible et permet aussi l'obtention de la quantité pour cinq personne.\par
    Par contre, ce calcul s'avère délicat pour le second exemple. On a alors tout intérêt à calculer une quatrième proportionnelle.
\end{remarque}


