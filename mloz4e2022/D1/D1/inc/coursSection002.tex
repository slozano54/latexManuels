\section{Tableaux}
\begin{definition}
    Un tableau en ligne traduit une \textbf{situation de proportionnalité} lorsqu'on passe d'une ligne à l'autre en multipliant toujours par le même nombre.\\
    Ces deux "multiplicateurs" s'appellent \textbf{coefficient de proportionnalité}.
\end{definition}

\begin{remarque}
    \textit{L'un des coefficients est supérieur à 1 et l'autre et inférieur à 1.}\par
    Que dire de leur produit?
\end{remarque}

\begin{exemple*1}
    On veut compléter le tableau de proportionnalité ci-dessous en utilisant l'égalité des produits en croix.
    \begin{center}
        \begin{tabular}{|c|c|c|}
            \hline
            14&21&b\\
            \hline
            8&a&28\\
            \hline
        \end{tabular}
    \end{center}

    \correction
    Ce tableau est un tableau de proportionnalité donc :
    \begin{align*}
        \dfrac{14}{8}           &=\dfrac{21}{a}             &\\
        14\times a              &=8\times 21                &\text{\bfseries \color{OrangeRed}Les produits en croix sont égaux.}\\
        \dfrac{14\times a}{14}  &= \dfrac{8\times 21}{14}   &\text{\bfseries \color{OrangeRed}On divise les deux membres par 14.}\\
        a                       &=\num{12}                  &\text{\bfseries \color{OrangeRed}On simplifie et on calcule.}
    \end{align*}

    On calculerait de même $b=49$
\end{exemple*1}

\begin{remarque}
    Pour compléter un tableau de proportionnalité, il suffit de connaître un couple de valeurs \og{} associées \fg{} non nulles.
\end{remarque}

