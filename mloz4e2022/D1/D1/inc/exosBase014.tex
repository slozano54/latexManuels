\begin{exercice*}
    Un opérateur téléphonique propose les trois formules suivantes :
    \begin{itemize}
        \item \textbf{\textcolor{red}{Tarif 1 : }} \Prix{0.40}/min sans abonnement.
        \item \textbf{\textcolor{blue}{Tarif 2 : }} \Prix{35} d'abonnement pour un forfait de 2 h de communication, puis \Prix{0.40}/min au-delà du forfait.
        \item \textbf{\textcolor{DarkGreen}{Tarif 3 : }} \Prix{48} d'abonnement pour 4 h de communication, puis \Prix{0.40}/min au-delà.
    \end{itemize}
    \begin{enumerate}
        \item Compléter le tableau ci-dessous.
        \par\smallskip
        \begin{tabular}{|>{\centering\arraybackslash\columncolor{LightGray}}m{0.3\linewidth}|*{5}{>{\centering\arraybackslash}m{0.07\linewidth}|}}
            \hline
            Durée en min&60&150&200&250&300\\\hline
            Prix au tarif 1&&&&&\\\hline
            Prix au tarif 2&&&&&\\\hline
            Prix au tarif 3&&&&&\\\hline
        \end{tabular}
        \par\smallskip
        \item Le tarif 2 a été représenté en bleu sur le graphique ci-dessous. Représente les tarifs 1 et 3, respectivement en rouge et en vert.\par\smallskip
        \Fonction[%
            Trace,%
            Grille,PasGrilleX=0.25,PasGrilleY=0.25,%
            Origine={(1,1)},%            
            Xmin=0,Xmax=6,Xstep=1,%
            Ymin=0,Ymax=7,Ystep=2,%
            Bornea=0,Borneb=0,%
            Traces={%
                for k=1 upto 5 :
                    label.bot(TEX("\num{"&decimal(k*60)&"}"),placepoint(k,0));
                        draw placepoint(k,-0.05)--placepoint(k,0.05) withpen pencircle scaled 1.5bp;
                endfor;
                for k=1 upto 12 :
                        label.lft(TEX("\num{"&decimal(k*10)&"}"),placepoint(0,k));
                        draw placepoint(-0.05,k)--placepoint(0.05,k) withpen pencircle scaled 1.5bp;
                endfor;
                label.bot(btex Durée (en min) etex,placepoint(4,-0.9));
                label(btex \rotatebox{90}{Prix (en \Prix{})} etex,placepoint(-0.9,10.5));
                draw placepoint(0,3.5)--placepoint(2,3.5) withcolor blue withpen pencircle scaled 1.5bp;
                draw placepoint(2,3.5)--placepoint(5.25,11.3) withcolor blue withpen pencircle scaled 1.5bp;
                trace appelation(placepoint(4,8.3),placepoint(5,10.7),-3mm,btex \textcolor{blue}{Tarif 2} etex);
                %%% Correction	              
                % draw placepoint(0,0)--placepoint(5.25,12.6) withcolor red withpen pencircle scaled 1.5bp;
                % trace appelation(placepoint(4,9.6),placepoint(5,12),3mm,btex \textcolor{red}{Tarif 1} etex);
                % draw placepoint(0,4.8)--placepoint(4,4.8) withcolor DarkGreen withpen pencircle scaled 1.5bp;
                % draw placepoint(4,4.8)--placepoint(5.25,7.8) withcolor DarkGreen withpen pencircle scaled 1.5bp;
                % trace appelation(placepoint(4,4.8),placepoint(5,7.2),-3mm,btex \textcolor{DarkGreen}{Tarif 3} etex);
            }
        ]{}\par\smallskip
        \item Déterminer un intervalle de durée de communication pour lequelle il mieux souscrire au tarif 2.
        \item Indiquer le tarif le plus avantageux pour 210 minutes de communication.
        \item Indiquer le(s) tarif(s) représentant une situation de proportionnalité. Justifier.
    \end{enumerate}
\end{exercice*}
\begin{corrige}
    %\setcounter{partie}{0} % Pour s'assurer que le compteur de \partie est à zéro dans les corrigés
    %\phantom{rrr}    
    Un opérateur téléphonique propose les trois formules suivantes :
    \begin{itemize}
        \item \textbf{\textcolor{red}{Tarif 1 : }} \Prix{0.40}/min sans abonnement.
        \item \textbf{\textcolor{blue}{Tarif 2 : }} \Prix{35} d'abonnement pour un forfait de 2 h de communication, puis \Prix{0.40}/min au-delà du forfait.
        \item \textbf{\textcolor{DarkGreen}{Tarif 3 : }} \Prix{48} d'abonnement pour 4 h de communication, puis \Prix{0.40}/min au-delà.
    \end{itemize}
    \begin{enumerate}
        \item Compléter le tableau ci-dessous.
        \par\smallskip
        \begin{tabular}{|>{\centering\arraybackslash\columncolor{LightGray}}m{0.3\linewidth}|*{5}{>{\centering\arraybackslash}m{0.07\linewidth}|}}
            \hline
            Durée en min&60&150&200&250&300\\\hline
            Prix au tarif 1&\textcolor{red}{24}&\textcolor{red}{60}&\textcolor{red}{80}&\textcolor{red}{100}&\textcolor{red}{120}\\\hline
            Prix au tarif 2&\textcolor{red}{35}&\textcolor{red}{47}&\textcolor{red}{67}&\textcolor{red}{87}&\textcolor{red}{107}\\\hline
            Prix au tarif 3&\textcolor{red}{48}&\textcolor{red}{48}&\textcolor{red}{48}&\textcolor{red}{52}&\textcolor{red}{72}\\\hline
        \end{tabular}
        \par\smallskip
        \item Le tarif 2 a été représenté en bleu sur le graphique ci-dessous. Représente les tarifs 1 et 3, respectivement en rouge et en vert.\par\smallskip
        \Fonction[%
            Trace,%
            Grille,PasGrilleX=0.25,PasGrilleY=0.25,%
            Origine={(1,1)},%            
            Xmin=0,Xmax=6,Xstep=1,%
            Ymin=0,Ymax=7,Ystep=2,%
            Bornea=0,Borneb=0,%
            Traces={%
                for k=1 upto 5 :
                    label.bot(TEX("\num{"&decimal(k*60)&"}"),placepoint(k,0));
                        draw placepoint(k,-0.05)--placepoint(k,0.05) withpen pencircle scaled 1.5bp;
                endfor;
                for k=1 upto 12 :
                        label.lft(TEX("\num{"&decimal(k*10)&"}"),placepoint(0,k));
                        draw placepoint(-0.05,k)--placepoint(0.05,k) withpen pencircle scaled 1.5bp;
                endfor;
                label.bot(btex Durée (en min) etex,placepoint(4,-0.9));
                label(btex \rotatebox{90}{Prix (en \Prix{})} etex,placepoint(-0.9,10.5));
                draw placepoint(0,3.5)--placepoint(2,3.5) withcolor blue withpen pencircle scaled 1.5bp;
                draw placepoint(2,3.5)--placepoint(5.25,11.3) withcolor blue withpen pencircle scaled 1.5bp;
                trace appelation(placepoint(4,8.3),placepoint(5,10.7),-3mm,btex \textcolor{blue}{Tarif 2} etex);
                %%% Correction	              
                draw placepoint(0,0)--placepoint(5.25,12.6) withcolor red withpen pencircle scaled 1.5bp;
                trace appelation(placepoint(4,9.6),placepoint(5,12),3mm,btex \textcolor{red}{Tarif 1} etex);
                draw placepoint(0,4.8)--placepoint(4,4.8) withcolor DarkGreen withpen pencircle scaled 1.5bp;
                draw placepoint(4,4.8)--placepoint(5.25,7.8) withcolor DarkGreen withpen pencircle scaled 1.5bp;
                trace appelation(placepoint(4,4.8),placepoint(5,7.2),-3mm,btex \textcolor{DarkGreen}{Tarif 3} etex);
            }
        ]{}\par\smallskip
    \end{enumerate}
    \Coupe    
    \begin{enumerate}
        \setcounter{enumi}{2}
        \item Déterminer un intervalle de durée de communication pour lequelle il mieux souscrire au tarif 2.
        \par\textcolor{red}{Par lecture graphique, le prix du tarif 2 est inférieur aux deux autres entre \Temps{;;;;90;} et \Temps{;;;;150;}.}        
        \item Indiquer le tarif le plus avantageux pour 210 minutes de communication.
        \par\textcolor{red}{Par lecture graphique, c'est le tarfi 3.}
        \item Indiquer le(s) tarif(s) représentant une situation de proportionnalité. Justifier.
        \par\textcolor{red}{La courbe représentant le tarif 1 est une droite passant par l'origine, c'est donc une situation de proportionnalité et pas les deux autres tarifs.}
    \end{enumerate}
\end{corrige}

