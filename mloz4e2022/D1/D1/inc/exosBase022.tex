\begin{exercice*}
    Sur son vélo, Léa parcourt 6,3 km en 21 min.
    \begin{enumerate}
        \item Convertir 21 minutes en heure décimale.
        \item Déterminer sa vitesse en km/h.
        \item À cette vitesse, quelle distance parcourt-elle :
        \begin{enumerate}
            \item en 48 min;
            \item en 2 h 12 min.
        \end{enumerate}
    \end{enumerate}
\end{exercice*}
\begin{corrige}
    %\setcounter{partie}{0} % Pour s'assurer que le compteur de \partie est à zéro dans les corrigés
    %\phantom{rrr}    
    Sur son vélo, Léa parcourt 6,3 km en 21 min.
    \begin{enumerate}
        \item Convertir 21 minutes en heure décimale.
        \par\textcolor{red}{$\dfrac{21\text{ min}}{60\text{ min}}=\num{0.35}$ donc 21 min sont \num{0.35}h.}
        \item Déterminer sa vitesse en km/h.
        \par\textcolor{red}{$\dfrac{\num{6.3}\text{ km}}{\num{0.35}\text{ h}}=\Vitesse[kmh]{18}$.}
        \item À cette vitesse, quelle distance parcourt-elle :
        \begin{enumerate}
            \item en 48 min;
            \par\textcolor{red}{48 min sont \num{0.8}h donc la distance sera de \Lg[km]{14.4}.}
            \item en 2 h 12 min.
            \par\textcolor{red}{2h 12 sont \num{2.2}h donc la distance sera de \Lg[km]{39.6}.}
        \end{enumerate}
    \end{enumerate}
\end{corrige}

