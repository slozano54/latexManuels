\begin{exercice*}
    Déterminer si les égalités suivantes sont vraies.
    \begin{enumerate}
        \item $\dfrac{\num{5,3}}{\num{6,1}}\overset{?}{=}\dfrac{\num{21.2}}{\num{24.4}}$\smallskip
        \item $\dfrac{72}{32}\overset{?}{=}\dfrac{360}{160}$\smallskip
        \item $\dfrac{\num{5,1}}{\num{9,8}}\overset{?}{=}\dfrac{\num{5,3}}{10}$\smallskip
        \item $\dfrac{\num{4,4}}{\num{8,7}}\overset{?}{=}\dfrac{\num{30,8}}{\num{60,9}}$
    \end{enumerate}
    \hrefMathalea{https://coopmaths.fr/alea/?uuid=7f2be&id=4C20-2&n=4&d=10&s=4&alea=pQhQ&cd=1&v=eleve&es=021100}
\end{exercice*}
\begin{corrige}
    %\setcounter{partie}{0} % Pour s'assurer que le compteur de \partie est à zéro dans les corrigés
    %\phantom{rrr}    
    \begin{enumerate}
        \item $\dfrac{\num{5,3}}{\num{6,1}}\overset{?}{=}\dfrac{\num{21.2}}{\num{24.4}}$\par
        D'une part, $\num{5,3}\times \num{24.4} = {\color{red}\boldsymbol{\num{129,32}}}$.\\
                    D'autre part, $\num{6,1}\times \num{21.2} = {\color{red}\boldsymbol{\num{129,32}}}$.\\
                    On constate que les produits en croix sont égaux.\\
                    Les quotients $\dfrac{\num{5,3}}{\num{6,1}}$ et $\dfrac{\num{21.2}}{\num{24.4}}$ sont donc égales.
        \item $\dfrac{72}{32}\overset{?}{=}\dfrac{360}{160}$\par
        D'une part, $72\times 160 = {\color{red}\boldsymbol{\num{11 520}}}$.\\
                    D'autre part, $32\times 360 = {\color{red}\boldsymbol{\num{11 520}}}$.\\
                    On constate que les produits en croix sont égaux.\\
                    Les fractions $\dfrac{72}{32}$ et $\dfrac{360}{160}$ sont donc égales.
    \end{enumerate}
    \Coupe
    \begin{enumerate}
        \setcounter{enumi}{2}
        \item $\dfrac{\num{5,1}}{\num{9,8}}\overset{?}{=}\dfrac{\num{5,3}}{10}$\par
        D'une part, $\num{5,1}\times 10 = {\color{red}\boldsymbol{51}}$.\\
                    D'autre part, $\num{9,8}\times \num{5,3} = {\color{red}\boldsymbol{\num{51,94}}}$.\\
                    On constate que les produits en croix ne sont pas égaux.\\
                    Les quotients $\dfrac{\num{5,1}}{\num{9,8}}$ et $\dfrac{\num{5,3}}{10}$ ne sont donc pas égales.
        \item $\dfrac{\num{4,4}}{\num{8,7}}\overset{?}{=}\dfrac{\num{30,8}}{\num{60,9}}$\par
        D'une part, $\num{4,4}\times \num{60,9} = {\color{red}\boldsymbol{\num{267,96}}}$.\\
                    D'autre part, $\num{8,7}\times \num{30,8} = {\color{red}\boldsymbol{\num{267,96}}}$.\\
                    On constate que les produits en croix sont égaux.\\
                    Les quotients $\dfrac{\num{4,4}}{\num{8,7}}$ et $\dfrac{\num{30,8}}{\num{60,9}}$ sont donc égales.
        \end{enumerate}
\end{corrige}

