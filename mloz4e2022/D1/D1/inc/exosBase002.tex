\begin{exercice*}
    Déterminer si les égalités suivantes sont vraies.
    \begin{enumerate}
        \item $\dfrac{5{,}3}{6{,}1}\overset{?}{=}\dfrac{21{,}2}{24{,}4}$\smallskip
        \item $\dfrac{72}{32}\overset{?}{=}\dfrac{360}{160}$\smallskip
        \item $\dfrac{5{,}1}{9{,}8}\overset{?}{=}\dfrac{5{,}3}{10}$\smallskip
        \item $\dfrac{4{,}4}{8{,}7}\overset{?}{=}\dfrac{30{,}8}{60{,}9}$
    \end{enumerate}
    \hrefMathalea{https://coopmaths.fr/alea/?uuid=7f2be&id=4C20-2&n=4&d=10&s=4&alea=pQhQ&cd=1&v=eleve&es=021100}
\end{exercice*}
\begin{corrige}
    %\setcounter{partie}{0} % Pour s'assurer que le compteur de \partie est à zéro dans les corrigés
    %\phantom{rrr}    
    \begin{enumerate}
        \item $\dfrac{5{,}3}{6{,}1}\overset{?}{=}\dfrac{21{,}2}{24{,}4}$\par
        D'une part, $5{,}3\times 24{,}4 = {\color[HTML]{f15929}\boldsymbol{129{,}32}}$.\\
                    D'autre part, $6{,}1\times 21{,}2 = {\color[HTML]{f15929}\boldsymbol{129{,}32}}$.\\
                    On constate que les produits en croix ne sont pas égaux.\\
                    Les quotients $\dfrac{5{,}3}{6{,}1}$ et $\dfrac{21{,}2}{24{,}4}$ sont donc égales.
        \item $\dfrac{72}{32}\overset{?}{=}\dfrac{360}{160}$\par
        D'une part, $72\times 160 = {\color[HTML]{f15929}\boldsymbol{11\,520}}$.\\
                    D'autre part, $32\times 360 = {\color[HTML]{f15929}\boldsymbol{11\,520}}$.\\
                    On constate que les produits en croix sont égaux.\\
                    Les fractions $\dfrac{72}{32}$ et $\dfrac{360}{160}$ sont donc égales.
    \end{enumerate}
    \Coupe
    \begin{enumerate}
        \setcounter{enumi}{2}
        \item $\dfrac{5{,}1}{9{,}8}\overset{?}{=}\dfrac{5{,}3}{10}$\par
        D'une part, $5{,}1\times 10 = {\color[HTML]{f15929}\boldsymbol{51}}$.\\
                    D'autre part, $9{,}8\times 5{,}3 = {\color[HTML]{f15929}\boldsymbol{51{,}94}}$.\\
                    On constate que les produits en croix ne sont pas égaux.\\
                    Les quotients $\dfrac{5{,}1}{9{,}8}$ et $\dfrac{5{,}3}{10}$ ne sont donc pas égales.
        \item $\dfrac{4{,}4}{8{,}7}\overset{?}{=}\dfrac{30{,}8}{60{,}9}$\par
        D'une part, $4{,}4\times 60{,}9 = {\color[HTML]{f15929}\boldsymbol{267{,}96}}$.\\
                    D'autre part, $8{,}7\times 30{,}8 = {\color[HTML]{f15929}\boldsymbol{267{,}96}}$.\\
                    On constate que les produits en croix ne sont pas égaux.\\
                    Les quotients $\dfrac{4{,}4}{8{,}7}$ et $\dfrac{30{,}8}{60{,}9}$ sont donc égales.
        \end{enumerate}
\end{corrige}

