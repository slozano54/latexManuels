\begin{exercice*}
    La loi d'Ohm indique que la tension U (en Volts) aux bornes d'un conducteur ohmique est égale au produit de la résistance R (en Ohms) du
    conducteur et de l'intensité I (en Ampères) du courant qui traverse ce conducteur.\par
    \vspace*{-5mm}
    \begin{center}
        \begin{Geometrie}            
            pair A[];
            A0=u*(3,4);
            draw A0 withpen pencircle scaled 4bp;
            A1-A0=u*(-2,0);
            draw A0--A1;
            A2-A1=u*(0,-0.6);
            A3-A2=u*(0,-0.4);
            drawarrow A1--A2;
            label.rt(btex \textbf{I} etex, iso(A1,A3));
            draw A2--A3;
            A4-A3=u*(0.2,0);
            A5-A4=u*(0,-1);
            A6-A5=u*(-0.4,0);
            A7-A6=u*(0,1);
            draw A3--A4--A5--A6--A7--A3;
            pair B;
            B=iso(A6,A7);
            label.lft(btex \textbf{R} etex,B);
            A8=iso(A5,A6);
            A9-A8=u*(0,-1);
            A10-A9=u*(2,0);
            draw A8--A9--A10;
            draw A10 withpen pencircle scaled 4bp;
            A11-A10=u*(0,0.3);
            A12-A0=u*(0,-0.3);
            drawarrow A11--A12;
            pair C;
            C=iso(A11,A12);
            label.rt(btex \textbf{U} etex, C);
        \end{Geometrie}
    \end{center}
    \vspace*{-8mm}
    \begin{enumerate}
        \item À l'aide du texte ci-dessus, indiquer la relation reliant les variables U , R et I.
        \item On réalise un montage expérimental permettant de mesurer la tension U (à l'aide d'un voltmètre) et l'intensité I (à l'aide d'un ampèremètre).
        \begin{enumerate}
            \item Si on mesure U = \SI{12}{V} et I = \SI{0.24}{A}, calculer la valeur de la résistance du conducteur ohmique.
            \item Si R = \SI{200}{\ohm} et U = \SI{220}{V}, déterminer l'intensité de courant traversant le dipôle.
        \end{enumerate}
    \end{enumerate}
\end{exercice*}
\begin{corrige}
    %\setcounter{partie}{0} % Pour s'assurer que le compteur de \partie est à zéro dans les corrigés
    %\phantom{rrr}    
    La loi d'Ohm indique que la tension U (en Volts) aux bornes d'un conducteur ohmique est égale au produit de la résistance R (en Ohms) du
    conducteur et de l'intensité I (en Ampères) du courant qui traverse ce conducteur.\par
    \begin{center}
        \begin{Geometrie}            
            pair A[];
            A0=u*(3,4);
            draw A0 withpen pencircle scaled 4bp;
            A1-A0=u*(-2,0);
            draw A0--A1;
            A2-A1=u*(0,-0.6);
            A3-A2=u*(0,-0.4);
            drawarrow A1--A2;
            label.rt(btex \textbf{I} etex, iso(A1,A3));
            draw A2--A3;
            A4-A3=u*(0.2,0);
            A5-A4=u*(0,-1);
            A6-A5=u*(-0.4,0);
            A7-A6=u*(0,1);
            draw A3--A4--A5--A6--A7--A3;
            pair B;
            B=iso(A6,A7);
            label.lft(btex \textbf{R} etex,B);
            A8=iso(A5,A6);
            A9-A8=u*(0,-1);
            A10-A9=u*(2,0);
            draw A8--A9--A10;
            draw A10 withpen pencircle scaled 4bp;
            A11-A10=u*(0,0.3);
            A12-A0=u*(0,-0.3);
            drawarrow A11--A12;
            pair C;
            C=iso(A11,A12);
            label.rt(btex \textbf{U} etex, C);
        \end{Geometrie}
    \end{center}
    \Coupe
    \begin{enumerate}
        \item À l'aide du texte ci-dessus, indiquer la relation reliant les variables U , R et I.
        \par\textcolor{red}{$U=R\times I$}
        \item On réalise un montage expérimental permettant de mesurer la tension U (à l'aide d'un voltmètre) et l'intensité I (à l'aide d'un ampèremètre).\\
        \begin{enumerate}
            \item Si on mesure U = \SI{12}{V} et I = \SI{0.24}{A}, calculer la valeur de la résistance du conducteur ohmique.
            \par\textcolor{red}{$R=\dfrac{U}{I}=\dfrac{\SI{12}{V}}{\SI{0.24}{A}}=\SI{50}{\ohm}$}
            \item Si R = \SI{200}{\ohm} et U = \SI{220}{V}, déterminer l'intensité de courant traversant le dipôle.
            \par\textcolor{red}{$I=\dfrac{U}{R}=\dfrac{\SI{220}{V}}{\SI{200}{\ohm}}=\SI{1.1}{A}$}
        \end{enumerate}
    \end{enumerate}
\end{corrige}

