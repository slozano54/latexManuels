\begin{exercice*}
    La loi d'Ohm indique que la tension U (en Volts) aux bornes d'un conducteur ohmique est égale au produit de la résistance R (en Ohms) du
    conducteur et de l'intensité I (en Ampères) du courant qui traverse ce conducteur.\par\smallskip
    \begin{center}
        \begin{circuitikz}
            \draw (2,3) -- (0,3) 
                to [R, a=$R$, i>^=$I$] (0,0) 
        -- (2,0);
        \draw (2,0) 
            to[open, *-*, v>=$U$] (2,3);
        \end{circuitikz}
    \end{center}
    \begin{enumerate}
        \item À l'aide du texte ci-dessus, indiquer la relation reliant les variables U , R et I.
        \item On réalise un montage expérimental permettant de mesurer la tension U (à l'aide d'un voltmètre) et l'intensité I (à l'aide d'un ampèremètre).
        \begin{enumerate}
            \item Si on mesure U = \SI{12}{V} et I = \SI{0.24}{A}, calculer la valeur de la résistance du conducteur ohmique.
            \item Si R = \SI{200}{\ohm} et U = \SI{220}{V}, déterminer l'intensité de courant traversant le dipôle.
        \end{enumerate}
    \end{enumerate}
\end{exercice*}
\begin{corrige}
    %\setcounter{partie}{0} % Pour s'assurer que le compteur de \partie est à zéro dans les corrigés
    %\phantom{rrr}    
    La loi d'Ohm indique que la tension U (en Volts) aux bornes d'un conducteur ohmique est égale au produit de la résistance R (en Ohms) du
    conducteur et de l'intensité I (en Ampères) du courant qui traverse ce conducteur.\par
    \begin{center}
        \begin{circuitikz}
            \draw (2,3) -- (0,3) 
                to [R, a=$R$, i>^=$I$] (0,0) 
        -- (2,0);
        \draw (2,0) 
            to[open, *-*, v>=$U$] (2,3);
        \end{circuitikz}
    \end{center}
    \begin{enumerate}
        \item À l'aide du texte ci-dessus, indiquer la relation reliant les variables U , R et I.
        \par\textcolor{red}{$U=R\times I$}
        \item On réalise un montage expérimental permettant de mesurer la tension U (à l'aide d'un voltmètre) et l'intensité I (à l'aide d'un ampèremètre).\\
        \begin{enumerate}
            \item Si on mesure U = \SI{12}{V} et I = \SI{0.24}{A}, calculer la valeur de la résistance du conducteur ohmique.
            \par\textcolor{red}{$R=\dfrac{U}{I}=\dfrac{\SI{12}{V}}{\SI{0.24}{A}}=\SI{50}{\ohm}$}
            \item Si R = \SI{200}{\ohm} et U = \SI{220}{V}, déterminer l'intensité de courant traversant le dipôle.
            \par\textcolor{red}{$I=\dfrac{U}{R}=\dfrac{\SI{220}{V}}{\SI{200}{\ohm}}=\SI{1.1}{A}$}
        \end{enumerate}
    \end{enumerate}
\end{corrige}

