\begin{exercice*}
    \begin{enumerate}
        \item Le but (ou cochonnet) d'un jeu de pétanque est en bois, de masse volumique \MasseVol[kgdm]{0.7}, et a un volume de \Vol[cm]{14.1}.
        Déterminer sa masse.
        \item Une boule de pétanque a une masse de \Masse[g]{650} et un volume de \Vol[dm]{0.183}.
        Sachant que l'acier avec lequel cette boule est fabriquée a une masse volumique de \MasseVol[kgdm]{7.850},
        souligner un aspect surprenant de cette boule de pétanque.
    \end{enumerate}
\end{exercice*}
\begin{corrige}
    %\setcounter{partie}{0} % Pour s'assurer que le compteur de \partie est à zéro dans les corrigés
    %\phantom{rrr}    
    \begin{enumerate}
        \item Le but (ou cochonnet) d'un jeu de pétanque est en bois, de masse volumique \MasseVol[kgdm]{0.7}, et a un volume de \Vol[cm]{14.1}.
        Déterminer sa masse.
        \par\textcolor{red}{\Vol[cm]{1} de ce bois pèse\Masse[g]{0.7} et $\num{14.1}\times\num{0.7}=\num{9.87}$ donc le but pèse \Masse[g]{9.87}.}
        \item Une boule de pétanque a une masse de \Masse[g]{650} et un volume de \Vol[dm]{0.183}.
        Sachant que l'acier avec lequel cette boule est fabriquée a une masse volumique de \MasseVol[kgdm]{7.850},
        souligner un aspect surprenant de cette boule de pétanque.
        \par\textcolor{red}{\Vol[dm]{0.183} d'acier pèse $\num{7.85}\times\num{0.183}=\Masse[kg]{1.43655}$ or la boule ne pèse que \Masse[g]{650}, elle
        est donc sûrement creuse.}
    \end{enumerate}
\end{corrige}

