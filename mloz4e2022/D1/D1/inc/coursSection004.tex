\section{Représentation graphique}
\begin{remarque}
    Dans un tableau de nombres, si l'on peut passer des nombres d'une ligne aux nombres de l'autre ligne par une même multiplication alors ce tableau est un tableau de proportionnalité.    
    \tikzstyle{FlechePropor}=[-{Stealth}]
    \vspace*{-10mm}
    \begin{center}
        \Propor[Stretch=1.2,Simple]{3/7.5,5/12.5,9/22.5}
    \end{center}
    \FlechesPD{1}{2}{$\times\num{2,5}$}
\end{remarque}
\vspace*{-10mm}
\begin{propriete}[\admise]
    Sur une représentation graphique, on reconnaît une situation de proportionnalité lorsque tous les points sont alignés avec l'origine du repère.
\end{propriete}

\begin{multicols}{2}
    \begin{center}
        \Propor[Stretch=1.2,Simple]{10/9,25/22.5,40/36}
        $$\underbrace{\frac9{10}=\frac{22,5}{25}=\frac{36}{40}}_{\mbox{les quotients sont égaux}}$$
    \end{center}
    \columnbreak
    \begin{center}
        \Fonction[%
            Trace,%
            Grille,PasGrilleX=0.1,PasGrilleY=0.1,%
            Origine={(0.7,0.5)},%
            CouleurTrace=red,
            Calcul=.9*x,%
            Xmin=0,Xmax=5,%
            Ymin=0,Ymax=4.5,%
            Bornea=0,Borneb=5,%
            Traces={%
                marque_p:="plein";
                marque_r:=0.7*marque_r;
                draw placepoint(1,0)--placepoint(1,0.9) dashed evenly;
                draw placepoint(1,0.9)--placepoint(0,0.9) dashed evenly;			
                pointe(placepoint(1,0),placepoint(1,0.9),placepoint(0,0.9));
                label.bot(btex \num{10} etex,placepoint(1,0));
                label.lft(btex \num{9} etex,placepoint(0,0.9));
                %
                draw placepoint(2.5,0)--placepoint(2.5,2.25) dashed evenly;
                draw placepoint(2.5,2.25)--placepoint(0,2.25) dashed evenly;			
                pointe(placepoint(2.5,0),placepoint(2.5,2.25),placepoint(0,2.25));
                label.bot(btex \num{25} etex,placepoint(2.5,0));
                label.lft(btex \num{22.5} etex,placepoint(0,2.25));
                %
                draw placepoint(4,0)--placepoint(4,3.6) dashed evenly;
                draw placepoint(4,3.6)--placepoint(0,3.6) dashed evenly;			
                pointe(placepoint(4,0),placepoint(4,3.6),placepoint(0,3.6));
                label.bot(btex \num{40} etex,placepoint(4,0));
                label.lft(btex \num{36} etex,placepoint(0,3.6));
            }
        ]{}
        \psshadowbox{{\sc Proportionnalité}}
    \end{center}
\end{multicols}
\medskip
\begin{multicols}{2}
    \begin{center}
        \Fonction[%
            Trace,%
            Grille,PasGrilleX=0.1,PasGrilleY=0.1,%
            Origine={(0.5,0.5)},%
            CouleurTrace=red,
            Calcul=7*x/15+10/3/10,%
            Xmin=0,Xmax=5,%
            Ymin=0,Ymax=3,%
            Bornea=0,Borneb=5,%
            Traces={%
                marque_p:="plein";
                marque_r:=0.7*marque_r;
                draw placepoint(1,0)--placepoint(1,0.8) dashed evenly;
                draw placepoint(1,0.8)--placepoint(0,0.8) dashed evenly;			
                pointe(placepoint(1,0),placepoint(1,0.8),placepoint(0,0.8));
                label.bot(btex \num{10} etex,placepoint(1,0));
                label.lft(btex \num{8} etex,placepoint(0,0.8));
                %
                draw placepoint(2.5,0)--placepoint(2.5,1.5) dashed evenly;
                draw placepoint(2.5,1.5)--placepoint(0,1.5) dashed evenly;			
                pointe(placepoint(2.5,0),placepoint(2.5,1.5),placepoint(0,1.5));
                label.bot(btex \num{25} etex,placepoint(2.5,0));
                label.lft(btex \num{15} etex,placepoint(0,1.5));
                %
                draw placepoint(4,0)--placepoint(4,2.2) dashed evenly;
                draw placepoint(4,2.2)--placepoint(0,2.2) dashed evenly;			
                pointe(placepoint(4,0),placepoint(4,2.2),placepoint(0,2.2));
                label.bot(btex \num{40} etex,placepoint(4,0));
                label.lft(btex \num{22} etex,placepoint(0,2.2));
            }
        ]{}
        \psshadowbox{{\sc Non} {\sc Proportionnalité}}
    \end{center}
    \columnbreak
    \begin{center}
        \Propor[Stretch=1.2,Simple]{10/8,25/15,40/22}
        $$\underbrace{\frac8{10}\quad  \text{et} \quad \frac{15}{25}=\frac35=\frac6{10}}_{\mbox{les quotients ne sont pas égaux}}$$
    \end{center}
\end{multicols}
\medskip
\begin{multicols}{2}
    \begin{center}
        \Propor[Stretch=1.2,Simple]{10/2,25/12.5,40/32}
        $$\underbrace{\frac2{10}\quad \text{et} \quad \frac{12,5}{25}=\frac12=\frac5{10}}_{\mbox{les quotients ne sont pas égaux}}$$
    \end{center}
    \columnbreak
    \begin{center}
        \Fonction[%
            Trace,%
            Grille,PasGrilleX=0.1,PasGrilleY=0.1,%
            Origine={(0.7,0.5)},%
            CouleurTrace=red,
            Calcul=10*(x*x)/50,%
            Xmin=0,Xmax=5,%
            Ymin=0,Ymax=4,%
            Bornea=0,Borneb=5,%
            Traces={%
                marque_p:="plein";
                marque_r:=0.7*marque_r;
                draw placepoint(1,0)--placepoint(1,0.2) dashed evenly;
                draw placepoint(1,0.2)--placepoint(0,0.2) dashed evenly;			
                pointe(placepoint(1,0),placepoint(1,0.2),placepoint(0,0.2));
                label.bot(btex \num{10} etex,placepoint(1,0));
                label.lft(btex \num{2} etex,placepoint(0,0.2));
                %
                draw placepoint(2.5,0)--placepoint(2.5,1.25) dashed evenly;
                draw placepoint(2.5,1.25)--placepoint(0,1.25) dashed evenly;			
                pointe(placepoint(2.5,0),placepoint(2.5,1.25),placepoint(0,1.25));
                label.bot(btex \num{25} etex,placepoint(2.5,0));
                label.lft(btex \num{12.5} etex,placepoint(0,1.25));
                %
                draw placepoint(4,0)--placepoint(4,3.2) dashed evenly;
                draw placepoint(4,3.2)--placepoint(0,3.2) dashed evenly;			
                pointe(placepoint(4,0),placepoint(4,3.2),placepoint(0,3.2));
                label.bot(btex \num{40} etex,placepoint(4,0));
                label.lft(btex \num{32} etex,placepoint(0,3.2));
            }
        ]{}
        \psshadowbox{{\sc Non} {\sc Proportionnalité}}
    \end{center}
\end{multicols}

