\section{Poucentages}
% \subsection{Calculer le pourcentage d'un nombre}
% \proprNum{(admise)}{on veut prendre "$t\%$" d'un nombre}{on le multiplie par $\dfrac{t}{100}$}

% \Exemples{}{
% \begin{mylist}
% \item $35\%$ des élèves d'un collège de $560$ élèves sont demi-pensionnaires, donc $560\times \dfrac{35}{100}=196$ élèves.
% \item $100 \%$ d'une classe de $5^{eme}$ de $23$ élèves est externe, c'est à dire $23\times \dfrac{100}{100}= 23$ élèves!
% \item \textit{C'est Mardi gras et aujourd'hui on a $40\%$ sur les crêpes car la boulangère est de bonne humeur. D'ordinaire la crêpe coûte $2$ \textgreek{\euro}.}
% \begin{enumerate}
% \item \`{A} combien s'élève la réduction?
% \item Combien payera t'on alors une crêpe?
% \end{enumerate}
% \par\vspace{0.5cm}
% \begin{tabular}{|c|c|c|}
% \hline
% Montant de la réduction&$R$&$40$ \textgreek{\euro}\\
% \hline
% Prix payé sans réduction&$2$ \textgreek{\euro}&$100$ \textgreek{\euro}\\
% \hline
% \end{tabular}
% \par\vspace{0.5cm}
% C'est un tableau de proportionnalité, donc les rapports $\dfrac{R}{2}$ et $\dfrac{40}{100}$ sont égaux,

% d'où $\dfrac{R}{2}=\dfrac{40}{100}$ d'où en les multipliant par 2, $\dfrac{R}{2} \times 2=\dfrac{40}{100} \times 2$ d'où $R=\dfrac{40}{100} \times 2$
% \end{mylist}
% }

% \subsection{Déterminer un pourcentage}
% \Exemples[Exemple]{}{
% \textit{On se demande quel est le pourcentage d'élèves de $5^{\grave{e}me}$ qui sont demi-pensionnaires. Aujourd'hui sur les \ldots \ldots \ldots présents, il y a \ldots \ldots \ldots demi-pensionnaires.}

% "Si dans la classe il y avait eu $100$ présents alors combien y aurait-il de demi-pensionnaires?"

% C'est une situation de proportionnalité.

% \begin{center}
% \begin{tabular}{|c|c|c|}
% \hline
% Nombre d'élèves présents&$100$&$n_{pr\acute{e}sents}$\\
% \hline
% Nombre de demi-pensionnaires&$n$&$n_{DP}$\\
% \hline
% \end{tabular}
% \end{center}

% Les rapports $\dfrac{n}{100}$ et $\dfrac{n_{DP}}{n_{pr\acute{e}sents}}$ sont égaux d'où
% $n=\dfrac{n_{DP}}{n_{pr\acute{e}sents}}\times 100$
% }

