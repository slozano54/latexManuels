\section{Poucentages}
\begin{methode*1}[Calculer le pourcentage d'un nombre]
    Si on veut prendre "$t\%$" d'un nombre alors on le multiplie par $\dfrac{t}{100}$.
    \exercice
    \begin{enumerate}    
        \item Dans un collège de $560$ élèves, $35\%$ des élèves sont demi-pensionnaires. Calculer le nombre de demi-pensionnaires.
        \item Dans une classe de \num{23} élèves, $100 \%$ sont externes. Déterminer le nombre d'externes.
        \item \textit{C'est Mardi gras et aujourd'hui on a $40\%$ sur les crêpes car la boulangère est de bonne humeur. D'ordinaire la crêpe coûte \Prix{2}.}
        \begin{enumerate}
            \item Déterminer le montnat de la réduction.
            \item Calculer alors le prix d'une crêpe?
        \end{enumerate}
    \end{enumerate}
    \correction
    \begin{enumerate}    
        \item \Pourcentage[Fractionnaire]{35}{560} 
        \num{196} élèves sont donc demi-pensionnaires.
        \item \Pourcentage[Fractionnaire]{100}{23}
        Ou alors on pense que $100\%$ c'est la totalité ! \num{23} élèves sont donc externes !
        \item \textit{C'est Mardi gras et aujourd'hui on a $40\%$ sur les crêpes car la boulangère est de bonne humeur. D'ordinaire la crêpe coûte \Prix{2}.}
        \par\medskip
        {\renewcommand{\arraystretch}{1.2}
            \begin{tabular}{|>{\columncolor{LightGray}}m{0.35\linewidth}|>{\centering\arraybackslash}m{0.1\linewidth}|>{\centering\arraybackslash}m{0.1\linewidth}|}
                \hline
                Montant de la réduction&$R$&\Prix{40}\\
                \hline
                Prix payé sans réduction&\Prix{2}&\Prix{100}\\
                \hline
            \end{tabular}        
        }
        \par\medskip
        C'est un tableau de proportionnalité, donc les rapports $\dfrac{R}{2}$ et $\dfrac{40}{100}$ sont égaux, 
        d'où $\dfrac{R}{2}=\dfrac{40}{100}$ d'où en les multipliant par 2, $\dfrac{R}{2} \times 2=\dfrac{40}{100} \times 2$ d'où $R=\dfrac{40}{100} \times 2=\num{0.8}$.
        \par\medskip
        \begin{enumerate}
            \item Le montant de la réduction est donc de \Prix{0.8}.
            \item La crêpe coûtera donc \Prix{1.2}.
        \end{enumerate}
    \end{enumerate}
\end{methode*1}

\begin{methode*1}[Déterminer un pourcentage]    
    \exercice
    Dans un collège de \num{475} élèves, il y a \num{323} externes.
    Déterminer le pourcentage d'externes.
    \correction
    Si dans ce collège il y avait $100$ élèves alors combien y aurait-il d'externes ?\par
    C'est une situation de proportionnalité.
    \par\medskip
    {\renewcommand{\arraystretch}{1.2}
    \begin{tabular}{|>{\columncolor{LightGray}}m{0.25\linewidth}|>{\centering\arraybackslash}m{0.1\linewidth}|>{\centering\arraybackslash}m{0.1\linewidth}|}
        \hline
        Nombre d'externes&\num{323}&{\bfseries\color{OrangeRed}$x$ ?}\\
        \hline
        Nombre d'élèves&\num{475}&\num{100}\\
        \hline            
    \end{tabular}
    }
    \par\medskip
    On détermine la quatrième proportionnelle à l'aide des produits en croix.
    \begin{align*}
        \dfrac{323}{475}                            &=\dfrac{\color{OrangeRed} x}{100}&\\
        475\times {\color{OrangeRed} x}             &= 323\times 100                  &\text{\bfseries \color{OrangeRed}Les produits en croix sont égaux.}\\
        \dfrac{475\times {\color{OrangeRed} x}}{475}&= \dfrac{323\times 100}{475}     &\text{\bfseries \color{OrangeRed}On divise les deux membres par 6.}\\
        {\color{OrangeRed} x}                       &=\num{68}                        &\text{\bfseries \color{OrangeRed}On simplifie et on calcule.}
    \end{align*}
    \par\medskip
    Il y a donc $68\%$ d'externes dans ce collège.
\end{methode*1}

