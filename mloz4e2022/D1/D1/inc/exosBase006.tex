\begin{exercice*}    
    Dans chacun des cas suivants, déterminer la quatrième proportionnelle.
    \begin{enumerate}
        \item {\renewcommand{\arraystretch}{1.2}
            \begin{tabular}{|>{\centering\arraybackslash}m{0.1\linewidth}|>{\centering\arraybackslash}m{0.1\linewidth}|}
                \hline
                $9$&{\bfseries\color{OrangeRed}$x$ ?}\\\hline
                $4$&$2$\\\hline
            \end{tabular}
        }\smallskip
        \item {\renewcommand{\arraystretch}{1.2}
            \begin{tabular}{|>{\centering\arraybackslash}m{0.1\linewidth}|>{\centering\arraybackslash}m{0.1\linewidth}|}
                \hline
                $-4$&$-8$\\\hline
                {\bfseries\color{OrangeRed}$t$ ?}&$-2$\\\hline            
            \end{tabular}
        }\smallskip
        \item {\renewcommand{\arraystretch}{1.2}
            \begin{tabular}{|>{\centering\arraybackslash}m{0.1\linewidth}|>{\centering\arraybackslash}m{0.1\linewidth}|}
                \hline                
                \num{10}&\num{3.5}\\\hline            
                \num{8.9}&{\bfseries\color{OrangeRed}$z$ ?}\\\hline
            \end{tabular}
        }
    \end{enumerate}
    \hrefMathalea{https://coopmaths.fr/alea/?uuid=a6b5b&id=4P10-2&n=3&d=10&s=4&alea=wSve&cd=1&v=eleve&es=021100}
\end{exercice*}
\begin{corrige}
    %\setcounter{partie}{0} % Pour s'assurer que le compteur de \partie est à zéro dans les corrigés
    %\phantom{rrr}
    Dans chacun des cas suivants, déterminer la quatrième proportionnelle.\par\smallskip
    \begin{enumerate}
        \item {\renewcommand{\arraystretch}{1.2}
            \begin{tabular}{|>{\centering\arraybackslash}m{0.1\linewidth}|>{\centering\arraybackslash}m{0.1\linewidth}|}
                \hline
                $9$&{\bfseries\color{OrangeRed}$x$ ?}\\\hline
                $4$&$2$\\\hline
            \end{tabular}
        }\smallskip
        \begin{align*}
            \dfrac{9}{4}                              &=\dfrac{\color{OrangeRed} x}{2}&\\
            4\times {\color{OrangeRed} x}             &= 9\times 2                    &\text{\tiny\bfseries\color{red}Les produits en croix sont égaux.}\\
            \dfrac{4\times {\color{OrangeRed} x}}{4}  &= \dfrac{9\times 2}{4}         &\text{\tiny\bfseries\color{red}On divise les deux membres par 4.}\\
            {\color{OrangeRed} x}                     &=\num{4.5}                     &\text{\tiny\bfseries\color{red}On simplifie et on calcule.}
        \end{align*}        
        \item {\renewcommand{\arraystretch}{1.2}
            \begin{tabular}{|>{\centering\arraybackslash}m{0.1\linewidth}|>{\centering\arraybackslash}m{0.1\linewidth}|}
                \hline
                $-4$&$-8$\\\hline
                {\bfseries\color{OrangeRed}$t$ ?}&$-2$\\\hline            
            \end{tabular}
        }\smallskip
        \begin{align*}
            \dfrac{-4}{\color{OrangeRed} t}                              &=\dfrac{-8}{-2}&\\
            -8\times {\color{OrangeRed} t}             &= -4\times (-2)                    &\text{\tiny\bfseries\color{red}Les produits en croix sont égaux.}\\
            \dfrac{-8\times {\color{OrangeRed} t}}{-8}  &= \dfrac{-4\times (-2)}{-8}         &\text{\tiny\bfseries\color{red}On divise les deux membres par -8.}\\
            {\color{OrangeRed} t}                     &=\num{1}                     &\text{\tiny\bfseries\color{red}On simplifie et on calcule.}
        \end{align*}        
        \item {\renewcommand{\arraystretch}{1.2}
            \begin{tabular}{|>{\centering\arraybackslash}m{0.1\linewidth}|>{\centering\arraybackslash}m{0.1\linewidth}|}
                \hline
                \num{10}&\num{3.5}\\\hline
                \num{8.9}&{\bfseries\color{OrangeRed}$z$ ?}\\\hline            
            \end{tabular}
        }\smallskip
        \begin{align*}
            \dfrac{10}{\num{8.9}}                              &=\dfrac{\num{3.5}}{\color{OrangeRed} z}&\\
            10\times {\color{OrangeRed} z}             &= \num{8.9}\times\num{3.5}                    &\text{\tiny\bfseries\color{red}Les produits en croix sont égaux.}\\
            \dfrac{10\times {\color{OrangeRed} z}}{10}  &= \dfrac{\num{8.9}\times\num{3.5}}{10}         &\text{\tiny\bfseries\color{red}On divise les deux membres par 10.}\\
            {\color{OrangeRed} z}                     &=\num{3.115}                     &\text{\tiny\bfseries\color{red}On simplifie et on calcule.}
        \end{align*}        
    \end{enumerate}    
\end{corrige}

