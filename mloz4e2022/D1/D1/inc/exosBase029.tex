\begin{exercice*}
    F11ex4
    
    Lancé le 26 novembre 2011, le rover Curiosity de la NASA est chargé d’analyser la planète Mars, appelée aussi planète Rouge.
    Il a atterri sur la planète Rouge le 6 aout 2012, parcourant ainsi une distance d’environ 560 millions de km en 255 jours.
\begin{enumerate}
    \item Quelle a été la durée en heures du vol ?
    \item Calcule la vitesse moyenne du rover en km/h. Arrondis à la centaine près.
    \item Via le satellite Mars Odyssey, des images prises et envoyées par le rover ont été retransmises au centre de la NASA. Les premières images ont été
    émises de Mars à 7 h 48 min le 6 aout 2012. La distance parcourue par le signal a été de 248 × 10 6 km, à une vitesse moyenne de 300 000 km/s environ (vitesse de la lumière).
    À quelle heure ces premières images sont-elles parvenues au centre de la NASA ? (On donnera l’arrondi à la minute près.)
\end{enumerate}
\end{exercice*}
\begin{corrige}
    %\setcounter{partie}{0} % Pour s'assurer que le compteur de \partie est à zéro dans les corrigés
    %\phantom{rrr}    
    \dots
\end{corrige}

