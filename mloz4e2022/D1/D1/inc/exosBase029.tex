\begin{exercice*}
    Lancé le 26 novembre 2011, le rover Curiosity de la NASA est chargé d’analyser la planète Mars, appelée aussi planète Rouge.
    Il a atterri sur la planète Rouge le 6 aout 2012, parcourant ainsi une distance d’environ 560 millions de km en 255 jours.
    \begin{enumerate}
        \item Calculer la durée du vol en heures.
        \item Calculer la vitesse moyenne du rover en km/h. Arrondir à la centaine près.
        \item Via le satellite Mars Odyssey, des images prises et envoyées par le rover ont été retransmises au centre de la NASA.
        Les premières images ont été émises de Mars à \Temps{;;;7;48;} le 6 aout 2012. La distance parcourue par le signal a été de
        $248\times 10^6$ \Lg[km]{}, à une vitesse moyenne de \Vitesse[kms]{300 000} environ (vitesse de la lumière).
        Déterminer l'heure d'arrivée de ces premières images au centre de la NASA. (On donnera l’arrondi à la minute près.)
    \end{enumerate}
\end{exercice*}
\begin{corrige}
    %\setcounter{partie}{0} % Pour s'assurer que le compteur de \partie est à zéro dans les corrigés
    %\phantom{rrr}    
    Lancé le 26 novembre 2011, le rover Curiosity de la NASA est chargé d’analyser la planète Mars, appelée aussi planète Rouge.
    Il a atterri sur la planète Rouge le 6 aout 2012, parcourant ainsi une distance d’environ 560 millions de km en 255 jours.
    \begin{enumerate}
        \item Calculer la durée du vol en heures.
        \par\textcolor{red}{$\Temps{;;255;;;} = 255\times 24 = \Temps{;;;6120;;}$}
        \item Calculer la vitesse moyenne du rover en km/h. Arrondir à la centaine près.
        \par\textcolor{red}{$\text{Vitesse}=\dfrac{Distance}{Temps}$ d'où $V=\dfrac{\Lg[km]{560000000}}{\Temps{;;;6120}}=\Vitesse[kmh]{91500}$}
        \item Via le satellite Mars Odyssey, des images prises et envoyées par le rover ont été retransmises au centre de la NASA.
        Les premières images ont été émises de Mars à \Temps{;;;7;48;} le 6 aout 2012. La distance parcourue par le signal a été de
        $248\times 10^6$ \Lg[km]{}, à une vitesse moyenne de \Vitesse[kms]{300 000} environ (vitesse de la lumière).
        Déterminer l'heure d'arrivée de ces premières images au centre de la NASA. (On donnera l’arrondi à la minute près.)
        \par\textcolor{red}{$\text{Temps}=\dfrac{Distance}{Vitesse}$ d'où $T=248\times 10^6~\text{km}\div\Vitesse[kms]{300000}\simeq \Temps{;;;;;827}$ soit environ \Temps{;;;;14;}\\
        $\Temps{;;;7;48;}+\Temps{;;;;14;} = \Temps{;;;8;02;}$, les première images sont donc arrivées à la NASA à environ \Temps{;;;8;02;}.}
    \end{enumerate}
\end{corrige}

