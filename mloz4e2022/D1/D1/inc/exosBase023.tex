\begin{exercice*}
    Lors d'un marathon, un coureur parcourt le premier kilomètre de course, en quatre minutes et trente secondes. La longueur officielle d'un
    marathon est de \Lg[km]{42,195}. Si le coureur garde cette allure tout au long de sa course, déterminer s'il mettra moins de \Temps{;;;3;30;} pour effectuer le marathon.
\end{exercice*}
\begin{corrige}
    %\setcounter{partie}{0} % Pour s'assurer que le compteur de \partie est à zéro dans les corrigés
    %\phantom{rrr}    
    Lors d'un marathon, un coureur parcourt le premier kilomètre de course, en quatre minutes et trente secondes. La longueur officielle d'un
    marathon est de \Lg[km]{42,195}. Si le coureur garde cette allure tout au long de sa course, déterminer s'il mettra moins de \Temps{;;;3;30;} pour effectuer le marathon.
    \par\textcolor{red}{%
    4 min 30 s = \num{4.5} min, 3 h 30 = 210 min et $210 \text{ min}\div \num{4.5}\text{ min} \simeq \num{46.7}$. En gardant la même allure, il fera \Lg[km]{46.7} en 3 h 30, donc
    il fera le marathon en moins de 3 h 30.}
\end{corrige}

