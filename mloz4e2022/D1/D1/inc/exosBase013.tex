\begin{exercice*}
    Lorsqu’on fait geler de l’eau, le volume de glace obtenu est proportionnel au volume d’eau utilisé.\par 
    En faisant geler \Capa[L]{1.5} d’eau, on obtient \Capa[L]{1.62} de glace.
    \begin{enumerate}
	    \item Montrer qu’en faisant geler \Capa[L]{1} d’eau, on obtient \Capa[L]{1.08} de glace.
	    \item On souhaite compléter le tableau ci-dessous à l’aide d’un tableur.\par\medskip	    
        \scalebox{0.7}{
            \parbox{\linewidth}{
                \begin{Tableur}[LargeurUn=80pt,Largeur=20pt,Colonnes=7,Cellule=B2,Colonne=2,Ligne=2,PasL=1,PasC=1]
                    Volume d'eau initial (en \Capa[L]{})&\num{0.5}&\num{1}&\num{1.5}&\num{2}&\num{2.5}&\num{3}\\
                    Volume de glace obtenu (en \Capa[L]{})&&&&&&\\
                \end{Tableur}
            }
        }        
		\begin{enumerate}\medskip
			\item Proposer une formule à saisir en cellule B2 avant de la recopier vers la droite jusqu'à la cellule G2.
			\item Compléter le tableau.
		\end{enumerate}
		\item Déterminer le graphique représentant le volume de glace obtenu (en \Capa[L]{}) en fonction du volume d'eau contenu au départ dans la bouteille (en \Capa[L]{}). Justifier.\par\medskip
		\begin{minipage}{0.3\linewidth}
            \begin{center}
                \begin{tikzpicture}[scale=0.7]
                    % Axes
                    \draw[thick,>=latex,->](0,0)--+(2.5,0);
                    \draw[thick,>=latex,->](0,0)--+(0,2.5);
                    \foreach \x in {0,0.5,...,2} \draw[thick] (\x,-.05) -- (\x,0.05);
                    \foreach \y in {0,0.5,...,2} \draw[thick] (-.05,\y) -- (0.05,\y);
                    \foreach \x in {0,...,2} \draw (\x,-0.4) node {\scriptsize\x};
                    \foreach \y in {0,...,2} \draw (-0.4,\y) node {\scriptsize\y};
                    % Courbe
                    \draw [domain=0:2,blue,very thick] plot (\x,{0.6*\x*\x});
                    % Fond
                    \begin{pgfonlayer}{background}
                        \draw[step=5mm,very thin,brown!50] (current bounding box.south west) grid (current bounding box.north east);
                        \draw[step=1cm,thin,brown!70] (current bounding box.south west) grid (current bounding box.north east);
                    \end{pgfonlayer}
                \end{tikzpicture}\par
                {\sc\footnotesize Graphique n°1}
            \end{center}
        \end{minipage}
        \begin{minipage}{0.3\linewidth}
            \begin{center}
                \begin{tikzpicture}[scale=0.7]
                    % Axes
                    \draw[thick,>=latex,->](0,0)--+(2.5,0);
                    \draw[thick,>=latex,->](0,0)--+(0,2.5);
                    \foreach \x in {0,0.5,...,2} \draw[thick] (\x,-.05) -- (\x,0.05);
                    \foreach \y in {0,0.5,...,2} \draw[thick] (-.05,\y) -- (0.05,\y);
                    \foreach \x in {0,...,2} \draw (\x,-0.4) node {\scriptsize\x};
                    \foreach \y in {0,...,2} \draw (-0.4,\y) node {\scriptsize\y};
                    % Courbe
                    \draw [domain=0:2.3,blue,very thick] plot (\x,{1.08*\x});
                    % Fond
                    \begin{pgfonlayer}{background}
                        \draw[step=5mm,very thin,brown!50] (current bounding box.south west) grid (current bounding box.north east);
                        \draw[step=1cm,thin,brown!70] (current bounding box.south west) grid (current bounding box.north east);
                    \end{pgfonlayer}
                \end{tikzpicture}\par
                {\sc\footnotesize Graphique n°2}
            \end{center}
        \end{minipage}
        \begin{minipage}{0.3\linewidth}
            \begin{center}
                \begin{tikzpicture}[scale=0.7]
                    % Axes
                    \draw[thick,>=latex,->](0,0)--+(2.5,0);
                    \draw[thick,>=latex,->](0,0)--+(0,2.5);
                    \foreach \x in {0,0.5,...,2} \draw[thick] (\x,-.05) -- (\x,0.05);
                    \foreach \y in {0,0.5,...,2} \draw[thick] (-.05,\y) -- (0.05,\y);
                    \foreach \x in {0,...,2} \draw (\x,-0.4) node {\scriptsize\x};
                    \foreach \y in {0,...,2} \draw (-0.4,\y) node {\scriptsize\y};
                    % Courbe
                    \draw [domain=0:2.5,blue,very thick] plot (\x,{0.5+0.7*\x});
                    % Fond
                    \begin{pgfonlayer}{background}
                        \draw[step=5mm,very thin,brown!50] (current bounding box.south west) grid (current bounding box.north east);
                        \draw[step=1cm,thin,brown!70] (current bounding box.south west) grid (current bounding box.north east);
                    \end{pgfonlayer}
                \end{tikzpicture}\par
                {\sc\footnotesize Graphique n°3}
            \end{center}
    \end{minipage}
    \end{enumerate}
\end{exercice*}
\begin{corrige}
    %\setcounter{partie}{0} % Pour s'assurer que le compteur de \partie est à zéro dans les corrigés
    %\phantom{rrr}    
    Lorsqu’on fait geler de l’eau, le volume de glace obtenu est proportionnel au volume d’eau utilisé.\par 
    En faisant geler \Capa[L]{1.5} d’eau, on obtient \Capa[L]{1.62} de glace.\par
    \begin{enumerate}
	    \item Montrer qu’en faisant geler \Capa[L]{1} d’eau, on obtient \Capa[L]{1.08} de glace.\par
	    \textcolor{red}{$V_\text{de glace}\div V_\text{d'eau}=\Capa[L]{1.62}\div \Capa[L]{1.5} = \num{1.08}$.}
	    \item On souhaite compléter le tableau ci-dessous à l’aide d’un tableur.\par\medskip	    
        \scalebox{0.65}{
            \parbox{\linewidth}{
                \begin{Tableur}[LargeurUn=80pt,Largeur=20pt,Colonnes=7,Cellule=B2,Colonne=2,Ligne=2,PasL=1,PasC=1]
                    Volume d'eau initial (en \Capa[L]{})&\num{0.5}&\num{1}&\num{1.5}&\num{2}&\num{2.5}&\num{3}\\
                    Volume de glace obtenu (en \Capa[L]{})&\textcolor{red}{\num{0.54}}&\textcolor{red}{\num{1.08}}&\textcolor{red}{\num{1.62}}&\textcolor{red}{\num{2.16}}&\textcolor{red}{\num{2.7}}&\textcolor{red}{\num{3.24}}\\
                \end{Tableur}
            }
        }\par\medskip        
		\begin{enumerate}
			\item Proposer une formule à saisir en cellule B2 avant de la recopier vers la droite jusqu'à la cellule G2.\par
			\textcolor{red}{Une formule possible : {\sc =1.08*B1}.}
			\item Compléter le tableau.\par
			\textcolor{red}{cf ci-dessus.}
		\end{enumerate}
        \setcounter{enumi}{2}
		\item Déterminer le graphique représentant le volume de glace obtenu (en \Capa[L]{}) en fonction du volume d'eau contenu au départ dans la bouteille (en \Capa[L]{}). Justifier.
    \end{enumerate}    
    \Coupe
		\begin{minipage}{0.3\linewidth}
            \begin{center}
                \begin{tikzpicture}[scale=0.7]
                    % Axes
                    \draw[thick,>=latex,->](0,0)--+(2.5,0);
                    \draw[thick,>=latex,->](0,0)--+(0,2.5);
                    \foreach \x in {0,0.5,...,2} \draw[thick] (\x,-.05) -- (\x,0.05);
                    \foreach \y in {0,0.5,...,2} \draw[thick] (-.05,\y) -- (0.05,\y);
                    \foreach \x in {0,...,2} \draw (\x,-0.4) node {\scriptsize\x};
                    \foreach \y in {0,...,2} \draw (-0.4,\y) node {\scriptsize\y};
                    % Courbe
                    \draw [domain=0:2,blue,very thick] plot (\x,{0.6*\x*\x});
                    % Fond
                    \begin{pgfonlayer}{background}
                        \draw[step=5mm,very thin,brown!50] (current bounding box.south west) grid (current bounding box.north east);
                        \draw[step=1cm,thin,brown!70] (current bounding box.south west) grid (current bounding box.north east);
                    \end{pgfonlayer}
                \end{tikzpicture}\par
                {\sc\footnotesize Graphique n°1}
            \end{center}
        \end{minipage}
        \begin{minipage}{0.3\linewidth}
            \begin{center}
                \begin{tikzpicture}[scale=0.7]
                    % Axes
                    \draw[thick,>=latex,->](0,0)--+(2.5,0);
                    \draw[thick,>=latex,->](0,0)--+(0,2.5);
                    \foreach \x in {0,0.5,...,2} \draw[thick] (\x,-.05) -- (\x,0.05);
                    \foreach \y in {0,0.5,...,2} \draw[thick] (-.05,\y) -- (0.05,\y);
                    \foreach \x in {0,...,2} \draw (\x,-0.4) node {\scriptsize\x};
                    \foreach \y in {0,...,2} \draw (-0.4,\y) node {\scriptsize\y};
                    % Courbe
                    \draw [domain=0:2.3,blue,very thick] plot (\x,{1.08*\x});
                    % Fond
                    \begin{pgfonlayer}{background}
                        \draw[step=5mm,very thin,brown!50] (current bounding box.south west) grid (current bounding box.north east);
                        \draw[step=1cm,thin,brown!70] (current bounding box.south west) grid (current bounding box.north east);
                    \end{pgfonlayer}
                \end{tikzpicture}\par
                {\sc\footnotesize Graphique n°2}
            \end{center}
        \end{minipage}
        \begin{minipage}{0.3\linewidth}
            \begin{center}
                \begin{tikzpicture}[scale=0.7]
                    % Axes
                    \draw[thick,>=latex,->](0,0)--+(2.5,0);
                    \draw[thick,>=latex,->](0,0)--+(0,2.5);
                    \foreach \x in {0,0.5,...,2} \draw[thick] (\x,-.05) -- (\x,0.05);
                    \foreach \y in {0,0.5,...,2} \draw[thick] (-.05,\y) -- (0.05,\y);
                    \foreach \x in {0,...,2} \draw (\x,-0.4) node {\scriptsize\x};
                    \foreach \y in {0,...,2} \draw (-0.4,\y) node {\scriptsize\y};
                    % Courbe
                    \draw [domain=0:2.5,blue,very thick] plot (\x,{0.5+0.7*\x});
                    % Fond
                    \begin{pgfonlayer}{background}
                        \draw[step=5mm,very thin,brown!50] (current bounding box.south west) grid (current bounding box.north east);
                        \draw[step=1cm,thin,brown!70] (current bounding box.south west) grid (current bounding box.north east);
                    \end{pgfonlayer}
                \end{tikzpicture}\par
                {\sc\footnotesize Graphique n°3}
            \end{center}
    \end{minipage}
    \par\medskip    
    \textcolor{red}{Les graphiques n°1 et n°3 ne conviennent pas car ils ne représentent pas une situation de proportionnalité, c'est donc le graphique n°2.}
\end{corrige}

