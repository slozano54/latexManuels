\begin{activite}[La cabine]    
    La figure suivante représente le déplacement d'une cabine de téléphérique. La droite $(d)$ représente le câble porteur.
    Le point $A$ représente le point d'attache de la cabine au câble au départ. À l'arrivée, ce point est représenté par le point $B$.

    \begin{center}
        \begin{Geometrie}[CoinHD={(11u,5u)}]
            trace grille(0.5) withcolor black;
            pair A,B,M,N,P,R,S,M',N';
            A=u*(1.5,2.5);
            B=u*(7.5,4);
            labeloffset := 5bp ;
            label.top(btex $A$ etex, A);
            label.top(btex $B$ etex, B);
            marque_p:="croix";
            pointe(A,B);
            trace droite(A,B) withcolor black;
            drawarrow A--B withcolor blue dashed evenly;
            labeloffset := 3bp ;
            label.top(btex $(d)$ etex,1.2[B,A]);
            M=u*(1,2);
            N=u*(2,2);
            P=u*(2.5,1.5);
            R=u*(2.5,0.5);
            S=u*(1,0.5);
            trace segment(A,projection(A,M,N));
            trace polygone(M,N,P,R,S) withpen pencircle scaled 1.5bp;
            remplis polygone(M,N,P,R,S) withcolor gris;
            labeloffset := 5bp ;
            pointe(M,N,P,R,S);
            label.llft(btex  $M$ etex, M);
            label.urt(btex  $N$ etex, N);
            label.urt(btex  $P$ etex, P);
            label.lrt(btex  $R$ etex, R);
            label.llft(btex $S$ etex, S);
            M'=M shifted (6u,1.5u);
            N'=N shifted (6u,1.5u);
            drawoptions(withpen pencircle scaled 1.5bp withcolor black);
            trace segment(B,projection(B,M',N'));
            label.lft(btex $M'$ etex, M');
            label.rt(btex $N'$ etex, N');
            trace segment(M',N');
            pointe(M',N');
            trace segment(A,projection(A,M,N));
            trace segment(M,N);
        \end{Geometrie}
    \end{center}
        Directement sur la figure ci-dessus :        
        \begin{enumerate}
        \item Compléter la figure avec la cabine dans sa position d'arrivée.
        
        \smallskip
        {\bfseries On notera $P'$, $R'$ et $S'$ les points d'arrivée correspondant aux points $P$, $R$ et $S$.}
        \smallskip
        \item Tracer le segment $[MM']$. Qu'a-t-il de particulier ?
        
        \smallskip\pointilles
        \item Tracer d'autres segments ayant cette particularité.
        \item Indiquer la nature du quadrilatère $ABM'M$ : \pointilles
        \item Nommer d'autres quadrilatères ayant cette particularité dont un avec les sommets $A$ et $B$.
        
        \smallskip\pointilles
        
        \smallskip
        {\bfseries Cette transformation permettant d'amener la cabine de sa position de départ à sa position d'arrivée s'appelle translation qui transforme $A$ en $B$.}
        \smallskip

        \smallskip
        \item Déterminer l'image du segment $[NP]$ par la translation qui transforme $A$ en $B$ : \pointilles
        
        \smallskip\pointilles

        \smallskip
        \item Faire une remarque sur ce segment et son image : \pointilles
        
        \smallskip\pointilles
        \item Comparer la figure $MNPRS$ et son image par la translation qui transforme $A$ en $B$. Leurs longueurs, leurs angles \dots
        
        \smallskip\pointilles

        \smallskip\pointilles

        \smallskip
        {\bfseries La translation revient à glisser sans tourner.}
        \smallskip
    \end{enumerate}
\end{activite}
