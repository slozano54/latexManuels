\section{Propriétés}
\begin{propriete}[\admise]
    L'image du point $M$, par la translation qui transforme $A$ en $B$, est le point $M'$, tel que les segments $[MB]$ et $[AM']$ ont le même milieu.
\end{propriete}

\begin{remarque}
    Si les points ne sont pas alignés, alors $ABM'M$ est un parallélogramme.
\end{remarque}

\begin{propriete}[\admise]
    L'image d'un segment par une translation est un segment parallèle et de même longueur.
\end{propriete}

\begin{remarque}
    Une figure et son image par translation sont superposables.
\end{remarque}

\begin{propriete}[Conservations \admise]
    La translation conserve :
    \begin{itemize}
        \item les longueurs donc les périmètres et les aires,
        \item l'alignement donc les milieux,
        \item la mesure des angles
    \end{itemize}
\end{propriete}
