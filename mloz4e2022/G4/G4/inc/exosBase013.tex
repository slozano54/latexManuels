\begin{exercice*}[Hexagone d'Escher]
    \begin{changemargin}{0mm}{-5mm}
        \begin{enumerate}
            \item Colorier en rouge le motif permettant d'obtenir cette frise en effectuant la translation qui transforme $C$ en $D$.\\            
            \begin{Geometrie}[CoinHD={(8u,3u)}]
                labeloffset := 5bp ;
                trace grille(0.5) withcolor white;
                pair O;
                O = u*(1.5,1);
                pair A[];
                path cO;
                rayon:=0.5u;
                cO = cercles(O,rayon);
                A0 = pointarc(cO,0);
                A2 = pointarc(cO,120);
                A4 = pointarc(cO,240);
                A1 = 0.3[A2,A0] rotatedaround(A2,30);
                A3 = A1 rotatedaround(A2,-120);
                A5 = A3 rotatedaround(A4,-120);        
                picture hexagoneEscher;
                hexagoneEscher = image(
                    draw A3--A2--A1 withcolor red;        
                        draw A3--A4--A5 withcolor blue;        
                        draw A5--A0--A1 withcolor DarkGreen;
                );
                picture maille[];
                maille[0] = image(
                    draw hexagoneEscher;
                    draw hexagoneEscher rotatedaround(A2,-120);
                    draw hexagoneEscher rotatedaround(A2,-240);
                );        
                draw maille[0];
                for l=1 upto 4:
                    maille[l] = maille[0] shifted (1.5*l*u,0);
                    draw maille[l];
                endfor;
                pair C,D;
                C = A5 rotatedaround(A2,120);
                D = A5 rotatedaround(A2,120) shifted (1.5u,0);
                label.urt(btex $C$ etex,C);
                label.urt(btex $D$ etex,D);
                marque_p:="croix";
                marque_s:=3;
                pointe(C,D);
            \end{Geometrie}
            \item On repart de la frise précédetne.
            \begin{itemize}
                \item Représenter par une flèche la translation à effectuer sur la frise pour obtenir le pavage.
                \item Colorier la frise initiale avec une couleur et sa translatée avec une autre.
            \end{itemize}
            \begin{Geometrie}[CoinHD={(9u,4u)}]
                labeloffset := 5bp ;
                trace grille(0.5) withcolor white;
                pair O;
                O = u*(1.5,1);
                pair A[];
                path cO;
                rayon:=0.5u;
                cO = cercles(O,rayon);
                A0 = pointarc(cO,0);
                A2 = pointarc(cO,120);
                A4 = pointarc(cO,240);
                A1 = 0.3[A2,A0] rotatedaround(A2,30);
                A3 = A1 rotatedaround(A2,-120);
                A5 = A3 rotatedaround(A4,-120);        
                picture hexagoneEscher;
                hexagoneEscher = image(
                    draw A3--A2--A1 withcolor red;        
                        draw A3--A4--A5 withcolor blue;        
                        draw A5--A0--A1 withcolor DarkGreen;
                );
                picture maille[];
                maille[0] = image(
                    draw hexagoneEscher;
                    draw hexagoneEscher rotatedaround(A2,-120);
                    draw hexagoneEscher rotatedaround(A2,-240);
                );        
                draw maille[0];
                for l=1 upto 4:
                    maille[l] = maille[0] shifted (1.5*l*u,0);
                    draw maille[l];
                endfor;
                pair C,D;
                C = A5 rotatedaround(A2,120);
                D = A5 rotatedaround(A2,120) shifted (1.5u,0);
                marque_p:="croix";
                marque_s:=3;
                pair z;
                z=C-A2;
                maille[5] = maille[0] shifted 1.2z;
                draw maille[5];
                for l=1 upto 4:
                    maille[l+5] = maille[5] shifted (1.5*l*u,0);
                        draw maille[l+5];
                endfor;            
            \end{Geometrie}
        \end{enumerate}
    \end{changemargin}
\end{exercice*}
\begin{corrige}
    %\setcounter{partie}{0} % Pour s'assurer que le compteur de \partie est à zéro dans les corrigés
    % \phantom{rrr}
    \begin{enumerate}
        \item Colorier en rouge le motif permettant d'obtenir cette frise en effectuant la translation qui transforme $C$ en $D$.\\
        \hspace*{-7mm}
        \scalebox{0.9}{
        \begin{Geometrie}[CoinHD={(8u,3u)}]
            labeloffset := 5bp ;
            trace grille(0.5) withcolor white;
            pair O;
            O = u*(1.5,1);
            pair A[];
            path cO;
            rayon:=0.5u;
            cO = cercles(O,rayon);
            A0 = pointarc(cO,0);
            A2 = pointarc(cO,120);
            A4 = pointarc(cO,240);
            A1 = 0.3[A2,A0] rotatedaround(A2,30);
            A3 = A1 rotatedaround(A2,-120);
            A5 = A3 rotatedaround(A4,-120);        
            picture hexagoneEscher;
            hexagoneEscher = image(
                draw A3--A2--A1 withcolor red;        
                    draw A3--A4--A5 withcolor blue;        
                    draw A5--A0--A1 withcolor DarkGreen;
            );
            picture maille[];
            maille[0] = image(
                draw hexagoneEscher;
                draw hexagoneEscher rotatedaround(A2,-120);
                draw hexagoneEscher rotatedaround(A2,-240);
            );        
            draw maille[0];
            for l=1 upto 4:
                maille[l] = maille[0] shifted (1.5*l*u,0);
                draw maille[l];
            endfor;
            pair C,D;
            C = A5 rotatedaround(A2,120);
            D = A5 rotatedaround(A2,120) shifted (1.5u,0);
            label.urt(btex $C$ etex,C);
            label.urt(btex $D$ etex,D);
            marque_p:="croix";
            marque_s:=3;
            pointe(C,D);            
            %%% Correction
            path hexagoneEscherPath[];
            hexagoneEscherPath[0] = A0--A1--A2--A3--A4--A5--cycle;
            fill hexagoneEscherPath[0] withcolor LightPink;
            draw maille[0];
            hexagoneEscherPath[1] = hexagoneEscherPath[0] rotatedaround(A2,-120);
            hexagoneEscherPath[2] = hexagoneEscherPath[0] rotatedaround(A2,-240);
            fill hexagoneEscherPath[1] withcolor LightPink;
            fill hexagoneEscherPath[2] withcolor LightPink;            
            pointe(C);
            drawarrow C--D;
        \end{Geometrie}
        }
        \item On repart de la frise précédetne.
        \begin{itemize}
            \item Représenter par une flèche la translation à effectuer sur la frise pour obtenir le pavage.
            \item Colorier la frise initiale avec une couleur et sa translatée avec une autre.
        \end{itemize}
        \hspace*{-7mm}
        \scalebox{0.9}{
        \begin{Geometrie}[CoinHD={(9u,4u)}]
            labeloffset := 5bp ;
            trace grille(0.5) withcolor white;
            pair O;
            O = u*(1.5,1);
            pair A[];
            path cO;
            rayon:=0.5u;
            cO = cercles(O,rayon);
            A0 = pointarc(cO,0);
            A2 = pointarc(cO,120);
            A4 = pointarc(cO,240);
            A1 = 0.3[A2,A0] rotatedaround(A2,30);
            A3 = A1 rotatedaround(A2,-120);
            A5 = A3 rotatedaround(A4,-120);        
            picture hexagoneEscher;
            hexagoneEscher = image(
                draw A3--A2--A1 withcolor red;        
                    draw A3--A4--A5 withcolor blue;        
                    draw A5--A0--A1 withcolor DarkGreen;
            );
            picture maille[];
            maille[0] = image(
                draw hexagoneEscher;
                draw hexagoneEscher rotatedaround(A2,-120);
                draw hexagoneEscher rotatedaround(A2,-240);
            );        
            draw maille[0];
            for l=1 upto 4:
                maille[l] = maille[0] shifted (1.5*l*u,0);
                draw maille[l];
            endfor;
            pair C,D;
            C = A5 rotatedaround(A2,120);
            D = A5 rotatedaround(A2,120) shifted (1.5u,0);
            marque_p:="croix";
            marque_s:=3;
            pair z;
            z=C-A2;
            maille[5] = maille[0] shifted 1.2z;
            draw maille[5];
            for l=1 upto 4:
                maille[l+5] = maille[5] shifted (1.5*l*u,0);
                    draw maille[l+5];
            endfor;            
            %%% Correction
            picture corr;
            corr = image(
            path hexagoneEscherPath[];            
            hexagoneEscherPath[0] = A0--A1--A2--A3--A4--A5--cycle;
            fill hexagoneEscherPath[0] withcolor LightPink;
            hexagoneEscherPath[1] = hexagoneEscherPath[0] shifted 1.2z;
            fill hexagoneEscherPath[1] withcolor LightGreen;
            for l=1 upto 2:
                    hexagoneEscherPath[2*l] = hexagoneEscherPath[0] rotatedaround(A2,-120*l);
                fill hexagoneEscherPath[2*l] withcolor LightPink;
                hexagoneEscherPath[2*l+1] = hexagoneEscherPath[2*l] shifted 1.2z;
                fill hexagoneEscherPath[2*l+1] withcolor LightGreen;
            endfor;
            draw maille[0];
            draw maille[5];
            );
            for l=0 upto 4:
                draw corr shifted (1.5*l*u,0);
            endfor;  
%            pointe(C);
            drawarrow A5 rotatedaround(A2,-120)-- A5 rotatedaround(A2,-120) shifted 1.2z;
        \end{Geometrie}
        }
    \end{enumerate}
\end{corrige}

