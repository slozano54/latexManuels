\begin{exercice*}
    On se base sur la figure de l'exercice {\bfseries 1}.
    
    \Reseau[%
    Colonnes=5,%
    Lignes=8,%
    Traces={%	
        pair A,B,C,D,E,F,G,H; % Placement dans le quadrillage en calculant
        A=ppreseau(0,8);
        B=ppreseau(1,8);
        C=ppreseau(1,7);
        D=ppreseau(0,7);
        E=ppreseau(2,5);
        F=ppreseau(1,2);
        G=ppreseau(3,2);
        H=ppreseau(4,6);
        marque_p:="croix";
        drawoptions(withcolor red);
        pointe(A,B,C,D,E,F,G,H);
        label.urt(btex $A$ etex,A);
        label.urt(btex $B$ etex,B);
        label.urt(btex $C$ etex,C);
        label.urt(btex $D$ etex,D);
        label.urt(btex $E$ etex,E);
        label.urt(btex $F$ etex,F);
        label.urt(btex $G$ etex,G);
        label.urt(btex $H$ etex,H);
        drawoptions();
    }]{}

    Compléter les phrases suivantes.
    \begin{enumerate}
        \item On considère la translation qui transforme $B$ en $C$.
        \begin{itemize}
            \item L'image du motif \num{51} est le motif \pointilles
            \item L'image du motif \num{41} est le motif \pointilles
            \item L'image du motif \num{32} est le motif \pointilles
            \item L'image du motif \num{26} est le motif \pointilles
        \end{itemize}
        \item On considère la translation qui transforme $E$ en $G$.
        \begin{itemize}
            \item L'image du motif \num{51} est le motif \pointilles
            \item L'image du motif \num{41} est le motif \pointilles
            \item L'image du motif \num{32} est le motif \pointilles
            \item L'image du motif \num{26} est le motif \pointilles
        \end{itemize}
        \item Déterminer la translation par laquelle le motif \num{44} est l'image du motif \num{9}, c'est la translation qui 
        
        \pointilles
        \item En considérant la translation de la question précédente :
        \begin{itemize}
            \item L'image du motif \num{16} est le motif \pointilles
            \item L'image du motif \num{1} est le motif \pointilles
        \end{itemize}
    \end{enumerate}
\end{exercice*}
\begin{corrige}
    %\setcounter{partie}{0} % Pour s'assurer que le compteur de \partie est à zéro dans les corrigés
    % \phantom{rrr}
    On se base sur la figure de l'exercice {\bfseries 1}.\\ Compléter les phrases suivantes.\\
    \begin{changemargin}{0mm}{-5mm}
        \begin{enumerate}
            \item On considère la translation qui transforme $B$ en $C$.
            \begin{itemize}
                \item L'image du motif \num{51} est le motif {\red 43}.
                \item L'image du motif \num{41} est le motif {\red 33}.
                \item L'image du motif \num{32} est le motif {\red 24}.
                \item L'image du motif \num{26} est le motif {\red 18}.
            \end{itemize}
            \item On considère la translation qui transforme $E$ en $G$.
            \begin{itemize}
                \item L'image du motif \num{51} est le motif {\red 28}.
                \item L'image du motif \num{41} est le motif {\red 18}.
                \item L'image du motif \num{32} est le motif {\red 9}.
                \item L'image du motif \num{26} est le motif {\red 3}.
            \end{itemize}
            \item Déterminer la translation par laquelle le motif \num{44} est l'image du motif \num{9}.\\
            {\red La translation qui transforme $F$ en $H$.}
            \item En considérant la translation de la question précédente :
            \begin{itemize}
                \item L'image du motif \num{16} est le motif {\red 51}.
                \item L'image du motif \num{1} est le motif {\red 36}.
            \end{itemize}
        \end{enumerate}
    \end{changemargin}
\end{corrige}

