\begin{activite}[Construction de parallélogrammes]
    \begin{changemargin}{-10mm}{-40mm}
        \partie[Avec quadrillage]
        Reproduire à l’identique les parallélogrammes \textbf{à l'aide des quadrillages} :\par\smallskip
        %1    
        \begin{minipage}{0.22\linewidth}
            \scalebox{0.9}{
            \begin{Geometrie}[CoinHD={(4u,4u)}]
                trace feuillet withpen pencircle scaled 2bp;
                pair A,B,C,D;
                A=u*(1,1);
                B-A=u*(2,0.5);
                D-A=u*(-0.5,1.5);
                C-D=u*(2,0.5);
                trace polygone(A,B,C,D) withpen pencircle scaled 2bp;
                fill polygone(A,B,C,D) withcolor gris;
                trace grille(0.5) withcolor black;
            \end{Geometrie}
            }
            \par\vspace*{-5mm}
            \tikz[remember picture,overlay]{\coordinate[name=A,xshift=20mm,yshift=13mm];}
            \par
            \tikz[remember picture,overlay]{\coordinate[name=B,xshift=20mm,yshift=1mm];}
            \par
            \scalebox{0.9}{
            \begin{Geometrie}[CoinHD={(4u,4u)}]
                trace feuillet withpen pencircle scaled 2bp;
                pair A,B,C,D;
                A=u*(1,1);
                B-A=u*(2,0.5);
                D-A=u*(-0.5,1.5);
                C-D=u*(2,0.5);
                trace C--D withpen pencircle scaled 1bp;
                trace grille(0.5) withcolor black;
            \end{Geometrie}
            \begin{tikzpicture}[remember picture,overlay]%
                \draw[Gray,-{Triangle[width=25pt,length=10pt]}, line width=15pt](A) -- (B);
            \end{tikzpicture}%
            }
        \end{minipage}
        %2
        \begin{minipage}{0.22\linewidth}
            \scalebox{0.9}{
            \begin{Geometrie}[CoinHD={(4u,4u)}]
                trace feuillet withpen pencircle scaled 2bp;
                pair A,B,C,D;
                A=u*(2.5,1);
                B-A=u*(1,0.5);
                D-A=u*(-2,1.5);
                C-D=u*(1,0.5);
                trace polygone(A,B,C,D) withpen pencircle scaled 2bp;
                fill polygone(A,B,C,D) withcolor gris;
                trace grille(0.5) withcolor black;
            \end{Geometrie}
            }
            \par\vspace*{-5mm}
            \tikz[remember picture,overlay]{\coordinate[name=A,xshift=20mm,yshift=13mm];}
            \par
            \tikz[remember picture,overlay]{\coordinate[name=B,xshift=20mm,yshift=1mm];}
            \par
            \scalebox{0.9}{
            \begin{Geometrie}[CoinHD={(4u,4u)}]
                trace feuillet withpen pencircle scaled 2bp;
                pair A,B,C,D;
                A=u*(2.5,1);
                B-A=u*(1,0.5);
                D-A=u*(-2,1.5);
                C-D=u*(1,0.5);
                trace C--D withpen pencircle scaled 1bp;
                trace grille(0.5) withcolor black;
            \end{Geometrie}
            \begin{tikzpicture}[remember picture,overlay]%
                \draw[Gray,-{Triangle[width=25pt,length=10pt]}, line width=15pt](A) -- (B);
            \end{tikzpicture}%
            }
        \end{minipage}
        %3
        \begin{minipage}{0.22\linewidth}
            \scalebox{0.9}{
            \begin{Geometrie}[CoinHD={(4u,4u)}]
                trace feuillet withpen pencircle scaled 2bp;
                pair A,B,C,D;
                A=u*(3,1);
                B-A=u*(0.5,2);
                D-A=u*(-2.5,0.5);
                C-D=u*(0.5,2);
                trace polygone(A,B,C,D) withpen pencircle scaled 2bp;
                fill polygone(A,B,C,D) withcolor gris;
                trace grille(0.5) withcolor black;
            \end{Geometrie}
            }
            \par\vspace*{-5mm}
            \tikz[remember picture,overlay]{\coordinate[name=A,xshift=20mm,yshift=13mm];}
            \par
            \tikz[remember picture,overlay]{\coordinate[name=B,xshift=20mm,yshift=1mm];}
            \par
            \scalebox{0.9}{
            \begin{Geometrie}[CoinHD={(4u,4u)}]
                trace feuillet withpen pencircle scaled 2bp;
                pair A,B,C,D;
                A=u*(3,1);
                B-A=u*(0.5,2);
                D-A=u*(-2.5,0.5);
                C-D=u*(0.5,2);
                trace A--B withpen pencircle scaled 1bp;
                trace grille(0.5) withcolor black;
            \end{Geometrie}
            \begin{tikzpicture}[remember picture,overlay]%
                \draw[Gray,-{Triangle[width=25pt,length=10pt]}, line width=15pt](A) -- (B);
            \end{tikzpicture}%
            }
        \end{minipage}
        %4
        \begin{minipage}{0.22\linewidth}
            \scalebox{0.9}{
            \begin{Geometrie}[CoinHD={(4u,4u)}]
                trace feuillet withpen pencircle scaled 2bp;
                pair A,B,C,D;
                A=u*(2,1);
                B-A=u*(1.5,2);
                D-A=u*(-1.5,0.5);
                C-D=u*(1.5,2);
                trace polygone(A,B,C,D) withpen pencircle scaled 2bp;
                fill polygone(A,B,C,D) withcolor gris;
                trace grille(0.5) withcolor black;
            \end{Geometrie}
            }
            \par\vspace*{-5mm}
            \tikz[remember picture,overlay]{\coordinate[name=A,xshift=20mm,yshift=13mm];}
            \par
            \tikz[remember picture,overlay]{\coordinate[name=B,xshift=20mm,yshift=1mm];}
            \par
            \scalebox{0.9}{
            \begin{Geometrie}[CoinHD={(4u,4u)}]
                trace feuillet withpen pencircle scaled 2bp;
                pair A,B,C,D;
                A=u*(2,1);
                B-A=u*(1.5,2);
                D-A=u*(-1.5,0.5);
                C-D=u*(1.5,2);
                trace grille(0.5) withcolor black;
            \end{Geometrie}
            \begin{tikzpicture}[remember picture,overlay]%
                \draw[Gray,-{Triangle[width=25pt,length=10pt]}, line width=15pt](A) -- (B);
            \end{tikzpicture}%
            }
        \end{minipage}

        \partie[Sans quadrillage]
        Reproduire à l’identique les parallélogrammes \textbf{à l'aide du compas} :\par\smallskip
        %1    
        \begin{minipage}{0.33\linewidth}
            \scalebox{0.9}{
            \begin{Geometrie}[CoinHD={(5u,5u)}]
                trace feuillet withpen pencircle scaled 2bp;
                pair A,B,C,D;
                A=u*(2.5,1);
                B-A=u*(2,3);
                D-A=u*(-2,0.5);
                C-D=u*(2,3);
                trace polygone(A,B,C,D) withpen pencircle scaled 2bp;
                fill polygone(A,B,C,D) withcolor gris;
            \end{Geometrie}
            }
            \par\vspace*{-5mm}
            \tikz[remember picture,overlay]{\coordinate[name=A,xshift=25mm,yshift=13mm];}
            \par
            \tikz[remember picture,overlay]{\coordinate[name=B,xshift=25mm,yshift=1mm];}
            \par
            \scalebox{0.9}{
            \begin{Geometrie}[CoinHD={(5u,5u)}]
                trace feuillet withpen pencircle scaled 2bp;
                pair A,B,C,D;
                A=u*(2.5,1);
                B-A=u*(2,3);
                D-A=u*(-2,0.5);
                C-D=u*(2,3);
                trace B--C--D withpen pencircle scaled 1bp;
            \end{Geometrie}
            \begin{tikzpicture}[remember picture,overlay]%
                \draw[Gray,-{Triangle[width=25pt,length=10pt]}, line width=15pt](A) -- (B);
            \end{tikzpicture}%
            }
        \end{minipage}
        %2
        \begin{minipage}{0.33\linewidth}
            \scalebox{0.9}{
            \begin{Geometrie}[CoinHD={(5u,5u)}]
                trace feuillet withpen pencircle scaled 2bp;
                pair A,B,C,D;
                A=u*(4,0.5);
                B-A=u*(0.5,2);
                D-A=u*(-3.5,2);
                C-D=u*(0.5,2);
                trace polygone(A,B,C,D) withpen pencircle scaled 2bp;
                fill polygone(A,B,C,D) withcolor gris;
            \end{Geometrie}
            }
            \par\vspace*{-5mm}
            \tikz[remember picture,overlay]{\coordinate[name=A,xshift=25mm,yshift=13mm];}
            \par
            \tikz[remember picture,overlay]{\coordinate[name=B,xshift=25mm,yshift=1mm];}
            \par
            \scalebox{0.9}{
            \begin{Geometrie}[CoinHD={(5u,5u)}]
                trace feuillet withpen pencircle scaled 2bp;
                pair A,B,C,D;
            A=u*(4,0.5);
                B-A=u*(0.5,2);
                D-A=u*(-3.5,2);
                C-D=u*(0.5,2);
                trace A--B--C withpen pencircle scaled 1bp;
            \end{Geometrie}
            \begin{tikzpicture}[remember picture,overlay]%
                \draw[Gray,-{Triangle[width=25pt,length=10pt]}, line width=15pt](A) -- (B);
            \end{tikzpicture}%
            }
        \end{minipage}
        %3
        \begin{minipage}{0.33\linewidth}
            \scalebox{0.9}{
            \begin{Geometrie}[CoinHD={(5u,5u)}]
                trace feuillet withpen pencircle scaled 2bp;
                pair A,B,C,D;
                A=u*(4.5,0.5);
                B-A=u*(0,3);
                D-A=u*(-3.5,1);
                C-D=u*(0,3);
                trace polygone(A,B,C,D) withpen pencircle scaled 2bp;
                fill polygone(A,B,C,D) withcolor gris;
            \end{Geometrie}
            }
            \par\vspace*{-5mm}
            \tikz[remember picture,overlay]{\coordinate[name=A,xshift=25mm,yshift=13mm];}
            \par
            \tikz[remember picture,overlay]{\coordinate[name=B,xshift=25mm,yshift=1mm];}
            \par
            \scalebox{0.9}{
            \begin{Geometrie}[CoinHD={(5u,5u)}]
                trace feuillet withpen pencircle scaled 2bp;
                pair A,B,C,D;
            A=u*(4.5,0.5);
                B-A=u*(0,3);
                D-A=u*(-3.5,1);
                C-D=u*(0,3);
                trace D--A--B withpen pencircle scaled 1bp;
            \end{Geometrie}
            \begin{tikzpicture}[remember picture,overlay]%
                \draw[Gray,-{Triangle[width=25pt,length=10pt]}, line width=15pt](A) -- (B);
            \end{tikzpicture}%
            }
        \end{minipage}

        \partie[Une caractérisation du parallélogramme]
        \begin{enumerate}
            \item Placer trois points non alignés A, B et C, puis un point D pour que ABCD soit un parallélogramme.
            \item Faire une figure à main levée dans le cadre prévu à cet effet.
            \item En utilisant la règle non graduée et le compas, construire le parallélogramme ABCD.
        \end{enumerate}
        \fbox{
        \parbox{0.9\linewidth}{
        \begin{Geometrie}[CoinHD={(6u,5u)}]
            trace feuillet withpen pencircle scaled 2bp;
            pair A;
            A=u*(2.5,4.5);
            label(btex \textbf{Figure à main levée ici.} etex,A);
        \end{Geometrie}
        \vspace*{65mm}        
        }
        }
        \begin{enumerate}
            \setcounter{enumi}{3}
            \item En déduire une propriété \textbf{caractéristique} du parallélogramme.
        \end{enumerate}
        \fbox{\parbox{0.9\linewidth}{\phantom{rrr}\vspace*{45mm}}}
    \end{changemargin}
\end{activite}
