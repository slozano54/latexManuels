\section{Translation}

\begin{minipage}{0.6\linewidth}
    \begin{definition}
        $H$ est l’image de $G$ par la translation \textbf{qui transforme $\mathbf{A}$ en $\mathbf{B}$} signifie que $ABHG$ est un parallélogramme.
    \end{definition}
\end{minipage}
\begin{minipage}{0.4\linewidth}
    \hspace*{-10mm}
    \coursVecteurFigUn{0.5}
\end{minipage}

\begin{remarques}
    \begin{itemize}
        \item Une translation fait glisser une figure dans une direction, un sens et une longueur donnés.
        \item La translation qui transforme $A$ en $B$ se traduit par un glissement de $A$ vers $B$.
        % \item On parle de la translation de vecteur $\overrightarrow{AB}$.
    \end{itemize}
    \smallskip
    \hrefConstruction{https://www.geogebra.org/classic/efdft7qt}{Une animation Geogebra pour visualiser}
    \creditGeogebra{Gauthier}
\end{remarques}

\begin{notation}    
    Une translation est symbolisée par un vecteur (une flèche) qui donne la direction, le sens et la longueur de ce déplacement.
    
    \begin{Geometrie}[CoinHD={(11u,3u)}]
        % trace grille(0.5) withcolor gris;
        pair A,B;
        A=u*(3.5,1);
        B=u*(7.5,2);
        labeloffset := 5bp ;
        label.top(btex $A$ etex, A);
        label.top(btex $B$ etex, B);
        marque_p:="croix";
        pointe(A,B);
        trace droite(A,B) withcolor red;
        trace segment(A,B) withcolor black withpen pencircle scaled 2bp;        
        labeloffset := 6bp ;
        label.top(btex $(d)$ direction etex,1.5[B,A]) withcolor red;
        label.top(btex longueur etex,0.5[A,B]);
        fill B -- rotation(0.1[B,A],B,20) -- rotation(0.1[B,A],B,-20) -- cycle withcolor DarkGreen;
        label.bot(btex sens etex,0.9[A,B]) withcolor DarkGreen;
    \end{Geometrie}    
\end{notation}

\begin{methode*1}[Construction de l'image d'un quadrilatère par translation]    
    \exercice
    Construire l'image de $ACEG$ par la translation qui transforme $A$ en $B$.
    \correction
    \begin{minipage}{0.45\linewidth}
        \begin{center}
            \coursVecteurFigUn{0.4}
        \end{center}
    \end{minipage}
    \begin{minipage}{0.55\linewidth}
        \textbf{Programme de construction}

        Pour construire l'image de $ACEG$ par \mbox{la translation de vecteur $\overrightarrow{AB}$}, il faut reproduire ce vecteur à partir de chaque sommet de la figure.
    \end{minipage}
    \smallskip
    \hrefConstruction{http://lozano.maths.free.fr/iep_local/figures_html/scr_iep_122.html}{Image d'un point à l'aide du compas}
    \creditInstrumentPoche
\end{methode*1}
