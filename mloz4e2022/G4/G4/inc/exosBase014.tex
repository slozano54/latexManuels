\begin{exercice*}[Une frise]    
    \begin{enumerate}
        \item Constuire le symétrique du motif par la symétrie d'axe $(AE)$.
        \item Puis construire le translaté du motif obtenu par la translation qui transforme $A$ en $E$.
        \item Répéter l'opération avec le nouveau motif et ainsi de suite.
    \end{enumerate}
    \begin{Geometrie}[CoinHD={(9u,4u)}]
        labeloffset := 5bp ;
        trace papierpointe withcolor Grey;
        pair A,E,M[];
        A = u*(1,2);
        E = u*(2.5,2);
        marque_p:="croix";
        pointe(A,E);
        M0 = A;
        M1 = M0 shifted (0.5u,u);
        M2 = M1 shifted (-0.5u,0);
        M3 = M2 shifted (-u,0.5u);
        path motif[];
        motif[0] = M0--M1--M2--M3--cycle;
        draw motif[0];
        fill motif[0] withcolor LightGrey;
        label.lft(btex $A$ etex,A);
        label.lft(btex $E$ etex,E);           
    \end{Geometrie}
\end{exercice*}
\begin{corrige}
    %\setcounter{partie}{0} % Pour s'assurer que le compteur de \partie est à zéro dans les corrigés
    % \phantom{rrr}
    \begin{enumerate}
        \item Constuire le symétrique du motif par la symétrie d'axe $(AE)$.
        \item Puis construire le translaté du motif obtenu par la translation qui transforme $A$ en $E$.
        \item Répéter l'opération avec le nouveau motif et ainsi de suite.
    \end{enumerate}
    \begin{Geometrie}[CoinHD={(9u,4u)}]
        labeloffset := 5bp ;
        trace papierpointe withcolor Grey;
        pair A,E,M[];
        A = u*(1,2);
        E = u*(2.5,2);
        marque_p:="croix";
        pointe(A,E);
        M0 = A;
        M1 = M0 shifted (0.5u,u);
        M2 = M1 shifted (-0.5u,0);
        M3 = M2 shifted (-u,0.5u);
        path motif[];
        motif[0] = M0--M1--M2--M3--cycle;
        draw motif[0];
        fill motif[0] withcolor LightGrey;
        label.lft(btex $A$ etex,A);
        label.lft(btex $E$ etex,E);
        %%% Correction
        motif[1] = symetrie(motif[0],A,E);
        draw motif[1];        
        pair zAE;
        zAE=E-A;
        for l=1 upto 4:
            motif[2*l] = motif[2*l-2] shifted zAE;
            draw motif[2*l];
            motif[2*l+1] = motif[2*l+1-2] shifted zAE;
            draw motif[2*l+1];
        endfor;            
    \end{Geometrie}
\end{corrige}

