\begin{exercice*}
    On considère la grille ci-dessous, où les droites qui semblent parallèles le sont.

    \Reseau[%
        Colonnes=5,%
        Lignes=8,%
        Traces={%	
            pair A,B,C,D,E,F,G,H; % Placement dans le quadrillage en calculant
            A=ppreseau(0,8);
            B=ppreseau(1,8);
            C=ppreseau(1,7);
            D=ppreseau(0,7);
            E=ppreseau(2,5);
            F=ppreseau(1,2);
            G=ppreseau(3,2);
            H=ppreseau(4,6);
            marque_p:="croix";
            drawoptions(withcolor red);
            pointe(A,B,C,D,E,F,G,H);
            label.urt(btex $A$ etex,A);
            label.urt(btex $B$ etex,B);
            label.urt(btex $C$ etex,C);
            label.urt(btex $D$ etex,D);
            label.urt(btex $E$ etex,E);
            label.urt(btex $F$ etex,F);
            label.urt(btex $G$ etex,G);
            label.urt(btex $H$ etex,H);
            drawoptions();
        }]{}
    \begin{enumerate}
        \item Justifier la nature du quadrilatère $ABCD$.
        \item Déterminer l'image de $D$ par la translation qui transforme $C$ en $B$. Justifier.
        \item Déterminer l'image de $C$ par la translation qui transforme $B$ en $A$. Justifier.
    \end{enumerate}    
\end{exercice*}
\begin{corrige}
    %\setcounter{partie}{0} % Pour s'assurer que le compteur de \partie est à zéro dans les corrigés
    % \phantom{rrr}
    On considère la grille ci-dessous, où les droites qui semblent parallèles le sont.

    \scalebox{0.7}{
    \Reseau[%
        Colonnes=5,%
        Lignes=8,%
        Traces={%	
            pair A,B,C,D,E,F,G,H; % Placement dans le quadrillage en calculant
            A=ppreseau(0,8);
            B=ppreseau(1,8);
            C=ppreseau(1,7);
            D=ppreseau(0,7);
            E=ppreseau(2,5);
            F=ppreseau(1,2);
            G=ppreseau(3,2);
            H=ppreseau(4,6);
            marque_p:="croix";
            drawoptions(withcolor red);
            pointe(A,B,C,D,E,F,G,H);
            label.urt(btex $A$ etex,A);
            label.urt(btex $B$ etex,B);
            label.urt(btex $C$ etex,C);
            label.urt(btex $D$ etex,D);
            label.urt(btex $E$ etex,E);
            label.urt(btex $F$ etex,F);
            label.urt(btex $G$ etex,G);
            label.urt(btex $H$ etex,H);
            drawoptions();
        }]{}
    }
    
    \begin{enumerate}
        \item Justifier la nature du quadrilatère $ABCD$.\\
        {\red Les côtés opposés de du quadrilatère $ABCD$ sont parallèles deux à deux donc c'est un parallélogramme.}        
        \item Déterminer l'image de $D$ par la translation qui transforme $C$ en $B$. Justifier.\\
        {\red $CBAD$ étant un parallélogramme, $A$ est l'image de $D$ par la translation qui transforme $C$ en $B$.}
        \item Déterminer l'image de $C$ par la translation qui transforme $B$ en $A$. Justifier.\\
        {\red $BADC$ étant un parallélogramme, $D$ est l'image de $C$ par la translation qui transforme $B$ en $A$.}
    \end{enumerate}
\end{corrige}

