\begin{exercice*}
    On considère la grille ci-dessous, où les droites qui semblent parallèles le sont.

    \Reseau[%
        Colonnes=5,%
        Lignes=8,%
        Traces={%	
            pair A,B,C,D,E,F,G,H; % Placement dans le quadrillage en calculant
            A=ppreseau(0,8);
            B=ppreseau(1,8);
            C=ppreseau(1,7);
            D=ppreseau(0,7);
            E=ppreseau(2,5);
            F=ppreseau(1,2);
            G=ppreseau(3,2);
            H=ppreseau(4,6);
            marque_p:="croix";
            drawoptions(withcolor red);
            pointe(A,B,C,D,E,F,G,H);
            label.urt(btex $A$ etex,A);
            label.urt(btex $B$ etex,B);
            label.urt(btex $C$ etex,C);
            label.urt(btex $D$ etex,D);
            label.urt(btex $E$ etex,E);
            label.urt(btex $F$ etex,F);
            label.urt(btex $G$ etex,G);
            label.urt(btex $H$ etex,H);
            drawoptions();
        }]{}
\end{exercice*}
\begin{corrige}
    %\setcounter{partie}{0} % Pour s'assurer que le compteur de \partie est à zéro dans les corrigés
    % \phantom{rrr}
\dots
\end{corrige}

