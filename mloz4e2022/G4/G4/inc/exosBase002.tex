\begin{exercice*}
    On se base sur la figure de l'exercice {\bfseries 1}.

    \Reseau[%
    Colonnes=5,%
    Lignes=8,%
    Traces={%	
        pair A,B,C,D,E,F,G,H; % Placement dans le quadrillage en calculant
        A=ppreseau(0,8);
        B=ppreseau(1,8);
        C=ppreseau(1,7);
        D=ppreseau(0,7);
        E=ppreseau(2,5);
        F=ppreseau(1,2);
        G=ppreseau(3,2);
        H=ppreseau(4,6);
        marque_p:="croix";
        drawoptions(withcolor red);
        pointe(A,B,C,D,E,F,G,H);
        label.urt(btex $A$ etex,A);
        label.urt(btex $B$ etex,B);
        label.urt(btex $C$ etex,C);
        label.urt(btex $D$ etex,D);
        label.urt(btex $E$ etex,E);
        label.urt(btex $F$ etex,F);
        label.urt(btex $G$ etex,G);
        label.urt(btex $H$ etex,H);
        drawoptions();
    }]{}
    \begin{enumerate}
        \item On considère la translation qui transforme $F$ en $G$. Colorier  :
        \begin{itemize}
            \item en rouge, l'image du motif \num{42}, c'est le \pointilles
            \item en bleu, l'image du motif \num{32}, c'est le \pointilles
            \item en vert, l'image du motif \num{9}, c'est le \pointilles
        \end{itemize}
        \item On considère la translation qui transforme $E$ en $H$. Colorier  :
        \begin{itemize}
            \item en jaune, l'image du motif \num{42}, c'est le \pointilles
            \item en orange, l'image du motif \num{32}, c'est le \pointilles
            \item en gris, l'image du motif \num{9}, c'est le \pointilles
        \end{itemize}
    \end{enumerate}
    % \hrefMathalea{https://coopmaths.fr/alea/?uuid=3bfb6&id=4G11&n=3&d=10&s=1&s2=false&s3=8&cd=1}
    \hrefMathaleaVIII{48253}{eleve}{1211}
\end{exercice*}
\begin{corrige}
    %\setcounter{partie}{0} % Pour s'assurer que le compteur de \partie est à zéro dans les corrigés
    % \phantom{rrr}
    On se base sur la figure de l'exercice {\bfseries 1}.\\
    \begin{enumerate}
        \item On considère la translation qui transforme $F$ en $G$. Colorier  :
        \begin{itemize}
            \item en rouge, l'image du motif \num{42}.
            \item en bleu, l'image du motif \num{32}.
            \item en vert, l'image du motif \num{9}.
        \end{itemize}
        \item On considère la translation qui transforme $E$ en $H$. Colorier  :
        \begin{itemize}
            \item en jaune, l'image du motif \num{42}.
            \item en orange, l'image du motif \num{32}.
            \item en gris, l'image du motif \num{9}.
        \end{itemize}
    \end{enumerate}
    \scalebox{0.7}{
    \Reseau[%
    Colonnes=5,%
    Lignes=8,%
    Traces={%	
        pair A,B,C,D,E,F,G,H; % Placement dans le quadrillage en calculant
        A=ppreseau(0,8);
        B=ppreseau(1,8);
        C=ppreseau(1,7);
        D=ppreseau(0,7);
        E=ppreseau(2,5);
        F=ppreseau(1,2);
        G=ppreseau(3,2);
        H=ppreseau(4,6);
        % Coloriages
        alpha = 3/8; % alpha: 0=invisible, 1=opaque
        picture bg, fgrouge, fgbleu, fgvert, fgjaune, fgorange, fggris;
        bg = currentpicture; % save the current pic *before* filling the box
        % par F-> G image de 42 en rouge
        path boxrouge; boxrouge= H--ppreseau(4,5)--ppreseau(5,5)--ppreseau(5,6)--cycle;        
        fill boxrouge withcolor LightPink;
        fgrouge = image( 
        for e within bg:
            draw e if 5=colormodel e: withcolor alpha[colorpart e, LightPink] fi;
        endfor
        );
        clip fgrouge to boxrouge; 
        % par F-> G  image de 32 en bleu
        path boxbleu; boxbleu= E--ppreseau(3,5)--ppreseau(3,4)--ppreseau(2,4)--cycle;
        fill boxbleu withcolor LightBlue;
        fgbleu = image( 
            for e within bg:
                draw e if 5=colormodel e: withcolor alpha[colorpart e, LightBlue] fi;
            endfor
        );
        clip fgbleu to boxbleu;
        % par F-> G  image de 9 en vert
        path boxvert; boxvert= G--ppreseau(4,2)--ppreseau(4,1)--ppreseau(3,1)--cycle;
        fill boxvert withcolor LightGreen;
        fgvert = image( 
            for e within bg:
                draw e if 5=colormodel e: withcolor alpha[colorpart e, LightGreen] fi;
            endfor
        );
        clip fgvert to boxvert;
        % par E-> H image de 42 en jaune
        path boxjaune; boxjaune= H--ppreseau(4,7)--ppreseau(5,7)--ppreseau(5,6)--cycle;        
        fill boxrouge withcolor LightGoldenrodYellow;
        fgjaune = image( 
        for e within bg:
            draw e if 5=colormodel e: withcolor alpha[colorpart e, LightGoldenrodYellow] fi;
        endfor
        );
        clip fgjaune to boxjaune; 
        % par E-> H  image de 32 en orange
        path boxorange; boxorange= E--ppreseau(3,5)--ppreseau(3,6)--ppreseau(2,6)--cycle;
        fill boxorange withcolor LightSalmon;
        fgorange = image( 
            for e within bg:
                draw e if 5=colormodel e: withcolor alpha[colorpart e, LightSalmon] fi;
            endfor
        );
        clip fgorange to boxorange;
        % par E-> H  image de 9 en gris
        path boxgris; boxgris= G--ppreseau(4,2)--ppreseau(4,3)--ppreseau(3,3)--cycle;
        fill boxgris withcolor LightGrey;
        fggris = image( 
            for e within bg:
                draw e if 5=colormodel e: withcolor alpha[colorpart e, LightGrey] fi;
            endfor
        );
        clip fggris to boxgris;
        % On finit
        fill boxrouge withcolor LightPink;
        fill boxbleu withcolor LightBlue;
        fill boxvert withcolor LightGreen;
        fill boxjaune withcolor LightGoldenrodYellow;
        fill boxorange withcolor LightSalmon;
        fill boxgris withcolor LightGrey;
        draw fgrouge;
        draw fgbleu;
        draw fgvert;
        draw fgjaune;
        draw fgorange;
        draw fggris;
        % Les points
        marque_p:="croix";
        drawoptions(withcolor red);
        pointe(A,B,C,D,E,F,G,H);
        label.urt(btex $A$ etex,A);
        label.urt(btex $B$ etex,B);
        label.urt(btex $C$ etex,C);
        label.urt(btex $D$ etex,D);
        label.urt(btex $E$ etex,E);
        label.urt(btex $F$ etex,F);
        label.urt(btex $G$ etex,G);
        label.urt(btex $H$ etex,H);
        drawoptions();
    }]{}
    }
\end{corrige}

