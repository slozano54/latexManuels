\begin{exercice*}[Soustraire deux entiers relatifs]
    Transformer la soustraction en addition puis calculer.
    \begin{multicols}2
        \begin{enumerate}            
            \item $ (-6) - (-9) $
            \item $ (-16) - (+4) $
            \item $ (+12) - (-2) $
            \item $ (-12) - (+20) $
            \item $ (+5) - (-4) $
            \item $ (-15) - (-10) $
            \item $ (-16) - (+14) $
            \item $ (+5) - (-11) $
        \end{enumerate}
    \end{multicols}
\end{exercice*}
\begin{corrige}
    %\setcounter{partie}{0} % Pour s'assurer que le compteur de \partie est à zéro dans les corrigés
    \phantom{rrr}    
    \begin{multicols}2
        \begin{enumerate}
            \item $ (-6) - (-9) = \mathbf{{\color[HTML]{f15929}(-6)}} + \mathbf{{\color{blue}(+9)}} = \mathbf{{\color{blue}(+3)}} $
            \item $ (-16) - (+4) = \mathbf{{\color[HTML]{f15929}(-16)}} + \mathbf{{\color[HTML]{f15929}(-4)}} = \mathbf{{\color[HTML]{f15929}(-20)}} $
            \item $ (+12) - (-2) = \mathbf{{\color{blue}(+12)}} + \mathbf{{\color{blue}(+2)}} = \mathbf{{\color{blue}(+14)}} $
            \item $ (-12) - (+20) = \mathbf{{\color[HTML]{f15929}(-12)}} + \mathbf{{\color[HTML]{f15929}(-20)}} = \mathbf{{\color[HTML]{f15929}(-32)}} $
            \item $ (+5) - (-4) = \mathbf{{\color{blue}(+5)}} + \mathbf{{\color{blue}(+4)}} = \mathbf{{\color{blue}(+9)}} $
            \item $ (-15) - (-10) = \mathbf{{\color[HTML]{f15929}(-15)}} + \mathbf{{\color{blue}(+10)}} = \mathbf{{\color[HTML]{f15929}(-5)}} $
            \item $ (-16) - (+14) = \mathbf{{\color[HTML]{f15929}(-16)}} + \mathbf{{\color[HTML]{f15929}(-14)}} = \mathbf{{\color[HTML]{f15929}(-30)}} $
            \item $ (+5) - (-11) = \mathbf{{\color{blue}(+5)}} + \mathbf{{\color{blue}(+11)}} = \mathbf{{\color{blue}(+16)}} $
        \end{enumerate}
    \end{multicols}
\end{corrige}