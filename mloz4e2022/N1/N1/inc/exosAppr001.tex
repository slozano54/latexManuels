\begin{exercice*}[Signe d'un produit de deux relatifs]    
    % \begin{multicols}2
        \begin{enumerate}
            \item Donne le signe de $ m $ pour que A soit negatif.
            
            A = $ (-3) \times (+13)\times m\times (-8) $
            \item Donne le signe de $ n $ pour que B soit negatif. 
            
            B = $ (-9) \times n\times (-8)\times (-6)\times (-10) $ 
            \item Donne le signe de $ n $ pour que C soit positif. 
            
            C = $ (-4) \times (-11)\times n $
        \end{enumerate}
    % \end{multicols}

    \href{https://coopmaths.fr/mathalea.html?ex=4C10-6,s=1,n=3,cd=1,i=1&v=l}{Cliquer pour vous entraîner sur le site \mathaleaLogo} 
\end{exercice*}
\begin{corrige}
    %\setcounter{partie}{0} % Pour s'assurer que le compteur de \partie est à zéro dans les corrigés
    \phantom{rrr}    
    % \begin{multicols}2
        \begin{enumerate}
            \item {\bfseries \color[HTML]{f15929}Supposons que m soit positif : }\\
            Il y a 2 facteurs négatifs, le nombre de facteurs négatifs est pair donc le produit est positif.
           
           \medskip
            Donc si {\bfseries \color{black}m est positif} $ (-3) \times (+13)\times m\times (-8) $ est {\bfseries \color{black}positif}.
           
           \medskip
            {\bfseries \color[HTML]{f15929}Supposons maintenant que m soit négatif : }
           
           \medskip
            Il y a 3 facteurs négatifs, le nombre de facteurs négatifs est impair donc le produit est négatif.
           
           \medskip
            Donc si {\bfseries \color{black}m est négatif} $ (-3) \times (+13)\times m\times (-8) $ est {\bfseries \color{black}négatif}.
           
           \medskip
            {\bfseries \color[HTML]{f15929}Conclusion :} \\
           {\bfseries \color{black}Il faut donc que $ m $ soit négatif pour que A soit négatif}
               \item {\bfseries \color[HTML]{f15929}Supposons que n soit positif : }\\
            Il y a 4 facteurs négatifs, le nombre de facteurs négatifs est pair donc le produit est positif.
           
           \medskip
            Donc si {\bfseries \color{black}n est positif} $ (-9) \times n\times (-8)\times (-6)\times (-10) $ est {\bfseries \color{black}positif}.
           
           \medskip
            {\bfseries \color[HTML]{f15929}Supposons maintenant que n soit négatif : }
           
           \medskip
            Il y a 5 facteurs négatifs, le nombre de facteurs négatifs est impair donc le produit est négatif.
           
           \medskip
            Donc si {\bfseries \color{black}n est négatif} $ (-9) \times n\times (-8)\times (-6)\times (-10) $ est {\bfseries \color{black}négatif}.
           
           \medskip
            {\bfseries \color[HTML]{f15929}Conclusion :} \\
           {\bfseries \color{black}Il faut donc que $ n $ soit négatif pour que B soit négatif}
               \item {\bfseries \color[HTML]{f15929}Supposons que n soit positif : }\\
            Il y a 2 facteurs négatifs, le nombre de facteurs négatifs est pair donc le produit est positif.
           
           \medskip
            Donc si {\bfseries \color{black}n est positif} $ (-4) \times (-11)\times n $ est {\bfseries \color{black}positif}.
           
           \medskip
            {\bfseries \color[HTML]{f15929}Supposons maintenant que n soit négatif : }
           
           \medskip
            Il y a 3 facteurs négatifs, le nombre de facteurs négatifs est impair donc le produit est négatif.
           
           \medskip
            Donc si {\bfseries \color{black}n est négatif} $ (-4) \times (-11)\times n $ est {\bfseries \color{black}négatif}.
           
           \medskip
            {\bfseries \color[HTML]{f15929}Conclusion :} \\
           {\bfseries \color{black}Il faut donc que $ n $ soit positif pour que C soit positif}
        \end{enumerate}
    % \end{multicols}
\end{corrige}

