\begin{exercice*}[Produits à trou]
    Calculer.
    \begin{multicols}2
        \begin{enumerate}
            \item $ \ldots\ldots\ldots \times (-10) = (-50) $
            \item $ \ldots\ldots\ldots \times (-1) = (-9) $
            \item $ \ldots\ldots\ldots \times (+4) = (-16) $
            \item $ (-1) \times \ldots\ldots\ldots = (+3) $
            \item $ (+2) \times \ldots\ldots\ldots = (-18) $
            \item $ (-7) \times \ldots\ldots\ldots = (-14) $
            \item $ \ldots\ldots\ldots \times (-9) = (+90) $
            \item $ (+8) \times \ldots\ldots\ldots = (-40) $
        \end{enumerate}
    \end{multicols}

    % \href{https://coopmaths.fr/mathalea.html?ex=4C10-3,s=10,s2=false,n=10,i=1&v=l}{Cliquer pour vous entraîner sur le site \mathaleaLogo} 
    \hrefMathalea{https://coopmaths.fr/mathalea.html?ex=4C10-10,s=10,s2=false,n=8,i=1&v=l}    
\end{exercice*}
\begin{corrige}
    %\setcounter{partie}{0} % Pour s'assurer que le compteur de \partie est à zéro dans les corrigés
    \phantom{rrr}    
    \begin{multicols}2
        \begin{enumerate}
            \item $ \mathbf{{\color{blue}(+5)}} \times \mathbf{{\color[HTML]{f15929}(-10)}} = \mathbf{{\color[HTML]{f15929}(-50)}} $
            \item $ \mathbf{{\color{blue}(+9)}} \times \mathbf{{\color[HTML]{f15929}(-1)}} = \mathbf{{\color[HTML]{f15929}(-9)}} $
            \item $ \mathbf{{\color[HTML]{f15929}(-4)}} \times \mathbf{{\color{blue}(+4)}} = \mathbf{{\color[HTML]{f15929}(-16)}} $
            \item $ \mathbf{{\color[HTML]{f15929}(-1)}} \times \mathbf{{\color[HTML]{f15929}(-3)}} = \mathbf{{\color{blue}(+3)}} $
            \item $ \mathbf{{\color{blue}(+2)}} \times \mathbf{{\color[HTML]{f15929}(-9)}} = \mathbf{{\color[HTML]{f15929}(-18)}} $
            \item $ \mathbf{{\color[HTML]{f15929}(-7)}} \times \mathbf{{\color{blue}(+2)}} = \mathbf{{\color[HTML]{f15929}(-14)}} $
            \item $ \mathbf{{\color[HTML]{f15929}(-10)}} \times \mathbf{{\color[HTML]{f15929}(-9)}} = \mathbf{{\color{blue}(+90)}} $
            \item $ \mathbf{{\color{blue}(+8)}} \times \mathbf{{\color[HTML]{f15929}(-5)}} = \mathbf{{\color[HTML]{f15929}(-40)}} $      
        \end{enumerate}
    \end{multicols}
\end{corrige}