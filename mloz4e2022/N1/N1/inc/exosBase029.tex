\begin{exercice*}[Priorités opératoires]
    Calculer.
    \begin{itemize}
        \item[] $ A = 25\div(-5)\times4$
        \item[] $ B = (-8+9-5)\times(-5)$
        \item[] $ C = 5\times4\div(-30+28)$
        \item[] $ D = -2\times(-2)\times(22-25)$
        \item[] $ E = 7\times(-8)-27\div9$
    \end{itemize}

    \hrefAleaTeX{https://urls.mathslozano.fr/4n12025ex29}
\end{exercice*}
\begin{corrige}
    %\setcounter{partie}{0} % Pour s'assurer que le compteur de \partie est à zéro dans les corrigés
    \phantom{rrr}
    \begin{multicols}2
    \begin{list}{}{}
        \item 
        $ A = \mathbf{{\color[HTML]{f15929}25\div(-5)}}\times4$ \\
        $ A = -5\times4$ \\
        $ \psshadowbox{A = -20} $ 
        \item 
        $ B = (\mathbf{{\color[HTML]{f15929}-8+9-5}})\times(-5)$ \\
        $ B = -4\times(-5)$ \\
        $ \psshadowbox{B = 20} $ 
        \columnbreak
        \item 
        $ C = 5\times4\div(\mathbf{{\color[HTML]{f15929}-30+28}})$ \\
        $ C = \mathbf{{\color[HTML]{f15929}5\times4}}\div(-2)$ \\
        $ C = 20\div(-2)$ \\
        $ \psshadowbox{C = -10} $ 
        \item 
        $ D = -2\times(-2)\times(\mathbf{{\color[HTML]{f15929}22-25}})$ \\
        $ D = -2\times(-2)\times(-3)$ \\
        $ \psshadowbox{D = -12} $ 
        \item
        $ E = 7\mathbf{{\color[HTML]{f15929}\times}}(-8)-27\mathbf{{\color[HTML]{f15929}\div}}9$ \\
        $ E = -56-3$ \\
        $ \psshadowbox{E = -59} $ 
    \end{list}
\end{multicols}
\end{corrige}

