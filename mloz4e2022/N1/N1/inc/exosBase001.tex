\begin{exercice*}[Comparer des entiers]
    Comparer les nombres suivants.
    \begin{multicols}2
        \begin{enumerate}
            \item $+4$ et $+6$
            \item $-6$ et $-2$
            \item $-2$ et $-4$
            \item $0$ et $-8$
            \item $+3$ et $-4$
            \item $+4$ et $-14$
            \item $-12$ et $-18$
            \item $-212$ et $+212$
        \end{enumerate}
    \end{multicols}
\end{exercice*}
\begin{corrige}
    %\setcounter{partie}{0} % Pour s'assurer que le compteur de \partie est à zéro dans les corrigés
    \phantom{rrr}    
    
    \begin{enumerate}
        \item $+4<+6$ : Du côté positif, le nombre le plus éloigné de zéro est le plus grand.
        \item $-6<-2$ : Du côté négatif, le nombre le plus éloigné de zéro est le plus petit.
        \item $-2>-4$
        \item $0>-8$ : Un nombre négatif est un nombre inférieur à zéro.
        \item $+3>-4$ : Un nombre positif est toujours plus grand qu'un nombre négatif.
        \item $+4>-14$
        \item $-12>-18$
        \item $-212<+212$
    \end{enumerate}    
\end{corrige}

