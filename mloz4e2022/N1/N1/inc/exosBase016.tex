\begin{exercice*}[Signe d'un produit de relatifs]
    Pour chacun des produits suivants :
    \begin{itemize}
        \item Donner le signe du produit à l'aide d'un schéma.
        \item Indiquer le nombre de facteurs négatifs.
    \end{itemize}
    Une fois tous les produits traités, essayer de faire le lien entre le signe du produit et le nombre de facteur négatifs.
    \begin{enumerate}
            \item $ (-13) \times (-10) \times (-14) \times (-16) $
            \item $ (+20) \times (-11) \times (+15) $
            \item $ (+19) \times (+11) $
            \item $ (-1) \times (+12) $
            \item $ (+14) \times (-2) \times (-18) \times (-1) $
            \item $ (-8) \times (-1) \times (+19) $
            \item $ (+17) \times (-4) $
            \item $ (-4) \times (-10) \times (+4) \times (+14) $
        \end{enumerate}
    
    % \href{https://coopmaths.fr/mathalea.html?ex=4C10-1,s=1,n=8,i=1&v=l}{Cliquer pour vous entraîner sur le site \mathaleaLogo} 
    \hrefMathalea{https://coopmaths.fr/mathalea.html?ex=4C10-1,s=1,n=8,i=1&v=l}
\end{exercice*}
\begin{corrige}
    %\setcounter{partie}{0} % Pour s'assurer que le compteur de \partie est à zéro dans les corrigés
    \phantom{rrr}    
    \begin{multicols}2
        \begin{enumerate}
            \item $ (-13) $ est négatif, $ (-10) $ est négatif, $ (-14) $ est négatif et $ (-16) $ est négatif.\\
            Il y a 4 facteurs négatifs, le nombre de facteurs négatifs est pair donc le produit est positif.\\
           Donc $ (-13) \times (-10) \times (-14) \times (-16) $ est {\bfseries \color[HTML]{f15929}positif}.
               \item $ (+20) $ est positif, $ (-11) $ est négatif et $ (+15) $ est positif.\\
            Il y a 1 facteur négatif, le nombre de facteurs négatifs est impair donc le produit est négatif.\\
           Donc $ (+20) \times (-11) \times (+15) $ est {\bfseries \color[HTML]{f15929}négatif}.
               \item $ (+19) $ est positif et $ (+11) $ est positif.\\
            Les deux facteurs ont le même signe donc le produit est positif.\\
           Donc $ (+19) \times (+11) $ est {\bfseries \color[HTML]{f15929}positif}.
               \item $ (-1) $ est négatif et $ (+12) $ est positif.\\
            Les deux facteurs ont un signe différent donc le produit est négatif.\\
           Donc $ (-1) \times (+12) $ est {\bfseries \color[HTML]{f15929}négatif}.
               \item $ (+14) $ est positif, $ (-2) $ est négatif, $ (-18) $ est négatif et $ (+1) $ est positif.\\
            Il y a 2 facteurs négatifs, le nombre de facteurs négatifs est pair donc le produit est positif.\\
           Donc $ (+14) \times (-2) \times (-18) \times (+1) $ est {\bfseries \color[HTML]{f15929}positif}.
               \item $ (-8) $ est négatif, $ (-1) $ est négatif et $ (+19) $ est positif.\\
            Il y a 2 facteurs négatifs, le nombre de facteurs négatifs est pair donc le produit est positif.\\
           Donc $ (-8) \times (-1) \times (+19) $ est {\bfseries \color[HTML]{f15929}positif}.
               \item $ (+17) $ est positif et $ (-4) $ est négatif.\\
            Les deux facteurs ont un signe différent donc le produit est négatif.\\
           Donc $ (+17) \times (-4) $ est {\bfseries \color[HTML]{f15929}négatif}.
               \item $ (-4) $ est négatif, $ (-10) $ est négatif, $ (+4) $ est positif et $ (+14) $ est positif.\\
            Il y a 2 facteurs négatifs, le nombre de facteurs négatifs est pair donc le produit est positif.\\
           Donc $ (-4) \times (-10) \times (+4) \times (+14) $ est {\bfseries \color[HTML]{f15929}positif}.
        \end{enumerate}
    \end{multicols}
\end{corrige}