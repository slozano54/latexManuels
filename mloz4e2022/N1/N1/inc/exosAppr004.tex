\begin{exercice*}[Signe d'un produit de deux relatifs]    
    % \begin{multicols}2
        \begin{enumerate}
            \item Donne le signe de $ n $ pour que A soit negatif. \\
            A = $ \dfrac {(+1)\times (+8)\times (+7)\times n}{(-3)\times (+14)} $ 
            \item Donne le signe de $ n $ pour que B soit positif. \\
            B = $ \dfrac {(-3)\times (-16)\times (+13)}{(+6)\times n\times (+7)} $ 
            \item Donne le signe de $ a $ pour que C soit negatif. \\
            C = $ \dfrac {(+6)\times (+6)\times (+20)}{(+11)\times (+12)\times a} $ 
            \item Donne le signe de $ a $ pour que D soit positif. \\
            D = $ \dfrac {(+7)\times (-16)\times (-10)\times a\times (+1)}{(-2)} $ 
        \end{enumerate}
    % \end{multicols}
    
    \hrefMathalea{https://coopmaths.fr/mathalea.html?ex=4C10-6,s=2,n=4,cd=0,i=1&v=l}
\end{exercice*}
\begin{corrige}
    %\setcounter{partie}{0} % Pour s'assurer que le compteur de \partie est à zéro dans les corrigés
    % \begin{multicols}2
        \begin{enumerate}
            \item {\bfseries \color[HTML]{f15929}Supposons que n soit positif : }\\
            Quand on compte les facteurs négatifs du numérateur et du dénominateur, on trouve 1, ce nombre est impair donc le quotient est négatif.
           
           \medskip
            Donc si {\bfseries \color{black}n est positif} $ \dfrac {(+1)\times (+8)\times (+7)\times n}{(-3)\times (+14)} $ est {\bfseries \color{black}négatif}.
           
           \medskip
            {\bfseries \color[HTML]{f15929}Supposons maintenant que n soit négatif : }\\
            Quand on compte les facteurs négatifs du numérateur et du dénominateur, on trouve 2, ce nombre est pair donc le quotient est positif.
           
           \medskip
            Donc si {\bfseries \color{black}n est négatif} $ \dfrac {(+1)\times (+8)\times (+7)\times n}{(-3)\times (+14)} $ est {\bfseries \color{black}positif}.
           
           \medskip
            {\bfseries \color[HTML]{f15929}Conclusion :} \\
           {\bfseries \color{black}Il faut donc que $ n $ soit positif pour que A soit négatif}
            \item {\bfseries \color[HTML]{f15929}Supposons que n soit positif : }\\
            Quand on compte les facteurs négatifs du numérateur et du dénominateur, on trouve 2, ce nombre est pair donc le quotient est positif.
           
           \medskip
            Donc si {\bfseries \color{black}n est positif} $ \dfrac {(-3)\times (-16)\times (+13)}{(+6)\times n\times (+7)} $ est {\bfseries \color{black}positif}.
           
           \medskip
            {\bfseries \color[HTML]{f15929}Supposons maintenant que n soit négatif : }\\
            Quand on compte les facteurs négatifs du numérateur et du dénominateur, on trouve 3, ce nombre est impair donc le quotient est négatif.
           
           \medskip
            Donc si {\bfseries \color{black}n est négatif} $ \dfrac {(-3)\times (-16)\times (+13)}{(+6)\times n\times (+7)} $ est {\bfseries \color{black}négatif}.
           
           \medskip
            {\bfseries \color[HTML]{f15929}Conclusion :} \\
           {\bfseries \color{black}Il faut donc que $ n $ soit positif pour que B soit positif}
            \item {\bfseries \color[HTML]{f15929}Supposons que a soit positif : }\\
            Tous les facteurs du numérateur et tous les facteurs du dénominateur sont positifs donc le quotient est positif.
           
           \medskip
            Donc si {\bfseries \color{black}a est positif} $ \dfrac {(+6)\times (+6)\times (+20)}{(+11)\times (+12)\times a} $ est {\bfseries \color{black}positif}.
           
           \medskip
            {\bfseries \color[HTML]{f15929}Supposons maintenant que a soit négatif : }\\
            Quand on compte les facteurs négatifs du numérateur et du dénominateur, on trouve 1, ce nombre est impair donc le quotient est négatif.
           
           \medskip
            Donc si {\bfseries \color{black}a est négatif} $ \dfrac {(+6)\times (+6)\times (+20)}{(+11)\times (+12)\times a} $ est {\bfseries \color{black}négatif}.
           
           \medskip
            {\bfseries \color[HTML]{f15929}Conclusion :} \\
           {\bfseries \color{black}Il faut donc que $ a $ soit négatif pour que C soit négatif}
            \item {\bfseries \color[HTML]{f15929}Supposons que a soit positif : }\\
            Quand on compte les facteurs négatifs du numérateur et du dénominateur, on trouve 3, ce nombre est impair donc le quotient est négatif.
           
           \medskip
            Donc si {\bfseries \color{black}a est positif} $ \dfrac {(+7)\times (-16)\times (-10)\times a\times (+1)}{(-2)} $ est {\bfseries \color{black}négatif}.
           
           \medskip
            {\bfseries \color[HTML]{f15929}Supposons maintenant que a soit négatif : }\\
            Quand on compte les facteurs négatifs du numérateur et du dénominateur, on trouve 4, ce nombre est pair donc le quotient est positif.
           
           \medskip
            Donc si {\bfseries \color{black}a est négatif} $ \dfrac {(+7)\times (-16)\times (-10)\times a\times (+1)}{(-2)} $ est {\bfseries \color{black}positif}.
           
           \medskip
            {\bfseries \color[HTML]{f15929}Conclusion :} \\
           {\bfseries \color{black}Il faut donc que $ a $ soit négatif pour que D soit positif}
           
        \end{enumerate}
    % \end{multicols}
\end{corrige}

