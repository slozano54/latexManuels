\section{Quotient de deux nombres relatifs}

\begin{definition}[Quotient de deux relatifs]
  Le nombre $x$ qui vérifie $ax=b$, avec $a\not=0$, s'appelle \MotDefinition{le quotient}{} de $b$ par $a$.
\end{definition}

\begin{notation}
  Le quotient de $b$ par $a$ se note $\dfrac{b}{a}$. 
\end{notation}

\begin{remarques}
  \begin{itemize}
    \item Si le nombre $x$ vérifie $ax=b$, avec $a\not=0$, alors $x=\dfrac{b}{a}$
    \item Le quotient de $b$ par $a$, avec $a\not=0$, est le nombre qui (lorqu'il est) multiplié par $a$ donne $b$.
  \end{itemize}  
\end{remarques}

\begin{propriete}[Signe d'un quotient de deux nombres de même signe \admise]
  Si on effectue le \textbf{quotient} de deux nombres relatifs de \textbf{m\^{e}me signe} alors il est \textbf{positif}.
\end{propriete}

\begin{propriete}[Signe d'un quotient de deux nombres de signes contraires \admise]
  Si on effectue le \textbf{quotient} de deux nombres relatifs de \textbf{signes contraires} alors il est \textbf{n\'egatif}
\end{propriete}

\begin{exemple*1}
  Déterminer les signes de : $\dfrac{(+7)}{(+3)}$\hfill ;\hfill$\dfrac{(-2)}{(-3)}$\hfill ;\hfill $\dfrac{(-4)}{(+6)}$\hfill et \hfill $\dfrac{(+7)}{(-3)}$
  \correction
  \setstretch{1.5}
  \begin{itemize}
    \item $\dfrac{(+7)}{(+3)}$ et $\dfrac{(-2)}{(-3)}$ sont des quotients de deux nombres de même signe.
    
    \textbf{Ces deux qotients sont donc positifs}.
    \item $\dfrac{(-4)}{(+6)}$ et $\dfrac{(+7)}{(-3)}$ sont des quotients de deux nombres de signes contraires.
    
    \textbf{Ces deux quotients sont donc négatifs}.
  \end{itemize}
  \setstretch{1}
\end{exemple*1}

\begin{propriete}[Distance à zéro d'un quotient \admise]
  Le \textbf{quotient} de deux nombres relatifs $b$ et $a$ avec $a\not=0$ est un nombre relatif 
  dont \textbf{la distance à zéro} est égale au quotient des distances à zéro de $b$ et de $a$.
\end{propriete}

\begin{exemple*1}
  Déterminer la distance à zéro de $\dfrac{(-7)}{(+4)}$
  \correction  
  \begin{itemize}
    \item La distance à zéro de $(-7)$ vaut $7$.
    \item La distance à zéro de $(+4)$ vaut $4$.
  \end{itemize}  
  Donc la distance à zéro de $\dfrac{(-7)}{(+4)}$ vaut $\dfrac{7}{4}$

  Ici $\dfrac{7}{4}$ a une écriture décimale donc la distance à zéro de $\dfrac{(-7)}{(+4)}$ vaut aussi $1,75$.
\end{exemple*1}

\begin{methode*1}[Calculer le quotient de deux relatifs]
  Pour calculer le quotient de deux nombres relatifs :
  \begin{enumerate}
    \item On détermine son signe.
    \item On calcule sa distance à zéro
  \end{enumerate}
  \exercice
  Calculer $\dfrac{(-7)}{(+4)}$
  \correction
  \begin{enumerate}
    \item $\dfrac{(-7)}{(+4)}$ est un quotient de deux nombres de signes contraires donc il est \textbf{négatif}.
    \item La distance à zéro de $\dfrac{(-7)}{(+4)}$ est égale au quotient des distances à zéro de $(-7)$ et de $(+4)$,
    soit $\dfrac{7}{4} = 1,75$
  \end{enumerate}
  Donc $\dfrac{(-7)}{(+4)} = -\dfrac{7}{4} = -1,75$
\end{methode*1}

\begin{definition}[Inverse d'un nombre relatif non nul]
  \textbf{L'inverse} d'un nombre relatif $x$, avec $x\not=0$, est le quotient de 1 par $x$.
\end{definition}

\begin{notation}
  L'inverse de $x$, avec $x\not=0$ se note $\dfrac{1}{x}$ ou $x^{-1}$.
\end{notation}

\begin{remarque}
  C'est le nombre qui (lorsqu'il est) multiplié par $x$ donne 1.
  $$x\times\frac{1}{x}=1$$
\end{remarque}

\begin{exemple*1}
  Déterminer l'inverse de $\dfrac{1}{-2}$
  \correction
  $(-2)\times \dfrac{1}{-2}=1$ donc $(-2)$ est l'inverse de $\dfrac{1}{-2 }$
  
  \begin{remarque}
    Tout comme $\dfrac{1}{-2 }$ est l'inverse de $(-2)$ !
  \end{remarque}
\end{exemple*1}

