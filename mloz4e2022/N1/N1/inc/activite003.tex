\phantomsection \label{N1activite003}
\begin{activite}[Produit de deux nombres relatifs]           
    Il faut distinguer deux cas :
    \begin{itemize}
        \item Les deux nombres sont de \textbf{signes contraires}.
        \item Les deux nombres sont de \textbf{même signe}.
    \end{itemize}

    \begin{enumerate}
        \item Produit de deux nombres de signes contraires, par exemple $3\times(-7)$
        \begin{enumerate}
            \item Calculer $(-7)+(-7)+(-7)$
            \item En déduire $3\times (-7)$
        \end{enumerate}
        \item Justifier que $(-4)\times 5=-20$
        \item Si l'un des facteurs n'est pas entier, par exemple $1,5\times (-3)$
        \begin{enumerate}
            \item Calculer $1,5\times ((-3)+3)$
            \item Développer sans calculer l'expression $1,5\times ((-3)+3)$ à l'aide de la distributivité.
            \item En déduire $1,5\times (-3)$
        \end{enumerate}
        \item Produit de deux nombres de même signe
        \begin{enumerate}
            \item Nous savons déjà le faire pour deux nombres positifs.
            \item Si les deux nombres sont négatifs, par exemple $(-2)\times (-7)$
            \begin{enumerate}
                \item Calculer $(-2)\times ((-7)+7)$
                \item Développer sans calculer l'expression $(-2)\times ((-7)+7)$ à l'aide de la distributivité.
                \item En déduire $(-2)\times (-7)$
            \end{enumerate}
        \end{enumerate}
    \end{enumerate}

    \textbf{Compléter la trace écrite.}

    \textbf{Voir la correction p. \hyperref[corrN1activite003]{\pageref{corrN1activite003}}}

\end{activite}
