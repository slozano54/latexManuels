\begin{exercice*}[Additions à trou]
    Compléter les additions suivantes.
    \begin{multicols}2
        \begin{enumerate}            
            \item $ (-3) + \ldots\ldots = (-16) $
            \item $ \ldots\ldots + (+2) = (-14) $
            \item $ \ldots\ldots + (+1) = (-14) $
            \item $ (-4) + \ldots\ldots = (+13) $
            \item $ (-3) + \ldots\ldots = (-12) $
            \item $ (-14) + \ldots\ldots = (-26) $
            \item $ \ldots\ldots + (-2) = (-3) $
            \item $ \ldots\ldots + (-17) = (-15) $          
        \end{enumerate}
    \end{multicols}

    \href{https://coopmaths.fr/mathalea.html?ex=5R20-2,s=20,s2=false,n=10,i=1&v=ex&z=1}{Cliquer pour vous entraîner sur le site \mathaleaLogo} 
\end{exercice*}
\begin{corrige}
    %\setcounter{partie}{0} % Pour s'assurer que le compteur de \partie est à zéro dans les corrigés
    \phantom{rrr}    
    \begin{multicols}2
        \begin{enumerate}
            \item $ \mathbf{{\color[HTML]{f15929}(-3)}} + \mathbf{{\color[HTML]{f15929}(-13)}} = \mathbf{{\color[HTML]{f15929}(-16)}} $
            \item $ \mathbf{{\color[HTML]{f15929}(-16)}} + \mathbf{{\color{blue}(+2)}} = \mathbf{{\color[HTML]{f15929}(-14)}} $
            \item $ \mathbf{{\color[HTML]{f15929}(-15)}} + \mathbf{{\color{blue}(+1)}} = \mathbf{{\color[HTML]{f15929}(-14)}} $
            \item $ \mathbf{{\color[HTML]{f15929}(-4)}} + \mathbf{{\color{blue}(+17)}} = \mathbf{{\color{blue}(+13)}} $
            \item $ \mathbf{{\color[HTML]{f15929}(-3)}} + \mathbf{{\color[HTML]{f15929}(-9)}} = \mathbf{{\color[HTML]{f15929}(-12)}} $
            \item $ \mathbf{{\color[HTML]{f15929}(-14)}} + \mathbf{{\color[HTML]{f15929}(-12)}} = \mathbf{{\color[HTML]{f15929}(-26)}} $
            \item $ \mathbf{{\color[HTML]{f15929}(-1)}} + \mathbf{{\color[HTML]{f15929}(-2)}} = \mathbf{{\color[HTML]{f15929}(-3)}} $
            \item $ \mathbf{{\color{blue}(+2)}} + \mathbf{{\color[HTML]{f15929}(-17)}} = \mathbf{{\color[HTML]{f15929}(-15)}} $
        \end{enumerate}   
    \end{multicols}
\end{corrige}