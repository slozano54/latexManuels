% Les enigmes ne sont pas numérotées par défaut donc il faut ajouter manuellement la numérotation
% si on veut mettre plusieurs enigmes
\refstepcounter{exercice}
\numeroteEnigme
\begin{enigme}[Pyramides de nombres additives]
    \begin{enumerate}
        \item Compléter les pyramides sachant que chaque nombre est la somme
        des nombres se trouvant dans les deux cases juste en dessous.    

        \PyramideNombre[Etages=4,Hauteur=8mm,Largeur=1.5cm]{-1,\num{6.5},3,~,~,~,\num{-5.5},~,~,~}
        \hfill
        \PyramideNombre[Etages=4,Hauteur=8mm,Largeur=1.5cm]{7-3,5-1-9,\num{3.1}-\num{2.8},\num{-0.1}-\num{1.4},~,~,~,~,~,~}

        \item Inventer un contenu et échanger avec son voisin.
    
        \PyramideNombre[Etages=4,Hauteur=8mm,Largeur=1.5cm]{~,~,~,~,~,~,~,~,~,~}
        \hfill
        \PyramideNombre[Etages=4,Hauteur=8mm,Largeur=1.5cm]{~,~,~,~,~,~,~,~,~,~}
    \end{enumerate}
\end{enigme}
  
% Pour le corrigé, il faut décrémenter le compteur, sinon il est incrémenté deux fois
\addtocounter{exercice}{-1}
\begin{corrige}
    \begin{enumerate}
        \item \phantom{rrr}
        \PyramideNombre[Etages=4,Hauteur=8mm,Largeur=1.5cm]{-1,\num{6.5},3,*-\num{8.5},*\num{5.5},*\num{9.5},\num{-5.5},*\num{14.5},*4,*\num{18.5}}
        
        On calcule d'abord dans les cases :

        \PyramideNombre[Etages=4,Hauteur=8mm,Largeur=1.5cm]{*4,*-5,*\num{0.3},*\num{-1.5},~,~,~,~,~,~}

        Puis on complète

        \PyramideNombre[Etages=4,Hauteur=8mm,Largeur=1.5cm]{4,-5,\num{0.3},\num{-1.5},*-1,*-\num{4.7},*-\num{1.2},*-\num{5.7},*\num{5.9},*\num{0.2}}

        \item Pas de correction \ldots
    \end{enumerate}
\end{corrige}