\section{Multiplication de deux nombres relatifs}
\begin{propriete}[(admise)]
  Si on effectue le \textbf{produit} de deux nombres relatifs de \textbf{m\^{e}me signe} alors il est \textbf{positif}.
\end{propriete}

\begin{propriete}[(admise)]
  Si on effectue le \textbf{produit} de deux nombres relatifs de \textbf{signes contraires} alors il est \textbf{n\'egatif}
\end{propriete}

\begin{exemple*1}
  Quels sont les signes de : $(+7)\times (+3)$\hfill ;\hfill$(-2)\times(-3)$\hfill ;\hfill $(-4)\times(+6)$\hfill et \hfill $(+7)\times(-3)$ ?
  \correction
  \begin{itemize}
    \item $(+7)\times (+3)$ et $(-2)\times(-3)$ sont des produits de deux nombres de même signe.\\
    \textbf{Ces deux produits sont donc positifs}.
    \item $(-4)\times(+6)$ et $(+7)\times(-3)$ sont des produits de deux nombres de signes contraires.\\
    \textbf{Ces deux produits sont donc négatifs}.
  \end{itemize}
\end{exemple*1}

\begin{remarques}
  \begin{itemize}
    \item remarque.
    \item remarque.
  \end{itemize}
\end{remarques}

\subsection{Sous-section 1.2}
\begin{theoreme}[Titre optionnel]
  Dans le cours, on utilise assez souvent des cadres du type
  définition, comme ici par exemple pour un théorème.
\end{theoreme}
\begin{notation}
  notation
\end{notation}
\begin{notations}
  \begin{itemize}
    \item notation.
    \item notation.
  \end{itemize}
\end{notations}
\begin{preuve}
  Ceci est une preuve\par Deuxième ligne de la preuve
\end{preuve}
\begin{exemple}
  Texte de l’exemple
  \correction
  Texte de la correction en vis à vis
\end{exemple}

\begin{exemple*1}
  Texte de l’exemple
  \correction
  Texte de la correction, le tout verticalement affiché
\end{exemple*1}

\begin{exemple}[0.6]
  Texte de l’exemple très long sur une ligne, très très très long.
  On peut modifier la répartition horizontale  à l'aide d'un argument optionnel valant par défaut 0,4, valant ici 0,6.
  \correction
  Texte de la correction en vis à vis
\end{exemple}