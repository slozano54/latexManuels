\section{Addition de deux nombres relatifs}
\begin{definition}[Distance à zéro]
  La \MotDefinition{distance à zéro}{} d'un nombre relatif, c’est la distance qui le sépare de zéro !

  Une distance est \textbf{toujours positive}.
\end{definition}

\begin{propriete}[\admise]
  Si deux nombres relatifs sont de \textbf{même signe} et qu’ils sont \textbf{positifs} 
  alors leur somme est \textbf{positive} et on calcule sa distance à zéro en additionnant les distances à zéro.
\end{propriete}

\begin{propriete}[\admise]
  Si deux nombres relatifs sont de \textbf{même signe} et qu’ils sont \textbf{négatifs}
  alors leur somme est \textbf{négative} et on calcule sa distance à zéro en additionnant les distances à zéro.
\end{propriete}

\begin{propriete}[\admise]
  Si deux nombres relatifs sont de \textbf{signes contraires}
  alors leur somme est du \textbf{signe du plus éloigné de zéro} et on calcule sa distance à zéro en calculant la différence positive des distances à zéro.
\end{propriete}

\begin{exemple*1}
  \begin{itemize}
    %\begin{minipage}{12cm}
    \item[] $A=(\textcolor{red}{+}17,7)+ (\textcolor{red}{+}1,5)=\textcolor{red}{+}(17,7+1,5)=\psshadowbox{\textcolor{red}{+}19,2}$
    \item[] $B=(\textcolor{green}{-}23,6)+ (\textcolor{green}{-}7,2)=\textcolor{green}{-}(23,6+7,2)=\psshadowbox{\textcolor{green}{-}30,8}$
    \item[] $C=(\textcolor{red}{+}14,3 )+ (\textcolor{green}{-}4,36)=\textcolor{red}{+}(14,3-4,36)=\psshadowbox{\textcolor{red}{+}9,94}$
    \\
    Pour le C, le nombre le plus loin de zéro est le nombre positif donc la somme est positive.
    \par\vspace{0.5cm}
    \item[] $D=(\textcolor{green}{-}11,2)+ (\textcolor{red}{+}7,6)=\textcolor{green}{-}(11,2-7,6 )=\psshadowbox{\textcolor{green}{-}3,6}$
    \\
    Pour le D, le nombre le plus loin de zéro est le nombre négatif donc la somme est négative.
    \\
    %\end{minipage}
    \par\vspace{0.5cm}
    \begin{minipage}{7cm}
    \item[] $E=(+14,9)+(-5,1)+(1,75)$
    \item[] $E=(+9,8)+(1,75)$
    \item[] \psshadowbox{$E=+11,55$}
    \end{minipage}
    \begin{minipage}{5cm}
    \item[] $F=(-\dfrac38)+(-\dfrac78)$
    \item[] $F=-\dfrac{10}{8}$
    \item[] \psshadowbox{$F=-\dfrac{5}{4}$}
    \end{minipage}
    \begin{minipage}{6cm}
    \item[] $G=(-\dfrac37)+(+\dfrac{5}{14})$
    \item[] $G=(-\dfrac{6}{14})+(+\dfrac{5}{14})$
    \item[] \psshadowbox{$G=-\dfrac{1}{14}$}
    \end{minipage}
    \end{itemize} 
\end{exemple*1}