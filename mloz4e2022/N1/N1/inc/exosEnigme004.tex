% Les enigmes ne sont pas numérotées par défaut donc il faut ajouter manuellement la numérotation
% si on veut mettre plusieurs enigmes
\refstepcounter{exercice}
\numeroteEnigme
\begin{enigme}[Pyramides de nombres multiplicatives]
    \begin{enumerate}
        \item Compléter les pyramides sachant que chaque nombre est le produit
        des nombres se trouvant dans les deux cases juste en dessous.    

        \PyramideNombre[Etages=4,Hauteur=8mm,Largeur=1.2cm]{-2,+2,-2,-2,~,~,~,~,~,~}
        \hfill
        \PyramideNombre[Etages=4,Hauteur=8mm,Largeur=1.2cm]{-1,+1,-1,-1,~,~,~,~,~,~}
        \hfill
        \PyramideNombre[Etages=4,Hauteur=8mm,Largeur=1.2cm]{-3,2,-1,-5,~,~,~,~,~,~}

        \item Inventer un contenu et échanger avec son voisin.
    
        \PyramideNombre[Etages=4,Hauteur=8mm,Largeur=1.5cm]{~,~,~,~,~,~,~,~,~,~}
        \hfill
        \PyramideNombre[Etages=4,Hauteur=8mm,Largeur=1.5cm]{~,~,~,~,~,~,~,~,~,~}
    \end{enumerate}
\end{enigme}
  
% Pour le corrigé, il faut décrémenter le compteur, sinon il est incrémenté deux fois
\addtocounter{exercice}{-1}
\begin{corrige}
    \begin{enumerate}
        \item \phantom{rrr}
        \PyramideNombre[Etages=4,Hauteur=8mm,Largeur=1.2cm]{-2,+2,-2,-2,*-4,*-4,-+4,*16,*-16,*-256}
        \hfill
        \PyramideNombre[Etages=4,Hauteur=8mm,Largeur=1.2cm]{-1,+1,-1,-1,*-1,*-1,*+1,*+1,*-1,*-1}
        \hfill
        \PyramideNombre[Etages=4,Hauteur=8mm,Largeur=1.2cm]{-3,2,-1,-5,*-6,*-2,*5,*12,*-10,*-120}
        \item Pas de correction \ldots
    \end{enumerate}
\end{corrige}