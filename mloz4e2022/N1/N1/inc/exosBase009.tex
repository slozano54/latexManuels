\begin{exercice*}[Sommes de deux entiers relatifs simplifiées]
    Calculer.
    \begin{multicols}2
        \begin{enumerate}
            \item $ 8-4 $
            \item $ -20+20 $
            \item $ -6+11 $
            \item $ -10+3 $
            \item $ -12+18 $
            \item $ -2-18 $
            \item $ 4-10 $
            \item $ -5+13 $
        \end{enumerate}
    \end{multicols}

    % \href{https://coopmaths.fr/mathalea.html?ex=5R20,s=20,s2=true,s3=false,n=8,i=1&v=l}{Cliquer pour vous entraîner sur le site \mathaleaLogo} 
    \hrefMathalea{https://coopmaths.fr/mathalea.html?ex=5R20,s=20,s2=true,s3=false,n=8,i=1&v=l}
\end{exercice*}
\begin{corrige}
    %\setcounter{partie}{0} % Pour s'assurer que le compteur de \partie est à zéro dans les corrigés
    \phantom{rrr}    
    \begin{multicols}4
        \begin{enumerate}
            \item $ 8-4 = 4 $
            \item $ -20+20 = 0 $
            \item $ -6+11 = 5 $
            \item $ -10+3 = -7 $
            \item $ -12+18 = 6 $
            \item $ -2-18 = -20 $
            \item $ 4-10 = -6 $
            \item $ -5+13 = 8 $
        \end{enumerate}
    \end{multicols}
\end{corrige}