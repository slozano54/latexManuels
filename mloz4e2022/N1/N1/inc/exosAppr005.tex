\begin{exercice*}[Signe d'un produit de deux relatifs]    
    % \begin{multicols}2
        \begin{itemize}
            \item[] A = $-6-(-2+7)$
            \item[] B = $9+110\div(-11)$
            \item[] C = $2\times4\times(-3)-9$
            \item[] D = $-17+14+10\times8$
            \item[] E = $9\times(-6)+8\times2$
        \end{itemize}
    % \end{multicols}
    
    \hrefMathalea{https://coopmaths.fr/mathalea.html?ex=4C11,s=3,s2=true,n=5,i=1&v=l}
\end{exercice*}
\begin{corrige}
    %\setcounter{partie}{0} % Pour s'assurer que le compteur de \partie est à zéro dans les corrigés
    % \begin{multicols}2
        \begin{list}{}{}
            \item A = $-6-(\mathbf{{\color[HTML]{f15929}-2+7}})$ \\
            A = $-6-(+5)$ \\
            A = $-6-5$ \\
            A = $-11$ \\
            \item B = $9\mathbf{{\color[HTML]{f15929}~+110\div(-11)}}$ \\
            B = $9-10$ \\
            B = $-1$ \\
            \item C = $\mathbf{{\color[HTML]{f15929}2\times4}}\times(-3)-9$ \\
            C = $\mathbf{{\color[HTML]{f15929}8\times(-3)}}-9$ \\
            C = $-24-9$ \\
            C = $-33$ \\
            \item D = $-17+14\mathbf{{\color[HTML]{f15929}+10\times8}}$ \\
            D = $-17+14+80$ \\
            D = $77$ \\
            \item E = $9\mathbf{{\color[HTML]{f15929}\times}}(-6)+8\mathbf{{\color[HTML]{f15929}\times}}2$ \\
            E = $-54+16$ \\
            E = $-38$ \\
        \end{list}
    % \end{multicols}
\end{corrige}
