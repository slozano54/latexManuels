\begin{exercice*}[Programme de calcul \tice]
    Voici un programme élaboré avec le logiciel Scratch.
    \begin{Scratch}[Echelle=0.75]
        Place Drapeau;
        Place Demander("Choisir un nombre");
        Place Dire(OpMul(OpMul(OvalCap("réponse"),"-7"),OpMul("2",OvalCap("réponse"))));
    \end{Scratch}
    % \begin{multicols}2
        \begin{enumerate}
            \item Que répond le programme si on choisit $-1$ ?
            \item Écris le programme de calcul correspondant.
        \end{enumerate}
    % \end{multicols}    
\end{exercice*}
\begin{corrige}
    %\setcounter{partie}{0} % Pour s'assurer que le compteur de \partie est à zéro dans les corrigés
    \phantom{rrr}    
    % \begin{multicols}2
        \begin{enumerate}
            \item Si on choisit $-1$ la variable
            \raisebox{-0.3\totalheight}[0.7\totalheight]{\raisebox{\depth}{
            \begin{Scratch}[Echelle=0.75]
                Place OvalCap("réponse");
            \end{Scratch}
            }}
            contient $-1$, donc le calcul à faire est : $ R = \left( (-1)\times (-7) \right)\times\left(2\times (-1)\right) $

            $ R = \left( (-1)\times (-7) \right)\times\left(2\times (-1)\right) $
            
            $ R = \left( +7 \right)\times\left(-2 \right) $

            $ \psshadowbox{R = -14} $

            \medskip
            \item 
        \end{enumerate}
        % La commande doit être en dehors de l'environnement enumerate sinon bug !
        \myProgCalcul{$\leadsto$}{Programme de calcul}{%
            \ProgCalcul[Enonce,ThemePerso]{%                
                Choisir un nombre.,
                Multiplier le nombre de départ par $(-7)$.,
                Multiplier le nombre de départ par $2$.,
                Multiplier les deux nombres obtenus précédemment.,
            }
        } 
    % \end{multicols}
\end{corrige}