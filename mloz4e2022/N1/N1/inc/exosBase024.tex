\begin{exercice*}[Programme de calcul]
    Voici un programme de calcul :
    \begin{myProgCalculBox}{Programme de calcul}
        \begin{itemize}
            \item Choisir un nombre.
            \item Multiplier ce nombre par (-5).
            \item Doubler le résultat obtenu.
        \end{itemize}
    \end{myProgCalculBox}
   % \ProgCalcul[Enonce,ThemePerso]{}

    % \begin{multicols}2
        Applique ce programme à chacun de ces nombres :
        \begin{enumerate}
            \item $ 5 $
            \item $ 0 $
            \item $ (-5) $
            \item $ (-1,2) $
            \item Que remarques-tu ? Expliquer pourquoi.
        \end{enumerate}
    % \end{multicols}    
\end{exercice*}
\begin{corrige}
    %\setcounter{partie}{0} % Pour s'assurer que le compteur de \partie est à zéro dans les corrigés
    \phantom{rrr}    
    % \begin{multicols}2
        \begin{enumerate}
            \item \ProgCalcul{5,*(-5) *2}
            \item \ProgCalcul{0,*(-5) *2}
            \item \ProgCalcul{-5,*(-5) *2}
            \item \ProgCalcul{-1.2,*(-5) *2}
            \item On remarque que chaque fois le nombre de départ est multiplié par $(-10)$. En effet :
            \begin{myProgCalculBox}{Programme de calcul}
                \begin{itemize}
                    \item Choisir un nombre \dotfill $x$.
                    \item Multiplier ce nombre par (-5)\dotfill $(-5)\times x $.
                    \item Doubler le résultat obtenu\dotfill $(-10)\times x$.
                \end{itemize}
            \end{myProgCalculBox}        
        \end{enumerate}
    % \end{multicols}
\end{corrige}