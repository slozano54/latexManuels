\begin{exercice*}[Signe d'un produit de relatifs]
    Indiquer le signe du produit lorsqu'on multiplie :
    % \begin{multicols}2
        \begin{enumerate}
            \item un nombre négatif par un nombre positif.
            \item quatre nombres négatifs entre eux.
            \item un nombre positif et deux nombres négatifs.
            \item un nombre relatif par lui-même.
            \item trois nombres négatifs entre eux.
        \end{enumerate}
    % \end{multicols}    
\end{exercice*}
\begin{corrige}
    %\setcounter{partie}{0} % Pour s'assurer que le compteur de \partie est à zéro dans les corrigés
    \phantom{rrr}    
    Lorsqu'on multiplie : 
    % \begin{multicols}2
        \begin{enumerate}
            \item un nombre négatif par un nombre positif, on obtient un nombre {\bfseries \color[HTML]{f15929}négatif}.
            \item quatre nombres négatifs entre eux, on obtient un nombre {\bfseries \color[HTML]{f15929}positif}..
            \item un nombre positif et deux nombres négatifs, on obtient un nombre {\bfseries \color[HTML]{f15929}positif}.
            \item un nombre relatif par lui-même, on obtient un nombre {\bfseries \color[HTML]{f15929}positif}.
            \item trois nombres négatifs entre eux, on obtient un nombre {\bfseries \color[HTML]{f15929}négatif}.
        \end{enumerate}
    % \end{multicols}
\end{corrige}

