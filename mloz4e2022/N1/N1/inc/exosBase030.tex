\begin{exercice*}[Signe d'un quotient (fraction)]
    Donner le signe des quotient suivants sans les calculer.
    \begin{multicols}4
      \begin{spacing}{2}
        \begin{enumerate}
            \item $ \dfrac{(-7)}{(+6)} $            
            \item $ \dfrac{(+4)}{(-16)} $            
            \item $ \dfrac{(-19)}{(+15)} $
            \item $ \dfrac{(+8)}{(-1)} $     
            \item $ \dfrac{(+11) \times (+12)}{(+13)} $
	    	    \item $ \dfrac{(+2)}{(+7) \times (-16)} $
	    	    \item $ \dfrac{(+1)}{(-2)} $
	    	    \item $ \dfrac{(+18) \times (+13)}{(-14) \times (+3)} $	
        \end{enumerate}
      \end{spacing}
    \end{multicols}
      
      \hrefMathalea{https://coopmaths.fr/mathalea.html?ex=4C10-2,s=1,n=4,i=1&ex=4C10-2,s=5,n=4,i=1&v=l}
\end{exercice*}
\begin{corrige}
    %\setcounter{partie}{0} % Pour s'assurer que le compteur de \partie est à zéro dans les corrigés
        \begin{enumerate}
            \item $ (-7) $ est négatif et $ (+6) $ est positif.\\
            Les numérateur et le dénominateur ont un signe différent donc le quotient est négatif.\\
           Donc $ \dfrac{(-7)}{(+6)} $ est {\bfseries \color[HTML]{f15929}négatif}.
            \item $ (+4) $ est positif et $ (-16) $ est négatif.\\
            Les numérateur et le dénominateur ont un signe différent donc le quotient est négatif.\\
           Donc $ \dfrac{(+4)}{(-16)} $ est {\bfseries \color[HTML]{f15929}négatif}.
            \item $ (-19) $ est négatif et $ (+15) $ est positif.\\
            Les numérateur et le dénominateur ont un signe différent donc le quotient est négatif.\\
           Donc $ \dfrac{(-19)}{(+15)} $ est {\bfseries \color[HTML]{f15929}négatif}.
            \item $ (+8) $ est positif et $ (-1) $ est négatif.\\
            Les numérateur et le dénominateur ont un signe différent donc le quotient est négatif.\\
           Donc $ \dfrac{(+8)}{(-1)} $ est {\bfseries \color[HTML]{f15929}négatif}.
            \item $ (+11) $ est positif, $ (+12) $ est positif et $ (+13) $ est positif.\\
           Tous les facteurs du numérateur et tous les facteurs du dénominateur sont positifs donc le quotient est positif.\\
          Donc $ \dfrac{(+11) \times (+12)}{(+13)} $ est {\bfseries \color[HTML]{f15929}positif}.
            \item $ (+2) $ est positif, $ (+7) $ est positif et $ (-16) $ est négatif.\\
           Quand on compte les facteurs négatifs du numérateur et du dénominateur, on trouve 1, ce nombre est impair donc le quotient est négatif.\\
          Donc $ \dfrac{(+2)}{(+7) \times (-16)} $ est {\bfseries \color[HTML]{f15929}négatif}.
            \item $ (+1) $ est positif et $ (-2) $ est négatif.\\
           Les numérateur et le dénominateur ont un signe différent donc le quotient est négatif.\\
          Donc $ \dfrac{(+1)}{(-2)} $ est {\bfseries \color[HTML]{f15929}négatif}.
            \item $ (+18) $ est positif, $ (+13) $ est positif, $ (-14) $ est négatif et $ (+3) $ est positif.\\
           Quand on compte les facteurs négatifs du numérateur et du dénominateur, on trouve 1, ce nombre est impair donc le quotient est négatif.\\
          Donc $ \dfrac{(+18) \times (+13)}{(-14) \times (+3)} $ est {\bfseries \color[HTML]{f15929}négatif}.                
        \end{enumerate}
\end{corrige}

