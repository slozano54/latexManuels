\annexe{Corrections d'activités}
\phantomsection \label{corrN1activite003}
\section*{Produit de deux nombres relatifs}
Retour à l'activité p. \hyperref[N1activite003]{\pageref{N1activite003}}

\begin{enumerate}
    \item Produit de deux nombres de signes contraires, par exemple $3\times(-7)$
    \begin{enumerate}
        \item $(-7)+(-7)+(-7) = -(7+7+7) = -21$.
        \item $3\times (-7)$ vaut donc $-21$.
    \end{enumerate}
    \item $(-4)\times 5= 5\times (-4)$ et $5\times (-4) = -20$ d'après ce qui précède, donc $(-4)\times 5 = -20$.
    \item Si l'un des facteurs n'est pas entier, par exemple $1,5\times (-3)$
    \begin{enumerate}
        \item $1,5\times ((-3)+3) = 1,5\times 0 = 0$.
        \item $1,5\times ((-3)+3) = 1,5\times (-3) + 1,5\times 3$ 
        \item D'où $1,5\times (-3) + 1,5\times 3 = 0 $ donc $1,5\times (-3)$ et $1,5\times 3$ sont des nombres opposés donc $1,5\times (-3) = - 1,5\times 3 = -4,5$
    \end{enumerate}
    \item Produit de deux nombres de même signe
    \begin{enumerate}
        \item Nous savons déjà le faire pour deux nombres positifs.
        \item Si les deux nombres sont négatifs, par exemple $(-2)\times (-7)$
        \begin{enumerate}
            \item $(-2)\times ((-7)+7) = (-2)\times 0 = 0$
            \item $(-2)\times ((-7)+7) = (-2)\times (-7) + (-2)\times 7$
            \item D'où $(-2)\times (-7) + (-2)\times 7=0$ donc $(-2)\times (-7) = - (-2)\times 7 = -(-14) = 14$
        \end{enumerate}
    \end{enumerate}
\end{enumerate}
