\begin{exercice*}[Simplification des écritures fractionnaires]
    Simplifier l'écriture de chaque fraction.

    \begin{exemple*1}
      $ \dfrac{-2}{+9} = - \dfrac29 $
    \end{exemple*1}

    \begin{multicols}3
      \begin{spacing}{2}
        \begin{enumerate}
          \item $-\dfrac{+4}{+5}$
          \item $-\dfrac{-1}{-7}$
          \item $\dfrac{7}{-3}$
          \item $-\dfrac{8}{11}$
          \item $-\dfrac{1}{-20}$
          \item $-\dfrac{5}{-15}$
        \end{enumerate}              
      \end{spacing}
    \end{multicols}
\end{exercice*}
\begin{corrige}
    %\setcounter{partie}{0} % Pour s'assurer que le compteur de \partie est à zéro dans les corrigés
        \begin{enumerate}
          \item $-\dfrac{+4}{+5} = -\dfrac{4}{5}$
          \item $-\dfrac{-1}{-7} = -\dfrac{1}{7}$
          \item $ \dfrac{7}{-3}  = -\dfrac{7}{3} $
          \item $-\dfrac{-8}{11}  = \dfrac{8}{11} $
          \item $-\dfrac{1}{-20} = \dfrac{1}{20}$
          \item $-\dfrac{5}{-15} = \dfrac{5}{15}$
        \end{enumerate}
\end{corrige}

