% Les enigmes ne sont pas numérotées par défaut donc il faut ajouter manuellement la numérotation
% si on veut mettre plusieurs enigmes
\refstepcounter{exercice}
\numeroteEnigme
\begin{enigme}[Rose multiplicative]
    Les nombres situés à l'extrémité des flèches sont les produits des nombres dont les flèches sont issues.     

    \begin{minipage}{0.5\linewidth}
        \begin{center}
            \RoseMul[Aide,Rayon=18mm]
        \end{center}
    \end{minipage}
    \hfill
    \begin{minipage}{0.5\linewidth}
        \begin{center}
            \RoseMul[Aide,Produits,Rayon=18mm]
        \end{center}
    \end{minipage}  
    
\end{enigme}

% Pour le corrigé, il faut décrémenter le compteur, sinon il est incrémenté deux fois
\addtocounter{exercice}{-1}
\begin{corrige}
    Pas de corrections pour les roses multiplicatives.
    % \RoseMul[Aide,Solution]
    % \hfill
    % \RoseMul[Aide,Produits,Solution]

\end{corrige}