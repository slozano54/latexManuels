% Les enigmes ne sont pas numérotées par défaut donc il faut ajouter manuellement la numérotation
% si on veut mettre plusieurs enigmes
\refstepcounter{exercice}
\numeroteEnigme
\begin{enigme}[Rose multiplicative]
    Les nombres situés à l'extrémité des flèches sont les produits des nombres dont les flèches sont issues.     

    \RoseMul[Aide,Rayon=18mm]    
    \hfill
    \RoseMul[Aide,Produits,Rayon=18mm]    

    % \href{https://coopmaths.fr/mathalea.html?ex=5R20-6,s=10,s2=1,s3=2,s4=false,n=3,i=1&v=l}{Cliquer pour vous entraîner sur le site \mathaleaLogo} 
    \hrefMathalea[\emoji{star-struck} Calculer les produits - ]{https://coopmaths.fr/mathalea.html?ex=4C10-9,s=10,s2=7,s3=1,n=1,i=1&v=l}

    \hrefMathalea[\emoji{star-struck} Déterminer les facteurs - ]{https://coopmaths.fr/mathalea.html?ex=4C10-9,s=10,s2=7,s3=2,n=1,i=1&v=l}
\end{enigme}

% Pour le corrigé, il faut décrémenter le compteur, sinon il est incrémenté deux fois
\addtocounter{exercice}{-1}
\begin{corrige}
    Pas de corrections pour les roses multiplicatives.
    % \RoseMul[Aide,Solution]
    % \hfill
    % \RoseMul[Aide,Produits,Solution]

\end{corrige}