\begin{exercice*}[Signe d'un quotient (fraction)]
    Donner le signe des quotient suivants sans les calculer.
    \begin{multicols}2
        \begin{spacing}{2}
            \begin{enumerate}
                \item $ \dfrac{(-7)}{(+6)} $            
                \item $ \dfrac{(+4)}{(-16)} $            
                \item $ \dfrac{(-19)}{(+15)} $
                \item $ \dfrac{(+8)}{(-1)} $            
            \end{enumerate}
        \end{spacing}
    \end{multicols}
\end{exercice*}
\begin{corrige}
    %\setcounter{partie}{0} % Pour s'assurer que le compteur de \partie est à zéro dans les corrigés
        \begin{enumerate}
            \item $ (-7) $ est négatif et $ (+6) $ est positif.\\
            Les numérateur et le dénominateur ont un signe différent donc le quotient est négatif.\\
           Donc $ \dfrac{(-7)}{(+6)} $ est {\bfseries \color[HTML]{f15929}négatif}.
               \item $ (+4) $ est positif et $ (-16) $ est négatif.\\
            Les numérateur et le dénominateur ont un signe différent donc le quotient est négatif.\\
           Donc $ \dfrac{(+4)}{(-16)} $ est {\bfseries \color[HTML]{f15929}négatif}.
               \item $ (-19) $ est négatif et $ (+15) $ est positif.\\
            Les numérateur et le dénominateur ont un signe différent donc le quotient est négatif.\\
           Donc $ \dfrac{(-19)}{(+15)} $ est {\bfseries \color[HTML]{f15929}négatif}.
               \item $ (+8) $ est positif et $ (-1) $ est négatif.\\
            Les numérateur et le dénominateur ont un signe différent donc le quotient est négatif.\\
           Donc $ \dfrac{(+8)}{(-1)} $ est {\bfseries \color[HTML]{f15929}négatif}.
        \end{enumerate}
\end{corrige}

