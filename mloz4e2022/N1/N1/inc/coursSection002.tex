\section{Soustraction de deux nombres relatifs}
\begin{definition}[nombres opposés]
    Deux nombres relatifs sont dits \MotDefinition{opposés}{} quand leur somme vaut zéro.
\end{definition}

\begin{notation}
    On note $-a$ l'opposé du nombre relatif $a$.
\end{notation}

\begin{remarque}
    $-a$ peut être positif ! Par exemple lorsque $a$ vaut $-4$.
\end{remarque}

\begin{propriete}[(admise)]
    Si un nombre relatif est positif alors son opposé est négatif.
\end{propriete}

\begin{propriete}[(admise)]
    Si un nombre relatif est négatif alors son opposé est positif.
\end{propriete}

\begin{exemple*1}
    \begin{enumerate}
        \item $-5,28$ est l'opposé de $+5,28$ mais $+5,28$ est aussi l'opposé de $-5,28$ \\en particulier $-(-5,28)=5,28$
        \item si $a=+2,14$ alors $-a=-2,14$ et si $a=-7,81$ alors $-a=+7,81$
    \end{enumerate}
\end{exemple*1}

\begin{propriete}[géométriquement (admise)]
    Si deux nombres relatifs sont opposés alors ils correspondent à des points symétriques par rapport à l'origine.
\end{propriete}

\begin{exemple*1}
        $-7$ et $+7$ sont opposés.
        \par\vspace{0.25cm}
        \pspicture(-4,-.5)(4,1)
            \psline{->}(-4,.5)(4,.5)
            \rput(0,0.5){+}
            \uput[90](0,0.5){O} 
            \uput[-90](0,0.5){0} 
            \rput(-3.5,.5){+}
            \uput[90](-3.5,0.5){A} 
            \uput[-90](-3.5,0.5){-7} 
            \rput(3.5,.5){+}
            \uput[90](3.5,0.5){B} 
            \uput[-90](3.5,0.5){+7} 
            \psline[linecolor=red]{<->}(-3.5,-0.2)(0,-0.2)
            \uput[-90](-1.5,-0.2){7 unités} 
            \psline[linecolor=red]{<->}(0,-0.2)(3.5,-0.2)
            \uput[-90](1.5,-0.2){7 unités} 
         \endpspicture
        \par\vspace{0.25cm}
        $A$ et $B$ sont symétriques par rapport à $O$
\end{exemple*1}

\begin{exemple*1}
        $\dfrac13$ et $-\dfrac13$ sont opposés.
        \par\vspace{0.25cm}
        \pspicture(-4,-.5)(4,1)
            \psline{->}(-4,.5)(4,.5)
            \multirput(-3,0.5)(1,0){7}{+}
            \uput[90](0,0.5){O} 
            \uput[-90](0,0.5){0} 
            \uput[-90](3,0.2){+1} 
            \uput[90](-1,0.5){C} 
            \uput[-90](-1,0.5){$-\frac13$} 
            \uput[90](1,0.5){D} 
            \uput[-90](1,0.5){$\frac13$} 
            \psline[linecolor=red]{<->}(-1,-0.2)(0,-0.2)
            \psline[linecolor=red]{<->}(0,-0.2)(1,-0.2)
         \endpspicture
        \par\vspace{0.25cm}
        $C$ et $D$ sont symétriques par rapport à $O$
\end{exemple*1}

\begin{propriete}[(admise)]
    Si on soustrait un nombre relatif alors cela revient à additionner son nombre opposé.
\end{propriete}

\begin{exemple*1}
    \begin{itemize}
        \item l'opposé de $+8,2$ est $-8,2$ donc soustraire $+8,2$ revient à ajouter $-8,2$.
        $$(+14)-(+8,2)=(+14)+(-8,2)=+5,8$$
        \item l'opposé de $-8,2$ est $+8,2$ donc soustraire $-8,2$ revient à ajouter $+8,2$.
        $$(-17,2)-(-8,2)=(-17,2)+(+8,2)=-9$$
        \item $+\dfrac54 -(+\dfrac34)=+\dfrac54 +(-\dfrac34)=-\dfrac14$
    \end{itemize}
\end{exemple*1}