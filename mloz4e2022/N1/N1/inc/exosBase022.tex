\begin{exercice*}[Suite de multiplications]
    Calculer.
    \begin{itemize}
        \item[] $I = (-2)\times (-3)\times (+5) $
        \item[] $J = (-3)\times (-2)\times (-4) $
        \item[] $K = (-3,2)\times (-10)\times (+2)\times (-0,5) $
        \item[] $L = (-75)\times (-0,25)\times (+4)\times (+2) $
        \item[] $M = (-3)\times (-0,1)\times (+5)\times (+4) $
        \item[] $N = (+2)\times (-10)\times (+3)\times (-1)\times (-1) $
    \end{itemize}
\end{exercice*}
\begin{corrige}
    %\setcounter{partie}{0} % Pour s'assurer que le compteur de \partie est à zéro dans les corrigés
    \phantom{rrr}
    \begin{multicols}2
    \begin{list}{}{}
        \item $I = (-2)\times (-3)\times (+5) $
        \item $I = (+6)\times (+5) $
        \item $\psshadowbox{I = 30} $
        \item 
        \item $J = (-3)\times (-2)\times (-4) $
        \item $J = (+6)\times (-4) $
        \item $\psshadowbox{J = -24} $
        \item 
        \item $K = (-3,2)\times (-10)\times (+2)\times (-0,5) $
        \item $K = (+32)\times (+2)\times (-0,5) $
        \item $K = (+32)\times (-1) $
        \item $\psshadowbox{K = -32} $
        \columnbreak
        \item $L = (-75)\times (-0,25)\times (+4)\times (+2) $
        \item $L = (-75)\times (-1)\times (+2) $
        \item $L = (+75)\times (+2) $
        \item $\psshadowbox{L = 150} $
        \item 
        \item $M = (-3)\times (-0,1)\times (+5)\times (+4) $
        \item $M = (+0,3)\times (+5)\times (+4) $
        \item $M = (+0,3)\times (+20) $
        \item $\psshadowbox{M = 6} $
        \item 
        \item $N = (+2)\times (-10)\times (+3)\times (-1)\times (-1) $
        \item $N = (-20)\times (+3)\times (-1)\times (-1) $
        \item $N = (-60)\times (-1)\times (-1) $
        \item $N = (+60)\times (-1) $
        \item $\psshadowbox{N = -60} $
    \end{list}
\end{multicols}
\end{corrige}

