\begin{exercice*}[Programme de calcul]    
    Voici un programme de calcul :
    \myProgCalcul{$\leadsto$}{Programme de calcul}{%
        \ProgCalcul[Enonce,ThemePerso]{%
            Choisir un nombre.,
            Augmenter le nombre de $-5$.,
            Multiplier le résultat par $4$.,
            Soustraire le double du nombre choisi au départ.,
            Diviser le résultat par $-2$,
            Ajouter $-10$.,
        }
    }  
    Applique ce programme à chacun de ces nombres :
    \begin{multicols}2
        \begin{enumerate}
            \item $ 12 $
            \item $ -3 $
        \end{enumerate}
    \end{multicols}
    \begin{enumerate}
        \setcounter{enumi}{4}
        \item Que remarques-tu ? Expliquer pourquoi.
    \end{enumerate}
        
\end{exercice*}
\begin{corrige}
    %\setcounter{partie}{0} % Pour s'assurer que le compteur de \partie est à zéro dans les corrigés
    \phantom{rrrr}    
    % \begin{multicols}4
        \begin{spacing}2
            \begin{enumerate}
                \item \ProgCalcul{12,+(-5) *4 -24 /(-2) +(-10)}            
                \item \ProgCalcul{-3,+(-5) *4 +6 /(-2) +(-10)}            
            \end{enumerate}
        \end{spacing}
    % \end{multicols}
        \begin{enumerate}
            \setcounter{enumi}{2}
            \item On remarque que chaque fois on obtient l'opposé du nombre de départ. En effet :
        \end{enumerate}
        % La commande doit être en dehors de l'environnement enumerate sinon bug !
        \myProgCalcul{$\leadsto$}{Programme de calcul}{%
            \ProgCalcul[Application,SansCalcul,ThemePerso]{%                
                Augmenter le nombre de $-5$.,
                Multiplier le résultat par $4$.,
                Soustraire le double du nombre choisi au départ.,
                Diviser le résultat par $-2$,
                Ajouter $-10$.,
                §%
                x,+(-5) *4 -(2x) /(-2) +(-10),x-5 4x-20 2x-20 -x+10 -x 
            }
        } 
\end{corrige}