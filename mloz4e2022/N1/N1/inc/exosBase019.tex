\begin{exercice*}[Produit de relatifs en écriture simplifiée]
    Calculer.
    \begin{multicols}2
        \begin{enumerate}
            \item $ -10 \times  5,13 $
            \item $ -1 \times  (-3,17) $
            \item $ -5 \times  8 $
            \item $ 5 \times  (-10) $
            \item $ -2 \times  4 $
            \item $ 10,7 \times  (-10) $
            \item $ 5 \times  (-1) $
            \item $ -5 \times  4 $
            \item $ -4 \times  (-9) $
            \item $ 3 \times  (-4) $          
        \end{enumerate}
    \end{multicols}

    \hrefAleaTeX{https://urls.mathslozano.fr/4n12025ex19}
\end{exercice*}
\begin{corrige}
    %\setcounter{partie}{0} % Pour s'assurer que le compteur de \partie est à zéro dans les corrigés
    \phantom{rrr}    
    \begin{multicols}2
        \begin{enumerate}
            \item $ -10 \times  5,13 = -51,3 $
            \item $ -1 \times  (-3,17) = 3,17 $
            \item $ -5 \times  8 = -40 $
            \item $ 5 \times  (-10) = -50 $
            \item $ -2 \times  4 = -8 $
            \item $ 10,7 \times  (-10) = -107 $
            \item $ 5 \times  (-1) = -5 $
            \item $ -5 \times  4 = -20 $
            \item $ -4 \times  (-9) = 36 $
            \item $ 3 \times  (-4) = -12 $
        \end{enumerate}
    \end{multicols}
\end{corrige}