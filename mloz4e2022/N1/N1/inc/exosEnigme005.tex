% Les enigmes ne sont pas numérotées par défaut donc il faut ajouter manuellement la numérotation
% si on veut mettre plusieurs enigmes
\refstepcounter{exercice}
\numeroteEnigme
\begin{enigme}
    \begin{enumerate}
        \item Compléter les pyramides sachant que chaque nombre est le produit
        des nombres se trouvant dans les deux cases d'où partent les flêches.    

        \PyramideNombre[Multiplication,Aide]{-3,5,7}
        \hfill
        \PyramideNombre[Multiplication,Aide]{3,-5,7}
        \hfill
        \PyramideNombre[Multiplication,Produit,Aide]{4,-5,-3}
        \pagebreak
        \item Inventer un contenu et échanger avec son voisin.
        \begin{center}
            \scalebox{0.7}{
                \PyramideNombre[Multiplication,Aide]{}
                \hfill
                \PyramideNombre[Multiplication,Aide]{}
                \hfill
                \PyramideNombre[Multiplication,Aide]{}
            }
        \end{center}
    \end{enumerate}
    \vspace*{-10mm}
\end{enigme}
  
% Pour le corrigé, il faut décrémenter le compteur, sinon il est incrémenté deux fois
\addtocounter{exercice}{-1}
\begin{corrige}
    \begin{enumerate}
        \item \phantom{rrr}
        \PyramideNombre[Multiplication,Aide,Solution]{-3,5,7}
        \hfill
        \PyramideNombre[Multiplication,Aide,Solution]{3,-5,7}
        \hfill
        \PyramideNombre[Multiplication,Produit,Aide,Solution]{4,-5,-3}
        \item Pas de correction \ldots
    \end{enumerate}
\end{corrige}