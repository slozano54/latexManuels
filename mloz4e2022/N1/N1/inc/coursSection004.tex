\section{Multiplication de deux nombres relatifs}

\begin{propriete}[Signe d'un produit de deux nombres de même signe (admise)]
  Si on effectue le \textbf{\MotDefinition{produit}{}} de deux nombres relatifs de \textbf{m\^{e}me signe} alors il est \textbf{positif}.
\end{propriete}

\begin{propriete}[Signe d'un produit de deux nombres de signes contraires (admise)]
  Si on effectue le \textbf{produit} de deux nombres relatifs de \textbf{signes contraires} alors il est \textbf{n\'egatif}
\end{propriete}

\begin{exemple*1}
  Déterminer les signes de : $(+7)\times (+3)$\hfill ;\hfill$(-2)\times(-3)$\hfill ;\hfill $(-4)\times(+6)$\hfill et \hfill $(+7)\times(-3)$
  \correction
  \begin{itemize}
    \item $(+7)\times (+3)$ et $(-2)\times(-3)$ sont des produits de deux nombres de même signe.\\
    \textbf{Ces deux produits sont donc positifs}.
    \item $(-4)\times(+6)$ et $(+7)\times(-3)$ sont des produits de deux nombres de signes contraires.\\
    \textbf{Ces deux produits sont donc négatifs}.
  \end{itemize}
\end{exemple*1}

\begin{propriete}[Distance à zéro d'un produit (admise)]
  Le \textbf{produit} de deux nombres relatifs est un nombre relatif dont \textbf{la distance à zéro} est égale au produit des distances à zéro de ses facteurs.
\end{propriete}

\begin{exemple*1}
  Déterminer la distance à zéro de $(-7)\times (+3)$
  \correction
  \begin{itemize}
    \item La distance à zéro de $(-7)$ vaut $7$.
    \item La distance à zéro de $(+3)$ vaut $3$.
  \end{itemize}
  Donc la distance à zéro de $(-7)\times (+3)$ vaut $7\times 3= 21$
\end{exemple*1}

\begin{methode*1}
  Pour calculer le produit de deux nombres relatifs :
  \begin{enumerate}
    \item On détermine son signe.
    \item On calcule sa distance à zéro
  \end{enumerate}
  \exercice
  Calculer $(-7)\times (+3)$
  \correction
  \begin{enumerate}
    \item $(-7)\times (+3)$ est un produit de deux nombres de signes contraires donc il est \textbf{négatif}.
    \item La distance à zéro de $(-7)\times (+3)$ est égale au produit des distances à zéro des facteurs, soit $7\times 3 =21$
  \end{enumerate}
  Donc $(-7)\times (+3) = -21$
\end{methode*1}

\begin{propriete}[Multiplication par $\mathbf{0}$ (admise)]
  Si $a$ est un nombre relatif quelconque alors $a\times0=0\times a=0$.
\end{propriete}

\begin{multicols}2
  \begin{exemple*1}
    $(-2)\times 0=0\times (-2)=0$
  \end{exemple*1}
  \begin{exemple*1}
    $0\times (+6)=(+6)\times 0=0$
  \end{exemple*1}
\end{multicols}

\begin{propriete}[Multiplication par $\mathbf{(-1)}$ (admise)]
  Si on multiplie un nombre relatif par $-1$ alors on obtient l'opposé de ce nombre.
\end{propriete}

\begin{exemple*1}
  $(-22)\times (-1)=(+22)$ donc $(+22)$ est l'opposé de $(-22)$
\end{exemple*1}

\begin{exemple*1}
  $(-1)\times(+6)=(-6)$donc  $(-6)$ est l'opposé de $(+6)$
\end{exemple*1}

\begin{propriete}[Signe d'un produit de plusieurs facteurs]
  Lorsque l'on multiplie des nombres relatifs différents de $0$ :
  \begin{itemize}
    \item s'il y a un nombre {\bf pair} de facteurs {\bf négatifs} alors le produit est {\bf positif}.
    \item s'il y a un nombre {\bf impair} de facteurs {\bf négatifs} alors le produit est {\bf n\'egatif}.
  \end{itemize}
\end{propriete}

\begin{preuve}
  Aucun facteur n'est supposé nul donc le produit est lui aussi non nul.
  \begin{itemize}
    \item s'il y a un nombre {\bf pair} de facteurs {\bf négatifs} alors on peut les grouper par deux 
    et ainsi onbtenir un produit dont tous les facteurs sont positifs.



    Le produit est donc \textbf{positif}   
    
    Illustration :
    $\underbrace{\underbrace{a_0\times a_1}_\textrm{positif}\times \underbrace{a_2\times a_3}_\textrm{positif} \times \ldots \times \underbrace{a_{2n-2}\times a_{2n-1}}_\textrm{positif}}_\textrm{$2n$ facteurs négatifs en tout}$

    \item s'il y a un nombre {\bf impair} de facteurs {\bf négatifs} alors on peut séparer le produit en
    un produit de facteurs négatifs en nombre pair et un facteur négatif tout seul. 
    
    La première partie du produit est un nombre positif d'après ce qui précède.

    Il reste à faire un produit de deux nombres de signes contraires.

    Le produit final est donc {\bf négatif}.

    Illustration :
    $\underbrace{\underbrace{a_0\times a_1\times a_2\times a_3\times \ldots \times a_{2n-2}\times a_{2n-1}}_\textrm{$2n$ facteurs négatifs donc positif} \times \underbrace{a_{2n+1}}_\textrm{négatif}}_\textrm{Produit de deux relatifs de signes contraires donc négatifs}$
  \end{itemize}
\end{preuve}

\begin{exemple*1}
  Déterminer le signe de $A=(-4)\times3\times(-7)\times(-110)\times(-17)$.
  \correction
  $A$ est positif car il y a 4 facteurs négatifs au total, c'est à dire un nombre pair de facteurs négatifs.
\end{exemple*1}

\begin{exemple*1}
  Déterminer le signe de $B=13\times(-19)\times(-53)\times(-15)$.
  \correction
  $B$ est négatif car il y a 3 facteurs négatifs au total, c'est à dire un nombre impair de facteurs négatifs.
\end{exemple*1}