\begin{exercice*}
    Synthèse d'une vente de sapins de différentes tailles.
    \par\smallskip    
    \Stat[%        
        Tableau,%
        Donnee=\scantokens{%
            \begin{minipage}{0.1\textwidth}%
                \smallskip
                \textbf{Nombre de}\par\textbf{sapins}%
                \par\smallskip
            \end{minipage}},%
        Effectif=\scantokens{%
        \begin{minipage}{0.1\textwidth}%
            \smallskip
            \textbf{Prix du}\par\textbf{sapin en \Prix{}}%
            \par\smallskip
        \end{minipage}},%        
    ]{20/15,10/25,40/30,40/50,30/55}
    \smallskip
    \begin{enumerate}
        \item Calculer le prix de vente moyen d'un sapin. Arrondir au centime d'euro.
        \item Interpréter ce résultat.
        \item Modifier une seule valeur pour que ce prix moyen soit un nombre entier d'euros.
    \end{enumerate}
\end{exercice*}
\begin{corrige}
    %\setcounter{partie}{0} % Pour s'assurer que le compteur de \partie est à zéro dans les corrigés
    %\phantom{rrr}    
    Synthèse d'une vente de sapins de différentes tailles.
    \par\smallskip
    \Stat[%        
        Tableau,%
        Donnee=\textbf{Nombre de sapins},%
        Effectif=\textbf{Prix du sapin en \Prix{}}%
    ]{20/15,10/25,40/30,40/50,30/55}
    \smallskip
    \begin{enumerate}
        \item Calculer le prix de vente moyen d'un sapin. Arrondir au centime d'euro.
        \par\smallskip
        \textcolor{red}{
            \Stat[Moyenne,Precision=2]{20/15,10/25,40/30,40/50,30/55}
            Le prix de vente moyen est donc de \Prix{38.57}.
        }
        \smallskip        
        \item Interpréter ce résultat.
        \par\smallskip
        \textcolor{red}{Cela signifie que si tous les sapins étaient vendus au même prix, ils seraient vendus \Prix{38.57}.}
        \item Modifier une seule valeur pour que ce prix moyen soit un nombre entier d'euros.
        \par\smallskip
        \textcolor{red}{
            Il y a \EffectifTotal sapins. Pour que le prix moyen soit de \Prix{39}, il faut un total des ventes de $39\times 140=\Prix{5460}$
            soit \Prix{60} de plus. Il y a plusieurs solutions, remplacer \Prix{15} par \Prix{18} ou \Prix{25} par \Prix{31} ou \Prix{55} par \Prix{57}.
        }
    \end{enumerate}
\end{corrige}

