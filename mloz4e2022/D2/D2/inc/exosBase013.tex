\begin{exercice*}[\tableurLogo]
    Ce tableau rassemble dans une feuille de calcul le nombre de médailles gagnées par les sportifs calédoniens lors des Jeus du Pacifique.
    \begin{center}
      \setlength{\tabcolsep}{0.5\tabcolsep}
      \begin{Tableur}[Bandeau=false,LargeurUn=43,Largeur=43,Colonnes=5]
        Année des Jeux du Pacifique&Nombre de médailles d'or&Nombre de médailles d'argent&Nombre de médailles de bronze&Total\\
        1963&7&9&11&27\\
        1966&39&30&30&\\
        1969&36&20&21&\\
        1971&33&32&27&\\
        1975&37&31&34&\\
        1979&33&43&26&\\
        1983&24&20&19&\\
        1987&82&48&38&\\
        1991&29&29&27&\\
        1995&82&57&43&\\
        1999&73&55&44&\\
        2003&93&73&74&\\
        2007&90&69&68&\\
        &&&&\\
        Total : &658&&&\\
        &&&&\\
        Moyennes&&&&\\
      \end{Tableur}
    \end{center}

    \begin{enumerate}
        \item Reproduire dette feuille de calcul à l'aide d'un tableur.
        \item Pour obtenir \num{27} dans la cellule {\ttfamily E2}, on a saisi la formule \fbox{{\ttfamily = somme(B2:D2)}}.

        Proposer une formule à écrire en {\ttfamily B16} pour obtenir \num{658}.
        
        Programmer la cellule {\ttfamily B16}.
        \item Déterminer la formule à saisir en {\ttfamily B18} pour calculer la moyenne des médailles d'or obtenues sur ces 13 années.
        
        Programmer {\ttfamily B18}.
        \item Compléter la feuille de calcul en étirant ces formules.
    \end{enumerate}
\end{exercice*}
\begin{corrige}
    %\setcounter{partie}{0} % Pour s'assurer que le compteur de \partie est à zéro dans les corrigés
    %\phantom{rrr}    
    Pas de correction.
\end{corrige}

