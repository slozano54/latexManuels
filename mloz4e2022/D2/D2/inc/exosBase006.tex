\begin{exercice*}
    Durant une compétition d'athlétisme, les concurrents ont courru les \Lg[m]{200} en :

    \Temps{;;;;;20.25} - \Temps{;;;;;20.12} - \Temps{;;;;;20.48} - \Temps{;;;;;20.09} - \Temps{;;;;;20.69} - \Temps{;;;;;20.19} - \Temps{;;;;;20.38}
    \begin{enumerate}
        \item Déterminer l'effectif total.
        \item Ranger cette série dans l'ordre croissant, puis déterminer sa médiane.
        \item Interpréter cette médiane.
    \end{enumerate}
\end{exercice*}
\begin{corrige}
    %\setcounter{partie}{0} % Pour s'assurer que le compteur de \partie est à zéro dans les corrigés
    %\phantom{rrr}    
    \begin{enumerate}
        \item L'effectif total est \textcolor{red}{7}.
        \item Il y a un nombre impair de valeurs donc la médiane est la valeur centrale, ici \textcolor{red}{\Temps{;;;;;20.25}}.
        \item Il y a autant de temps inférieurs à \Temps{;;;;;20.25}, que de temps supérieurs à \Temps{;;;;;20.25}.
    \end{enumerate}
\end{corrige}

