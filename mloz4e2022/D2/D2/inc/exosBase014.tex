\begin{exercice*}
    Les informations suivantes concernent les salaires des hommes et des femmes d'une même entreprise.

    \fbox{\parbox{\linewidth}{
        % \setlength{\parskip}{.5cm}
        \begin{center}
            \textbf{Salaires des femmes}

            \Prix{1200} ; \Prix{1230} ; \Prix{1250} ; \Prix{1310} ; \Prix{1370}

            \Prix{1400} ; \Prix{1440} ; \Prix{1500} ; \Prix{1700} ; \Prix{2100}
        \end{center}
    }}

    \fbox{\parbox{\linewidth}{
        % \setlength{\parskip}{.5cm}
        \begin{center}
            \textbf{Salaires des hommes}

            Effectif total : 20
            
            Moyenne : \Prix{1769} 
            
            Médiane : \Prix{2000}
        \end{center}
    }}

    \begin{enumerate}
        \item Déterminer le salaire moyen des femmes.
        \item Déterminer le salaire moyen de tous les salariés de cette entreprise.
        \item Déterminer le salaire médian des femmes.
        \item Comparer les salaires moyen et médian des hommes et des femmes.
    \end{enumerate}
\end{exercice*}
\begin{corrige}
    %\setcounter{partie}{0} % Pour s'assurer que le compteur de \partie est à zéro dans les corrigés
    %\phantom{rrr}    
    Les informations suivantes concernent les salaires des hommes et des femmes d'une même entreprise.

    \fbox{\parbox{\linewidth}{
        \setlength{\parskip}{.5cm}
        \begin{center}
            \textbf{Salaires des femmes}

            \Prix{1200} ; \Prix{1230} ; \Prix{1250} ; \Prix{1310} ; \Prix{1370}

            \Prix{1400} ; \Prix{1440} ; \Prix{1500} ; \Prix{1700} ; \Prix{2100}
        \end{center}
    }}

    \fbox{\parbox{\linewidth}{
        \setlength{\parskip}{.5cm}
        \begin{center}
            \textbf{Salaires des hommes}

            Effectif total : 20
            
            Moyenne : \Prix{1769} 
            
            Médiane : \Prix{2000}
        \end{center}
    }}

    \begin{enumerate}
        \item Déterminer le salaire moyen des femmes.
        
        \textcolor{red}{\Stat[Liste,Moyenne]{1200,1230,1250,1310,1370,1400,1440,1500,1700,2100}}

        \textcolor{red}{Le salaire moyen des femmes est donc de \Prix{\Moyenne}.}
        \item Déterminer le salaire moyen de tous les salariés de cette entreprise.
        
        \textcolor{red}{\Stat[Moyenne,Precision=0]{1200/1,1230/1,1250/1,1310/1,1370/1,1400/1,1440/1,1500/1,1700/1,2100/1,1769/20}}
        
        \textcolor{red}{Le salaire moyen des salariés est donc de \Prix{1663}.}
        \item Déterminer le salaire médian des femmes.
        
        Le salaire médian des femmes est la moyenne de la 5e et la 6e valeur soit \textcolor{red}{\Prix{1385}.}
    \end{enumerate}
    \Coupe
    \begin{enumerate}
        \setcounter{enumi}{3}
        \item Comparer les salaires moyen et médian des hommes et des femmes.
        
        \textcolor{red}{Le salaire médian des hommes est plus élévé que celui des femmes. Globalement les hommes sont donc mieux payés.}
    \end{enumerate}
\end{corrige}

