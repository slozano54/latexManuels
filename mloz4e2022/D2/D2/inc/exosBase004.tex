\begin{exercice*}
    Ce diagramme en barres donne la répartition des notes obtenues à un contrôle de matématiques d'une classe de 3\up{e}.
    \par\smallskip
    \Stat[%        
        Graphique,%
        Donnee=Note,%
        Effectif=Effectif,%
        Unitex=1,%
        Unitey=0.4,Pasy=1,%
        Grille,PasGrilley=1,LectureFine,%        
        EpaisseurBatons=4,%
        CouleurDefaut=Grey,
        Origine=7
    ]{8/2,9/5,10/2,11/2,12/3,13/2,14/7,15/2}
    \begin{enumerate}
        \item Déterminer le nombre d'élèves de la classe.
        \item Déterminer la note moyenne de ce contrôle. Interpréter cette valeur.        
    \end{enumerate}
\end{exercice*}
\begin{corrige}
    %\setcounter{partie}{0} % Pour s'assurer que le compteur de \partie est à zéro dans les corrigés
    %\phantom{rrr}    
    Ce diagramme en barres donne la répartition des notes obtenues à un contrôle de matématiques d'une classe de 3\up{e}.
    \par\smallskip
    \Stat[%        
        Graphique,%
        Donnee=Note,%
        Effectif=Effectif,%
        Unitex=1,%
        Unitey=0.4,Pasy=1,%
        Grille,PasGrilley=1,LectureFine,%        
        EpaisseurBatons=4,%
        CouleurDefaut=Grey,
        Origine=7
    ]{8/2,9/5,10/2,11/2,12/3,13/2,14/7,15/2}
    \textcolor{red}{\Stat[Moyenne]{8/2,9/5,10/2,11/2,12/3,13/2,14/7,15/2}}
    \begin{enumerate}
        \item Déterminer le nombre d'élèves de la classe.
        \par\smallskip\textcolor{red}{Le nombre d'élèves de cette classe est donc de \EffectifTotal.}
        \item Déterminer la note moyenne de ce contrôle. Interpréter cette valeur.
        \par\smallskip\textcolor{red}{%
        La note moyenne de ce contrôle est donc de \num{\Moyenne}.\\        
        Cela signifie que si tous les élèves avaient eu la même note, ils auraient eu \num{\Moyenne}.
        }
    \end{enumerate}
\end{corrige}

