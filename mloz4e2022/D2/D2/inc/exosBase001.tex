\begin{exercice*}
    Calculer la moyenne podérée de chaque série statistique. Arrondir au dixième si nécessaire.
    \begin{enumerate}
        \item Série 1\par\smallskip
        \Stat[Tableau]{12/3,34/2,50/4,75/3,100/1}
        \smallskip
        \item Série 2\par\smallskip
        \Stat[Tableau]{3/7,5/4,8/2,11/6,11/2}
        \smallskip
        \item Série 3\par\smallskip
        \Stat[Tableau]{3.4/7,9.4/3,7.2/2,13.2/6,18.1/1}
        \smallskip
    \end{enumerate}
\end{exercice*}
\begin{corrige}
    %\setcounter{partie}{0} % Pour s'assurer que le compteur de \partie est à zéro dans les corrigés
    %\phantom{rrr}    
    Calculer la moyenne podérée de chque série statistique. Arrondir au dixième si nécessaire.
    \begin{enumerate}
        \item Série 1\par\smallskip
        \Stat[Tableau]{12/3,34/2,50/4,75/3,100/1}
        \par\smallskip
        \textcolor{red}{
            \Stat[Moyenne,Precision=1]{12/3,34/2,50/4,75/3,100/1}
        }
        \smallskip
        \item Série 2\par\smallskip
        \Stat[Tableau]{3/7,5/4,8/2,11/6,11/2}
        \par\smallskip
        \textcolor{red}{
            \Stat[Moyenne,Precision=1]{3/7,5/4,8/2,11/6,11/2}
        }
    \end{enumerate}
        \Coupe
    \begin{enumerate}
        \setcounter{enumi}{2}
        \item Série 3\par\smallskip
        \hspace*{-10mm}\Stat[Tableau]{3.4/7,9.4/3,7.2/2,13.2/6,18.1/1}
        \par\smallskip
        \textcolor{red}{
            \Stat[Moyenne,Precision=1]{3.4/7,9.4/3,7.2/2,13.2/6,18.1/1}
        }

    \end{enumerate}
\end{corrige}

