\begin{exercice*}
    Une station de ski sonde 300 skieurs. Le bilan de l'enquête est représenté sur ce graphique. Il indique la répartition en classe des skieurs en fonction de leur âge.

    \Stat[%
    Graphique,%
    Histogramme,%
    UniteAire=1,%
    Pasx=1,%
    Unitex=0.09,%
    Unitey=0.5,%
    Lecture,%
    DonneesSup,%
    Donnee=Âge,%
    Effectif=Nombre de skieurs,%
    ]{%
    0/10/27,%
    10/20/45,%
    20/30/48,%
    30/40/39,%
    40/50/42,%
    50/60/36,%
    60/70/33,%
    70/80/24,%
    80/90/6
    }
    \begin{enumerate}
        \item Compléter le tableau ci-dessous
        
        \Stat[Classes,Tableau,Donnee=Âge,ColVide={1,2,3},Stretch=1.5]{%
        0/10/27,%
        10/20/45,%
        20/30/48%
        }
        
        \medskip
        \Stat[Classes,Tableau,Donnee=Âge,ColVide={1,2,3},Stretch=1.5]{%
        30/40/39,%
        40/50/42,%
        50/60/36%
        }
        
        \smallskip
        \Stat[Classes,Tableau,Donnee=Âge,ColVide={1,2,3},Stretch=1.5]{%
        60/70/33,%
        70/80/24,%
        80/90/6
        }
        
        \smallskip
        \item Déterminer la classe d'âge médiane.
        \item Déterminer la fréquence, en pourcentage, de skieurs ayant un âge strictement inférieur à 20 ans.
    \end{enumerate}
\end{exercice*}
\begin{corrige}
    %\setcounter{partie}{0} % Pour s'assurer que le compteur de \partie est à zéro dans les corrigés
    %\phantom{rrr}    
    Une station de ski sonde 300 skieurs. Le bilan de l'enquête est représenté sur ce graphique. Il indique la répartition en classe des skieurs en fonction de leur âge.

    \begin{enumerate}
        \item Compléter le tableau ci-dessous
        
        \hspace*{-5mm}\Stat[Classes,Tableau,Donnee=Âge]{%
        0/10/27,%
        10/20/45,%
        20/30/48%
        }

        \hspace*{-5mm}\Stat[Classes,Tableau,Donnee=Âge]{%
        30/40/39,%
        40/50/42,%
        50/60/36%
        }

        \hspace*{-5mm}\Stat[Classes,Tableau,Donnee=Âge]{%
        60/70/33,%
        70/80/24,%
        80/90/6
        }

        \item Déterminer la classe d'âge médiane.
        
        L’effectif total de la série est 300. Or,300 = 150 + 150. La classe d'âge médiane est donc celle qui compte le 150e et le 151e skieur.

        C'est $30\leq ... \leq 40$.
        \item Déterminer la fréquence, en pourcentage, de skieurs ayant un âge strictement inférieur à 20 ans.
        
        $\dfrac{27+45}{300}=\num{0.24}=24 \%$
    \end{enumerate}
    

\end{corrige}

