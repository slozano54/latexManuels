\begin{exercice*}
    Ce diagramme donne la répartition de 36 enfants inscrits à une centre de loisir en fonction de leur âge.
    \par\smallskip
    \hspace*{-5mm}\Stat[%    
        Qualitatif,%    
        Graphique,%
        Donnee=Âge,%
        Effectif=Effectif,%
        Unitex=1.2,%
        Unitey=0.5,Pasy=1,%
        Grille,PasGrilley=1,LectureFine,%        
        EpaisseurBatons=7,%
        CouleurDefaut=Grey,
        Origine=7
    ]{8 ans/2,9 ans/3,10 ans/9,11 ans/8,12 ans/9,13 ans/5}
    \begin{enumerate}
        \item Déterminer l'âge moyen des enfants. Arrondir au dixième.
        \item Déterminer l'âge médian des enfants.
    \end{enumerate}
\end{exercice*}
\begin{corrige}
    %\setcounter{partie}{0} % Pour s'assurer que le compteur de \partie est à zéro dans les corrigés
    %\phantom{rrr}    
    \begin{enumerate}
        \item \Stat[Moyenne,Precision=1]{8/2,9/3,10/9,11/8,12/9,13/5}
        \item \Stat[Mediane]{8/2,9/3,10/9,11/8,12/9,13/5}
    \end{enumerate}
\end{corrige}

