\begin{exercice*}
    Ce diagramme circulaire représente la répartition par âge des élèves de l'association sportive d'un collège.

    \Stat[Qualitatif,Graphique,Angle,Rayon=1.5cm,Hachures]{%
        11 ans/37,%
        12 ans/13,%
        13 ans/32,%
        14 ans/18%
    }
    \begin{enumerate}
        \item Déterminer l'âge des élèves les plus nombreux de l'association.
        \item Répondre par Vrai ou Faux. Justifier.
        \begin{enumerate}
            \item Plus de la moitié des élèves ont 11 ans.
            \item Plus d'un quart des élèves ont 13 ans.
            \item Moins d'un quart des élèves ont 12 ans.
            \item Plus de la moitié des élèves ont entre 11 ans et 12 ans.
        \end{enumerate}
    \end{enumerate}
\end{exercice*}
\begin{corrige}
    %\setcounter{partie}{0} % Pour s'assurer que le compteur de \partie est à zéro dans les corrigés
    %\phantom{rrr}    
    Ce diagramme circulaire représente la répartition par âge des élèves de l'association sportive d'un collège.

    \Stat[Qualitatif,Graphique,Angle,Rayon=1.5cm,Hachures]{%
        11 ans/37,%
        12 ans/13,%
        13 ans/32,%
        14 ans/18%
    }
    \begin{enumerate}
        \item Déterminer l'âge des élèves les plus nombreux de l'association.
        
        Le secteur dont la surface est la plus importante est celui des élèves de \textcolor{red}{11ans}
        \item Répondre par Vrai ou Faux. Justifier.
        \begin{enumerate}
            \item Plus de la moitié des élèves ont 11 ans.
            
            Le secteur correspondant aux élèves de 11 ans est inférieure à celle d'un demi-disque donc c'est \textcolor{red}{Faux}.
            \item Plus d'un quart des élèves ont 13 ans.
            
            Le secteur correspondant aux élèves de 13 ans est supérieure à celle d'un quart de disque donc c'est \textcolor{red}{Vrai}.
            \item Moins d'un quart des élèves ont 12 ans.
            
            Le secteur correspondant aux élèves de 12 ans est inférieure à celle d'un quart de disque donc c'est \textcolor{red}{Vrai}.
            \item Plus de la moitié des élèves ont entre 11 ans et 12 ans.
            
            Le secteur correspondant aux élèves de 11 ans et 12 ans est supérieure à celle d'un demi-disque donc c'est \textcolor{red}{Vrai}.
        \end{enumerate}
    \end{enumerate}
\end{corrige}

