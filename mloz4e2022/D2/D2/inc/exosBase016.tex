\begin{exercice*}
    Un jeu de fléchettes consiste à lancer trois fléchettes sur une cible. La position des fléchettes sur la cible détermine le nombre de points obtenus.

    On a relevé dans ce tableau les points obtenus par Denis et Nadia lors de sept parties. Par mégarde, on a effacé le score de la 6e partie de Nadia.


    \begin{tabular}{|>{\columncolor{LightGray}\centering\arraybackslash}m{0.1\linewidth}|*{7}{>{\centering\arraybackslash}m{0.04\linewidth}|}*{2}{>{\centering\arraybackslash}m{0.08\linewidth}|}}        
        \hline
        Partie&1&2&3&4&5&6&7&Moy.&Méd.\\\hline
        Denis&40&35&85&67&28&74&28&&\\\hline
        Nadia&12&62&7&100&81&&30&51&\\\hline
    \end{tabular}
    \begin{enumerate}
        \item Calculer le nombre moyen de points obtenus par Denis.
        \item Sachant que Nadia a obtenu en moyenne 51 points par partie, calculer son nombre de points à la 6e parite.
        \item Déterminer la médiane de la série de points obtenus par Denis.
        \item Déterminer la médiane de la série de points obtenus par Nadia.
    \end{enumerate}
\end{exercice*}
\begin{corrige}
    %\setcounter{partie}{0} % Pour s'assurer que le compteur de \partie est à zéro dans les corrigés
    %\phantom{rrr}    
    Un jeu de fléchettes consiste à lancer trois fléchettes sur une cible. La position des fléchettes sur la cible détermine le nombre de points obtenus.

    On a relevé dans ce tableau les points obtenus par Denis et Nadia lors de sept parties. Par mégarde, on a effacé le score de la 6e partie de Nadia.

    \hspace*{-10mm}
    \begin{tabular}{|>{\columncolor{LightGray}\centering\arraybackslash}m{0.12\linewidth}|*{7}{>{\centering\arraybackslash}m{0.04\linewidth}|}*{2}{>{\centering\arraybackslash}m{0.08\linewidth}|}}        
        \hline
        Partie&1&2&3&4&5&6&7&Moy.&Méd.\\\hline
        Denis&40&35&85&67&28&74&28&&\\\hline
        Nadia&12&62&7&100&81&&30&51&\\\hline
    \end{tabular}
    \begin{enumerate}
        \item Calculer le nombre moyen de points obtenus par Denis.
        
        \textcolor{red}{\Stat[Liste,Moyenne]{40,35,85,67,28,74,28}}
        \item Sachant que Nadia a obtenu en moyenne 51 points par partie, calculer son nombre de points à la 6e parite.
        
        \textcolor{red}{Notons $p$ le nombre de points de la 6e partie.}

        \textcolor{red}{$12+62+7+100+81+p+30=7\times 51 = 357$}

        \textcolor{red}{d'où $p=357-292=65$}

        \textcolor{red}{Nadia a obtenu 65 points à la 6e partie.}
    \end{enumerate}
    \Coupe
    \begin{enumerate}
        \setcounter{enumi}{3}
        \item Déterminer la médiane de la série de points obtenus par Denis.
        
        \textcolor{red}{\Stat[Liste,Mediane]{40,35,85,67,28,74,28}}
        \item Déterminer la médiane de la série de points obtenus par Nadia.
        
        \textcolor{red}{\Stat[Liste,Mediane]{12,62,7,100,81,65,30}}
    \end{enumerate}
\end{corrige}

