\begin{exercice*}
    L'IMC est une grandeur internationale permettant de déterminer la corpulence d'une personne adulte entre 18 ans et 65 ans.

    \fbox{\parbox{\linewidth}{
        %\setlength{\parskip}{.5cm}
        \begin{center}
            \textbf{Normes}

            $\num{18.5}\leq \text{IMC} < 25 \rightarrow$ corpulence normale\\
            $\num{25}\leq \text{IMC} < 30 \rightarrow$ surpoids\\
            $\num{30}\leq \text{IMC} \rightarrow$ obésité
        \end{center}
    }}

    Dans une entreprise, le médecin a fait le bilan de l'IMC des 41 employés. Ce tableau résume les informations recueillies arrondies à l'unité près.

    \begin{tabular}{|>{\columncolor{LightGray}\centering\arraybackslash}m{0.1\linewidth}|*{8}{>{\centering\arraybackslash}m{0.04\linewidth}|}>{\columncolor{LightGray}\centering\arraybackslash}m{0.1\linewidth}|}
        \hline
        IMC&20&22&23&24&25&29&30&33&Total\\\hline
        Effectif&9&12&6&8&2&1&1&2&41\\\hline
    \end{tabular}
    
    \columnbreak
    \begin{enumerate}
        \item Calculer l'IMC moyen arrondi à l'entier près.
        \item Calculer l'IMC médian. Interpréter sa valeur.
        \item On peut parfois lire \og{}\textit{On estime qu'au moins 5\% de lapopulation modiale est en surpoids ou est obèse}\fg{}.
        
        Déterminer si c'est le cas pour cette entreprise.
    \end{enumerate}
\end{exercice*}
\begin{corrige}
    %\setcounter{partie}{0} % Pour s'assurer que le compteur de \partie est à zéro dans les corrigés
    %\phantom{rrr}    
    L'IMC est une grandeur internationale permettant de déterminer la corpulence d'une personne adulte entre 18 ans et 65 ans.

    \fbox{\parbox{\linewidth}{
        %\setlength{\parskip}{.5cm}
        \begin{center}
            \textbf{Normes}

            $\num{18.5}\leq \text{IMC} < 25 \rightarrow$ corpulence normale\\
            $\num{25}\leq \text{IMC} < 30 \rightarrow$ surpoids\\
            $\num{30}\leq \text{IMC} \rightarrow$ obésité
        \end{center}
    }}

    Dans une entreprise, le médecin a fait le bilan de l'IMC des 41 employés. Ce tableau résume les informations recueillies arrondies à l'unité près.

    \begin{tabular}{|>{\columncolor{LightGray}\centering\arraybackslash}m{0.1\linewidth}|*{8}{>{\centering\arraybackslash}m{0.04\linewidth}|}>{\columncolor{LightGray}\centering\arraybackslash}m{0.1\linewidth}|}
        \hline
        IMC&20&22&23&24&25&29&30&33&Total\\\hline
        Effectif&9&12&6&8&2&1&1&2&41\\\hline
    \end{tabular}

    \begin{enumerate}
        \item Calculer l'IMC moyen arrondi à l'entier près.
        
        \textcolor{red}{\Stat[Moyenne,Precision=0]{20/9,22/12,23/6,24/8,25/2,29/1,30/1,33/2}}

        \textcolor{red}{L'IMC moyen est d'environ \Moyenne.}
        \item Calculer l'IMC médian. Interpréter sa valeur.
        
        \textcolor{red}{\Stat[Mediane]{20/9,22/12,23/6,24/8,25/2,29/1,30/1,33/2}}

        \textcolor{red}{L'IMC médian vaut \Mediane, cela signifie que la moitié des employés a un IMC supérieur à 22.} 
    \end{enumerate}
    \Coupe
    \begin{enumerate}
        \setcounter{enumi}{2}
        \item On peut parfois lire \og{}\textit{On estime qu'au moins 5\% de lapopulation modiale est en surpoids ou est obèse}\fg{}.
        
        Déterminer si c'est le cas pour cette entreprise.

        \textcolor{red}{6 employés sur 41 sont en surpoids ou obèse, soit $6\div41\approx15\%$, donc c'est le cas pour cette entreprise.}
    \end{enumerate}
\end{corrige}

