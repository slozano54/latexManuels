\begin{exercice*}
    Sur ce graphique, on a reporté les résultats de Tom tout au long de l'année scolaire.

    \Stat[%
    Representation,%
    Grille,%
    RelieSegment,%
    Graduations,%    
    LabelY=Note,%
    Xmin=0,%
    Xmax=12,%
    Xstep=1.5,%
    Ymin=0,%
    Ymax=20,%
    Ystep=3,%
    PasGrilleY=2%
    ]{1/13,2/12,3/9,4/11,5/6,6/11,7/11,8/17,9/19,10/14,11/3,12/12}

    Déterminer sa note médiane et l'interpréter.
\end{exercice*}
\begin{corrige}
    %\setcounter{partie}{0} % Pour s'assurer que le compteur de \partie est à zéro dans les corrigés
    %\phantom{rrr}    
    \Stat[Mediane]{1/13,2/12,3/9,4/11,5/6,6/11,7/11,8/17,9/19,10/14,11/3,12/12}

    Tom a eu autant de notes supérieures que de notes inférieures à sa not médiane.
\end{corrige}

