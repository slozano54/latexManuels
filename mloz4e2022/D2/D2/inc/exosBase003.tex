\begin{exercice*}
    Ci-dessous, les résultats du dernier contrôle commun du collège Jean Lurçat.

    Calculer la moyenne du collège à ce contrôle. Arrondir au dixième.
    \par\smallskip
    \Stat[Tableau,Donnee=Note,Largeur=5mm]{1/1,2/0,3/3,4/2,5/3,6/5,7/6}
    \par\smallskip
    \Stat[Tableau,Donnee=Note,Largeur=5mm]{8/9,9/15,10/23,11/12,12/15,13/16,14/11}
    \par\smallskip
    \Stat[Tableau,Donnee=Note,Largeur=5mm]{15/7,16/3,17/0,18/2,19/1,20/1}    
\end{exercice*}
\begin{corrige}
    %\setcounter{partie}{0} % Pour s'assurer que le compteur de \partie est à zéro dans les corrigés
    %\phantom{rrr}    
    Ci-dessous, les résultats du dernier contrôle commun du collège Jean Lurçat.
    
    Calculer la moyenne du collège à ce contrôle. Arrondir au dixième.
    \par\smallskip
    \Stat[Tableau,Donnee=Note,Largeur=5mm]{1/1,2/0,3/3,4/2,5/3,6/5,7/6}
    \par\smallskip
    \Stat[Tableau,Donnee=Note,Largeur=5mm]{8/9,9/15,10/23,11/12,12/15,13/16,14/11}
    \par\smallskip
    \Stat[Tableau,Donnee=Note,Largeur=5mm]{15/7,16/3,17/0,18/2,19/1,20/1}  
    \par\medskip
    \textcolor{red}{
        \Stat[]{1/1,2/0,3/3,4/2,5/3,6/5,7/6,8/9,9/15,10/23,11/12,12/15,13/16,14/11,15/7,16/3,17/0,18/2,19/1,20/1}
        La somme des données de la série est : $1\times 1 + 2\times 0 + \dots + 1\times 20=\num{1444}$.\\
        L'effectif total est \EffectifTotal.\\
        Donc la moyenne est : $\dfrac{\num{1444}}{135}\simeq \num{10.7}$
    }
\end{corrige}

