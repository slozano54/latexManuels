\begin{exercice*}
    Soit la série statistique donnant le SMIC\footnote[1]{SMIC : Salaire Minimum Interprofessionnel de Croissance} horaire brut en euros de 2017 à 2023 \textit{(source : INSEE)}

    \Stat[Tableau,Donnee=SMIC,Effectif=Année,Largeur=8mm]{9.88/2018,10.03/2019,10.15/2020,10.25/2021,10.45/2021}

    \Stat[Tableau,Donnee=SMIC,Effectif=Année,Largeur=8mm]{10.57/2022,10.85/2022,11.07/2022,11.27/2023,11.52/2023}

    \begin{enumerate}
        \item Déterminer le SMIC médian entre 2018 et 2023.
        \item Interpréter cette valeur.
    \end{enumerate}
\end{exercice*}
\begin{corrige}
    %\setcounter{partie}{0} % Pour s'assurer que le compteur de \partie est à zéro dans les corrigés
    %\phantom{rrr}    
    \begin{enumerate}
        \item Il y a 10 valeurs, donc le SMIC médian peut être la moyenne des deux SMIC centraux rangés dans l'ordre croissant.Ici, \textcolor{red}{$(\num{10.48}+\num{10.57})\div 2= \num{10.525}$}
        \item La moitié des SMIC est inférieur à \num{10.525}, l'autre moitié est supérieure.
    \end{enumerate}
\end{corrige}

