% ============================================================================================
% ======= Lexique
% ============================================================================================

% ============================================================================================
% LES LIGNES SUIVANTES SERVENT ÉVENTUELLEMENT DE DOCUMENTATION
% ============================================================================================

% % Classement forcé
% % Impose le symbile int directement après le mot intégrale
% \MotDefinition{$\int$}[integrale]{symbole}
% \MotDefinition{intégrale}{mot intégrale}
% \MotDefinition{intégration}{autre mot intégrale}

% % Utilisation d’un mot au pluriel dans le texte alors que le lexique présente normalement le mot au singulier.
% % Par exemple, on utilise le code pour ajouter l'entrée au lexique :
% \MotDefinition[mot]{mots}{Exemple mot lexique distinct du texte}
% % Et comme ceci dans un texte
% Utilisation de \MotDefinition[mot]{mots}{} au pluriel dans le texte alors que le lexique présente normalement le mot au singulier.

% ============================================================================================
% On centralise les entrées du lexique ici
% Quelques définitions pour la maquette ...
% ============================================================================================

% ============================================================================================
% ======= Chapitre N1
% ============================================================================================

\MotDefinition{produit}{Un produit est le résultat d'une multiplication.}

\MotDefinition[distance à zéro (d'un relatif)]{distance à zéro}{
La distance à zéro d’un nombre relatif, c’est la distance qui le sépare de zéro !
Une distance est toujours positive.
}

\MotDefinition{nombre positif}{Un nombre relatif positif est un nombre supérieur à zéro.}

\MotDefinition{nombre négatif}{Un nombre relatif négatif est un nombre inférieur à zéro.}

\MotDefinition{nombres opposés}{Deux nombres relatifs sont dits opposés quand leur somme vaut zéro.}
