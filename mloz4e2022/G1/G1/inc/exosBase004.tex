\begin{exercice*}
    \begin{enumerate}
        \item Justifier que les triangles sont isométriques.
        
        \hspace*{-10mm}
        \begin{tikzpicture}[scale = 0.5]
            % \draw[help lines, color=gray!30, dashed] (0,0) grid (18,10);
            \coordinate[label=above:$A$] (A) at (3,6);
            \coordinate[label=below:$B$] (B) at (1,1);
            \coordinate[label=below:$C$] (C) at (10,3);
            \coordinate[label=below:$D$] (D) at (15,4);
            \coordinate[label=above:$E$] (E) at (17,9);
            \coordinate[label=above:$F$] (F) at (8,7);
            \draw (A) -- (B) -- node[sloped,below] {\large\Lg{4.3}} (C) -- node[sloped,above] {\large\Lg{3.4}} (A);
            \draw (D) -- (E) -- node[sloped,above] {\large\Lg{4.3}} (F) -- node[sloped,below] {\large\Lg{3.4}} (D);
            \pic [draw=red, text=red, -, "\ang{32}",angle eccentricity=1.5] {angle = A--C--B};
            \pic [draw=red, text=red, -, "\ang{32}",angle eccentricity=1.5] {angle = D--F--E};
        \end{tikzpicture}
        % \pagebreak
        % \columnbreak
        \item Justifier que les triangles ne sont pas isométriques.
        
        % \hspace*{-2cm}%
        \begin{tikzpicture}[scale = 0.4]
            % \draw[help lines, color=gray!30, dashed] (0,0) grid (18,10);
            \coordinate[label=above:$G$] (A) at (8,7);
            \coordinate[label=above:$H$] (B) at (1,7);
            \coordinate[label=below:$J$] (C) at (2,1);
            \coordinate[label=below:$K$] (D) at (8,1);
            \coordinate[label=below:$L$] (E) at (15,1);
            \coordinate[label=above:$M$] (F) at (16.375,9.375);
            \draw (A) -- node[sloped,above] {\large\Lg{7}} (B) -- (C) -- node[sloped,above] {\large\Lg{8.5}} (A);
            \draw (D) -- node[sloped,below] {\large\Lg{7}} (E) -- node[sloped,above] {\large\Lg{8.5}} (F) -- (D);
            \pic [draw=red, text=red, -, "\ang{45}",angle eccentricity=1.5] {angle = B--A--C};
            \pic [draw=red, text=red, -, "\ang{45}",angle eccentricity=1.5] {angle = E--D--F};
        \end{tikzpicture}
    \end{enumerate}
\end{exercice*}
\begin{corrige}
    %\setcounter{partie}{0} % Pour s'assurer que le compteur de \partie est à zéro dans les corrigés
    \phantom{rrr}    
    \begin{multicols}2
        \begin{enumerate}
            \item $\widehat{ABC}=\widehat{FED}$; $AB=DE$ et $BC=EF$, les triangles $ABC$ et $DEF$ ont
            un angle égal compris entre deux côtés égaux deux à deux
            
            donc \psshadowbox{ils sont isométriques.}
            \item $\widehat{JHG}=\widehat{MLK}$; $GH=LK$ et $JH\neq LM$, les triangles $GHJ$ et $MK$ ont
            un angle égal compris entre deux côtés qui ne sont pas égaux deux à deux
            
            donc \psshadowbox{ils ne sont pas isométriques.}

        \end{enumerate}
    \end{multicols}
\end{corrige}

