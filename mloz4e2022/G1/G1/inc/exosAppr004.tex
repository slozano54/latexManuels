\begin{exercice*}[Triangles jumeaux]

    \begin{tikzpicture}[scale = 0.5]
        % \draw[help lines, color=black!30, dashed] (0,0) grid (18,10);
        \coordinate[label=below:$S$] (S) at (3,1);
        \coordinate[label=below:$V$] (V) at (9,1);
        \coordinate[label=below:$T$] (T) at (15,1);
        \coordinate[label=above:$U$] (U) at (9,9);
        \draw (S) -- (U) -- (V) -- (S);
        \draw (V) -- (T) -- (U) -- (V);
        \draw[fill=gray!20] (V) -- (10,1) -- (10,2) -- (9,2) -- (V);
        \pic [draw=black, -, "\ang{56}",angle eccentricity=1.5] {angle = U--T--V};        
        \pic [draw=black, -, "\ang{34}",angle eccentricity=1.5, angle radius=7mm] {angle = S--U--V};        
    \end{tikzpicture}

    % \begin{multicols}2
        \begin{enumerate}
            \item Expliquer pourquoi les triangles $SUV$ et $TUV$ sont égaux.
            \item Déterminer la nature du triangle $SUT$. Justifier.
        \end{enumerate}
    % \end{multicols}
\end{exercice*}
\begin{corrige}
    %\setcounter{partie}{0} % Pour s'assurer que le compteur de \partie est à zéro dans les corrigés
    \phantom{rrr}    

    % \begin{multicols}2
        \begin{enumerate}
            \item LA somme des angles du triangle $VUT$ est égale à \ang{180} 
            
            donc $\widehat{VUT} = \ang{180} - (\ang{90} + \ang{56}) = \ang{34}$,

            $\widehat{SVU} = \widehat{UVT} = \ang{90}$ et $\widehat{SUV} = \widehat{VUT} = \ang{34}$

            $SVU$ et $VUT$ ont un côté égal $[VU]$ adjacent à deux angles égaux deux à deux donc \psshadowbox{ils sont égaux}.
            
            \item Les triangles $SVU$ et $VUT$ sont isométriques donc $SU=UT$.
            
            \psshadowbox{Le triangle $SUT$ est donc isocèle en $U$}.
        \end{enumerate}
    % \end{multicols}
\end{corrige}

