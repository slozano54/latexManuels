\definecolor{myGreen}{rgb}{0,0.6,0} %green

\begin{definition}[Triangles isométriques]
    Deux triangles sont \textbf{isométriques} lorsque leurs côtés sont deux à deux de même longueur.
\end{definition}

\begin{exemple*1}
    Les triangles $ABC$, $IJK$, $LMN$ et $XYZ$ se superposent par glissement et ou retournement, ils sont donc isométriques.
    \begin{center}
    \scalebox{0.6}{    
        \begin{tikzpicture}[line cap=round,line join=round,>=triangle 45,x=1cm,y=1cm]
            \clip(-9,-7) rectangle (9,5.5);
            \draw [line width=2pt,color=myGreen] (-7.74,1.07)-- (-1.28,2.07);
            \draw [line width=2pt,color=red] (-1.28,2.07)-- (-4,4.51);
            \draw [line width=2pt,color=blue] (-4,4.51)-- (-7.74,1.07);
            \draw [line width=2pt,color=myGreen] (0.7443407477719717,0.8432900996677757)-- (7.204340747771972,1.8432900996677755);
            \draw [line width=2pt,color=red] (7.204340747771972,1.8432900996677755)-- (4.484340747771972,4.283290099667775);
            \draw [line width=2pt,color=blue] (4.484340747771972,4.283290099667775)-- (0.7443407477719717,0.8432900996677757);
            \draw [line width=2pt,color=myGreen] (-2.1824604714443954,-1.569871362126245)-- (-8.642460471444396,-2.5698713621262446);
            \draw [line width=2pt,color=red] (-8.642460471444396,-2.5698713621262446)-- (-5.922460471444396,-5.009871362126245);
            \draw [line width=2pt,color=blue] (-5.922460471444396,-5.009871362126245)-- (-2.1824604714443954,-1.569871362126245);
            \draw [line width=2pt,color=myGreen] (1.273312430294113,-1.521587110006146)-- (7.758729359192739,-2.340713021179218);
            \draw [line width=2pt,color=red] (7.758729359192739,-2.340713021179218)-- (5.107961464351075,-4.855753690368624);
            \draw [line width=2pt,color=blue] (5.107961464351075,-4.855753690368624)-- (1.273312430294113,-1.521587110006146);
            \begin{small}
                \draw (-8,0.8) node {$A$};
                \draw (-1.1,2.4) node {$B$};
                \draw (-3.8,4.9) node {$C$};
                \draw (0.1,1) node {$I$};
                \draw (7.6,2.2) node {$K$};
                \draw (4.7,4.7) node {$J$};
                \draw (-1.9,-1.1) node {$M$};
                \draw (-8.9,-1.9) node {$L$};
                \draw (-5.9,-5.5) node {$N$};
                \draw (1.3,-1.1) node {$X$};
                \draw (8,-1.9) node {$Y$};
                \draw (5.1,-5.3) node {$Z$};
            \end{small}
        \end{tikzpicture}
    }
    \end{center}
\end{exemple*1}

\begin{propriete}[Premier cas d'égalité \admise]
    Si deux triangles ont un côté de même mesure compris entre deux angles de même mesure deux à deux alors ils sont \textbf{isométriques}.
\end{propriete}

\begin{exemple*1}
    \begin{itemize}
        \item \textcolor{red}{$AC$}=\textcolor{red}{$IK$}
        \item Le côté $[AC]$ est compris entre les angles \textcolor{blue}{$\widehat{CAB}$} et \textcolor{myGreen}{$\widehat{BCA}$}
        \item Le côté $[IK]$ est compris entre les angles \textcolor{blue}{$\widehat{IKJ}$} et \textcolor{myGreen}{$\widehat{JIK}$}
        \item \textcolor{blue}{$\widehat{CAB}$}=\textcolor{blue}{$\widehat{IKJ}$} et \textcolor{myGreen}{$\widehat{BCA}$}=\textcolor{myGreen}{$\widehat{JIK}$}
    \end{itemize}
    Donc les triangles $ABC$ et $IJK$ sont isométriques.
    \begin{center}
        \scalebox{0.6}{            
            \begin{tikzpicture}[line cap=round,line join=round,>=triangle 45,x=1cm,y=1cm]
                \clip(-11,-4) rectangle (5,2);
                \draw [shift={(-10,-2)},line width=2pt,color=blue,fill=blue,fill opacity=0.5] (0,0) -- (-9.462322208025617:0.6) arc (-9.462322208025617:56.309932474020215:0.6) -- cycle;
                \draw [shift={(-4,-3)},line width=2pt,color=myGreen,fill=myGreen,fill opacity=0.5] (0,0) -- (135:0.6) arc (135:170.53767779197437:0.6) -- cycle;
                \draw [shift={(-2,1)},line width=2pt,color=myGreen,fill=myGreen,fill opacity=0.5] (0,0) -- (-45:0.6) arc (-45:-9.462322208025615:0.6) -- cycle;
                \draw [shift={(4,0)},line width=2pt,color=blue,fill=blue,fill opacity=0.5] (0,0) -- (170.53767779197437:0.6) arc (170.53767779197437:236.3099324740202:0.6) -- cycle;
                \draw [line width=2pt] (-10,-2)-- (-8,1);
                \draw [line width=2pt] (-8,1)-- (-4,-3);
                \draw [line width=2pt,color=red] (-4,-3)-- (-10,-2);
                \draw [line width=2pt,color=red] (-2,1)-- (4,0);
                \draw [line width=2pt] (4,0)-- (2,-3);
                \draw [line width=2pt] (2,-3)-- (-2,1);
                \begin{small}
                \draw (-10.4,-2.12) node {$A$};
                \draw (-7.84,1.42) node {$B$};
                \draw (-3.72,-3.34) node {$C$};
                \draw (-2.22,1.5) node {$I$};
                \draw (2.5,-3.34) node {$J$};
                \draw (4.24,0.42) node {$K$};
                \end{small}
            \end{tikzpicture}
        }
    \end{center}
\end{exemple*1}

\begin{propriete}[Deuxième cas d'égalité \admise]
    Si deux triangles ont un angle de même mesure compris entre deux côtés de même mesure deux à deux alors ils sont \textbf{isométriques}.
\end{propriete}

\begin{exemple*1}
    \begin{itemize}
        \item \textcolor{red}{$AC$}=\textcolor{red}{$IK$}
        \item \textcolor{myGreen}{$AB$}=\textcolor{myGreen}{$JK$}
        \item L'angle \textcolor{blue}{$\widehat{CAB}$} est compris entre les côtés \textcolor{red}{$[AC]$} et \textcolor{myGreen}{$[AB]$}
        \item L'angle \textcolor{blue}{$\widehat{IKJ}$} est compris entre les côtés \textcolor{red}{$[IK]$} et \textcolor{myGreen}{$[KJ]$}
        \item \textcolor{red}{$AC$}=\textcolor{red}{$IK$} et \textcolor{myGreen}{$AB$}=\textcolor{myGreen}{$JK$}
    \end{itemize}
    Donc les triangles $ABC$ et $IJK$ sont isométriques.
    \begin{center}
        \scalebox{0.6}{            
            \begin{tikzpicture}[line cap=round,line join=round,>=triangle 45,x=1cm,y=1cm]
                \clip(-11,-4) rectangle (5,2);
                \draw [shift={(-10,-2)},line width=2pt,color=blue,fill=blue,fill opacity=0.5] (0,0) -- (-9.462322208025617:0.6) arc (-9.462322208025617:56.309932474020215:0.6) -- cycle;
                \draw [shift={(4,0)},line width=2pt,color=blue,fill=blue,fill opacity=0.5] (0,0) -- (170.53767779197437:0.6) arc (170.53767779197437:236.3099324740202:0.6) -- cycle;
                \draw [line width=2pt,color=myGreen] (-10,-2)-- (-8,1);
                \draw [line width=2pt] (-8,1)-- (-4,-3);
                \draw [line width=2pt,color=red] (-4,-3)-- (-10,-2);
                \draw [line width=2pt,color=red] (-2,1)-- (4,0);
                \draw [line width=2pt,color=myGreen] (4,0)-- (2,-3);
                \draw [line width=2pt] (2,-3)-- (-2,1);
                \begin{small}
                    \draw (-10.4,-2.12) node {$A$};
                    \draw (-7.84,1.42) node {$B$};
                    \draw (-3.72,-3.34) node {$C$};
                    \draw (-2.22,1.5) node {$I$};
                    \draw (2.5,-3.34) node {$J$};
                    \draw (4.24,0.42) node {$K$};
                \end{small}
            \end{tikzpicture}
        }
    \end{center}
\end{exemple*1}

\begin{propriete}[Triangles isométriques et conséquences \admise]
    Si deux triangles sont isométriques alors :
    \begin{itemize}
        \item leurs angles sont de même mesure;
        \item leurs aires sont égales.
    \end{itemize}
\end{propriete}

\begin{remarques}
    \emoji{warning} \emoji{warning} \emoji{warning}
    
    La réciproque n'est pas forcément vraie.
    \begin{itemize}
        \item Deux triangles peuvent avoir des angles deux à deux de même mesure, sans pour autant être isométriques. Nous verrons ça avec le théorème de Thalès.
        \item Deux triangles peuvent avoir la même aire sans être isométriques.\par
        \textbf{Contre-exemple ci-dessous} :  les triangles $ABC$ et $DEF$ ont une aire de 12 unités d'aire mais ne sont pas isométriques.
    \end{itemize}
    \begin{center}
        \scalebox{0.7}{
            \begin{tikzpicture}[line cap=round,line join=round,>=triangle 45,x=1cm,y=1cm]
                \clip(-8,0) rectangle (6,5);
                \draw[line width=2pt,color=myGreen,fill=myGreen,fill opacity=0.10000000149011612] (-3,1.5998547467921123) -- (-3.5998547467921123,1.5998547467921125) -- (-3.5998547467921123,1) -- (-3,1) -- cycle; 
                \draw[line width=2pt,color=myGreen,fill=myGreen,fill opacity=0.10000000149011612] (4,1.5998547467921123) -- (3.4001452532078877,1.5998547467921125) -- (3.4001452532078877,1) -- (4,1) -- cycle; 
                \draw [line width=2pt] (-7,1)-- (-3,1);
                \draw [line width=2pt] (-3,1)-- (-3,4);
                \draw [line width=2pt] (-3,4)-- (-7,1);
                \draw [line width=2pt] (1,1)-- (5,1);
                \draw [line width=2pt] (5,1)-- (4,4);
                \draw [line width=2pt] (4,4)-- (1,1);
                \draw [line width=2pt] (4,4)-- (4,1);
                \draw [->,line width=2pt] (3.7,0.35) -- (5,0.35);
                \draw [->,line width=2pt] (3,0.35) -- (1,0.35);
                \draw (-4.905632943307733,0.7) node {$4$};
                \draw (-2.4454970600426087,2.7) node {$3$};
                \draw (-5.160129758817918,3.202963742906823) node {$5$};
                \draw (3.4,0.35) node {$4$};
                \draw (5.359071948936407,2.976744351342214) node {$\num{3.16}$};
                \draw (1.9092262275761174,3.202963742906823) node {$\num{4.24}$};
                \draw (3.7,2.8) node {$3$};
                \begin{small}
                    \draw (-7.478878522355162,1) node {$A$};
                    \draw (-2.162722820586847,1) node {$B$};
                    \draw (-2.7848261473895217,4.616834940185631) node {$C$};
                    \draw (0.3822453345150058,1.0256020990974606) node {$D$};
                    \draw (5.359071948936406,1) node {$E$};
                    \draw (4.227974991113361,4.616834940185631) node {$F$};
                    \draw (4.199697567167784,0.7711052835872751) node {$G$};
                \end{small}
            \end{tikzpicture}
        }
    \end{center}

\end{remarques}
