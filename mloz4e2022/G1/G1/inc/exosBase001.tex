\begin{exercice*}[Côtés homologues]
    Compléter les phrases suivantes.
    \begin{enumerate}
        % \begin{spacing}{2}
            \item Ci-dessous les triangles $JCZ$ et $UNI$ sont égaux.\\
            Les côtés homologues aux côtes $[JZ]$, $[ZC]$ et $[CJ]$ sont respectivement \ldots\\
            \hspace*{-1cm}%
            \begin{tikzpicture}[baseline,scale = 0.3]
            
                \tikzset{
                  point/.style={
                    thick,
                    draw,
                    cross out,
                    inner sep=0pt,
                    minimum width=5pt,
                    minimum height=5pt,
                  },
                }
                \clip (-10.286546551104232,-10.831123286479347) rectangle (19.53237615266523,6.302459592272257);
            
            
                    \draw[color={black},fill opacity = 1.1] (0,0)--(-1.8419417397532436,-6.024719962567122)--(-7.286546551104232,3.3024595922722564)--cycle;
                \draw[color={black},fill opacity = 1.1] (10.17547508481482,-2.974189190409857)--(13.325475084814821,2.481770853432104)--(16.53237615266523,-7.831123286479347)--cycle;
                \draw [color={black}] (-0.9209708698766218,-3.012359981283561) node[anchor = center, rotate = -287] {|};
                \draw [color={black}] (11.75047508481482,-0.24620916848887653) node[anchor = center, rotate = 60] {|};
                \draw[color={black}] (-1.6080443759750538,-5.259676157796694) arc (73:120.27000000000001:0.8000000000000002) ;
                \draw[color={black}] (12.92547508481482,1.7889505304045534) arc (-120:-72.72999999999999:0.8) ;
                \draw [color={black}] (-0.7290727207752279,-0.3293219819893699) node[anchor = center, rotate = -245.69] {||};
            \draw[color={black}] (-0.7286546551104233,0.33024595922722566) arc (155.62:253:0.8000000000000002) ;
                \draw [color={black}] (10.959943046056846,-2.8173133867154103) node[anchor = center, rotate = -78.69] {||};
            \draw[color={black}] (10.811165191599862,-3.4598826000168064) arc (-37.38:59.99999999999999:0.8000000000000008) ;
                
                \draw [color={black}] (0.4791477793286072,0.14288948724263872) node[anchor = center,scale=1] {$J$};
                \draw [color={black}] (-1.7277089751239407,-6.511495963911834) node[anchor = center,scale=1] {$Z$};
                \draw [color={black}] (-7.641512274346455,3.6545949857264284) node[anchor = center,scale=1] {$C$};
                
                \draw [color={black}] (9.676464691218818,-3.0056316312051043) node[anchor = center,scale=1] {$I$};
                \draw [color={black}] (13.323670873274153,2.9817675982422247) node[anchor = center,scale=1] {$U$};
                \draw [color={black}] (16.799030899387986,-8.254083386311732) node[anchor = center,scale=1] {$N$};
            
            \end{tikzpicture}
                \item Ci-dessous les triangles $DEY$ et $TXM$ sont égaux.\\
                Les côtés homologues aux côtes $[DE]$, $[EY]$ et $[YD]$ sont respectivement \ldots\\
            \hspace*{-1cm}%
            \begin{tikzpicture}[baseline,scale = 0.3]
            
                \tikzset{
                  point/.style={
                    thick,
                    draw,
                    cross out,
                    inner sep=0pt,
                    minimum width=5pt,
                    minimum height=5pt,
                  },
                }
                \clip (-3.601374624958842,-3) rectangle (26.303708727331852,12.386049285302493);
            
            
                    \draw[color={black},fill opacity = 1.1] (0,0)--(-0.6013746249588419,6.873743416833045)--(7.003242424676053,2.0602416220476423)--cycle;
                \draw[color={black},fill opacity = 1.1] (20.49722589010883,9.386049285302493)--(23.303708727331852,3.0825856275685464)--(14.546279150778291,5.158019850015734)--cycle;
                \draw [color={black}] (-0.30068731247942093,3.4368717084165223) node[anchor = center, rotate = -85] {|};
                \draw [color={black}] (21.90046730872034,6.2343174564355195) node[anchor = center, rotate = -66] {|};
                \draw [color={black}] (3.2009338998586054,4.4669925194403435) node[anchor = center, rotate = -32] {||};
                \draw [color={black}] (18.92499393905507,4.12030273879214) node[anchor = center, rotate = -14] {||};
                \draw [color={black}] (3.5016212123380264,1.0301208110238211) node[anchor = center, rotate = -343] {|||};
                \draw [color={black}] (17.52175252044356,7.272034567659113) node[anchor = center, rotate = 35] {|||};
                \draw[color={black}] (-0.5316500307607153,6.076787658359648) arc (-85:-32.33:0.8000000000000003) ;
                \draw[color={black}] (22.978319412871212,3.8134219936826272) arc (114:166.67000000000002:0.8) ;
                \draw [color={black}] (6.2109651798962275,2.171132412486204) node[anchor = center, rotate = 82.03] {x};
            \draw[color={black}] (6.327276464708507,2.4881084482507894) arc (147.67:196.39999999999998:0.8) ;
                \draw [color={black}] (15.331494512731641,5.311110788194552) node[anchor = center, rotate = -78.97] {x};
            \draw[color={black}] (15.32471733536083,4.9735368080204285) arc (-13.33:35.4:0.8000000000000004) ;
                \draw [color={black}] (0.4508434757598554,0.6608631933802733) node[anchor = center, rotate = -34.3] {||};
            \draw[color={black}] (0.767478621882307,0.22577990378604296) arc (16.39:95:0.7999999999999999) ;
                \draw [color={black}] (20.2861010193076,8.614410581679374) node[anchor = center, rotate = -195.3] {||};
            \draw[color={black}] (19.84506734333288,8.922703593764219) arc (-144.61:-66.00000000000001:0.8000000000000009) ;
                
                \draw [color={black}] (-0.2912347803382618,-0.4064262574211024) node[anchor = center,scale=1] {$D$};
                \draw [color={black}] (-0.8886910914708968,7.282949047379628) node[anchor = center,scale=1] {$E$};
                \draw [color={black}] (7.49459125762071,1.9676331755434349) node[anchor = center,scale=1] {$Y$};
                
                \draw [color={black}] (20.640274338227727,9.865149631253917) node[anchor = center,scale=1] {$M$};
                \draw [color={black}] (23.708596105425087,2.7892151943944015) node[anchor = center,scale=1] {$X$};
                \draw [color={black}] (14.051549341114553,5.085615323706388) node[anchor = center,scale=1] {$T$};
            
            \end{tikzpicture}
                \item Ci-dessous les triangles $AZG$ et $AVH$ sont égaux.\\
                Les côtés homologues aux côtes $[AZ]$, $[ZG]$ et $[GA]$ sont respectivement \ldots\\
            \begin{tikzpicture}[baseline,scale = 0.3]
            
                \tikzset{
                  point/.style={
                    thick,
                    draw,
                    cross out,
                    inner sep=0pt,
                    minimum width=5pt,
                    minimum height=5pt,
                  },
                }
                \clip (-9.947823061489254,-9.597266067154896) rectangle (5.340102655689827,8.943711570994452);
            
            
                    \draw[color={black},fill opacity = 1.1] (0,0)--(-6.947823061489254,-0.8530854038360342)--(-4.89512949377508,5.943711570994452)--cycle;
                \draw[color={black},fill opacity = 1.1] (0,0)--(2.340102655689827,-6.597266067154896)--(-4.740145371803367,-6.068032782885334)--cycle;
                \draw [color={black}] (-3.473911530744627,-0.4265427019180171) node[anchor = center, rotate = -353] {|};
                \draw [color={black}] (1.1700513278449134,-3.298633033577448) node[anchor = center, rotate = -70] {|};
                \draw [color={black}] (-5.921476277632167,2.545313083579209) node[anchor = center, rotate = 73] {||};
                \draw [color={black}] (-1.20002135805677,-6.332649425020115) node[anchor = center, rotate = -5] {||};
                \draw[color={black}] (-6.153786140176196,-0.7555899291119159) arc (7:73.2:0.8000000000000004) ;
                \draw[color={black}] (2.072662352182418,-5.843292802337194) arc (109.53:175.73000000000002:0.7999999999999998) ;
                
                \draw [color={black}] (0.4414540380249563,0.23477293777490793) node[anchor = center,scale=1] {$A$};
                \draw [color={black}] (-7.445832567057559,-0.8085148713777794) node[anchor = center,scale=1] {$Z$};
                \draw [color={black}] (-5.0602546448326,6.415658267654395) node[anchor = center,scale=1] {$G$};
                \draw [color={black}] (0.4414540380249563,0.23477293777490793) node[anchor = center,scale=1] {$ $};
                \draw [color={black}] (2.6711795702720176,-6.97195003616176) node[anchor = center,scale=1] {$H$};
                \draw [color={black}] (-4.96102383935966,-6.516600171992013) node[anchor = center,scale=1] {$V$};
            
            \end{tikzpicture}
        % \end{spacing}
        \end{enumerate}

    \hrefMathalea{https://coopmaths.fr/mathalea.html?ex=5G24-1,s=1,n=3&v=l} % On peut mettre une ref en option si on veut
\end{exercice*}
\begin{corrige}
    %\setcounter{partie}{0} % Pour s'assurer que le compteur de \partie est à zéro dans les corrigés
    \phantom{rrr}    
    \begin{multicols}2
        \begin{enumerate}
            \item Les côtés homologues aux côtes $[JZ]$, $[ZC]$ et $[CJ]$ sont respectivement 
            $[IU]$, $[UN]$ et $[NI]$.
            \item Les côtés homologues aux côtes $[DE]$, $[EY]$ et $[YD]$ sont respectivement 
            $[MX]$, $[XT]$ et $[TM]$.
            \item Les côtés homologues aux côtes $[AZ]$, $[ZG]$ et $[GA]$ sont respectivement 
            $[AH]$, $[HV]$ et $[VA]$.
        \end{enumerate}
    \end{multicols}
\end{corrige}

