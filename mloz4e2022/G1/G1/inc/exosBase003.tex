\begin{exercice*}
    \begin{enumerate}
        \item Justifier que les triangles sont isométriques.        
        \begin{tikzpicture}[scale = 0.3]
            % \draw[help lines, color=gray!30, dashed] (0,0) grid (18,10);
            \coordinate[label=above:$A$] (A) at (3,6);
            \coordinate[label=below:$B$] (B) at (1,1);
            \coordinate[label=below:$C$] (C) at (10,3);
            \coordinate[label=below:$D$] (D) at (15,4);
            \coordinate[label=above:$E$] (E) at (17,9);
            \coordinate[label=above:$F$] (F) at (8,7);
            \draw (A) -- (B) -- node[sloped,below] {\large\Lg{3}} (C) -- (A);
            \draw (D) -- (E) -- node[sloped,above] {\large\Lg{3}} (F) -- (D);
            \pic [draw=red, text=red, -, "\ang{50}",angle eccentricity=1.5] {angle = C--B--A};
            \pic [draw=blue, text=blue, -, "\ang{30}",angle eccentricity=1.5] {angle = A--C--B};
            \pic [draw=red, text=red, -, "\ang{50}",angle eccentricity=1.5] {angle = F--E--D};
            \pic [draw=blue, text=blue, -, "\ang{30}",angle eccentricity=1.5] {angle = D--F--E};
        \end{tikzpicture}
        
        \item Justifier que les triangles ne sont pas isométriques.
        
        \begin{tikzpicture}[scale = 0.35]
            % \draw[help lines, color=gray!30, dashed] (0,0) grid (18,10);
            \coordinate[label=above:$G$] (A) at (1,7);
            \coordinate[label=below:$H$] (B) at (1,1);
            \coordinate[label=below:$J$] (C) at (6.33,1.66);
            \coordinate[label=above:$K$] (D) at (10,7.75);
            \coordinate[label=below:$L$] (E) at (16,1.75);
            \coordinate[label=below:$M$] (F) at (10,1);
            \draw (B) -- node[sloped,above] {\large\Lg{6}} (A) -- (C) -- (B);
            \draw (D) -- (E) -- node[sloped,below] {\large\Lg{6}} (F) -- (D);
            \pic [draw=red, text=red, -, "\ang{83}",angle eccentricity=1.5] {angle = C--B--A};
            \pic [draw=blue, text=blue, -, "\ang{45}",angle eccentricity=1.5] {angle = B--A--C};
            \pic [draw=red, text=red, -, "\ang{83}",angle eccentricity=1.5] {angle = E--F--D};
            \pic [draw=blue, text=blue, -, "\ang{45}",angle eccentricity=1.5] {angle = F--D--E};
        \end{tikzpicture}
    \end{enumerate}
\end{exercice*}
\begin{corrige}
    %\setcounter{partie}{0} % Pour s'assurer que le compteur de \partie est à zéro dans les corrigés
    \phantom{rrr}    
    \begin{multicols}2
        \begin{enumerate}
            \item $AC=FD$; $\widehat{BAC}=\widehat{EDF}$ et $\widehat{BCA}=\widehat{EFD}$, les triangles $ABC$ et $DEF$ ont
            un côté égal adjacent à deux angles égaux deux à deux
            
            donc \psshadowbox{ils sont isométriques.}
            \columnbreak
            \item Les triangles $GHJ$ et $MLK$ ont deux angles égaux deux à deux.
            
            Ces angles sont adjacents à deux côtés qui n'ont pas la même mesure

            donc \psshadowbox{les triangles ne sont pas isométriques.}
        \end{enumerate}
    \end{multicols}
\end{corrige}

