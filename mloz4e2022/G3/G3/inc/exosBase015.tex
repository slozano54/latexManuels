\begin{exercice*}
    {\it Léonard de Pise, dit Fibonacci, mathématicien italien du Moyen Âge serait a l'origine de ce problème.}

    Une lance, longue de 20 pieds\footnote{Un pied est une unité de mesure anglo-saxonne valant environ \Lg{30}}, est posée
    verticalement le long d'une tour considérée comme perpendiculaire au sol. Si on éloigne l'extrémité de la lance qui repose
    sur le sol de 12 pieds de la tour, déterminer de combien descend l'autre extrémité de la lance le long du mur.

    \includegraphics[scale=0.5]{\currentpath/images/tourDePise}
    
\end{exercice*}
\begin{corrige}
    %\setcounter{partie}{0} % Pour s'assurer que le compteur de \partie est à zéro dans les corrigés
    % \phantom{rrr}
    {\it Léonard de Pise, dit Fibonacci, mathématicien italien du Moyen Âge serait a l'origine de ce problème.}

    Une lance, longue de 20 pieds\footnote{Un pied est une unité de mesure anglo-saxonne valant environ \Lg{30}}, est posée
    verticalement le long d'une tour considérée comme perpendiculaire au sol. Si on éloigne l'extrémité de la lance qui repose
    sur le sol de 12 pieds de la tour, déterminer de combien descend l'autre extrémité de la lance le long du mur.

    \includegraphics[scale=0.5]{\currentpath/images/tourDePiseCorr}

    {\red La lance verticale et le sol hoizontl sont perpendiculaires donc on peut appliquer le théorème de Pythagore dans le triangle $ABC$.

    $AB^2+BC^2=AC^2$\\
    $AB^2+12^2=20^2$\\
    $AB^2+144=400$\\
    $AB^2=400-144$\\
    $AB^2=256$\\
    $AB^2=\sqrt{256}$, $AB$ étant une longueur c'est un nombre positif.\\
    $AB=16$ pieds or $AD=BD-AB$ donc $AD=20-16=4$ pieds soit $4\times 30=\Lg{120}$.\\
    La lance descend donc d'environ \Lg{120} ou \Lg[m]{1.20}.
    }
\end{corrige}

