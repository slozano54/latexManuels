\begin{exercice*}
    On dispose de ces informations :\\
    \includegraphics[scale=0.5]{\currentpath/images/remorqueFusil}\\
    Toutes les longueurs indiquées sont en millimètres. On suppose le fond de la remorque rectangulaire.

    \medskip
    Déterminer si le fusil sous-marin peut être placé \og à plat \fg{} dans la remorque. Justifier.
\end{exercice*}
\begin{corrige}
    %\setcounter{partie}{0} % Pour s'assurer que le compteur de \partie est à zéro dans les corrigés
    % \phantom{rrr}
        On dispose de ces informations :\\
    \includegraphics[scale=0.5]{\currentpath/images/remorqueFusilCorr}\\
    Toutes les longueurs indiquées sont en millimètres. On suppose le fond de la remorque rectangulaire.

    \medskip
    Déterminer si le fusil sous-marin peut être placé \og à plat \fg{} dans la remorque. Justifier.

    {\red Le fond de la remorque est un rectangle donc $ABC$ est un triangle rectangle en $B$. La plus grande dimension est la longueur de l'hypoténuse $AC$ de $ABC$.

    Le théorème de Pythagore permet d'écrire : $AC^2=AB^2+BC^2$\\
    $AC^2=\num{1800}^2+\num{1350}^2$\\$AC^2=\num{3240000}+\num{1822500}$\\$AC^2=\num{5062500}$, $AC$ est une longueur.\\$AC=\sqrt{\num{5062500}}$\\
    $AC=\Lg[mm]{2250}$, cette longueur étant supérieure à $\Lg[mm]{2100}$, le fusil peut être poser à plat dans la remorque.    
    }
\end{corrige}

