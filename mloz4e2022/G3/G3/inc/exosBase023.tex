\begin{exercice*}
    Pour vérifier s'il a bien posé une étagère de \Lg{20} de profondeur sur un mur parfaitement vertical, Mme Laure LOGE a pris les mersures indiquées
    sur ce schéma. 

    \begin{minipage}{0.5\linewidth}
        \includegraphics[scale=0.4]{\currentpath/images/etagere.pdf}
    \end{minipage}
    \begin{minipage}{0.5\linewidth}
        Déterminer si son étagère est parfaitement horizontale.
    \end{minipage}    
\end{exercice*}
\begin{corrige}
    %\setcounter{partie}{0} % Pour s'assurer que le compteur de \partie est à zéro dans les corrigés
    % \phantom{rrr}
    Pour vérifier s'il a bien posé une étagère de \Lg{20} de profondeur sur un mur parfaitement vertical, Mme Laure LOGE a pris les mersures indiquées
    sur ce schéma. 

    \begin{minipage}{0.45\linewidth}
        \includegraphics[scale=0.4]{\currentpath/images/etagere.pdf}
    \end{minipage}
    \hfill
    \begin{minipage}{0.5\linewidth}
        Déterminer si son étagère est parfaitement horizontale.
    \end{minipage}  

    \medskip
    {\red Le plus grand côté du triangle dessiné mesure \Lg{29}.\\
    $29^2=841$ et $20^2+21^2=400+441=841$\\
    Le théorème de Pythagore permet donc d'affirmer que ce triangle est rectangle.\\
    Le mur étant vertical, l'étagère est bien horizontale.}
\end{corrige}

