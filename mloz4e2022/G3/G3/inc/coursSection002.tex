\section{Réciproque du théorème de Pythagore}

\begin{propriete}[Réciproque du théorème de Pythagore en français]
    Si dans un triangle, le carré du plus grand côté est égal à la somme des carrés des deux autres côtés alors, ce triangle est rectangle.
\end{propriete}

\begin{propriete}[Réciproque du théorème de Pythagore symboliquement]
    Si dans un triangle $ABC$ tel que $[BC]$ soit le plus grand côté, $BC^2=AC^2+AB^2$ alors, le triangle $ABC$ est rectangle en $A$.
\end{propriete}

\begin{methode*1}
    \exercice
    Justifier que le triangle $RST$, tel que $RS=\Lg{3}$, $ST=\Lg{4}$ et $TR=\Lg{5}$ est rectangle.
    \correction
    Dans le triangle $RST$, dont $[RT]$ est le plus grand côté : 
    $$\left.\begin{array}{l}
    \text{d'une part : }TR^2=5^2=25\\
    \\
    \text{d'autre part : }RS^2+ST^2=3^2+4^2=9+16=25\\
    \end{array}
    \right\rbrace\text{on constate que }TR^2=RS^2+ST^2$$
    Comme $TR^2=RS^2+ST^2$, d'après la réciproque du théorème de Pythagore :\\
    \psshadowbox{le triangle $RST$ est rectangle en $S$.}
\end{methode*1}
