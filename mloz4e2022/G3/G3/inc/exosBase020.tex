\begin{exercice*}
    \begin{enumerate}
        \item Sachant que :
        \begin{itemize}
            \item $ABC$ est un triangle.
            \item $AB=\Lg{17}$, $AC=\Lg{15}$ et $BC=\Lg{8}$.
        \end{itemize}
        En supposant que $ABC$ soit un triangle rectangle, dire quel serait son hypoténuse.Justifier.
        \item Recopier et compléter après avoir calculer $AB^2$ et $AC^2+CB^2$.
        
            Dans le triangle $ABC$, $[AB]$ est le plus \makebox[0.1\linewidth]{\dotfill}
            $$\left.\begin{array}{l}
            AB^2=\makebox[0.1\linewidth]{\dotfill}\\
            \\
            \ldots^2+\ldots^2=\makebox[0.1\linewidth]{\dotfill}\\
            \end{array}
            \right\rbrace \makebox[0.2\linewidth]{\dotfill}$$
        \item Conclure
    \end{enumerate}
    \hrefMathalea{https://coopmaths.fr/mathalea.html?ex=4G21,s=1,n=1,i=0&v=l}
\end{exercice*}
\begin{corrige}
    %\setcounter{partie}{0} % Pour s'assurer que le compteur de \partie est à zéro dans les corrigés
    % \phantom{rrr}
    \begin{enumerate}
        \item Sachant que :
        \begin{itemize}
            \item $ABC$ est un triangle.
            \item $AB=\Lg{17}$, $AC=\Lg{15}$ et $BC=\Lg{8}$.
        \end{itemize}
        En supposant que $ABC$ soit un triangle rectangle, dire quel serait son hypoténuse.Justifier.

        {\red Son plus grand côté étant $[AB]$, ce serait son hypoténuse.}
        \item Recopier et compléter après avoir calculer $AB^2$ et $AC^2+CB^2$.
        
            Dans le triangle $ABC$, $[AB]$ est le plus {\red grand côté.}
            $$\left.\begin{array}{l}
            AB^2={\red 17^2}\\
            AB^2={\red 289}\\
            \\
            {\red AC}^2+{\red CB}^2={\red 15^2+8^2}\\
            {\red AC^2+ CB^2=225+64}\\
            {\red AC^2+ CB^2= 289}\\
            \end{array}
            \right\rbrace {\red AB^2=AC^2+CB^2}$$
        \item Conclure
        
        {\red Dans le triangle $ABC$, puisque $[AB]$ est le plus grand côté et que $AB^2=AC^2+CB^2$, la réciproque du théorème de Pythagore garantit que 
        le triangle $ABC$ est rectangle en $C$.}
    \end{enumerate}
\end{corrige}

