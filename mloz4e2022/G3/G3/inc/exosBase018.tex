\begin{exercice*}
    Recopier et compléter les phrases suivantes.
    \begin{enumerate}
        \item $AB^2=AC^2+CB^2$ donc d'après \makebox[0.1\linewidth]{\dotfill} le triangle $ABC$ est \makebox[0.1\linewidth]{\dotfill} en \makebox[0.1\linewidth]{\dotfill}.
        \item $AB^2\neq AC^2+CB^2$ donc d'après \makebox[0.1\linewidth]{\dotfill} le triangle $ABC$ n'est pas \makebox[0.1\linewidth]{\dotfill} en \makebox[0.1\linewidth]{\dotfill}.
        \item $AB^2=AC^2+CB^2$ et dans le triangle $ABC$, $[AB]$ est le plus grand côté donc d'après \makebox[0.1\linewidth]{\dotfill} le triangle $ABC$ n'est pas \makebox[0.1\linewidth]{\dotfill}.
        \item $MR^2=ME^2+ER^2$ donc \makebox[0.1\linewidth]{\dotfill}
        \item $MR^2\neq ME^2+ER^2$ et \makebox[0.1\linewidth]{\dotfill} donc \makebox[0.1\linewidth]{\dotfill}
    \end{enumerate}    
\end{exercice*}
\begin{corrige}
    %\setcounter{partie}{0} % Pour s'assurer que le compteur de \partie est à zéro dans les corrigés
    % \phantom{rrr}
    Recopier et compléter les phrases suivantes.

    \begin{enumerate}
        \item $AB^2=AC^2+CB^2$ donc d'après {\red la réciproque du théorème de Pythagore,} le triangle $ABC$ est {\red rectangle} en {\red $C$}.
        \item $AB^2\neq AC^2+CB^2$ donc d'après {\red la contraposée du théorème de Pythagore,} le triangle $ABC$ n'est pas {\red rectangle} en {\red $C$}.
    \end{enumerate}
    \Coupe
    \begin{enumerate}
        \setcounter{enumi}{2}
        \item $AB^2=AC^2+CB^2$ et dans le triangle $ABC$, $[AB]$ est le plus grand côté donc d'après {\red la copntraposée du théorème de Pythagore} le triangle $ABC$ n'est pas {\red rectangle}.
        \item $MR^2=ME^2+ER^2$ donc {\red d'après la réciproqe du théorème de Pythagore, le triangle $MER$ est rectrangle en $E$.}
        \item $MR^2\neq ME^2+ER^2$ et {\red $[MR]$ est le plus grand côté du triangle $MER$} donc {\red d'après la contraposée du théorème de Pythagore, le triangle $MER$ n'est pas rectangle.}
    \end{enumerate}
\end{corrige}

