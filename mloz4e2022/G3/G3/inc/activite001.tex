\begin{activite}[Prérequis - Connaissances antérieures]
    \vspace*{-10mm}
    \begin{spacing}{1.5}
        \begin{enumerate}
            \item Calculer l'aire du carré $ABCD$ de côté \Lg{8}.
            \item Déterminer la valeur de $8^2$.
            \item $IJKL$ est un carré d'aire \Aire{25}. Déterminer la longueur $IJ$.
            \item Déterminer les nombres dont le carré vaut $25$.
            \item Exprimer l'aire du carré $EFGH$ en fonction de $c$.
            
            \begin{Geometrie}[CoinBG={(-u,-u)},CoinHD={(6u,6u)}]
                u:=0.4cm;
                pair E,F,G,H;
                E=u*(1,1);
                F=u*(5,1);
                G=1[F,rotation(E,F,-90)];
                H=1[E,rotation(F,E,90)];
                label.llft(btex E etex,E);
                label.lrt(btex  F etex,F);
                label.urt(btex  G etex,G);
                label.ulft(btex H etex,H);
                trace E--F--G--H--cycle;
                trace codeperp(E,F,G,5);
                trace codeperp(F,G,H,5);
                trace codeperp(G,H,E,5);
                marque_s:=marque_s/3;
                trace codesegments(E,F,F,G,2);
                trace codesegments(G,H,H,E,2);
                trace cotation(E,F,-4mm,-4mm,btex $c$ etex);
            \end{Geometrie}
            \item Exprimer l'aire du carré $EFGH$ en fonction de $EF$.
            
            \begin{Geometrie}[CoinBG={(-u,-u)},CoinHD={(6u,6u)}]
                u:=0.4cm;
                pair E,F,G,H;
                E=u*(1,4);
                F=u*(4,1);
                G=1[F,rotation(E,F,-90)];
                H=1[E,rotation(F,E,90)];
                label.llft(btex E etex,E);
                label.lrt(btex  F etex,F);
                label.urt(btex  G etex,G);
                label.ulft(btex H etex,H);
                trace E--F--G--H--cycle;
                trace codeperp(E,F,G,5);
                trace codeperp(F,G,H,5);
                trace codeperp(G,H,E,5);
                marque_s:=marque_s/3;
                trace codesegments(E,F,F,G,2);
                trace codesegments(G,H,H,E,2);            
            \end{Geometrie}
            \item Dans le triangle $IJK$ rectangle en $I$, déterminer le côté opposé à l'angle droit $\widehat{JIK}$.
            
            \begin{Geometrie}[CoinBG={(-u,-u)},CoinHD={(8u,4u)}]
                u:=0.4cm;
                pair I,J,K;
                I=u*(1,1);
                J=u*(7,1);            
                K=3/5[I,rotation(J,I,90)];
                label.llft(btex I etex,I);
                label.lrt(btex  J etex,J);
                label.ulft(btex  K etex,K);
                trace I--J--K--cycle;
                trace codeperp(K,I,J,5);
            \end{Geometrie}       
            \item Dans le triangle $ABC$ rectangle en $A$ :
            \begin{enumerate}
                \item Déterminer le côté le plus long.
                \item Déterminer le côté opposé à l'angle droit.
                \item Rappeler comment s'appelle ce côté de manière générale.
            \end{enumerate}
            \item Déterminer le type d'opération permettant de trouver le nombre manquant dans $48 + \dots = 57$.
            \item Nommer tous les triangles rectangles de cette figure et indiquer leur hypoténuse. FIGURE A FAIRE.
        \end{enumerate}
    \end{spacing}
\end{activite}