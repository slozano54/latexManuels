\begin{activite}[Théorème direct - Puzzle de Henry PÉRIGAL]
    \begin{myBox}{Henry Perigal (1801-1898)}
        Henry Perigal était un mathématicien amateur qui est principalement connu pour une élégante preuve du théorème de Pythagore.
        Son schéma a été gravé, sur sa pierre tombale.
        
        \smallskip
        \hrefLien{http://plus.maths.org/issue16/features/perigal/}{Pour les curieux, un article en anglais.}
    \end{myBox}

    Sur la figure ci dessous, le triangle $ABC$ rectangle en $A$. On a construit les carrés sur chacun des côtés du triangle $ABC$.
    Nous allons construire de quoi découper ces carrés et établir une preuve par dissection du théorème de Pythagore.

    \begin{minipage}{0.4\linewidth}
        \begin{enumerate}
            \item Sur la figure ci-contre, coller les pièces \Circled{1} \Circled{2} \Circled{3} \Circled{4} \Circled{5}
            obtenues puis découpées en suivant les consignes de la page suivante afin de recouvrir la surface du carré $BCIJ$.
            \item Exprimer l'aire du carré $AGHC$ en fonction de $AC$.\smallskip\\\makebox[\linewidth]{\dotfill}
            \item Exprimer l'aire du carré $BEFA$ en fonction de $AB$.\smallskip\\\makebox[\linewidth]{\dotfill}
            \item Exprimer l'aire du carré $CIJB$ en fonction de $BC$.\smallskip\\\makebox[\linewidth]{\dotfill}
            \item Écrire une égalité liant les aires des trois carrés.\smallskip\\\makebox[\linewidth]{\dotfill}
            \item Énoncer le théorème de Pythagore à l'aide d'une phrase conditionnelle.\smallskip\\\makebox[1.7\linewidth]{\dotfill}\medskip\\\makebox[1.7\linewidth]{\dotfill}\medskip\\\makebox[1.7\linewidth]{\dotfill}
        \end{enumerate}
    \end{minipage}
    \hfill
    \begin{minipage}{0.55\linewidth}
        \begin{tikzpicture}[baseline,scale=0.75]
            % \draw[help lines, color=black!30, dashed] (0,-3) grid (15,16);
            % On définit les points
            \coordinate[label=above left:$A$] (A) at (4,8);
            \coordinate[label=above right:$B$] (B) at (11,8);
            \coordinate[label=below left:$C$] (C) at (4,5);
            % \draw (A)--(B)--(C)--cycle;
            \tkzMarkRightAngles[size=0.3](C,A,B);
            \coordinate[label=above right:$E$] (E) at (11,15);
            \coordinate[label=above left:$F$] (F) at (4,15);
            % \draw (B)--(E)--(F)--(A);
            \coordinate[label=above left:$G$] (G) at (1,8);
            \coordinate[label=below left:$H$] (H) at (1,5);
            % \draw (A)--(G)--(H)--(C);
            \tkzDefPointBy[rotation=center C angle -90](B);
            \tkzGetPoint{I};
            \tkzDefPointBy[rotation=center B angle 90](C);
            \tkzGetPoint{J};
            \tkzLabelPoint[below left](I){$I$};
            \tkzLabelPoint[below right](J){$J$};
            % Tracés
            \draw (C)--(I)--(J)--(B)--cycle;
            % Construction demandée
            \tkzInterLL(F,B)(A,E);
            \tkzGetPoint{O};            
            \tkzDefPointBy[projection=onto B--C](O);
            \tkzGetPoint{O1};
            \tkzInterLL(A,B)(O,O1);
            \tkzGetPoint{K};
            \tkzLabelPoint[below left](K){$K$};
            \tkzInterLL(E,F)(O,O1);
            \tkzGetPoint{L};
            \tkzLabelPoint[above right](L){$L$};
            \draw (K)--(L);
            \tkzDefPointBy[rotation=center O angle 90](L);
            \tkzGetPoint{M};
            \tkzLabelPoint[above left](M){$M$};
            \tkzInterLL(M,O)(E,B);
            \tkzGetPoint{N};
            \tkzLabelPoint[below right](N){$N$};
            \draw (M)--(N);                        
            % Coloriage
            \draw[ultra thick,color=blue,fill=blue!10] (A)--(G)--(H)--(C)--cycle;
            \draw[ultra thick,color=red,fill=red!10] (F)--(L)--(O)--(M)--cycle;
            \draw[ultra thick,color=red,fill=red!10] (M)--(O)--(K)--(A)--cycle;
            \draw[ultra thick,color=red,fill=red!10] (K)--(O)--(N)--(B)--cycle;
            \draw[ultra thick,color=red,fill=red!10] (L)--(O)--(N)--(E)--cycle;
            \tkzLabelPoint[below right,xshift=2mm](O){$O$};            
            \tkzMarkRightAngles[size=0.3](L,O,M);
            % Marquage des pièces
            \tkzDefMidPoint(A,H);
            \tkzGetPoint{Un};
            \tkzLabelPoint[below, yshift=3mm](Un){\Circled{1}}
            \coordinate[label=below:\Circled{2}, yshift=3mm] (Deux) at (5.5,12.5);
            \coordinate[label=below:\Circled{3}, yshift=3mm] (Trois) at (6.5,9.5);
            \coordinate[label=below:\Circled{4}, yshift=3mm] (Quatre) at (8.5,13.5);
            \coordinate[label=below:\Circled{5}, yshift=3mm] (Cinq) at (9.5,10.5);
        \end{tikzpicture}    
    \end{minipage}
    
    \pagebreak
    \vspace*{-15mm}
    Sur la figure ci-dessous : 
    \begin{itemize}
        \item Construire le point $O$ à l'intersection des diagonales du carré $ABEF$.\\ ATTENTION : NE PAS TRACER LES DIAGONALES ENTIÈREMENT.
        \item Construire la perpenduiculaire à $(BC)$ passant par $O$. Elle coupe $[AB]$ en $K$ et $[FE]$ en $L$.
        \item Construire la perpendiculaire à $(KL)$ passant par $O$. Elle coupe $[AF]$ en $M$ et $[EB]$ en $N$.
        \item Numéroter le quadrilatère $AGHC$ \Circled{1}.
        \item Numéroter le quadrilatère $FLOM$ \Circled{2}.
        \item Numéroter le quadrilatère $MOKA$ \Circled{3}.
        \item Numéroter le quadrilatère $LONE$ \Circled{4}.
        \item Numéroter le quadrilatère $KONB$ \Circled{5}.
        \item Découper les cinq pièces et finir l'activité.
    \end{itemize}
    \begin{tikzpicture}[baseline,scale=0.75]
        % \draw[help lines, color=black!30, dashed] (0,-3) grid (15,16);
        \coordinate[label=above left:$A$] (A) at (4,8);
        \coordinate[label=above right:$B$] (B) at (11,8);
        \coordinate[label=below left:$C$] (C) at (4,5);
        \draw (A)--(B)--(C)--cycle;
        \tkzMarkRightAngles[size=0.3](C,A,B);
        \coordinate[label=above right:$E$] (E) at (11,15);
        \coordinate[label=above left:$F$] (F) at (4,15);
        \draw (B)--(E)--(F)--(A);
        \coordinate[label=above left:$G$] (G) at (1,8);
        \coordinate[label=below left:$H$] (H) at (1,5);
        \draw (A)--(G)--(H)--(C);
        \tkzDefPointBy[rotation=center C angle -90](B);
        \tkzGetPoint{I};
        \tkzDefPointBy[rotation=center B angle 90](C);
        \tkzGetPoint{J};
        \tkzLabelPoint[below left](I){$I$};
        \tkzLabelPoint[below right](J){$J$};
        \draw (C)--(I)--(J)--(B);
    \end{tikzpicture}
    
    \begin{remarque}
        $O$ est un centre de symétrie de la figure obtenue dans le carré $ABEF$. De plus $(MC)$//$(NB)$ et $(MN)$//$(BC)$ donc $MNBC$ est un parallélogramme. 
        Par conséquent, $MN=BC$.\\ On en déduit donc que $OM=ON=OK=OL=\dfrac{BC}{2}$.
    \end{remarque}
\end{activite}