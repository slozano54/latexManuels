\begin{activite}[Réciproque du théorème direct]
    \partie[Énoncé de la réciproque du théorème de Pythagore]
    \begin{enumerate}
        \item Écrire le théorème de Pythagore sous la forme : \og Si \makebox[0.1\linewidth]{\dotfill} alors, \makebox[0.1\linewidth]{\dotfill} \fg
        \item Écrire la réciproque du théorème de Pythoagore.
        
        Pour le moment, on n'est pas sûrs que la réciproque du théorème de Pythagore est vraie.
        \item Énumérer les conditions de cette réciproque.
        \item Énumérer les conclusions de cette réciproque.
    \end{enumerate}

    \partie[Conjecture]
    Pour chacun des triangles ci-après :
    \begin{itemize}
        \item Triangle ABC tel que $AB=\Lg{3}$, $AC=\Lg{4}$ et $BC=\Lg{5}$.
        \item Triangle GTP tel que $GT=\Lg{5}$, $TP=\Lg{13}$ et $GP=\Lg{12}$.
        \item Triangle XMZ tel que $XM=\Lg{6}$, $XZ=\Lg{6.8}$ et $MZ=\Lg{3.2}$.
    \end{itemize}
    \begin{enumerate}
        \item Construire le triangle.
        \item Vérifier par le calcul que le triangle vérifie les conditions de la réciproque du théorème de Pythagore.
        \item Dire si la nature du triangle semble correspondre aux conclusions attendues dans la réciproque du théorème de Pythagore.
    \end{enumerate}

    \partie[Démonstration]
    $ABC$ est un triangle particulier dont les côtés vérifient les conditions de la réciproque du théorème de Pythagore, c'est à dire que $AC^2=AB^2+BC^2$.

    On constuit un triangle $BCD$ rectangle en $B$ tel que $BD=AB$.

    \begin{minipage}{0.3\linewidth}
        \begin{tikzpicture}[scale=0.7]
            % \draw[help lines, color=black!30, dashed] (0,0) grid (8,6);        
            \coordinate[label=left:$B$] (B) at (1,4);
            \coordinate[label=right:$C$] (C) at (5,4);
            \coordinate[label=left:$D$] (D) at (1,1);
            \draw (B)--(C)--(D)--cycle;
            \tkzMarkRightAngles[size=0.3](D,B,C);
            \tkzDefPointBy[rotation=center B angle 170](D);
            \tkzGetPoint{A};
            \tkzLabelPoint[above](A){$A$};
            \draw (B)--(A)--(C);
            \tkzMarkSegments[mark=oo,size=6pt](B,A B,D);        
        \end{tikzpicture}
    \end{minipage}
    \begin{minipage}{0.7\linewidth}        
        \begin{enumerate}
            \item À l'aide du théorème de Pythagore, montrer que $DC=AC$.
            
            \emoji{light-bulb} \mbox{Si $DC^2=AC^2$ et que $DC$ et $AC$ sont des longueurs, alors $DC=AC$.}
            \item En déduire que $(BC)$ est la médiatrice de $[AD]$.
            \item En déduire que $(AB)$ est perpendiculaire à $(BC)$.
            \item En déduire que le triangle $ABC$ est rectangle en $B$.
        \end{enumerate}
    \end{minipage}
\end{activite}