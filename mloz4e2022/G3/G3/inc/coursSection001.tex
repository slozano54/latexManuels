\section{Théorème direct}

\begin{definition}
    Le {\bfseries carré}  d'un nombre est le produit du nombre par lui-même.
    $$a^2=a\times a$$
\end{definition}

\begin{definition}
    La {\bfseries racine carrée} d'un nombre positif $a$ est le nombre positif, noté $\sqrt{a}$, dont le carré vaut $a$.
    $$\left(\sqrt{a}\right)^2=a\kern20mm(a>0)$$
\end{definition}

\begin{propriete}
    Si un \textbf{triangle est rectangle} alors, le carré de l'hypoténuse est égal à la somme des carrés des deux côtés de l'angle droit.
\end{propriete}

\begin{preuve}
    Voir Activité 4
\end{preuve}

\begin{propriete}[Relation de Pythagore]
    \begin{minipage}{0.5\linewidth}
        \begin{tikzpicture}[scale=1]
            \coordinate[label=below left:$A$] (A) at (1,1);
            \coordinate[label=above:$B$] (B) at (4,5);
            \tkzDefPointWith[orthogonal normed,K=3](B,A);
            \tkzGetPoint{C};
            \tkzLabelPoint[below right](C){$C$};
            \draw (A)--(B)--(C)--cycle;
            \tkzMarkRightAngles[size=0.3](A,B,C);
        \end{tikzpicture}
    \end{minipage}
    \begin{minipage}{0.5\linewidth}
        Dans le triangle $ABC$ rectangle en $B$,\\le théorème de Pythagore permet d'écrire\\ une \textbf{relation de Pythagore} :
        $$AC^2=AB^2+BC^2$$
    \end{minipage}
\end{propriete}

\begin{methode*1}[Calculer l'hypoténuse]
    Texte commun Penser à utiliser le paquet ProfCollege
    \exercice
    Exo
    \correction
    Corr
\end{methode*1}

\begin{methode*1}[Calculer un côté de l'angle droit]
    Texte commun
    \exercice
    Exo
    \correction
    Corr
\end{methode*1}
