\section{Contraposée du théorème de Pythagore}

\begin{propriete}[Contraposée du théorème de Pythagore en français]
    Si dans un triangle, le carré du plus grand côté n'est pas égal à la somme des carrés des deux autres côtés alors, ce triangle n'est pas rectangle.
\end{propriete}

\begin{propriete}[Contraposée du théorème de Pythagore symboliquement]    
    Si dans un triangle $ABC$ tel que $[BC]$ soit le plus grand côté, $BC^2\not=AC^2+AB^2$ alors, le triangle $ABC$ n'est pas rectangle.
\end{propriete}

\begin{methode*1}
    \exercice
    Justifier que le triangle $EHI$ tel que $EH=\Lg{10.5}$, $EI=\Lg{8}$\\ et $HI=\Lg{6}$ n'est pas rectangle.
    \correction
    \begin{enumerate}
        \item \underline{Première façon de rédiger} :
        Dans le triangle $EHI$, dont $[EH]$ est le plus grand côté :
        $$\left.\begin{array}{l}
        \text{d'une part : }EH^2=10,5^2=110,25\\
        \\
        \text{d'autre part : }EI^2+IH^2=8^2+6^2=64+36=100\\
        \end{array}
        \right\rbrace \text{on constate que }EH^2\not=EI^2+IH^2$$
        donc d'après la contraposée du théorème de Pythagore :\\ \psshadowbox{le triangle $EHI$ n'est pas un triangle rectangle.}\medskip
        \item \underline{Seconde façon de rédiger} :
        Dans le triangle $EHI$, dont $[EH]$ est le plus grand côté.
        $$\left.\begin{array}{l}
        \text{d'une part : }EH^2=10,5^2=110,25\\
        \\
        \text{d'autre part : }EI^2+IH^2=8^2+6^2=64+36=100\\
        \end{array}
        \right\rbrace \text{on constate que }EH^2\not=EI^2+IH^2$$
        \textbf{or}, si le triangle était rectangle,d'après le théorème de Pythagore on aurait égalité\\
        \textbf{donc} \psshadowbox{le triangle $EHI$ n'est pas un triangle rectangle.}
    \end{enumerate}    
\end{methode*1}