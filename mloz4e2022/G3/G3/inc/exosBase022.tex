\begin{exercice*}
    Pour savoir si un mur est bien vertical, un ma\c con utilise une règle de \Lg[m]{1} et fait une marque à \Lg{60} sur le sol ainsi qu'une autre à 
    \Lg{80} du sol sur le mur.

    \begin{minipage}{0.5\linewidth}
        \includegraphics[scale=0.4]{\currentpath/images/maconnerie.pdf}
    \end{minipage}
    \begin{minipage}{0.5\linewidth}
        Expliquer comment placer la règle pour vérifier la verticalité du mur.
    \end{minipage}    
\end{exercice*}
\begin{corrige}
    %\setcounter{partie}{0} % Pour s'assurer que le compteur de \partie est à zéro dans les corrigés
    % \phantom{rrr}
    Pour savoir si un mur est bien vertical, un ma\c con utilise une règle de \Lg[m]{1} et fait une marque à \Lg{60} sur le sol ainsi qu'une autre à 
    \Lg{80} du sol sur le mur.

    \begin{minipage}{0.45\linewidth}
        \includegraphics[scale=0.4]{\currentpath/images/maconnerie.pdf}
    \end{minipage}
    \hfill
    \begin{minipage}{0.5\linewidth}
        Expliquer comment placer la règle pour vérifier la verticalité du mur.
    \end{minipage}  

    \medskip
    {\red \Lg[m]{1} = \Lg{100}.\\
    $60^2+80^2=\num{10000}$ et $100^2=\num{10000}$\\
    La réciproque du théorème de Pythagore permet de justifier que le triangle formé par les mesures \Lg{60}, \Lg{80} et \Lg{100} est rectangle.\\
    Il suffit donc de placer la règle entre les deux marques.}
\end{corrige}

