\begin{exercice*}
    \begin{enumerate}
        \item Compléter la figure sachant que :
        \begin{itemize}
            \item $TOC$ est un triangle.
            \item $TO=\Lg[mm]{77}$, $OC=\Lg[mm]{35}$ et $CT=\Lg[mm]{85}$.
        \end{itemize}

        \medskip
        \begin{tikzpicture}
            \coordinate[label=above left:$\ldots$] (O) at (1,3);
            \coordinate[label=below left:$\ldots$] (C) at (1,1);
            \coordinate[label=above right:$\ldots$] (T) at (5,3);
            \draw (T)--(O)--(C)--cycle;
        \end{tikzpicture}
        \item En supposant que $TOC$ soit un triangle rectangle, dire quel serait son hypoténuse.
        \item Recopier et compléter après avoir calculer $CT^2$ et $CO^2+OT^2$.
            $$\left.\begin{array}{l}
            CT^2=\makebox[0.1\linewidth]{\dotfill}\\
            \\
            \ldots^2+\ldots^2=\makebox[0.1\linewidth]{\dotfill}\\
            \end{array}
            \right\rbrace CT^2\neq \ldots^2+\ldots^2$$
        \item Conclure.
    \end{enumerate}
    \hrefMathalea{https://coopmaths.fr/mathalea.html?ex=4G21,s=2,n=1,i=0&v=l}
\end{exercice*}
\begin{corrige}
    %\setcounter{partie}{0} % Pour s'assurer que le compteur de \partie est à zéro dans les corrigés
    % \phantom{rrr}
    \begin{enumerate}
        \item Compléter la figure sachant que :
        \begin{itemize}
            \item $TOC$ est un triangle.
            \item $TO=\Lg[mm]{77}$, $OC=\Lg[mm]{35}$ et $CT=\Lg[mm]{85}$.
        \end{itemize}

        \medskip
        \begin{tikzpicture}
            \coordinate[label=above left: {\red $O$}] (O) at (1,3);
            \coordinate[label=below left: {\red $C$}] (C) at (1,1);
            \coordinate[label=above right:{\red $T$}] (T) at (5,3);
            \draw (T)--(O)--(C)--cycle;
        \end{tikzpicture}
        \item En supposant que $TOC$ soit un triangle rectangle, dire quel serait son hypoténuse.
        
        {\red Ce serait son plus grand côté, c'est à dire $[CT]$.}
        \item Recopier et compléter après avoir calculer $CT^2$ et $CO^2+OT^2$.
            $$\left.\begin{array}{l}
            CT^2={\red 85^2}\\
            CT^2={\red \num{7225}}\\
            \\
            {\red CO}^2+{\red OT}^2={\red 35^2 + 77^2}\\
            {\red CO}^2+{\red OT}^2={\red \num{1225} + \num{5929}}\\
            {\red CO}^2+{\red OT}^2={\red \num{7154}}\\
            \end{array}
            \right\rbrace CT^2\neq {\red CO}^2+{\red OT}^2$$
        \item Conclure.
        
        {\red Si le triangle $TOC$ était rectangle, il le serait en $O$ et d'après la théorème de Pythagore, on aurait $CT^2=CO^2+OT^2$.\\
        La contraposée du théorème de Pythagore permet donc de conclure que le triangle $TOC$ n'est pas rectangle.}
    \end{enumerate}
\end{corrige}
    
    