\begin{exercice*}
    Dans chaque cas, calculer la longueur manquante. Arrondir au millimètre près si nécessaire.\\
    Les longueurs sont exprimées en \Lg{}.

    {\it Les figures sont données à titre indicatif.}
    \begin{multicols}{2}
        \begin{enumerate}
            \item \phantom{rrr}\\\Pythagore[FigureSeule,Echelle=8mm,Angle=170]{SRT}{3.7}{2.7}{}
            
            % \Pythagore[Precision=1]{SRT}{3.7}{2.7}{}
            \item \phantom{rrr}\\\Pythagore[FigureSeule,Echelle=8mm]{HFG}{3.6}{2.5}{}
            
            % \Pythagore[Exact]{HFG}{3.6}{2.5}{}
        \end{enumerate}
    \end{multicols}
    \hrefMathalea{https://coopmaths.fr/mathalea.html?ex=4G20,s=3,s2=2,n=2,video=M9sceJ8gzNc,i=0&v=l}
\end{exercice*}
\begin{corrige}
    %\setcounter{partie}{0} % Pour s'assurer que le compteur de \partie est à zéro dans les corrigés
    % \phantom{rrr}
    Dans chaque cas, calculer la longueur manquante. Arrondir au millimètre près si nécessaire.\\
    Les longueurs sont exprimées en \Lg{}.

    {\it Les figures sont données à titre indicatif.}

    \begin{enumerate}
        \item \phantom{rrr}\\
        \begin{minipage}{0.15\linewidth}
            \Pythagore[FigureSeule,Echelle=6mm,Angle=170]{SRT}{3.7}{2.7}{}
        \end{minipage}
        \begin{minipage}{0.8\linewidth}        
            {\red \Pythagore[Precision=1]{SRT}{3.7}{2.7}{}}
        \end{minipage}
        \item \phantom{rrr}\\
        \begin{minipage}{0.15\linewidth}
            \Pythagore[FigureSeule,Echelle=6mm]{HFG}{3.6}{2.5}{}
        \end{minipage}
        \begin{minipage}{0.8\linewidth}        
            {\red \Pythagore[Precision=1]{HFG}{3.6}{2.5}{}}
        \end{minipage}       
    \end{enumerate}
\end{corrige}