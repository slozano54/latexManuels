    %\setcounter{partie}{0} % Pour s'assurer que le compteur de \partie est à zéro dans les corrigés
    % \phantom{rrr}
    Utiliser la calculatrice pour calculer les expressions suivantes.

    \begin{enumerate}
        \item $A=8^2+12^2$\\{\red $A=64+144$\\$A=208$}
        \item $B=8^2-12^2$\\{\red $B=64-144$\\$B=-80$}
        \item $C=\num{5.9}^2+3^2$\\{\red $C=\num{34.81}+9$\\$C=\num{43.81}$}
        \item $D=\num{5.9}^2-3^2$\\{\red $D=\num{34.81}-9$\\$D=\num{25.81}$}
        \item $E=\num{4.6}^2+\num{8.5}^2$\\{\red $E=\num{21.16}+\num{72.25}$\\$E=\num{93.41}$}
        \item $F=\num{4.6}^2-\num{8.5}^2$\\{\red $E=\num{21.16}-\num{72.25}$\\$E=\num{-51.09}$}
        \item $G=4^2+5^2$ puis $H=9^2$.\\
        {\red
            $G=16+25$\\$G=41$\\$H=81$
        }
        \item Établir une affirmation grâce à la question précédente.
        {\red La question précédente fournit un contre-exemple permettant d'établir que d'une façon générale, $a^2+b^2\neq c^2$}
    \end{enumerate}
