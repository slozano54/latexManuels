    %\setcounter{partie}{0} % Pour s'assurer que le compteur de \partie est à zéro dans les corrigés
    \phantom{rrr}

    \begin{minipage}{0.65\linewidth}
        \emoji{light-bulb} Sur la figure ci-contre :
        \begin{itemize}
            \item Les points $A$, $P$ et $B$ sont alignés.
            \item Les points $A$, $R$ et $C$ sont alignés.
        \end{itemize}
        \begin{enumerate}
            \item Expliquer pourquoi on peut appliquer le théorème de Thalès.

            {\color{red} Dans le triangle $ABC$ :
            \begin{itemize}
                \item $R \in [AC]$.
                \item $P \in [AB]$.
                \item Les droites $(BC)$ et $(PR)$ sont toutes le deux perpendiculaires à la même droite $(AC)$,
                elles sont donc parallèles.
            \end{itemize}
            On peut donc appliquer le théorème de Thalès.
            }
            \item Écrire alors les rapports égaux.

            {\color{red} $\dfrac{AR}{AC}=\dfrac{AP}{AB}=\dfrac{RP}{CB}$}
        \end{enumerate}
    \end{minipage}
    \begin{minipage}{0.3\linewidth}
        \hspace*{-3mm}
        \vspace*{-10mm}
        \begin{tikzpicture}[scale=0.4]
            % \draw[help lines, color=black!30, dashed] (0,0) grid (10,8);
            \coordinate[label=below:$A$] (A) at (9,6);
            \coordinate[label=above left:$R$] (R) at (6,6);
            \coordinate[label=below left:$P$] (P) at (6,4.5);
            \tkzDefPointOnLine[pos=1.5](P,R) \tkzGetPoint{R1};
            \tkzDefPointBy[homothety=center A ratio 2.5](R)        \tkzGetPoint{C};
            \tkzDefPointBy[homothety=center A ratio 2.5](P)        \tkzGetPoint{B};
            \tkzDefPointOnLine[pos=1.5](B,C) \tkzGetPoint{C1};
            \tkzLabelPoints[above left](C,B);
            \tkzDrawLine(C,A);
            \tkzDrawLine(B,A);
            \tkzDrawLine[add=0.7 and 0.7](R,P);
            \tkzDrawLine[add=0.4 and 0.4](C,B);
            \tkzMarkRightAngle[fill=gray](A,R,R1);
            \tkzMarkRightAngle[fill=gray](A,C,C1);
        \end{tikzpicture}
    \end{minipage}
