    %\setcounter{partie}{0} % Pour s'assurer que le compteur de \partie est à zéro dans les corrigés
    \phantom{rrr}

    Soit un triangle $EFG$ tel que :
    \begin{itemize}
        \item $EG=\Lg{4}$ ; $FG=\Lg{3.3}$ et $EF=\Lg{5}$.
        \item $M$ est le ploint de $[EG)$ tel que $EM=\Lg{6}$.
        \item La parallèle à $(FG)$ passant par le point $M$ coupe $[EF)$ en $N$.
    \end{itemize}
    \begin{multicols}{2}
    \begin{enumerate}
        \item Faire la figure aux instruments.

        \begin{tikzpicture}[scale=0.4]
            % \draw[help lines, color=black!30, dashed] (0,0) grid (10,8);
            \coordinate[label=below left:$E$] (E) at (1,1);
            \coordinate[label=below left:$N$] (N) at (9,1);
            \coordinate[label=right:$M$] (M) at (7,7);
            \tkzDrawLine[add = 0 and 0.3](E,M);
            \tkzDrawLine[add = 0 and 0.3](E,N);
            \tkzDefPointBy[homothety=center E ratio 0.6](M)        \tkzGetPoint{G};
            \tkzDefPointBy[homothety=center E ratio 0.6](N)        \tkzGetPoint{F};
            \tkzLabelPoints[above](G);
            \tkzLabelPoints[below right](F);
            \tkzDrawLine(M,N);
            \begin{scope}[ dim style/.append style={dashed},
                dim fence style/.style={dashed}]
                \tkzDrawSegment[dim={\(\Lg{6}\),10mm,sloped,above=1mm}](E,M);
                \tkzDrawSegment[dim={\(\Lg{4}\),5mm,sloped,above=1mm}](E,G);
                \tkzDrawSegment[dim={\(\Lg{5}\),-2mm,sloped,below=1mm}](E,F);
                \tkzDrawSegment[dim={\(\Lg{3.3}\),2mm,sloped,above=1mm}](G,F);
            \end{scope}
        \end{tikzpicture}

        \columnbreak
        \item Calculer $EN$ et $MN$.

        {\color{red}
        Dans la configuration ci-dessus :
        \begin{itemize}
            \item les deux triangles sont \textcolor{black}{$EGF$} et \textcolor{blue}{$EMN$}.
            \item $G \in [EM]$ et $F \in [EN]$.
            \item les droites $(GF)$ et $(MN)$ sont parallèles.
        \end{itemize}
        D'après le théorème de Thalès, on peut donc écrire l'égalité des trois rapports :
        $$\frac{\textcolor{black}{EG}}{\textcolor{blue}{EM}}=\frac{\textcolor{black}{EF}}{\textcolor{blue}{EN}}=\frac{\textcolor{black}{GF}}{\textcolor{blue}{MN}}\qquad\mbox{c'est à dire}\qquad
        \frac{\textcolor{black}{4}}{\textcolor{blue}{6}}=\frac{\textcolor{black}{5}}{\textcolor{blue}{EN}}=\frac{\textcolor{black}{\num{3.3}}}{\textcolor{blue}{MN}}$$
        }
    \end{enumerate}
    \end{multicols}
    \begin{multicols}{2}
        \begin{list}{}{}
        \item \phantom{rrr}

        {\color{red}
        $$\mbox{\bf Pour le calcul de $EN$}, \mbox{ on utilise} \quad \dfrac{4}{6}=\dfrac{5}{EN}$$
        $$\mbox{\bf Les produits en croix sont égaux}$$
        $$4\times EN=6\times 5$$
        $$\mbox{\bf On divise les deux membres par $4$}$$
        $$EN=\frac{6\times 5}{4}=\Lg{7.5}$$
        }
        \columnbreak
        \item \phantom{rrr}

        {\color{red}
        $$\mbox{\bf Pour le calcul de $MN$}, \mbox{ on utilise} \quad \dfrac{4}{6}=\dfrac{\num{3.3}}{MN}$$
        $$\mbox{\bf Les produits en croix sont égaux}$$
        $$4\times MN=\num{3.3}\times 6$$
        $$\mbox{\bf On divise les deux membres par $4$}$$
        $$MN=\frac{\num{3.3}\times 6}{4}=\Lg{4.95}$$
        }
        \end{list}
    \end{multicols}

