    %\setcounter{partie}{0} % Pour s'assurer que le compteur de \partie est à zéro dans les corrigés
    % \phantom{rrr}

    Sur la figure ci-dessous :
    \begin{itemize}
        \item $SABCD$ et $SIJKL$ sont deux pyramides régulières à bases carrées.
        \item $[SM]$ et $[SO]$ sont les hauteurs respectives de $SIJKL$ et $SABCD$.
    \end{itemize}
    \begin{minipage}{0.5\linewidth}
        \scalebox{0.63}{
            \begin{tikzpicture}
                % \draw[help lines, color=black!30, dashed] (0,0) grid (9,5);
                \coordinate[label=below:$A$] (A) at (0.1,0.1);
                \coordinate[label=below:$B$] (B) at (5.1,0.1);
                \coordinate[label=left:$D$] (D) at (2.1,1.5);
                \coordinate[label=right:$C$] (C) at (7.1,1.5);
                \coordinate[label=below:$O$] (O) at (3.6,0.8);
                \coordinate[label=above:$S$] (S) at (3.6,5);
                \tkzDrawSegment(A,B);
                \tkzDrawSegment(B,C);
                \tkzDrawSegment[dashed](C,D);
                \tkzDrawSegment[dashed](D,A);
                \tkzDrawSegment(A,S);
                \tkzDrawSegment(B,S);
                \tkzDrawSegment(C,S);
                \tkzDrawSegment[dashed](D,S);
                \tkzDrawSegment[dashed](D,B);
                \tkzDrawSegment[dashed](A,C);
                \tkzDrawSegment[dashed](O,S);
                \tkzDefPointBy[homothety=center S ratio 0.6](A)        \tkzGetPoint{I};
                \tkzDefPointBy[homothety=center S ratio 0.6](B)        \tkzGetPoint{J};
                \tkzDefPointBy[homothety=center S ratio 0.6](C)        \tkzGetPoint{K};
                \tkzDefPointBy[homothety=center S ratio 0.6](D)        \tkzGetPoint{L};
                \tkzDefPointBy[homothety=center S ratio 0.6](O)        \tkzGetPoint{M};
                \tkzDrawSegment(I,J);
                \tkzDrawSegment(J,K);
                \tkzDrawSegment[dashed](K,L);
                \tkzDrawSegment[dashed](L,I);
                \tkzDrawSegment[dashed](L,J);
                \tkzDrawSegment[dashed](I,K);
                \tkzLabelPoints[left](I,L);
                \tkzLabelPoints[right](K);
                \tkzLabelPoints[above right](M);
                \tkzLabelPoints[below right](J);
                \tkzMarkRightAngle[fill=gray,size=0.15](J,M,S);
                \tkzMarkRightAngle[fill=gray,size=0.15](B,O,S);
                \fill[opacity=0.2,gray] (I)--(L)--(S)--(I);
                \fill[opacity=0.2,gray] (L)--(S)--(K)--(L);
                \fill[opacity=0.2,gray] (I)--(L)--(K)--(J)--(J)--(I);
            \end{tikzpicture}
        }
    \end{minipage}
    \begin{minipage}{0.45\linewidth}
        \begin{itemize}
            \item $M \in [SO]$.
            \item $SM=\Lg{1.5}$.
            \item $SO=\Lg{4.5}$.
            \item $DB=\Lg{5}$.
        \end{itemize}
    \end{minipage}

    \begin{enumerate}
        \item Justifier la position relative de $(MJ)$ et $(OB)$.

        {\color{red} Les droites $(MJ)$ et $(OB)$ sont perpendiculaires à la même droite, $(OS)$.

        Elles sont donc parallèles.
        }
        \item Justifier le calcul la valeur exacte de $MJ$.

        {\color{red} Dans le triangle $SOB$ :
        \begin{itemize}
            \item $M \in [SO]$ et $J \in [SB]$.
            \item Les droites $(MJ)$ et $(OB)$ sont parallèles.
        \end{itemize}
        On peut donc appliquer le théorème de Thalès et écrire : $\dfrac{SM}{SO}=\dfrac{SJ}{SB}=\dfrac{MJ}{OB}$.

        \smallskip
        De plus $ABCD$ est un carré, ses diagonales se coupent donc en leur milieu $O$ donc $OB=DB\div 2 = 5\div 2 = \Lg{2.5}$.

        On peut alors écrire l'égalité de Thalès avec les valeurs numériques : $\dfrac{\num{1.5}}{\num{4.5}}=\dfrac{SJ}{SB}=\dfrac{MJ}{\num{2.5}}$.

        \textbf{On utilise l'égalité} $\dfrac{\num{1.5}}{\num{4.5}}=\dfrac{MJ}{\num{2.5}}$

        \textbf{Les produits en croix sont égaux} donc $\num{4.5}\times MJ=\num{1.5}\times \num{2.5}$

        \textbf{On divise par $\num{4.5}$}

        \smallskip
        d'où $MJ=\dfrac{\num{1.5}\times \num{2.5}}{\num{4.5}}=\dfrac{\num{3.75}}{\num{4.5}}=\dfrac{5}{6}$

        $MJ=\dfrac{5}{6} \Lg{}$
        }
    \end{enumerate}
