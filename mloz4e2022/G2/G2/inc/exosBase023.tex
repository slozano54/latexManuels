\begin{exercice*}
    Le triangle $I'J'K'$ est un agrandissement du triangle $IJK$.

    \hspace*{-10mm}
    \begin{tikzpicture}[scale = 0.55]
            % \draw[help lines, color=black!70, dashed] (0,0) grid (16,6);                            
            % points
            \coordinate (J) at (6,2);
            \coordinate (I) at (2,2);
            \coordinate (K) at (1,5);
            \coordinate (J') at (15,1);
            \coordinate (I') at (9,1);
            \coordinate (K') at (7.5,5.5);
            % triangle IJK
            \tkzDrawSegment[style=red, dim={$\Lg{3}$,-10pt,midway,font=\normalsize,sloped}](K,I)        
            \tkzDrawSegment[style=red, dim={$\Lg{2}$,-10pt,midway,font=\normalsize,sloped}](I,J)        
            \draw[ultra thick] (I)--(J)--(K)--(I);
            \tkzMarkAngles(K,J,I);            
            \tkzLabelAngle[pos=1.6](K,J,I){\ang{43}}
            \tkzLabelPoints[above](K);
            \tkzLabelPoints[below left](I);
            \tkzLabelPoints[below right](J);
            % triangle I'J'K'
            \tkzDrawSegment[style=red, dim={$\Lg{4.5}$,-10pt,midway,font=\normalsize,sloped}](K',I')        
            \tkzDrawSegment[style=red, dim={$\Lg{6}$,10pt,midway,font=\normalsize,sloped}](K',J')        
            \draw[ultra thick] (I')--(J')--(K')--(I');
            \tkzLabelPoints[above](K');
            \tkzLabelPoints[below](I');
            \tkzLabelPoints[below right](J');
    \end{tikzpicture}
    \begin{enumerate}
        \item Déterminer $k$, le rapport d'agrandissement, sous forme fractionnaire puis décimale.
        \item Calculer $I'J'$.
        \item Déterminer la mesure de $\widehat{I'J'K'}$.
    \end{enumerate}
\end{exercice*}
\begin{corrige}
    %\setcounter{partie}{0} % Pour s'assurer que le compteur de \partie est à zéro dans les corrigés
    % \phantom{rrr}  

    Le triangle $I'J'K'$ est un agrandissement du triangle $IJK$.

    \begin{tikzpicture}[scale = 0.6]
            % \draw[help lines, color=black!70, dashed] (0,0) grid (16,6);                            
            % points
            \coordinate (J) at (6,2);
            \coordinate (I) at (2,2);
            \coordinate (K) at (1,5);
            \coordinate (J') at (15,1);
            \coordinate (I') at (9,1);
            \coordinate (K') at (7.5,5.5);
            % triangle IJK
            \tkzDrawSegment[style=red, dim={$\Lg{3}$,-10pt,midway,font=\normalsize,sloped}](K,I)        
            \tkzDrawSegment[style=red, dim={$\Lg{2}$,-10pt,midway,font=\normalsize,sloped}](I,J)        
            \draw[ultra thick] (I)--(J)--(K)--(I);
            \tkzMarkAngles(K,J,I);            
            \tkzLabelAngle[pos=1.6](K,J,I){\ang{43}}
            \tkzLabelPoints[above](K);
            \tkzLabelPoints[below left](I);
            \tkzLabelPoints[below right](J);
            % triangle I'J'K'
            \tkzDrawSegment[style=red, dim={$\Lg{4.5}$,-10pt,midway,font=\normalsize,sloped}](K',I')        
            \tkzDrawSegment[style=red, dim={$\Lg{6}$,10pt,midway,font=\normalsize,sloped}](K',J')        
            \draw[ultra thick] (I')--(J')--(K')--(I');
            \tkzLabelPoints[above](K');
            \tkzLabelPoints[below](I');
            \tkzLabelPoints[below right](J');
    \end{tikzpicture}

    \begin{enumerate}
        \item Déterminer $k$, le rapport d'agrandissement, sous forme fractionnaire puis décimale.
        
        {\color{red}On sait que $I'J'K'$ est un agrandissement de $IJK$, donc les longueurs de leurs côtés sont proportionnelles.
        $k=\dfrac{\num{4.5}}{3}=\dfrac{3}{2}=\num{1.5}$.
        }
        \item Calculer $I'J'$.
        
        {\color{red}$I'J'=k\times IJ=\num{1.5}\times 2=3$, $[I'J']$ mesure donc \Lg{3}}
        \item Déterminer la mesure de $\widehat{I'J'K'}$.
        
        {\color{red}L'agrandissement ou la réduction conservent les angles donc $\widehat{I'J'K'}=\widehat{IJK}=\ang{43}$.}
    \end{enumerate}
\end{corrige}