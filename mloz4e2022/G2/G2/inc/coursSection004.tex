\subsection{Agrandissement/Réduction dans le plan}

\begin{definition}
On dit qu'un objet est un \textbf{agrandissement} ou une \textbf{réduction} d'un autre objet lorsque les longueurs entre les deux objets sont proportionnelles.\par\vspace{0.25cm}
Le coefficient de proportionnalité est alors appelé \textbf{coefficient de réduction}\\
ou \textbf{coefficient d'agrandissement} suivant le cas :
\begin{itemize}
\item si le coefficient de proportionnalité entre les longueurs de deux objets est\\ \textbf{strictement supérieur à 1} alors c'est un \textbf{coefficient d'agrandissement}.
\item si le coefficient de proportionnalité entre les longueurs de deux objets est\\ \textbf{strictement inférieur à 1} alors c'est un \textbf{coefficient de réduction}.
\end{itemize}
\end{definition}

\begin{remarques}
\begin{itemize}
\item Si le coefficient de proportionnalité entre les longueurs de deux objets est 1 alors les deux objets ont les mêmes dimensions.
\item Les triangles d'une configuration de Thalès sont homothétiques.
\item Les triangles d'une configuration de Thalès sont semblables.
\end{itemize}
\end{remarques}

\begin{propriete}[\admise]
Les agrandissements et les réductions conservent les angles, donc la perpendicularité et le parallélisme!
\end{propriete}