\begin{exercice*}[Rédaction complète] 
    \begin{minipage}{0.7\linewidth}
        Sur la figure ci-contre :
        \begin{itemize}
            \item $(GI)$ et $(FD)$ sont parallèles.
            \item $EI=4$, $ED=7$ et $GI=5$.
        \end{itemize}

        \begin{enumerate}
            \item Calculer la longueur $FD$.
            \item Justifier que l'on ne peut calculer ni $EG$ ni $EF$.
        \end{enumerate}
    \end{minipage}
    \begin{minipage}{0.3\linewidth}
        \hspace*{-10mm}   
        \Thales[FigureSeule,Angle=-45]{EFDGI}{}{}{}{}{}{}
    \end{minipage}
\end{exercice*}
\begin{corrige}
    %\setcounter{partie}{0} % Pour s'assurer que le compteur de \partie est à zéro dans les corrigés
    \phantom{rrr}

    \begin{minipage}{0.6\linewidth}
        Sur la figure ci-contre :
        \begin{itemize}
            \item $(GI)$ et $(FD)$ sont parallèles.
            \item $EI=4$, $ED=7$ et $GI=5$.
        \end{itemize}
    \end{minipage}
    \begin{minipage}{0.3\linewidth}
        \scalebox{0.65}{
            \Thales[FigureSeule,Angle=-45]{EFDGI}{}{}{}{}{}{}
        }           
    \end{minipage}
    
    \begin{enumerate}
        \item Calculer la longueur $FD$.
        
        {\color{red}
        Dans la configuration ci-dessus : 
        \begin{itemize}
            \item les deux triangles sont $EGI$ et $EFD$.
            \item $G \in [EF]$ et $I \in [ED]$.
            \item les droites $(GI)$ et $(FD)$ sont parallèles.                
        \end{itemize}
        D'après le théorème de Thalès, on peut donc écrire l'égalité des trois rapports :
        $\dfrac{EG}{EF}=\dfrac{EI}{ED}=\dfrac{GI}{FD}\qquad\mbox{c'est à dire}\qquad\dfrac{EG}{EF}=\dfrac{4}{7}=\dfrac{5}{FD}$
    
            $\mbox{\bf Pour le calcul de $FD$, on utilise} \quad \dfrac{4}{7}=\dfrac{5}{FD}$

            $\mbox{\bf Les produits en croix sont égaux donc }4\times FD=7\times5$
            $\mbox{\bf On divise les deux membres par $4$ donc }FD=\dfrac{7\times5}{4}=\num{8.75}$
        }            
        \item Justifier que l'on ne peut calculer ni $EG$ ni $EF$.
        
        {\color{red} Pour caluler $EG$ ou $EF$ il faut utiliser le rapport $\dfrac{EG}{EF}$ dont on ne connait ni le 
        numérateur ni le dénominateur donc au mieux on peut établir une relation de proportionnalité entre $EG$ et $EF$.}
    \end{enumerate}
    
\end{corrige}

