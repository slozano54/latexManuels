\begin{exercice*}[Proportionnalité]
    Deux triangles ont les longueurs suivantes, rassemblées dans ce tableau.
    
    \hspace*{-10mm}
    \begin{tabular}{|>{\centering\arraybackslash\columncolor{gray!40}}m{0.29\linewidth}|*{3}{>{\centering\arraybackslash}m{0.2\linewidth}|}}        
        \hline
        \textbf{Triangle $RST$}&$RS=\num{5.4}$&$RT=\num{8.1}$&$TS=\num{10.8}$\\\hline
        \textbf{Triangle $FGH$}&$FG=\num{4.5}$&$FH=\num{6.75}$&$GH=\num{9}$\\\hline
    \end{tabular}
    
    \begin{enumerate}
        \item Justifier que c'est un tableau de proportionnalité.
        \item Préciser si le triangle $RST$ est un agrandissement ou une réduction du triangle $FGH$, ainsi que le rapport.        
    \end{enumerate}
\end{exercice*}
\begin{corrige}
%\setcounter{partie}{0} % Pour s'assurer que le compteur de \partie est à zéro dans les corrigés
% \phantom{rrr}  
Deux triangles ont les longueurs suivantes, rassemblées dans ce tableau.
    
\begin{tabular}{|>{\centering\arraybackslash}m{0.4\linewidth}|*{3}{>{\centering\arraybackslash}m{0.2\linewidth}|}}        
    \hline
    \textbf{Triangle $RST$}&$RS=\num{5.4}$&$RT=\num{8.1}$&$TS=\num{10.8}$\\\hline
    \textbf{Triangle $FGH$}&$FG=\num{4.5}$&$FH=\num{6.75}$&$GH=\num{9}$\\\hline
\end{tabular}

\begin{enumerate}
    \item Justifier que c'est un tableau de proportionnalité.
    
    {\color{red}En calculant tous les quotients, on constate que $\dfrac{RS}{FG}=\dfrac{RT}{FH}=\dfrac{TS}{GH}=\dfrac{6}{5}$.
    Tous les quotients sont égaux, c'est donc un tableau de proportionnalité.
    }
    \item Préciser si le triangle $RST$ est un agrandissement ou une réduction du triangle $FGH$, ainsi que le rapport.        
    
    {\color{red}$RST$ est un agrandissement de $FGH$ de rapport $k=\dfrac{6}{5}=\num{1.2}$}
\end{enumerate}

\end{corrige}

