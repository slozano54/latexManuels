\begin{exercice*}
    Un cube a pour arête \Lg{2}.
    
    On l'agrandit dans un rapport $3$.
    \begin{enumerate}
        \item Calculer le volume du cube initial.
        \item Calculer l'arête du cube agrandi.
        \item Calculer le volume du cube agrandi.
        \item Recopier et compléter :
        
        \smallskip
        $\dfrac{\mathcal{V}_{\text{cube agrandi}}}{\mathcal{V}_{\text{cube initial}}}=\dfrac{\makebox[1cm]{\dotfill}}{\makebox[1cm]{\dotfill}}=\left(\dfrac{\makebox[1cm]{\dotfill}}{\makebox[1cm]{\dotfill}}\right)^3=(\makebox[1cm]{\dotfill})^3$
    \end{enumerate}
\end{exercice*}
\begin{corrige}
    %\setcounter{partie}{0} % Pour s'assurer que le compteur de \partie est à zéro dans les corrigés
    \phantom{rrr} 
    
    \begin{multicols}{2}
        Un cube a pour arête \Lg{2}. On l'agrandit dans un rapport $3$.

        \begin{enumerate}
            \item Calculer le volume du cube initial.
            
            {\color{red}$\mathcal{V}_{\text{cube initial}}=2\times 2\times 2 = 8$
            
            Le volume du cube initial vaut \Vol{8}.
            }
            \item Calculer l'arête du cube agrandi.        
            
            {\color{red} $\text{arête du cube agrandi} = 3\times 2 = 6$
            
            L'arête du cube agrandi mesure \Lg{6}.
            }
            \columnbreak
            \item Calculer le volume du cube agrandi.
            
            {\color{red}$\mathcal{V}_{\text{cube agrandi}}=6\times 6\times 6 = 216$
            
            Le volume du cube initial vaut \Vol{216}.
            }
            \item Recopier et compléter :
            
            \smallskip
            $\dfrac{\mathcal{V}_{\text{cube initial}}}{\mathcal{V}_{\text{cube agrandi}}}=\dfrac{{\color{red}216}}{{\color{red}8}}=\left(\dfrac{{\color{red}6}}{{\color{red}2}}\right)^3=({\color{red}3})^3$
        \end{enumerate}
    \end{multicols}
\end{corrige}