\begin{activite}[Réciproque du théorème direct]
    
        \begin{remarque}
            L'activité peut se faire :
            \begin{itemize}
                \item En classe entière au vidéoprojecteur.
                \item En devoir à la maison.
                \item En salle informatique par deux.
                \item En salle informatique individuellement.
            \end{itemize}
        \end{remarque}
        
        \href{https://www.geogebra.org/m/nztvwr9b}{\emoji{link} Ouvrir l'activité Geogebra}

        \begin{enumerate}
            \item Déplacer le point $N$ sur le segment $[AC]$.
            \item Déplacer le point $M$ sur le segment $[AB]$ tel que $\dfrac{AM}{AB}=\dfrac{AN}{AC}$.
            \item Déplacer plusieurs fois N, et reprendre les questions 1) et 2).
            \item Formuler une conjecture sur les droites $(MN)$ et $(BC)$.
            \begin{spacing}{1.2}
            \item Fixer un point $N$ sur le segment $[AC]$ , déplacer le point $M$ sur la droite $[AB]$
            de manière à ce que $\dfrac{AM}{AB}=\dfrac{AN}{AC}$ et que la conjecture précédente soit fausse.
            \end{spacing}
            \item Énoncer une conjecture qui tient compte des questions 1) et 2).
        \end{enumerate}

        \begin{center}
            \includegraphics[scale=0.5]{\currentpath/images/reciproqueThales.png}
        \end{center}

        \textbf{Compléter la trace écrite, réciproque théorème de Thalès.}
\end{activite}