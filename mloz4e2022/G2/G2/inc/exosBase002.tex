\begin{exercice*}[Quatrième proportionnelle $\mathbb{Z}$]
    Déterminer la quatrième proportionnelle dans les égalités suivantes.
    \begin{multicols}{2}
        \begin{enumerate}
            \begin{spacing}{2}
                \item $\dfrac{-4}{2}=\dfrac{x}{-8}$
                \item $\dfrac{-9}{w}=\dfrac{-8}{5}$
            \end{spacing}
        \end{enumerate}
    \end{multicols}
    \hrefMathalea{https://coopmaths.fr/mathalea.html?ex=4L15-1,s=2,n=2,i=0&v=l} % On peut personnaliser le texte entre crochets si on veut sinon supprimer les crochets
\end{exercice*}
\begin{corrige}
    %\setcounter{partie}{0} % Pour s'assurer que le compteur de \partie est à zéro dans les corrigés
    % \phantom{rrr}
    Déterminer la quatrième proportionnelle dans les égalités suivantes.
    \begin{multicols}{2}
    \begin{enumerate}
        \begin{spacing}{1.2}
            \item 
            $\dfrac{-4}{2}=\dfrac{x}{-8}$\\
            {\bfseries \color[HTML]{f15929}Les produits en croix sont égaux.}\\
            $2\times x = -4\times -8$\\
            {\bfseries \color[HTML]{f15929}On divise les deux membres par 2}.\\
            $\dfrac{2\times x}{2}= \dfrac{-4\times -8}{2}$\\
            {\bfseries \color[HTML]{f15929}On simplifie et on calcule.}\\
            $x=16$            
            \item 
            $\dfrac{-9}{w}=\dfrac{-8}{5}$\\
            {\bfseries \color[HTML]{f15929}Les produits en croix sont égaux.}\\
            $-8\times w = 5\times -9$\\
            {\bfseries \color[HTML]{f15929}On divise les deux membres par -8}.\\
            $\dfrac{-8\times w}{-8}= \dfrac{5\times -9}{-8}$\\
            {\bfseries \color[HTML]{f15929}On simplifie et on calcule.}\\
            $w=5{,}63$       
        \end{spacing}
    \end{enumerate}
    \end{multicols}
\end{corrige}

