\begin{activite}[Théorème direct]
    \partie[Configuration emboîtée - {$M \in [AB]$}]
        \begin{minipage}{0.6\textwidth}
            \begin{enumerate}
                \item Tracer un triangle $ABC$.
                \item Placer un point $M$ sur le segment $[AB]$.
                \item Tracer la droite parallèle à la droite $(BC)$ passant par $M$,
                
                elle coupe $(AC)$ en $N$.
                \item Que dire des longeurs des côtés des triangles $AMN$ et $ABC$ ?
                \item Appliquer le théorème vu en $4$\up{e} pour justifier ce résultat.
            \end{enumerate}
        \end{minipage}
        \hspace*{1cm}
        \begin{minipage}{0.4\textwidth}
            \Thales[FigureSeule,Angle=-90]{ABCMN}{AM}{AN}{MN}{AB}{AC}{BC}
        \end{minipage}

    \partie[Configuration emboîtée - {$M \in [AB)$ ; $M \notin [AB]$}]
        \begin{minipage}{0.6\textwidth}
            \begin{enumerate}
                \item Tracer un triangle $ABC$.
                \item Placer un point $M$ sur la demi-droite $[AB)$ n'appartenant pas au segment $[AB]$.
                \item Tracer la droite parallèle à la droite $(BC)$ passant par $M$,
                
                elle coupe $(AC)$ en $N$.
                \item Se placer dans le triangle $AMN$ et démontrer que les quotients
                $\dfrac{AB}{AM}$, $\dfrac{AC}{AN}$ et $\dfrac{BC}{MN}$ sont égaux.
                \item Faire une déduction concernant les quotients $\dfrac{AM}{AB}$, $\dfrac{AN}{AC}$ et $\dfrac{MN}{BC}$. Justifier.
            \end{enumerate}
        \end{minipage}
        \hspace*{1cm}
        \begin{minipage}{0.4\textwidth}
            \Thales[FigureSeule,Angle=-90]{AMNBC}{AM}{AN}{MN}{AB}{AC}{BC}
        \end{minipage}

        \textbf{Compléter la trace écrite, théorème de Thalès.}
\end{activite}