\begin{exercice*}[Triangles en situation de Thalès]    
    \begin{minipage}{0.2\linewidth}
        \hspace*{-15mm}
        \begin{tikzpicture}[scale=0.3]
            % \draw[help lines, color=black!30, dashed] (0,0) grid (8,12);        
            \coordinate[label=below:$V$] (V) at (7,1);
            \coordinate[label=above:$U$] (U) at (1,9);
            \coordinate[label=above:$S$] (S) at (6.5,10);
            \tkzDefPointBy[homothety=center V ratio 0.6](U)	\tkzGetPoint{H};
            \tkzDefPointBy[homothety=center V ratio 0.6](S)	\tkzGetPoint{R};
            \tkzLabelPoints[left](H);
            \tkzLabelPoints[right](R);
            \tkzDrawSegment(V,U);
            \tkzDrawSegment(U,S);
            \tkzDrawSegment[dashed,ultra thick,color=red](U,S);
            \tkzDrawSegment(S,V);
            \tkzDrawSegment(H,R);
            \tkzDrawSegment[dashed,ultra thick,color=red](H,R);
        \end{tikzpicture}
    \end{minipage}    
    \hfill
    \begin{minipage}{0.8\linewidth}
        \emoji{light-bulb} Sur la figure ci-contre, les droites portant les segments en pontillés sont parallèles.
        \begin{enumerate}
            \item Nommer les triangles dont les longueurs sont proportionnelles.
            \item Écrire les rapports de proportionnalités égaux.
        \end{enumerate}
    \end{minipage}

\end{exercice*}
\begin{corrige}
    %\setcounter{partie}{0} % Pour s'assurer que le compteur de \partie est à zéro dans les corrigés
    \phantom{rrr}

    \begin{minipage}{0.3\linewidth}
        \begin{tikzpicture}[scale=0.4]
            % \draw[help lines, color=black!30, dashed] (0,0) grid (8,12);        
            \coordinate[label=below:$V$] (V) at (7,1);
            \coordinate[label=above:$U$] (U) at (1,9);
            \coordinate[label=above:$S$] (S) at (6.5,10);
            \tkzDefPointBy[homothety=center V ratio 0.6](U)	\tkzGetPoint{H};
            \tkzDefPointBy[homothety=center V ratio 0.6](S)	\tkzGetPoint{R};
            \tkzLabelPoints[left](H);
            \tkzLabelPoints[right](R);
            \tkzDrawSegment(V,U);
            \tkzDrawSegment(U,S);
            \tkzDrawSegment[dashed,ultra thick,color=red](U,S);
            \tkzDrawSegment(S,V);
            \tkzDrawSegment(H,R);
            \tkzDrawSegment[dashed,ultra thick,color=red](H,R);
        \end{tikzpicture}
    \end{minipage}
    % \hfill
    \begin{minipage}{0.6\linewidth}
        \emoji{light-bulb} Sur la figure ci-contre, les droites portant les segments en pontillés sont parallèles.

        \begin{enumerate}
            \item Nommer les triangles dont les longueurs sont proportionnelles.
            
            {\color{red} Les triangles $USV$ et $HRV$ ont leurs longeurs proportionnelles.}
            \item Écrire les rapports de proportionnalités égaux.
            
            \smallskip
            {\color{red} $\dfrac{UV}{HV} = \dfrac{SV}{RV} = \dfrac{US}{HR}$}
        \end{enumerate}
    \end{minipage}

\end{corrige}

