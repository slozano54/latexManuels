\begin{exercice*}[Deux longueurs inconnues]    
    \begin{minipage}{0.35\linewidth}
        \hspace*{-15mm}
        \begin{tikzpicture}[scale=0.4]
            % \draw[help lines, color=black!30, dashed] (0,0) grid (10,8);        
            \coordinate[label=below left:$R$] (R) at (8,1);
            \coordinate[label=above:$T$] (T) at (4,6);
            \coordinate[label=below right:$V$] (V) at (1,1);
            \tkzDrawLine(R,T);
            \tkzDrawLine(R,V);
            \tkzDrawLine(V,T);
            \tkzDrawLine[ultra thick](V,T);
            \tkzLabelSegment[above,sloped](V,T){$\Lg{5}$} 
            \tkzDefPointBy[homothety=center R ratio 0.7](T)	\tkzGetPoint{S};
            \tkzDefPointBy[homothety=center R ratio 0.7](V)	\tkzGetPoint{U};
            \tkzLabelPoints[right](S);
            \tkzLabelPoints[below right](U);
            \tkzDrawLine(U,S);
            \tkzDrawLine[ultra thick](U,S);
            \tkzLabelSegment[above,sloped](U,S){$\Lg{4}$};
            \begin{scope}[ dim style/.append style={red, dashed},
                dim fence style/.style={red, dashed}]                
                \tkzDrawSegment[dim={\(\Lg{2.5}\),-5mm,sloped,above=1mm}](R,T);
                \tkzDrawSegment[dim={\(\Lg{3}\),-5mm,sloped,above=1mm}](U,R);
            \end{scope}
        \end{tikzpicture}
    \end{minipage}    
    \hfill
    \begin{minipage}{0.65\linewidth}
        \emoji{light-bulb} Sur la figure ci-contre :
        \begin{itemize}
            \item Les points $R$, $S$ et $T$ sont alignés.
            \item Les points $R$, $U$ et $V$ sont alignés.
            \item Les droites $(SU)$ et $(TV)$ sont
            
            parallèles.
        \end{itemize}
        \begin{enumerate}
            \item Justifier le calcul de $RS$.
            \item Jusitifier le calcul de $RV$.
        \end{enumerate}

    \end{minipage}

\end{exercice*}
\begin{corrige}
    %\setcounter{partie}{0} % Pour s'assurer que le compteur de \partie est à zéro dans les corrigés
    \phantom{rrr}

    \begin{minipage}{0.3\linewidth}
        \hspace*{-15mm}
        \begin{tikzpicture}[scale=0.4]
            % \draw[help lines, color=black!30, dashed] (0,0) grid (10,8);        
            \coordinate[label=below left:$R$] (R) at (8,1);
            \coordinate[label=above:$T$] (T) at (4,6);
            \coordinate[label=below right:$V$] (V) at (1,1);
            \tkzDrawLine(R,T);
            \tkzDrawLine(R,V);
            \tkzDrawLine(V,T);
            \tkzDrawLine[ultra thick](V,T);
            \tkzLabelSegment[above,sloped](V,T){$\Lg{5}$} 
            \tkzDefPointBy[homothety=center R ratio 0.7](T)	\tkzGetPoint{S};
            \tkzDefPointBy[homothety=center R ratio 0.7](V)	\tkzGetPoint{U};
            \tkzLabelPoints[right](S);
            \tkzLabelPoints[below right](U);
            \tkzDrawLine(U,S);
            \tkzDrawLine[ultra thick](U,S);
            \tkzLabelSegment[above,sloped](U,S){$\Lg{4}$};
            \begin{scope}[ dim style/.append style={red, dashed},
                dim fence style/.style={red, dashed}]                
                \tkzDrawSegment[dim={\(\Lg{2.5}\),-5mm,sloped,above=1mm}](R,T);
                \tkzDrawSegment[dim={\(\Lg{3}\),-5mm,sloped,above=1mm}](U,R);
            \end{scope}
        \end{tikzpicture}
    \end{minipage}    
    \hfill
    \begin{minipage}{0.65\linewidth}
        \emoji{light-bulb} Sur la figure ci-contre :
        \begin{itemize}
            \item Les points $R$, $S$ et $T$ sont alignés.
            \item Les points $R$, $U$ et $V$ sont alignés.
            \item Les droites $(SU)$ et $(TV)$ sont
            
            parallèles.
        \end{itemize}
    \end{minipage}

    \medskip
    {\color{red}    
        Dans la configuration ci-dessus : 
        \begin{itemize}
            \item les deux triangles sont \textcolor{black}{$RSU$} et \textcolor{blue}{$RTV$}.
            \item $S \in [RT]$ et $U \in [RV]$.
            \item les droites $(SU)$ et $(TV)$ sont parallèles.                
        \end{itemize}
        D'après le théorème de Thalès, on peut donc écrire l'égalité des trois rapports :
        $$\frac{\textcolor{black}{RS}}{\textcolor{blue}{RT}}=\frac{\textcolor{black}{RU}}{\textcolor{blue}{RV}}=\frac{\textcolor{black}{SU}}{\textcolor{blue}{TV}}\qquad\mbox{c'est à dire}\qquad
        \frac{\textcolor{black}{RS}}{\textcolor{blue}{\num{2.5}}}=\frac{\textcolor{black}{3}}{\textcolor{blue}{RV}}=\frac{\textcolor{black}{4}}{\textcolor{blue}{5}}$$
    }
    \begin{multicols}{2}
        \begin{enumerate}
            \item Justifier le calcul de $RS$.
            
            {\color{red}    
            $$\mbox{\bf Pour le calcul de $RS$}, \mbox{ on utilise} \quad \dfrac{RS}{\num{2.5}}=\dfrac{4}{5}$$
            $$\mbox{\bf Les produits en croix sont égaux}$$
            $$5\times RS=\num{2.5}\times4$$
            $$\mbox{\bf On divise les deux membres par $5$}$$
            $$RS=\frac{\num{2.5}\times4}{5}=\Lg{2}$$
            }
            \columnbreak
            \item Jusitifier le calcul de $RV$.
            
            {\color{red}    
            $$\mbox{\bf Pour le calcul de $RV$}, \mbox{ on utilise} \quad \dfrac{3}{RV}=\dfrac{4}{5}$$
            $$\mbox{\bf Les produits en croix sont égaux}$$
            $$4\times RV=5\times3$$
            $$\mbox{\bf On divise les deux membres par $4$}$$
            $$RV=\frac{5\times3}{4}=\Lg{3.75}$$
            }
        \end{enumerate}   
    \end{multicols}

\end{corrige}

