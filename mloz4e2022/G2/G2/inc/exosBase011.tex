\begin{exercice*}[Monumental]    
    Pour mesurer la hauteur de la tour Eiffel, on utilise un bâton. On relève les mesures suivantes :
    \begin{itemize}
        \item L'ombre du bâton, $OM$, mesure \Lg[m]{1.10}.
        \item L'ombre de la tour Eiffel, $OH$, mesure \Lg[m]{166}.
        \item Le bâton, $[MN]$, mesure \Lg[m]{2}.
    \end{itemize}

    On suppose que :
    \begin{itemize}
        \item $O$, $M$, $H$ sont alignés.
        \item $O$, $N$, $S$ sont alignés.
        \item Le bâton et la tour Eiffel sont perpendiculaires au sol.
    \end{itemize}

    \medskip
    Calculer la hauteur de la tour Eiffel arrondie à l'unité.

    \scalebox{0.95}{
    \begin{tikzpicture}
        % \draw[help lines, color=black!30, dashed] (0,0) grid (9,5);                
        \node[anchor=south west, inner sep=0] (tourEiffel) at (7.5,0) {\includegraphics[scale=0.3]{\currentpath/images/tourEiffel.png}};
        \coordinate[label=above:$S$] (S) at (8.5,4.7);
        \tkzDrawPoint[shape=cross out](S);
        \coordinate[label=below:$H$] (H) at (8.5,0.08);
        \tkzDrawPoint[shape=cross out](H);
        \tkzDrawSegment[ultra thick](H,S);
        \coordinate[label=left:$O$] (O) at (0.5,0.08);
        \tkzDrawSegment[yellow, ultra thick](O,S);
        \tkzLabelSegment[sloped,above](O,S){Rayons du Soleil};        
        \tkzDrawSegment[ultra thick](O,H);
        \tkzMarkRightAngle[fill=gray,size=0.2](S,H,O);
        \tkzDefPointBy[homothety=center O ratio 0.1](S)	\tkzGetPoint{N};            
        \tkzDefPointBy[homothety=center O ratio 0.1](H)	\tkzGetPoint{M};
        \tkzDrawSegment[brown,ultra thick](M,N);
        \tkzMarkRightAngle[fill=gray,size=0.2](H,M,N);
        \tkzLabelPoints(M);
        \tkzLabelPoints[above](N);
    \end{tikzpicture}    
    }

    % https://tex.stackexchange.com/questions/9559/drawing-on-an-image-with-tikz
\end{exercice*}
\begin{corrige}
    %\setcounter{partie}{0} % Pour s'assurer que le compteur de \partie est à zéro dans les corrigés
    \phantom{rrr}
    
    \begin{minipage}{0.4\linewidth}
        \hspace*{-20mm}
        \scalebox{0.8}{
        \begin{tikzpicture}
            % \draw[help lines, color=black!30, dashed] (0,0) grid (9,5);                
            \node[anchor=south west, inner sep=0] (tourEiffel) at (7.5,0) {\includegraphics[scale=0.3]{\currentpath/images/tourEiffel.png}};
            \coordinate[label=above:$S$] (S) at (8.5,4.7);
            \tkzDrawPoint[shape=cross out](S);
            \coordinate[label=below:$H$] (H) at (8.5,0.08);
            \tkzDrawPoint[shape=cross out](H);
            \tkzDrawSegment[ultra thick](H,S);
            \coordinate[label=left:$O$] (O) at (0.5,0.08);
            \tkzDrawSegment[yellow, ultra thick](O,S);
            \tkzLabelSegment[sloped,above](O,S){Rayons du Soleil};        
            \tkzDrawSegment[ultra thick](O,H);
            \tkzMarkRightAngle[fill=gray,size=0.2](S,H,O);
            \tkzDefPointBy[homothety=center O ratio 0.1](S)	\tkzGetPoint{N};            
            \tkzDefPointBy[homothety=center O ratio 0.1](H)	\tkzGetPoint{M};
            \tkzDrawSegment[brown,ultra thick](M,N);
            \tkzMarkRightAngle[fill=gray,size=0.2](H,M,N);
            \tkzLabelPoints(M);
            \tkzLabelPoints[above](N);
        \end{tikzpicture}    
        }
    \end{minipage}
    \begin{minipage}{0.55\linewidth}
        Pour mesurer la hauteur de la tour Eiffel, on utilise un bâton. On relève les mesures suivantes :
        \begin{itemize}
            \item L'ombre du bâton, $OM$, mesure \Lg[m]{1.10}.
            \item L'ombre de la tour Eiffel, $OH$, mesure \Lg[m]{166}.
            \item Le bâton, $[MN]$, mesure \Lg[m]{2}.
        \end{itemize}

        \medskip
        On suppose que :
        \begin{itemize}
            \item $O$, $M$, $H$ sont alignés.
            \item $O$, $N$, $S$ sont alignés.
            \item Le bâton et la tour Eiffel sont perpendiculaires au sol.
        \end{itemize}

        \medskip
        Calculer la hauteur de la tour Eiffel arrondie à l'unité.
    \end{minipage}

    \medskip
    {\color{red}
    Dans la configuration ci-dessus : 
    \begin{itemize}
        \item les deux triangles sont $OMN$ et $OHS$.
        \item $M \in [OH]$ et $N \in [OS]$.
        \item les droites $(MN)$ et $(SH)$ sont toutes les deux perpendicualires à la même droite $(OH)$
        
        donc elles sont parallèles.                
    \end{itemize}
    D'après le théorème de Thalès, on peut donc écrire l'égalité des trois rapports :
    $$\frac{ON}{OS}=\frac{OM}{OH}=\frac{MN}{HS}\qquad\mbox{c'est à dire}\qquad\frac{ON}{OS}=\frac{\num{1.10}}{166}=\frac{2}{HS}$$
    
        $$\mbox{on utilise} \quad \dfrac{\num{1.10}}{166}=\dfrac{2}{HS}$$
        $$\mbox{\bf Les produits en croix sont égaux}$$
        $$\num{1.10}\times HS=166\times 2$$
        $$\mbox{\bf On divise les deux membres par \num{1.10}}$$                
        $$HS=\frac{166\times 2}{\num{1.10}}\simeq\Lg[m]{302}$$ 
        
        \textbf{La hauteur de la tour Eiffel est d'environ \Lg[m]{302}.}
    }
\end{corrige}

