\section{Quotients égaux}
\begin{remarque}
    La notation fractionnaire sert notamment à écrire le quotient des divisions "interminables" par exemple le quotient de $5\div 3$
    ne peut s'écrire sous forme décimale, on pourra donc l'écrire $\dfrac{5}{3}$.
\end{remarque}

\begin{definition}
    On note $\dfrac{a}{b}$, avec $b\neq 0$, le nombre qui, lorsqu'il est multiplié par $b$, donne a.

    C'est à dire $\dfrac{a}{b} \times b=a$
\end{definition}

\begin{exemples*1}
    \phantom{rrr}

    \medskip
    $\dfrac{{\color{red}7}}{{\color{blue}11}} \times {\color{blue}11}={\color{red}7}$
    \hfill
    $\dfrac{{\color{red}5}}{{\color{blue}9}} \times {\color{blue}9}={\color{red}5}$
    \hfill
    $\dfrac{{\color{red}11}}{{\color{blue}7}} \times {\color{blue}7}={\color{red}11}$
    \hfill
    $\dfrac{{\color{red}2}}{{\color{blue}5}} \times {\color{blue}5}={\color{red}2}$
    \hfill
    $\dfrac{{\color{red}a}}{{\color{blue}b}} \times {\color{blue}b}={\color{red}a}$, $b\neq 0$
\end{exemples*1}

\begin{exemple*1}
    Justifier, à l'aide de la définition, que $\dfrac{4}{7}=\dfrac{12}{21}$.
    \correction
    $\dfrac{4}{7}\times 21=\dfrac{4}{7}\times 7\times 3$ or par définition, $\dfrac{4}{7}\times 7=4$ donc $\dfrac{4}{7}\times 21=4\times 3 = 12$.

    $\dfrac{4}{7}$ est donc le nombre qui multiplié par $21$ donne $12$ or ce nombre est par définition $\dfrac{12}{21}$.

    Conslusion : $\dfrac{4}{7}=\dfrac{12}{21}$.
\end{exemple*1}

\begin{propriete}[\admise]
    Le quotient de deux nombres relatifs ne change pas lorsque l'on multiplie (ou on divise) ces deux nombres par un même nombre
    relatif non nul (différent de 0).
    $$\frac{a}{b}=\frac{a\times c}{b\times c}\qquad ou \qquad \frac{a}{b}=\frac{a\div
    c}{b\div c}\qquad avec \quad  (b\not=0; c\not=0)$$
\end{propriete}

\begin{exemples*1}
    \begin{multicols}{2}
        \begin{list}{}{}
            \item $A=\dfrac{0,3}{-20}$
            \item $A=\dfrac{0,3\times10}{-20\times10}$
            \item $A=\dfrac3{-200}$
            \item $A=-\dfrac3{200}$
            \columnbreak
            \item $B=\dfrac{-18}{12}$
            \item $B=\dfrac{(-3)\times6}{2\times6}$
            \item $B=\dfrac{-3}2$
            \item $B=-\dfrac32$ (Simplification)
        \end{list}
    \end{multicols}
\end{exemples*1}
