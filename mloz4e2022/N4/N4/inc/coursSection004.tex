\section{Division de deux fractions}
\begin{definition}[Inverse d'un nombre relatif différent de 0]
    \textbf{L'inverse} d'un nombre relatif $x$ (avec $x\not=0$) est le quotient de 1 par $x$; on le note $\dfrac{1}{x}$.
    \par C'est le nombre qui (lorsqu'il est multiplié par $x$ donne 1 : $x\times\dfrac{1}{x}=1$.
\end{definition}

\begin{exemple*1}
    $(-2)\times \dfrac{1}{-2}=1$; $(-2)$ est l'inverse de $\dfrac{1}{-2 }$ tout comme $\dfrac{1}{-2 }$ est l'inverse de $(-2)$.
\end{exemple*1}

\begin{minipage}{0.7\linewidth}
    \begin{propriete}[\admise]
        Diviser par un nombre relatif différent de 0, c'est multiplier par l'inverse de ce nombre.
        $$\frac{a}{b}=a\times\frac{1}{b}\quad(b\not=0)$$
    \end{propriete}
\end{minipage}
\begin{minipage}{0.4\linewidth}
    \begin{exemple*1}
        \phantom{rrr}
        
        \vspace*{5mm}
        $A=\dfrac{-2}{0,25}$\\$A=-2\times\dfrac{1}{0,25}$\\$A=-2\times4$\\$A=-8$
    \end{exemple*1}
\end{minipage}

\begin{remarque}
    L'inverse de la fraction $\dfrac{c}{d}$ est la fraction $\dfrac{d}{c}$ (avec $c\not=0;\,d\not=0$).
\end{remarque}

\begin{minipage}{0.6\linewidth}
    \begin{propriete}[\admise]
        Diviser par une fraction $\dfrac{c}{d}$ ($c\not=0;\,d\not=0$), c'est multiplier par l'inverse de cette fraction.
        $$\frac{a}{b}\div\frac{c}{d}=\frac{a}{b}\times\frac{d}{c}$$
    \end{propriete}
\end{minipage}
\begin{minipage}{0.5\linewidth}
    \begin{exemples*1}
        \begin{multicols}{2}
            \begin{list}{}{}
                \item $A=\dfrac34\div\dfrac75$
                \item $A=\dfrac34\times\dfrac57$
                \item $A=\dfrac{3\times5}{4\times7}$
                \item $A=\dfrac{15}{28}$
                \columnbreak
                \item $B=\dfrac43\div9$
                \item $B=\dfrac43\div\dfrac91$
                \item $B=\dfrac43\times\dfrac19$
                \item $B=\dfrac4{27}$
            \end{list}
        \end{multicols}
    \end{exemples*1}
\end{minipage}

