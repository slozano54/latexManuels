\section{Division de deux fractions}
% \subsection{Inverse d'un nombre relatif diff\'erent de 0}
% \definNum{\par 
% \textbf{L'inverse} d'un nombre relatif $x$ (avec $x\not=0$) est
% le quotient de 1 par $x$; on le note $\dfrac{1}{x}$.
% \par C'est le nombre qui (lorsqu'il est multipli\'e par $x$ donne 1 : $x\times\frac{1}{x}=1$
% }

% \Exemples[Exemple]{}{
% $(-2)\times \dfrac{1}{-2}=1$; $(-2)$ est l'inverse de $\dfrac{1}{-2 }$ tout comme $\dfrac{1}{-2 }$ est l'inverse de $(-2)$.
% }


% \proprNumBis{(admise)}{
% Diviser par un nombre relatif diff\'erent de 0,
% c'est multiplier par l'inverse de ce nombre.
% $$\frac{a}{b}=a\times\frac{1}{b}\quad(b\not=0)$$
% }

% \Exemples[Exemple]{}{$\dfrac{-2}{0,25}=-2\times\dfrac{1}{0,25}=-2\times4=-8$.}

% \Remarques[Remarque]{
% L'inverse de la fraction $\dfrac{c}{d}$ est la fraction $\dfrac{d}{c}$ (avec $c\not=0;\,d\not=0$).
% }

% \proprNumBis{(admise)}{
% Diviser par une fraction $\dfrac{c}{d}$
% ($c\not=0;\,d\not=0$), c'est multiplier par l'inverse de cette
% fraction.
% $$\frac{a}{b}\div\frac{c}{d}=\frac{a}{b}\times\frac{d}{c}$$
% }

% \Exemples{}{
% $\dfrac34\div\dfrac75=\dfrac34\times\dfrac57=\dfrac{3\times5}{4\times7}=\dfrac{15}{28}$\hfill$\dfrac43\div9=\dfrac43\div\dfrac91=\dfrac43\times\dfrac19=\dfrac4{27}$.
% }

