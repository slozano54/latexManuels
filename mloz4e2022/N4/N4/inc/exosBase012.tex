\begin{exercice*}
    Traduire chaque phrase par une fraction.
    \begin{enumerate}
        \item L'inverse du quart de l'opposé de $5$.
        \item L'opposé du tiers de l'inverse de $5$.
        \item L'opposé de l'inverse de $\dfrac{13}{15}$.
        \item L'inverse du quart de l'opposé de $-\dfrac{12}{10}$.
    \end{enumerate}
\end{exercice*}
\begin{corrige}
    %\setcounter{partie}{0} % Pour s'assurer que le compteur de \partie est à zéro dans les corrigés
    % \phantom{rrr}    
    Traduire chaque phrase par une fraction.

    \begin{enumerate}
        \item L'inverse du quart de l'opposé de $5$.\\
        {\red L'opposé de $5$ : $-5$.\\Le quart de l'opposé de $5$ : $\dfrac{-5}{4}$.\\Donc $\dfrac{4}{-5}$}
        \item L'opposé du tiers de l'inverse de $5$.\\
        {\red L'inverse de $5$ : $\dfrac{1}{5}$.\\Le tiers de l'inverse de $5$ : $\dfrac{\dfrac{1}{5}}{3}=\dfrac{1}{15}$.\\Donc $-\dfrac{1}{15}$}
        \item L'opposé de l'inverse de $\dfrac{13}{15}$.\\
        {\red L'inverse de $\dfrac{13}{15}$ : $\dfrac{15}{13}$.\\Donc $-\dfrac{15}{13}$}
        \item L'inverse du quart de l'opposé de $-\dfrac{12}{10}$.\\
        {\red L'opposé de $-\dfrac{12}{10}$ : $\dfrac{12}{10}$.\\Le quart de l'opposé de $-\dfrac{12}{10}$ : $\dfrac{\dfrac{12}{10}}{4}=\dfrac{3}{10}$.\\Donc $\dfrac{10}{3}$.}
    \end{enumerate}
    \vspace*{-7mm}
\end{corrige}

