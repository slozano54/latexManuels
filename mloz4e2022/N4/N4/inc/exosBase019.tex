\begin{exercice*}
    Au collège du Lagon, \num{180} élèves ont été présents aux épreuves du brevet des collèges.
    \begin{enumerate}
        \item Les trois quarts ont été orientés en seconde. Détermniner le nombre d'entre eux que cela réprésente.
        \item Parmi ces derniers, \num{80} \% d'entre eux sont reçus à l'examen. Déterminer le nombre d'élèves admis en seconde ayant échoué à l'examen.
    \end{enumerate}
\end{exercice*}
\begin{corrige}
    %\setcounter{partie}{0} % Pour s'assurer que le compteur de \partie est à zéro dans les corrigés
    % \phantom{rrr}    
    Au collège du Lagon, \num{180} élèves ont été présents aux épreuves du brevet des collèges.
    
    \begin{enumerate}
        \item Les trois quarts ont été orientés en seconde. Détermniner le nombre d'entre eux que cela réprésente.\\
        {\red $\dfrac{3}{4}\times 180 = 135$ donc 135 élèves sont orientés en seconde.}        
        \item Parmi ces derniers, \num{80} \% d'entre eux sont reçus à l'examen. Déterminer le nombre d'élèves admis en seconde ayant échoué à l'examen.\\
        {\red Si \num{100} \% sont reçus, c'est que \num{20} \% ont échoué, soit $\dfrac{20}{100}\times 135 = 27$, donc 27 élèves admis en seconde ont échoué au brevet.}
    \end{enumerate}
\end{corrige}

