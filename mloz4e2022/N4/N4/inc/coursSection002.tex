\section{Multiplication des écritures fractionnaires.}
\begin{propriete}[\admise]
    Si on veut multiplier deux écritures fractionnaires alors il suffit de multiplier les numérateurs entre eux et les dénominateurs entre eux.

    \bigskip
    {\bfseries Plus symboliquement :}\\
    Si $a, b, c, d$ sont 4 nombres, $b, d$ étant non nuls. alors $\dfrac{a}{b}\times \dfrac{c}{d} = \dfrac{a\times c}{b\times d}$.
\end{propriete}


\begin{exemples*1}
    \begin{multicols}{3}
        \begin{spacing}{1.3}
            \begin{list}{}{}                
                \item $A=\dfrac{10}{-3}\times \dfrac{7}{11}$
                \item $A=\dfrac{10\times 7}{3\times 11}$
                \item $A= \psshadowbox{\dfrac{70}{-33}}$
                \columnbreak
                \item $B=5\times \dfrac{4}{7}$
                \item $B = \dfrac{5}{1}\times \dfrac{4}{7}$
                \item $B = \dfrac{5\times 4}{1\times 7}$
                \item $B = \psshadowbox{\dfrac{20}{7}}$
                \columnbreak
                \item $C=\dfrac{4}{-3}\times \dfrac{-9}{11}$
                \item $C = \dfrac{4\times (-9)}{(-3)\times 11}$
                \item $C = \dfrac{4\times 3\times \textcolor{red}{(-3)}}{11\times \textcolor{red}{(-3)}}$
                \item $C = \dfrac{4\times 3}{11}$
                \item $C = \psshadowbox{\dfrac{12}{11}}$
            \end{list}
        \end{spacing}
    \end{multicols}
\end{exemples*1}

