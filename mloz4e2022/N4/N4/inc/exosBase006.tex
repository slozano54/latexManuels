\begin{exercice*}[Niveau BOSS]
    Calculer astucieusement les nombres suivants. 

    \hspace*{-10mm}
    \begin{minipage}{\linewidth}   
        \begin{list}{}{}
            \begin{spacing}{1.5}
                \item \mbox{$A=\left(1-\dfrac15\right)\times\left(1-\dfrac25\right)\left(1-\dfrac35\right)\left(1-\dfrac45\right)\left(1-\dfrac55\right)\left(1-\dfrac65\right)$}
                \item $B=\left(2-\dfrac{1+1}{2}\right)\left(2-\dfrac{1+2}{3}\right)\left(2-\dfrac{1+3}{4}\right)\left(\dfrac{1+4}{5}-2\right)\left(\dfrac{5}{5}-2\right)$
            \end{spacing}
        \end{list}    
    \end{minipage}
\end{exercice*}
\begin{corrige}
    %\setcounter{partie}{0} % Pour s'assurer que le compteur de \partie est à zéro dans les corrigés
    % \phantom{rrr}    
    Calculer astucieusement les nombres suivants.
    \hspace*{-7mm}
    \begin{minipage}{\linewidth}   
        \begin{spacing}{2}
            \begin{itemize}
                \def\item{}
                \item $A=\left(1-\dfrac15\right)\times\left(1-\dfrac25\right)\left(1-\dfrac35\right)\left(1-\dfrac45\right)\left(1-\dfrac55\right)\left(1-\dfrac65\right)$\\
                {\red $\left(1-\dfrac{5}{5}\right)=0$ donc ce produit est nul.\\$A=0$}\\
                \item $B=\left(2-\dfrac{1+1}{2}\right)\left(2-\dfrac{1+2}{3}\right)\left(2-\dfrac{1+3}{4}\right)\left(\dfrac{1+4}{5}-2\right)\left(\dfrac{5}{5}-2\right)$\\
                {\red $B=(2-1)(2-1)(2-1)(1-2)(1-2)$\\$B=1\times 1\times 1\times (-1)\times (-1)$\\$B=1$}
            \end{itemize}
        \end{spacing}
    \end{minipage}
\end{corrige}
