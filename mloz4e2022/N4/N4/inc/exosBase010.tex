\begin{exercice*}
    Compléter, si possible, le tableau suivant.
    
    \smallskip
    {\renewcommand{\arraystretch}{1.4}
    \begin{tabular}{|*{3}{>{\hsize=0.5\hsize\centering\arraybackslash}m{0.5\linewidth}|}}%
        \hline
        \rowcolor{gray!20} $x$ & Inverse de $x$ & Opposé de $x$ \\
        \hline
        $-7$&&\\[2mm]
        \hline
        $0$&&\\[2mm]
        \hline
        $\dfrac{1}{3}$&&\\[2mm]
        \hline
        $-\dfrac{5}{2}$&&\\[2mm]
        \hline
    \end{tabular}
    }
\end{exercice*}
\begin{corrige}
    %\setcounter{partie}{0} % Pour s'assurer que le compteur de \partie est à zéro dans les corrigés
    % \phantom{rrr}    
    Compléter, si possible, le tableau suivant.
    
    \smallskip
    {\renewcommand{\arraystretch}{1.5} 
    \begin{tabular}{|*{3}{>{\hsize=0.5\hsize\centering\arraybackslash}m{0.5\linewidth}|}}%
        \hline
        \rowcolor{gray!20} $x$ & Inverse de $x$ & Opposé de $x$ \\
        \hline
        $-7$&{\red $-\dfrac{1}{7}$}&{\red $7$}\\[2mm]
        \hline
        $0$&{\red N'existe pas}&{\red $0$}\\[2mm]
        \hline
        $\dfrac{1}{3}$&{\red $3$}&{\red $-\dfrac{1}{3}$}\\[2mm]
        \hline
        $-\dfrac{5}{2}$&{\red $-\dfrac{2}{5}$}&{\red $\dfrac{5}{2}$}\\[2mm]
        \hline
    \end{tabular}
    }
\end{corrige}

