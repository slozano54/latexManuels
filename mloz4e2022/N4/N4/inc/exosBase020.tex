\begin{exercice*}
    Entre 1890 et 1990, la population d'un village a triplé. Puis entre 1990 et 2010, elle a perdu un tiers de ses habitants.
    \begin{enumerate}
        \item Déterminer si la population a augmenté ou diminué entre 1890 et 2010.
        \item Calculer alors dans quelle proportion.
    \end{enumerate}
\end{exercice*}
\begin{corrige}
    %\setcounter{partie}{0} % Pour s'assurer que le compteur de \partie est à zéro dans les corrigés
    % \phantom{rrr}    
    Entre 1890 et 1990, la population d'un village a triplé. Puis entre 1990 et 2010, elle a perdu un tiers de ses habitants.
    
    \begin{enumerate}
        \item Déterminer si la population a augmenté ou diminué entre 1890 et 2010.\\
        {\red 
        \begin{itemize}
            \item Entre 1890 et 1990, la population est multipliée par 3.
            \item Entre 1990 et 2010, la population diminue d'un tiers donc il en reste les deux tiers, elle est multipliée par $\dfrac{2}{3}$.
        \end{itemize}
        Donc entre 1890 et 2010, elle est multipliée par $3\times\dfrac{2}{3}$, c'est à dire par $2$. On en déduit que la population augmente.
        }        
        \item Calculer alors dans quelle proportion.\\
        {\red Entre 1890 et 2010, la population a été multipliée par $2$, donc elle a doublé.}
    \end{enumerate}
\end{corrige}

