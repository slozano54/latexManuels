\begin{exercice*}[Trois triangles]
    On considère cette figure.

    \begin{center}
        \begin{Geometrie}
            pair A,B,D,E,F,G,P;
            A=u*(1,1);
            D-A=u*(4,3);
            B=0.7[D,rotation(A,D,90)];            
            G=.3[A,B];
            P=.7[A,D];
            F=projection(G,A,D);
            E=projection(P,A,B);
            trace polygone(A,B,D);
            trace segment(F,G);
            trace segment(P,E);
            trace codeperp(A,F,G,5);
            trace codeperp(B,E,P,5);
            trace codeperp(A,D,B,5);
            label.lft(TEX("A"),A);
            label.top(TEX("F"),F);
            label.bot(TEX("G"),G);
            label.top(TEX("P"),P);
            label.top(TEX("D"),D);
            label.bot(TEX("E"),E);
            label.rt(TEX("B"),B);
        \end{Geometrie}
    \end{center}
    \begin{enumerate}
        \item Écrire le cosinus de l'angle $\widehat{DAB}$ de trois façons différentes, en précisant le triangle utilisé.
        \item Faire une remarque justifiée sur ces trois rapports.
    \end{enumerate}
\end{exercice*}
\begin{corrige}
    %\setcounter{partie}{0} % Pour s'assurer que le compteur de \partie est à zéro dans les corrigés
    %\phantom{rrr}    
    On considère cette figure.

    \begin{center}
        \begin{Geometrie}
            pair A,B,D,E,F,G,P;
            A=u*(1,1);
            D-A=u*(4,3);
            B=0.7[D,rotation(A,D,90)];            
            G=.3[A,B];
            P=.7[A,D];
            F=projection(G,A,D);
            E=projection(P,A,B);
            trace polygone(A,B,D);
            trace segment(F,G);
            trace segment(P,E);
            trace codeperp(A,F,G,5);
            trace codeperp(B,E,P,5);
            trace codeperp(A,D,B,5);
            label.lft(TEX("A"),A);
            label.top(TEX("F"),F);
            label.bot(TEX("G"),G);
            label.top(TEX("P"),P);
            label.top(TEX("D"),D);
            label.bot(TEX("E"),E);
            label.rt(TEX("B"),B);
        \end{Geometrie}
    \end{center}
    \begin{enumerate}
        \item Écrire le cosinus de l'angle $\widehat{DAB}$ de trois façons différentes, en précisant le triangle utilisé.
        
        {\color{red}
        \begin{itemize}
            \def\item{$\bullet$}
            \item Dans le triangle $ADB$ rectangle en $D$, $cos(\widehat{DAB})=\dfrac{AD}{AB}$
            
            \item Dans le triangle $APE$ rectangle en $E$, $cos(\widehat{DAB})=\dfrac{AE}{AP}$
            
            \item Dans le triangle $AFG$ rectangle en $F$, $cos(\widehat{DAB})=\dfrac{AF}{AG}$
        \end{itemize}
        }
        \item Faire une remarque justifiée sur ces trois rapports.
        
        \textcolor{red}{Ces trois rapports représentent tous le cosinus de l'angle $\widehat{DAB}$, ils sont donc égaux. $$\dfrac{AD}{AB}=\dfrac{AE}{AP}=\dfrac{AF}{AG}$$}
    \end{enumerate}
\end{corrige}

