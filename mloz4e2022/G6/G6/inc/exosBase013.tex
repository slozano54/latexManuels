\begin{exercice*}
    On considère cette figure.

    \begin{center}
        \begin{Geometrie}
            pair I,V,U;
            U=u*(1,1);
            I-U=u*(2,2);
            V=1.3[I,rotation(U,I,90)];
            trace polygone(V,I,U);
            trace codeperp(V,I,U,5);
            trace cotationmil(U,V,-3mm,15,TEX("\Lg[cm]{6}"));
            marque_a:=0.8*marque_a;                
            trace Codeangle(I,V,U,0,TEX("\textbf{?}")) withpen pencircle scaled 1bp;
            trace Codeangle(V,U,I,0,TEX("\ang{55}")) withpen pencircle scaled 1bp;
            label.lrt(TEX("V"),V);
            label.llft(TEX("U"),U);
            label.top(TEX("I"),I);
        \end{Geometrie}
    \end{center}
    \begin{enumerate}
        \item Avec ces données, indiquer la longueur que l'on peut calculer. La calculer et arrondir au millimètre.
        \item Détermne la la mesure de l'angle $\widehat{IVU}$. Justifier.
        \item En déduire la longueur du troisième côté du triangle $IVU$.
    \end{enumerate}
\end{exercice*}
\begin{corrige}
    %\setcounter{partie}{0} % Pour s'assurer que le compteur de \partie est à zéro dans les corrigés
    %\phantom{rrr}    
    On considère cette figure.

    \begin{Geometrie}
        pair I,V,U;
        U=u*(1,1);
        I-U=u*(2,2);
        V=1.3[I,rotation(U,I,90)];
        trace polygone(V,I,U);
        trace codeperp(V,I,U,5);
        trace cotationmil(U,V,-3mm,15,TEX("\Lg[cm]{6}"));
        marque_a:=0.8*marque_a;                
        trace Codeangle(I,V,U,0,TEX("\textbf{?}")) withpen pencircle scaled 1bp;
        trace Codeangle(V,U,I,0,TEX("\ang{55}")) withpen pencircle scaled 1bp;
        label.lrt(TEX("V"),V);
        label.llft(TEX("U"),U);
        label.top(TEX("I"),I);
    \end{Geometrie}

    \begin{enumerate}
        \item Avec ces données, indiquer la longueur que l'on peut calculer. La calculer et arrondir au millimètre.
        
        {\color{red}Dans le triangle $VUI$ rectangle en $I$, on connait la longueur de l'hypoténuse et l'angle $\widehat{VUI}$ on peut donc calculer le côté adjacent à cet angle, $UI$.

        \Trigo[Cosinus,Precision=1]{UIV}{}{6}{55}
        }
    % \end{enumerate}
    % \Coupe
    % \begin{enumerate}
    %     \setcounter{enumi}{1}
        \item Détermne la la mesure de l'angle $\widehat{IVU}$. Justifier.
        
        {\color{red} Dans un triangle rectangle, les angles aigus sont complémentaires, donc $\widehat{IVU}=\ang{90}-\widehat{IUV}=\ang{90}-\ang{55}=\ang{35}$.}
        \item En déduire la longueur du troisième côté du triangle $IVU$.
        
        {\color{red}\Trigo[Cosinus,Precision=1]{VIU}{}{6}{35}}
    \end{enumerate}
\end{corrige}

