% https://www.coursdeprofs.fr/publication/P-57f99zn2/TP-Activites-decouvertes-Trigonometrie.html
\begin{changemargin}{-10mm}{-10mm}
    \begin{activite}[Découverte avec GeoGebra]
        \begin{minipage}{0.6\linewidth}
            \begin{remarque}
                L'activité peut se faire :
                \begin{itemize}
                    \item En classe entière au vidéoprojecteur.
                    \item En devoir à la maison.
                    \item En salle informatique par deux.
                    \item En salle informatique individuellement.
                \end{itemize}
            \end{remarque}
        \end{minipage}
        \begin{minipage}{0.35\linewidth}
            \begin{center}
                \includegraphics[scale=0.3]{\currentpath/images/trigoIntro.png}
            \end{center}
        \end{minipage}
        \begin{center}
            % {\Huge \href{https://www.geogebra.org/m/ed8skenp}{\emoji{link} Ouvrir l'activité Geogebra}}    
            {\Huge \hrefLien{https://www.geogebra.org/classic/ed8skenp}{Ouvrir l'activité Geogebra}}    
            
        \end{center}
        \begin{enumerate}
            \item Déplacer le point$A$. Faire une remarque.
            \par\medskip
            \pointilles\par\medskip
            \pointilles\par\medskip
            \pointilles\par\medskip
            \pointilles
            \item Compléter la phrase suivante :\\
            Les rapports des longueurs de deux côtés d'un triangle rectangle \pointilles[0.3\linewidth] de la longueur des côtés.
            \item Modifier la valeur de l'angle $\widehat{ABC}$ à l'aide du curseur. Faire une remarque.
            \par\medskip
            \pointilles\par\medskip
            \pointilles\par\medskip
            \pointilles\par\medskip
            \pointilles\par\medskip        
            \item Compléter la phrase suivante :\\
            Les rapports des longueurs de deux côtés d'un triangle rectangle \pointilles[0.3\linewidth] de la mesure de l'angle $\widehat{ABC}$.
        \end{enumerate}
    
        % \textbf{Compléter la trace écrite.}
        \clearpage
        \begin{enumerate}
            \setcounter{enumi}{4}
            \item Compléter le tableau suivant en faisant bouger le curseur. Les trois derniers angles sont à choisir librement.
            
            \begin{center}
                \begin{tabular}{|c|*{7}{@{}>{\vrule width0pt height\dimexpr.65cm-.2pt\relax depth\dimexpr.35cm-.2pt\relax\centering\arraybackslash}p{\dimexpr2cm-.4pt\relax}@{}|}}
                    \hline
                    &$\widehat{ABC}$&{\red $BC$}&{\blue $BA$}&{\color{mygreen} $CA$}&$\dfrac{BA}{BC}$&$\dfrac{CA}{BC}$&$\dfrac{CA}{BA}$ \\
                    \hline
                    {\bfseries 1}&\ang{30}&{\red\num{8.155}}&{\blue\num{7.062}}&{\color{mygreen}\num{4.077}}&\num{0.866}&\num{0.5}&\num{0.577}\\
                    \hline
                    {\bfseries 2}&\ang{45}&&&&&&\\
                    \hline
                    {\bfseries 3}&\ang{60}&&&&&&\\
                    \hline
                    {\bfseries 4}&\dots&&&&&&\\
                    \hline
                    {\bfseries 5}&\dots&&&&&&\\
                    \hline
                    {\bfseries 6}&\dots&&&&&&\\
                    \hline
                \end{tabular}
            \end{center}        
            \item Compléter le tableau suivant en reprenant les angles de la question précédente et à l'aide de la calculatrice.
            
            \begin{center}
                \begin{tabular}{|c|*{4}{@{}>{\vrule width0pt height\dimexpr.65cm-.2pt\relax depth\dimexpr.35cm-.2pt\relax\centering\arraybackslash}p{\dimexpr2cm-.4pt\relax}@{}|}}
                    \hline
                    &$\widehat{ABC}$&$\cos(\widehat{ABC})$&$\sin(\widehat{ABC})$&$\tan(\widehat{ABC})$\\
                    \hline
                    {\bfseries 1}&\ang{30}&&&\\
                    \hline
                    {\bfseries 2}&\ang{45}&&&\\
                    \hline
                    {\bfseries 3}&\ang{60}&&&\\
                    \hline
                    {\bfseries 4}&\dots&&&\\
                    \hline
                    {\bfseries 5}&\dots&&&\\
                    \hline
                    {\bfseries 6}&\dots&&&\\
                    \hline
                \end{tabular}
            \end{center}        
            \item Faire une remarque en comparant les deux tableaux.
            \par\medskip
            \pointilles\par\medskip
            \pointilles\par\medskip
            \item Écrire trois égalités entre $\cos(\widehat{ABC})$; $\sin(\widehat{ABC})$; $\tan(\widehat{ABC})$ et deux côtés du triangle, $AB$, $AC$ ou $BC$.
        \end{enumerate}
    \end{activite}
    \end{changemargin}
    