\begin{exercice*}
    Pour chaque figure, calculer la mesure de l'angle indiqué par un \og{}\textbf{?}\fg{}, arrondir au degré. Détailler la rédaction.
    \begin{enumerate}
        \item \textbf{Figure 1}
        
        \begin{center}
            \begin{Geometrie}
                pair I,K,S;
                I=u*(6,1);
                S-I=u*(0,2);
                K-I=u*(-sqrt(32),0);
                trace polygone(I,K,S);
                trace codeperp(S,I,K,5);
                trace cotationmil(I,S,-3mm,15,TEX("\Lg[cm]{2}"));
                trace cotationmil(K,S,3mm,15,TEX("\Lg[cm]{6}"));
                marque_a:=0.6*marque_a;                
                trace Codeangle(K,S,I,0,TEX("\textbf{?}")) withpen pencircle scaled 1bp;
                label.urt(TEX("S"),S);
                label.llft(TEX("K"),K);
                label.lrt(TEX("I"),I);
            \end{Geometrie}
        \end{center}
        \item \textbf{Figure 2}
        
        \begin{center}
            \begin{Geometrie}
                pair F,U,N;
                F=u*(1,3.5);
                N-F=u*(4.15,0);
                U-F=u*(0,-0.5*sqrt(31.11));
                trace polygone(F,U,N);
                trace codeperp(U,F,N,5);
                trace cotationmil(F,N,3mm,15,TEX("\Lg[cm]{8.3}"));
                trace cotationmil(U,N,-3mm,15,TEX("\Lg[cm]{10}"));
                marque_a:=0.7*marque_a;
                trace Codeangle(F,N,U,0,TEX("\textbf{?}")) withpen pencircle scaled 1bp;
                label.urt(TEX("N"),N);
                label.llft(TEX("U"),U);
                label.ulft(TEX("F"),F);
            \end{Geometrie}
        \end{center}
    \end{enumerate}
\end{exercice*}
\begin{corrige}
    %\setcounter{partie}{0} % Pour s'assurer que le compteur de \partie est à zéro dans les corrigés
    %\phantom{rrr}
    Pour chaque figure, calculer la mesure de l'angle indiqué par un \og{}\textbf{?}\fg{}, arrondir au degré. Détailler la rédaction.

    \begin{enumerate}
        \item \textbf{Figure 1}
        
        \scalebox{0.8}{
        \begin{Geometrie}
            pair I,K,S;
            I=u*(6,1);
            S-I=u*(0,2);
            K-I=u*(-sqrt(32),0);
            trace polygone(I,K,S);
            trace codeperp(S,I,K,5);
            trace cotationmil(I,S,-3mm,15,TEX("\Lg[cm]{2}"));
            trace cotationmil(K,S,3mm,15,TEX("\Lg[cm]{6}"));
            marque_a:=0.6*marque_a;                
            trace Codeangle(K,S,I,0,TEX("\textbf{?}")) withpen pencircle scaled 1bp;
            label.urt(TEX("S"),S);
            label.llft(TEX("K"),K);
            label.lrt(TEX("I"),I);
        \end{Geometrie}
        }

        \textcolor{red}{\Trigo[Cosinus]{SIK}{2}{6}{}}
        \item \textbf{Figure 2}
        
        \scalebox{0.8}{
        \begin{Geometrie}
            pair F,U,N;
            F=u*(1,3.5);
            N-F=u*(4.15,0);
            U-F=u*(0,-0.5*sqrt(31.11));
            trace polygone(F,U,N);
            trace codeperp(U,F,N,5);
            trace cotationmil(F,N,3mm,15,TEX("\Lg[cm]{8.3}"));
            trace cotationmil(U,N,-3mm,15,TEX("\Lg[cm]{10}"));
            marque_a:=0.7*marque_a;
            trace Codeangle(F,N,U,0,TEX("\textbf{?}")) withpen pencircle scaled 1bp;
            label.urt(TEX("N"),N);
            label.llft(TEX("U"),U);
            label.ulft(TEX("F"),F);
        \end{Geometrie}
        }

        \textcolor{red}{\Trigo[Cosinus]{NFU}{8.3}{10}{}}
    \end{enumerate}    
\end{corrige}

