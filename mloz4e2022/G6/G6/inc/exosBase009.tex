\begin{exercice*}[Mesure d'un angle]
    On considère cette figure.
    \begin{center}
        \begin{Geometrie}
            pair E,F,G;
            F=u*(1,1);
            E-F=u*(0,2);
            G-F=u*(0.5*sqrt(33),0);
            trace polygone(E,F,G);
            trace codeperp(G,F,E,5);
            trace cotationmil(F,E,3mm,15,TEX("\Lg[cm]{4}"));
            trace cotationmil(E,G,3mm,15,TEX("\Lg[cm]{7}"));
            label.ulft(TEX("E"),E);
            label.llft(TEX("F"),F);
            label.lrt(TEX("G"),G);
        \end{Geometrie}
    \end{center}
    \begin{enumerate}
        \item Exprimer le cosinus de l'angle $\widehat{FEG}$.
        \vspace*{20mm}
        \item Calculer la mesure de l'angle $\widehat{FEG}$, arrondir au degré près.
        \vfill
    \end{enumerate}
\end{exercice*}
\begin{corrige}
    %\setcounter{partie}{0} % Pour s'assurer que le compteur de \partie est à zéro dans les corrigés
    %\phantom{rrr}    
    On considère cette figure.
    \begin{center}
        \begin{Geometrie}
            pair E,F,G;
            F=u*(1,1);
            E-F=u*(0,2);
            G-F=u*(0.5*sqrt(33),0);
            trace polygone(E,F,G);
            trace codeperp(G,F,E,5);
            trace cotationmil(F,E,3mm,15,TEX("\Lg[cm]{4}"));
            trace cotationmil(E,G,3mm,15,TEX("\Lg[cm]{7}"));
            label.ulft(TEX("E"),E);
            label.llft(TEX("F"),F);
            label.lrt(TEX("G"),G);
        \end{Geometrie}
    \end{center}
    \begin{enumerate}
        \item Exprimer le cosinus de l'angle $\widehat{FEG}$.
        
        {\color{red}Dans le triangle $EFG$ rectangle en $F$, $cos(\widehat{FEG})=\dfrac{FE}{EG}=\dfrac47$}
        \item Calculer la mesure de l'angle $\widehat{FEG}$, arrondir au degré près.
        
        {\color{red}$cos(\widehat{FEG})=\dfrac47$ d'où $\widehat{FEG}=cos^{-1}\left(\dfrac47\right)\approx\ang{55}$.}
    \end{enumerate}
\end{corrige}

