\begin{exercice*}[Longueur de l'hypoténuse]
    On considère cette figure.

    \begin{center}
        \begin{Geometrie}
            pair I,V,U;
            V=u*(1,1);
            I-V=u*(3,1);
            U=0.4[I,rotation(V,I,90)];
            trace polygone(V,I,U);
            trace codeperp(V,I,U,5);
            trace cotationmil(V,I,3mm,15,TEX("\Lg[cm]{3.5}"));
            marque_a:=1.2*marque_a;                
            trace Codeangle(U,V,I,0,TEX("\ang{21}")) withpen pencircle scaled 1bp;
            label.llft(TEX("V"),V);
            label.lrt(TEX("U"),U);
            label.urt(TEX("I"),I);
        \end{Geometrie}
    \end{center}
    \begin{enumerate}
        \item Dans le triangle $VUI$, rectangle en $I$, exprimer le cosinus de l'angle $\widehat{IVU}$ en fonction des longueurs des côtés.
        \item Exprimer alors la longueur $VU$ en fonction de $IV$ et du cosinus de l'angle $\widehat{IVU}$.
        \item À l'aide de la calculatrice, déduire la mesure, arrondie au millimètre, de la longueur $VU$.
    \end{enumerate}
\end{exercice*}
\begin{corrige}
    %\setcounter{partie}{0} % Pour s'assurer que le compteur de \partie est à zéro dans les corrigés
    %\phantom{rrr}    
    On considère cette figure.

    \begin{Geometrie}
        pair I,V,U;
        V=u*(1,1);
        I-V=u*(3,1);
        U=0.4[I,rotation(V,I,90)];
        trace polygone(V,I,U);
        trace codeperp(V,I,U,5);
        trace cotationmil(V,I,3mm,15,TEX("\Lg[cm]{3.5}"));
        marque_a:=1.2*marque_a;                
        trace Codeangle(U,V,I,0,TEX("\ang{21}")) withpen pencircle scaled 1bp;
        label.llft(TEX("V"),V);
        label.lrt(TEX("U"),U);
        label.urt(TEX("I"),I);
    \end{Geometrie}

    \begin{enumerate}
        \item Dans le triangle $VUI$, rectangle en $I$, exprimer le cosinus de l'angle $\widehat{IVU}$ en fonction des longueurs des côtés.
        
        {\color{red} $cos(\widehat{IVU})=\dfrac{IV}{UV}$}
        \item Exprimer alors la longueur $VU$ en fonction de $IV$ et du cosinus de l'angle $\widehat{IVU}$.
        
        {\color{red} $UV=\dfrac{IV}{cos(\widehat{IVU})}$}
        \item À l'aide de la calculatrice, déduire la mesure, arrondie au millimètre, de la longueur $VU$.
        
        {\color{red} $UV=\dfrac{IV}{cos(\widehat{IVU})}=\dfrac{\num{3.5}}{cos(\ang{21})}\approx\Lg[cm]{3.7}$}
    \end{enumerate}
\end{corrige}

