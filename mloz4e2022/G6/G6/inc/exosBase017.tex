\begin{exercice*}
    $[A'B']$ est l'image de $[AB]$ sur l'écran d'une chambre noire d'un appareil photo d'orifice $O$.
    \begin{center}
        \begin{Geometrie}
            pair A,A',B,B',O;
            O=u*(2,2);
            B-O=u*(sqrt(9.8),0);
            A-B=u*(0,2.8);
            B'=1.5[A,O];
            A'=1.5[B,O];
            trace segment(A',B);
            trace segment(A,B') dashed evenly;
            drawarrow A'--B' withpen pencircle scaled 1bp;
            drawarrow B--A withpen pencircle scaled 1bp;
            trace codeperp(A,B,O,5);
            trace codeperp(B',A',O,5);
            trace appelation(O,A,3mm,TEX("\Lg[m]{2.10}"));
            trace appelation(B,A,-3mm,TEX("\Lg[m]{1.40}"));
            trace appelation(B',O,-3mm,TEX("\Lg[m]{0.03}"));
            label.top(TEX("A'"),A');
            label.urt(TEX("A"),A);
            label.rt(TEX("B"),B);
            label.ulft(TEX("O"),O);
            label.bot(TEX("B'"),B');
        \end{Geometrie}
    \end{center}
    \begin{enumerate}
        \item Démontrer l'égalité des angles $\widehat{A'B'O}$ et $\widehat{OAB}$.
        \item Écrire $\text{cos}(\widehat{A'B'O})$ en fonction de $A'B'$ puis, en utilisant $\text{cos}(\widehat{OAB})$, en déduire la valeur exacte de la longueur $A'B'$.
    \end{enumerate}
\end{exercice*}
\begin{corrige}
    %\setcounter{partie}{0} % Pour s'assurer que le compteur de \partie est à zéro dans les corrigés
    %\phantom{rrr}    
    $[A'B']$ est l'image de $[AB]$ sur l'écran d'une chambre noire d'un appareil photo d'orifice $O$.

    \begin{Geometrie}
        pair A,A',B,B',O;
        O=u*(2,2);
        B-O=u*(sqrt(9.8),0);
        A-B=u*(0,2.8);
        B'=1.5[A,O];
        A'=1.5[B,O];
        trace segment(A',B);
        trace segment(A,B') dashed evenly;
        drawarrow A'--B' withpen pencircle scaled 1bp;
        drawarrow B--A withpen pencircle scaled 1bp;
        trace codeperp(A,B,O,5);
        trace codeperp(B',A',O,5);
        trace appelation(O,A,3mm,TEX("\Lg[m]{2.10}"));
        trace appelation(B,A,-3mm,TEX("\Lg[m]{1.40}"));
        trace appelation(B',O,-3mm,TEX("\Lg[m]{0.03}"));
        label.top(TEX("A'"),A');
        label.urt(TEX("A"),A);
        label.rt(TEX("B"),B);
        label.ulft(TEX("O"),O);
        label.bot(TEX("B'"),B');
    \end{Geometrie}

    \begin{enumerate}
        \item Démontrer l'égalité des angles $\widehat{A'B'O}$ et $\widehat{OAB}$.
        
        {\color{red}$(AB)$ et $(A'B')$ sont toutes les deux perpendiculaires à la même droite, $(A'B)$, elles sont donc parallèles.
        Les parallèles $(AB)$ et $(A'B')$ déterminent donc avec la sécante $(AB')$, des angles alternes-internes $\widehat{A'B'O}$ et $\widehat{OAB}$ de même mesure.}
        \item Écrire $\text{cos}(\widehat{A'B'O})$ en fonction de $A'B'$ puis, en utilisant $\text{cos}(\widehat{OAB})$, en déduire la valeur exacte de la longueur $A'B'$.
        
        {\color{red}LE triangle $OA'B'$ est rectangle en $A'$ sonc : $\text{cos}(\widehat{A'B'O})=\dfrac{A'B'}{B'O}=\dfrac{A'B'}{\num{0.03}}$.
        De plus, $\text{cos}(\widehat{A'B'O})=\text{cos}(\widehat{OAB})=\dfrac{\num{1.4}}{\num{2.1}}$ d'où $A'B'=\dfrac{\num{0.03}\times\num{1.4}}{\num{2.1}}=\Lg[m]{0.02}$
        }
    \end{enumerate}
\end{corrige}

