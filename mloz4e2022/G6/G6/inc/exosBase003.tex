\begin{exercice*}
    Entourer les triangles dans lesquels \mbox{$\text{cos}(\widehat{EGF})=\dfrac{GF}{GE}$} et barrer les autres.
    
    \begin{minipage}{0.45\linewidth}
        \begin{Geometrie}
            pair E,F,G,C;
            E=u*(1,2);
            F-E=u*(1,1);            
            G=1.4[F,rotation(E,F,90)];
            C=iso(E,G);            
            trace polygone(E,F,G);
            trace segment(F,C);
            marque_s:=0.3*marque_s;
            trace Codelongueur(E,C,C,G,C,F,2);
            label.lft(TEX("E"),E);
            label.top(TEX("F"),F);
            label.rt(TEX("G"),G);
            label.bot(TEX("C"),C);
        \end{Geometrie}

        \bigskip
        \begin{Geometrie}
            pair E,F,G;
            E=u*(3,1);
            F-E=u*(-2,1);            
            G=0.7[E,rotation(F,E,-90)];            
            trace polygone(E,F,G);            
            trace cotationmil(F,E,-3mm,15,TEX("\Lg[cm]{4}"));
            trace cotationmil(E,G,-3mm,15,TEX("\Lg[cm]{3}"));
            trace cotationmil(F,G,3mm,15,TEX("\Lg[cm]{5}"));
            label.bot(TEX("E"),E);
            label.lft(TEX("F"),F);
            label.urt(TEX("G"),G);
        \end{Geometrie}
    \end{minipage}
    \hfill
    \begin{minipage}{0.45\linewidth}
        \begin{Geometrie}
            pair E,F,G;
            E=u*(1,2);
            F-E=u*(1,1);            
            G=1.4[F,rotation(E,F,90)];
            trace codeperp(E,F,G,5);
            trace polygone(E,F,G);
            label.lft(TEX("E"),E);
            label.top(TEX("G"),F);
            label.rt(TEX("F"),G);
        \end{Geometrie}

        \bigskip
        \begin{Geometrie}
            pair E,F,G;
            E=u*(1,1);
            F-E=u*(3,1);            
            G=0.6[F,rotation(E,F,-90)];
            trace codeperp(E,F,G,5);
            trace polygone(E,F,G);
            label.lft(TEX("E"),E);
            label.rt(TEX("F"),F);
            label.top(TEX("G"),G);
        \end{Geometrie}
    \end{minipage}
\end{exercice*}
\begin{corrige}
    %\setcounter{partie}{0} % Pour s'assurer que le compteur de \partie est à zéro dans les corrigés
    %\phantom{rrr}    
    Entourer les triangles dans lesquels \mbox{$\text{cos}(\widehat{EGF})=\dfrac{GF}{GE}$} et barrer les autres.
    
    \begin{minipage}{0.45\linewidth}
        \begin{Geometrie}
            pair E,F,G,C;
            E=u*(1,2);
            F-E=u*(1,1);            
            G=1.4[F,rotation(E,F,90)];
            C=iso(E,G);            
            trace polygone(E,F,G);
            trace segment(F,C);
            marque_s:=0.3*marque_s;
            trace Codelongueur(E,C,C,G,C,F,2);
            label.lft(TEX("E"),E);
            label.top(TEX("F"),F);
            label.rt(TEX("G"),G);
            label.bot(TEX("C"),C);
            trace cercles(iso(E,F,G),1.5u) withcolor red;
        \end{Geometrie}
        \textcolor{red}{$EFG$ est bien rectangle, on peut le justifier à l'aide des triangles isocèles et de la somme des angles d'un triangle.}

        \bigskip
        \begin{Geometrie}
            pair E,F,G;
            E=u*(3,1);
            F-E=u*(-2,1);            
            G=0.7[E,rotation(F,E,-90)];            
            trace polygone(E,F,G);            
            trace cotationmil(F,E,-3mm,15,TEX("\Lg[cm]{4}"));
            trace cotationmil(E,G,-3mm,15,TEX("\Lg[cm]{3}"));
            trace cotationmil(F,G,3mm,15,TEX("\Lg[cm]{5}"));
            label.bot(TEX("E"),E);
            label.lft(TEX("F"),F);
            label.urt(TEX("G"),G);
            trace segment(F+(0.2u,0.5u),E+(0.2u,-0.2u)) withcolor red;
            trace segment(F+(0.2u,-u),G+(0.2u,0.2u)) withcolor red;
        \end{Geometrie}
        \textcolor{red}{$EFG$ est bien rectangle grâce à la réciproque du théorème de Pythagore mais $\text{cos}(\widehat{EGF})\neq\dfrac{GF}{GE}$.}
    \end{minipage}
    \hfill
    \begin{minipage}{0.45\linewidth}
        \begin{Geometrie}
            pair E,F,G;
            E=u*(1,2);
            F-E=u*(1,1);            
            G=1.4[F,rotation(E,F,90)];
            trace codeperp(E,F,G,5);
            trace polygone(E,F,G);
            label.lft(TEX("E"),E);
            label.top(TEX("G"),F);
            label.rt(TEX("F"),G);
            trace segment(G+(0,-0.2u),E+(0,1.5u)) withcolor red;
            trace segment(F+(u,0),E+(-0.2u,-0.2u)) withcolor red;
        \end{Geometrie}
        \textcolor{red}{$EFG$ est bien rectangle mais $\widehat{EGF}=\ang{90}$.}

        \bigskip
        \begin{Geometrie}
            pair E,F,G;
            E=u*(1,1);
            F-E=u*(3,1);            
            G=0.6[F,rotation(E,F,-90)];
            trace codeperp(E,F,G,5);
            trace polygone(E,F,G);
            label.lft(TEX("E"),E);
            label.rt(TEX("F"),F);
            label.top(TEX("G"),G);
            trace cercles(iso(E,F,G),2u) withcolor red;
        \end{Geometrie}
        \textcolor{red}{$EFG$ est bien rectangle et $\text{cos}(\widehat{EGF})\neq\dfrac{GF}{GE}$.}
    \end{minipage}
\end{corrige}

