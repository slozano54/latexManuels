\begin{exercice*}[Longueur du côté adjacent]
    On considère cette figure.

    \begin{center}
        \begin{Geometrie}
            pair I,K,J;
            I=u*(1,1);
            K-I=u*(3,0.5);
            J=0.8[K,rotation(I,K,-90)];
            trace polygone(I,K,J);
            trace codeperp(J,K,I,5);
            trace cotationmil(I,J,3mm,15,TEX("\Lg[cm]{6}"));
            marque_a:=0.6*marque_a;                
            trace Codeangle(I,J,K,0,TEX("\ang{53}")) withpen pencircle scaled 1bp;
            label.urt(TEX("J"),J);
            label.lrt(TEX("K"),K);
            label.llft(TEX("I"),I);
        \end{Geometrie}
    \end{center}
    \begin{enumerate}
        \item Dans le triangle $IJK$, rectangle en $K$, exprimer le cosinus de l'angle $\widehat{IJK}$ en fonction des longueurs des côtés.
        \item Exprimer alors la longueur $JK$ en fonction de $IJ$ et du cosinus de l'angle $\widehat{IJK}$.
        \item À l'aide de la calculatrice, déduire la mesure, arrondie au millimètre, de la longueur $JK$.
    \end{enumerate}
\end{exercice*}
\begin{corrige}
    %\setcounter{partie}{0} % Pour s'assurer que le compteur de \partie est à zéro dans les corrigés
    %\phantom{rrr}    
    On considère cette figure.

    \begin{Geometrie}
        pair I,K,J;
        I=u*(1,1);
        K-I=u*(3,0.5);
        J=0.8[K,rotation(I,K,-90)];
        trace polygone(I,K,J);
        trace codeperp(J,K,I,5);
        trace cotationmil(I,J,3mm,15,TEX("\Lg[cm]{6}"));
        marque_a:=0.6*marque_a;                
        trace Codeangle(I,J,K,0,TEX("\ang{53}")) withpen pencircle scaled 1bp;
        label.urt(TEX("J"),J);
        label.lrt(TEX("K"),K);
        label.llft(TEX("I"),I);
    \end{Geometrie}

    \begin{enumerate}
        \item Dans le triangle $IJK$, rectangle en $K$, exprimer le cosinus de l'angle $\widehat{IJK}$ en fonction des longueurs des côtés.
        
        {\color{red} $\text{cos}(\widehat{IJK})=\dfrac{JK}{IJ}$}
        \item Exprimer alors la longueur $JK$ en fonction de $IJ$ et du cosinus de l'angle $\widehat{IJK}$.
        
        {\color{red} $JK=IJ\times \text{cos}(\widehat{IJK})$}
        \item À l'aide de la calculatrice, déduire la mesure, arrondie au millimètre, de la longueur $JK$.
        
        {\color{red} $JK=IJ\times \text{cos}(\widehat{IJK})=6\times \text{cos}(\ang{53})\approx\Lg[cm]{3.6}$}
    \end{enumerate}
\end{corrige}

