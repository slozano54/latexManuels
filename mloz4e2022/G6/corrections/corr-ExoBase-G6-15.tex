    %\setcounter{partie}{0} % Pour s'assurer que le compteur de \partie est à zéro dans les corrigés
    %\phantom{rrr}
    On considère cette figure.
    \begin{Geometrie}
        pair L,K,J[],P[];
        L=u*(1,1);
        K-L=u*(4.25,0);
        P0=rotation(K,L,35);
        P1=projection(K,L,P0);
        J0=rotation(K,P1,37);
        J1=projection(K,P1,J0);
        trace polygone(L,K,P1);
        trace polygone(J1,K,P1);
        trace codeperp(L,P1,K,5);
        trace codeperp(P1,J1,K,5);
        marque_a:=0.8*marque_a;
        trace Codeangle(P1,K,L,0,TEX("\ang{55}")) withpen pencircle scaled 1bp;
        trace Codeangle(K,P1,J1,0,TEX("\ang{37}")) withpen pencircle scaled 1bp;
        trace appelation(L,K,-3mm,TEX("\Lg[cm]{8.5}"));
        label.llft(TEX("L"),L);
        label.lrt(TEX("K"),K);
        label.top(TEX("P"),P1);
        label.urt(TEX("J"),J1);
    \end{Geometrie}

    \begin{enumerate}
        \item Calculer la longueur $PK$, arrondie au millimètre.

        {\color{red}\Trigo[Cosinus,Precision=1]{KPL}{}{8.5}{55}}
        \item En déduire la longueur $PJ$, arrondie au mm.

        {\color{red}\Trigo[Cosinus,Precision=1]{PJK}{}{4.9}{37}}
    \end{enumerate}
