    %\setcounter{partie}{0} % Pour s'assurer que le compteur de \partie est à zéro dans les corrigés
    %\phantom{rrr}
    On considère cette figure.

    \begin{center}
        \scalebox{0.8}{
        \begin{Geometrie}
            pair R,S,T,U,V,P;
            numeric rayon;
            rayon:=2;
            P=u*(3,3);
            U-P=u*(-rayon,0);
            S-P=u*(rayon,0);
            path cc;
            cc=cercles(P,rayon*u);
            R=pointarc(cc,60);
            T=pointarc(cc,210);
            trace cc;
            V=projection(P,U,R);
            trace polygone(U,R,S,T);
            trace segment(V,P);
            trace segment(U,S);
            trace codeperp(U,V,P,5);
            trace codeperp(S,T,U,5);
            trace codeperp(U,R,S,5);
            label.top(TEX("R"),R);
            label.rt(TEX("S"),S);
            label.lft(TEX("T"),T);
            label.lft(TEX("U"),U);
            label.ulft(TEX("V"),V);
            label.bot(TEX("P"),P);
        \end{Geometrie}
        }
    \end{center}
    \begin{enumerate}
        \item Exprimer le cosinus des angles $\widehat{TUS}$ et $\widehat{TSU}$.

        {\color{red}Dans le triangle $TSU$ rectangle en $T$, $$\text{cos}(\widehat{TUS})=\dfrac{TU}{US} \text{ et } \text{cos}(\widehat{TSU})=\dfrac{TS}{US}$$.}
        \item Exprimer le cosinus des angles $\widehat{VPU}$.

        {\color{red}Dans le triangle $VPU$ rectangle en $V$, $$\text{cos}(\widehat{VPU})=\dfrac{VP}{PU}$$.}
        \item Exprimer de deux façons différentes le cosinus des angles $\widehat{RUS}$.

        {\color{red}Dans le triangle $RUS$ rectangle en $R$, $$\text{cos}(\widehat{RUS})=\dfrac{UR}{US}$$.
        Dans le triangle $VPU$ rectangle en $V$, $$\text{cos}(\widehat{RUS})=\dfrac{UV}{UP}$$.}
    \end{enumerate}
