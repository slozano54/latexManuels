    %\setcounter{partie}{0} % Pour s'assurer que le compteur de \partie est à zéro dans les corrigés
    %\phantom{rrr}
    $TICE$ est un losange de côté \Lg[cm]{7} tel que $\widehat{TIC}=\ang{64}$.

    \scalebox{1}{
    \begin{Geometrie}
        pair T,I,C,E,O;
        T=u*(1,4);
        C-T=u*(2.5,0);
        O=iso(T,C);
        I-O=u*(0,2.5);
        E-O=u*(0,-2.5);
        trace polygone(T,I,C,E);
        trace segment(I,E);
        trace segment(T,C);
        label.lft(TEX("T"),T);
        label.rt(TEX("C"),C);
        label.top(TEX("I"),I);
        label.bot(TEX("E"),E);
        label.urt(TEX("O"),O);
    \end{Geometrie}
    }

    \begin{enumerate}
        \item Justifier la position relative des droites $(IE)$ et $(TC)$.

        {\color{red}Les diagonales d'un losange sont perpendiculaires, or $TICE$ est un losange, donc $(IE)$ et $(TC)$ sont perpendiculaires.}
        \item Déterminer les mesures des angles $\widehat{TIE}$ et $\widehat{EIC}$. Justifier.

        {\color{red}Les diagonales d'un losange sont des axes de symétrie de ce losange, or $TICE$ est un losange, donc $\widehat{TIE}=\widehat{EIC}=\dfrac{\widehat{TIC}}{2}=\ang{32}$.}
    \end{enumerate}
    \Coupe
    \begin{enumerate}
        \setcounter{enumi}{2}
        \item Calculer la longueur $IO$, arrondie au millimètre.

        {\color{red}$(IE)$ et $(TC)$ sont perpendiculaires donc le triangle $TIO$ est rectangle en $O$.

        \Trigo[Cosinus,Precision=1]{OIT}{}{7}{32}}
        \item En déduire, en justifiant, la longueur de la diagonale $[IE]$, arrondie au millimètre.

        {\color{red}Les diagonales d'un losange ont le même milieu, or $TICE$ est un losange donc O est e milieu de $[IE]$ et $[TC]$ d'où $IE=IO\times 2\approx\Lg[cm]{5.9}\times 2$ soit \Lg[cm]{11.8}.}
        \item Calculer $TO$, puis $TC$, en arrondissant au millimètre.

        {\color{red}\Pythagore[Precision=1]{TOI}{7}{5.9}{}

        Comme dans la question précédente, $TC=2\times TO\approx\Lg[cm]{7.5}$.
        }
    \end{enumerate}
