    %\setcounter{partie}{0} % Pour s'assurer que le compteur de \partie est à zéro dans les corrigés
    %\phantom{rrr}
    En utilisant cette figure, compléter les phrases.

    \begin{Geometrie}
        pair E,F,G,H;
        E=u*(1,1);
        F-E=u*(1,1.5);
        G=1.4[F,rotation(E,F,90)];
        H=projection(F,E,G);
        trace codeperp(E,F,G,5);
        trace codeperp(G,H,F,5);
        trace polygone(E,F,G);
        trace segment(F,H);
        label.llft(TEX("E"),E);
        label.top(TEX("F"),F);
        label.lrt(TEX("G"),G);
        label.bot(TEX("H"),H);
    \end{Geometrie}

    \begin{enumerate}
        \item Dans le triangle $EGF$, rectangle en $F$, on a : $$cos(\widehat{FEG})=\textcolor{red}{\dfrac{EF}{EG}}$$
        \item Dans le triangle $FHE$, rectangle en $H$, on a : $$cos(\widehat{FEG})=\textcolor{red}{\dfrac{EH}{EF}}$$
        \item Dans le triangle \textcolor{red}{$FGH$ rectangle en $H$}, on a : $$\textcolor{red}{cos(\widehat{HGF})}=\dfrac{GH}{FG}$$
    \end{enumerate}
    \Coupe
    \begin{enumerate}
        \setcounter{enumi}{3}
        \item Dans le triangle \textcolor{red}{$FGH$ rectangle en $H$}, on a : $$\textcolor{red}{cos(\widehat{HFG})}=\dfrac{FH}{FG}$$
    \end{enumerate}
