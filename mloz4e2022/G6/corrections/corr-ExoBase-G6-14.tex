    %\setcounter{partie}{0} % Pour s'assurer que le compteur de \partie est à zéro dans les corrigés
    %\phantom{rrr}
    On considère cette figure sur laquelle on admet que le triangle $PEO$ est rectangle en $P$ et le triangle $EOF$ est rectangle en $F$.

    \begin{Geometrie}
        pair P,O,F,E,centre;
        numeric rayon;
        rayon=2.5;
        E=u*(1,1);
        O-E=u*(rayon*sqrt(2),rayon*sqrt(2));
        centre=iso(O,E);
        path cc;
        cc=cercle(centre,rayon*u);
        trace cc;
        trace segment (O,E);
        P=pointarc(cc,165);
        F=pointarc(cc,-30);
        trace polygone(P,E,F,O);
        trace appelation(E,O,2mm,TEX("\Lg[cm]{55}"));
        trace appelation(F,O,2mm,TEX("\Lg[cm]{36}"));
        marque_a:=0.8*marque_a;
        trace Codeangle(P,O,E,0,TEX("\ang{37}")) withpen pencircle scaled 1bp;
        label.llft(TEX("E"),E);
        label.urt(TEX("O"),O);
        label.ulft(TEX("P"),P);
        label.lrt(TEX("F"),F);
        trace codeperp(E,P,O,5) withcolor red;
        trace codeperp(O,F,E,5) withcolor red;
    \end{Geometrie}

    \begin{enumerate}
        \item Coder la figure.
        \item Indiquer ce qu'ont les triangles $PEO$ et $EOF$ en commun.

        \textcolor{red}{les triangles $PEO$ et $EOF$ ont la même hypoténuse.}
        \item Calculer la mesure de l'angle $\widehat{EOF}$. Arrondir au degré.

        {\color{red}\Trigo[Cosinus,Precision=0]{FOE}{36}{55}{}}
        \item Calculer la longueur $PO$. Arrondir au millimètre.

        {\color{red}\Trigo[Cosinus,Precision=1]{POE}{}{55}{37}}
    \end{enumerate}
    \Coupe
    \begin{enumerate}
        \setcounter{enumi}{4}
        \item Calculer de deux façons différentes la longueur $EF$. Arrondir au millimètre.

        {\color{red}\textbf{Méthode 1 :}

        $\widehat{EOF}\approx\ang{49}$ donc $\widehat{OEF}\approx\ang{90}-\ang{49} \text{ soit } \ang{41}$
        \Trigo[Cosinus,Precision=1]{FOE}{}{55}{41}

        \textbf{Méthode 2 :} avec le théorème de Pythagore

        \Pythagore[Precision=1]{EFO}{55}{36}{}
        }
    \end{enumerate}
