\begin{changemargin}{-10mm}{-10mm}
\begin{activite}[Le triangle de Sierpinski]
{\bfseries
Triangle de Sierpinski act1p52 ...
cf 6e G4 recreation pour base
}

{\it
    La figure initiale est un triangle équilatéral.
    On construit à l'intérieur de celui-ci un triangle en joignant les milieux du triangle initial.
    On réitère le procédé avec chacun des triangles périphériques.
    Lorsqu'on choisit d'arrêter les recursion à une étape, on colorie comme ci-dessous.
}
\partie[tracé de quelques états]
    \begin{description}
       \item[Étape 1 : ] Sur une feuille unie, tracer un triangle équilatéral de 16 cm de côté.
       \item[Étape 2 : ] \ \\ [-7mm]
       \begin{enumerate}
          \item Placer le milieu de chacun des trois côtés du triangle.
          \item Tracer le triangle équilatéral passant par les milieux des trois côtés à l'intérieur du grand triangle.
        \end{enumerate}
      \item[Étape 3 : ] \ \\ [-7mm]
      \begin{enumerate}
         \item Pour les trois triangles situés aux sommets du triangle principal, placer les milieux des côtés.
         \item Construire les trois triangles à l'intérieur.
      \end{enumerate}
      \item[Étape 4 : ] Reproduire l'étape 3 pour chacun des neuf triangles ainsi construits.      
    \end{description}
    {\bfseries Décorer/colorier les triangles à votre guise en utilisant une ou deux couleurs.}
  
   \multido{\iA=0+1,\iB=1+1}{4}{%
        \begin{pspicture}(4.5,4)
            \psSier(0,0){4cm}{\iA}            
            \uput[0](1.3,-0.5){Étape \iB}
        \end{pspicture}   
    }

    \partie[dénombrement]
    Nous allons utiliser la notation \og puissance \fg{}, définie de la façon suivante : $3^4 = \underbrace{3\times 3\times 3\times 3}_\textrm{4 facteurs}$
    Cette expression se lit \og 3 puissance 4 \fg{} ou \og 3 exposant 4 \fg{}

    \begin{enumerate}
        \item Combien y a-t-il de triangles noirs à l'étape 1 ? Écrire la réponse sous la forme d'une puissance de \num{3}.
        \item Combien y a-t-il de triangles noirs à l'étape 2 ? Écrire la réponse sous la forme d'une puissance de \num{3}.
        \item Combien y a-t-il de triangles noirs à l'étape 3 ? Écrire la réponse sous la forme d'une puissance de \num{3}.
        
        {\bfseries On continue à construire des triangles noirs.}
        \item Combien y a-t-il de triangles noirs à l'étape 4 ? Écrire la réponse sous la forme d'une puissance de \num{3}.
        \item Combien y a-t-il de triangles noirs à l'étape 7 ? Écrire la réponse sous la forme d'une puissance de \num{3}.
        \item Combien y a-t-il de triangles noirs à l'étape 20 ? Écrire la réponse sous la forme d'une puissance de \num{3}.
        
        {\bfseries À présent utilisons la calculatrice et sa touche \Calculatrice{/$x^{\blacksquare}$}.}
        \item Combien y a-t-il de triangles noirs à l'étape 10 ?
        \item Combien y a-t-il de triangles noirs à l'étape 13 ?
        \item Combien y a-t-il de triangles noirs à l'étape 15 ?
        \item Combien y a-t-il de triangles noirs à l'étape 18 ?
    \end{enumerate}

    \begin{center}
        \fbox{
           \begin{minipage}{13cm}
              {\bf Wacław Franciszek Sierpiński} (1882-1969) est un mathématicien polonais. \\
              Il s'intéresse entre autre aux fractales (une fractale est une forme géométrique qui se répète à l'identique lorsqu'on zoome ou qu'on \og dezoome\fg) dont certaines portent son nom : le triangle de Sierpiński, le tapis de Sierpiński et la courbe de Sierpiński. \\
              \begin{pspicture}(-0.7,0)(3,3.5)
                 \psSier(0,0){3cm}{5}
              \end{pspicture}
              \qquad
              \includegraphics[width=3cm]{\currentpath/images/Sierpinski_carpet}
              \quad
              \begin{pspicture}(-2,-1.5)(2,1.5)
                 \psSier[unit=0.1,n=4,fillstyle=solid,fillcolor=black] 
              \end{pspicture}
              \medskip
           \end{minipage}}
     \end{center}
\end{activite}
\end{changemargin}