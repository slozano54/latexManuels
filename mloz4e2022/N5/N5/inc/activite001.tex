\begin{changemargin}{-15mm}{-15mm}   
\begin{activite}[Le triangle de Sierpinski]
   \vspace*{-7mm}
   \partie[tracé de quelques états]
   \begin{minipage}{0.8\linewidth}
      \begin{description}
         \item[Étape 1 : ] Sur une feuille unie, tracer un triangle équilatéral de 16 cm de côté.
         \item[Étape 2 : ] \ \\ [-7mm]
         \begin{enumerate}
            \item Placer le milieu de chacun des trois côtés du triangle.
            \item Tracer le triangle équilatéral passant par les milieux des trois côtés\\ à l'intérieur du grand triangle.
         \end{enumerate}
      \item[Étape 3 : ] \ \\ [-7mm]
      \begin{enumerate}
         \item Pour les trois triangles situés aux sommets du triangle principal,\\ placer les milieux des côtés.
         \item Construire les trois triangles à l'intérieur.
      \end{enumerate}
      \item[Étape 4 : ] Reproduire l'étape 3 pour chacun des neuf triangles ainsi construits.      
      \end{description}
      {\bfseries Décorer/colorier les triangles à votre guise en utilisant une ou deux couleurs.}
   \end{minipage}
   \hspace*{-10mm}
   \begin{minipage}{0.6\linewidth}
      \vspace*{-20mm}
      \begin{pspicture}(4.5,2)
         \psSier(0,0){2cm}{0}            
         \uput[0](0.4,-0.5){Étape 1}
      \end{pspicture}
      \hspace*{-15mm}
      \begin{pspicture}(4.5,2)
         \psSier(0,0){2cm}{1}            
         \uput[0](0.4,-0.5){Étape 2}
      \end{pspicture}   

      \vspace*{10mm}
      \begin{pspicture}(4.5,2)
         \psSier(0,0){2cm}{2}            
         \uput[0](0.4,-0.5){Étape 3}
      \end{pspicture}
      \hspace*{-15mm}
      \begin{pspicture}(4.5,2)
         \psSier(0,0){2cm}{3}            
         \uput[0](0.4,-0.5){Étape 4}
      \end{pspicture}   
   \end{minipage}  

   \partie[dénombrement]
   Nous allons utiliser la notation \og puissance \fg{}, définie de la façon suivante : $3^4 = \underbrace{3\times 3\times 3\times 3}_\textrm{4 facteurs}$.\\    
   Cette expression se lit \og 3 puissance 4 \fg{} ou \og 3 exposant 4 \fg{}

   \begin{enumerate}
      \item Combien y a-t-il de triangles noirs à l'étape 1 ? à l'étape 2 ? à l'étape 3 ?\\Écrire la réponse sous la forme d'une puissance de \num{3}.        
      \begin{center}
      {\bfseries On continue à construire des triangles noirs.}
      \end{center}
      \item Combien y a-t-il de triangles noirs à l'étape 4 ? à l'étape 7 ? à l'étape 20 ?\\Écrire la réponse sous la forme d'une puissance de \num{3}.
      \begin{center}
      {\bfseries À présent utilisons la calculatrice et sa touche \Calculatrice{/$x^{\blacksquare}$}.}
      \end{center}
      \item Combien y a-t-il de triangles noirs à l'étape 10 ? à l'étape 13 ? à l'étape 15 ? à l'étape 18 ?\\Écrire la réponse sous la forme d'une puissance de \num{3} puis sous forme décimale.
   \end{enumerate}
   \vspace*{-10mm}
   \begin{center}
      \fbox{
         \begin{minipage}{\linewidth}
            {\bf Wacław Franciszek Sierpiński} (1882-1969) est un mathématicien polonais. \\
            Il s'intéresse entre autre aux fractales (une fractale est une forme géométrique qui se répète à l'identique lorsqu'on zoome ou qu'on \og dezoome\fg) dont certaines portent son nom : le triangle de Sierpiński, le tapis de Sierpiński et la courbe de Sierpiński.
            \begin{center}
               \begin{pspicture}(-0.7,0)(3,2)
                  \psSier(0,0){2.5cm}{5}
               \end{pspicture}
               \qquad
               \includegraphics[width=2.5cm]{\currentpath/images/Sierpinski_carpet}
               \quad
               \begin{pspicture}(-2,-1.3)(2,1)
                  \psSier[unit=0.083,n=4,fillstyle=solid,fillcolor=black] 
               \end{pspicture}
            \end{center}
            \vspace*{2mm}
         \end{minipage}
      }
   \end{center}
\end{activite}
\end{changemargin}