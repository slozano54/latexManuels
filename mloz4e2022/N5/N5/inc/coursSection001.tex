\section{Puisssances d'un entier relatif}
\begin{definition}
    Soit $n$ un nombre entier positif.
$$a^n=\underbrace{a\times a\times a\times\dots\times a}_{n\mbox{ facteurs}}\quad (n\geq 2)\qquad a^1=a\qquad a^0=1$$
\\ Le nombre n s'appelle \textbf{exposant}, $a^n$ se lit \textbf{a exposant n} ou \textbf{a puissance n}
\\ On note $a^{-n}$ l'inverse de $a^n$, c'est \`{a} dire le nombre qui lorsqu'il est multipli\'e par $a^n$ donne $1$. 
$$a^{-n}=\frac1{a^n}$$
\end{definition}

\begin{exemples*1}
    \begin{minipage}{0.5\linewidth}
        $2^5=2\times2\times2\times2\times2=32$ \\
        $(-5)^3=(-5)\times(-5)\times(-5)=-125$
    \end{minipage}
    \begin{minipage}{0.5\linewidth}
        $5^{-3}=\dfrac1{5^3}=\num{0.008}$ \\\smallskip
        $(-2)^{-4}=\dfrac1{(-2)^4}=\num{-0.0625}$
    \end{minipage}
\end{exemples*1}

\begin{methode}[Utilisation de la calculatrice]
    Touche \Calculatrice{//$x^{\blacksquare}$} ou \Calculatrice{//$x^n$} ou \Calculatrice{//$y^x$} ou \Calculatrice{//$\hat{}$} ou \Calculatrice{//$\uparrow$} ou \dots
% {\hfill$(-5)^3=\quad\underbrace{(\,5\,\pm\,)\,\uparrow\,3\,=}_{\mbox{calculatrice}}\quad-125$\hfill$2^5=\quad\underbrace{2\,\uparrow\,5=}_{\mbox{calculatrice}}\quad32$\hfill}
    \exercice

    À l'aide de la calculatrice, calculer :
    \begin{itemize}
        \item $(-5)^3$
        \item $2^5$
    \end{itemize}
    \correction

    $\underbrace{\Calculatrice{/(,//$(-)$,/5,/),//$x^{\blacksquare}$,/3}}_{\mbox{calculatrice}}$

    \smallskip
    \Calculatrice[Ecran]{"(-5)^3"/"-125"}

    \bigskip
    $\underbrace{\Calculatrice{/2,//$\hat{}$,/5}}_{\mbox{calculatrice}}$

    \smallskip
    \Calculatrice[Ecran]{"2^5"/"32"}

\end{methode}

\begin{propriete}[Opérations sur les puissances - \admise]
    Si $a$ est un entier relatif non nul et si $m$ et $n$ sont des entiers relatifs alors
    $$a^m\times a^n=a^{m+n}\text{;}\kern1cm\frac{a^m}{a^n}=a^{m-n}\text{;}\kern1cm\left(a^m\right)^n=a^{m\times n}\text{;}\kern1cm a^m\times b^m=(a\times b)^m$$
\end{propriete}


\begin{exemples*1}

    Les résultats précédents ne sont pas exigibles mais il faut savoir faire le genre de calculs suivants :
    \begin{enumerate}
        \item $A=\dfrac{(-9)^1}{(-9)^6}= \dfrac{\mathbf{\color{blue}{(-9)}}}{\mathbf{\color{red}{(-9)}} \times \mathbf{\color{red}{(-9)}}\times \mathbf{\color{red}{(-9)}}\times \mathbf{\color{red}{(-9)}}\times \mathbf{\color{red}{(-9)}}\times \mathbf{\color{red}{(-9)}}}$

        \medskip
        Il y a donc $\mathbf{\color{blue}{1}}$ simplifications par $(-9)$ possible.

        \medskip
        $A=\dfrac{\mathbf{\color{blue}{\cancel{(-9)}}}}{\mathbf{\color{red}{\cancel{(-9)}}}\times\mathbf{\color{red}{(-9)}} \times \mathbf{\color{red}{(-9)}}\times \mathbf{\color{red}{(-9)}}\times \mathbf{\color{red}{(-9)}}\times \mathbf{\color{red}{(-9)}}}$

        \medskip
        $A=\dfrac{1}{(-9)^{6-1}}=\dfrac{1}{(-9)^{5}}=\psshadowbox{(-9)^{-5}}$

        \medskip
        \item $B=((-6)^2)^{2}=\color{red}{\underbrace{\mathbf{\color{red}{((-6)^2)}} \times \mathbf{\color{red}{((-6)^2)}}}_{2\thickspace\text{facteurs}}}$

        \medskip
        $B=\color{red}{\underbrace{\mathbf{\color{red}{(\color{blue}{\underbrace{\mathbf{\color{blue}{(-6)}} \times \mathbf{\color{blue}{(-6)}}}_{2\thickspace\text{facteurs}}}\color{red})}} \times \mathbf{\color{red}{(\color{blue}{\underbrace{\mathbf{\color{blue}{(-6)}} \times \mathbf{\color{blue}{(-6)}}}_{2\thickspace\text{facteurs}}}\color{red})}}}_{2\times\color{blue}{2}\thickspace\color{black}{\text{facteurs}}}}$

        \medskip
        Il y a donc $\mathbf{\color{red}{2}~\color{black}{\times}~\color{blue}{2}}$ facteurs tous égaux à $(-6)$

        \medskip
        $B=(-6)^{2\times2} = (-6)^{4}= \psshadowbox{ 6^{4}}$

        \medskip
        \item $C=4^4\times 4^1=\mathbf{\color{red}{4}} \times \mathbf{\color{red}{4}}\times \mathbf{\color{red}{4}}\times \mathbf{\color{red}{4}} \times \mathbf{\color{blue}{4}}$

        \medskip
        Il y a donc $\mathbf{\color{red}{4}~\color{black}{+}~\color{blue}{1}}$ facteurs tous égaux à $4$

        \medskip
        $C=4^{4+1} = \psshadowbox{4^{5}}$

        \medskip
        \item $D=5^5\times 8^5=\mathbf{\color{red}{5}} \times \mathbf{\color{red}{5}}\times \mathbf{\color{red}{5}}\times \mathbf{\color{red}{5}}\times \mathbf{\color{red}{5}} \times \mathbf{\color{blue}{8}} \times \mathbf{\color{blue}{8}}\times \mathbf{\color{blue}{8}}\times \mathbf{\color{blue}{8}}\times \mathbf{\color{blue}{8}}$

        \medskip
        $D=\mathbf{(\color{red}{5}} \times \mathbf{\color{blue}{8}}) \times (\mathbf{\color{red}{5}} \times \mathbf{\color{blue}{8}})\times (\mathbf{\color{red}{5}} \times \mathbf{\color{blue}{8}})\times (\mathbf{\color{red}{5}} \times \mathbf{\color{blue}{8}})\times (\mathbf{\color{red}{5}} \times \mathbf{\color{blue}{8}})$

        \medskip
        $D= (\color{red}{\mathbf{5}} \color{black}{\times} \color{blue}{\mathbf{8}}\color{black}{)^{5}}=\psshadowbox{40^5}$

        \medskip
        \item $E=((-2)^2)^{4}=\color{red}{\underbrace{\mathbf{\color{red}{((-2)^2)}} \times \mathbf{\color{red}{((-2)^2)}}\times \mathbf{\color{red}{((-2)^2)}}\times \mathbf{\color{red}{((-2)^2)}}}_{4\thickspace\text{facteurs}}}$

        \medskip
        $E=\color{red}{\underbrace{\mathbf{\color{red}{(\color{blue}{\underbrace{\mathbf{\color{blue}{(-2)}} \times \mathbf{\color{blue}{(-2)}}}_{2\thickspace\text{facteurs}}}\color{red})}} \times \mathbf{\color{red}{(\color{blue}{\underbrace{\mathbf{\color{blue}{(-2)}} \times \mathbf{\color{blue}{(-2)}}}_{2\thickspace\text{facteurs}}}\color{red})}}\times \mathbf{\color{red}{(\color{blue}{\underbrace{\mathbf{\color{blue}{(-2)}} \times \mathbf{\color{blue}{(-2)}}}_{2\thickspace\text{facteurs}}}\color{red})}}\times \mathbf{\color{red}{(\color{blue}{\underbrace{\mathbf{\color{blue}{(-2)}} \times \mathbf{\color{blue}{(-2)}}}_{2\thickspace\text{facteurs}}}\color{red})}}}_{4\times\color{blue}{2}\thickspace\color{black}{\text{facteurs}}}}$

        \medskip
        Il y a donc $\mathbf{\color{red}{4}~\color{black}{\times}~\color{blue}{2}}$ facteurs tous égaux à $(-2)$

        \medskip
        $E=(-2)^{2\times4} = (-2)^{8}= \psshadowbox{ 2^{8}}$
    \end{enumerate}
\end{exemples*1}