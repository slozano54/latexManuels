\begin{exercice*}
    La masse d’un atome de cuivre est de \Masse{1.05d{-30}}.

    Déterminer le nombre d'atomes de cuivre se trouvant dans dans \Masse[kg]{1.47} de cuivre.
\end{exercice*}
\begin{corrige}
    %\setcounter{partie}{0} % Pour s'assurer que le compteur de \partie est à zéro dans les corrigés
    %\phantom{rrr}        
    La masse d’un atome de cuivre est de \Masse{1.05d{-30}}.

    Déterminer le nombre d'atomes de cuivre se trouvant dans dans \Masse[kg]{1.47} de cuivre.

    {\red \Masse[kg]{1.47} = \Masse{1.47d3} donc $\dfrac{\num{1.47}\times 10^3}{\num{1.05}\times 10^{-30}}\approx \num{1.4}\times 10^{33}$
    
    Il y a environ $\num{1.4}\times 10^{33}$ atomes de cuivre dans $\Masse[kg]{1.47}$.}
\end{corrige}

