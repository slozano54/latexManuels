\begin{exercice*}    
    Écrire sous la forme d'une puissance d'un nombre.
    \begin{multicols}{2}
        \begin{spacing}{2}
            $A=\dfrac{5^{-4}}{5^2}$

            $B=\dfrac{(-4)^{-2}}{(-4)^{-6}}$

            \columnbreak
            $C=\dfrac{2^{-5}}{2^{-3}}$

            $D=\dfrac{(-5)^{3}}{(-5)^{-2}}$
        \end{spacing}
    \end{multicols}
    \hrefMathalea{https://coopmaths.fr/alea/?uuid=bae57&id=4C33-1&n=4&s=2&cd=1&v=eleve&title=Exercices&es=1211}
\end{exercice*}
\begin{corrige}
    %\setcounter{partie}{0} % Pour s'assurer que le compteur de \partie est à zéro dans les corrigés
    %\phantom{rrr}    
    Écrire sous la forme d'une puissance d'un nombre.
        \begin{spacing}{2}
            $A=\dfrac{5^{-4}}{5^2}          {\red = 5^{-4-2} = 5^{-6}}$
            $B=\dfrac{(-4)^{-2}}{(-4)^{-6}} {\red = (-4)^{-2-(-6)} = (-4)^{-2+6} = (-4)^4}$            
            $C=\dfrac{2^{-5}}{2^{-3}}       {\red = 2^{-5-(-3)} = 2^{-5+3} = 2^{-2}}$            
            $D=\dfrac{(-5)^{3}}{(-5)^{-2}}  {\red = (-5)^{3-(-2)} = (-5)^{3+2} = (-5)^5}$
        \end{spacing}
\end{corrige}

