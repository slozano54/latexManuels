\begin{exercice*}
    Écrire sous la forme $10^n$.
    \begin{multicols}{2}
        \begin{enumerate}
            \begin{spacing}{1.5}
                \item $A=(10^3)^{3}$
                \item $B=10^6\times 10^2$
                \item $C=\dfrac{10^1}{10^2}$
                \item $D=10^7\times 10^4$
                \item $E=(10^2)^{4}$
                \item $F=\dfrac{10^5}{10^4}$
            \end{spacing}
        \end{enumerate}
    \end{multicols}
    \hrefMathalea{https://coopmaths.fr/alea/?uuid=1d078&id=4C33-0&n=4&s=1&s2=3&v=eleve&title=Exercices&es=1211}
\end{exercice*}
\begin{corrige}
    %\setcounter{partie}{0} % Pour s'assurer que le compteur de \partie est à zéro dans les corrigés
    %\phantom{rrr}    
    
    \begin{enumerate}        
            \item $A=(10^3)^{3}$\\
            $A=\color{red}{\underbrace{\mathbf{\color{red}{(10^3)}} \times \mathbf{\color{red}{(10^3)}}\times \mathbf{\color{red}{(10^3)}}}_{3\thickspace\text{facteurs}}}$\\
            $A=\color{red}{\underbrace{\mathbf{\color{red}{(\color{blue}{\underbrace{\mathbf{\color{blue}{10}} \times \mathbf{\color{blue}{10}}\times \mathbf{\color{blue}{10}}}_{3\thickspace\text{facteurs}}}\color{red})}} \times \mathbf{\color{red}{(\color{blue}{\underbrace{\mathbf{\color{blue}{10}} \times \mathbf{\color{blue}{10}}\times \mathbf{\color{blue}{10}}}_{3\thickspace\text{facteurs}}}\color{red})}}\times \mathbf{\color{red}{(\color{blue}{\underbrace{\mathbf{\color{blue}{10}} \times \mathbf{\color{blue}{10}}\times \mathbf{\color{blue}{10}}}_{3\thickspace\text{facteurs}}}\color{red})}}}_{3\times\color{blue}{3}\thickspace\color{black}{\text{facteurs}}}}$\\
            Il y a donc $\mathbf{\color{red}{3}~\color{black}{\times}~\color{blue}{3}}$ facteurs tous égaux à $10$.\\
            $A=10^{3\times3} = 10^{9}$
            \item $B=10^6\times 10^2$\\
            $B=\mathbf{\color{red}{10}} \times \mathbf{\color{red}{10}}\times \mathbf{\color{red}{10}}\times \mathbf{\color{red}{10}}\times \mathbf{\color{red}{10}}\times \mathbf{\color{red}{10}} \times \mathbf{\color{blue}{10}} \times \mathbf{\color{blue}{10}}$\\
            Il y a donc $\mathbf{\color{red}{6}~\color{black}{+}~\color{blue}{2}}$ facteurs tous égaux à $10$.\\
            $B=10^{6+2} = 10^{8}$     
            \end{enumerate}
            \Coupe
            \begin{enumerate}
                \setcounter{enumi}{2}         
            \item $C=\dfrac{10^1}{10^2}$
            
            \medskip
            $C=\dfrac{\mathbf{\color{blue}{10}}}{\mathbf{\color{red}{10}} \times \mathbf{\color{red}{10}}}$
            
            \medskip
            Il y a donc $\mathbf{\color{blue}{1}}$ simplifications par $10$ possibles.
            
            \medskip
            $C=\dfrac{\mathbf{\color{blue}{\cancel{10}}}}{\mathbf{\color{red}{\cancel{10}}}\times\mathbf{\color{red}{10}}}$
            
            \medskip
            $C=\dfrac{1}{10^{2-1}}=\dfrac{1}{10^{1}}=10^{-1}$            
            \item $D=10^7\times 10^4$\\
            \hspace*{-5mm}$D=\mathbf{\color{red}{10}} \times \mathbf{\color{red}{10}}\times \mathbf{\color{red}{10}}\times \mathbf{\color{red}{10}}\times \mathbf{\color{red}{10}}\times \mathbf{\color{red}{10}}\times \mathbf{\color{red}{10}} \times \mathbf{\color{blue}{10}} \times \mathbf{\color{blue}{10}}\times \mathbf{\color{blue}{10}}\times \mathbf{\color{blue}{10}}$\\
            Il y a donc $\mathbf{\color{red}{7}~\color{black}{+}~\color{blue}{4}}$ facteurs tous égaux à $10$.\\
            $D=10^{7+4} = 10^{11}$
            \item $E=(10^2)^{4}$\\
            $E=\color{red}{\underbrace{\mathbf{\color{red}{(10^2)}} \times \mathbf{\color{red}{(10^2)}}\times \mathbf{\color{red}{(10^2)}}\times \mathbf{\color{red}{(10^2)}}}_{4\thickspace\text{facteurs}}}$\\
            $E=\color{red}{\underbrace{\mathbf{\color{red}{(\color{blue}{\underbrace{\mathbf{\color{blue}{10}} \times \mathbf{\color{blue}{10}}}_{2\thickspace\text{facteurs}}}\color{red})}} \times \mathbf{\color{red}{(\color{blue}{\underbrace{\mathbf{\color{blue}{10}} \times \mathbf{\color{blue}{10}}}_{2\thickspace\text{facteurs}}}\color{red})}}\times \mathbf{\color{red}{(\color{blue}{\underbrace{\mathbf{\color{blue}{10}} \times \mathbf{\color{blue}{10}}}_{2\thickspace\text{facteurs}}}\color{red})}}\times \mathbf{\color{red}{(\color{blue}{\underbrace{\mathbf{\color{blue}{10}} \times \mathbf{\color{blue}{10}}}_{2\thickspace\text{facteurs}}}\color{red})}}}_{4\times\color{blue}{2}\thickspace\color{black}{\text{facteurs}}}}$\\
            Il y a donc $\mathbf{\color{red}{4}~\color{black}{\times}~\color{blue}{2}}$ facteurs tous égaux à $10$.\\
            $E=10^{2\times4} = 10^{8}$
            \item $F=\dfrac{10^5}{10^4}$
            
            \medskip
            $F=\dfrac{\mathbf{\color{red}{10}} \times \mathbf{\color{red}{10}}\times \mathbf{\color{red}{10}}\times \mathbf{\color{red}{10}}\times \mathbf{\color{red}{10}}}{\mathbf{\color{blue}{10}} \times \mathbf{\color{blue}{10}}\times \mathbf{\color{blue}{10}}\times \mathbf{\color{blue}{10}}}$
            
            \medskip
            Il y a donc $\mathbf{\color{blue}{4}}$ simplifications par $10$ possibles.
            
            \medskip
            $F=\dfrac{\mathbf{\color{red}{\cancel{10}}} \times \mathbf{\color{red}{\cancel{10}}}\times \mathbf{\color{red}{\cancel{10}}}\times \mathbf{\color{red}{\cancel{10}}}\times\mathbf{\color{red}{10}}}{\mathbf{\color{blue}{\cancel{10}}} \times \mathbf{\color{blue}{\cancel{10}}}\times \mathbf{\color{blue}{\cancel{10}}}\times \mathbf{\color{blue}{\cancel{10}}}}$
            
            \medskip
            $F=10^{5-4}=10^{1}$        
    \end{enumerate}

\end{corrige}

