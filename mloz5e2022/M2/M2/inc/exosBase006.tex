\begin{exercice*} %6
   Calculer l'aire et le périmètre de ces figures. \\
   {\psset{unit=0.45}
   \small
   \begin{pspicture}(-1,0)(13,6)
      \psframe(1,1)(7,5)
      \psframe(1,1)(1.4,1.4)
      \rput(4,1){/\!\!/}
      \rput(4,5){/\!\!/}
      \rput(1,3){$\times$}
      \rput(7,3){$\times$}
      \rput(4,0.2){\Lg[m]{8}}
      \rput{90}(0.2,2.7){\Lg[m]{4,5}}
      \rput(4,3){A}
      
      \psframe(10,1)(13,4)
      \psframe(10,1)(10.4,1.4)
      \rput(11.5,1){$\bullet$}
      \rput(11.5,4){$\bullet$}
      \rput(10,2.5){$\bullet$}
      \rput(13,2.5){$\bullet$}
      \rput(11.5,0.2){\Lg[mm]{12,3}}
      \rput(11.5,2.5){B}
   \end{pspicture} \\

   \begin{pspicture}(-1,0)(13,5)
      \pspolygon(1,1)(7,1)(2,5)
      \psframe(2,1)(2.3,1.3)
      \psline(2,1)(2,5)
      \rput{-40}(5.2,3.5){\Lg[cm]{6,9}}
      \rput{78}(0.8,3.2){\Lg[cm]{4,6}}
      \rput(3.5,0){\Lg[cm]{6,6}}
      \rput{90}(2.5,2.7){\Lg[cm]{4,4}}
      \rput(3.8,2.2){C}
      
      \pspolygon(9,1)(14,1)(14,6)
      \psframe(14,1)(13.7,1.3)
      \rput(11.5,0){\Lg[km]{8,5}}
      \rput{45}(11.3,4){\Lg[km]{12,5}}
      \rput(11.5,1){|\!\!|}
      \rput(14,3.5){=}
      \rput(12.5,2.5){D}
   \end{pspicture} \\
   
   \begin{pspicture}(-1,1)(13,5.5)
      \pscircle(3,3){2.5}
      \psline(3,3)(5.5,3)
      \rput(4.25,2.5){\Lg[dm]{1,5}}
      \rput(3,3.75){E}
      \pscircle(12,3){2}
      \psline(10,3)(14,3)
      \rput(12,2.5){\Lg[m]{5,6}}
      \rput(12,3.5){F}
   \end{pspicture}}
\end{exercice*}

\begin{corrige}
   \textcolor{G1}{$\bullet$} Figure A : on a un rectangle de côtés \Lg[m]{8} et \Lg[m]{4,5}. \\
      $\mathcal{P} =2\times(L+\ell) =2\times(\Lg[m]{8}+\Lg[m]{4,5}) =\red\Lg[m]{25}$. \\
      $\mathcal{A} =L\times\ell =\Lg[m]{8}\times\Lg[m]{4,5} =\red\Aire[m]{36}$. \\
   \textcolor{G1}{$\bullet$} Figure B : on a un carré de côté \Lg[mm]{12,3}. \\
      $\mathcal{P} =4\times c =4\times\Lg[mm]{12,3} =\red\Lg[mm]{49,2}$. \\ 
      $\mathcal{A} =c\times c =\Lg[mm]{12,3}\times\Lg[mm]{12,3} =\red\Aire[mm]{151,29}$. \\ 
   \textcolor{G1}{$\bullet$} Figure C : on a un triangle de base \Lg[cm]{6,6} et de hauteur associée \Lg[cm]{4,4}. \\
      $\mathcal{P} =\Lg[cm]{6,6}+\Lg[cm]{6,9}+\Lg[cm]{4,6} =\red \Lg[cm]{18,1}$. \\ [1mm]
      $\mathcal{A} =\dfrac{b\times h}{2} =\dfrac{\Lg[cm]{6,6}\times\Lg[cm]{4,4}}{2}=\red \Aire[cm]{14,52}$. \\ [1mm]
   \textcolor{G1}{$\bullet$} Figure D : on a un triangle de base \Lg[km]{8,5} et de hauteur associée \Lg[km]{8,5}. \\
      $\mathcal{P} =2\times\Lg[km]{8,5}+\Lg[km]{12} =\red \Lg[km]{29}$. \\ [1mm]
      $\mathcal{A} =\dfrac{b\times h}{2}  =\dfrac{\Lg[km]{8,5}\times\Lg[km]{8,5}}{2}=\red \Aire[km]{36,125}$. \\ [1mm]
   \textcolor{G1}{$\bullet$}  Figure E : on a un disque de rayon \Lg[dm]{1,5}. \\
      $\mathcal{P} =2\times\pi\times r =2\times\pi\times\Lg[dm]{1,5} \approx\red \Lg[dm]{9,42}$. \\
      $\mathcal{A} =\pi\times r^2 =\pi\times(\Lg[dm]{1,5})^2 \approx\red \Aire[dm]{7,07}$. \\
   \textcolor{G1}{$\bullet$}  Figure F : on a un disque de rayon \Lg[m]{2,8}. \\
      $\mathcal{P} =2\times\pi\times r =2\times\pi\times\Lg[m]{2,8} \approx\red \Lg[m]{17,59}$. \\
      $\mathcal{A} =\pi\times r^2 =\pi\times(\Lg[m]{2,8})^2 \approx\red \Aire[m]{24,63}$. \\       
\end{corrige}
