\begin{exercice*} %7
   L'unité d'aire est le carré.
   \begin{enumerate}
      \item On peut regrouper les surfaces ci-dessous par deux ou par trois sauf une, laquelle ?
      \item Calculer alors l'aire de chaque surface.
   \end{enumerate}
   \begin{center}
      {\psset{unit=0.65}
      \small
      \begin{pspicture}(0,4.5)(11,12.8)
         \psgrid[subgriddiv=0,gridlabels=0,gridcolor=lightgray](0,4)(11,13)
         \psset{fillstyle=solid,fillcolor=lightgray}
         
         \pscustom{\psarc(2,11){1}{0}{180} \psarcn(1,10){1}{90}{0} \psarcn(3,10){1}{180}{90}}
         \rput(2,11.25){\small A}
         \pscustom{\psarc(5,11){1}{0}{180} \psline(4,11)(4,10) \psarc(4,11){1}{270}{0} \psarc(6,11){1}{180}{270} \psline(6,10)(6,11)}
         \rput(5,11.5){\small B}
         \pscircle(8,11){1}
         \rput(8,11){\small C}
         
         \pscustom{\psarc(2,8){1}{90}{180} \psarcn(1,7){1}{90}{0} \psarc(2,8){1}{270}{0} \psarcn(3,9){1}{270}{180}}
         \rput(2,8){\small D}
         \pscustom{\psarc(5,8){1}{180}{0} \psline(6,8)(7,8) \psarc(6,8){1}{0}{180} \psline(5,8)(4,8)}
         \rput(5.5,8){\small E}
         \pscustom{\psarcn(8,7){1}{90}{0} \psarcn(10,7){1}{180}{90} \psarcn(10,9){1}{270}{180} \psarcn(8,9){1}{0}{270}}
         \rput(9,8){\small F}  
         
         \pscustom{\psline(1,5)(1,6)(3,6) \psarcn(3,5){1}{90}{0} \psline(4,5)(2,5)(2,4) \psarc(1,4){1}{0}{90}}
         \rput(2.5,5.5){\small G}
         \pscustom{\psarc(6,5){1}{90}{180} \psarcn(5,4){1}{90}{0} \psline(6,4)(6,5)(7,5)(7,4) \psarcn(8,4){1}{180}{90} \psarc(7,5){1}{0}{90} \psline(7,6)(6,6)}
         \rput(6.5,5.5){\small H}
           
      \end{pspicture}}
   \end{center}
\end{exercice*}

\begin{corrige}
\ \\ [-5mm]
   \begin{enumerate}
      \item On peut regrouper A et D ; B, C et E ; G et H. \\
         {\red La surface Fest seule}. \smallskip
      \item \textcolor{G1}{$\bullet$} Les figures A et D peuvent être vues comme un rectangle de mesures $2u$ et $u$, leur aire vaut {\red $2u^2$}. \\
         \textcolor{G1}{$\bullet$} Les figures B, C et E peuvent être recomposées en un disque de rayon $u$, leur aire veut {\red $\pi u^2$}. \\
         \textcolor{G1}{$\bullet$} Les figures G et H peuvent être recomposées en un rectangle de mesures $3u$ et $u$, leur aire veut {\red $3u^2$}. \\
         \textcolor{G1}{$\bullet$} La figure F peut être vue comme un carré de côté $2u$ auquel on enlève quatre quarts de disque, soit un disque de rayon $u$. $\mathcal{A} =\red 4u^2-\pi u^2$.
   \end{enumerate}
\end{corrige}
