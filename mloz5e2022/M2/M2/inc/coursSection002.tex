\section{Conversions}

On peut mesurer une longueur grâce au mètre (m) et ses multiples et sous-multiples :

\Tableau[Metre,FlechesH]{7321/3}
\vspace*{-8mm}

\begin{exemple*1}
   $\Lg[km]{0,07321} =\Lg[dam]{7,321} =\Lg[m]{73,21}=\Lg[dm]{732,1} =\Lg[cm]{7321} =\Lg[mm]{73210}$. 
\end{exemple*1}

\clearpage

L'aire est une grandeur composée, correspondant au produit de deux longueurs. Chaque unité d'aire dans le tableau comporte donc deux colonnes. Pour désigner une aire, on utilise le mètre carré (\Aire[m]{}) et ses multiples et sous-multiples. \\
Pour les mesures agraires, on utilise l'are (a) qui équivaut à \Aire[m]{100} et l'hectare (ha) qui vaut 100 ares. \smallskip

\Tableau[Carre,Colonnes,Are,FlechesH]{3701/4}
\vspace*{-8mm}

\begin{exemple*1}
   \Aire[km]{0,03701} = \Aire[ha]{3,701} = \Aire[hm]{3,701} = \Aire[a]{370,1} = \Aire[m]{37010} = \Aire[dm]{3701000}\dots
\end{exemple*1}