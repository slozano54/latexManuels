\begin{changemargin}{-10mm}{-10mm}
\begin{activite}[Une aire multiforme]
    {\bf Objectifs :} différencier périmètre et aire ; comparer des périmètres et des aires sans utiliser la mesure.
 
    Le curvica est un jeu-puzzle de 24 pièces inventé par {\bf Jean Fromentin} en 1982 afin de travailler sur les notions de périmètre, d’aire et de symétrie, entre autres ! À partir d’un carré, on obtient une pièce du puzzle curvica en \og creusant \fg, en \og bombant \fg ou en laissant droits les côtés. Les arcs produits sont tous identiques.
       
       \begin{center}
        \scalebox{0.8}{
          \begin{pspicture}(0,0)(12,8)
             \psline(0,8)(0,0)(12,0)
             \curvicaMDeux{1}{1}{\hm}{\vg}{K} \curvicaMDeux{3}{1}{\hm}{\vd}{T} \curvicaMDeux{5}{1}{\hb}{\vd}{W} \curvicaMDeux{7}{1}{\hh}{\vg}{V} \curvicaMDeux{9}{1}{\hb}{\vd}{N} \curvicaMDeux{11}{1}{\hb}{\vm}{E} %ligne 1
             \curvicaMDeux{1}{3}{\hb}{\vd}{H} \curvicaMDeux{3}{3}{\hm}{\vd}{F} \curvicaMDeux{5}{3}{\hb}{\vg}{R} \curvicaMDeux{7}{3}{\hb}{\vd}{L} \curvicaMDeux{9}{3}{\hh}{\vd}{C} \curvicaMDeux{11}{3}{\hb}{\vm}{X} %ligne 2
             \curvicaMDeux{1}{5}{\hm}{\vd}{U} \curvicaMDeux{3}{5}{\hm}{\vg}{D} \curvicaMDeux{5}{5}{\hh}{\vd}{A} \curvicaMDeux{7}{5}{\hb}{\vd}{M} \curvicaMDeux{9}{5}{\hb}{\vg}{Q} \curvicaMDeux{11}{5}{\hb}{\vm}{O} %ligne 3
             \curvicaMDeux{1}{7}{\hm}{\vm}{I} \curvicaMDeux{3}{7}{\hm}{\vd}{J} \curvicaMDeux{5}{7}{\hm}{\vg}{S} \curvicaMDeux{7}{7}{\hm}{\vg}{P} \curvicaMDeux{9}{7}{\hm}{\vd}{B} \curvicaMDeux{11}{7}{\hm}{\vm}{G} %ligne 4
          \end{pspicture}
        }
       \end{center}
       \vspace*{-10mm}
       
       \begin{spacing}{1.5}
       \begin{enumerate}
          \item On s’intéresse uniquement aux pièces A, B, C, D, E et F que l'on pourra colorier.
             \begin{enumerate}
                \item Classer ces six pièces du plus petit au plus grand périmètre. \pointilles
                \item Classer ces six pièces de la plus petite à la plus grande aire. \pointilles
                \item Quelles sont les deux pièces qui ont la même aire et le même périmètre ? \pointilles
                \item Trouver deux pièces qui ont le même périmètre, mais des aires différentes. \pointilles 
             \end{enumerate}
          \item On s'intéresse maintenant à toutes les pièces.
             \begin{enumerate}
                \item Y a-t-il une unique pièce d’aire minimale ? \pointilles
                \item Y a-t-il une unique pièce de périmètre minimum ? \pointilles             
                \item Y a-t-il une unique pièce d’aire maximale ? \pointilles           
                \item Y a-t-il une unique pièce de périmètre maximum ? \pointilles         
                \item Y a-t-il une ou des pièces de même aire que la carré I ? \pointilles
                \item Y a-t-il une ou des pièces de même périmètre que la carré I ? \pointilles \\ [-15mm]
             \end{enumerate}
       \end{enumerate}
       \end{spacing}
 \end{activite}
 
 \vfill\hfill{\small\it Source : \href{https://irem.univ-reunion.fr/spip.php?article802}{Curvica - activités mathématiques ludiques}, Yves Martin, IREM de la Réunion.}
\end{changemargin}