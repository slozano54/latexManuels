% Les enigmes ne sont pas numérotées par défaut donc il faut ajouter manuellement la numérotation
% si on veut mettre plusieurs enigmes
% \refstepcounter{exercice}
% \numeroteEnigme
\begin{enigme}[Tel père, tel fils]

   \medskip
   
   C’est l’histoire d’un petit rectangle de dimensions $\Lg[mm]{2}\times\Lg[mm]{3}$. \\
      Chaque jour, il s’agrandit pour devenir un rectangle plus grand : sa nouvelle largeur est égale à son ancienne longueur ; sa nouvelle longueur est égale à la somme de ses deux anciennes dimensions. \\
      
      {\bf Au bout de combien de jours son aire dépassera-t-elle \Aire[m]{1,5} ?} \\
   
      \begin{center}
         \includegraphics[width=10cm]{\currentpath/images/fantome}
      \end{center}
      
      Par groupes, vous effectuerez les recherches, puis rendrez une fiche récapitulative où figureront votre raisonnement, la modélisation et les calculs effectués.
        
   \end{enigme}
   
   \vfill \hfill {\it\small D'après \og La résolution de problèmes mathématiques au collège \fg, MENJS, 2021}
   

% Pour le corrigé, il faut décrémenter le compteur, sinon il est incrémenté deux fois
% \addtocounter{exercice}{-1}
\begin{corrige}
   On peut, par exemple, calculer les aires une par une jusqu'à obtenir $\Aire[m]{1,5} = \Aire[mm]{1500000}$. \\
   Le tableau ci-dessous récapitule la largeur et la longueur obtenue par chaque rectangle en mm ainsi que son aire en \Aire[mm]. \medskip
   
   \begin{tabular}{|c|c|c|c|}
      \hline
      Jour & Largeur & Longueur & Aire \\
      \hline
      0 & 2 & 3 & 6 \\
      \hline
      1 & 3 & 5 & 15 \\
      \hline
      2 & 5 & 8 & 40 \\
      \hline
      3 & 8 & 13 & 104 \\
      \hline
      4 & 13 & 22 & 286 \\
      \hline
      5 & 22 & 35 & 770 \\
      \hline
      6 & 35 & 57 & 1 995 \\
      \hline
      7 & 57 & 94 & 5 358 \\
      \hline
      8 & 94 & 151 & 14 194 \\
      \hline
      9 & 151 & 245 & 36 995 \\
      \hline
      10 & 245 & 396 & 97 020 \\
      \hline
      11 & 396 & 641 & 253 836 \\
      \hline
      12 & 641 & 1 037 & 664 717 \\
      \hline
      13 & 1 037 & 1 678 & 1 740 086 \\
      \hline
   \end{tabular}
   \medskip
   
   {\red Il faudra 13 jours pour que l'aire du rectangle dépasse \Aire[m]{1,5}.}
\end{corrige}