\begin{exercice*} %8
   Dans cette figure, on a : \\
   \begin{minipage}{2cm}
      $AB =\Lg[cm]{9}$ \\
      $BC =\Lg[cm]{21}$ \\
      $CD =\Lg[cm]{11}$ \\ 
      $DE =\Lg[cm]{9}$ \\
      $EF =\Lg[cm]{11}$ \\
      $GH =\Lg[cm]{7}$
   \end{minipage}
   \qquad
   \begin{minipage}{5cm}
   {\psset{unit=1.3}
      \small
      \begin{pspicture}(-0.5,-0.3)(3.5,2.5)
         \psframe(0,0)(3,2)
         \pspolygon(2,0)(3,1)(1,2)(0,0.8)
         \rput(-0.2,-0.2){$G$}
         \rput(-0.2,2.2){$A$}
         \rput(3.2,-0.2){$E$}
         \rput(3.2,2.2){$C$}
         \rput(2,-0.2){$F$}
         \rput(3.2,1){$D$}
         \rput(1,2.2){$B$}
         \rput(-0.2,0.8){$H$}
      \end{pspicture}}
   \end{minipage} \\
   \begin{enumerate}
      \item Calculer le périmètre du rectangle $ACEG$.
      \item Calculer l'aire du quadrilatère $BDFH$.
   \end{enumerate}
\end{exercice*}

\begin{corrige}
   \ \\ [-5mm]
   \begin{enumerate}
      \item $\mathcal{P}_{ACEG} =2\times(AC+CE)$. \\
         Or, $AC =AB+BC =\Lg[cm]{9}+\Lg[cm]{21} =\Lg[cm]{30}$ ;\\
         et $CE =CD+DE =\Lg[cm]{11}+\Lg[cm]{9} =\Lg[cm]{20}$. \\
         D'où $\mathcal{P}_{ACEG} =2\times(\Lg[cm]{30}+\Lg[cm]{20}) =\Lg[cm]{100}$. \\
         {\red Le périmètre du rectangle $ACEG$ est de \Lg[cm]{100}}.
      \item Pour calculer l'aire du quadrilatère $BDFH$, on calcule l'aire du rectangle $ACEG$ auquel on soustrait l'aire de chacun des triangles rectangles $ABH$, $BCD$, $DEF$ et $FGH$. \\
      $\mathcal{A}_{ACEG} =AC\times CE =\Lg[cm]{30}\times\Lg[cm]{20} =\Aire[cm]{600}$. \\ [1.5mm]
      $\mathcal{A}_{ABH} =\dfrac{HA\times AB}{2} =\dfrac{\Lg[cm]{13}\times\Lg[cm]{9}}{2} =\Aire[cm]{58,5}$. \\ [1.5mm]
      $\mathcal{A}_{BCD} =\dfrac{BC\times CD}{2} =\dfrac{\Lg[cm]{21}\times\Lg[cm]{11}}{2} =\Aire[cm]{115,5}$. \\ [1.5mm]
      $\mathcal{A}_{DEF} =\dfrac{DE\times EF}{2} =\dfrac{\Lg[cm]{9}\times\Lg[cm]{11}}{2} =\Aire[cm]{49,5}$. \\ [1mm]
      $\mathcal{A}_{FGH} = \dfrac{FG\times GH}{2} =\dfrac{\Lg[cm]{19}\times\Lg[cm]{7}}{2} =\Aire[cm]{66,5}$. \\ [1.5mm]
      $\mathcal{A}_{BDFH} =\Aire[cm]{600}-\Aire[cm]{58,5}-\Aire[cm]{115,5}$ \\
      \hspace*{14mm} $-\Aire[cm]{49,5}-\Aire[cm]{66,5} =\Aire[cm]{310}$. \\
      {\red L'aire du quadrilatère $BDFH$ est de \Aire[cm]{310}}.
   \end{enumerate}
\end{corrige}