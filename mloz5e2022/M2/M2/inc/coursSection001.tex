\section{Périmètres et aires usuelles}

$\bullet$ Le périmètre d'un polygone est la somme des longueurs des segments qui le compose ; \\
$\bullet$ la circonférence d'un cercle de rayon $r$ se calcule grâce à la formule $\mathcal{P} =2\times\pi\times r$. \medskip

Pour les aires, on a les formules suivantes :

\begin{minipage}[t]{5cm}
   \Formule[Aire,Surface=rectangle,Ancre={(2.5,-2)},Couleur=yellow!10]
\end{minipage}
\qquad
\begin{minipage}[t]{5cm}
   \Formule[Aire,Surface=triangle,Ancre={(2.5,-2)},Couleur=yellow!10]
\end{minipage}
\qquad
\begin{minipage}[t]{5cm}
   \Formule[Aire,Surface=disque,Ancre={(2.5,-2)},Couleur=yellow!10]
\end{minipage}

\vspace*{40mm}
\hspace*{-11mm}
\begin{minipage}[t]{5cm}
   \begin{exemple}[1.5]
      Aire d'un rectangle de longueur \Lg[dm]{0,3} et de largeur \Lg[dm]{0,2} : \\ [1mm]
      $\Lg[dm]{0,3}\times\Lg[dm]{0,2} =\Aire[dm]{0,06}$.
   \end{exemple}
\end{minipage}
\qquad
\begin{minipage}[t]{5cm}
   \begin{exemple}[1.5]
      Aire d'un triangle de base \Lg[mm]{28} et de hauteur relative \Lg[mm]{13} : \\ [2mm]
   $\dfrac{\Lg[mm]{28}\times\Lg[mm]{13}}{2} =\Aire[mm]{182}$.
   \end{exemple}
\end{minipage}
\qquad
\begin{minipage}[t]{5cm}
   \begin{exemple}[1.5]
      Aire et périmètre d'un disque de rayon \Lg[cm]{1,2} : \\
      $\mathcal{A} =\pi\times(\Lg[cm]{1,2})^2\approx\Aire[cm]{4,52}$. \\
      $\mathcal{P} =2\times\pi\times(\Lg[cm]{1,2})\approx\Lg[cm]{7,54}$.
   \end{exemple}
\end{minipage}

\begin{remarque}
   pour calculer l'aire d'une figure complexe, il suffit de la \og découper \fg{} en figures usuelles et d'additionner ou de soustraire les aires qui la constituent.
\end{remarque}
