\vspace*{-7mm}
\begin{changemargin}{-10mm}{-10mm}
%pre-001
\begin{prerequis}[Connaisances \emoji{red-heart} et compétences \emoji{diamond-suit} du cycle 3]    
   \begin{itemize}        
       \item[\emoji{red-heart}] Vocabulaire associé à ces objets et à leurs propriétés : côté, sommet, angle, hauteur.
       \columnbreak
       \item[\emoji{diamond-suit}] Reconnaître, nommer, décrire des triangles, dont les triangles particuliers (triangle rectangle, triangle isocèle, triangle équilatéral).       
   \end{itemize}
\end{prerequis}
\vspace*{-3mm}
%pre-002
\begin{prerequis}[Connaisances \emoji{red-heart} et compétences \emoji{diamond-suit} du cycle 4]    
    \begin{itemize}        
        \item[\emoji{diamond-suit}] Mener des calculs impliquant des grandeurs mesurables, exprimer les résultats dans des les unités adaptées.
        \item[\emoji{diamond-suit}] Exprimer et vérifier la cohérence des résultats du point de vue des unités.
    \end{itemize}
\end{prerequis}
\vspace*{-3mm}
\begin{debat}[Haïku]
    \vspace*{-7mm}
    Un haïku est un petit poème traditionnel japonais, de forme très concise (dix-sept syllabes en trois vers : 5-7-5 ou 7-5-5 ou 5-5-7). {\it Richard Cauche} est enseignant de mathématiques et membre de l’IREM de Paris. Il nous fait partager ces haïkus mathématiques sur le périmètre et l'aire du disque :
    \begin{center}
    \includegraphics[height=4cm]{\currentpath/images/circonference}
    \hspace{10mm}
    \includegraphics[height=4cm]{\currentpath/images/surface}
    \end{center}
    \begin{cadre}[B2][J4]
    \begin{center}
        \hrefVideo{https://www.yout-ube.com/watch?v=TcNfC8b4hUg}{\bf Archimède et le nombre $\pi$}, {\it m@ths et tiques}, Yvan Monka.
    \end{center}
    \end{cadre}
\end{debat}
\end{changemargin}