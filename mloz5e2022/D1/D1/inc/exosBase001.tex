\begin{exercice*}[Dés]
    On a réalisé $200$ lancers de $2$ dés à $8$ faces.\\
    Les résultats sont inscrits dans le tableau ci-dessous :
    
    \medskip
    \hspace*{-5mm}
    \begin{tabularx}{\linewidth}{|c|*{8}{>{\centering\arraybackslash}X|}}
        \hline  
        Scores                &2&3&4 &5 &6&7 &8 &9 \\\hline
        Nombre d'apparitions  &4&6&11&15&9&16&20&26\\\hline
    \end{tabularx}
   
    \medskip
    \hspace*{-5mm}
    \begin{tabularx}{\linewidth}{|c|*{7}{>{\centering\arraybackslash}X|}}
        \hline  
        Scores                &10&11&12&13&14&15&16\\\hline
        Nombre d'apparitions  &18&17&20&15&11&8 &4 \\\hline
    \end{tabularx}

    \medskip
     Calculer la fréquence de la valeur $9$.

    \hrefMathalea{https://coopmaths.fr/mathalea.html?ex=5S13,s=1,n=1,i=0&v=l} % On peut personnaliser le texte entre crochets si on veut sinon supprimer les crochets
\end{exercice*}
\begin{corrige}
    %\setcounter{partie}{0} % Pour s'assurer que le compteur de \partie est à zéro dans les corrigés
    %\phantom{rrr}
    On a réalisé $200$ lancers de $2$ dés à $8$ faces.\\
    Les résultats sont inscrits dans le tableau ci-dessous :
    
    \medskip
    \hspace*{-5mm}
    \begin{tabularx}{\linewidth}{|c|*{15}{>{\centering\arraybackslash}X|}}
        \hline  
        Scores                &2&3&4 &5 &6&7 &8 &9 &10&11&12&13&14&15&16\\\hline
        Nombre d'apparitions  &4&6&11&15&9&16&20&26&18&17&20&15&11&8 &4 \\\hline
    \end{tabularx}
   
    \medskip
     Calculer la fréquence de la valeur $9$.

     {\red        
        La valeur $9$ apparaît $26$ fois.\\
        Le nombre total de lancers est $200$.\\
        La fréquence de la valeur $9$ est $\dfrac{26}{200}=0{,}13$\\
        Soit $13\thickspace\%$.    
     }
\end{corrige}

