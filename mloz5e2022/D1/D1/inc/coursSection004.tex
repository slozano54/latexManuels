\section{Représentations graphiques}
\subsection{Tableaux}

On souhaite connaître le nombre d'enfants qui constitue une fratrie d'une classe de 5\up{e} d'un collège composée ce jour là de 25 élèves. On obtient les résultats suivants :
   \begin{center}
      1 ; 2 ; 5 ; 1 ; 2 ; 2 ; 3 ; 4 ; 1 ; 3 ; 2 ; 6 ; 2 ; 3 ; 4 ; 2 ; 6 ; 1 ; 3 ; 2 ; 1 ; 2 ; 3 ; 3 ; 2.
   \end{center}
   
On {\bf organise} les résultats : pour rassembler les données, on les présente sous forme d'un tableau où l'on regroupe ensemble les différentes valeurs obtenues.

L'{\bf effectif} d'une donnée est le nombre de fois qu'elle apparaît.
\begin{center}
   \Stat[Sondage,Tableau,Largeur=8mm,Donnee=Nombre d'enfants,Total]{1,2,5,1,2,2,3,4,1,3,2,6,2,3,4,2,6,1,3,2,1,2,3,3,2}
\end{center}

\subsection{Diagramme en bandes}

\subsection{Diagramme en barres ou diagramme bâtons}

\subsection{Diagramme circulaire ou camembert}

\subsection{Diagramme semi-circulaire}

\subsection{Graphique cartésien}