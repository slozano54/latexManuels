\section{Caractéristiques de position : Moyenne}
% \subsection{Moyenne arithmétique}
\begin{definition}[Moyenne arithmétique]
    La {\bf moyenne} d'une série statistique est le quotient de la somme de toutes les valeurs par l'effectif total de cette série :
    $${\text{moyenne} =\frac{\text{Somme de toutes les valeurs}}{\text{Effectif total de la série}}}$$
\end{definition}

\begin{exemple*1}
    Dans une usine, sept employés calculent le salaire moyen (en €) des salaires de leur atelier :
    \Prix[0]{760}; \Prix[0]{825}; \Prix[0]{915}; \Prix[0]{990}; \Prix[0]{1065}; \Prix[0]{1160}; \Prix[0]{1296}.
    \correction
    $$\frac{760 + 825 + 915 + 990 + 1065 + 1160 + 1296}{7}\approx 1002$$
    \psshadowbox{Le salaire moyen des  employés de cet atelier s'élève environ à \Prix[0]{1002}}
\end{exemple*1}

\begin{definition}[Moyenne pondérée]
    La {\bf moyenne pondérée} d'une série statistique est le quotient de la somme des  valeurs, affectées chacune de leur coefficient, par la somme totale des coefficients.
\end{definition}

\begin{exemple*1}
    Dans une classe de 28 élèves, les notes d'un devoir se répartissent ainsi :

    {\renewcommand{\arraystretch}{1.5}
    \begin{tabular}{|m{2.5cm}|c|c|c|c|c|c|c|c|c|c|c|}
    \cline{12-12}
    \multicolumn{11}{c|}{}&total\\
    \hline
    Notes&6&7&8&9&10&11&12&13&15&17&\begin{LARGE}$\otimes$\end{LARGE}\\
    \hline
    Effectifs& \color{red} 2&\color{red} 1&\color{red} 2&\color{red} 4&\color{red} 6&\color{red} 3&\color{red} 3&\color{red} 3&\color{red} 2&\color{red} 2&\color{red} 28\\
    \hline
    \end{tabular}
    }

    \smallskip
    Pour calculer la moyenne, on effectue le calcul suivant :
    
    \smallskip
    $M = \dfrac{\textcolor{red}{2}\times 6 + \textcolor{red}{1}\times 7 + \textcolor{red}{2}\times 8 + \textcolor{red}{4}\times 9 + \textcolor{red}{6}\times 10 + \textcolor{red}{3}\times 11 +
    \textcolor{red}{3}\times 12 + \textcolor{red}{3}\times 13 + \textcolor{red}{2}\times 15 + \textcolor{red}{2}\times 17}{{\red 2}+{\red 1}+{\red 2}+{\red 4}+{\red 6}+{\red 3}+{\red 3}+{\red 3}+{\red 2}+{\red 2}}$

    $M=\dfrac{303}{28}\approx10,8$
\end{exemple*1}

\begin{remarque}\titreRemarque{répartition en classes et moyenne}

    Dans certains cas, une série statistique peut être répartie en \textbf{classes} qui sont des intervalles de valeurs.
\end{remarque}

\begin{exemple*1}
    \titreExemple{Répartition de 2000 adultes suivant leur taille}

    \begin{longtable}{|>{\columncolor{gray!20}\centering}m{0.3\textwidth}|*{4}{>{\centering\arraybackslash}m{0.125\textwidth}|}}
        \hline
        \rowcolor{gray!20}Tailles en \Lg{} &$\left[140\, ; 150\right[$&$\left[150\, ; 160\right[$&$\left[160\, ; 170\right[$&$\left[170\, ; 195\right[$\\
        \hline
        Effectifs&48&397&913&642\\
        \hline
    \end{longtable}
    

Il est impossible, a priori, de calculer la moyenne de cette série puisqu'on ne connaît ni les valeurs des tailles et ni leurs effectifs.

On peut donc, par exemple, considèrer que la valeur au centre de la classe va représenter la classe.

On calcule alors la moyenne pondérée pour obtenir une valeur approchée de la moyenne de la série.

\begin{longtable}{|>{\columncolor{gray!20}\centering}m{0.3\textwidth}|*{4}{>{\centering\arraybackslash}m{0.125\textwidth}|}}
    \hline
    \rowcolor{gray!20}Tailles en \Lg{} &$\left[140\, ; 150\right[$&$\left[150\, ; 160\right[$&$\left[160\, ; 170\right[$&$\left[170\, ; 195\right[$\\
    \hline
    Centre de la classe&145&155&165&182,5\\
    \hline
    Effectifs&48&397&913&642\\
    \hline
\end{longtable}

Pour calculer la taille moyenne, on effectue le calcul suivant : 
$$\dfrac{48\times 145 + 397\times 155 + 913\times 165 + 642\times 145}{2000}\approx168$$

\psshadowbox{La taille moyenne de ce groupe d'adultes est d'environ \Lg{168}.}

\end{exemple*1}