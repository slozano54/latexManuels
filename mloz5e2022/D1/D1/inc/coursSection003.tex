\section{Caractéristiques de position : Moyenne}
\subsection{Moyenne arithmétique}
% \definNum{
% La {\bf moyenne} d'une s\'erie statistique est le quotient de la somme de toutes les valeurs par l'effectif total de cette s\'erie :
% $${\mbox{moyenne} =\frac{\mbox{Somme de toutes les valeurs}}{\mbox{Effectif total de la s\'erie}}}$$
% }

% \Exemples[Exemple]{}{Dans une usine, sept employ\'es calculent le salaire moyen (en {\textgreek{\euro}}) des salaires de leur atelier.
% $$\frac{760 + 825 + 915 + 990 + 1065 + 1160 + 1296}{7}\approx 1002$$

% \psshadowbox{Le salaire moyen des  employ\'es de cet atelier s'\'el\`{e}ve environ \`{a} 1002 {\textgreek{\euro}}.}
% }

\subsection{Moyenne pond\'er\'ee}

% \definNum{
% La {\bf moyenne pond\'er\'ee} d'une s\'erie statistique est le quotient de la somme des  valeurs, affect\'ees chacune de leur coefficient, par la somme totale des coefficients.
% }

% \Exemples[Exemple]{}{ Dans une classe de 28 \'el\`{e}ves, les notes \`{a} un devoir se r\'epartissent de la mani\`{e}re suivante :
% \begin{center}
% \renewcommand{\arraystretch}{1.5}
% \begin{tabular}{|m{2.5cm}|c|c|c|c|c|c|c|c|c|c|c|}
% \cline{12-12}
% \multicolumn{11}{c|}{}&total\\
% \hline
% Notes&6&7&8&9&10&11&12&13&15&17&\begin{LARGE}$\otimes$\end{LARGE}\\
% \hline
% Effectifs& \color{red} 2&\color{red} 1&\color{red} 2&\color{red} 4&\color{red} 6&\color{red} 3&\color{red} 3&\color{red} 3&\color{red} 2&\color{red} 2&\color{red} 28\\
% \hline
% \end{tabular}
% \renewcommand{\arraystretch}{1}
% \end{center}
% Nous repr\'esentons cette s\'erie par un diagramme en barres.
% \begin{center}
% \psscalebox{0.7}{%
% \begin{pspicture}(0,0)(13,8.5)
% \psline{->}(0,0)(13.5,0)
% \multido{\n=1+1}{7}{\psline[linecolor=red](0,\n)(13,\n)}
% \psline{->}(0,0)(0,8)
% \multido{\n=1+1}{7}{\uput{0.2}[180](0,\n){\textcolor{red}\n}}
% \uput{0.2}[270](1,0){6}\uput{0.2}[270](2,0){7}
% \uput{0.2}[270](3,0){8}\uput{0.2}[270](4,0){9}
% \uput{0.2}[270](5,0){10}\uput{0.2}[270](6,0){11}\uput{0.2}[270](7,0){12}
% \uput{0.2}[270](8,0){13}\uput{0.2}[270](9,0){14}\uput{0.2}[270](10,0){15}
% \uput{0.2}[270](11,0){16}\uput{0.2}[270](12,0){17}
% \psframe[fillstyle=solid,fillcolor=blue](0.7,0)(1.3,2)\psframe[fillstyle=solid,fillcolor=blue](1.7,0)(2.3,1)
% \psframe[fillstyle=solid,fillcolor=blue](2.7,0)(3.3,2)\psframe[fillstyle=solid,fillcolor=blue](3.7,0)(4.3,4)
% \psframe[fillstyle=solid,fillcolor=blue](4.7,0)(5.3,6)\psframe[fillstyle=solid,fillcolor=blue](5.7,0)(6.3,3)
% \psframe[fillstyle=solid,fillcolor=blue](6.7,0)(7.3,3)\psframe[fillstyle=solid,fillcolor=blue](7.7,0)(8.3,3)
% \psframe[fillstyle=solid,fillcolor=blue](9.7,0)(10.3,2)\psframe[fillstyle=solid,fillcolor=blue](11.7,0)(12.3,2)
% \uput{0.2}[180](0,7.5){{\textit{Effectifs}}}\uput{0.2}[270](13,0){{\textit{Notes}}}
% %\caption[pspicture]{\label{illustration}{\textbf{Diagramme en barres :}} La hauteur des barres est \'egale aux effectifs}
% \end{pspicture}
% }
% \end{center}
% Pour calculer la moyenne, on effectue le calcul suivant :\par\vspace{0.25cm}
% $M = \dfrac{6\times\textcolor{red}{2}+7\times\textcolor{red}{1}+8\times\textcolor{red}{2}+9\times\textcolor{red}{4}+10\times\textcolor{red}{6}+11\times\textcolor{red}{3}+
% 12\times\textcolor{red}{3}+13\times\textcolor{red}{3}+15\times\textcolor{red}{2}+17\times\textcolor{red}{2}}{2+1+2+4+6+3+3+3+2+2}$\par\vspace{0.25cm}
% $M=\dfrac{303}{28}\approx10,8$
% }

\subsection{R\'epartition en classes et moyenne}
% \Remarques[Vocabulaire]{
% Dans certains cas une s\'erie statistique peut être r\'epartie en \textbf{classes} (par intervalles de valeurs).
% }

% \Exemples[Exemple]{r\'epartition de 2000 adultes suivant leur taille}{
% \begin{center}
% \begin{tabular}{|m{3cm}|c|c|c|c|}
% \hline
% Tailles en $cm$ &$\left[140\, ; 150\right[$&$\left[150\, ; 160\right[$&$\left[160\, ; 170\right[$&$\left[170\, ; 195\right[$\\
% \hline
% Effectifs&48&397&913&642\\
% \hline
% \end{tabular}
% \end{center}
% \vspace{0.25cm}
% Il est impossible, a priori, de calculer la moyenne de cette s\'erie puisqu'on ne connaît pas les valeurs des tailles et leurs effectifs. On peut donc, par exemple, consid\`{e}rer que la valeur au centre de la classe va repr\'esenter la classe. On calcule alors la moyenne pond\'er\'ee pour obtenir une valeur approch\'ee de la moyenne de la s\'erie.
% \begin{center}
% \begin{tabular}{|c|c|c|c|c|}
% \hline
% Tailles en $cm$ &$\left[140\, ; 150\right[$&$\left[150\, ; 160\right[$&$\left[160\, ; 170\right[$&$\left[170\, ; 195\right[$\\
% \hline
% Centre de la classe&145&155&165&182,5\\
% \hline
% Effectifs&48&397&913&642\\
% \hline
% \end{tabular}
% \end{center}
% Pour calculer la taille moyenne, on effectue le calcul suivant :$$\dfrac{145\times48+155\times397+165\times913+145\times642}{2000}\approx168$$
% \psshadowbox{La taille moyenne de ce groupe d'adultes est d'environ $168\,cm$.}
% }