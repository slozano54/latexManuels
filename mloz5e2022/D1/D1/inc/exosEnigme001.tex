% Les enigmes ne sont pas numérotées par défaut donc il faut ajouter manuellement la numérotation
% si on veut mettre plusieurs enigmes
\refstepcounter{exercice}
% \numeroteEnigme
\begin{enigme}[Le tableur]
    Un tableur est un logiciel d'édition et de présentation de tableaux. Il comporte des \textbf{feuilles de calcul} composées de multiples lignes et colonnes formant des \textbf{cellules}. Chaque cellule est repérée par son adresse : une lettre désignant la colonne et un numéro désignant la ligne. Par exemple, la cellule {\bf A1} fait référence à la colonne A ligne numéro 1. \medskip
 
 \partie[écrire dans une cellule] 
    Ouvrir une nouvelle feuille de tableur, écrire les textes suivants et appuyer sur entrée.     
    $$\text{Dans la cellule A1 : }  \fcolorbox[gray]{0.1}{0.9}{\makebox[2.5cm][l]{\texttt{1+2}}} \qquad ; \qquad \text{ Dans la cellule A2 : } \fcolorbox[gray]{0.1}{0.9}{\makebox[2.5cm][l]{\texttt{=1+2}}}$$
    Quelle est la différence entre ces deux écritures ? Quelle est la différence d'interprétation du tableur ? \par     
 \partie[utilisation du tableur-grapheur]
    Le ministère de la culture a effectué une enquête sur les équipements utilisés le plus souvent pour se connecter à Internet en 2019, en fonction de l'âge des utilisateurs.
    \begin{center}
       \begin{Tableur}[Bandeau=false,Colonnes=7,LargeurUn=60pt,Largeur=48pt]
          & 12-17 ans & 18-24 ans & 25-39 ans	 & 40-59 ans & 60-69 ans & 70 ans + \\
          Smartphone & 78\,\% & 89\,\% & 79\,\% & 45\,\% & 24\,\% & 12\,\% \\
          Ordinateur & 16\,\% & 10\,\% & 15\,\% & 42\,\% & 45\,\% & 38\,\% \\
       \end{Tableur}
    \end{center}
    On souhaite représenter, sous différentes formes, les résultats de cette enquête.
    \begin{enumerate}
       \item Reproduire ce tableau dans un tableur. \\
          {\it Enregistre ton travail dans un fichier nommé \texttt{Nom\_prenom\_classe\_tableur\_grapheur}.}
       \item Sélectionner le tableau en entier (cellules A1 à G3) et insérer trois graphiques : 
       \begin{itemize}
            \item un diagramme en barres verticales (graphique 1)
            \item un diagramme en barres horizontales (graphique 2)
            \item un diagramme circulaire (graphique 3)
       \end{itemize}
       \item Laquelle de ces trois représentations vous parait-elle la plus lisible, pourquoi ?
       \item Choisir deux autres représentations (graphique 4 et graphique 5) et les décrire brièvement.
       \item Créer un graphique 6 de votre choix représentant uniquement les données concernant les ordinateurs.
       \item Répondre aux questions suivantes :
          \begin{enumerate}
             \item Quelle tranche d'âge utilise le plus son smartphone pour aller sur Internet ? 
             \item Selon toi, pourquoi \og les jeunes \fg{} utilisent-ils plus un smartphone qu'un ordinateur pour aller sur Internet ?                
             \item Selon toi, pourquoi \og les seniors \fg{} utilisent-ils plus un ordinateur qu'un smartphone pour aller sur Internet ?
          \end{enumerate}
    \end{enumerate}
    \dotfill\par
    \dotfill\par
    \dotfill\par
    \dotfill\par
    \dotfill\par
    \dotfill\par    
 \end{enigme}
 
 

% Pour le corrigé, il faut décrémenter le compteur, sinon il est incrémenté deux fois
\addtocounter{exercice}{-1}
\begin{corrige}
    \ldots
\end{corrige}