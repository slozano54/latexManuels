\begin{exercice*}[Notes]
    Rémi a obtenu ces notes ce trimestre-ci en mathématiques :\\
    $15$; $7$ ; $10$ ; $16$ ; $20$ ; $5$ ; $8$ ; $17$ ; $4$ ; $19$ ; $10$ et $14$.\\
    Calculer la moyenne de cet élève en mathématiques.

    \hrefMathalea{https://coopmaths.fr/mathalea.html?ex=5S14,s=1,n=1,i=0&v=l} % On peut personnaliser le texte entre crochets si on veut sinon supprimer les crochets
\end{exercice*}
\begin{corrige}
    %\setcounter{partie}{0} % Pour s'assurer que le compteur de \partie est à zéro dans les corrigés
    %\phantom{rrr}
        Rémi a obtenu ces notes ce trimestre-ci en mathématiques :\\
    $15$; $7$ ; $10$ ; $16$ ; $20$ ; $5$ ; $8$ ; $17$ ; $4$ ; $19$ ; $10$ et $14$.\\
    Calculer la moyenne de cet élève en mathématiques.
    {\red
    \begin{spacing}{1.5}
        La somme des notes est : $145$.\\
         Il y a $12$ notes.\\
        Donc la moyenne de cet élève est : $\dfrac{145}{12}$ $\approx12{,}08$
    \end{spacing}
    }
\end{corrige}

