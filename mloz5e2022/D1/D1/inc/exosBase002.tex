\begin{exercice*}[Notes]
    Elsa a obtenu ces notes ce trimestre-ci en mathématiques :\\
    $12$; $13$ ; $9$ ; $9$ ; $13$ ; $13$ ; $8$ ; $10$ ; $12$ et $13$.
    
    \medskip
    Calculer la fréquence de la note $12$.

    \hrefMathalea{https://coopmaths.fr/mathalea.html?ex=5S13,s=2,n=1,i=0&v=l} % On peut personnaliser le texte entre crochets si on veut sinon supprimer les crochets
\end{exercice*}
\begin{corrige}
    %\setcounter{partie}{0} % Pour s'assurer que le compteur de \partie est à zéro dans les corrigés
    %\phantom{rrr}
    Elsa a obtenu ces notes ce trimestre-ci en mathématiques :\\
    $12$; $13$ ; $9$ ; $9$ ; $13$ ; $13$ ; $8$ ; $10$ ; $12$ et $13$.
    
    \medskip
    Calculer la fréquence de la note $12$.
    
    {\red    
        La note $12$ a été obtenue $2$ fois.\\
         Il y a $10$ notes.\\
        Donc la fréquence de la note $12$ est : $\dfrac{2}{10}$$=0{,}2$\\
        Soit $20\thickspace\%$.    
    }
\end{corrige}

