\begin{exercice*}
   Le tableau ci-dessous représente la répartition des médailles françaises aux Jeux olympiques d'été de 1896 à 2016 pour les dix sports ayant eu le plus de médailles.
   \begin{enumerate}
      \item Compléter le tableau.
      \item Comment est établi le classement des sports aux Jeux olympiques.
      \item Construire trois diagrammes circulaires : celui du cyclisme, du tir et du canoë-kayak en fonction de la couleur de médaille obtenue. Les comparer.
   \end{enumerate}
   \begin{center}
      \small
         {\renewcommand{\arraystretch}{2.1}
         \begin{tabular}{|*{7}{c|}}
            \hline 
            Pl. & & Sport & \pscircle[fillstyle=solid,fillcolor=yellow](0,0.1){0.2} & \pscircle[fillstyle=solid,fillcolor=lightgray](0,0.1){0.2} & \pscircle[fillstyle=solid,fillcolor=brown](0,0.1){0.2} & T. \\
            \hline
            1 & \includegraphics[height=4mm]{\currentpath/images/S1} & & & \, 51 \, & \, 35 \, & \, {\bf 118} \, \\
            \hline 
            2 & \includegraphics[height=4mm]{\currentpath/images/S2} & & 41 & & 23 & {\bf 91} \\
            \hline
            3 & \includegraphics[height=4mm]{\currentpath/images/S3} & & 14 & 25 & & {\bf 68} \\
            \hline  
            4 & \includegraphics[height=4mm]{\currentpath/images/S4} & & 14 & 13 & 10 & \\
            \hline  
            5 & \includegraphics[height=4mm]{\currentpath/images/S5} & & 14 & 10 & & {\bf 49} \\
            \hline  
            6 & \includegraphics[height=4mm]{\currentpath/images/S6} & & 13 & & 17 & {\bf 41} \\
            \hline  
            7 & \includegraphics[height=4mm]{\currentpath/images/S7} & & & 14 & 10 & {\bf 33} \\
            \hline  
            8 & \includegraphics[height=4mm]{\currentpath/images/S8} & & 9 & & 3 & {\bf 15} \\
            \hline  
            9 & \includegraphics[height=4mm]{\currentpath/images/S9} & & 8 & 15 & & {\bf 43} \\
            \hline  
            10 & \includegraphics[height=4mm]{\currentpath/images/S10} & {\textcolor{white}{Canoé-kayak}} & 8 & 9 & 19 & \\
            \hline    
         \end{tabular}} \\ [2mm]
      \hfill {\footnotesize\it Source : France aux Jeux olympiques, Wikipedia, 2019}
   \end{center}
\end{exercice*}
\begin{corrige}
    %\setcounter{partie}{0} % Pour s'assurer que le compteur de \partie est à zéro dans les corrigés
    %\phantom{rrr}
    Le tableau ci-dessous représente la répartition des médailles françaises aux Jeux olympiques d'été de 1896 à 2016 pour les dix sports ayant eu le plus de médailles.

    \begin{enumerate}
       \item Compléter le tableau.
       
       {\red
       Voici le tableau complété : \\ \smallskip
       }
         {\footnotesize
         \renewcommand{\arraystretch}{2}
         \begin{tabular}{|*{7}{c|}}
            \hline 
             & & Sport & \pscircle[fillstyle=solid,fillcolor=yellow](0,0.1){0.2} & \pscircle[fillstyle=solid,fillcolor=lightgray](0,0.1){0.2} & \pscircle[fillstyle=solid,fillcolor=brown](0,0.1){0.2} & T. \\
            \hline
            1 & \includegraphics[scale=0.3]{\currentpath/images/S1} & {\red Escrime} & \, {\red 32} \, & \, 51 \, & \, 35 \, & \, {\bf 118} \, \\
            \hline 
            2 & \includegraphics[scale=0.3]{\currentpath/images/S2} & {\red Cyclisme} & 41 & {\red 27} & 23 & {\bf 91} \\
            \hline
            3 & \includegraphics[scale=0.3]{\currentpath/images/S3} & {\red Athlétisme} & 14 & 25 & {\red 29} & {\bf 68} \\
            \hline  
            4 & \includegraphics[scale=0.3]{\currentpath/images/S4} & {\red Équitation} & 14 & 13 & 10 & {\red \bf 37} \\
            \hline  
            5 & \includegraphics[scale=0.3]{\currentpath/images/S5} & {\red Judo} & 14 & 10 & {\red 25} & {\bf 49} \\
            \hline  
            6 & \includegraphics[scale=0.3]{\currentpath/images/S6} & {\red Voile} & 13 & {\red 11} & 17 & {\bf 41} \\
            \hline  
            7 & \includegraphics[scale=0.3]{\currentpath/images/S7} & {\red Tir} & {\red 9} & 14 & 10 & {\bf 33} \\
            \hline  
            8 & \includegraphics[scale=0.3]{\currentpath/images/S8} & {\red Haltérophilie} & 9 & {\red 3} & 3 & {\bf 15} \\
            \hline  
            9 & \includegraphics[scale=0.3]{\currentpath/images/S9} & {\red Natation} & 8 & 15 & {\red 20} & {\bf 43} \\
            \hline  
            10 & \includegraphics[scale=0.3]{\currentpath/images/S10} & {\red Canoé-kayak} & 8 & 9 & 19 & {\red \bf 36} \\
            \hline    
         \end{tabular}}       
       \item Comment est établi le classement des sports aux Jeux olympiques.
       
       {\red Le classement des sports est établi grâce au {\red nombre de médailles d'or}, puis d'argent, puis de bronze.}
       \item Construire trois diagrammes circulaires : celui du cyclisme, du tir et du canoë-kayak en fonction de la couleur de médaille obtenue. Les comparer.
         
       {\red On récapitule dans un tableau les angles : \\ \smallskip
       
       \small
      {\renewcommand{\arraystretch}{1.5}
      \begin{Ltableau}{0.95\linewidth}{5}{c}
         \hline   
         Couleur & Or & Argent & Bronze & Total \\
         \hline   
         Cyclisme & 41 & 27 & 23 & 91 \\
         Angle & \ang{162} & \ang{107} & \ang{91} & \ang{360} \\
         \hline
         Tir & 9 & 14 & 10 & 33 \\
         Angle & \ang{98} & \ang{153} & \ang{109} & \ang{360} \\
         \hline
         Canoé-kayak & 8 & 9 & 19 & 36 \\
         Angle & \ang{80} & \ang{90} & \ang{190} & \ang{360} \\
        \hline
      \end{Ltableau}}
      {\psset{unit=0.4}
      \quad
      \begin{pspicture}(-2,-3.3)(3,3.8)
         \pscircle(0,0){3}
         \pswedge[fillstyle=solid,fillcolor=yellow](0,0){3}{0}{162}
         \pswedge[fillstyle=solid,fillcolor=lightgray](0,0){3}{162}{269}
         \pswedge[fillstyle=solid,fillcolor=brown](0,0){3}{269}{360}      
      \end{pspicture}
      \hfill
      \begin{pspicture}(-3.5,-3)(3,3.5)
         \pscircle(0,0){3}
         \pswedge[fillstyle=solid,fillcolor=yellow](0,0){3}{0}{98}
         \pswedge[fillstyle=solid,fillcolor=lightgray](0,0){3}{98}{251}
         \pswedge[fillstyle=solid,fillcolor=brown](0,0){3}{251}{360}      
      \end{pspicture}
      \hfill
      \begin{pspicture}(-3.5,-3)(2.5,3.5)
         \pscircle(0,0){3}
         \pswedge[fillstyle=solid,fillcolor=yellow](0,0){3}{0}{80}
         \pswedge[fillstyle=solid,fillcolor=lightgray](0,0){3}{80}{170}
         \pswedge[fillstyle=solid,fillcolor=brown](0,0){3}{170}{360}      
      \end{pspicture}} \hspace*{24mm} \\
      \quad Cyclisme \hfill Tir \hfill Canoé-kayak \hspace*{24mm} \\ \medskip
      On remarque par exemple que chacun de ces sports à une couleur très dominante : l'or pour le cyclisme, l'argent pour le tir et le bronze pour le canoé-kayak.  
       }
    \end{enumerate}
\end{corrige}

