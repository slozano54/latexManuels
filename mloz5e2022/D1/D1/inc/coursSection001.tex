\section{Vocabulaire}

\begin{definition}
Une \textbf{population statistique}, les élèves d'un collège, mais aussi les voitures d'un parc automobiles,\dots 
est constituée d'\textbf{individus}, un élève, une voiture du parc,\dots dont on étudie un \textbf{caractère} 
le régime alimentaire, la couleur des yeux, la taille, la pointure, le nombre de portes,\dots

\smallskip
Ce caractère peut être :
\begin{itemize}
\item \textbf{qualitatif} : couleur des yeux, sexe, port de lunettes,\dots
\item \textbf{quantitatif} : taille, masse,poids,pointure,nombre de frères et soeurs,\dots
\end{itemize}

Les différentes valeurs prises par le caractère s'appellent des \textbf{modalités}.

Parfois on peut regrouper les individus selon certains critères, l'âge, la classe, le niveau, la couleurs des yeux,\dots on dit qu'on fait un \textbf{regroupement en classes}.

\textbf{Attention au cas discret !}

\smallskip
Lorsque la série est numérique, l'\textbf{étendue} est égale â la différence entre la plus grande et la plus petite des modalités du caractère.
\begin{itemize}
\item L'étude d'une \textbf{série statistique} c'est l'étude d'un caractère d'une population.
\item L'\textbf{effectif} d'une modalité c'est le nombre d'individus dont le caractère est cette modalité.
\item L'\textbf{effectif total} c'est le nombre total d'individus de la série.
\item Le \textbf{mode} d'une série statistique, c'est la modalité qui présente le plus grand effectif.
\end{itemize}
\end{definition}