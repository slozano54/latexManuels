\begin{exercice*}[Températures]
    En septembre 1989, à Berlin, on a relevé les températures suivantes.\\
    $\def\arraystretch{1.5}\begin{array}{|c|c|c|c|c|c|c|c|c|c|c|c|c|c|c|c|c}\hline  \text{Jour}&1&2&3&4&5&6&7&8&9&10&11&12&13&14&15\\
    \hline \text{Température en} ^\circ\text{C}&23&22&20&22&23&25&24&26&27&28&26&28&30&28&30\\
    \hline\end{array}$
    
    \medskip
    $\def\arraystretch{1.5}\begin{array}{|c|c|c|c|c|c|c|c|c|c|c|c|c|c|c|c|c}\hline  \text{Jour}&16&17&18&19&20&21&22&23&24&25&26&27&28&29&30\\
    \hline \text{Température en} ^\circ\text{C}&30&28&27&29&31&33&35&35&36&35&33&33&33&32&32\\
    \hline\end{array}$
    
    \medskip
    Calculer la température moyenne de ce mois.

    \hrefMathalea{https://coopmaths.fr/mathalea.html?ex=5S14,s=2,n=1,i=0&v=l} % On peut personnaliser le texte entre crochets si on veut sinon supprimer les crochets
\end{exercice*}
\begin{corrige}
    %\setcounter{partie}{0} % Pour s'assurer que le compteur de \partie est à zéro dans les corrigés
    %\phantom{rrr}
    En septembre 1989, à Berlin, on a relevé les températures suivantes.\\
$\def\arraystretch{1.5}\begin{array}{|c|c|c|c|c|c|c|c|c|c|c|c|c|c|c|c|c}\hline  \text{Jour}&1&2&3&4&5&6&7&8&9&10&11&12&13&14&15\\
\hline \text{Température en} ^\circ\text{C}&23&22&20&22&23&25&24&26&27&28&26&28&30&28&30\\
\hline\end{array}$

\medskip
$\def\arraystretch{1.5}\begin{array}{|c|c|c|c|c|c|c|c|c|c|c|c|c|c|c|c|c}\hline  \text{Jour}&16&17&18&19&20&21&22&23&24&25&26&27&28&29&30\\
\hline \text{Température en} ^\circ\text{C}&30&28&27&29&31&33&35&35&36&35&33&33&33&32&32\\
\hline\end{array}$

\medskip
Calculer la température moyenne de ce mois.
    {\red
    \begin{spacing}{1.5}
        En septembre 1989, la somme des températures est $864^\circ\text{C}$.\\
         Il y a $30$ jours ce mois-ci.\\
         La température moyenne est :\\
        $\dfrac{864^\circ\text{C}}{30}$$=28{,}8^\circ\text{C}$
    \end{spacing}
    }
\end{corrige}

