\begin{exercice*}[Températures]
    En janvier 2009, à Belgrade, on a relevé les températures suivantes.

    \smallskip    
    \begin{tabularx}{\linewidth}{|c|*{7}{>{\centering\arraybackslash}X|}}
        \hline  
        Jour                   &1&2&3&4&5&6&7\\\hline
        Température en \Temp{} &1&2&4&2&2&2&2\\\hline
    \end{tabularx}

    \smallskip    
    \begin{tabularx}{\linewidth}{|c|*{6}{>{\centering\arraybackslash}X|}}
        \hline
        Jour                   &8&9&10&11&12&13\\\hline 
        Température en \Temp{} &3&3&2 &1 &1 &-1\\\hline
    \end{tabularx}   
    
    \smallskip    
    \begin{tabularx}{\linewidth}{|c|*{6}{>{\centering\arraybackslash}X|}}
        \hline
        Jour                   &14&15&16&17&18&19\\\hline 
        Température en \Temp{} &-2&-2&-1&-2&-1&-2\\\hline
    \end{tabularx}
    
    \smallskip    
    \begin{tabularx}{\linewidth}{|c|*{6}{>{\centering\arraybackslash}X|}}
        \hline
        Jour                   &20&21&22&23&24&25\\\hline 
        Température en \Temp{} &-1&-3&-5&-4&-3&-2\\\hline
    \end{tabularx}
    
    \smallskip    
    \begin{tabularx}{\linewidth}{|c|*{6}{>{\centering\arraybackslash}X|}}
        \hline
        Jour                   &26&27&28&29&30&31\\\hline 
        Température en \Temp{} &-4&-5&-6&-5&-6&-6\\\hline
    \end{tabularx}

    \medskip
    Calculer la fréquence de la température \Temp{-2}.

    \hrefMathalea{https://coopmaths.fr/mathalea.html?ex=5S13,s=3,n=1,i=0&v=l} % On peut personnaliser le texte entre crochets si on veut sinon supprimer les crochets
\end{exercice*}
\begin{corrige}
    %\setcounter{partie}{0} % Pour s'assurer que le compteur de \partie est à zéro dans les corrigés
    %\phantom{rrr}
    En janvier 2009, à Belgrade, on a relevé les températures suivantes.

    \smallskip    
    \begin{tabularx}{\linewidth}{|c|*{7}{>{\centering\arraybackslash}X|}}
        \hline  
        Jour                   &1&2&3&4&5&6&7\\\hline
        Température en \Temp{} &1&2&4&2&2&2&2\\\hline
    \end{tabularx}

    \smallskip    
    \begin{tabularx}{\linewidth}{|c|*{6}{>{\centering\arraybackslash}X|}}
        \hline
        Jour                   &8&9&10&11&12&13\\\hline 
        Température en \Temp{} &3&3&2 &1 &1 &-1\\\hline
    \end{tabularx}   
    
    \smallskip    
    \begin{tabularx}{\linewidth}{|c|*{6}{>{\centering\arraybackslash}X|}}
        \hline
        Jour                   &14&15&16&17&18&19\\\hline 
        Température en \Temp{} &-2&-2&-1&-2&-1&-2\\\hline
    \end{tabularx}
    
    \smallskip    
    \begin{tabularx}{\linewidth}{|c|*{6}{>{\centering\arraybackslash}X|}}
        \hline
        Jour                   &20&21&22&23&24&25\\\hline 
        Température en \Temp{} &-1&-3&-5&-4&-3&-2\\\hline
    \end{tabularx}
    
    \smallskip    
    \begin{tabularx}{\linewidth}{|c|*{6}{>{\centering\arraybackslash}X|}}
        \hline
        Jour                   &26&27&28&29&30&31\\\hline 
        Température en \Temp{} &-4&-5&-6&-5&-6&-6\\\hline
    \end{tabularx}

    \medskip
    Calculer la fréquence de la température \Temp{-2}.
    
    {\red
    \begin{spacing}{1.5}
        En janvier 2009, à Bruxelles, la température \Temp{-2} a été relevée $5$ fois.\\
        Il y a $31$ jours ce mois-ci.\\
         La fréquence de la température \Temp{-2} est :\\
        $\dfrac{5}{31}$$\approx0{,}161$\\
        Soit environ $16{,}1\thickspace\%$.
    \end{spacing}
    }
\end{corrige}

