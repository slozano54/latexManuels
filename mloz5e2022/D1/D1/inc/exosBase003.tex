\begin{exercice*}[Températures]
    En janvier 2009, à Belgrade, on a relevé les températures suivantes.

    \smallskip
    $\def\arraystretch{1.5}
    \begin{array}{|c|c|c|c|c|c|c|c|c|}
        \hline  
        \text{Jour}                             &1&2&3&4&5&6&7&8\\\hline
        \text{Température en}  ^\circ\text{C}   &1&2&4&2&2&2&2&3\\\hline
    \end{array}$
    
    $\def\arraystretch{1.5}
    \begin{array}{|c|c|c|c|c|c|c|c|c|}
        \hline
        \text{Jour}                             &9&10&11&12&13&14&15&16\\\hline 
        \text{Température en}  ^\circ\text{C}   &3&2 &1 &1 &-1&-2&-2&-1\\\hline
    \end{array}$

    $\def\arraystretch{1.5}
    \begin{array}{|c|c|c|c|c|c|c|c|c|}
        \hline
        \text{Jour}                             &17&18&19&20&21&22&23&24\\\hline
        \text{Température en}  ^\circ\text{C}   &-2&-1&-2&-1&-3&-5&-4&-3\\\hline
    \end{array}$

    $\def\arraystretch{1.5}
    \begin{array}{|c|c|c|c|c|c|c|c|}
        \hline
        \text{Jour}                             &25&26&27&28&29&30&31\\\hline
        \text{Température en}  ^\circ\text{C}   &-2&-4&-5&-6&-5&-6&-6\\\hline
    \end{array}$

    \medskip
    Calculer la fréquence de la température $-2^\circ\text{C}$.

    \hrefMathalea{https://coopmaths.fr/mathalea.html?ex=5S13,s=3,n=1,i=0&v=l} % On peut personnaliser le texte entre crochets si on veut sinon supprimer les crochets
\end{exercice*}
\begin{corrige}
    %\setcounter{partie}{0} % Pour s'assurer que le compteur de \partie est à zéro dans les corrigés
    %\phantom{rrr}
    En janvier 2009, à Belgrade, on a relevé les températures suivantes.

    \smallskip
    $\def\arraystretch{1.5}
    \begin{array}{|c|c|c|c|c|c|c|c|c|}
        \hline  
        \text{Jour}                             &1&2&3&4&5&6&7&8\\\hline
        \text{Température en}  ^\circ\text{C}   &1&2&4&2&2&2&2&3\\\hline
    \end{array}$
    
    $\def\arraystretch{1.5}
    \begin{array}{|c|c|c|c|c|c|c|c|c|}
        \hline
        \text{Jour}                             &9&10&11&12&13&14&15&16\\\hline 
        \text{Température en}  ^\circ\text{C}   &3&2 &1 &1 &-1&-2&-2&-1\\\hline
    \end{array}$

    $\def\arraystretch{1.5}
    \begin{array}{|c|c|c|c|c|c|c|c|c|}
        \hline
        \text{Jour}                             &17&18&19&20&21&22&23&24\\\hline
        \text{Température en}  ^\circ\text{C}   &-2&-1&-2&-1&-3&-5&-4&-3\\\hline
    \end{array}$

    $\def\arraystretch{1.5}
    \begin{array}{|c|c|c|c|c|c|c|c|}
        \hline
        \text{Jour}                             &25&26&27&28&29&30&31\\\hline
        \text{Température en}  ^\circ\text{C}   &-2&-4&-5&-6&-5&-6&-6\\\hline
    \end{array}$

    \medskip
    Calculer la fréquence de la température $-2^\circ\text{C}$.
    
    {\red
    \begin{spacing}{1.5}
        En janvier 2009, à Bruxelles, la température $-2^\circ\text{C}$ a été relevée $5$ fois.\\
        Il y a $31$ jours ce mois-ci.\\
         La fréquence de la température $-2^\circ\text{C}$ est :\\
        $\dfrac{5}{31}$$\approx0{,}161$\\
        Soit environ $16{,}1\thickspace\%$.
    \end{spacing}
    }
\end{corrige}

