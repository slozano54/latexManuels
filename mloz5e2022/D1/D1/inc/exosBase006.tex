\begin{exercice*}[Pointures]
    Pour passer une commande de chaussures de foot, Nawel a noté les pointures des membres de son club et les a regroupées dans un tableau :

    \medskip
    $\def\arraystretch{1.5}\begin{array}{|c|c|c|c|c|c|} \hline \text{Pointure} & 35 & 36 & 37 & 38 & 39 \\
     \hline \text{Effectif} & 9 & 8 & 7 & 4 & 9 \\
    \hline \end{array}$
    
    \medskip
    Calculer la pointure moyenne des membres de ce club.

    \hrefMathalea{https://coopmaths.fr/mathalea.html?ex=5S14,s=3,n=1,i=0&v=l} % On peut personnaliser le texte entre crochets si on veut sinon supprimer les crochets
\end{exercice*}
\begin{corrige}
    %\setcounter{partie}{0} % Pour s'assurer que le compteur de \partie est à zéro dans les corrigés
    %\phantom{rrr}
    Pour passer une commande de chaussures de foot, Nawel a noté les pointures des membres de son club et les a regroupées dans un tableau :

\medskip
$\def\arraystretch{1.5}\begin{array}{|c|c|c|c|c|c|} \hline \text{Pointure} & 35 & 36 & 37 & 38 & 39 \\
 \hline \text{Effectif} & 9 & 8 & 7 & 4 & 9 \\
\hline \end{array}$

\medskip
Calculer la pointure moyenne des membres de ce club.

    {\red
    \begin{spacing}{1.5}
        $\text{Moyenne} = \dfrac{\text{Somme des valeurs}}{\text{Effectif total}} =\dfrac{35 \times 9+ 36 \times 8+ 37 \times 7+ 38 \times 4+ 39 \times 9}{9+ 8+ 7+ 4+ 9} = \dfrac{1365}{37} \approx36{,}89$\\
    \end{spacing}
    }
\end{corrige}

