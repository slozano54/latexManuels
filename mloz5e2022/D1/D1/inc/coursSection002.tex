\section{Effectifs et fréquences}
% \definNumTitre{Effectifs et fr\'equences}{
% Dans un tableau statistique, l'{\bf effectif} est le nombre de r\'eponses associ\'ees \`{a} chaque valeur.\par
% L'ensemble des valeurs et des effectifs forme une {\bf s\'erie statistique}.\par
% En divisant l'effectif d'une valeur par l'effectif total, on obtient la {\bf fr\'equence}.
% $$f=\frac{\mbox{valeur de l'effectif}}{\mbox{valeur de l'effectif total}}$$
% }

% \Remarques[Remarque]{
% Une fréquence peut s'exprimer :
% \begin{mylist}
% \item sous forme fractionnaire.
% \item sous forme de pourcentage, éventuellement en arrondissant.
% \item sous forme décimale, éventuellement en arrondissant.
% \end{mylist}
% }

% \newpage
% \Exemples[Exemple]{}{La standardiste d'une radio FM a not\'e le nombre d'appels t\'el\'ephoniques reçus par tranches d'heures au cours d'une matin\'ee. Elle obtient les r\'esultats suivants :
% \begin{center}
% \renewcommand{\arraystretch}{3}
% \begin{tabular}{|m{3.5cm}|c|c|c|c|c|}
% \hline
% Tranches horaires&9h-10h&10h-11h&11h-12h&12h-13h&{\bf Total}\\
% \hline
% Effectifs\par(nombres d'appels)&19&37&46&28&{\bf 130}\\
% \hline
% Fr\'equences& $\dfrac{19}{130}$&&&&{\bf 1}\\
% \hline
% Fr\'equences en \% &14,6&&&&{\bf 100}\\
% \hline
% \end{tabular}
% \renewcommand{\arraystretch}{1}
% \end{center}
% }