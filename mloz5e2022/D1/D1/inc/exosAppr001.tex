\begin{exercice*}
  On a demandé à des élèves de trois classes de 5\up{e} combien d'animaux de compagnie vivaient avec eux. \\
  Le résultat est représenté dans le diagramme suivant :

  \smallskip
  \hspace*{-15mm}
  \scalebox{0.9}{
  \Stat[Graphique,Grille,Unitex=1.2,Unitey=0.14,Origine=-0.5,Pasy=5,EpaisseurBatons=8,Donnee={Nombre d'animaux},Effectif=Nombre d'élèves,LectureFine,Tiret,ListeCouleursB={Gray,Gray,Gray,Gray,Gray}]{0/32,1/21,2/15,3/4,4/3}\\
  }
  Les affirmations suivantes sont-elles vraies ou fausses ?
  \begin{enumerate}
     \item 21 élèves ont un seul animal de compagnie.
     \item Il y a 75 élèves en 5\up{e} dans ce collège.
     \item Les élèves qui ont deux animaux de compagnie sont trois fois plus nombreux que les élèves qui en ont trois.
     \item 70 élèves ont moins de trois animaux de compagnie.
     \item Plus de la moitié des élèves ont au moins un animal de compagnie.
     \item Parmi les élèves qui ont au moins un animal de compagnie, la moitié en ont plusieurs.
  \end{enumerate}

\end{exercice*}
\begin{corrige}
    %\setcounter{partie}{0} % Pour s'assurer que le compteur de \partie est à zéro dans les corrigés
    %\phantom{rrr}
    On a demandé à des élèves de trois classes de 5\up{e} combien d'animaux de compagnie vivaient avec eux. \\
    Le résultat est représenté dans le diagramme suivant :
  
    \smallskip    
    \scalebox{0.8}{      
      \Stat[Graphique,Grille,Unitex=1.2,Unitey=0.14,Origine=-0.5,Pasy=5,EpaisseurBatons=8,Donnee={Nombre d'animaux},Effectif=Nombre d'élèves,LectureFine,Tiret,ListeCouleursB={Gray,Gray,Gray,Gray,Gray}]{0/32,1/21,2/15,3/4,4/3}
    }

    Les affirmations suivantes sont-elles vraies ou fausses ?

    \begin{enumerate}
       \item 21 élèves ont un seul animal de compagnie.
       
       {\red Vrai. Par lecture de la deuxième barre.}
       \item Il y a 75 élèves en 5\up{e} dans ce collège.
       
       {\red Faux. Le diagramme ne nous permet pas de connaître le nombre d'élèves de ce collège puisqu'il s'agit des données de trois classes seulement.}
       \item Les élèves qui ont deux animaux de compagnie sont trois fois plus nombreux que les élèves qui en ont trois.
       
       {\red Faux. 15 élèves ont deux animaux de compagnie et  4 en ont trois. or, 15 n'est pas le triple de 4.}
       \item 70 élèves ont moins de trois animaux de compagnie.
       
       {\red Faux. $32+21+15 =68$. 68 élèves ont moins de trois animaux de compagnie.}
       \item Plus de la moitié des élèves ont au moins un animal de compagnie.
       
       {\red Vrai. 32 élèves n'ont pas d'animal de compagnie et 43 en ont au moins un.}
       \item Parmi les élèves qui ont au moins un animal de compagnie, la moitié en ont plusieurs.
       
       {\red Vrai. Parmi les 43 élèves qui ont au moins un animal de compagnie, 22 en ont plusieurs, donc plus de la moitié.}
    \end{enumerate}
\end{corrige}

