\vspace*{-9mm}
\begin{changemargin}{-10mm}{-10mm}
%pre-001
\begin{prerequis}[Connaisances \emoji{red-heart} et compétences \emoji{diamond-suit} du cycle 3]    
   \begin{itemize}        
       \item[\emoji{red-heart}] Vocabulaire associé à ces objets et à leurs propriétés : côté, sommet, angle, hauteur.
       \columnbreak
       \item[\emoji{diamond-suit}] Reconnaître, nommer, décrire des triangles, dont les triangles particuliers (triangle rectangle, triangle isocèle, triangle équilatéral).       
   \end{itemize}
\end{prerequis}
\vspace*{-4mm}
%pre-002
\begin{prerequis}[Connaisances \emoji{red-heart} et compétences \emoji{diamond-suit} du cycle 4]    
    \begin{itemize}        
        \item[\emoji{diamond-suit}] Mener des calculs impliquant des grandeurs mesurables, exprimer les résultats dans des les unités adaptées.
        \item[\emoji{diamond-suit}] Exprimer et vérifier la cohérence des résultats du point de vue des unités.
    \end{itemize}
\end{prerequis}
\end{changemargin}
\vspace*{-13mm}
\begin{debat}[Le premier tableur]   
   \begin{changemargin}{-15mm}{-15mm}
   Des données brutes récoltées ont souvent peu de sens si elles sont utilisées ainsi, d'où la nécessité
   de les disposer d'une manière plus lisible à l'aide de tableaux et diagrammes.

   Avec l'avènement de l'informatique, les tableaux deviennent numériques grâce aux {\bf tableurs}.
%    : logiciels qui permettent de manipuler des données numériques, d'effectuer des opérations de façon automatisée,
%    de créer des représentations graphiques à partir des données : diagrammes , histogrammes, courbes\dots{}
   
   Le premier tableur fut créé en 1978 par {\it Daniel Bricklin}, étudiant à Harvard qui devait établir des tableaux
   comptables pour une étude de cas sur Pepsi-Cola sans pour autant établir tous les calculs \og à la main \fg. 
   Son premier prototype, {\it VisiCalc} (pour Visible Calculator), pouvait manipuler un tableau de vingt
   lignes et cinq colonnes !   
   \begin{center}    
      \scalebox{0.7}{   
      \begin{Tableur}[Bandeau=false,Colonnes=5]
         & & & & \\
         & & & & \\
         & & & & \\
      \end{Tableur}
      }
   \end{center}
   \begin{cadre}[B2][J4]
      \begin{center}
         \hrefVideo{https://www.ted.com/talks/dan_bricklin_meet_the_inventor_of_the_electronic_spreadsheet}{\bf Meet the inventor of electronic spreadsheet}
         
         Site Internet de {\it Ted talks}.
      \end{center}
   \end{cadre}
   \end{changemargin}
\end{debat}