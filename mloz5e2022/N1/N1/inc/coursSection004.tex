\section{Calcul des quotients}
% \CadreLampe{Rappels}{
% \begin{list}{*}{}
% \item Un quotient c'est le résultat d'une division.
% \item Une division peut être interminable, par exemple $10\div 3$. \par
% Dans ce cas on note le quotitent $\dfrac{10}{3}$
% \item $\dfrac{\OPoval{A}{0,1}{\textcolor{mygreen}{10}}}{\OPoval{B}{0,1}{\textcolor{red}{3}}}$\qquad
% \begin{minipage}[c]{8cm}
% 	\pnode[0,0](0,1.5em){C}\psdot(C)\uput[20](C){\textcolor{mygreen}{NUM\'{E}RATEUR}}
% 	%{ \textcolor{mygreen}{NUM\'{E}RATEUR}}\psdot(C)
% 	\ncarc{->}{C}{A}
% 	\pnode(0,0.2em){D}{ \textcolor{red}{D\'{E}NOMINATEUR}}\psdot(D)
% 	\ncarc{->}{D}{B}
% \end{minipage}
% \end{list}
% \par
% }

% \proprNumBis{(admise)}{
% Dans un calcul de quotient, on commence par effectuer les suites d'opérations figurant au numérateur et au dénominateur.
% }
% \CadreLampe{Remarque}{
% Il y a des parenthèses implicites autour du numérateur et du dénominateur.
% }
% \newpage
% \Exemples[Exemple]{}{
% On aura donc deux fa\c cons de calculer les quotients.
% \par
% \begin{tabular}{ccc}
% \begin{minipage}{8cm}
% $$\Eqalign{
% A=&\dfrac{10\:000+ 3\times 20\:000}{3\times 10 - 27}\cr
% A=&\dfrac{10\:000+ 60\:000}{3\times 10 - 27}\cr
% A=&\dfrac{70\:000}{3\times 10 - 27}\cr
% A=&\dfrac{70\:000}{30 - 27}\cr
% A=&\psshadowbox{\dfrac{70\:000}{3}}\cr
% }
% $$
% \end{minipage}
% &
% \begin{minipage}{8cm}
% $$\Eqalign{
% A=&\dfrac{10\:000+ 3\times 20\:000}{3\times 10 - 27}\cr
% A=&{\color{red}(}10\:000+ 3\times 20\:000{\color{red})}\div{\color{red}(}3\times 10 - 27{\color{red})}\cr
% A=&{\color{red}(}10\:000+ 60\:000{\color{red})}\div{\color{red}(}3\times 10 - 27{\color{red})}\cr
% &\cr
% A=&70\:000\div{\color{red}(}3\times 10 - 27{\color{red})}\cr
% A=&70\:000\div{\color{red}(}30 - 27{\color{red})}\cr
% A=&70\:000\div 3\cr
% A=&\psshadowbox{\dfrac{70 000}{3}}\cr
% }
% $$
% \end{minipage}
% \end{tabular}
% }

% \Exemples{Pour chaque calcul essayer l'autre méthode}{
% \begin{tabular}{cc}
% \begin{minipage}{8cm}
% $$\Eqalign{
% B=&\dfrac{10+20\times 2}{3\times 10 - 27}\cr
% B=&{\color{red}(}10+20\times 2{\color{red})}\div {\color{red}(}3\times 10 - 27{\color{red})}\cr
% B=&{\color{red}(}10+40{\color{red})}\div {\color{red}(}3\times 10 - 27{\color{red})}\cr
% B=&50\div {\color{red}(}3\times 10 - 27{\color{red})}\cr
% B=&50\div {\color{red}(}30 - 27{\color{red})}\cr
% B=&50\div 3\cr
% B=&\psshadowbox{\dfrac{50}{3}}\cr
% }
% $$
% \end{minipage}
% &
% \begin{minipage}{8cm}
% $$\Eqalign{
% C=&\dfrac{37-3\times 4+5}{37-3\times (4+5)}\cr
% C=&\dfrac{37-12+5}{37-3\times 9}\cr
% C=&\dfrac{30}{37-27}\cr
% C=&\dfrac{30}{10}\cr
% C=&\psshadowbox{3}\cr
% }
% $$
% \end{minipage}
% \end{tabular}
% }
