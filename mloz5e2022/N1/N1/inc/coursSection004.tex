\section{Calcul des quotients}

\begin{myBox}{\emoji{light-bulb} Rappels}
    \begin{list}{$\bullet$}{}
        \item Un quotient c'est le résultat d'une division.
        \item Une division peut être interminable, par exemple $10\div 3$. Dans ce cas on note le quotitent $\dfrac{10}{3}$
        \item $\dfrac{\OPoval{A}{0,1}{10}}{\OPoval{B}{0,1}{\textcolor{red}{3}}}$\qquad
        \begin{minipage}[c]{8cm}
            \pnode[0,0](0,1.5em){C}\psdot(C)\uput[20](C){NUMÉRATEUR}            
            \ncarc{->}{C}{A}
            \pnode(0,0.2em){D}{ \textcolor{red}{DÉNOMINATEUR}}\psdot(D)
            \ncarc{->}{D}{B}
        \end{minipage}
    \end{list}
\end{myBox}

\begin{propriete}[\admise]
    Dans un calcul de quotient, on commence par effectuer les suites d'opérations figurant au numérateur et au dénominateur.
\end{propriete}

\begin{remarque}
    Il y a des parenthèses implicites autour du numérateur et du dénominateur.
\end{remarque}

\begin{exemple*1}
    On aura donc deux façons de calculer les quotients.

    \medskip
    Calculer $A=\dfrac{\num{10000}+ 3\times \num{20000}}{3\times 10 - 27}$

    \medskip
    \correction    
    \begin{minipage}{0.45\linewidth}
        \begin{spacing}{1.5}
            \begin{list}{}{}
                \item $\phantom{A}$
                \item $A=\dfrac{\num{10000}+ 3\times \num{20000}}{3\times 10 - 27}$
                \item $A=\dfrac{\num{10000}+ \num{60000}}{30 - 27}$            
                \item $A=\psshadowbox{\dfrac{\num{70000}}{3}}$
            \end{list}
        \end{spacing}
    \end{minipage}      
    \begin{minipage}{0.5\linewidth}        
        \begin{list}{}{}
            \item $A=\dfrac{\num{10000}+ 3\times \num{20000}}{3\times 10 - 27}$
            
            \smallskip
            \item $A={\color{red}(}\num{10000}+ 3\times \num{20000}{\color{red})}\div{\color{red}(}3\times 10 - 27{\color{red})}$
            \item $A={\color{red}(}\num{10000}+ \num{60000}{\color{red})}\div{\color{red}(}30 - 27{\color{red})}$
            \item $A=\num{70000}\div 3$
            \item $A=\psshadowbox{\dfrac{\num{70000}}{3}}$
        \end{list}
    \end{minipage}   
\end{exemple*1}

\begin{exemple*1}
    Pour chaque calcul, $B=\dfrac{10+20\times 2}{3\times 10 - 27}$  et $C=\dfrac{37-3\times 4+5}{37-3\times (4+5)}$, 
    essayer l'autre façon que celle utilisée ci-dessous : 

    \bigskip
    % \correction
    \begin{minipage}{0.45\linewidth}
        \begin{spacing}{1.5}
            \begin{list}{}{}
                \item $B=\dfrac{10+20\times 2}{3\times 10 - 27}$            
                \item $B={\color{red}(}10+40{\color{red})}\div {\color{red}(}30 - 27{\color{red})}$            
                \item $B=50\div 3$
                \item $B=\psshadowbox{\dfrac{50}{3}}$
            \end{list}
        \end{spacing}
    \end{minipage}      
    \begin{minipage}{0.5\linewidth}
        \begin{spacing}{1.5}
            \begin{list}{}{}
                \item $C=\dfrac{37-3\times 4+5}{37-3\times (4+5)}$            
                \item $C=\dfrac{30}{37-27}$
                \item $C=\dfrac{30}{10}$
                \item $C=\psshadowbox{3}$            
            \end{list}
        \end{spacing}
    \end{minipage}   
\end{exemple*1}
