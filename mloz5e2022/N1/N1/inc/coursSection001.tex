\section{Rappels sur les nombres décimaux}

\begin{definition}
   Une {\bf fraction décimale} est une fraction dont le dénominateur est  $1$, $10$, $100$, $\num{1000}$ \ldots \\
   Un {\bf nombre décimal} est un nombre qui peut s'écrire sous forme d'une fraction décimale.
\end{definition}

\medskip
\begin{remarque}
    Un nombre a une seule valeur numérique mais a plusieurs écritures.
\end{remarque}
   
\begin{exemple*1}
   Voilà plusieurs écritures du nombre seize et quatre-vingt-deux centièmes : \par\smallskip
%     {\renewcommand{\arraystretch}{1.3}
%     \begin{tabular}{cp{13cm}}
%       16,82 & $=16+\dfrac{82}{100} =\dfrac{1\,682}{100}$ \\
%       & $=1\times10+6\times1+8\times\dfrac{1}{10}+2\times\dfrac{1}{100}$ \\
%       & $=1\times10+6\times1+8\times0,1+2\times0,01$ \\ [-5mm]
%    \end{tabular}}
    \begin{list}{}{}
        \item $\num{16.82} = 16 + \dfrac{82}{100} = \dfrac{\num{1682}}{\num{1000}}$
        \item $\phantom{\num{16.82}} = 1\times 10 + 6\times 1 + 8\times \dfrac{1}{10} + 2\times \dfrac{1}{100}$
        \item $\phantom{\num{16.82}} = 1\times 10 + 6\times 1 + 8\times \num{0.1} + 2\times \num{0.01}$
    \end{list}
\end{exemple*1}