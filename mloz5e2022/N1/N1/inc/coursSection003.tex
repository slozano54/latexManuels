\section{Reconnaître la nature d'une expression}

\begin{propriete}[\admise]
    La nature d'une expression est donnée par la dernière opération effectuée.\par
    Si la dernière opération est :
    \begin{list}{$\bullet$}{}
        \item une \textbf{addition} alors l'expression est une \textbf{somme}.
        \item une \textbf{soustraction} alors l'expression est une \textbf{différence}.
        \item une \textbf{multiplication} alors l'expression est un \textbf{produit}.
        \item une \textbf{division} alors l'expression est un \textbf{quotient}.
    \end{list}
\end{propriete}

\begin{exemple*1}
    \begin{multicols}2
        \begin{list}{}{}
        \item $A=12\times (5+2)$
        \item A est un produit car la 
        \item dernière opération est une multiplication
        \item $A=12\times7$
        \item 
        \item $B=(7-3)\div2$
        \item B est un quotient car la 
        \item dernière opération est une division
        \item $B=4\div2$
        \item 
        \item $C=5\times 8 + \dfrac{12}2$
        \item C est une somme car la 
        \item dernière opération est une addition
        \item $C=40+6$
        \item 
        \item $D=(8+5)-2\times3$
        \item D est une différence car la 
        \item dernière opération est une soustraction
        \item $D=13-6$
        \item 
        \end{list}
    \end{multicols}
\end{exemple*1}