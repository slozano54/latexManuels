\section{Priorités dans les calculs}

\begin{definition}
   \begin{itemize}
      \item Lorsqu'on effectue l'addition de deux {\bf termes}, le résultat est une {\bf somme}.
      \item Lorsqu'on effectue la soustraction de deux {\bf termes}, le résultat est une {\bf différence}.
      \item Lorsqu'on effectue la multiplication de deux {\bf facteurs}, le résultat est un {\bf produit}.
      \item Lorsqu'on effectue la division d'un {\bf dividende} par un {\bf diviseur}, le résultat est un {\bf quotient}. \\ [-10mm]
   \end{itemize}
\end{definition}

\bigskip

{\setlength{\tabcolsep}{4pt}
\begin{tabular}{*{7}{c}|*{7}{c}|*{8}{c}|*{10}{c}}
   12 & $+$ & 3 & = &15 &&&& 12 & $-$ & 3 & = & 9 &&& 12 & $\times$ & 3 & = & 36 &&&&& 12 & $\div$ & 3  & = & $\dfrac{12}{3}$ & = & 4 & \\
   \multicolumn{3}{c}{$\nwarrow \quad \nearrow$} & & $\uparrow$ &&&& \multicolumn{3}{c}{$\nwarrow \quad \nearrow$} & & $\uparrow$ &&& \multicolumn{3}{c}{$\nwarrow \quad \nearrow$} & & $\uparrow$ &&&&& $\uparrow$ & & $\uparrow$ & & & & $\uparrow$ & \\
   \multicolumn{3}{c}{\small termes} & \multicolumn{3}{c}{\small somme} &&& \multicolumn{3}{c}{\small termes} & \multicolumn{3}{c|}{\small différence} && \multicolumn{3}{c}{\small facteurs} & \multicolumn{4}{c|}{\small produit} & \multicolumn{3}{c}{\small dividende} & \multicolumn{3}{c}{\small diviseur} & & \multicolumn{3}{c}{\small quotient} \\
\end{tabular}}

\subsection{Calculs sans parenthèses}

\begin{propriete}[\admise]
Dans une suite d'additions et de soustractions \textbf{sans parenthèses}, on effectue les opérations de gauche à droite, c'est à dire dans le sens d'écriture, les unes après les autres.
\end{propriete}

\begin{exemple}[0.35]
   Calculer les expressions suivantes, en soulignant chaque fois l'opération à effectuer.
   
   $A=4-3+7-4$ 
   
   $B=\num{5.1}-\num{3.5}+\num{3.54}$
   \correction
   
   \smallskip
   \begin{minipage}{0.52\linewidth}
      \begin{list}{}{}
               \item $A=\underline{4-3}+7-4$
               \item $A=\underline{1+7}-4$
               \item $A=\underline{8-4}$
               \item $A=\psshadowbox{4}$             
      \end{list}
   \end{minipage}      
   \begin{minipage}{0.47\linewidth}
      \begin{list}{}{}
         \item $B=\underline{5,1-3,5}+3,54$
         \item $B=\underline{1,6+3,54}$
         \item $B=\psshadowbox{5,14}$
      \end{list}
   \end{minipage}   
\end{exemple}

\begin{propriete}[\admise]   
   Dans une suite de multiplications et de divisions \textbf{sans parenthèses}, on effectue les opérations de gauche à droite, c'est à dire dans le sens d'écriture, les unes après les autres.
\end{propriete}

\begin{exemple}[0.35]
   Calculer les expressions suivantes, en soulignant chaque fois l'opération à effectuer.
   
   $A=4\times 3\times 7\div 4$
   
   $B=55\div 11\times 3$
   \correction
   
   \bigskip
   \begin{minipage}{0.54\linewidth}
      \begin{list}{}{}
               \item $A=\underline{4\times 3}\times 7\div 4$
               \item $A=\underline{12\times 7}\div 4$
               \item $A=\underline{84\div 4}$
               \item $A=\psshadowbox{21}$               
      \end{list}
   \end{minipage}      
   \begin{minipage}{0.45\linewidth}
      \begin{list}{}{}
         \item $B=\underline{55\div 11}\times 3$
         \item $B=\underline{5\times 3}$
         \item $B=\psshadowbox{15}$
      \end{list}
   \end{minipage}   
\end{exemple}

\begin{propriete}[\admise]
   Dans une suite d'opérations \textbf{sans parenthèses}, les multiplications et les divisions sont prioritaires ( on les calcule en premier) par rapport aux additions et aux soustractions.
\end{propriete}

\begin{exemple}[0.35]
   Calculer les expressions suivantes, en soulignant chaque fois les opérations à effectuer.
   
   $A=10,5+8\times 3$
   
   $B=8,5+4\div 2-4\times 2,4$
   
   $C=2\times 5+3\times 8+7\times 9$
   \correction

   \smallskip
   \begin{minipage}{0.54\linewidth}
      \begin{list}{}{}
         \item $A=10,5+\underline{8\times 3}$
         \item $A=\underline{10,5+24}$
         \item $A=\psshadowbox{34,5}$               
      \end{list}
   \end{minipage}      
   \begin{minipage}{0.44\linewidth}
      \begin{list}{}{}
         \item $B=8,5+\underline{4\div 2}-\underline{4\times 2,4}$
         \item $B=\underline{8,5+2}-9,6$
         \item $B=\underline{10,5-9,6}$
         \item $B=\psshadowbox{0,9}$         
      \end{list}
   \end{minipage}
   \begin{minipage}{0.5\linewidth}
      \begin{list}{}{}
         \item $C=\underline{2\times 5}+\underline{3\times 8}+\underline{7\times 9}$
         \item $C=\underline{10+24}+63$
         \item $C=\underline{34+63}$
         \item $C=\psshadowbox{97}$         
      \end{list}
   \end{minipage}   
\end{exemple}

\pagebreak
\subsection{Calculs avec parenthèses}

\begin{propriete}[\admise]
   Dans une suite d'opérations \textbf{avec des parenthèses}, on commence par effectuer les suites d'opérations qui sont dans les parenthèses en commençant par les parenthèses les plus intérieures.
\end{propriete}

\begin{exemple}[0.35]
   Calculer les expressions suivantes, en soulignant chaque fois les opérations à effectuer.
   
   $A=15\times(28-(3+3)\times 3)$ ;
   
   $B=64-(38,5-(9-1,5\times 3))$

   $C=14+(28+5\times (7-3\times 2))\times 4$ ;   
   \correction

   \bigskip
   \begin{minipage}{0.47\linewidth}
      \begin{list}{}{}
         \item $A=15\times(28-(\underline{3+3})\times 3)$
         \item $A=15\times(28-\underline{6\times 3})$
         \item $A=15\times(\underline{28-18})$
         \item $A=\underline{15\times 10}$
         \item $A=\psshadowbox{150}$
      \end{list}
   \end{minipage}      
   \begin{minipage}{0.52\linewidth}
      \begin{list}{}{}
         \item $B=64-(38,5-(9-\underline{1,5\times 3}))$
         \item $B=64-(38,5-(\underline{9-4,5}))$
         \item $B=64-(\underline{38,5-4,5})$
         \item $B=\underline{64-34}$
         \item $B=\psshadowbox{30}$
      \end{list}
   \end{minipage}

   \medskip
   \begin{minipage}{1\linewidth}
      \begin{list}{}{}
         \item $C=14+(28+5\times (7-\underline{3\times 2}))\times 4$
         \item $C=14+(28+5\times (\underline{7-6}))\times 4$
         \item $C=14+(28+\underline{5\times 1})\times 4$
         \item $C=14+(\underline{28+5})\times 4$
         \item $C=14+\underline{33\times 4}$
         \item $C=\underline{14+132}$
         \item $C=\psshadowbox{146}$      
      \end{list}
   \end{minipage}   
\end{exemple}

\bigskip

\begin{methode}[Priorités opératoires]
   Dans un calcul, on effectue dans l'ordre :
   \begin{itemize}
      \item les calculs entre parenthèses, en commençant par les plus intérieures ;
      \item les multiplications et les divisions ;
      \item les additions et soustractions.
   \end{itemize}
   Les calculs s'effectuent généralement de gauche à droite, mais une expression comportant uniquement des multiplications
   ou uniquement des additions peut s'effectuer dans l'ordre que l'on veut.
   \exercice
      Calculer la valeur de $A$ : \\
      $A =8\times5+3\times((15-9)\times2)$
   \correction
      $A =8\times5+3\times(\underline{(15-9)}\div2)$ \\
      $A =8\times5+3\times\underline{(\psframebox*[fillcolor=yellow]{6}\div2)}$ \\
      $A =\underline{8\times5}+\underline{3\times\psframebox*[fillcolor=yellow]{3}}$ \\
      $A =\underline{\psframebox*[fillcolor=yellow]{40}+\psframebox*[fillcolor=yellow]{9}}$ \\
      $A = \psframebox*[fillcolor=yellow]{49}$
\end{methode}

\begin{remarque}
   une expression qui figure au numérateur et/ou au dénominateur d'un quotient est considérée comme une expression entre parenthèses : \\
   $\dfrac{8+4}{3,5+2,5} = (8+4)\div(3,5+2,5) =12\div6 =2$.
\end{remarque}