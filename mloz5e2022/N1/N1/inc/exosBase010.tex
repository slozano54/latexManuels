\begin{exercice}%10
   Compléter les calculs suivants pour que chaque égalité soit vraie.
   \begin{enumerate}
      \item Avec les signes $+, -$ ou $\times$ : \bigskip
      \begin{itemize}
         \item $3 \pointilles 3 \pointilles 3 \pointilles 3 = 6$ \bigskip
         \item $3 \pointilles 3 \pointilles 3 \pointilles 3 = 81$ \medskip
      \end{itemize}
      \item Avec les signes $+, -$ ou $\times$ et des parenthèses : \bigskip
      \begin{itemize}
         \item $\pointilles 3 \pointilles 3 \pointilles 3 \pointilles 3 \pointilles = 9$ \bigskip
         \item $\pointilles 3 \pointilles 3 \pointilles 3 \pointilles 3 \pointilles = 27$ \medskip
      \end{itemize}
      \item Avec les signes $+, -,\times$ ou $\div$ et des parenthèses : \bigskip
      \begin{itemize}
         \item $\pointilles 3 \pointilles 3 \pointilles 3 \pointilles 3 \pointilles = 1$ \bigskip
         \item $\pointilles 3 \pointilles 3 \pointilles 3 \pointilles 3 \pointilles = 12$
      \end{itemize}
   \end{enumerate}
\end{exercice}

\begin{corrige}
   \ \\ [-5mm]
   \begin{enumerate}
      \item $3 {\blue \,+\,} 3 {\blue \,+\,} 3 {\blue \,-\,} 3 = 6$ \\
         $3 {\blue \,\times\,} 3  {\blue \,\times\,} 3  {\blue \,\times\,} 3 = 81$ \smallskip
      \item ${\blue (} 3 {\blue \,+\,} 3 {\blue \,-\,} 3 {\blue )\,\times\,} 3 = 9$ \\
         ${\blue (} 3 {\blue \,+\,} 3 {\blue \,+\,} 3 {\blue )\,\times\,} 3 = 27$ \smallskip
      \item ${\blue (} 3 {\blue \,+\,} 3 {\blue \,-\,} 3 {\blue )\,\div\,} 3 = 1$ \\
         ${\blue (} 3 {\blue \,+\,} 3 {\blue \,\div\,} 3 {\blue )\,\times\,} 3 = 12$
   \end{enumerate}
\end{corrige}