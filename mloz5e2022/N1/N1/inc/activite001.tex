\begin{activite}[Construction et repérage d'une droite graduée]
    {\bf Objectifs :} comprendre et utiliser le principe de construction d'une graduation en dixièmes et en centièmes ; savoir situer des nombres décimaux sous différentes écritures ; ordonner, encadrer, intercaler des nombres décimaux.
    \partie[construction d'une droite graduée]
        \begin{enumerate}
            \item Tracer au stylo une droite la plus longue possible sur \textbf{la bande de papier fournie}\footnote{Au moins 6 feuilles de brouillon A4 scotchées dans la largeur du côté imprimé puis découpées en longueur.}.
            \item Placer à gauche sur cette droite le repère de l'origine, inscrire la valeur 0 en dessous. 
            \item Grâce à la petite bande de couleur \og $\dfrac1{10}$ \fg, qui correspond à un dixième d'une unité, placer le nombre $1$.
            \begin{center}
            \begin{pspicture}(-1,-0.25)(5,1.75)
                \multido{\n=0+0.5}{11}{\psline(\n,0)(\n,0.5)}
                \psframe[fillstyle=solid,fillcolor=C1](0,0.5)(5,1.5)
                \psline(0,0)(5,0)
                \rput(2.5,1){\white\small $\dfrac{1}{10}$}
            \end{pspicture}
            \end{center}
            \item Placer ensuite les nombres $2$ et $3$, toujours en dessous de la droite.
        \end{enumerate}
    \partie[placer des nombres décimaux sur la droite graduée]
        \begin{enumerate}
            \item Sur la droite graduée, placer au crayon à papier et au-dessus les nombres suivants : \\ [1mm]
            $\dfrac{8}{10} \hfill\text{;}\hfill 0,3 \hfill\text{;}\hfill \text{cinq dixièmes} \hfill\text{;}\hfill \dfrac{23}{10} \hfill\text{;}\hfill 1,7 \hfill\text{;}\hfill 2+\dfrac{1}{10} \hfill\text{;}\hfill \text{douze dixièmes} \hfill\text{.}\hfill$ \smallskip
            \item Trouver un moyen pour placer $\dfrac{143}{100}$ sur la droite graduée. \smallskip
            \item Placer au crayon les nombres suivants : $\hfill \dfrac{255}{100} \hfill\text{;}\hfill 0,23 \hfill\text{;}\hfill \text{cent-six centièmes} \hfill\text{;}\hfill 1+\dfrac{9}{10}+\dfrac{8}{100} \hfill\text{.}\hfill$
        \end{enumerate}
    \partie[ordonner, encadrer, intercaler des nombres décimaux]
        \begin{enumerate}
            \item Écrire dans l'ordre croissant les nombres inscrits sur la droite graduée.
            
            \medskip\dotfill
            \pagebreak            
            \item Encadrer chacun des nombres suivants par deux nombres entiers consécutifs.
            \item 
                \makebox[0.35\linewidth]{\dotfill} < \makebox[0.2\linewidth]{$\dfrac{8}{10}$} <                   \makebox[0.35\linewidth]{\dotfill} \par\bigskip
                \makebox[0.35\linewidth]{\dotfill} < \makebox[0.2\linewidth]{0,3} <                               \makebox[0.35\linewidth]{\dotfill} \par\medskip
                \makebox[0.35\linewidth]{\dotfill} < \makebox[0.2\linewidth]{cinq dixièmes} <                     \makebox[0.35\linewidth]{\dotfill} \par\bigskip
                \makebox[0.35\linewidth]{\dotfill} < \makebox[0.2\linewidth]{$\dfrac{23}{10}$} <                  \makebox[0.35\linewidth]{\dotfill} \par\bigskip
                \makebox[0.35\linewidth]{\dotfill} < \makebox[0.2\linewidth]{1,7} <                               \makebox[0.35\linewidth]{\dotfill} \par\medskip
                \makebox[0.35\linewidth]{\dotfill} < \makebox[0.2\linewidth]{$2+\dfrac{1}{10}$} <                 \makebox[0.35\linewidth]{\dotfill} \par\bigskip
                \makebox[0.35\linewidth]{\dotfill} < \makebox[0.2\linewidth]{douze dixièmes} <                    \makebox[0.35\linewidth]{\dotfill} \par\bigskip
                \makebox[0.35\linewidth]{\dotfill} < \makebox[0.2\linewidth]{$\dfrac{143}{100}$} <                \makebox[0.35\linewidth]{\dotfill} \par\bigskip
                \makebox[0.35\linewidth]{\dotfill} < \makebox[0.2\linewidth]{$\dfrac{255}{100}$} <                \makebox[0.35\linewidth]{\dotfill} \par\bigskip
                \makebox[0.35\linewidth]{\dotfill} < \makebox[0.2\linewidth]{0,23} <                              \makebox[0.35\linewidth]{\dotfill} \par\medskip
                \makebox[0.35\linewidth]{\dotfill} < \makebox[0.2\linewidth]{106 centièmes} <                     \makebox[0.35\linewidth]{\dotfill} \par\bigskip
                \makebox[0.35\linewidth]{\dotfill} < \makebox[0.2\linewidth]{$1+\dfrac{9}{10}+\dfrac{8}{100}$} <  \makebox[0.35\linewidth]{\dotfill}

            \medskip
            \item Intercaler un nombre vérifiant chacune des inégalités.
            
            \medskip
            \makebox[0.2\linewidth]{cinq dixièmes}  < \makebox[0.35\linewidth]{\dotfill} <   \makebox[0.2\linewidth]{$\dfrac{8}{10}$} \\
            \makebox[0.2\linewidth]{$2$}            < \makebox[0.35\linewidth]{\dotfill} <   \makebox[0.2\linewidth]{$2+\dfrac{1}{10}$} \\
            \makebox[0.2\linewidth]{$0,23$}         < \makebox[0.35\linewidth]{\dotfill} <   \makebox[0.2\linewidth]{$0,3$}
        \end{enumerate}
    \vspace*{-3mm}
    \vfill\hrule\hfill{\it\footnotesize Source : Apprentissages numériques et résolution de problèmes au CM2, Ermel, Hatier 2001}.
 \end{activite}
 