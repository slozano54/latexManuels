\vspace*{-7mm}
\begin{changemargin}{-15mm}{-15mm} 
%pre-001
\begin{prerequis}[Connaisances \emoji{red-heart} et compétences \emoji{diamond-suit} du cycle 3]    
   \begin{itemize}        
       \item[\emoji{red-heart}] Vocabulaire associé à ces objets et à leurs propriétés : côté, sommet, angle, hauteur.
       \columnbreak
       \item[\emoji{diamond-suit}] Reconnaître, nommer, décrire des triangles, dont les triangles particuliers (triangle rectangle, triangle isocèle, triangle équilatéral).       
   \end{itemize}
\end{prerequis}
%pre-002
\begin{prerequis}[Connaisances \emoji{red-heart} et compétences \emoji{diamond-suit} du cycle 4]    
    \begin{itemize}        
        \item[\emoji{diamond-suit}] Mener des calculs impliquant des grandeurs mesurables, exprimer les résultats dans des les unités adaptées.
        \item[\emoji{diamond-suit}] Exprimer et vérifier la cohérence des résultats du point de vue des unités.
    \end{itemize}
\end{prerequis}
\end{changemargin}
\vspace*{-11mm}
\begin{debat}[Un peu d'histoire]   
   \begin{changemargin}{-15mm}{-15mm}
   Le système de numération que nous employons actuellement et qui nous semble si naturel est le fruit d'une longue évolution des concepts mathématiques. En effet, un nombre est une entité abstraite qui peut surprendre : on a déjà vu {\bf un} élève, {\bf un} animal donné, on sait ce qu'est {\bf un} jour, mais qu'est-ce que {\bf un} ? C'est une entité qui, prise seule, n'a pas vraiment de sens. De nombreuses civilisations ont imaginé des systèmes de numération plus ou moins compliqués, plus ou moins pratiques : des systèmes utilisant des bases différentes, des systèmes utilisant le principe additif\dots{} jusqu'à notre système de numération positionnel de base dix maintenant utilisé de manière universelle.
   \end{changemargin}
   \begin{center}
      \textcolor{B1}{{\huge 19\textcircled{\Large 0}1\textcircled{\Large 1}7\textcircled{\Large2}8\textcircled{\Large 3}} \\
      \it Notation décimale de Simon Stevin représentant le nombre 19,178.}
   \end{center}   
   \begin{cadre}[B2][J4]
      \begin{center}
         % \hrefVideo{https://www.yout-ube.com/watch?v=bkGMa1EJkSA}{\bf Histoire de la virgule}, chaîne Youtube de {\it Maths 28}.
         \hrefVideo{https://www.yout-ube.com/watch?v=2s2HSdtzBzg}{A'Rieka Enchaînements d'opérations}
      \end{center}
   \end{cadre}
\end{debat}