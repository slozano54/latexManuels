\begin{exercice*}
    Colorier le minimum de cases pour que chacune des figures ci-dessous admette le point $O$ pour centre de symétrie.
    {\psset{unit=0.45,subgriddiv=0,gridlabels=0,gridcolor=darkgray}
    \begin{center}
       \begin{pspicture}(0,0)(8,8)
         \psgrid[subgriddiv=0,gridlabels=0](0,0)(8,8)
         \pstGeonode[PosAngle=-45](4,4){O}
         \psset{fillstyle=solid,fillcolor=black!80}
         \psframe(0,0)(1,1)
         \rput(5,7){\psframe(0,0)(1,1)}
         \rput(6,3){\psframe(0,0)(1,1)}
         \rput(3,2){\psframe(0,0)(1,1)}
         \psset{fillcolor=gray!50}
         \rput(1,5){\psframe(0,0)(1,1)}
         \rput(4,4){\psframe(0,0)(1,1)}
         \rput(5,2){\psframe(0,0)(1,1)}
         \rput(3,7){\psframe(0,0)(1,1)}   
      \end{pspicture}
      \;
      \begin{pspicture}(0,0)(8,8)        
         \psset{fillstyle=solid,fillcolor=black!80}
         \psframe(0,0)(1,3)
         \rput(6,3){\psframe(0,0)(1,2)}
         \rput(2,2){\psframe(0,0)(2,1)}
         \rput(1,5){\psframe(0,0)(1,2)}
         \rput(4,4){\psframe(0,0)(3,1)}
         \rput(5,2){\psframe(0,0)(1,2)}
         \rput(3,7){\psframe(0,0)(4,1)}
         \psgrid[subgriddiv=0,gridlabels=0](0,0)(8,8)
         \pstGeonode[PosAngle=-45](4,4){O}
      \end{pspicture}
    \end{center}}

    \hrefMathalea{https://coopmaths.fr/alea/?uuid=2a611&id=5G11-6&n=3&d=10&s2=4&cd=1&v=eleve&title=Exercices&es=3211}
\end{exercice*}
\begin{corrige}
    %\setcounter{partie}{0} % Pour s'assurer que le compteur de \partie est à zéro dans les corrigés
    \ \\ [2mm]
    {\psset{unit=0.45,subgriddiv=0,gridlabels=0,gridcolor=darkgray}
    \begin{pspicture}(1.4,0)(9,8)
      \psgrid[subgriddiv=0,gridlabels=0](0,0)(8,8)
      \pstGeonode[PosAngle=-45](4,4){O}
      \psset{fillstyle=solid,fillcolor=black!80}
      \psframe(0,0)(1,1)
      \rput(5,7){\psframe(0,0)(1,1)}
      \rput(6,3){\psframe(0,0)(1,1)}
      \rput(3,2){\psframe(0,0)(1,1)}
      \rput(4,5){\psframe(0,0)(1,1)}
      \rput(2,0){\psframe(0,0)(1,1)}
      \rput(1,4){\psframe(0,0)(1,1)}
      \rput(7,7){\psframe(0,0)(1,1)}
      \psset{fillcolor=gray!50}
      \rput(1,5){\psframe(0,0)(1,1)}
      \rput(4,4){\psframe(0,0)(1,1)}
      \rput(5,2){\psframe(0,0)(1,1)}
      \rput(3,7){\psframe(0,0)(1,1)}  
      \rput(3,3){\psframe(0,0)(1,1)}
      \rput(6,2){\psframe(0,0)(1,1)}
      \rput(4,0){\psframe(0,0)(1,1)}
      \rput(2,5){\psframe(0,0)(1,1)}
   \end{pspicture}
   \begin{pspicture}(0,0)(8,8)        
      \psset{fillstyle=solid,fillcolor=black!80}
      \pspolygon(0,0)(5,0)(5,1)(1,1)(1,3)(0,3)
      \pspolygon(1,3)(2,3)(2,2)(4,2)(4,4)(3,4)(3,6)(2,6)(2,7)(1,7)
      \pspolygon(3,7)(3,8)(8,8)(8,5)(7,5)(7,7)
      \pspolygon(4,4)(4,6)(6,6)(6,5)(7,5)(7,1)(6,1)(6,2)(5,2)(5,4)
      \psgrid[subgriddiv=0,gridlabels=0](0,0)(8,8)
      \pstGeonode[PosAngle=-45](4,4){O}
   \end{pspicture}}
\end{corrige}

