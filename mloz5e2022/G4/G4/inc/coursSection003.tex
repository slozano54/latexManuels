\section{Centre de symétrie}
\begin{definition}
    Si une figure $\cal F$ est transformée en elle-même par la symétrie centrale de centre $O$, alors $O$ est le \textbf{centre de symétrie} de la figure $\cal F$.
 \end{definition}
 
 \begin{exemple*1}
    Si une figure possède un centre de symétrie, alors il est unique.
    {\psset{unit=0.75}
    \begin{multicols}{3}
       \begin{center}
          \begin{pspicture}(-2.5,-0.5)(2.5,2.3)
             \pstGeonode[PointName=none,PointSymbol=none,CurveType=polygon](-1.156,0){A}(1.156,0){B}(0,2){C}
             \pstGeonode[PointName=none,PointSymbol=none,CurveType=polygon,linecolor=B1](0,-0.5){D}(0,2.5){E}
             \psline[linecolor=B1](-1.156,0)(2;50)
             \psline[linecolor=B1](1.156,0)(2;130)
          \end{pspicture}
    
          3 axes de symétrie \\
          0 centre de symétrie
          \begin{pspicture}(-2.5,-0.5)(2.5,2.3)
             \pscircle(0,1){1}
             \psset{linecolor=B1}
             \psline(-1.5,1)(1.5,1)
             \psline(0,-0.5)(0,2.5)
             \psline(-1.5,-0.5)(1.5,2.5)
             \psline(-1.5,2.5)(1.5,-0.5)
             \psline(-0.5,-0.5)(0.5,2.5)
             \psline(-0.5,2.5)(0.5,-0.5)
             \psline(-1.5,0)(1.5,2)
             \psline(-1.5,0.5)(1.5,1.5)
             \psline(-1.5,2)(1.5,0)
             \psline(-1.5,1.5)(1.5,0.5)
             \psdot[linecolor=A1,linewidth=1mm](0,1)  
          \end{pspicture}
    
          $\infty$ axes de symétrie \\
          1 centre de symétrie  
       \end{center}
    
       \begin{pspicture}(-2.5,-0.5)(2.5,2.3)
          \pspolygon(-1.5,0)(1.5,0)(1.5,2)(-1.5,2)
          \psset{linecolor=B1}
          \psline(-2,1)(2,1)
          \psline(0,-0.5)(0,2.5)
          \psdot[linecolor=A1,linewidth=1mm](0,1)  
       \end{pspicture}
    
       \quad 2 axes de symétrie
       
       \quad 1 centre de symétrie
    \end{multicols}}
 \vspace*{-5mm}
 \end{exemple*1}