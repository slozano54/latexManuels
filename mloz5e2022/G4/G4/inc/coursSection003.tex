\section{Centre de symétrie}
\begin{definition}
    Si une figure $\cal F$ est transformée en elle-même par la symétrie centrale de centre $O$, alors $O$ est le \textbf{centre de symétrie} de la figure $\cal F$.
 \end{definition}
 
 \begin{exemple*1}
    Si une figure possède un centre de symétrie, alors il est unique.
    {\psset{unit=0.75}
    \begin{multicols}{3}
       \begin{center}
         \begin{Geometrie}[CoinHD={u*(3,2.5)}]
            pair A[];
            A1=u*(0.5,0.5);
            A2-A1=u*(2,0);
            A3=rotation(A2,A1,60);
            trace polygone(A1,A2,A3);
            trace droite(A1,iso(A2,A3)) withcolor red;
            trace droite(A2,iso(A1,A3)) withcolor red;
            trace droite(A3,iso(A2,A1)) withcolor red;
         \end{Geometrie}
    
          3 axes de symétrie \\
          0 centre de symétrie
          \begin{Geometrie}[CoinHD={u*(3,2.5)}]
            path co;
            pair O,A[];
            O=u*(1.5,1.25);
            co=cercles(O,u);
            trace co;
            A1=pointarc(co,0);
            A2=pointarc(co,20);
            A3=pointarc(co,30);
            A4=pointarc(co,40);
            A5=pointarc(co,70);
            A6=pointarc(co,90);
            A7=pointarc(co,110);
            A8=pointarc(co,140);
            A9=pointarc(co,150);
            A10=pointarc(co,160);
            trace droite(O,A1) withcolor red;
            trace droite(O,A2) withcolor red;
            trace droite(O,A3) withcolor red;
            trace droite(O,A4) withcolor red;
            trace droite(O,A5) withcolor red;
            trace droite(O,A6) withcolor red;
            trace droite(O,A7) withcolor red;
            trace droite(O,A8) withcolor red;
            trace droite(O,A9) withcolor red;
            trace droite(O,A10) withcolor red;
            draw O withpen pencircle scaled 4bp withcolor blue;
         \end{Geometrie}
    
          $\infty$ axes de symétrie \\
          1 centre de symétrie
          \begin{Geometrie}[CoinHD={u*(3,2.5)}]
            path co;
            pair O,A[];            
            A1=u*(0.5,0.5);
            A2-A1=u*(2,0);
            A3-A2=u*(0,1.2);
            A4-A3=u*(-2,0);
            O=iso(A1,A2,A3,A4);
            trace polygone(A1,A2,A3,A4);
            trace droite(iso(A1,A2),iso(A3,A4)) withcolor red;
            trace droite(iso(A1,A4),iso(A3,A2)) withcolor red;
            draw O withpen pencircle scaled 4bp withcolor blue;
         \end{Geometrie}            
    
         2 axes de symétrie\\       
         1 centre de symétrie
      \end{center}
    \end{multicols}}
 \vspace*{-5mm}
 \end{exemple*1}