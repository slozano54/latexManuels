\begin{activite}[Le bonhomme inversé]
    \begin{changemargin}{-10mm}{-10mm}
    {\bf Objectifs :} transformer une figure par symétrie axiale ; observer l'effet de deux symétries axiales. \\ 
       Construire sur le quadrillage ci-dessous les bonhommes demandés.
       \begin{enumerate}
          \item Construire en vert le symétrique du bonhomme par rapport à la droite $(d)$. 
          \item Construire en rouge le symétrique du bonhomme vert par rapport à
 la droite $(\Delta)$.
          \item Reproduire sur du papier calque le bonhomme noir et le point $O$.
          \item En s'aidant du calque, sans le plier, trouver comment passer du bonhomme noir au bonhomme rouge.
          \item Sans utiliser les bonhommes noir et rouge ni les droites $(d)$ et $(\Delta)$, construire en bleu l'image du bonhomme vert par la symétrie centrale de centre $O$. \\
       \end{enumerate}
       \begin{center}
       {\psset{unit=0.6}
          \begin{pspicture}(-11,-11)(11,11)
             \psgrid[subgriddiv=0,gridlabels=0,gridcolor=gray](-11,-11)(11,11)
             \psline[linewidth=0.5mm](-11,0)(11,0)
             \psline[linewidth=0.5mm](0,-11)(0,11)
             \rput(0.5,-0.5){$O$}
             \rput(10.5,0.5){$(\Delta)$}
             \rput(0.7,10.5){$(d)$}
             \psset{linewidth=1mm}
             \psline(2,1)(3,1)(5,4)(7,1)(8,1)
             \psline(5,4)(5,7)
             \psframe(4,7)(6,9)
             \psline(3,5)(7,6)
             \psdots(3,5)(7,6)(4.5,8.5)
             \psline(5.25,8.5)(5.75,8.5)
             \psarc(5,8){0.5}{180}{0}
             \psline(3,9)(7,9)
             \psarc(5,9){1}{0}{180}
          \end{pspicture}}
       \end{center}
   \hfill{\it\footnotesize Source : \href{https://irem.univ-lille1.fr/IMG/pdf/fiche_eleve_symetrie.pdf}{À la découverte de la symétrie centrale, Nathalie Bernard, IREM de Lille.}}
\end{changemargin}
\end{activite}
