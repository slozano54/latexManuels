% Les enigmes ne sont pas numérotées par défaut donc il faut ajouter manuellement la numérotation
% si on veut mettre plusieurs enigmes
% \refstepcounter{exercice}
% \numeroteEnigme
\begin{changemargin}{-10mm}{-10mm}
\begin{enigme}[Le tapis mendiant]
    \partie[kesako ?]
       Un patchwork est une technique décorative de couture qui consiste à assembler des morceaux de tissus pour réaliser des objets divers, comme par exemple des taies d’oreillers ou des couvre-lits. \\
       On s'intéresse ici aux couvre-lits de Cilaos, (ville de la Réunion), plus communément appelés \og tapi mendian \fg. \\
       Le motif de base de ce tapis est un hexagone, et pour le réaliser, on assemble sept hexagones pour former une rosace, puis, on coud les rosaces entre elles.
       \begin{center}
          \includegraphics[height=4.5cm]{\currentpath/images/rosace}
          \qquad
          \includegraphics[height=4.5cm]{\currentpath/images/tapimendian}
       \end{center}
 
    \partie[construction d'un hexagone régulier]
       \begin{minipage}{9cm}
          \begin{enumerate}
             \item Qu'est-ce qu'un hexagone régulier ?
             \item Combien d'axe(s) et de centre de symétrie possède-t-il ?
             \item Construire le plus précisément un hexagone régulier.
          \end{enumerate}
       \end{minipage}
       \begin{minipage}{5cm}
          \begin{pspicture}(-3,-1)(2,1)
             \hexa{1.5}{White}
          \end{pspicture}
       \end{minipage}
    
    \partie[construction d'une rosace]
    \ \\ [-10mm]
       \begin{enumerate}
          \item Construire une rosace, dont le diamètre du cercle circonscrit au motif de base est de \ucm{6}.
          \item Décorer la rosace (couleurs, dessins, textes,motifs\dots), puis la découper.
          \item Combien d'axe(s) et de centre de symétrie possède-t-elle (avant décoration et après décoration) ?
       \end{enumerate}
       \begin{center}
            {\psset{unit=0.8}
          \begin{pspicture}(-2.5,-2.75)(2.5,2.5)
             \hexa{1}{lightgray}
             \rput(1.73;30){\hexa{1}{yellow!70}}
             \rput(1.73;90){\hexa{1}{orange!70}}
             \rput(1.73;150){\hexa{1}{red!70}}
             \rput(1.73;-30){\hexa{1}{Green!70}}
             \rput(1.73;-90){\hexa{1}{blue!70}}
             \rput(1.73;-150){\hexa{1}{violet!70}}
          \end{pspicture}}
       \end{center}
       
    \partie[construction du \og tapi mendian \fg]
       Assembler toutes les rosaces de la classe afin de créer un tapis mendiant.
 \end{enigme} 
 \vfill\hfill{\it\small Activité inspirée de \href{https://irem.univ-reunion.fr/spip.php?article786}{\og Les patchworks de Cilaos : enseignement et ethnogéométrie au collège \fg}, IREM de la Réunion}
\end{changemargin}

% Pour le corrigé, il faut décrémenter le compteur, sinon il est incrémenté deux fois
% \addtocounter{exercice}{-1}
% \begin{corrige}
%     \ldots
% \end{corrige}