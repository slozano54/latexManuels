\section{Symétrie centrale}
\begin{minipage}{0.5\linewidth}
\begin{center}
    \hrefLien{https://www.geogebra.org/m/fbfhreb8}{Geogebra d'Homer SIMPSON}

    \begin{Geometrie}[CoinHD={(8.4u,4.2u)}]
        % Echelle
        coeff:=0.7;
        u:=coeff*cm;        
        %%%
        trace grille(0.5) withcolor white;        
        pair O,A,B,C,D,E,F,G,H,I;
        pair A',B',C',D',E',F',G',H',I';
        O=u*(6,3);
        A=u*(3,5);
        B=u*(5,5);
        C=u*(4,4);
        D=u*(5,3);
        E=u*(5,1);
        F=u*(4,2);
        G=u*(3,1);
        H=u*(1,1);
        I=u*(3,3);        
        A'=rotation(A,O,180);
        B'=rotation(B,O,180);
        C'=rotation(C,O,180);
        D'=rotation(D,O,180);
        E'=rotation(E,O,180);
        F'=rotation(F,O,180);
        G'=rotation(G,O,180);
        H'=rotation(H,O,180);
        I'=rotation(I,O,180);
        trace polygone(A,B,C,D,E,F,G,H,I,A);
        trace chemin(A',B',C',D',E',F',G',H',I',A') withcolor \myMetapostGreen;
        marque_p:="croix";
        pointe(O);
        label.bot(btex O etex, O);
        label.lft(btex A etex, A);
        label.rt(btex B etex, B);
        label.rt(btex C etex, C);
        label.rt(btex D etex, D);
        label.bot(btex E etex, E);  
        label.rt(btex A' etex, A');
        label.lft(btex B' etex, B');
        label.lft(btex C' etex, C');
        label.lft(btex D' etex, D');
        label.lft(btex E' etex, E');
    \end{Geometrie}    
\end{center}
\end{minipage}
\begin{minipage}{0.6\linewidth}
\begin{changemargin}{0mm}{-5mm}
    \begin{vocabulaire}
        \begin{itemize}    
            \item La \textcolor{mygreen}{figure verte} est obtenue à partir de la \textcolor{black}{figure noire} par un \textbf{demi-tour} autour du point $O$.
            \item On dit que La \textcolor{mygreen}{figure verte} est la \textbf{symétrique} de la \textcolor{black}{figure noire} \textbf{par rapport au point $O$}.
            \item On peut également dire qu'elle est \textbf{l'image} de la figure noire \textbf{par la symétrie de centre $O$} .
            \item $O$ est un \textbf{centre de symétrie}.
        \end{itemize}
    \end{vocabulaire}
\end{changemargin}
\end{minipage}

\begin{definition}
    Le point $M'$ est l'image du point $M$ par la \textbf{symétrie de centre $O$} lorsque le point $O$ est le milieu du segment $[MM']$.
 \end{definition}