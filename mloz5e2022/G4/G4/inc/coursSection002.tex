\section{Constructions}
% Point
\begin{methode}[Symétrique d'un point par symétrie centrale]
    Pour construire le symétrique d'un point $M$ par la symétrie centrale de centre $O$ :
    \begin{itemize}
       \item tracer la demi-droite $[MO)$ ;
       \item reporter la longueur $MO$ de l'autre côté du point $O$ avec le compas ;
       \item placer le point $M'$ symétrique de $M$ par rapport à $O$.
    \end{itemize}
    \exercice
    Tracer le point $M'$, symétrique du point $M$ par rapport à $O$. 
    
    \bigskip
    \begin{Geometrie}[CoinHD={(5u,3u)}]
        pair M,O,M';
        M=u*(1,2.5);
        O=u*(2.5,1.75);
        M'=u*(4,1);
        marque_s:=2;
        marque_p:="croix";        
        pointe(M,O);
        label.lft(btex M etex, M);
        label.rt (btex O etex, O);
    \end{Geometrie}
    \correction
    \begin{Geometrie}[CoinHD={(5u,5u)}]
        pair M,O,M';
        M=u*(1,2.5);
        O=u*(2.5,1.75);
        M'=u*(4,1);
        marque_s:=2;
        trace demidroite (M,O);
        trace codesegments (M,O,O,M',2);        
        trace compas(O,M',1);
        marque_p:="croix";        
        pointe(M,O,M');
        label.lft(btex M etex, M);
        label.bot (btex O etex, O);
        label.bot (btex M' etex, M');
    \end{Geometrie}
\end{methode}

\begin{remarques}
    \begin{itemize}
        \item $O$ est le milieu du segment $[MM']$.
        \item $O$ est lui-même sont image !
    \end{itemize}
\end{remarques}

% Droite
\begin{methode*1}[Symétrique d'une droite par symétrie centrale]
    Pour construire le symétrique d'une droite $(d)$ par la symétrie centrale de centre $O$ :
    \begin{itemize}
        \item Placer deux points $A$ et $B$ sur la droite $(d)$.
        \item Construire les symétriques $A'$ et $B'$ de $A$ et $B$ par rapport à $O$.
        \item Tracer la droite $(A'B')$.
    \end{itemize}
    \exercice
    Tracer la droite $(d')$, symétrique de la droite $(d)$ par rapport à $O$. 
    
    \bigskip
    \begin{Geometrie}[CoinHD={(8u,4u)}]
        pair A,B,O,D;
        B=u*(5.5,3.5);
        A=u*(1.5,3.1);
        O=u*(4,2.5);
        D=u*(7,3.65);
        label.top(btex $(d)$ etex,D);
        trace droite(A,B);
        marque_p:="croix";
        pointe(O);
        label.top(btex O etex,O);
    \end{Geometrie}    
    \correction
    \phantom{rrr}

    \begin{Geometrie}[CoinHD={(8u,4u)}]
        pair A,B,C,O,A',B',C',D,D';
        C=u*(0.5,3);
        B=u*(5.5,3.5);
        A=u*(1.5,3.1);
        O=u*(4,2.5);
        A'=u*(6.5,1.9);
        B'=u*(2.5,1.5);
        C'=u*(7.5,2);
        D=u*(7,3.65);
        D'=u*(1,1.35);
        label.top(btex $(d)$ etex,D);
        label.top(btex $(d')$ etex,D');
        trace droite(B,C);
        trace droite(B',C');
        trace demidroite(C,O) withcolor red;
        trace demidroite(A,O) withcolor \myMetapostGreen;
        trace demidroite(B,O) withcolor \myMetapostGreen;
        marque_s:=2;
        trace codesegments (A,O,O,A',2) withcolor \myMetapostGreen;
        trace codesegments (B,O,O,B',3) withcolor \myMetapostGreen;
        marque_p:="croix";
        label.top(btex C  etex, C );
        label.top(btex A  etex, A );
        label.top(btex B  etex, B );
        label.top(btex O  etex, O );
        label.bot(btex B' etex, B');
        label.bot(btex A' etex, A');
        label.top(btex C' etex, C');
    \end{Geometrie}
\end{methode*1}

\begin{remarque}
    Pour construire le symétrique d'un point $C$ de $(d)$, connaissant $(d')$ la symétrique de $(d)$ par rapport à $O$, il suffit de construire l'intersection de $(d')$ et de $[CO)$.    
\end{remarque}

\begin{propriete}[\admise]
    Si deux droites sont symétriques par rapport à un point alors elles sont parallèles entre elles.
\end{propriete}

% Segment
\begin{methode}[Symétrique d'un segment par symétrie centrale]
    Pour construire le symétrique d'un segment $[AB]$ par la symétrie centrale de centre $O$ :
    \begin{itemize}
        \item Construire les symétriques $A'$ et $B'$ des extrémités $A$ et $B$ par rapport à $O$.
        \item Tracer $[A'B']$.
    \end{itemize}
    \exercice
    Tracer le segment $[A'B']$, symétrique du segment $[AB]$ par rapport à $O$.
    
    \bigskip
    \begin{Geometrie}[CoinHD={(8u,4u)}]
        pair A,B,O;
        B=u*(5.5,3.5);
        A=u*(1.5,3.1);
        O=u*(4,2.5);
        trace segment(A,B);
        trace marquesegment(A,B);
        marque_p:="croix";
        label.top(btex O etex, O);
        pointe(O);
        marque_p:="non";
        label.lft(btex A etex, A );
        label.rt(btex B etex, B );
    \end{Geometrie}
    \correction
    \begin{Geometrie}[CoinHD={(8u,4u)}]
        pair A,B,O,A',B';
        B=u*(5.5,3.5);
        A=u*(1.5,3.1);
        O=u*(4,2.5);
        A'=u*(6.5,1.9);
        B'=u*(2.5,1.5);
        trace segment(A,B);
        trace segment(A',B');
        trace demidroite(A,O) withcolor \myMetapostGreen;
        trace demidroite(B,O) withcolor \myMetapostGreen;
        trace marquesegment(A,B);
        trace marquesegment(A',B');
        marque_s:=2;
        trace codesegments (A,O,O,A',2) withcolor \myMetapostGreen;
        trace codesegments (B,O,O,B',3) withcolor \myMetapostGreen;
        marque_p:="croix";
        label.top(btex O etex, O);
        pointe(O);
        marque_p:="non";
        label.lft(btex A etex, A );
        label.rt(btex B etex, B );
        label.lrt(btex B' etex, B');
        label.llft(btex A'etex, A');
    \end{Geometrie}
\end{methode}

\begin{propriete}[\admise]
    Si deux segments sont symétriques par rapport à un point alors les droites qui les portent sont parallèles et ils ont la même longueur.
\end{propriete}

% Demi-droite
\begin{methode}[Symétrique d'une demi-droite par symétrie centrale]
    Pour construire le symétrique d'une demi-droite $[MN)$ par la symétrie centrale de centre $O$ :
    \begin{itemize}
        \item Construire le symétrique $M'$ de l'origine $M$ par rapport à $O$.
        \item Construire l'image d'un point de $[MN)$ par rapport à $O$, par exemple N.
        \item Tracer $[M'N')$
    \end{itemize}
    \exercice
    Tracer la demi-droite $[M'N')$, symétrique de la demi-droite $[MN)$ par rapport à $O$.
    
    \bigskip
    \begin{Geometrie}[CoinHD={(8u,4u)}]
        pair N,M,O;
        M=u*(5.5,3.5);
        N=u*(1.5,3.1);
        O=u*(4,2.5);
        trace demidroite(M,N);
        trace marquedemidroite(M,N);
        marque_s:=2;
        marque_p:="croix";
        label.top(btex O etex, O);
        pointe(O,N);
        marque_p:="non";
        label.top(btex N etex, N);        
        label.rt(btex M etex, M);
    \end{Geometrie}
    \correction
    \begin{Geometrie}[CoinHD={(8u,4u)}]
        pair N,M,O,N',M';
        M=u*(5.5,3.5);
        N=u*(1.5,3.1);
        O=u*(4,2.5);
        N'=u*(6.5,1.9);
        M'=u*(2.5,1.5);
        trace demidroite(M,N);
        trace demidroite(M',N');
        trace demidroite(M,O) withcolor \myMetapostGreen;
        trace demidroite(N,O) withcolor \myMetapostGreen;
        trace marquedemidroite(M,N);
        trace marquedemidroite(M',N');
        marque_s:=2;
        trace codesegments (M,O,O,M',2) withcolor \myMetapostGreen;
        trace codesegments (N,O,O,N',3) withcolor \myMetapostGreen;
        marque_p:="croix";
        label.top(btex O etex, O);
        pointe(O,N,N');
        marque_p:="non";
        label.top(btex N etex, N);
        label.rt(btex M etex, M);
        label.lft(btex M' etex, M');
        label.bot(btex N' etex, N');
    \end{Geometrie}
\end{methode}

\begin{propriete}[\admise]
    Si deux demi-droites sont symétriques par rapport à un point alors elles sont parallèles.
\end{propriete}

% Cercle
\begin{methode}[Symétrique d'un cercle par symétrie centrale]
    Pour construire le symétrique d'un cercle $(\mathcal{C})$ par la symétrie centrale de centre $O$ :
    \begin{itemize}
        \item Construire le symétrique $A'$ du centre $A$ de $(\mathcal{C})$ par rapport à $O$.
        \item Construire le cercle $(\mathcal{C}^{'})$ de centre $A'$ et de même rayon que $(\mathcal{C})$.
        \item {\bfseries Ou} l'image $B'$ d'un point $B$ de $(\mathcal{C})$ et tracer le cercle $(\mathcal{C}^{'})$ de centre $A'$ passant par $B'$.
    \end{itemize}
    \exercice
    Tracer le cercle $(\mathcal{C}^{'})$, symétrique du cercle $(\mathcal{C})$ par rapport à $O$.
    
    \bigskip
    \begin{Geometrie}[CoinHD={(8u,5u)}]
        pair A,B,O,C;
        A=u*(1.5,3.5);
        B=u*(3,3.5);
        O=u*(4,2.5);
        C=u*(1.5,2);
        label.bot(btex $({\mathcal C})$ etex,C);        
        path cc;
        cc=cercles(A,1.5u);
        trace cc;
        marque_s:=2;
        marque_p:="croix";
        pointe(O,A,B);
        label.top(btex O etex, O);
        label.bot(btex A etex, A);
        label.rt(btex B etex, B);
    \end{Geometrie}
    \correction
    \begin{Geometrie}[CoinHD={(8u,5u)}]
        pair A,B,O,A',B',C,C';
        A=u*(1.5,3.5);
        B=u*(3,3.5);
        O=u*(4,2.5);
        A'=u*(6.5,1.5);
        B'=u*(5,1.5);
        C=u*(1.5,2);
        C'=u*(6.5,3);
        label.bot(btex $({\mathcal C})$ etex,C);
        label.top(btex $({\mathcal C}^{'})$ etex,C') withcolor \myMetapostGreen;
        path cc;
        cc=cercles(A,1.5u);
        trace cc;
        path bb;
        bb=cercles(A',1.5u);
        trace bb withcolor \myMetapostGreen;
        trace demidroite(A,O) withcolor \myMetapostGreen;
        trace demidroite(B,O) withcolor \myMetapostGreen;
        marque_s:=2;
        trace codesegments (A,O,O,A',2) withcolor \myMetapostGreen;
        trace codesegments (B,O,O,B',3) withcolor \myMetapostGreen;
        marque_p:="croix";
        pointe(O,A,B,A',B');
        label.top(btex O etex, O);
        label.bot(btex A etex, A);
        label.rt(btex B etex, B);
        label.llft(btex B' etex, B');
        label.bot(btex A' etex, A'); 
    \end{Geometrie}
\end{methode}

\begin{propriete}[\admise]
    Si deux cercles sont symétriques par rapport à un point alors ils ont le même rayon.
\end{propriete}