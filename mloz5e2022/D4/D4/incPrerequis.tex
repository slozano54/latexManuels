\vspace*{-13mm}
%pre-001
\begin{prerequis}[Connaisances \emoji{red-heart} et compétences \emoji{diamond-suit} du cycle 3]    
   \begin{itemize}        
       \item[\emoji{red-heart}] Vocabulaire associé à ces objets et à leurs propriétés : côté, sommet, angle, hauteur.
       \columnbreak
       \item[\emoji{diamond-suit}] Reconnaître, nommer, décrire des triangles, dont les triangles particuliers (triangle rectangle, triangle isocèle, triangle équilatéral).       
   \end{itemize}
\end{prerequis}
\begin{debat}[D'où vient le ratio ?]
    {\bf Ratio} vient de l’anglais {\bf ratio} que l’on traduit par proportion qui lui-même vient du latin {\bf ratio} qui signifie calcul ou compte. Ce vocabulaire est plutôt utilisé dans le monde anglo-saxon. \\
    On le retrouve pour la première fois dans {\it Les éléments}, d'{\it Euclide}, soit il y a environ 2\,300 ans !
    \begin{center}
        \begin{pspicture}(0,0)(7.5,4.7)
          {\psset{unit=0.35}
          \psgrid[subgriddiv=0,gridcolor=lightgray,gridlabels=0pt](0,0)(21,13)
           \psframe(0,0)(21,13)
           \psline(8,0)(8,13)
           \psline(0,8)(8,8)
           \psline(5,8)(5,13)
           \psline(5,10)(8,10)
           \psline(6,8)(6,10)
           \psline(5,9)(6,9)
           \psset{linecolor=B1,linewidth=0.2}
           \psarc(8,13){13}{-90}{0}
           \psarc(8,8){8}{180}{-90}
           \psarc(5,8){5}{90}{180}
           \psarc(5,10){3}{0}{90}
           \psarc(6,10){2}{-90}{0}
           \psarc(6,9){1}{90}{-90}}
       \end{pspicture}
    \end{center}
    \begin{cadre}[B2][J4]
       \begin{center}
          \hrefVideo{https://www.yout-ube.com/watch?v=vDZje8o_eD4}{\bf Point commun entre un ananas, des lapins et la tour de Pise ?} chaîne {\it Unisciel}.
       \end{center}
    \end{cadre}  
 \end{debat}