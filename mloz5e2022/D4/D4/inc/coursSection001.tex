\section{Définition du ratio}

\begin{definition}
   \begin{itemize}
      \item On dit que {\bf deux nombres} $a$ et $b$ sont, par exemple, dans le {\bf ratio} 3 : 4 si $\dfrac{a}{3} =\dfrac{b}{4}$. \\
         On parle de ratio \og trois pour quatre \fg. \\ [2mm]
         On peut le modéliser ainsi : \parbox{7cm}{\Ratio[FigureCours,Longueur=7cm,CouleurUn=DeepSkyBlue,CouleurDeux=IndianRed]{3,4}}
      \item On dit que {\bf trois nombres} $a$, $b$ et $c$ sont, par exemple, dans le {\bf ratio} 1 : 3 : 6 si $\dfrac{a}{1} =\dfrac{b}{3} = \dfrac{c}{6}$. \\
      On parle de ratio \og un pour trois pour six \fg. \\ [2mm]
      On peut le modéliser ainsi : \parbox{7cm}{\Ratio[FigureCours,Longueur=8cm,CouleurUn=DeepSkyBlue,CouleurDeux=IndianRed,CouleurTrois=MediumSeaGreen]{1,3,6}}
      \vspace{-4mm}
   \end{itemize}
\end{definition}

\begin{exemple*1}
   Jules et Lyna se partagent des cookies dans un ratio 3 : 4, cela veut dire que, à chaque fois que Jules a 3 cookies, Lyna en a 4 si bien que le nombre de cookies que possède Jules divisé par 3 est toujours égal au nombre de cookies que possède Lyna divisé par 4.
\end{exemple*1}

\begin{remarque}
   attention à ne pas confondre les notations 3 : 4 ; $3\div4$ et $\dfrac34$, la première désigne un ratio, la deuxième une division et la troisième une fraction. \\
   Dans l'exemple, Jules possède $\dfrac37$ des cookies et Lyna $\dfrac47$. \\ [1mm]
   Chacune de ces fractions permet de comparer une partie à la totalité, ce ne sont pas des ratios.
\end{remarque}
