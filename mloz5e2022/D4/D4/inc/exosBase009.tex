\begin{exercice*}
   Les triangles.
   \begin{enumerate}
      \item Quel est le ratio des trois angles d’un triangle équilatéral ?
      \item Quel est le ratio des trois angles d’un triangle rectangle isocèle ?
      \item Quelle est la nature d’un triangle dont les angles sont dans le ratio 1 : 2 : 3 ?
   \end{enumerate}
\end{exercice*}

\begin{corrige}
\ \\ [-5mm]
   \begin{enumerate}
      \item Un triangle équilatéral a trois angles de \ang{60}. Les trois angles sont donc dans un ratio de 60 : 60 : 60, ou encore {\color{red} 1 : 1 : 1}.
      \item Un triangle rectangle isocèle a trois angles de \ang{90}, \ang{45} et \ang{45}. Les trois angles sont donc dans un ratio de 90 : 45 : 45, ou encore {\color{red} 2 : 1 : 1}.
      \item Le ratio 1 : 2 : 3 pour les angles du triangle signifie que pour 1 part pour le premier angle, on a 2 parts pour le second et 3 parts pour le troisième pour un total de 6 parts, correspondant à \ang{180}. \\ [2mm]
            \quad \Ratio[Figure,Longueur=7cm,TexteTotal=\ang{180},CouleurUn=LightSkyBlue,CouleurDeux=IndianRed,CouleurTrois=Gold]{1,2,3} \\
         $180\div6 =30$ donc, la valeur d'une part est de \ang{30}. Le triangle a donc trois angles de \ang{30}, \ang{60} et \ang{90}.\\
         {\color{red} Le triangle est rectangle.} 
   \end{enumerate}
\end{corrige}

\vfill\hfill{\footnotesize D'après \href{https://ent2d.ac-bordeaux.fr/disciplines/mathematiques/les-ratios-au-college/}{\og Les ratios au collège \fg}, académie de Bordeaux et \href{https://eduscol.education.fr/document/13132/download}{\og La résolution de problèmes mathématiques au collège \fg}, MENJS, 2021.}