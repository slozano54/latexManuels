\begin{exercice*}
   Ratios et bonbons.
   \begin{enumerate}
      \item Un paquet de bonbons contient 13 bonbons à la fraise et 8 au citron. Dans quel ratio sont les bonbons à la fraise et les bonbons au citron ?
      \item Un paquet de bonbons contient 28 bonbons à la fraise, 18 au citron et 14 au cola. Dans quel ratio sont les bonbons à la fraise, les bonbons au citron et les bonbons au cola ?
   \end{enumerate}
\end{exercice*}
\begin{corrige}
   \ \\ [-5mm]
   \begin{enumerate}
      \item Les bonbons à la fraise et au citron sont dans le ratio {\color{red} 13 : 8}
      \item Les bonbons à la fraise, au citron et au cola sont dans le ratio {\color{red} 28 : 18 : 14} ou encore 14 : 9 : 7
   \end{enumerate}
\end{corrige}
