\section{Méthode de partage suivant un ratio}

\begin{methode}
Pour partager une quantité suivant un ratio :
   \begin{itemize}
      \item on calcule le nombre de parts égales à distribuer en additionnant les nombres du ratio ;
      \item on divise la quantité par le nombre de parts à distribuer ce qui nous donne la quantité par part ;
      \item on distribue les parts selon le ratio.
   \end{itemize}
   \exercice
   On partage 15 pièces d'or suivant le ratio 2 : 3 entre deux pirates, Sambra et Piébo. 
   
   \bigskip
   Quel est le nombre de pièces d'or de chacun ?
   \correction
   \ \\ [-10mm]
   \begin{itemize}
      \item Les 15 pièces d'or sont partagées en 5 parts égales (2 parts pour Sambra et 3 parts pour Piébo).
      \item $15\text{ pièces d'or}\div 5 =3\text{ pièces d'or}$ pour une part.
     \item Sambra a 2 parts, soit $2\times3\text{ pièces d'or} =6\text{ pièces d'or}$ et Piébo a 3 parts, soit $3\times3\text{ pièces d'or} =9\text{ pièces d'or}$. 
   \end{itemize}
   \vspace*{-7mm}
    \begin{center}
        \Ratio[Figure,TexteTotal=15 pièces,Longueur=10cm,CouleurUn=DeepSkyBlue,CouleurDeux=IndianRed]{2,3}
    \end{center}
\end{methode}