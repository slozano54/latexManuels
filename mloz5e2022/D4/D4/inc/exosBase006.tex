\begin{exercice*}
   Des ratios \og doubles \fg.
      \begin{enumerate}
         \item Quelle quantité d'huile et de vinaigre utilise-t-on dans une vinaigrette de \Capa[mL]{500} réalisée dans le ratio 3~:~1 ?
         \item Deux amis ont joué au loto et leur mise s'est faite selon le ratio 3 : 5. Ils gagnent \Prix{64}. Quelle est la somme d'argent qui revient à chacun d'eux ?
   \end{enumerate}
\end{exercice*}
\begin{corrige}
\ \\ [-5mm]
   \begin{enumerate}
      \item Le ratio 3 : 1 pour l'huile est le vinaigre signifie que pour 3 parts d'huile, on a 1 part de vinaigre pour un total de 4 parts de vinaigrette correspondant à une capacité de \Capa[mL]{500}. \\ [2mm]
            \qquad \Ratio[Figure,Longueur=6cm,TexteTotal=\scantokens{\Capa[mL]{500}},CouleurUn=LightSkyBlue,CouleurDeux=IndianRed]{3,1} \\
         $500\div4 =125$ donc, une part vaut \Capa[mL]{125}. \\
         Il faut {\color{red} \Capa[mL]{375} d'huile pour \Capa[mL]{125} de vinaigre.}        
      \item Le ratio 3 : 5 signifie que pour 3 parts pour le premier ami, le second gagne 5 parts pour un total de 8 parts, correspondant à une somme de \Prix{64}. \\ [2mm]
            \qquad \Ratio[Figure,Longueur=6cm,TexteTotal=\Prix{64},CouleurUn=LightSkyBlue,CouleurDeux=IndianRed]{3,5} \\
            $64\div8 =8$ donc, une part vaut \Prix{8}. \\
            {\color{red} Le premier ami gagnera \Prix{24} et l'autre \Prix{40}.}
   \end{enumerate}
\end{corrige}
