\begin{exercice*}
   Utiliser ce tableau des matchs perdus ou gagnés d'un collège pour répondre aux questions.
   \begin{center}
      \begin{CLtableau}{0.6\linewidth}{4}{c}
         \hline
         & matchs gagnés & matchs perdus \\
         \hline
         Rugby & \quad\, 9 & \quad\, 6 \\
         \hline
         Judo & \quad 12 & \quad\, 8 \\
         \hline
         Handball & \quad 10 & \quad\, 5 \\
         \hline
      \end{CLtableau}
   \end{center}
   \begin{enumerate}
      \item Quels sports ont un ratio équivalent gains-pertes ?
      \item Pour le handball :
      \begin{enumerate}
         \item Quel est le ratio gains-matchs joués ?
         \item Quelle est la fraction de matchs gagnés ?
         \item Quel est le pourcentage de matchs gagnés ?
      \end{enumerate}
   \end{enumerate}
\end{exercice*}
\begin{corrige}
  \ \\ [-5mm]
  \begin{enumerate}
      \item Rugby : ratio gains - pertes de 9 : 6 $=$ 3 : 2 \\
      Judo : ratio gains - pertes de 12 : 8 $=$ 3 : 2 \\
      Handball : ratio gains - pertes de 10 : 5 $=$ 2 : 1 \\
      {\color{red} Le rugby et le judo ont le même ratio gains - pertes.}
      \item 
      \begin{enumerate}
         \item Pour le handball, le ratio gains - matchs joués est de 10 : 15 équivalent à {\color{red} 2 : 3} \\
         \item La fraction de matchs gagnés est de $\dfrac{10}{15} = {\color{red} \dfrac23}$ \\ [1mm]
         \item Le pourcentage de matchs gagnés est de $\dfrac{10}{15}\times100 \approx {\color{red} 67\,\%}$
      \end{enumerate}
   \end{enumerate}
\end{corrige}
