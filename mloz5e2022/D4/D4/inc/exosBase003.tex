\begin{exercice*}
   Utiliser le dessin ci-dessous pour répondre aux questions.
   \begin{center}
   \psset{unit=0.7,subgriddiv=0,gridlabels=0,fillstyle=solid}
      \begin{pspicture}(0,0.5)(7,3)
         \def\noir{\psframe[fillcolor=black](0,0)(1,1)}
         \rput(0,0){\noir}
         \rput(0,2){\noir}
         \rput(3,0){\noir}
         \rput(3,2){\noir}
         \rput(6,0){\noir}
         \rput(6,2){\noir}
         \psframe[fillcolor=B2](1,1)(3,2)
         \psframe[fillcolor=B2](4,1)(6,2)
         \psgrid(0,0)(7,3)
      \end{pspicture}
   \end{center}
   \begin{enumerate}
      \item Quel est le ratio carrés rouges - total de carré ?
      \item Que peut représenter le ratio 4 : 6 ?
      \item Selon quel ratio sont représentés les carrés noirs et les carrés blancs ?
   \end{enumerate}
\end{exercice*}
\begin{corrige} 
   \ \\ [-5mm]
   \begin{enumerate}
      \item Le ratio carrés rouges - total de carré est {\color{red} 4 : 21}
      \item 4 : 6 peut représenter le {\color{red} ratio des carrés rouges et} {\color{red} des carrés noirs.}
      \item Les carrés noirs et les carrés blancs sont dans le ratio {\color{red} 6 : 11}
   \end{enumerate}
\end{corrige}
