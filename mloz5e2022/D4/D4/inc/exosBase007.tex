\begin{exercice*}
   Des ratios \og triples \fg.
      \begin{enumerate}
         \item Une recette de biscuits sablés commence par la fabrication d'un \og sable \fg{} réalisé avec de la farine, du beurre et du sucre dans le ratio 10 : 6 : 5. Une pâte homogène est ensuite fabriquée avec ce sable et un peu de lait. \\
         Quelles masses de farine, de beurre et de sucre doit-on prendre pour créer un \og sable \fg{} de \Masse[g]{630} ?
      \item Pour récompenser leurs enfants Clémentine, Myrtille et Prune, qui les ont beaucoup aidés, M. et Mme Potager leur donnent un peu d'argent. Ils leur distribuent \Prix{120} selon le ratio 3 : 4 : 5 parce qu'ils n'ont pas aidé autant les uns que les autres.
      Combien chacun va-t-il recevoir ?
   \end{enumerate}
\end{exercice*}
\begin{corrige}
\ \\ [-5mm]
   \begin{enumerate}
      \item Le ratio 10 : 6 : 5 pour la farine, le beurre et le sucre signifie que pour 10 parts de farine, on a 6 parts de beurre et 5 parts de sucre pour un total de 21 parts correspondant à une masse de \Masse[g]{630}. \\ [2mm]
           \quad  \Ratio[Figure,Longueur=7cm,TexteTotal=\scantokens{\Masse[g]{630}},CouleurUn=LightSkyBlue,CouleurDeux=IndianRed,CouleurTrois=Gold]{10,6,5} \\
         $630\div21 =30$ donc, une part vaut \Masse[g]{30}. \\
         Il faut {\color{red} \Masse[g]{300} de farine, \Masse[g]{180} de beurre et \Masse[g]{150} de sucre.}
      \item Le ratio 3 : 4 : 5 pour Clémentine, Myrtille et Prune signifie que pour 3 parts pour Clémentine, on a 4 parts pour Myrtille et 5 parts pour Prune pour un total de 12 parts correspondant à \Prix{120}. \\ [2mm]
           \quad  \Ratio[Figure,Longueur=7cm,TexteTotal=\Prix{120},CouleurUn=LightSkyBlue,CouleurDeux=IndianRed,CouleurTrois=Gold]{3,4,5} \\
         $120\div12 =10$ donc, une part vaut \Prix{10}. \\
         {\color{red} Clémentine recoit \Prix{30}, Myrtille \Prix{40} et Prune \Prix{50}.}
   \end{enumerate}
\end{corrige}
