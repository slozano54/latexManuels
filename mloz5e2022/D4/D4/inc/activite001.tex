\begin{changemargin}{-10mm}{-10mm}
    \vspace*{-20mm}
    \begin{activite}[Représenter des ratios]
        \vspace*{-7mm}
        {\bf Objectifs} :  calculer un ratio ; partager une quantité en deux ou trois parts selon un ratio donné.

        \begin{minipage}{0.4\linewidth}
            {\psset{unit=0.7}
            \begin{pspicture}(-0.5,-0.5)(8,3.7)
                \psline(0,3)(0,0)(8,0)(8,3)
                \pscircle(1,0.5){0.5}
                \pscircle[fillstyle=solid,fillcolor=lightgray](2,0.5){0.5}
                \pscircle[fillstyle=solid,fillcolor=lightgray](3,0.5){0.5}
                \pscircle(4.5,0.5){0.5}
                \pscircle(5.5,0.5){0.5}
                \pscircle[fillstyle=solid,fillcolor=black](7,0.5){0.5}
                \pscircle(0.5,1.4){0.5}
                \pscircle[fillstyle=solid,fillcolor=lightgray](1.5,1.4){0.5}
                \pscircle(2.5,1.4){0.5}
                \pscircle(3.75,1.2){0.5}
                \pscircle[fillstyle=solid,fillcolor=black](5,1.4){0.5}
                \pscircle[fillstyle=solid,fillcolor=lightgray](6.25,1.2){0.5}
                \pscircle(7.5,1.4){0.5}
                \pscircle[fillstyle=solid,fillcolor=black](2,2.3){0.5}
                \pscircle(3.25,2.1){0.5}
                \pscircle[fillstyle=solid,fillcolor=lightgray](4.25,2.1){0.5}
                \pscircle(5.75,2.1){0.5}
                \pscircle(6.75,2.1){0.5}
                \pscircle[fillstyle=solid,fillcolor=black](5,2.8){0.5}
                \pscircle[fillstyle=solid,fillcolor=lightgray](3.75,3){0.5}
            \end{pspicture}}
        \end{minipage}
        % \quad
        \begin{minipage}{0.6\linewidth}
            Dans cette boite, il y a 4 balles noires pour 6 balles grises. \\
            On dit que la quantité de balles noires et grises est dans le ratio de 4 : 6 (on lit \og 4 pour 6 \fg{}) ou encore 2 : 3 (\og 2 pour 3 \fg{}). \\
            Inversement, le ratio des balles grises et noires est de 6 : 4.
        \end{minipage}
        \begin{enumerate}
            \item 
                \begin{enumerate}
                \item Quel est le ratio des balles noires et blanches ? Simplifier éventuellement ce ratio. \par \medskip
                    \pointilles \medskip
                \item Quel est le ratio des balles grises et blanches ? Simplifier éventuellement ce ratio. \par \medskip
                    \pointilles \medskip
                \item Comment pourrait-on écrire le ratio de balles noires, grises et blanches ? \par \medskip
                    \pointilles \medskip
                \item Quelle fraction du total des balles représente les balles noires ? Les balles grises ? Les balles blanches ? \par \medskip
                    \pointilles \medskip
                \end{enumerate}
            \item Dans cette question, on garde les mêmes ratios que dans les questions précédentes.
                \begin{enumerate}
                \item Si le bac contenait 40 balles, combien aurait-on de balles noires ? Grises ? Blanches ? \par \medskip
                    \pointilles \par \medskip
                    Quelle fraction du total des balles représente chaque sorte de balles ? \par \medskip
                    \pointilles \medskip
                \item Si le bac contenait 10 balles, combien aurait-on de balles noires ? Grises ? Blanches ? \par \medskip
                    \pointilles \par \medskip
                    Quelle fraction du total des balles représente chaque sorte de balles ? \par \medskip
                    \pointilles \medskip
                \item Si le bac contenait 130 balles, combien aurait-on de balles noires ? Grises ? Blanches ? \par \medskip
                    \pointilles \par \medskip
                Quelle fraction du total des balles représente chaque sorte de balles ? \par \medskip
                    \pointilles \medskip
                \end{enumerate}
            \item On partage 21 balles roses et violettes dans le ratio 3 : 4. Combien aura-t-on de balles roses ? violettes ? \par \medskip
                \pointilles \medskip
            \item On partage 48 balles bleues, blanches et rouges dans le ratio 1 : 2 : 3. Combien a-t-on de balles de chaque couleur ? \par \medskip
            \pointilles \medskip
        \end{enumerate}
    \end{activite}
\end{changemargin}
