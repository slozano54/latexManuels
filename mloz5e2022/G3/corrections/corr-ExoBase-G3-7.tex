   \begin{enumerate}
      \item Comme les droites $(VB)$ et $(TI)$ sont parallèles, les angles correspondants $\widehat{BVT}$ et $\widehat{ITY}$ sont égaux, donc $\widehat{ITY}$ mesure $\ang{45}$.\\
      \item Les angles $\widehat{VTI}$ et $\widehat{ITY}$ sont adjacents supplémentaires, donc $\widehat{VTI}$ mesure $\ang{180}-\ang{45}={\color{black}\boldsymbol{\ang{135}}}$.\\
      \item Dans un triangle, la somme des angles vaut $\ang{180}$ donc $\widehat{TIY}=\ang{180}-\widehat{ITY}-\widehat{TYI}=\ang{180}-\ang{45}-\ang{69}=\ang{66}$.\\
      \item Les angles $\widehat{TIY}$ et $\widehat{TIB}$ sont adjacents supplémentaires, donc $\widehat{TIB}$ mesure $\ang{180}-\ang{66}={\color{black}\boldsymbol{\ang{114}}}$.\\
      \item Comme les droites $(VB)$ et $(TI)$ sont parallèles, les angles correspondants $\widehat{TIY}$ et $\widehat{IBV}$ sont égaux, donc $\widehat{IBV}$ mesure ${\color{black}\boldsymbol{\ang{66}}}$.\\
      \item La somme des angles du quadrilatère vaut donc : $\ang{45}+{\color{black}\boldsymbol{\ang{135}}}+{\color{black}\boldsymbol{\ang{114}}}+{\color{black}\boldsymbol{\ang{66}}}=\ang{180}+\ang{180}=\ang{360}$.\\
      La conjecture est finalement vraie.
   \end{enumerate}
