   \begin{enumerate}
   \item Les angles $\widehat{VIY}$ et $\widehat{YIT}$ sont adjacents supplémentaires, donc $\widehat{VIY}$ mesure $\ang{180}-\ang{92}=\ang{88}$.\\
   \item Dans un triangle, la somme des angles vaut $\ang{180}$ donc $\widehat{VYI}=180-\widehat{IVY}-\widehat{VIY}=\ang{180}-\ang{60}-\ang{88}=\ang{32}$.\\
   \item Pour les droites $(VY)$ et $(BT)$ coupées par la sécante $(YB)$ les angles $\widehat{VYI}$ et $\widehat{TBI}$ sont des angles alternes-internes.\\
   Or, si des angles alternes-internes sont égaux, alors cela signifie que les droites coupées par la sécante sont parallèles.\\
   Les droites $(VY)$ et $(BT)$ sont donc parallèles.\\
   \item Les droites $(VY)$ et $(BT)$ sont parallèles et les segments $[VY]$ et $[BT]$ sont de même longueur.\\
   Or, un quadrilatère qui possède des côtés opposés parallèles et de même longueur est un parallèlogramme.\\
   Donc $VYTB$ est un parallèlogramme et $I$ est son centre.
   \end{enumerate}
