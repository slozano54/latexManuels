\begin{activite}[Des angles mouvants]
    {\bf Objectif :} faire découvrir la propriété de la somme des angles d'un triangle.
    \partie[construction du triangle]
       \begin{enumerate}
          \item Sur une feuille, tracer un triangle ABC quelconque puis le découper.
          \item Colorier les trois angles de trois couleurs différentes des deux côtés du triangle (une couleur par angle). \\
          {\psset{unit=1}
          \begin{pspicture}(-2,0.2)(10,5.8)
             \rput(8.3,4.8){$\leftarrow$ colorier des deux côtés}
             \psline[linestyle=dashed](6,5)(6,1)(6.3,1)
             \psframe(6,1)(6.25,1.25)
             \pswedge[fillstyle=solid,fillcolor=B1,linecolor=B1](1,1){1}{0}{38.5}
             \pswedge[fillstyle=solid,fillcolor=A1,linecolor=A1](6,5){1}{218.8}{306.8}
             \pswedge[fillstyle=solid,fillcolor=J1,linecolor=J1](9,1){1}{127}{180}
             \pspolygon(1,1)(9,1)(6,5)
             \rput(0.7,1){B}
             \rput(9.3,1){C}
             \rput(6,5.3){A}
             \rput(6,0.7){H}     
          \end{pspicture}}
          \item Tracer la hauteur issue du sommet A et nommer le pied de cette hauteur H.
          \item Plier le triangle ABC de manière à placer le point A sur le point H.
          \item Plier le triangle ABC de manière à placer le point B sur le point H.
          \item Plier le triangle ABC de manière à placer le point C sur le point H.
       \end{enumerate}
    \partie[observations]
       \begin{enumerate}
          \item Que forment les trois angles obtenus en H ? \\ [5mm]
          \makebox[\linewidth]{\dotfill}
          \item Formuler cette observation en utilisant les angles $\widehat{\text A}, \widehat{\text B}$ et $\widehat{\text C}$. \\ [5mm]
          \makebox[\linewidth]{\dotfill}
          \item Formuler cette observation par une phrase simple et générale sans utiliser le nom des angles. \\ [5mm]
          \makebox[\linewidth]{\dotfill} \\ [5mm]
          \makebox[\linewidth]{\dotfill}
       \end{enumerate}
\end{activite}