\begin{exercice}   
   Sur la figure,  les droites $(VB)$ et $(TI)$ sont parallèles.
   On veut déterminer la mesure des angles du quadrilatère $VTIB$ (toutes les réponses doivent être justifiées).
   \begin{enumerate}
      \item Déterminer la mesure de l'angle $\widehat{ITY}$.
      \item En déduire la mesure de l'angle $\widehat{VTI}$.
      \item En utilisant la première question, déterminer la mesure de l'angle $\widehat{TIY}$.
      \item En déduire la mesure de l'angle $\widehat{TIB}$.
      \item En utilisant la troisième question, déterminer la mesure de l'angle $\widehat{IBV}$.
      \item Vérifier la conjecture suivante : « La somme des angles d'un quadrilatère vaut 360°.»   
      \begin{tikzpicture}[baseline,scale = 0.5]

         \tikzset{
            point/.style={
            thick,
            draw,
            cross out,
            inner sep=0pt,
            minimum width=5pt,
            minimum height=5pt,
            },
         }
         \clip (-8,-1) rectangle (9.91136137041995,10.732661536483967);


         \draw[color={black}] (-12.096094779983385,-48.51478631379982)--(14.273391840380395,57.247447850283784);
         \draw[color={black}] (-38.78030515405258,37.41148335403628)--(48.86896358486953,-23.961195335525673);
         \draw[color={rgb,255:red,241;mygreen,89;blue,41}] (-42.944392744456735,-25.60818485972507)--(50.85575411487669,30.325811341751717);
         \draw[color={rgb,255:red,241;mygreen,89;blue,41}] (-41.97670516205807,-21.727001954621088)--(51.823441697275356,34.2069942468557);
         \draw [color={black},fill opacity = 1] (0.9538911900960678,1.2845589116335256) node[anchor = center,scale=1] {45°};
         \draw  [color={black},line width = 2,preaction={fill,color = {black},opacity = 0.1}] (0.24195474403776523,0.9702875356499328) -- (0,0) -- (0.8588878548891347,0.5121636971945014) arc (30.81:76:1) ;
         \draw [color={black},fill opacity = 1] (2.7376288704121574,7.233986033686531) node[anchor = center,scale=1] {69°};
         \draw  [color={black},line width = 2,preaction={fill,color = {black},opacity = 0.1}] (2.9964491046860005,8.15908510013292) -- (2.1772970603970094,8.732661536483967) -- (1.9353751647973416,7.762365810207971) arc (-104:-35:0.9999999999999994) ;
         \draw  [color={blue},line width = 2,preaction={fill,color = {blue},opacity = 0.2}] (0.7257328383609055,2.9108953694540523) -- (0.9676875823986708,3.881182905103985) -- (1.8265754372878056,4.393346602298486) arc (30.81:-104:1) ;
         \draw  [color={rgb,255:red,241;mygreen,89;blue,41},line width = 2,preaction={fill,color = {rgb,255:red,241;mygreen,89;blue,41},opacity = 0.2}] (6.182020969596496,5.9284823384778695) -- (5.3628883437430845,6.5020865062298965) -- (4.50400048885395,5.989922809035395) arc (-149.19:-35:1.0000000000000002) ;
         \draw  [color={mygreen},line width = 2,preaction={fill,color = {mygreen},opacity = 0.2}] (7.052490854910573,4.205433708172251) -- (7.91136137041995,4.717626482026644) -- (7.092209326130958,5.291202918377689) arc (145:210.81:0.9999999999999996) ;
         \draw  [color={pink},line width = 2,preaction={fill,color = {pink},opacity = 0.2}] (1.2096423264364364,4.851470440753917) -- (0.9676875823986708,3.881182905103985) -- (1.8265754372878056,4.393346602298486) arc (30.81:76:1) ;
         \draw  [color={red},line width = 2,preaction={fill,color = {red},opacity = 0.2}] (4.504017828233707,5.9898937323755055) -- (5.3628883437430845,6.5020865062298965) -- (4.5437362994540935,7.075662942580943) arc (145:210.81:0.9999999999999999) ;
         \draw (0,0) node[above left] {$V$};
         \draw (0.9676875823986708,3.881182905103985) node[above left] {$T$};
         \draw (2.1772970603970094,8.732661536483967) node[left] {$Y$};
         \draw (5.3628883437430845,6.5020865062298965) node[above right] {$I$};
         \draw (7.91136137041995,4.717626482026644) node[above right] {$B$};
      \end{tikzpicture}
   \end{enumerate}

   \hrefMathalea{https://coopmaths.fr/mathalea.html?ex=5G30-1,s=1,n=1&v=l}
\end{exercice}

\begin{corrige}
   \begin{enumerate}
      \item Comme les droites $(VB)$ et $(TI)$ sont parallèles, les angles correspondants $\widehat{BVT}$ et $\widehat{ITY}$ sont égaux, donc $\widehat{ITY}$ mesure $\ang{45}$.\\
      \item Les angles $\widehat{VTI}$ et $\widehat{ITY}$ sont adjacents supplémentaires, donc $\widehat{VTI}$ mesure $\ang{180}-\ang{45}={\color{black}\boldsymbol{\ang{135}}}$.\\
      \item Dans un triangle, la somme des angles vaut $\ang{180}$ donc $\widehat{TIY}=\ang{180}-\widehat{ITY}-\widehat{TYI}=\ang{180}-\ang{45}-\ang{69}=\ang{66}$.\\
      \item Les angles $\widehat{TIY}$ et $\widehat{TIB}$ sont adjacents supplémentaires, donc $\widehat{TIB}$ mesure $\ang{180}-\ang{66}={\color{black}\boldsymbol{\ang{114}}}$.\\
      \item Comme les droites $(VB)$ et $(TI)$ sont parallèles, les angles correspondants $\widehat{TIY}$ et $\widehat{IBV}$ sont égaux, donc $\widehat{IBV}$ mesure ${\color{black}\boldsymbol{\ang{66}}}$.\\
      \item La somme des angles du quadrilatère vaut donc : $\ang{45}+{\color{black}\boldsymbol{\ang{135}}}+{\color{black}\boldsymbol{\ang{114}}}+{\color{black}\boldsymbol{\ang{66}}}=\ang{180}+\ang{180}=\ang{360}$.\\
      La conjecture est finalement vraie.
   \end{enumerate}
\end{corrige}