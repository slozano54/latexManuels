\begin{exercice}   
   Sur la figure,  les droites $(VB)$ et $(TI)$ sont parallèles.
   On veut déterminer la mesure des angles du quadrilatère $VTIB$ (toutes les réponses doivent être justifiées).
   \begin{enumerate}
      \item Déterminer la mesure de l'angle $\widehat{ITY}$.
      \item En déduire la mesure de l'angle $\widehat{VTI}$.
      \item En utilisant la première question, déterminer la mesure de l'angle $\widehat{TIY}$.
      \item En déduire la mesure de l'angle $\widehat{TIB}$.
      \item En utilisant la troisième question, déterminer la mesure de l'angle $\widehat{IBV}$.
      \item Vérifier la conjecture suivante : « La somme des angles d'un quadrilatère vaut 360°.»         
      \begin{center}
         \scalebox{0.8}{
            \begin{Geometrie}[CoinBG={(u,2u)}]
               trace feuillet;
               pair A,B,C,D,E;
               A=u*(4,9);
               B-A=u*(-1,-3);
               D-A=u*(-2,-6);
               C-A=u*(2,-2.1);
               E-A=u*(4,-4.2);
               trace droite(A,D);
               trace droite(A,E);
               trace droite(B,C) dashed dashpattern(on6bp off3bp on1.5bp off3bp);
               trace droite(D,E) dashed dashpattern(on6bp off3bp on1.5bp off3bp);
               labeloffset:=3*labeloffset;
               label.rt(btex $Y$ etex,A);
               labeloffset:=2*labeloffset/3;
               label.ulft(btex $T$ etex,B);
               label.ulft(btex $V$ etex,D);
               label.urt(btex $I$ etex,C);
               label.urt(btex $B$ etex,E);
               %marquage des angles
               trace Codeangle(B,A,C,0,btex \ang{69} etex) withpen pencircle scaled 2bp;
               marque_ang:=1.5*marque_ang;
               trace Codeangle(E,D,B,0,btex \ang{45} etex) withpen pencircle scaled 2bp;
               marque_ang:=marque_ang/1.5;            
               trace marqueangle(A,C,B,0) withcolor red withpen pencircle scaled 2bp;
               trace marqueangle(C,B,A,0) withcolor Purple withpen pencircle scaled 2bp;
               trace marqueangle(C,E,D,0) withcolor ForestGreen withpen pencircle scaled 2bp;
               marque_a:=marque_a/1.5;
               trace marqueangle(B,C,E,0) withcolor orange withpen pencircle scaled 2bp;            
               trace marqueangle(D,B,C,0) withcolor blue withpen pencircle scaled 2bp;            
               marque_a:=marque_a*1.5;
            \end{Geometrie}
         }
     \end{center}
   \end{enumerate}

   \hrefMathalea{https://coopmaths.fr/mathalea.html?ex=5G30-1,s=1,n=1&v=l}
\end{exercice}

\begin{corrige}
   \begin{enumerate}
      \item Comme les droites $(VB)$ et $(TI)$ sont parallèles, les angles correspondants $\widehat{BVT}$ et $\widehat{ITY}$ sont égaux, donc $\widehat{ITY}$ mesure $\ang{45}$.\\
      \item Les angles $\widehat{VTI}$ et $\widehat{ITY}$ sont adjacents supplémentaires, donc $\widehat{VTI}$ mesure $\ang{180}-\ang{45}={\color{black}\boldsymbol{\ang{135}}}$.\\
      \item Dans un triangle, la somme des angles vaut $\ang{180}$ donc $\widehat{TIY}=\ang{180}-\widehat{ITY}-\widehat{TYI}=\ang{180}-\ang{45}-\ang{69}=\ang{66}$.\\
      \item Les angles $\widehat{TIY}$ et $\widehat{TIB}$ sont adjacents supplémentaires, donc $\widehat{TIB}$ mesure $\ang{180}-\ang{66}={\color{black}\boldsymbol{\ang{114}}}$.\\
      \item Comme les droites $(VB)$ et $(TI)$ sont parallèles, les angles correspondants $\widehat{TIY}$ et $\widehat{IBV}$ sont égaux, donc $\widehat{IBV}$ mesure ${\color{black}\boldsymbol{\ang{66}}}$.\\
      \item La somme des angles du quadrilatère vaut donc : $\ang{45}+{\color{black}\boldsymbol{\ang{135}}}+{\color{black}\boldsymbol{\ang{114}}}+{\color{black}\boldsymbol{\ang{66}}}=\ang{180}+\ang{180}=\ang{360}$.\\
      La conjecture est finalement vraie.
   \end{enumerate}
\end{corrige}