\begin{exercice}
  Dans la figure ci-dessous, les points $Y, M, I$ et $A$ sont alignés. Des mesures d'angle sont indiquées. \\
  Vrai ou faux : le triangle $KYA$ est rectangle en $K$.
  \begin{center}
  {\psset{unit=0.7}
  \small
  \begin{pspicture}(-1.5,0)(9.4,3.5)
     \psline[linestyle=dashed](4,0)(2,3.46)(0,0)
     \psline[linestyle=dashed](9.4,0)(2,3.46)(6.1,0)
     \psline(10,0)(-1,0)
     \rput(0,-0.3){$Y$}
     \rput(4,-0.3){$M$}
     \rput(2,3.9){$K$}
     \rput(6.2,-0.3){$I$}
     \rput(9.5,-0.3){$A$}
     \psarc(0,0){0.5}{60}{180}
     \psarc(4,0){0.6}{120}{180}
     \psarc(6.1,0){0.7}{140}{180}
     \psarc(9.4,0){0.8}{155}{180}
     \rput(5.1,0.3){\udeg{40}}
     \rput(8,0.25){\udeg{25}}
     \rput(-0.6,0.8){\udeg{120}}
     \rput(3.1,0.4){\udeg{60}}
  \end{pspicture}}
  \end{center}
\end{exercice}

\begin{corrige}
  \begin{itemize}
     \item L'angle $\widehat{KYM}$ est supplémentaire à l'angle de mesure \udeg{120}, donc il mesure $\udeg{180}-\udeg{120} =\udeg{60}$.
     \item Dans le triangle $KYA$ : \\
        $\widehat{KYA}+\widehat{YAK} = \udeg{60}+\udeg{25} =\udeg{85}$ soit : \\
        $\widehat{YKA} =\udeg{180}-\udeg{85} =\udeg{95} \neq\udeg{90}$. \\
        Donc, {\red le triangle $KYA$ n'est pas rectangle en $K$}.
  \end{itemize}
  \vspace*{-5mm}
\end{corrige}
