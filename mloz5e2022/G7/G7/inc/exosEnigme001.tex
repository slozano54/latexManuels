% Les enigmes ne sont pas numérotées par défaut donc il faut ajouter manuellement la numérotation
% si on veut mettre plusieurs enigmes
% \refstepcounter{exercice}
% \numeroteEnigme
\begin{enigme}[Les puzzles de Sam Loyd]
    \begin{changemargin}{-10mm}{-10mm}
        Sam Loyd (1841-1911) était un mathématicien américain adepte de casse-têtes mathématiques et d'échec. Il est à l'origine de milliers de casse-têtes publiés notamment dans des journaux. On considère l'énigme \og The royal road to mathematics \fg{}, issu du {\it Sam Loyd's cyclopedia of 5000 puzzles tricks and conundrums with answers}, page 60.
        \begin{center}
           \includegraphics[scale=0.5]{\currentpath/images/Sam_loyd} \quad \includegraphics[scale=0.8]{\currentpath/images/Sam_loyd_texte}
        \end{center}
        \begin{minipage}{0.5\linewidth}            
           \partie[composition du trapezium]
           On considère le quadrilatère de Sam Loyd \\
           construit sur le quadrillage suivant : \\
              {\psset{unit=0.9}
              \begin{pspicture}(-1.5,-1.3)(6,7.3)
                 \psgrid[subgriddiv=0,gridlabels=0,gridcolor=gray](-1,-1)(6,7)
                 \psset{linewidth=0.5mm}
                 \pspolygon(0,0)(5,0)(2,6)(0,5)
                 \psline(0,0)(4,2)(0,2)
                 \psline(0,4)(2,4)(2,1)
                 \psline{|-|}(4,5)(4,6)
                 \rput(4.5,5.5){\small\Lg[cm]{1}}
              \end{pspicture}}
              \begin{enumerate}
                 \item L'auteur appelle cette figure \og trapezium \fg{} que \\
                 l'on peut traduire par trapèze. Qu'en pensez-vous ? \\
                 Dans le texte, comme définit-il cette figure ?
                 \item De quoi est composé ce trapezium ?      
              \end{enumerate}
        \end{minipage}
        \begin{minipage}{0.5\linewidth}            
           \partie[résolution des puzzles]
           Sam Loyd propose, à partir des pièces du trapézium, de reconstituer le carré, la croix grecque, le parallélogramme, le rectangle et le triangle rectangle. \\
           {\psset{unit=0.45}
              \begin{pspicture}(-2,-1)(13,16.5)
                 \psset{fillstyle=solid,fillcolor=yellow!30,linewidth=0.8mm}
                 \pspolygon(3,0)(13,0)(11,4)
                 \pspolygon(0,4)(0,6)(2,6)(2,8)(4,8)(4,6)(6,6)(6,4)(4,4)(4,2)(2,2)(2,4)
                 \psframe(8,6)(13,10)
                 \pspolygon(2,9)(0,13)(4,15)(6,11)
                 \pspolygon(6,15)(11,15)(13,11)(8,11)
                 \psgrid[subgriddiv=0,gridlabels=0,gridcolor=gray](-1,-1)(14,16)
                 \psline[linewidth=0.5mm]{|-|}(0,0)(0,1)
                 \rput(1,0.5){\small\Lg[cm]{1}}         
              \end{pspicture}}
              \begin{enumerate}
              \setcounter{enumi}{2}
                 \item Reproduire le trapezium puis le découper.
                 \item Avec les pièces du puzzle, reconstituer chacune des figures ci-dessus.
              \end{enumerate}
        \end{minipage}
    \end{changemargin}
\end{enigme}

% Pour le corrigé, il faut décrémenter le compteur, sinon il est incrémenté deux fois
% \addtocounter{exercice}{-1}

 \begin{corrige}
    \begin{enumerate}
       \item Un trapèze est un quadrilatère possédant deux côtés parallèles. Ici, ce n'est pas le cas. \\
          D'ailleurs, Sam Loyd le définit ainsi : \og forme géométrique à quatre côtés, dont aucun pris deux à deux sont parallèles.\fg
       \item Le trapézium est composé :
          \begin{itemize}
             \item d'un {\color{red} carré} ;
             \item d'un {\color{red} trapèze rectangle} ;
             \item de deux {\color{red} triangles rectangles} ;
             \item d'un {\color{red} hexagone} (non convexe).
          \end{itemize}
       \item \dots
       \item Le carré :
    \end{enumerate}
    
    \begin{pspicture}(-0.77,7.5)(7,16.5)
    \psset{fillstyle=solid,fillcolor=yellow!30,linewidth=0.7mm}
       \pspolygon(2,9)(0,13)(4,15)(6,11)
       \psset{fillstyle=none}
       \psgrid[subgriddiv=0,gridlabels=0,gridcolor=gray](-1,8)(7,16)   
       \psline(1,11)(6,11)
       \psline(2,11)(2,13)(5,13)
       \psline(4,11)(4,15)
    \end{pspicture}
          
    Le parallélogramme : \\
    \begin{pspicture}(5.22,10)(14,15.5)
    \psset{fillstyle=solid,fillcolor=yellow!30,linewidth=0.7mm}
       \pspolygon(6,15)(11,15)(13,11)(8,11)
       \psset{fillstyle=none}
       \psgrid[subgriddiv=0,gridlabels=0,gridcolor=gray](5,10)(14,15) 
       \psline(9,11)(9,13)(12,13)
       \psline(11,11)(11,15)
       \psline(7,13)(11,15)
    \end{pspicture}
 
 \Coupe
 
    Le triangle : \\
    \begin{pspicture}(5.17,-0.5)(13,5.5)
    \psset{fillstyle=solid,fillcolor=yellow!30,linewidth=0.7mm}
       \pspolygon(3,0)(13,0)(11,4)
       \psset{fillstyle=none}
       \psgrid[subgriddiv=0,gridlabels=0,gridcolor=gray](5,0)(13,5)  
       \psline(8,0)(7,2)
       \psline(9,0)(9,2)(12,2)
       \psline(11,0)(11,4)
    \end{pspicture}    
    
    La croix : \\
    \begin{pspicture}(-0.8,0.5)(7,9.4)
    \psset{fillstyle=solid,fillcolor=yellow!30,linewidth=0.7mm}
       \pspolygon(0,4)(0,6)(2,6)(2,8)(4,8)(4,6)(6,6)(6,4)(4,4)(4,2)(2,2)(2,4)
       \psset{fillstyle=none}
       \psgrid[subgriddiv=0,gridlabels=0,gridcolor=gray](-1,1)(7,9)   
       \psline(3,2)(1,6)
       \psline(2,4)(6,6)
       \psline(2,6)(4,6)
    \end{pspicture}
          
    Le rectangle : \\
    \begin{pspicture}(6.2,7)(14,11.5)
    \psset{fillstyle=solid,fillcolor=yellow!30,linewidth=0.7mm}
      \psframe(8,6)(13,10)
       \psset{fillstyle=none}
       \psgrid[subgriddiv=0,gridlabels=0,gridcolor=gray](7,5)(14,11)   
       \psline(10,6)(8,10)
       \psline(8,8)(9,8)(13,10)
       \psline(11,6)(11,8)(13,8)
    \end{pspicture}
 \end{corrige}