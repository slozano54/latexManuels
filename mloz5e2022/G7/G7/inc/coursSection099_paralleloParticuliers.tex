\begin{changemargin}{0mm}{-15mm}
    % \subsection{Le losange}
    % \definNum{
    % \begin{minipage}{10cm}
    % Le \textbf{losange} est un \textcolor{red}{quadri}\textcolor{mygreen}{latère} (\textcolor{red}{quatre}\textcolor{mygreen}{côtés}) dont tous les côtés sont égaux.
    % \end{minipage}
    % \begin{minipage}{8cm}
    % \begin{center}
    % \includegraphics[scale=0.8]{coursparallelogrammesparticuliers.1} 
    % \end{center}
    % \end{minipage}
    % }
    
    % \proprNum{(admise)}{un quadrilatère est un losange}{c'est un parallélogramme.}
    % \Preuve[Démonstration]{
    % Soit $ABCD$ un losange, alors le triangle $ABC$ est isocèle en $B$ donc il a un axe de symétrie issue de son sommet principal, de même le triangle $BCD$ est isocèle en $C$ donc il a un axe de symétrie. Ces deux axes se coupent en un point qui constitue un centre de symétrie, donc les cotés opposés de $ABCD$ sont parallèles.$\square$
    % }
    
    % \proprNum{(admise)}{un parallélogramme a deux côtés consécutifs égaux}{c'est un losange.}
    % \Preuve[Démonstration]{
    % Soit $ABCD$ un parallélogramme tel que $AB=BC$, or un parallélogramme a ses côtés opposés de même longueur, donc $AB=CD$ et $BC=DA$ d'où $AB=BC=CD=DA$ $\square$
    % }
    
    % \proprNum{caractéristique du losange(admise)}{les diagonales d'un parallélogramme sont perpendiculaires}{c'est un losange.}
    
    % \Exemples[Exemple]{}{
    % \begin{minipage}{6.5cm}
    % \begin{center}
    % \includegraphics[scale=1]{coursparallelogrammesparticuliers.2}
    % \end{center}
    % \end{minipage}
    % \begin{minipage}{10cm}
    % \begin{enumerate}
    % \item GSXR a ses diagonales qui se coupent en leur milieu donc c'est un parallélogramme.
    % \item GSXR a ses diagonales perpendiculaires donc c'est un losange.
    % \end{enumerate}
    % \end{minipage}
    % }
    % \proprNum{Réciproque de la propriété précédente (admise)}{un quadrilatère est un losange}{c'est un parallélogramme dont les diagonales sont perpendiculaires.}
    % %\newpage
    
    % \subsection{Le rectangle}
    % \definNum{
    % \begin{minipage}{10cm}
    % Le \textbf{rectangle} est un quadrilatère ayant au moins trois angles droits.
    % \end{minipage}
    % \begin{minipage}{8cm}
    % \begin{center}
    % \includegraphics[scale=0.7]{coursparallelogrammesparticuliers.3} 
    % \end{center}
    % \end{minipage}
    % }
    
    % \proprNum{(admise)}{un quadrilatère est un rectangle}{c'est un parallélogramme.}
    % \Preuve[Démonstration]{
    % Soit $ABCD$ un rectangle, par exemple $\widehat{DAB}=\widehat{ABC}=\widehat{BCD}=90$\degre.
    % \par
    % $(AB)\perp (BC)$ et $(CD)\perp (BC)$ donc $(AB) \parallel (CD)$
    % \par
    % de même $(BC)\perp (AB)$ et $(AD)\perp (AB)$ donc $(BC) \parallel (AD)$
    % \par
    % donc $ABCD$ est un parallélogramme. $\square$
    % }
    
    % \proprNum{(admise)}{un parallélogramme a un angle droit}{c'est un rectangle.}
    % \Preuve[Démonstration]{
    % Soit $ABCD$ un parallélogramme tel que $\widehat{ABC}=90$\degre, or un parallélogramme a ses angles opposés de même mesure, donc $\widehat{ABC}=\widehat{CDA}=90$\degre
    % \par
    % d'autre part, $(AD) \parallel (BC)$ et $(AB) \perp (BC)$ donc $(AD)\perp (AB)$
    % \par
    % de même, $(BC)\perp (CD)$ donc les quatre angles sont droits. $\square$
    % }
    
    % \proprNum{caractéristique du rectangle (admise)}{les diagonales d'un parallélogramme sont de même longueur}{c'est un rectangle.}
    
    % \Exemples[Exemple]{}{
    % \begin{minipage}{6.5cm}
    % \begin{center}
    % \includegraphics[scale=1]{coursparallelogrammesparticuliers.4}
    % \end{center}
    % \end{minipage}
    % \begin{minipage}{10cm}
    % \begin{enumerate}
    % \item GSXR a ses diagonales qui se coupent en leur milieu donc c'est un parallélogramme.
    % \item GSXR a ses diagonales de même longueur donc c'est un rectangle.
    % \end{enumerate}
    % \end{minipage}
    % }
    % \proprNum{Réciproque de la propriété précédente (admise)}{un quadrilatère est un rectangle}{c'est un parallélogramme dont les diagonales sont de même longueur.}
    
    % \subsection{Le carré}
    % \definNum{
    % \begin{minipage}{10cm}
    % Le \textbf{carré} est un quadrilatère étant à la fois un rectangle et un losange.
    % \end{minipage}
    % \begin{minipage}{8cm}
    % \includegraphics[scale=0.8]{coursparallelogrammesparticuliers.5} 
    % \end{minipage}
    % }
    
    % \proprNum{(admise)}{un quadrilatère est un carré}{c'est un parallélogramme.}
    
    % \proprNum{caractéristique du carré (admise)}{les diagonales d'un parallélogramme sont de même longueur (rectangle) et forment un angle droit (losange)}{c'est un carré.}
    
    % \Exemples[Exemple]{}{
    % \begin{minipage}{6.5cm}
    % \begin{center}
    % \includegraphics[scale=0.7]{coursparallelogrammesparticuliers.6}
    % \end{center}
    % \end{minipage}
    % \begin{minipage}{10cm}
    % \begin{enumerate}
    % \item GSXR a ses diagonales qui se coupent en leur milieu donc c'est un parallélogramme.
    % \item GSXR a ses diagonales de même longueur donc c'est un rectangle.
    % \item GSXR a ses diagonales perpendiculaires donc c'est un losange.
    % \end{enumerate}
    % \end{minipage}
    % }
    
    % \proprNum{Réciproque de la propriété précédente (admise)}{un quadrilatère est un carré}{c'est un parallélogramme dont les diagonales sont de même longueur et forment un angle droit.}
    
    % \subsection{Axes et centres de symétrie}
    
    % \proprNum{(admise)}{un quadrilatère est un losange}{
    % \begin{enumerate}
    % \item son centre est un centre de symétrie ($\bigcap$ des diagonales).
    % \item ses diagonales sont des axes de symétrie.
    % \end{enumerate}
    % }
    % $$\includegraphics[scale=0.6]{coursparallelogrammesparticuliers.7}$$
    
    
    % %\begin{minipage}{6cm}
    % \proprNum{(admise)}{un quadrilatère est un rectangle}{
    % \begin{enumerate}
    % \item son centre est un centre de symétrie ($\bigcap$ des diagonales).
    % \item les médiatrices de ses côtés sont des axes de symétrie.
    % \end{enumerate}
    % }
    % %\end{minipage}
    % %\begin{minipage}{12cm}
    % $$\includegraphics[scale=0.6]{coursparallelogrammesparticuliers.8}$$
    % %\end{minipage}
    
    % %\begin{minipage}{6cm}
    % \proprNum{(admise)}{un quadrilatère est un carré}{
    % \begin{enumerate}
    % \item son centre est un centre de symétrie ($\bigcap$ des diagonales).
    % \item ses diagonales sont des axes de symétrie.
    % \item les médiatrices de ses côtés sont des axes de symétrie.
    % \end{enumerate}
    % }
    % %\end{minipage}
    % %\begin{minipage}{12cm}
    % $$\includegraphics[scale=0.6]{coursparallelogrammesparticuliers.9}$$
    % %\end{minipage}
    
    % \subsection{Compléments numériques - Constructions}
    % \Animations{
    % \lienCadre{http://lozano.maths.free.fr/iep_local/figures_html/scr_iep_089.html}{Parallélogrammes au compas}
    % \lienCadre{http://lozano.maths.free.fr/iep_local/figures_html/scr_iep_088.html}{Parallélogramme à partir de 3 points au compas}
    % \lienCadre{http://lozano.maths.free.fr/iep_local/figures_html/scr_iep_090.html}{Parallélogrammes à partir de 3 points avec les diagonales au compas}
    % \lienCadre{http://lozano.maths.free.fr/iep_local/figures_html/scr_iep_091.html}{Parallélogramme à partir de 3 points à la réquerre}
    % \creditInstrumentPoche
    % }
    
    
    
\end{changemargin}
 
