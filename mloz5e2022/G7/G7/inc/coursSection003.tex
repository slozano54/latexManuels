\begin{changemargin}{0mm}{-15mm}
    \section{Propriétés caractéristiques du parallélogramme}
    \begin{propriete}[caractérisation par les diagonales]
        Si les diagonales d'un quadrilatère ont le même milieu alors ce quadrilatère est un parallélogramme.
    \end{propriete}
    \begin{preuve}
        \phantom{rrr}

        \begin{minipage}{0.65\linewidth}
            On suppose que $[IK]$ et $[LJ]$ ont le même milieu $M$.
            \par \textbf{d'où} $I$ et $K$ sont symétriques par rapport à $M$, $L$ ,et $J$ aussi. 
            \par \textbf{donc} $(IL)$ et $(KJ)$ sont symétriques par rapport à $M$
            \par \textbf{or} la symétrie centrale transforme une droite en une droite parallèle
            \par \textbf{donc} $(IL)$ et $(KJ)$ sont parallèles, on montrerait de même que $(IJ)$ et $(KL)$ le sont.
            \par On peut donc conclure que le quadrilatère $IJKL$ est un parallélogramme. $\square$
        \end{minipage}
        \begin{minipage}{7cm}
            \begin{center}
                % \includegraphics[scale=1]{coursparallelogramme.5}
                \begin{Geometrie}[CoinHD={(7u,6u)}]
                    pair I,J,K,L,M;
                    I=u*(1,5);
                    J=u*(5,4);
                    K=u*(5,1);
                    L=u*(1,2);
                    M=u*(3,3);
                    marque_s:=0.2*marque_s;
                    trace segment(I,K);
                    trace marquesegment(I,K);
                    trace segment(L,J);                    
                    trace marquesegment(L,J);
                    trace codesegments(I,M,M,K,5);
                    trace codesegments(L,M,M,J,3);
                    label.ulft(btex $I$ etex,I);
                    label.urt(btex $J$ etex,J);
                    label.lrt(btex $K$ etex,K);
                    label.llft(btex $L$ etex,L);
                    label.top(btex $M$ etex,M);
                \end{Geometrie}  
            \end{center}
        \end{minipage}
    \end{preuve}
    \begin{propriete}[caractérisation par les côtés opposés \admise]
        \begin{minipage}{0.65\linewidth}    
            Si un quadrilatère, non croisé, a ses côtés opposés deux à deux égaux alors ce quadrilatère est un parallélogramme.
        \end{minipage}
        \begin{minipage}{0.35\linewidth}    
            % \includegraphics[scale=0.8]{coursparallelogramme.11}
            \begin{Geometrie}[CoinHD={(7u,6u)}]
                pair I,J,K,L,M;
                I=u*(1,5);
                J=u*(5,4);
                K=u*(5,1);
                L=u*(1,2);
                trace segment(I,J);
                trace segment(I,L);
                trace segment(L,K);
                trace segment(J,K);
                marque_s:=0.2*marque_s;
                trace codesegments(I,J,L,K,4);
                trace codesegments(I,L,J,K,1);
                label.ulft(btex $I$ etex,I);
                label.urt(btex $J$ etex,J);
                label.lrt(btex $K$ etex,K);
                label.llft(btex $L$ etex,L);
            \end{Geometrie} 
        \end{minipage}
    \end{propriete}
    \begin{propriete}[caractérisation par les côtés opposés bis \admise]
        \begin{minipage}{0.65\linewidth}    
            Si un quadrilatère, non croisé, a deux côtés opposés égaux ET parallèles alors ce quadrilatère est un parallélogramme.
        \end{minipage}
        \begin{minipage}{0.35\linewidth}    
            % \includegraphics[scale=0.8]{coursparallelogramme.6} 
            \begin{Geometrie}[CoinHD={(7u,6u)}]
                pair I,J,K,L,M;
                I=u*(1,5);
                J=u*(5,4);
                K=u*(5,1);
                L=u*(1,2);
                trace segment(I,J);
                trace segment(I,L) dashed evenly;
                trace segment(L,K);
                trace segment(J,K) dashed evenly;
                marque_s:=0.2*marque_s;
                trace codesegments(I,J,L,K,4);
                label.ulft(btex $I$ etex,I);
                label.urt(btex $J$ etex,J);
                label.lrt(btex $K$ etex,K);
                label.llft(btex $L$ etex,L);
            \end{Geometrie}
        \end{minipage}
    \end{propriete}
    \begin{propriete}[caractérisation par les angles opposés \admise]
        \begin{minipage}{0.65\linewidth}    
            Si un quadrilatère, non croisé, a ses angles opposés deux à deux égaux alors ce quadrilatère est un parallélogramme.
        \end{minipage}
        \begin{minipage}{0.35\linewidth}    
            % \includegraphics[scale=0.8]{coursparallelogramme.4} 
            \begin{Geometrie}[CoinHD={(7u,6u)}]
                pair A,B,C,D;
                A=u*(3,4);
                B=u*(6,5);
                C=u*(4,2);
                D=u*(1,1);
                trace polygone(A,B,C,D);
                marque_a:=0.5*marque_a;
                trace marqueangle(D,A,B,0) withcolor red;
                trace marqueangle(B,C,D,0) withcolor red;
                trace marqueangle(A,B,C,0) withcolor DarkGreen;
                trace marqueangle(C,D,A,0) withcolor DarkGreen;
                marque_a:=1.2*marque_a;
                trace marqueangle(A,B,C,0) withcolor DarkGreen;
                trace marqueangle(C,D,A,0) withcolor DarkGreen;
                label.ulft(btex $A$ etex,A);
                label.urt(btex $B$ etex,B);
                label.lrt(btex $C$ etex,C);
                label.llft(btex $D$ etex,D);
            \end{Geometrie}
        \end{minipage}
    \end{propriete}
\end{changemargin}
 
