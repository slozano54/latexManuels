\begin{changemargin}{0mm}{-15mm}
    \section{Centre de symétrie d'un parallélogramme}
    \begin{propriete}[\admise]
        Si on construit le point d'intersection des diagonales d'un parallélogramme alors c'est son centre de symétrie.
    \end{propriete}
    \begin{exemple*1}
        \phantom{rrr}

        \begin{minipage}{0.65\linewidth}
            \begin{itemize}
                \item $O$ est centre de symétrie du parallèlogramme $ABCD$.
                \item On dit parfois que $ABCD$ est un parallélogramme de centre $O$.
            \end{itemize}
        \end{minipage}
        \begin{minipage}{0.35\linewidth}
            % \includegraphics[scale=0.8]{coursparallelogramme.2}
            \scalebox{0.9}{
            \begin{Geometrie}[CoinBG={(0.5u,0.5u)},CoinHD={(5.5u,4u)}]
                pair A,B,C,D,O;
                path sac, sbd;
                D=u*(1,1);
                A-D=u*(1,2);
                C-D=u*(3,0.5);
                B-C=u*(1,2);  
                trace droite(A,B) withcolor red;
                trace droite(C,D) withcolor red;
                trace droite(A,D) withcolor DarkGreen;
                trace droite(C,B) withcolor DarkGreen;
                trace segment(A,C);
                trace segment(B,D);
                sac=segment(A,C);
                sbd=segment(B,D);
                O=sac intersectionpoint sbd;
                label.ulft(btex $A$ etex,A);
                label.urt(btex $B$ etex,B);
                label.lrt(btex $C$ etex,C);
                label.llft(btex $D$ etex,D);
                label.top(btex $O$ etex,O);
            \end{Geometrie}
            } 
        \end{minipage}
        \vspace*{-10mm}
    \end{exemple*1}
    \begin{propriete}[des diagonales]    
        Si un quadrilatère est un parallélogramme alors ses diagonales se coupent en leur milieu.
    \end{propriete}
    \begin{preuve}
        \phantom{rrr}

        \begin{minipage}{0.65\linewidth}
            Soit $ABCD$ un parallélogramme de centre $O$, $O$ est un centre de symétrie donc :
            \begin{itemize}
                \item $A$ et $C$ sont symétriques par rapport à $O$.
                \item $B$ et $D$ sont symétriques par rapport à $O$.
            \end{itemize}
            On peut donc conclure que $O$ est les milieu de $[AC]$ et de $[BD]$ $\square$
        \end{minipage}
        \begin{minipage}{0.35\linewidth}
            % \includegraphics[scale=0.8]{coursparallelogramme.10}
            \begin{Geometrie}[CoinBG={(0.5u,0.5u)},CoinHD={(5.5u,4u)}]
                pair A,B,C,D,O;
                path sac, sbd;
                D=u*(1,1);
                A-D=u*(1,2);
                C-D=u*(3,0.5);
                B-C=u*(1,2); 
                trace droite(A,B) withcolor red;
                trace droite(C,D) withcolor red;
                trace droite(A,D) withcolor DarkGreen;
                trace droite(C,B) withcolor DarkGreen;
                trace segment(A,C);
                trace segment(B,D);
                sac=segment(A,C);
                sbd=segment(B,D);
                O=sac intersectionpoint sbd;
                marque_s:=0.2*marque_s;
                trace codesegments(A,O,O,C,2);
                trace codesegments(B,O,O,D,4);
                label.ulft(btex $A$ etex,A);
                label.urt(btex $B$ etex,B);
                label.lrt(btex $C$ etex,C);
                label.llft(btex $D$ etex,D);
                label.top(btex $O$ etex,O);
            \end{Geometrie}
        \end{minipage}
    \end{preuve}
    \begin{propriete}[des côtés opposés]
        Si un quadrilatère est un parallélogramme alors ses côtés opposés sont de même longueur.
    \end{propriete}
    \begin{preuve}
        \phantom{rrr}

        \begin{minipage}{0.65\linewidth}
            Soit $ABCD$ un parallélogramme de centre $O$.
            \par
            $A$ et $C$ sont symétriques par rapport à $O$, $B$ et $D$ aussi
            \par \textbf{donc} $[AB]$ et $[CD]$ sont symétriques par rapport à $O$, $[AD]$ et $[BC]$ aussi
            \par \textbf{or} la symétrie centrale conserve les longueurs
            \par On peut donc conclure que, $[AB]$ et $[CD]$ ont la même longueur, $[AD]$ et $[BC]$ aussi. $\square$
        \end{minipage}
        \begin{minipage}{0.35\linewidth}
            \begin{center}
                \underline{Illustration} :\par 
                % \includegraphics[scale=0.9]{coursparallelogramme.3}
                \begin{Geometrie}[CoinHD={(7u,6u)}]
                    pair A,B,C,D;
                    D=u*(1,1);
                    A-D=u*(1,2);
                    C-D=u*(3,0.5);
                    B-C=u*(1,2);  
                    trace polygone(A,B,C,D);
                    marque_s:=0.2*marque_s;
                    trace codesegments(A,D,B,C,2);
                    trace codesegments(A,B,D,C,4);
                    label.ulft(btex $A$ etex,A);
                    label.urt(btex $B$ etex,B);
                    label.lrt(btex $C$ etex,C);
                    label.llft(btex $D$ etex,D);
                \end{Geometrie} 
            \end{center}
        \end{minipage}
    \end{preuve}
    \begin{propriete}[des angles opposés]
        Si un quadrilatère est un parallélogramme alors ses angles opposés sont de même mesure.
    \end{propriete}
    \begin{preuve}
        \phantom{rrr}

        \begin{minipage}{0.65\linewidth}
            Soit $ABCD$ un parallélogramme de centre $O$.
            \par
            $A$ et $C$ sont symétriques par rapport à $O$, $B$ et $D$ aussi
            \par \textbf{donc} $\widehat{ABC}$ et $\widehat{CDA}$ sont symétriques par rapport à $O$, $\widehat{BCD}$ et $\widehat{DAB}$ aussi
            \par \textbf{or} la symétrie centrale conserve les mesures d'angles
            \par On peut donc conclure que, $\widehat{ABC}$ et $\widehat{CDA}$ d'une part puis $\widehat{BCD}$ et $\widehat{DAB}$ d'autre part, ont la même mesure. $\square$
        \end{minipage}
        \begin{minipage}{0.35\linewidth}
            \begin{center}
                \underline{Illustration} :\par
                % \includegraphics[scale=1]{coursparallelogramme.4} 
                \begin{Geometrie}[CoinHD={(7u,6u)}]
                    pair A,B,C,D;
                    A=u*(3,4);
                    B=u*(6,5);
                    C=u*(4,2);
                    D=u*(1,1);
                    trace polygone(A,B,C,D);
                    marque_a:=0.5*marque_a;
                    trace marqueangle(D,A,B,0) withcolor red;
                    trace marqueangle(B,C,D,0) withcolor red;
                    trace marqueangle(A,B,C,0) withcolor DarkGreen;
                    trace marqueangle(C,D,A,0) withcolor DarkGreen;
                    marque_a:=1.2*marque_a;
                    trace marqueangle(A,B,C,0) withcolor DarkGreen;
                    trace marqueangle(C,D,A,0) withcolor DarkGreen;
                    label.ulft(btex $A$ etex,A);
                    label.urt(btex $B$ etex,B);
                    label.lrt(btex $C$ etex,C);
                    label.llft(btex $D$ etex,D);
                \end{Geometrie} 
            \end{center}
        \end{minipage}
    \end{preuve}
    \begin{propriete}[\admise]
        Dans un parallélogramme, deux angles consécutifs sont supplémentaires. \\
        Cela signifie que leur somme vaut \ang{180}.
    \end{propriete}
\end{changemargin}
 
