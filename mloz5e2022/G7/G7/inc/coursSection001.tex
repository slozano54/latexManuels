\begin{changemargin}{0mm}{-15mm}
    \section{Définition et vocabulaire}
    \begin{definition}
        Un \textbf{parallélogramme} est un quadrilatère dont les côtés opposés sont parallèles deux à deux.
    \end{definition}
    \begin{exemple*1}
        \phantom{rrr}    

        \begin{minipage}{0.6\linewidth}
            \begin{itemize}
                \item $(GS)$ est parallèle à $(RX)$
                \item $(SX)$ est parallèle à $(GR)$
            \end{itemize}
            Cette figure représente un parallélogramme nommé $GSXR$ ou $SXRG$ ou ... mais pas $GSRX$ ou $SRXG$, ...
        \end{minipage}
        \begin{minipage}{0.4\linewidth}
            % \includegraphics[scale=0.8]{coursparallelogramme.1} 
            \begin{Geometrie}[CoinBG={(0.5u,0.5u)},CoinHD={(5.5u,4u)}]
                pair G,S,X,R;
                R=u*(1,1);
                G-R=u*(1,2);
                X-R=u*(3,0.5);
                S-X=u*(1,2);                
                trace droite(G,S) withcolor red;
                trace droite(X,R) withcolor red;
                trace droite(G,R) withcolor DarkGreen;
                trace droite(X,S) withcolor DarkGreen;
                label.ulft(btex $G$ etex,G);
                label.ulft(btex $S$ etex,S);
                label.lrt(btex $X$ etex,X);
                label.ulft(btex $R$ etex,R);
            \end{Geometrie}
        \end{minipage}
        \vspace*{-10mm}
    \end{exemple*1}
    \begin{definition}[Vocabulaire]
        \begin{itemize}
            \item $[GS]$ et $[SX]$ sont des \colorbox{red!30}{\textbf{côtés consécutifs}} ( qui se suivent ).
            \item $[GS]$ et $[RX]$ sont des \colorbox{red!30}{\textbf{côtés opposés}} ( l'un en face de l'autre ).
            \item $G$ et $S$ sont des \colorbox{red!30}{\textbf{sommets consécutifs}}.
            \item $G$ et $X$ sont des \colorbox{red!30}{\textbf{sommets opposés}}.
            \item $\widehat{GSX}$ et $\widehat{SXR}$ sont des \colorbox{red!30}{\textbf{angles consécutifs}}.
            \item $\widehat{GSX}$ et $\widehat{XRG}$ sont des \colorbox{red!30}{\textbf{angles opposés}}.
            \item $[GX]$ et $[SR]$ sont les \colorbox{red!30}{\textbf{diagonales}}.
        \end{itemize}
    \end{definition}
\end{changemargin}
 
