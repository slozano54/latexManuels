\begin{activite}[Les polydrons]
    \begin{changemargin}{-10mm}{-15mm}
        {\bf Objectifs :} construire des solides fermés ; trier des solides selon leur forme.
           Les Polydrons sont des polygones en plastique dur qui peuvent se fixer entre eux à l'aide de charnières.
           
           Ce matériel permet de construire facilement des polyèdres et des patrons.
           \partie[construction de solides]
              \begin{enumerate}
                 \item Citer les différentes formes de Polydrons en précisant leur nature exacte.
                    
                 \vspace*{10mm}
                 \item Construire un premier solide, donner son nom si possible et le dessiner.
                 
                 \vspace*{35mm}
                 \item Construire d'autres solides en essayant de varier les formes.
              \end{enumerate}       
         \partie[classement des solides]
            \begin{enumerate}
               \item En regroupant tous les solides de la classe, déterminer un classement commun, discuter des choix.
               \item Citer les classes choisies en expliquant leurs caractéristiques.
               
                \vspace*{25mm}
            \end{enumerate}
         \begin{center}
              \includegraphics[width=15cm]{\currentpath/images/polydrons} \\
           \end{center}
    \end{changemargin}
\end{activite}