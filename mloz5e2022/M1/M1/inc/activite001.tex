\begin{activite}[De la seconde au siècle]
    \vspace*{-5mm}
    {\bf Objectif :} donner un ordre de grandeur dans le domaine des durées (plusieurs flèches peuvent arriver au même ordre de grandeur). \\
    Relier les durées suivantes à son ordre de grandeur.
    \begin{center}
        \begin{pspicture}(-8,-9)(8,9)
        {\psset{yunit=0.85}
            \rput(0,9){\cursive\large Le siècle}
            \rput(0,6){\cursive\large L'année}
            \rput(0,3){\cursive\large Le mois}
            \rput(0,0){\cursive\large Le jour}
            \rput(0,-3){\cursive\large L'heure}
            \rput(0,-6){\cursive\large La minute}
            \rput(0,-9){\cursive\large La seconde}
            \psdots(-1.5,-9)(-1.5,-6)(-1.5,-3)(-1.5,0)(-1.5,3)(-1.5,6)(-1.5,9)(1.5,-9)(1.5,-6)(1.5,-3)(1.5,0)(1.5,3)(1.5,6)(1.5,9)(-4,-10)(-4,-6)(-4,-2)(4,2)(4,6)(4,10)(4,-10)(4,-6)(4,-2)(-4,2)(-4,6)(-4,10)
            \rput(-6.25,-6){\parbox{4cm}{Temps mis par la lumière pour parcourir une distance équivalente à celle séparant la Terre et la Lune}}
            \rput(6.25,2){\parbox{4cm}{Durée d'une saison}}
            \rput(-6.25,2){\parbox{4cm}{Record du monde du \um{100}}}
            \rput(-6.25,-2){\parbox{4cm}{Temps de cuisson d'un \oe uf à la coque}}
            \rput(6.25,6){\parbox{4cm}{Intervalle entre deux battements de c\oe ur consécutifs}}
            \rput(-6.25,10){\parbox{4cm}{Durée d'un cycle complet de lune}}
            \rput(6.25,-6){\parbox{4cm}{Durée d'un film}}
            \rput(6.25,-10){\parbox{4cm}{Temps mis par la Terre pour faire le tour de son étoile : le Soleil}}
            \rput(6.25,10){\parbox{4cm}{Âge maximum atteint par un humain}}
            \rput(-6.25,6){\parbox{4cm}{Durée d'une grossesse}}
            \rput(6.25,-2){\parbox{4cm}{Durée d'un entraînement de sport}}
            \rput(-6.25,-10){\parbox{4cm}{Durée d'un weekend}}
            \multido{\n=-9+3}{7}{\rput(0,\n){\psframe(-1.2,-0.5)(1.2,0.5)}}}
        \end{pspicture}
    \end{center}
\end{activite}
