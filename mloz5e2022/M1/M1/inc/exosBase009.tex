
\begin{exercice*}
    Zayd part en promenade à \uh{9} 20. Il rentre à \uh{12}15, ne s'étant arrêté pour se reposer que lors de trois pauses de \umin{5} chacune. \\
    Pendant combien de temps a-t-il marché ?
 \end{exercice*}
 
 \begin{corrige}
    Zayd part en promenade à \uh{9} 20. Il rentre à \uh{12}15, ne s'étant arrêté pour se reposer que lors de trois pauses de \umin{5} chacune. \\
    Pendant combien de temps a-t-il marché ?

    {\red
    $\uh{9}\,\umin{20} \quad \xrightarrow{+\umin{40}} \quad \uh{10}\,\umin{00}$ \\
    $\uh{10}\,\umin{00} \quad \xrightarrow{+\uh{2} \, \umin{15}} \quad \uh{12}\,\umin{15}$ \\
    La promenade de Zayd a duré \uh{2}\,\umin{55}. \\
    Or, il s'est arrêtée $3\times\umin{5} =\umin{15}$ donc, il a marché durant \uh{2}\,\umin{40}.
    }
 \end{corrige}