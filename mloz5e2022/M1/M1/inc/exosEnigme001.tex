% Les enigmes ne sont pas numérotées par défaut donc il faut ajouter manuellement la numérotation
% si on veut mettre plusieurs enigmes
% \refstepcounter{exercice}
% \numeroteEnigme
\newcommand{\horloge}{\psset{fillstyle=none,linecolor=lightgray,linewidth=1.5mm}
      \multido{\n=0.0+2.8,\r=2.0+2.8}{4}{\psframe(0,\n)(11,\r)} %frames
      \multido{\n=2.75+2.75}{3}{\psline(\n,0)(\n,2)}
      \multido{\n=1+1}{10}{\psline(\n,2.8)(\n,4.8)}
      \multido{\n=2.75+2.75}{3}{\psline(\n,5.6)(\n,7.6)}
      \multido{\n=2.75+2.75}{3}{\psline(\n,8.4)(\n,10.4)}
      \pscircle(5.5,12.54){1.6}
      \multido{\n=2.4+2.8}{3}{\psarc(1.25,\n){0.4}{-90}{90}
         \psarc(4.25,\n){0.4}{90}{-90}
         \psline[linewidth=3mm](1.5,\n)(4,\n)
         \psarc(6.75,\n){0.4}{-90}{90}
         \psarc(9.75,\n){0.4}{90}{-90}
         \psline[linewidth=3mm](7,\n)(9.5,\n)}
      \psarc(4.75,10.75){0.35}{-90}{90}
      \psline[linewidth=2.5mm](5.1,10.75)(5.9,10.75)
      \psarc(6.25,10.75){0.35}{90}{-90}}


\begin{enigme}[Mengenlehreuhr]
   \partie[présentation]
      La {\bf Mengenlehreuhr} (en allemand \og Horloge de la théorie du jeu \fg) ou {\bf Berlin-Uhr} (\og Horloge de Berlin \fg) est la première horloge publique au monde qui indique l'heure au moyen de champs lumineux colorés, ce qui lui a valu d'entrer dans le livre Guinness des records lors de son installation le 17 juin 1975. \\
      {\it \small Source : \href{https://en.wikipedia.org/wiki/Mengenlehreuhr}{wikipedia}}
   \partie[recherche]
      À partir des images suivantes représentant l'horloge à différents moments de la journée, déterminer le fonctionnement de l'horloge. Par groupe, vous construirez une affiche récapitulant vos recherches. \\
      \begin{center}
      {\psset{unit=0.54}
      \begin{minipage}{6cm}
        \scalebox{0.8}{
         \begin{pspicture}(0,-1.5)(11,13.8)
            \rput(5.5,6.5){\includegraphics[width=6.7cm]{\currentpath/images/horloge_Berlin}}
         \end{pspicture}
        }
      \end{minipage}
      \hspace{2cm}
      \begin{minipage}{6cm}  
        \scalebox{0.8}{
         \begin{pspicture}(0,-1)(11,13.8)
            \psset{linecolor=yellow,fillstyle=solid,framearc=0.35}
               \psframe*(0,0)(2.75,2) %m1
               \psframe*(0,2.8)(5,4.8) %m5
               \pscircle*(5.5,12.54){1.6}
            \psset{linecolor=OrangeRed}            
               \psframe*(2,2.8)(3,4.8) \psframe*(5,2.8)(6,4.8) %mi
               \psframe*(0,8.4)(5.5,10.4) %h5  
            \horloge
            \rput(5.5,-1){\large Il est 10h31}
         \end{pspicture}
        }
      \end{minipage}
      
      \begin{minipage}{6cm}  
        \scalebox{0.8}{
         \begin{pspicture}(0,-1)(11,15.5)

            \psset{linecolor=yellow,fillstyle=solid,framearc=0.35}
               \psframe*(0,0)(2.75,2) %m1
               \psframe*(0,2.8)(1,4.8) %m5
            \psset{linecolor=OrangeRed}            
               \psframe*(0,5.6)(2.75,7.6) %h1
               \psframe*(0,8.4)(2.75,10.4) %h5  
            \horloge
            \rput(5.5,-1){\large Il est 6h06}
         \end{pspicture}
        }
      \end{minipage}
      \hspace{2cm}
      \begin{minipage}{6cm}  
        \scalebox{0.8}{
         \begin{pspicture}(0,-1)(11,15)
            \psset{linecolor=yellow,fillstyle=solid,framearc=0.35}
               \psframe*(0,2.8)(11,4.8) %m5
            \psset{linecolor=OrangeRed}            
               \psframe*(2,2.8)(3,4.8) \psframe*(5,2.8)(6,4.8) \psframe*(8,2.8)(9,4.8) %mi
               \psframe*(0,5.6)(8.25,7.6) %h1
            \horloge
            \rput(5.5,-1){\large Il est 3h55}
         \end{pspicture}
        }
      \end{minipage}}
      \end{center}
\end{enigme}

% Pour le corrigé, il faut décrémenter le compteur, sinon il est incrémenté deux fois
% \addtocounter{exercice}{-1}
\begin{corrige}
    {\red
    Chaque case allumée indique une durée écoulée :
    \begin{itemize}
       \item chaque case allumée de la première ligne en partant du haut représente 5 heures ;     
       \item chaque case allumée de la deuxième ligne représente 1 heure ;
       \item chaque case allumée de la troisième ligne représente 5 minutes. Les lumières rouges indiquent les quarts d’heure ;
       \item chaque case allumée de la dernière ligne représente 1 minute.
    \end{itemize}
   On additionne les durées pour obtenir l'heure.
    }
 \end{corrige}