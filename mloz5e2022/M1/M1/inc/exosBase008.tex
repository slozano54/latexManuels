\begin{exercice*}
    Douniya part du collège à pied à \uh{17}\,\umin{04}. Elle met \umin{15}\,\us{30} pour le trajet, \umin{5} pour acheter un pain au chocolat et \umin{7} pour dire au revoir aux copines (et copains !). \\
    À quelle heure arrivera-t-elle chez elle ?
 \end{exercice*}
 
 \begin{corrige}
    Douniya part du collège à pied à \uh{17}\,\umin{04}. Elle met \umin{15}\,\us{30} pour le trajet, \umin{5} pour acheter un pain au chocolat et \umin{7} pour dire au revoir aux copines (et copains !). \\
    À quelle heure arrivera-t-elle chez elle ?

    {\red Le temps de déplacement de Douniya est de \\
    $\umin{15}\,\us{30}+\umin{5}+\umin{7} =\umin{27}\,\us{30}$. \\
    Or, $\uh{17}\,\umin{04}+\umin{27}\,\us{30} =\uh{17}\,\umin{31}\,\us{30}$. \\
    Douniya devrait arriver chez elle à \uh{17}\,\umin{31}\,\us{30}.
    }
 \end{corrige}