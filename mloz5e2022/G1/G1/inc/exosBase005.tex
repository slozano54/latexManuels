\begin{exercice} %5
    Dans la figure ci-dessous, on sait que les droites $(xy)$ et
 $(tz)$ sont parallèles et on connaît la mesure de deux angles. En utilisant les données de la figure :
    \begin{enumerate}
        \item Donner la mesure en degrés des angles suivants : $\widehat{IJK}$, puis $\widehat{JKI}$, puis $\widehat{rIy}$, puis $\widehat{yIJ}$, puis $\widehat{xIK}$ et enfin $\widehat{KIJ}$.
        \item En déduire la nature du triangle $IJK$?
    \end{enumerate}
    \begin{center}    
    \begin{tikzpicture}
        \coordinate (z) at (-1,0);
        \coordinate (t) at (4,0);
        \coordinate (x) at (-1,1.73);
        \coordinate (y) at (4,1.73);
        \coordinate (K) at (0,0);
        \coordinate (J) at (2,0);        
        \coordinate (I) at (1,1.73); 
        \tkzDrawSegment(x,y);        
        \tkzDrawSegment(z,t);
        \tkzDrawLine[add=0.5 and 0.5](K,I);
        \tkzDrawLine[add=0.5 and 0.5](J,I);
        \tkzLabelPoint[left](x){$x$};
        \tkzLabelPoint[right](y){$y$};
        \tkzLabelPoint[left](z){$z$};
        \tkzLabelPoint[right](t){$t$};
        \tkzLabelPoint[above left](K){$K$};
        \tkzLabelPoints[xshift=-1](I);
        \tkzLabelPoint[below left](J){$J$};
        \tkzDefPointOnLine[pos=1.5](K,I) \tkzGetPoint{s}
        \tkzDefPointOnLine[pos=1.5](J,I) \tkzGetPoint{r}
        \tkzLabelPoint[above](r){$r$};
        \tkzLabelPoint[above](s){$s$};
        \pic [draw=black, -, "\ang{60}",angle eccentricity=1.6] {angle = y--I--s};        
        \pic [draw=black, -, "\ang{120}",angle eccentricity=1.6] {angle = t--J--r};        
     \end{tikzpicture}
    \end{center}
 \end{exercice}
 
 \begin{corrige}
 \ \\ [-5mm]
    \begin{enumerate}
        \item $\bullet$ Les angles $\widehat{IJK}$ et $\widehat{IJt}$ sont supplémentaires donc, ${\blue \widehat{IJK}} =180\up{\circ}-120\up{\circ}° ={\blue 60\up{\circ}}$. \\
           $\bullet$ Les angles $\widehat{JKI}$ et $\widehat{yIS}$ sont correspondants et $\widehat{yIS} =60\up{\circ}$ donc, {\blue $\widehat{JKI} =60\up{\circ}$}. \\
           $\bullet$ Les angles $\widehat{rIy}$ et $\widehat{IJt}$ sont correspondants et $\widehat{IJt} =120\up{\circ}$ donc, {\blue $\widehat{rIy} =120\up{\circ}$}. \\
           $\bullet$ Les angles $\widehat{yIJ}$ et $\widehat{rIy}$ sont supplémentaires donc, ${\blue \widehat{yIJ}} =180\up{\circ}-120\up{\circ}° ={\blue 60\up{\circ}}$. \\
           $\bullet$ Les angles $\widehat{xIK}$ et $\widehat{sIy}$ sont opposés par le sommet donc, ${\blue \widehat{xIK}} =\widehat{sIy} ={\blue 60\up{\circ}}$. Et enfin : \\
           ${\blue \widehat{KIJ}} =180\up{\circ} -\widehat{xIK}-\widehat{yIJ} =180\up{\circ}-60\up{\circ}-60\up{\circ} ={\blue 60\up{\circ}}$.
        \item Les trois angles du triangle sont égaux, donc {\blue le triangle $IJK$ est équilatéral}.
    \end{enumerate}
 \end{corrige}
 