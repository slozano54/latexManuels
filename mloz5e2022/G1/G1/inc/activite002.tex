\begin{activite}[Couples d'angles]
   \vspace*{-10mm}    
   {\bf Objectif :} faire découvrir la notion d'angles alternes-internes et d'angles correspondants. Des figures Geogebra seront projetées en même temps au tableau.
   {\renewcommand{\baselinestretch}{1.15}\selectfont        
   \partie[préparation]
   \vspace*{-5mm}
      La figure ci dessous représente deux droites $(d_1)$ et $(d_2)$ et une troisième droite $(\Delta)$ qui leur est sécante aux points $A$ et $B$. Placer une attache parisienne au niveau du point $A$ commun entre $(d_1)$ et $(\Delta)$ et une autre au niveau du point $B$ commun entre $(d_2)$ et $(\Delta)$. 
      \par\smallskip
      \begin{minipage}{0.7\linewidth}
         \begin{center}
            \begin{Geometrie}[CoinHD={(8.5u,4u)}]
               pair A[],B[];
               A0=u*(6,2.2);
               B0=u*(2,2);
               A1=A0 shifted (0.5u,1.2u);               
               A2=1.5[B0,A0];%pour le nom de (delta)  
               A3=1.3[A0,A1];%pour le nom de (d1)  
               B1=B0 shifted (-0.5u,1.2u);               
               B2=1.3[B0,B1];%pour le nom de (d2)  
               trace droite(A0,B0);
               trace droite(A0,A1);
               trace droite(B0,B1);
               label.rt(btex $(d_1)$ etex,A3);
               label.lft(btex $(d_2)$ etex,B2);
               label.bot(btex $(\Delta)$ etex,A2);
            \end{Geometrie}
      \end{center}
   \end{minipage}
   \hfill
   \begin{minipage}{0.45\linewidth}
      \begin{enumerate}
         \item Compléter la figure.
         \item Combien d'angles sont-ils matérialisés par cette configuration ?
         \par\pointilles[\linewidth]
         \par\pointilles[\linewidth]
         \par\pointilles[\linewidth]
      \end{enumerate}
   \end{minipage}
   \partie[angles alternes-internes]
   \vspace*{-10mm}
   \begin{enumerate}
      \item Au tableau, placer les deux aimants rouges sur deux angles vérifiant les conditions suivantes :
         \begin{itemize}
            \item les deux angles n'ont pas le même sommet ;
            \item ils sont situés de part et d'autre de la droite $(\Delta)$ ;
            \item ils sont situés \og entre \fg{} les droites $(d_1)$ et $(d_2)$.
         \end{itemize}
      Quelle est la mesure en degrés de chacun de ces deux angles ?
      \par\pointilles[\linewidth]
      \par\pointilles[\linewidth]               
      \item Combien y a-t-il de telles paires d'angles ?
      \par\pointilles[\linewidth]
      \begin{center}
         \fbox{Les angles ainsi construits sont dit {\bf alternes-internes}.}
      \end{center}
      \item Tracer ci-dessous une figure similaire à celle ci-dessus mais de telle sorte que les droites $(d_1)$ et $(d_2)$ soient parallèles, repérer deux angles alternes-internes par deux arcs puis donner la mesure de chacun de ces deux angles.
      \par\vspace*{30mm}
      \item En observant les résultats de la classe, quelle conjecture peut-on faire ?
      
      Au tableau nous allons afficher les mesures des angles choisis et modifier la figure.
      \par\pointilles[\linewidth]
      \par\pointilles[\linewidth]         
   \end{enumerate}
   \partie[angles correspondants]
   \vspace*{-10mm}
      \begin{enumerate}
         \item Au tableau, placer les deux aimants noirs sur deux angles vérifiant les conditions suivantes :
            \begin{itemize}
               \item les deux angles n'ont pas le même sommet ;
               \item ils sont situés du même côté que la droite $(\Delta)$ ;
               \item l'un est situé \og entre \fg{} les droites $(d_1)$ et $(d_2)$, l'autre à l'extérieur.
            \end{itemize}
         Quelle est la mesure en degrés de chacun de ces deux angles ?
         \par\pointilles[\linewidth]
         \par\pointilles[\linewidth]
         \par\pointilles[\linewidth]
         \item Combien y a-t-il de telles paires d'angles ?
         \par\pointilles[\linewidth]
         \begin{center}
         \fbox{Les angles ainsi construits sont dit {\bf correspondants}.}
         \end{center}
         \item  Tracer ci-dessous une figure similaire à celle ci-dessus mais de telle sorte que les droites $(d_1)$ et $(d_2)$ soient parallèles, repérer deux correspondants par deux arcs puis donner la mesure de chacun de ces deux angles.
         \par\vspace*{40mm}
         \item En observant les résultats de la classe, quelle conjecture peut-on faire ?
          
         Au tableau nous allons afficher les mesures des angles choisis et modifier la figure.
         \par\pointilles[\linewidth]
         \par\pointilles[\linewidth]         
      \end{enumerate}
   }
 \end{activite}