\section{Angles alternes-internes et correspondants}

\parbox{8cm}{Lorsque deux droites sont coupées par une droite sécante $(\Delta)$, on obtient huit angles. \\
{\it Dans la suite du cours, on se place dans cette configuration.}}
\hfill
\parbox{6.5cm}{\psset{unit=0.7}
   \begin{pspicture}(-0.5,-0.25)(6,3.5)
      \psline(0,1.5)(6,0.5)
      \psline(1,3)(6,3)
      \psline(1,0)(5,4)
      \psarc[linecolor=B2,doubleline=true](4,3){0.7}{0}{45}
      \rput(5,3.4){\textcolor{B2}{\small $A_2$}}
      \psarc[linecolor=J1,doubleline=true](4,3){0.7}{180}{225}
      \rput(3,2.5){\textcolor{J1}{\small $A_4$}}
      \psarc[linecolor=A1](4,3){0.5}{45}{180}
      \rput(3.7,3.8){\textcolor{A1}{\small $A_1$}}
      \psarc[linecolor=G1](4,3){0.5}{225}{0}
      \rput(4.35,2.25){\textcolor{G1}{\small $A_3$}}
      \psarc[linecolor=B2,doubleline=true](2.15,1.15){0.7}{-10}{45}
      \rput(3.2,1.4){\textcolor{B2}{\small $B_2$}}
      \psarc[linecolor=J1,doubleline=true](2.15,1.15){0.7}{170}{225}
      \rput(1.1,0.7){\textcolor{J1}{\small $B_4$}}
      \psarc[linecolor=A1](2.15,1.15){0.5}{45}{170}
      \rput(1.8,1.9){\textcolor{A1}{\small $B_1$}}
      \psarc[linecolor=G1](2.15,1.15){0.5}{225}{-10}
      \rput(2.4,0.3){\textcolor{G1}{\small $B_3$}}
      \rput(0.7,-0.3){$(\Delta)$}
      \rput(-0.5,1.5){$(d_1)$}
      \rput(0.5,3){$(d_2)$}
   \end{pspicture}}

\begin{definition}
   Deux angles sont {\bf alternes-internes} s'ils n'ont pas le même sommet, qu'ils sont situés de part et d'autre de la sécante $(\Delta)$ et qu'ils se situent \og entre \fg{} les droites $(d_1)$ et $(d_2)$.
\end{definition}

\begin{exemple*1}
Sur la figure, il y a deux couples d'angles alternes-internes : $A_4$ et $B_2$ ; $A_3$ et $B_1$.
\end{exemple*1}

\bigskip

\begin{definition}
   Deux angles sont {\bf correspondants} s'ils n'ont pas le même sommet, qu'ils sont situés du même côté de la sécante $(\Delta)$, l'un entre les deux droites $(d_1)$ et $(d_2)$ et l'autre à l'extérieur.
\end{definition}

\begin{exemple*1}
Sur la figure, il existe quatre couples d'angles correspondants :  $A_1$ et $B_1$ ; $A_2$ et $B_2$ ; $A_3$ et $B_3$ ; $A_4$ et $B_4$.
\end{exemple*1}