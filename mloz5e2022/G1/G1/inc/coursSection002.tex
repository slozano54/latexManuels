\section{Configurations particulières}

\begin{propriete}[Angles opposés par le sommet]
    Si deux angles sont opposés par le sommet alors ils ont la même mesure.
\end{propriete}

\begin{preuve}

    \vspace*{5mm}
    \begin{minipage}{0.3\linewidth}
        \begin{center}
            \begin{Geometrie}[CoinHD={(4u,3u)}]
                    pair A,A',B,B',O;
                    A=u*(0.5,0.5);
                    B=u*(1,2.5);
                    A'=u*(3.5,2.5);
                    B'=u*(3,0.5);
                    O=u*(2,1.5);
                    trace droite(A,A');
                    trace droite(B,B');
                    marque_s:=marque_s/3;
                    trace codesegments(A,O,O,A',2);
                    trace codesegments(B,O,O,B',3);                
                    trace marquesegment(A,A');
                    trace marquesegment(B,B');
                    trace marqueangle(B,O,A,0);
                    trace marqueangle(B',O,A',0);
                    fill coloreangle(B,O,A) withcolor red;
                    fill coloreangle(B',O,A') withcolor red;
                    marque_p:="non";
                    label.bot(btex $O$  etex,O);
                    label.bot(btex $B$  etex,B);
                    label.top(btex $B'$ etex,B');
                    label.top(btex $A$  etex,A);
                    label.bot(btex $A'$ etex,A');
            \end{Geometrie}
        \end{center}
    \end{minipage}
    \hfill
    \begin{minipage}{0.6\linewidth}
        Ils sont symétriques par rapport au point d'intersection des droites sécantes qui les forment.        
    \end{minipage}
\end{preuve}

\begin{propriete}[Angles alternes-internes]
    Si deux droites \textbf{parallèles} sont coupées par une troisième alors deux angles en position d'angles alternes-internes sont égaux.
\end{propriete}

\begin{preuve}

    \begin{minipage}{0.6\linewidth}
        On définit :
        \begin{itemize}
            \item  $(d_{1})$ et $(d_{2})$ les droites parallèles, $(d_{3})$ la sécante.
            \item  $I=(d_{1})\cap(d_{3})$
            \item  $I'=(d_{2})\cap(d_{3})$
            \item  $O$ le milieu de $[II']$
            \item  $J\in (d_{1})$ et $J'=(d_{2}\cap (JO)$ c'est le symétrique de J par rapport à O!
        \end{itemize}
        On démontre alors que des angles en position d'angles alternes-internes sont symétriques par rapport au poit O.
    \end{minipage}
    \hfill
    \begin{minipage}{0.3\linewidth}
        \begin{center}
            \begin{Geometrie}[CoinBG={(0,u)},CoinHD={(6u,6.5u)}]
                    pair I,I',O,J,J',A,B,C,D,E;
                    path d[];
                    z1=u*(0,1)=A;
                    z2=u*(6,2)=B;
                    z3=u*(0,5)=C;
                    z4=u*(6,6)=D;
                    z5=u*(1,7);
                    z6=u*(5,0)=E;
                    d1=z1--z2;d2=z3--z4;d3=z5--z6;
                    I=d1 intersectionpoint d3;
                    I'=d2 intersectionpoint d3;
                    z7=u*(1.2,1.2)=J;
                    z8=1/2[I,I']=O;
                    label.urt (btex $I'$ etex,I');
                    label.llft(btex $I $ etex,I );
                    label.llft(btex $J $ etex,J );
                    marque_p:="croix";
                    label.rt(btex $O$ etex,O);
                    trace d1;trace d2;trace d3;trace demidroite(J,O);
                    fill coloreangle(C,I',I) withcolor red;
                    fill coloreangle(B,I,I') withcolor red;
                    fill coloreangle(E,I',D) withcolor blue;
                    fill coloreangle(E,I,B) withcolor blue;
            \end{Geometrie}
        \end{center}
    \end{minipage}

\end{preuve}

\begin{propriete}[Angles correspondants]
    Si deux droites \textbf{parallèles} sont coupées par une troisième alors deux angles en position d'angles correspondants sont égaux.
\end{propriete}

\begin{preuve}
    On reprend l'idée précédente pour définir un centre de symétrie.
    Puis on passe par la propriété des angles opposés par le sommet puis celle des angles alternes-internes.
\end{preuve}
