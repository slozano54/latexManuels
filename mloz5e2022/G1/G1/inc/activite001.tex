\begin{activite}[Couples d'angles]
    \vspace*{-7mm}    
    {\bf Objectif :} faire découvrir la notion d'angles alternes-internes et d'angles correspondants.
    {\renewcommand{\baselinestretch}{1.15}\selectfont    
    \vspace*{-3mm}
    \partie[préparation]
    \vspace*{-5mm}
       Découper les trois bandelettes en bas de page. Ces bandes représentent deux droites $(d_1)$ et $(d_2)$ et une troisième droite $(\Delta)$ qui leur est sécante aux points $A$ et $B$. Placer une attache parisienne au niveau du point $A$ commun entre $(d_1)$ et $(\Delta)$ et une autre au niveau du point $B$ commun entre $(d_2)$ et $(\Delta)$. \\
       Combien d'angles sont-ils matérialisés par cette configuration ?
    \vspace*{-3mm}
    \partie[angles alternes-internes]
    \vspace*{-10mm}    
       \begin{enumerate}
          \item Prendre deux jetons, les placer sur deux angles vérifiant les conditions suivantes :
             \begin{itemize}
                \item les deux angles n'ont pas le même sommet ;
                \item ils sont situés de part et d'autre de la droite $(\Delta)$ ;
                \item ils sont situés \og entre \fg{} les droites $(d_1)$ et $(d_2)$.
             \end{itemize}
          Quelle est la mesure en degrés de chacun de ces deux angles ?
          \item Combien y a-t-il de telles paires d'angles ?
          
          Les angles ainsi construits sont dit {\bf alternes-internes}.
          \item Placer les bandelettes de telle sorte que les droites $(d_1)$ et $(d_2)$ soient parallèles, repérer deux angles alternes-internes par deux jetons puis donner la mesure de chacun de ces deux angles.
          \item En observant les résultats de la classe, quelle conjecture peut-on faire ?
       \end{enumerate}
       \vspace*{-7mm}    
     \partie[angles correspondants]
     \vspace*{-10mm}     
       \begin{enumerate}
          \item Prendre deux jetons, les placer sur deux angles vérifiant les conditions suivantes :
             \begin{itemize}
                \item les deux angles n'ont pas le même sommet ;
                \item ils sont situés du même côté que la droite $(\Delta)$ ;
                \item l'un est situé \og entre \fg{} les droites $(d_1)$ et $(d_2)$, l'autre à l'extérieur.
             \end{itemize}
          Quelle est la mesure en degrés de chacun de ces deux angles ?
          \item Combien y a-t-il de telles paires d'angles ? \\
             Les angles ainsi construits sont dit {\bf correspondants}.
          \item Placer les bandelettes de telle sorte que les droites $(d_1)$ et $(d_2)$ soient parallèles, repérer deux angles correspondants par deux jetons puis donner la mesure de chacun de ces deux angles.
          \item En observant les résultats de la classe, quelle conjecture peut-on faire ?
       \end{enumerate}
    }
    \vspace*{-9mm}    
    \begin{pspicture}(0.5,1)(17.5,5)
       \psline(0.75,0.75)(17.25,0.75)
       \rput(1,1.1){$(d_1)$}
       \rput(12,1.25){$A$}
       \psline(0.75,2.25)(17.25,2.25)
       \rput(1,2.6){$(\Delta)$}
       \rput(6,2.75){$B$}
       \rput(12,2.75){$A$}
       \psline(0.75,3.75)(17.25,3.75)
       \rput(1,4.1){$(d_2)$}
       \rput(6,4.25){$B$}
       \psdots(12,0.75)(6,2.25)(12,2.25)(6,3.75)
       \psset{linecolor=gray}
       \psframe(0,0)(18,4.5)
       \psline(0,1.5)(18,1.5)
       \psline(0,3)(18,3)
    \end{pspicture}
 \end{activite}