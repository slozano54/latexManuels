\begin{exercice*}
    Sur la figure ci-dessous :
    \begin{itemize}
       \item les droites $(ab), (cd)$ et $(ef)$ sont parallèles ;
       \item $R$ est un point de $(ab)$, $S$ un point de $(cd)$ et $T$ un point de $(ef)$ tels que $\widehat{bRS} =\ang{20}$ et $\widehat{RST} =\ang{57}$.
    \end{itemize}
    Calculer la mesure de l'angle $\widehat{STf}$. \\
    \begin{center}    
      \begin{tikzpicture}
         \coordinate (e) at (0,0.5);
         \coordinate (T) at (2,0.5);
         \coordinate (f) at (7,0.5);
         \coordinate (c) at (0,2);
         \coordinate (S) at (4,2);
         \coordinate (d) at (7,2);
         \coordinate (a) at (0,3);
         \coordinate (R) at (2,3);
         \coordinate (b) at (7,3);   
         \tkzDrawSegment(a,b);       
         \tkzDrawSegment(c,d);
         \tkzDrawSegment(e,f);
         \tkzDrawSegment(R,S);
         \tkzDrawSegment(S,T);
         \tkzLabelPoint[left](a){$a$};
         \tkzLabelPoint[left](c){$c$};
         \tkzLabelPoint[left](e){$e$};
         \tkzLabelPoint[right](b){$b$};
         \tkzLabelPoint[right](d){$d$};
         \tkzLabelPoint[right](f){$f$};
         \tkzLabelPoints[above](R);
         \tkzLabelPoints[above](S);
         \tkzLabelPoints[above](T);
         \pic [draw=black, -, "\ang{20}",angle eccentricity=2] {angle = S--R--b};                 
         \pic [draw=black, -, angle eccentricity=1.6,angle radius=0.6cm] {angle = R--S--T};        
         \pic [draw=black, -, "\ang{57}",angle eccentricity=1.5,angle radius=0.7cm,below] {angle = R--S--T};        
         \pic [draw=black, -, "?",angle eccentricity=1.6] {angle = f--T--S};        
       \end{tikzpicture}
   \end{center}
\end{exercice*}
 
\begin{corrige}
   \begin{itemize}
      \item Les angles $\widehat{bRS}$ et $\widehat{RSc}$ sont alternes-internes et les droites $(ab)$ et $(cd)$ sont parallèles donc : $\widehat{RSc} =\widehat{bRS} =\red \ang{20}$.
      \item On décompose l'angle $\widehat{RST}$ : $\widehat{RST} =\widehat{RSc}+\widehat{cST}$ donc, $\widehat{cST} =\widehat{RST} -\widehat{RSc} =\ang{57}-\ang{20} =\red \ang{37}$.
      \item Les angles $\widehat{cST}$ et $\widehat{STf}$ sont alternes-internes et les droites $(cd)$ et $(ef)$ sont parallèles donc : $\widehat{STf} =\widehat{cST} =\red \ang{37}$.
   \end{itemize}
\end{corrige}