\begin{exercice*}
   Anita pense que l'une des deux paires de droites $(d_1)$ et $(d_2)$ est parallèle. A-t-elle raison ? 
   \par\vspace*{-2mm}
   \begin{center}      
      \begin{tikzpicture}[scale=0.7]
         \coordinate (A) at (0,2.5);
         \coordinate (B) at (4,0);
         \coordinate (C) at (0,1);        
         \coordinate (D) at (3,2);
         \coordinate (E) at (1,0);
         \coordinate (F) at (4,1);
         \tkzDrawLine(A,B);        
         \tkzDrawLine(C,D);
         \tkzDrawLine(E,F);
         \tkzInterLL(A,B)(C,D);
         \tkzGetPoint{H};            
         \tkzInterLL(A,B)(E,F);
         \tkzGetPoint{I};            
         \pic [draw=black, -, "\ang{119}",angle eccentricity=1.6] {angle = D--H--A};        
         \pic [draw=black, -, "\ang{61}",angle eccentricity=1.6] {angle = B--I--F};        
         \tkzLabelLine[pos=1.5,left](C,D){$(d_1)$}
         \tkzLabelLine[pos=1.5,left](E,F){$(d_2)$}
      \end{tikzpicture}
      \hfill
      \begin{tikzpicture}[scale=0.7]
         \coordinate (A) at (0,1.5);
         \coordinate (B) at (3,1);
         \coordinate (C) at (0,0);        
         \coordinate (D) at (1.5,3);
         \coordinate (E) at (1.5,0);
         \coordinate (F) at (2.5,2.5);
         \tkzDrawLine(A,B);        
         \tkzDrawLine(C,D);
         \tkzDrawLine(E,F);
         \tkzInterLL(A,B)(C,D);
         \tkzGetPoint{H};            
         \tkzInterLL(A,B)(E,F);
         \tkzGetPoint{I};            
         \pic [draw=black, -, "\ang{59}",angle eccentricity=1.6] {angle = B--H--D};        
         \pic [draw=black, -, "\ang{111}",angle eccentricity=1.6] {angle = E--I--B};        
         \tkzLabelLine[pos=1.25,left](C,D){$(d_1)$}
         \tkzLabelLine[pos=1.25,right](E,F){$(d_2)$}
      \end{tikzpicture}
   \end{center}
 \end{exercice*}
 
 \begin{corrige}
    Oui, Anita a raison : \\
    \begin{itemize}
       \item Première figure : l'angle supplémentaire à \ang{119} de l'autre côté de $(d_1)$ vaut $\ang{180}-\ang{119} =\ang{61}$. \\
       On a deux angles correspondants de même mesure donc, {\red les droites $(d_1)$ et $(d_2)$ sont parallèles}.
       \item Deuxième figure : l'angle supplémentaire à \ang{111} de l'autre côté de $(d_2)$ vaut $\ang{180}-\ang{111} =\ang{69}$. \\
       On a deux angles alternes-internes de mesures différentes donc, {\red $(d_1)$ et $(d_2)$ ne sont pas parallèles}.
    \end{itemize}
 \end{corrige}