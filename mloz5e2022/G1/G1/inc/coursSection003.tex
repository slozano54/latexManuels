\section{Et si les droites sont parallèles ?}

\begin{propriete}
   \begin{itemize}
      \item Si les deux droites $(d_1)$ et $(d_2)$ sont parallèles, alors les angles alternes-internes et les angles correspondants sont égaux deux à deux.
      \item Si deux angles alternes-internes ou deux angles correspondants sont égaux, alors les droites $(d_1)$ et $(d_2)$ sont parallèles.
   \end{itemize}
   \ \\ [-14mm]
\end{propriete}

\begin{exemple}
   {\psset{unit=0.9}
   \begin{pspicture}(0,0.3)(6,3.2)
      \psline(1,1)(6,1)
      \psline(1,2.5)(6,2.5)
      \psline(2,0)(4,3)
      \rput(2.4,2.1){$\alpha =\udeg{56}$}
      \rput(3.5,1.5){$\beta$}
      \rput(1.8,0.5){$\gamma$}
      \psarc[linecolor=B2,doubleline=true](3.67,2.5){0.6}{182}{234}
      \psarc[linecolor=A1,doubleline=true](2.67,1){0.6}{2}{55}
      \psarc[linecolor=J1,doubleline=true](2.67,1){0.6}{182}{233}
      \rput(5.5,1.75){\parbox{1.5cm}{\small droites\\parallèles}}
   \end{pspicture}}
   \correction
   Mesures de $\beta$ et $\gamma$, sachant que les droites sont parallèles :
   \begin{itemize}
      \item $\alpha$ et $\beta$ sont des angles alternes-internes, ils ont donc même mesure. D'où : $\beta =\alpha =\udeg{56}$.
      \item $\alpha$ et $\gamma$ sont des angles correspondants, ils ont donc même mesure. D'où : $\gamma =\alpha =\udeg{56}$.
   \end{itemize}
\end{exemple}