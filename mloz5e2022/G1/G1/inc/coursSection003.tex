\section{Caractérisation angulaire du parallélisme}
\begin{remarque}
    $\ggg$ On les introduit à l'aide de la notion de réciproque, elles seront admises.
\end{remarque}

\begin{propriete}[Parallélisme et angles alternes-internes]
    \begin{minipage}{0.3\linewidth}
        \begin{center}           
            \begin{Geometrie}[CoinBG={(0,0.5u)},CoinHD={(6u,6u)}]
                pair I,I',O,J,J',A,B,C,D,E;
                path d[];
                z1=u*(0,1)=A;
                z2=u*(6,2)=B;
                z3=u*(0,5)=C;
                z4=u*(6,6)=D;
                z5=u*(1,7);
                z6=u*(5,0)=E;
                d1=z1--z2;d2=z3--z4;d3=z5--z6;
                I=d1 intersectionpoint d3;
                I'=d2 intersectionpoint d3;
                z7=u*(1.2,1.2)=J;
                z8=1/2[I,I']=O;
                label.urt (btex $I'$ etex,I');
                label.llft(btex $I $ etex,I );
                label.llft(btex $J $ etex,J );
                marque_p:="croix";
                pointe(O);
                label.rt(btex $O$ etex,O);
                trace d1;trace d2;trace d3;
                fill coloreangle(C,I',I) withcolor red;
                fill coloreangle(B,I,I') withcolor red;
                fill coloreangle(E,I',D) withcolor blue;
                fill coloreangle(I',I,J) withcolor blue;
           \end{Geometrie}
        \end{center}
    \end{minipage}
    \hfill
    \begin{minipage}{0.6\linewidth}
        Si deux droites et une sécante forment des angles alternes-internes égaux alors les droites qui les déterminent sont parallèles.
    \end{minipage}
\end{propriete}

\begin{preuve}
    À discuter oralement.
\end{preuve}

\begin{propriete}[Parallélisme et angles alternes-internes]
    \begin{minipage}{0.3\linewidth}
        \begin{center}
            \begin{Geometrie}[CoinBG={(0,0.5u)},CoinHD={(6u,6u)}]
                pair I,I',O,J,J',A,B,C,D,E;
                path d[];
                z1=u*(0,1)=A;
                z2=u*(6,2)=B;
                z3=u*(0,5)=C;
                z4=u*(6,6)=D;
                z5=u*(1,7);
                z6=u*(5,0)=E;
                d1=z1--z2;d2=z3--z4;d3=z5--z6;
                I=d1 intersectionpoint d3;
                I'=d2 intersectionpoint d3;
                z7=u*(1.2,1.2)=J;
                z8=1/2[I,I']=O;
                label.urt (btex $I'$ etex,I');
                label.llft(btex $I $ etex,I );
                label.llft(btex $J $ etex,J );
                marque_p:="croix";
                pointe(O);
                label.rt(btex $O$ etex,O);
                trace d1;trace d2;trace d3;
                fill coloreangle(E,I',D) withcolor blue;
                fill coloreangle(E,I,B) withcolor blue;
           \end{Geometrie}
        \end{center}
    \end{minipage}
    \hfill
    \begin{minipage}{0.6\linewidth}
        Si deux droites et une sécante forment des angles correspondants égaux alors les droites qui les déterminent sont parallèles.        
    \end{minipage}
\end{propriete}

\begin{preuve}
    À discuter oralement à l'aide de la propriété précédente.
\end{preuve}

\begin{exemple}
   {\psset{unit=0.9}
   \begin{pspicture}(0,0.3)(6,3.2)
      \psline(1,1)(6,1)
      \psline(1,2.5)(6,2.5)
      \psline(2,0)(4,3)
      \rput(2.4,2.1){$\alpha =\udeg{56}$}
      \rput(3.5,1.5){$\beta$}
      \rput(1.8,0.5){$\gamma$}
      \psarc[linecolor=B2,doubleline=true](3.67,2.5){0.6}{182}{234}
      \psarc[linecolor=A1,doubleline=true](2.67,1){0.6}{2}{55}
      \psarc[linecolor=J1,doubleline=true](2.67,1){0.6}{182}{233}
      \rput(5.5,1.75){\parbox{1.5cm}{\small droites\\parallèles}}
   \end{pspicture}}
   \correction
   Mesures de $\beta$ et $\gamma$, sachant que les droites sont parallèles :
   \begin{itemize}
      \item $\alpha$ et $\beta$ sont des angles alternes-internes, ils ont donc même mesure. D'où : $\beta =\alpha =\udeg{56}$.
      \item $\alpha$ et $\gamma$ sont des angles correspondants, ils ont donc même mesure. D'où : $\gamma =\alpha =\udeg{56}$.
   \end{itemize}
\end{exemple}