\begin{exercice*}
    Au regard de la figure, que peut-on dire des angles :
    \begin{multicols}{4}
    \begin{enumerate}
       \item 1 et 5 ?
       \item 2 et 6 ?
       \item 4 et 6 ?
       \item 8 et 1 ?
       \item 3 et 7 ?
       \item 3 et 5 ?
       \item 7 et 1 ?
       \item 4 et 8 ?
    \end{enumerate}
   \end{multicols}
    {\psset{unit=0.9}
    \begin{pspicture}(-0.5,0)(6,4.2)
       \psline(0,1.5)(6,0.5)
       \psline(1,3)(6,3)
       \psline(1,0)(5,4)
       \psarc(4,3){0.7}{0}{45}
       \rput(5,3.4){2}
       \psarc(4,3){0.7}{180}{225}
       \rput(3,2.5){4}
       \psarc(4,3){0.5}{45}{180}
       \rput(3.7,3.8){1}
       \psarc(4,3){0.5}{225}{0}
       \rput(4.35,2.25){3}
       \psarc(2.15,1.15){0.7}{-10}{45}
       \rput(3.2,1.4){6}
       \psarc(2.15,1.15){0.7}{170}{225}
       \rput(1.1,0.7){8}
       \psarc(2.15,1.15){0.5}{45}{170}
       \rput(1.8,1.9){5}
       \psarc(2.15,1.15){0.5}{225}{-10}
       \rput(2.4,0.3){7}
    \end{pspicture}}
 \end{exercice*}
 
 \begin{corrige}
    \ \\ [-5mm]
    \begin{enumerate}
       \item Les angles 1 et 5 sont {\red correspondants}.
       \item Les angles 2 et 6 sont {\red correspondants}.
       \item Les angles 4 et 6 sont {\red alternes-internes}.
       \item Les angles 8 et 1 ne sont {\red ni alternes-internes, ni adjacents}.
       \item Les angles 3 et 7 sont {\red correspondants}.
       \item Les angles 3 et 5 sont {\red alternes-internes}.
       \item Les angles 7 et 1 ne sont {\red ni alternes-internes, ni adjacents}.
       \item Les angles 4 et 8 sont {\red correspondants}.
    \end{enumerate}
 \end{corrige}