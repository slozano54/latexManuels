\begin{exercice} %2
   Dans la configuration suivante, citer :
   \begin{enumerate}
       \item la sécante ;
       \item deux angles correspondants ;
       \item deux angles alternes-internes.
   \end{enumerate}
   \begin{pspicture}(-3,-1.5)(3,1.2)
      \psline(-1.5,0)(2,0)
      \psline(-1,0.75)(2,-1.5)
      \psline(-1.5,-1)(2.5,-1)
      \rput(-1.75,0){$x$}
      \rput(-1.2,0.9){$y$}
      \rput(2.25,0){$t$}
      \rput(2.25,-1.6){$s$}
      \rput(2.75,-1){$u$}
      \rput(0.2,0.25){$O$}
      \rput(1.5,-0.75){$I$}
      \rput(-1.75,-1){$k$}
   \end{pspicture}
\end{exercice}

\begin{corrige}
   \ \\ [-5mm]
   \begin{enumerate}
       \item La sécante est {\blue la droite $(ys)$}.
       \item Il y a quatre couples d'angles correspondants : \\
          {\blue $\widehat{yOt}$ et $\widehat{OIu}$ ; \\
          $\widehat{yOx}$ et $\widehat{OIk}$ ; \\
          $\widehat{tOI}$ et $\widehat{uIs}$ ; \\
          $\widehat{xOI}$ et $\widehat{kIs}$}.
       \item Il y a deux couples d'angles alternes-internes : \\
          {\blue $\widehat{xOI}$ et $\widehat{OIu}$ \\
          $\widehat{tOI}$ et $\widehat{OIk}$}.
   \end{enumerate}
\end{corrige}