\begin{exercice*}
   Nora possède un champ en forme de quadrilatère $NORA$ dont les côtés $(NO)$ et $(RA)$ sont parallèles. Elle prend la mesure de deux angles et se demande si son quadrilatère peut être un parallélogramme.
   \begin{enumerate}
      \item Écrire sur le schéma ci-dessous la mesure de tous les angles existants.
      \item Les droites $(NA)$ et $(OR)$ sont-elles parallèles ?
      \item Quelle est alors la nature du quadrilatère $NORA$ ?
   \end{enumerate}
   \par\hspace*{-20mm}
   \begin{center}      
      \begin{tikzpicture}[scale=0.55]
         \coordinate (a) at (0,1);
         \coordinate (b) at (10,1);
         \coordinate (c) at (0,4);
         \coordinate (d) at (10,4);
         \coordinate (f) at (0.33,0);
         \coordinate (g) at (4,5.5);
         \coordinate (h) at (6.33,0);
         \coordinate (j) at (9,5);
         \tkzDrawLine(a,b);       
         \tkzDrawLine(c,d);
         \tkzDrawLine(f,g);
         \tkzDrawLine(h,j);
         \tkzInterLL(a,b)(f,g);
         \tkzGetPoint{A};            
         \tkzInterLL(a,b)(h,j);
         \tkzGetPoint{R};            
         \tkzInterLL(c,d)(f,g);
         \tkzGetPoint{N};            
         \tkzInterLL(c,d)(h,j);
         \tkzGetPoint{O};            
         \tkzLabelPoints[above left](N,A);
         \tkzLabelPoints[below right](O,R);
         \pic [draw=black, -, "\ang{121}",angle eccentricity=1.6] {angle = j--O--N};
         \pic [draw=black, -, "\ang{49}",angle eccentricity=1.6] {angle = R--A--N};
      \end{tikzpicture}
   \end{center}
\end{exercice*}
 
\begin{corrige}
 \ \\ [-5mm]
    \begin{enumerate}
      \item Schéma du terrain de Nora : \par
         \begin{tikzpicture}[scale=0.55]
            \coordinate (a) at (0,1);
            \coordinate (b) at (10,1);
            \coordinate (c) at (0,4);
            \coordinate (d) at (10,4);
            \coordinate (f) at (0.33,0);
            \coordinate (g) at (4,5.5);
            \coordinate (h) at (6.33,0);
            \coordinate (j) at (9,5);
            \tkzDrawLine(a,b);       
            \tkzDrawLine(c,d);
            \tkzDrawLine(f,g);
            \tkzDrawLine(h,j);
            \tkzInterLL(a,b)(f,g);
            \tkzGetPoint{A};            
            \tkzInterLL(a,b)(h,j);
            \tkzGetPoint{R};            
            \tkzInterLL(c,d)(f,g);
            \tkzGetPoint{N};            
            \tkzInterLL(c,d)(h,j);
            \tkzGetPoint{O};            
            \tkzLabelPoints[above left](N,A);
            \tkzLabelPoints[below right](O,R);
            \pic [draw=black, -, "\ang{121}",angle eccentricity=1.6] {angle = j--O--N};                 
            \pic [draw=black, -, "\ang{49}",angle eccentricity=1.6] {angle = R--A--N};                 
         \end{tikzpicture}
      \item Si les droites $(NA)$ ET $(OR)$ étaient parallèles, les angles correspondants en $N$ et $O$ par exemple seraient égaux, ce qui n'est pas le cas ici (\ang{131}$\neq$\ang{121}) donc, {\red ces droites ne sont pas parallèles}.
      \item Dans le quadrilatère $NORA$, les droites $(NO)$ et $(RA)$ sont parallèles, mais les droites $(NA)$ et $(OR)$ ne le sont pas donc, {\red le quadrilatère $NORA$ est un trapèze}.   
   \end{enumerate}
\end{corrige}