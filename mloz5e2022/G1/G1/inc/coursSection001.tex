\section{Vocabulaire}
\begin{changemargin}{0mm}{-25mm}

   \begin{myBox}{\emoji{light-bulb} Rappels}
     \vspace*{5mm} 
      \begin{pspicture}(-5,0.25)(5,2.5)
         \rput(0,-0.3){O}
         \rput(3.8,0){angle nul : \ang{0}}
         \pswedge[fillstyle=solid,fillcolor=B2,linecolor=B2](0,0){1.5}{0}{90}
         \rput(2.5,1.5){\parbox{2.1cm}{\textcolor{B2}{angle aigu : \\ \ang{0} < \pswedge[fillstyle=solid,fillcolor=B2,linecolor=B2](0.2,0){0.3}{0}{90} \qquad < \ang{90}}}}
         \pswedge[fillstyle=solid,fillcolor=A1,linecolor=A1](0,0){1.5}{90}{180}
         \rput(-2.5,1.5){\parbox{2.5cm}{\textcolor{A1}{angle obtus : \\ \ang{90} < \pswedge[fillstyle=solid,fillcolor=A1,linecolor=A1](0.4,0){0.3}{90}{180} \quad\; < \ang{180}}}}
         \rput(0,2.5){angle droit : \ang{90}}
         \rput(-3.8,0){angle plat : \ang{180}}
         \psline(-2.5,0)(2.5,0)
         \psline(0,2)
      \end{pspicture}   
      \vspace*{10mm} 
   \end{myBox}

   \begin{definition}[Angles adjacents]
      \begin{minipage}{0.3\linewidth}
         \begin{center}
            \begin{Geometrie}[CoinHD={(5u,3u)}]
               pair A,B,C,D;
               A=u*(2,0);
               B=u*(1,1);
               C=u*(2,3);
               D=u*(4,0.5);
               trace demidroite(A,B);
               trace demidroite(A,C);
               trace demidroite(A,D);
               trace marqueangle(C,A,B,0);
               trace marqueangle(D,A,C,0);
               fill coloreangle(C,A,B) withcolor red;
               fill coloreangle(D,A,C) withcolor blue;
            \end{Geometrie}
         \end{center}
      \end{minipage}
      \hfill
      \begin{minipage}{0.6\linewidth}
         Deux angles sont \textbf{\underline{adjacents}} si :
         \begin{itemize}
            \item  ils ont le même sommet;
            \item  ils ont un côté commun;
            \item  ils sont de part et d'autre de ce côté commun.
         \end{itemize}
      \end{minipage}
   \end{definition}

   \begin{remarques}
   Dans chacun des cas suivants, les angles ne sont pas adjacents.
      \begin{changemargin}{-10mm}{-10mm}
         \begin{minipage}{0.25\linewidth}
            \begin{center}
               \begin{Geometrie}[CoinHD={(4.5u,3.5u)}]
                  pair A,B,C,D,E;
                  A=u*(2,0);
                  B=u*(1,1);
                  C=u*(2,3);
                  D=u*(4,0.5);
                  E=u*(2,1);
                  trace demidroite(E,B);
                  trace demidroite(A,C);
                  trace demidroite(A,D);
                  trace marqueangle(C,E,B,0);
                  trace marqueangle(D,A,C,0);
                  fill coloreangle(C,E,B) withcolor red;
                  fill coloreangle(D,A,C) withcolor blue;
               \end{Geometrie}
               \par
               Les deux angles n'ont pas le même sommet. 
            \end{center}
         \end{minipage}
         \hfill
         \begin{minipage}{0.25\linewidth}
            \begin{center}
               \begin{Geometrie}[CoinHD={(4.5u,3.5u)}]
                  pair A,B,C,D,E;
                  A=u*(2,0);
                  B=u*(1,1);
                  C=u*(2,3);
                  D=u*(4,0.5);
                  E=u*(4,0);
                  trace demidroite(A,B);
                  trace demidroite(A,C);
                  trace demidroite(A,D);
                  trace demidroite(A,E);
                  trace marqueangle(C,A,B,0);
                  trace marqueangle(E,A,D,0);
                  fill coloreangle(C,A,B) withcolor red;
                  fill coloreangle(E,A,D) withcolor blue;
               \end{Geometrie}
               \par
               Les deux angles n'ont pas de côté commun. 
            \end{center}
         \end{minipage}
         \hfill
         \begin{minipage}{0.4\linewidth}
            \begin{center}
               \begin{Geometrie}[CoinHD={(5u,4.3u)}]
                  pair A,B,C,D;
                  A=u*(2,0.5);
                  B=u*(1,1.5);
                  C=u*(2,2.5);
                  D=u*(4,1);
                  trace demidroite(A,B);
                  trace demidroite(A,C);
                  trace demidroite(A,D);
                  marque_p:="croix";
                  pointe(B,C,D);            
                  label.bot(btex A etex,A);
                  label.lft(btex B etex,B);
                  label.lft(btex C etex,C);
                  label.bot(btex D etex,D);
               \end{Geometrie}
               \par
               Les deux angles $\widehat{DAC}$ et $\widehat{DAB}$ ne sont pas situés de part et d'autre du côté commun. 
            \end{center}
         \end{minipage}
      \end{changemargin}
   \end{remarques}

   \begin{definition}[Angles opposés par le sommet]
      \begin{minipage}{0.3\linewidth}
         \begin{center}
            \begin{Geometrie}[CoinHD={(5u,3u)}]
               pair A,A',B,B',O;
               A=u*(0.5,0.5);
               B=u*(1,2.5);
               A'=u*(3.5,2.5);
               B'=u*(3,0.5);
               O=u*(2,1.5);
               trace droite(A,A');
               trace droite(B,B');
               trace marqueangle(B,O,A,0);
               trace marqueangle(B',O,A',0);
               fill coloreangle(B,O,A) withcolor red;
               fill coloreangle(B',O,A') withcolor red;
               trace marqueangle(A',O,B,0);
               trace marqueangle(A,O,B',0);
               fill coloreangle(A',O,B) withcolor blue;
               fill coloreangle(A,O,B') withcolor blue;
               labeloffset:=labeloffset*1.5;
               label.lft(btex O etex,O);
            \end{Geometrie}
         \end{center}
      \end{minipage}
      \hfill
      \begin{minipage}{0.6\linewidth}
         Deux angles sont {\bfseries opposés par le sommet} si ils sont de part et d'autre du point d'intersection de deux droites sécantes.
      \end{minipage}
   \end{definition}

   \begin{definition}[Angles alternes-internes]
      \begin{minipage}{0.3\linewidth}
         \begin{center}
            \begin{Geometrie}[CoinHD={(4u,7u)}]
               pair A,B,C,D,E,F;
               path d[];
               a:=2.75cm ;
               b:=3.75cm ;
               c:=3.25cm ;
               d:=6.25cm ;
               e:=1.75cm ;
               z1=(0,a);C=z1;
               z2=(b,b);
               z3=(0,b)=E ;
               z4=(c,d)=D ;
               z5=(a,0)=F ;
               z6=(e,d);
               d1=z1--z2;d2=z3--z4;d3=z5--z6;
               A=d1 intersectionpoint d3;
               B=d2 intersectionpoint d3;
               label.ulft(btex $B$ etex, B);
               label.lrt(btex $A$ etex, A);
               trace d1;trace d2;trace d3;
               fill coloreangle(B,A,C) withcolor red;
               fill coloreangle(A,B,D) withcolor red;
            \end{Geometrie}
         \end{center}
      \end{minipage}
      \hfill
      \begin{minipage}{0.6\linewidth}
         Soient deux droites $(d)$ et $(d')$, coupées par une sécante $(\Delta)$.
         Deux angles sont \textbf{\red alternes-internes} si :
         \begin{itemize}
            \item  ils sont entre les droites $(d)$ et $(d')$;
            \item  ils sont de part et d'autre de la sécante $(\Delta)$.
         \end{itemize}
      \end{minipage}
   \end{definition}

   \begin{definition}[Angles correspondants]
      \begin{minipage}{0.3\linewidth}
         \begin{center}
            \begin{Geometrie}[CoinHD={(4u,7u)}]
               pair A,B,C,D,E,F;
               path d[];
               a:=2.75cm ;
               b:=3.75cm ;
               c:=3.25cm ;
               d:=6.25cm ;
               e:=1.75cm ;
               z1=(0,a);C=z1;
               z2=(b,b);
               z3=(0,b)=E ;
               z4=(c,d)=D ;
               z5=(a,0)=F ;
               z6=(e,d);
               d1=z1--z2;d2=z3--z4;d3=z5--z6;
               A=d1 intersectionpoint d3;
               B=d2 intersectionpoint d3;
               label.ulft(btex $B$ etex,B);
               label.lrt( btex $A$ etex,A);
               trace d1;trace d2;trace d3;
               fill coloreangle(C,A,F) withcolor blue;
               fill coloreangle(E,B,F) withcolor blue;
            \end{Geometrie}
         \end{center}
      \end{minipage}
      \hfill
      \begin{minipage}{0.6\linewidth}
         Soient deux droites $(d)$ et $(d')$, coupées par une sécante $(\Delta)$.
         Deux angles sont \textbf{\red correspondants} si :
         \begin{itemize}
            \item  ils sont du même côté de la sécante $(\Delta)$;
            \item  l'un est à l'extérieur des droites $(d)$ et $(d')$;
            \item  l'autre est entre les droites $(d)$ et $(d')$.
         \end{itemize}
      \end{minipage}
   \end{definition}

   \begin{definition}[Angles complémentaires]
      \begin{minipage}{0.3\linewidth}
         \begin{center}
            \begin{Geometrie}[CoinHD={(6u,5u)}]
               pair A,B,C,E,F;
               A=u*(0.5,1);
               B=u*(2,1);
               C=u*(3,1);
               E=u*(2,4);
               F=u*(4,4);
               trace droite(A,B);
               trace demidroite(B,E);
               trace demidroite(B,F);
               trace codeperp(C,B,E,25);
               fill coloreangle(F,B,E) withcolor red;
               fill coloreangle(C,B,F) withcolor blue;
               label.bot(btex $B$ etex, B);
               marque_p:="croix";
               pointe(A,E,F);
               label.lrt(btex $A$ etex, A);               
               label.lft(btex $E$ etex, E);
               label.lft(btex $F$ etex, F);
            \end{Geometrie}
         \end{center}
      \end{minipage}
      \hfill
      \begin{minipage}{0.6\linewidth}
         Deux angles dont la somme vaut \ang{90} sont dits \textbf{complémentaires}
      \end{minipage}
   \end{definition}

   \begin{definition}[Angles supplémentaires]
      \begin{minipage}{0.3\linewidth}
         \begin{center}
            \begin{Geometrie}[CoinHD={(6u,5u)}]
               pair B,C,D,G;
               B=u*(0.5,1);
               C=u*(3,1);
               D=u*(5,1);
               G=u*(2,4);
               trace droite(B,D);
               trace demidroite(C,G);
               fill coloreangle(D,C,G) withcolor red;
               fill coloreangle(G,C,B) withcolor blue;
               label.bot(btex $C$ etex,C);
               marque_p:="croix";
               pointe(B,D,G);
               label.bot(btex $B$ etex,B);               
               label.llft(btex $D$ etex,D);
               label.lft(btex $G$ etex,G);
            \end{Geometrie}
         \end{center}
      \end{minipage}
      \hfill
      \begin{minipage}{0.6\linewidth}
         Deux angles dont la somme vaut \ang{180} sont dits \textbf{supplémentaires}
      \end{minipage}
   \end{definition}
\end{changemargin}