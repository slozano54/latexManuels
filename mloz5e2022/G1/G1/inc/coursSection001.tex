\section{Mesure d'angles particuliers : rappels}

\begin{minipage}{10cm}
   \begin{pspicture}(-5,0.25)(5,2.5)
      \rput(0,-0.3){O}
      \rput(3.8,0){angle nul : \udeg{0}}
      \pswedge[fillstyle=solid,fillcolor=B2,linecolor=B2](0,0){1.5}{0}{90}
      \rput(2.5,1.5){\parbox{2.1cm}{\textcolor{B2}{angle aigu : \\ \udeg{0} < \pswedge[fillstyle=solid,fillcolor=B2,linecolor=B2](0.2,0){0.3}{0}{90} \qquad < \udeg{90}}}}
      \pswedge[fillstyle=solid,fillcolor=A1,linecolor=A1](0,0){1.5}{90}{180}
      \rput(-2.5,1.5){\parbox{2.5cm}{\textcolor{A1}{angle obtus : \\ \udeg{90} < \pswedge[fillstyle=solid,fillcolor=A1,linecolor=A1](0.4,0){0.3}{90}{180} \quad\; < \udeg{180}}}}
      \rput(0,2.5){angle droit : \udeg{90}}
      \rput(-3.8,0){angle plat : \udeg{180}}
      \psline(-2.5,0)(2.5,0)
      \psline(0,2)
   \end{pspicture}   
\end{minipage}
\begin{minipage}{5.5cm}
   Dans cette configuration, la somme des deux angles mesure \udeg{180}, on dit que ces angles sont supplémentaires. \\
   \begin{pspicture}(7,1.5)
      \psline(0.5,0)(5.5,0)
      \psline(3,0)(1.8,1.2)
      \psarc[linecolor=B1](3,0){0.7}{135}{180}
      \rput(2,0.4){\textcolor{B1}{\udeg{45}}}
      \psarc[linecolor=A1](3,0){0.5}{0}{135}
      \rput(3.6,0.7){\textcolor{A1}{\udeg{135}}}
   \end{pspicture}
\end{minipage}