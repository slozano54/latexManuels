\section{Vocabulaire}

\begin{definition}[Angles adjacents]
   \begin{minipage}{0.3\linewidth}
      \begin{center}
         \begin{Geometrie}[CoinHD={(5u,3u)}]
            pair A,B,C,D;
            A=u*(2,0);
            B=u*(1,1);
            C=u*(2,3);
            D=u*(4,0.5);
            trace demidroite(A,B);
            trace demidroite(A,C);
            trace demidroite(A,D);
            trace marqueangle(C,A,B,0);
            trace marqueangle(D,A,C,0);
            fill coloreangle(C,A,B) withcolor red;
            fill coloreangle(D,A,C) withcolor blue;
         \end{Geometrie}
      \end{center}
   \end{minipage}
   \hfill
   \begin{minipage}{0.6\linewidth}
      Deux angles sont \textbf{\underline{adjacents}} si :
      \begin{itemize}
         \item  ils ont le même sommet;
         \item  ils ont un côté commun;
         \item  ils sont de part et d'autre de ce côté commun.
      \end{itemize}
   \end{minipage}
\end{definition}

\begin{remarques}
Dans chacun des cas suivants, les angles ne sont pas adjacents.
   \begin{changemargin}{-10mm}{-10mm}
      \begin{minipage}{0.25\linewidth}
         \begin{center}
            \begin{Geometrie}[CoinHD={(4.5u,3.5u)}]
               pair A,B,C,D,E;
               A=u*(2,0);
               B=u*(1,1);
               C=u*(2,3);
               D=u*(4,0.5);
               E=u*(2,1);
               trace demidroite(E,B);
               trace demidroite(A,C);
               trace demidroite(A,D);
               trace marqueangle(C,E,B,0);
               trace marqueangle(D,A,C,0);
               fill coloreangle(C,E,B) withcolor red;
               fill coloreangle(D,A,C) withcolor blue;
            \end{Geometrie}
            \par
            Les deux angles n'ont pas le même sommet. 
         \end{center}
      \end{minipage}
      \hfill
      \begin{minipage}{0.25\linewidth}
         \begin{center}
            \begin{Geometrie}[CoinHD={(4.5u,3.5u)}]
               pair A,B,C,D,E;
               A=u*(2,0);
               B=u*(1,1);
               C=u*(2,3);
               D=u*(4,0.5);
               E=u*(4,0);
               trace demidroite(A,B);
               trace demidroite(A,C);
               trace demidroite(A,D);
               trace demidroite(A,E);
               trace marqueangle(C,A,B,0);
               trace marqueangle(E,A,D,0);
               fill coloreangle(C,A,B) withcolor red;
               fill coloreangle(E,A,D) withcolor blue;
            \end{Geometrie}
            \par
            Les deux angles n'ont pas de côté commun. 
         \end{center}
      \end{minipage}
      \hfill
      \begin{minipage}{0.4\linewidth}
         \begin{center}
            \begin{Geometrie}[CoinHD={(5u,4.3u)}]
               pair A,B,C,D;
               A=u*(2,0.5);
               B=u*(1,1.5);
               C=u*(2,2.5);
               D=u*(4,1);
               trace demidroite(A,B);
               trace demidroite(A,C);
               trace demidroite(A,D);
               marque_p:="croix";
               pointe(B,C,D);            
               label.bot(btex A etex,A);
               label.lft(btex B etex,B);
               label.lft(btex C etex,C);
               label.bot(btex D etex,D);
            \end{Geometrie}
            \par
            Les deux angles $\widehat{DAC}$ et $\widehat{DAB}$ ne sont pas situés de part et d'autre du côté commun. 
         \end{center}
      \end{minipage}
   \end{changemargin}
\end{remarques}
% \definNumTitre{Angles opposés par le sommet}{
% \begin{minipage}{5cm}
% \begin{center}
% \includegraphics[scale=1]{anglesparallelisme.5} 
% \end{center}
% \end{minipage}
% \begin{minipage}{11cm}
% Deux angles sont \textbf{\underline{opposés par le sommet}} si ils sont de part et d'autre du point d'intersection de deux droites sécantes.
% \end{minipage}
% }

% \definNumTitre{Angles alternes-internes}{
% \begin{minipage}{5cm}
% \begin{center}
% \includegraphics[scale=1]{anglesparallelisme.10} 
% \end{center}
% \end{minipage}
% \begin{minipage}{11cm}
% Soient deux droites $(d)$ et $(d')$, coupées par une sécante $(\Delta)$.\\
% Deux angles sont \colorbox{red!70}{\textbf{\underline{alternes-internes}}} si :
% \begin{mylist}
% \item  ils sont entre les droites $(d)$ et $(d')$;
% \item  ils sont de part et d'autre de la sécante $(\Delta)$.
% \end{mylist}
% \end{minipage}
% }
% \definNumTitre{Angles correspondants}{
% \begin{minipage}{5cm}
% \begin{center}
% \includegraphics[scale=1]{anglesparallelisme.11} 
% \end{center}
% \end{minipage}
% \begin{minipage}{11cm}
% Soient deux droites $(d)$ et $(d')$, coupées par une sécante $(\Delta)$.\\
% Deux angles sont \colorbox{blue!70}{\textbf{\underline{correspondants}}} si :
% \begin{mylist}
% \item  ils sont du même côté de la sécante $(\Delta)$;
% \item  l'un est à l'extérieur des droites $(d)$ et $(d')$;
% \item  l'autre est entre les droites $(d)$ et $(d')$.
% \end{mylist}
% \end{minipage}
% }

% \definNumTitre{Angles complémentaires}{
% \begin{minipage}{5cm}
% \begin{center}
% \includegraphics[scale=1]{anglesparallelisme.12} 
% \end{center}
% \end{minipage}
% \begin{minipage}{11cm}
% Deux angles dont la somme vaut 90\degre sont dits \textbf{\underline{complémentaires}}
% \end{minipage}
% }

% \definNumTitre{Angles supplémentaires}{
% \begin{minipage}{5cm}
% \begin{center}
% \includegraphics[scale=1]{anglesparallelisme.13} 
% \end{center}
% \end{minipage}
% \begin{minipage}{11cm}
% Deux angles dont la somme vaut 180\degre sont dits \textbf{\underline{supplémentaires}}
% \end{minipage}
% }

%%%%%%%%%%%%%%%%%%%%
% \section{Mesure d'angles particuliers : rappels}

% \begin{minipage}{10cm}
%    \begin{pspicture}(-5,0.25)(5,2.5)
%       \rput(0,-0.3){O}
%       \rput(3.8,0){angle nul : \udeg{0}}
%       \pswedge[fillstyle=solid,fillcolor=B2,linecolor=B2](0,0){1.5}{0}{90}
%       \rput(2.5,1.5){\parbox{2.1cm}{\textcolor{B2}{angle aigu : \\ \udeg{0} < \pswedge[fillstyle=solid,fillcolor=B2,linecolor=B2](0.2,0){0.3}{0}{90} \qquad < \udeg{90}}}}
%       \pswedge[fillstyle=solid,fillcolor=A1,linecolor=A1](0,0){1.5}{90}{180}
%       \rput(-2.5,1.5){\parbox{2.5cm}{\textcolor{A1}{angle obtus : \\ \udeg{90} < \pswedge[fillstyle=solid,fillcolor=A1,linecolor=A1](0.4,0){0.3}{90}{180} \quad\; < \udeg{180}}}}
%       \rput(0,2.5){angle droit : \udeg{90}}
%       \rput(-3.8,0){angle plat : \udeg{180}}
%       \psline(-2.5,0)(2.5,0)
%       \psline(0,2)
%    \end{pspicture}   
% \end{minipage}
% \begin{minipage}{5.5cm}
%    Dans cette configuration, la somme des deux angles mesure \udeg{180}, on dit que ces angles sont supplémentaires. \\
%    \begin{pspicture}(7,1.5)
%       \psline(0.5,0)(5.5,0)
%       \psline(3,0)(1.8,1.2)
%       \psarc[linecolor=B1](3,0){0.7}{135}{180}
%       \rput(2,0.4){\textcolor{B1}{\udeg{45}}}
%       \psarc[linecolor=A1](3,0){0.5}{0}{135}
%       \rput(3.6,0.7){\textcolor{A1}{\udeg{135}}}
%    \end{pspicture}
% \end{minipage}