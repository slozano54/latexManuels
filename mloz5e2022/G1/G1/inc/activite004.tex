\begin{changemargin}{-10mm}{-20mm}
    \begin{activite}[Configurations particulières]
        \begin{propriete}[Angles opposés par le sommet]
            Si deux angles sont \textbf{opposés par le sommet} alors \pointilles
            \par\medskip\pointilles
            \medskip
        \end{propriete}
        \begin{exemple*1}
            \phantom{rrr}\par\bigskip
            \begin{minipage}{0.8\linewidth}
                \pointilles[15mm] et \pointilles[15mm] sont opposés par le sommet donc
                \par\medskip\pointilles.
            \end{minipage}
            \hfill
            \begin{minipage}{0.15\linewidth}
                \begin{center}
                    \scalebox{0.9}{
                        \begin{Geometrie}[CoinHD={(5u,3u)}]
                            pair A,A',B,B',O;
                            A=u*(0.5,0.5);
                            O=u*(2,1.5);
                            A'=rotation(A,O,180);
                            %B=u*(1,2.5);
                            B=rotation(A,O,-30);
                            % A'=u*(3.5,2.5);
                            %B'=u*(3,0.5);
                            B'=rotation(B,O,180);
                            trace droite(A,A');
                            trace droite(B,B');
                            marque_a:=1.25*marque_a;
                            trace marqueangle(B,O,A,0);
                            trace marqueangle(B',O,A',0);                            
                            fill coloreangle(B,O,A) withcolor blue;
                            fill coloreangle(B',O,A') withcolor blue;
                            marque_a:=marque_a/1.25;
                            trace marqueangle(A',O,B,0);
                            trace marqueangle(A,O,B',0);                            
                            fill coloreangle(A',O,B) withcolor red;
                            fill coloreangle(A,O,B') withcolor red;
                            labeloffset:=labeloffset*1.5;
                            label.ulft(btex O etex,O);
                            marque_p:="croix";
                            pointe(A,A',B,B');
                            label.top(btex A etex,A);
                            label.top(btex B etex,A');
                            label.top(btex C etex,B);
                            label.top(btex D etex,B');
                        \end{Geometrie}            
                    }
                \end{center}
            \end{minipage}
        \end{exemple*1}
        %%%%%%%%%
        % Alternes-Internes
        %%%%%%%%%
        \begin{propriete}[(directe) Angles alternes-internes]
            Si deux droites \textbf{parallèles} sont coupées par une troisième alors \pointilles
            \par\medskip\pointilles
            \par\medskip\pointilles
            \medskip
        \end{propriete}
        \begin{exemple*1}
            \phantom{rrr}\par\bigskip
            \begin{minipage}{0.8\linewidth}
                Comme la sécante \pointilles[15mm] forme, avec les droites parallèles \pointilles[15mm] et \par\medskip
                \pointilles[15mm], les angles \pointilles[30mm] et \pointilles[15mm] alors \pointilles\par\medskip\pointilles.
            \end{minipage}            
            \begin{minipage}{0.15\linewidth}
                \vspace*{-15mm}
                \begin{center}
                    \scalebox{0.9}{
                        \begin{Geometrie}[CoinBG={(-1.5u,-1.5u)},CoinHD={(8u,2.5u)}]
                            pair A,B,C,D,E,F,G;
                            C=u*(1,0);
                            A-C=u*(1.5,0.5);
                            G-C=u*(4.5,1.5);
                            B=0.9[A,rotation(C,A,-80)];
                            E=rotation(A,B,-100);
                            D=rotation(A,B,80);
                            F=0.6[A,rotation(C,A,100)];
                            labeloffset:=1.2*labeloffset;
                            label.llft(btex $B$ etex, B);
                            label.llft(btex $A$ etex, A);
                            labeloffset:=labeloffset/1.2;
                            trace droite(D,E);
                            trace droite(C,G);
                            trace droite(A,B);
                            marque_a:=0.7*marque_a;
                            fill coloreangle(B,A,C) withcolor red;
                            fill coloreangle(A,B,D) withcolor red;
                            marque_p:="croix";
                            pointe(C,D,E,F,G);
                            label.lrt(btex $C$ etex, C);
                            label.lrt(btex $D$ etex, D);
                            label.lrt(btex $E$ etex, E);
                            label.rt(btex $F$ etex, F);
                            label.lrt(btex $G$ etex, G);
                            pair H;
                            H-F=u*(0,-1);
                            label.lft(btex $(DE)$ et $(CG)$  sont parallèles. etex,H);
                        \end{Geometrie}        
                    }
                \end{center}
            \end{minipage}
        \end{exemple*1}
        \begin{propriete}[(réciproque) Angles alternes-internes]
            Si deux droites et une sécante forment des angles alternes-internes \textbf{égaux} alors
            \par\medskip\pointilles
            \par\medskip\pointilles            
            \medskip
        \end{propriete}
        \begin{exemple*1}
            \phantom{rrr}\par\bigskip
            \begin{minipage}{0.8\linewidth}
                Comme la sécante \pointilles[15mm] forme avec les droites \pointilles[15mm] et \pointilles[15mm] \par\medskip
                les angles \pointilles[40mm] et \pointilles[15mm] de même \pointilles[20mm] \par\medskip alors \pointilles.                
            \end{minipage}
            \begin{minipage}{0.15\linewidth}
                \vspace*{-15mm}
                \begin{center}
                    \scalebox{0.9}{
                        \begin{Geometrie}[CoinBG={(-u,-0.5u)},CoinHD={(7u,2.5u)}]
                            pair A,B,C,D,E,F,G;
                            C=u*(1,0);
                            A-C=u*(1.5,0.5);
                            G-C=u*(4.5,1.5);
                            B=0.9[A,rotation(C,A,-80)];
                            E=rotation(A,B,-100);
                            D=rotation(A,B,80);
                            F=0.6[A,rotation(C,A,100)];
                            labeloffset:=1.2*labeloffset;
                            label.llft(btex $B$ etex, B);
                            label.llft(btex $A$ etex, A);
                            labeloffset:=labeloffset/1.2;
                            trace droite(D,E);
                            trace droite(C,G);
                            trace droite(A,B);
                            marque_a:=0.7*marque_a;
                            fill coloreangle(B,A,C) withcolor red;
                            fill coloreangle(A,B,D) withcolor red;
                            trace Codeangle(B,A,C,0,btex \textcolor{red}{\ang{80}} etex);
                            trace Codeangle(A,B,D,0,btex \textcolor{red}{\ang{80}} etex);
                            marque_p:="croix";
                            pointe(C,D,E,F,G);
                            label.lrt(btex $C$ etex, C);
                            label.lrt(btex $D$ etex, D);
                            label.lrt(btex $E$ etex, E);
                            label.rt(btex $F$ etex, F);
                            label.lrt(btex $G$ etex, G);
                         \end{Geometrie}             
                    }
                \end{center}    
            \end{minipage}            
        \end{exemple*1}
        %%%%%%%%%
        % Correspondants
        %%%%%%%%%
        \begin{propriete}[(directe) Angles correspondants]
            Si deux droites \textbf{parallèles} sont coupées par une troisième alors \pointilles
            \par\medskip\pointilles
            \par\medskip\pointilles
            \medskip
        \end{propriete}
        \begin{exemple*1}
            \begin{center}
                Comme la sécante \pointilles[15mm] forme avec les droites parallèles \pointilles[15mm] et \pointilles[15mm]\par\medskip 
                les angles \pointilles[30mm] et \pointilles[15mm] alors \pointilles.
                \par\bigskip
                \scalebox{1}{
                    \begin{Geometrie}[CoinBG={(-u,-0.5u)},CoinHD={(9u,2.5u)}]
                        pair A,B,C,D,E,F,G;
                        C=u*(1,0);
                        A-C=u*(1.5,0.5);
                        G-C=u*(4.5,1.5);
                        B=0.9[A,rotation(C,A,-80)];
                        E=rotation(A,B,-100);
                        D=rotation(A,B,80);
                        F=0.6[A,rotation(C,A,100)];
                        labeloffset:=1.2*labeloffset;
                        label.top(btex $B$ etex, B);
                        label.top(btex $A$ etex, A);
                        labeloffset:=labeloffset/1.2;
                        trace droite(D,E);
                        trace droite(C,G);
                        trace droite(A,B);
                        marque_a:=0.7*marque_a;
                        fill coloreangle(E,B,A) withcolor red;
                        fill coloreangle(C,A,F) withcolor red;
                        marque_p:="croix";
                        pointe(C,D,E,F,G);
                        label.lrt(btex $C$ etex, C);
                        label.lrt(btex $D$ etex, D);
                        label.lrt(btex $E$ etex, E);
                        label.rt(btex $F$ etex, F);
                        label.lrt(btex $G$ etex, G);
                        pair H;
                        H-F=u*(1,0.5);
                        label.rt(btex $(DE)$ et $(CG)$  sont parallèles. etex,H);
                     \end{Geometrie}        
                }
            \end{center}            
        \end{exemple*1}
        \vfill
        \begin{propriete}[(réciproque) Angles correspondants]
            Si deux droites et une sécante forment des angles correspodants \textbf{égaux} alors
            \par\medskip\pointilles
            \par\medskip\pointilles            
            \medskip
        \end{propriete}
        \begin{exemple*1}
            \begin{center}
                Comme la sécante \pointilles[15mm] forme avec les droites \pointilles[15mm] et \pointilles[15mm] \par\medskip
                les angles \pointilles[40mm] et \pointilles[15mm] de même \pointilles[20mm] alors \par\medskip
                \par\medskip\pointilles.
                \par\medskip
                \scalebox{1}{
                    \begin{Geometrie}[CoinBG={(-u,-0.5u)},CoinHD={(7u,2.5u)}]
                        pair A,B,C,D,E,F,G;
                        C=u*(1,0);
                        A-C=u*(1.5,0.5);
                        G-C=u*(4.5,1.5);
                        B=0.9[A,rotation(C,A,-80)];
                        E=rotation(A,B,-100);
                        D=rotation(A,B,80);
                        F=0.6[A,rotation(C,A,100)];
                        labeloffset:=1.2*labeloffset;
                        label.urt(btex $B$ etex, B);
                        label.urt(btex $A$ etex, A);
                        labeloffset:=labeloffset/1.2;
                        trace droite(D,E);
                        trace droite(C,G);
                        trace droite(A,B);
                        marque_a:=0.7*marque_a;
                        fill coloreangle(E,B,A) withcolor red;
                        fill coloreangle(C,A,F) withcolor red;
                        trace Codeangle(E,B,A,0,btex \textcolor{red}{\ang{80}} etex);
                        trace Codeangle(C,A,F,0,btex \textcolor{red}{\ang{80}} etex);
                        marque_p:="croix";
                        pointe(C,D,E,F,G);
                        label.lrt(btex $C$ etex, C);
                        label.lrt(btex $D$ etex, D);
                        label.lrt(btex $E$ etex, E);
                        label.rt(btex $F$ etex, F);
                        label.lrt(btex $G$ etex, G);
                     \end{Geometrie}             
                }
            \end{center}
        \end{exemple*1}
        \vfill
    \end{activite}
\end{changemargin}