\vspace*{-5mm} 
%pre-001
\begin{prerequis}[Connaisances \emoji{red-heart} et compétences \emoji{diamond-suit} du cycle 3]    
   \begin{itemize}        
       \item[\emoji{red-heart}] Vocabulaire associé à ces objets et à leurs propriétés : côté, sommet, angle, hauteur.
       \columnbreak
       \item[\emoji{diamond-suit}] Reconnaître, nommer, décrire des triangles, dont les triangles particuliers (triangle rectangle, triangle isocèle, triangle équilatéral).       
   \end{itemize}
\end{prerequis}
\vspace*{-5mm} 
\medskip
%pre-002
\begin{prerequis}[Connaisances \emoji{red-heart} et compétences \emoji{diamond-suit} du cycle 4]    
    \begin{itemize}        
        \item[\emoji{diamond-suit}] Mener des calculs impliquant des grandeurs mesurables, exprimer les résultats dans des les unités adaptées.
        \item[\emoji{diamond-suit}] Exprimer et vérifier la cohérence des résultats du point de vue des unités.
    \end{itemize}
\end{prerequis}
\begin{debat}[Débat : angles et coordonnées géographiques]
    Tout point à la surface de la Terre est déterminé par ses coordonnées géographiques (la latitude et la longitude) et par son altitude (élévation par rapport au niveau de la mer). \\
    \begin{minipage}{10.5cm}
       \begin{itemize}
          \item La {\bf latitude} d'un point sur la Terre est la mesure de l'angle que forment le plan de l'équateur et la demi-droite joignant le centre de la Terre à ce point.
          \item La {\bf longitude} d'un point est l'angle que fait le demi-plan passant par le méridien de ce point avec le plan du méridien de Greenwich.
       \end{itemize}
       Le collège Jean Lurçat de Frouard se trouve à une latitude de $\num{48.94}$ degrés Nord et $\num{2.38}$ degrés Est.
    \end{minipage}
    \hfill
    \begin{minipage}{5cm}
       \includegraphics[width=4cm]{\currentpath/images/Longitude_latitude}
    \end{minipage} 
    \bigskip
    \begin{cadre}[B2][F4]
       \begin{center}
          Vidéo : \href{https://www.yout-ube.com/watch?v=lpYEuHeecko}{{\bf Les fondamentaux : latitude et longitude}}

          Chaîne YouTube {\it La Classe d'Histoire}.
       \end{center}
    \end{cadre}
\end{debat}