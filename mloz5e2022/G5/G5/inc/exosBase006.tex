\begin{exercice*}
   Loutfi a trouvé un triangle sympa dont tous les angles ont pour mesure un entier pair : \ang{44}, \ang{66} et \ang{70}.
   \begin{enumerate}
      \item Trouver un autre exemple de triangle dont les mesures d'angles sont paires.
      \item En poursuivant ses recherches, elle a trouvé un triangle dont les mesures sont des multiples de trois : \ang{45}, \ang{51} et \ang{84}. Trouve un autre exemple de triangle dont les mesures d'angles sont des multiples de trois.
      \item Continuer les recherches en trouvant un triangle dont les mesures d'angles sont des multiples de quatre.
      \item Cela est-il possible avec tous les nombres entiers ?
   \end{enumerate}
\end{exercice*}
\begin{corrige}
   %\setcounter{partie}{0} % Pour s'assurer que le compteur de \partie est à zéro dans les corrigés
   \begin{enumerate}
      \item On peut choisir, par exemple, {\red $\ang{60} ; \ang{60}$ et $\ang{60}$}.
      \item On peut choisir, par exemple, {\red $\ang{30} ; \ang{60}$ et $\ang{90}$}.
      \item On peut choisir, par exemple, {\red $\ang{40} ; \ang{60}$ et $\ang{80}$}.
      \item {\red Non}, cela n'est pas possible par exemple avec des multiples de 7 : dans ce cas, la somme des angles est un multiple de 7 et doit être égale à \ang{180} ce qui n'est pas possible puisque 7 n'est pas un diviseur de 180.
   \end{enumerate}
\end{corrige}

