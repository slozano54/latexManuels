\begin{exercice*}
   Choisir trois nombres du tableau (chacun une fois) correspondant aux longueurs des côtés d'un triangle :
   \begin{enumerate}
      \item non constructible ;
      \item quelconque ;
      \item isocèle ;
      \item de périmètre \Lg[cm]{13}.
   \end{enumerate}
   \begin{center}
      {\renewcommand{\arraystretch}{1.5}
      \begin{tabular}{|*{4}{>{\centering\arraybackslash}p{1cm}|}}
         \hline
         \Lg[cm]{8} & \Lg[cm]{5} & \Lg[cm]{12} & \Lg[cm]{2} \\
         \hline
         \Lg[cm]{10} & \Lg[cm]{12} & \Lg[cm]{15} & \Lg[cm]{10} \\
         \hline
         \Lg[cm]{9} & \Lg[cm]{3} & \Lg[cm]{5} & \Lg[cm]{7} \\
         \hline
      \end{tabular}}
   \end{center}   
\end{exercice*}
\begin{corrige}
   %\setcounter{partie}{0} % Pour s'assurer que le compteur de \partie est à zéro dans les corrigés
   On a, par exemple (solution non unique) un triangle :
   
   \begin{enumerate}
      \item non constructible : {\red \Lg[cm]{15} ; \Lg[cm]{8} et \Lg[cm]{2}}.
      \item quelconque : {\red \Lg[cm]{10} ; \Lg[cm]{9} et \Lg[cm]{7}}.
      \item isocèle : {\red \Lg[cm]{12} ; \Lg[cm]{12} et \Lg[cm]{10}}.
      \item de périmètre \Lg[cm]{13} : {\red \Lg[cm]{5} ; \Lg[cm]{5} et \Lg[cm]{3}}.
   \end{enumerate}
\end{corrige}

