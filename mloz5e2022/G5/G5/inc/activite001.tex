\begin{activite}[Avec des allumettes]
   \begin{changemargin}{-10mm}{-10mm}
      {\bf Objectifs :} construire des triangles sous contraintes.

      Devant vous, vous avez dix allumettes. Pour chacune des questions suivantes, faire la construction si elle est possible avec des allumettes puis faire un dessin pour schématiser la situation.
      \begin{enumerate}
         \item
         \begin{enumerate}
            \item Aligner quatre allumettes en les plaçant les unes à côté des autres.
            \smallskip
            \begin{center}
               \includegraphics[width=3cm]{\currentpath/images/allumette}
               \includegraphics[width=3cm]{\currentpath/images/allumette}
               \includegraphics[width=3cm]{\currentpath/images/allumette}
               \includegraphics[width=3cm]{\currentpath/images/allumette}
            \end{center}
            \smallskip
            \item À partir de ce segment de longueur 4 allumettes, construire un triangle dont les deux autres côtés ont pour longueur trois allumettes.
            \vspace*{27mm}
            \item En utilisant les dix allumettes, construire un triangle différent du précédent dont un des côtés a pour longueur quatre allumettes. Quelles sont les longueurs de ses côtés ?
            \vspace*{27mm}
         \end{enumerate}
         \item En utilisant les dix allumettes, est-il possible de construire un triangle dont un des côtés a pour longueur six allumettes ? sept allumettes ? Expliquer.
         \vspace*{27mm}
         \item En utilisant les dix allumettes, peut-on construire un triangle dont un côté a pour longueur cinq allumettes ? Que constate-t-on dans ce cas ?
         \vspace*{27mm}
         \item On veut maintenant construire un triangle de périmètre 15 allumettes dont les côtés ont pour longueur un nombre entier d'allumettes. Donner toutes les solutions possibles.         
      \end{enumerate}
   \end{changemargin}
\end{activite}
