%\begin{changemargin}{-10mm}{-10mm}
%pre-001
\begin{prerequis}[Connaisances \emoji{red-heart} et compétences \emoji{diamond-suit} du cycle 3]    
   \begin{itemize}        
       \item[\emoji{red-heart}] Vocabulaire associé à ces objets et à leurs propriétés : côté, sommet, angle, hauteur.
       \columnbreak
       \item[\emoji{diamond-suit}] Reconnaître, nommer, décrire des triangles, dont les triangles particuliers (triangle rectangle, triangle isocèle, triangle équilatéral).       
   \end{itemize}
\end{prerequis}
\begin{debat}[Des instruments de navigation astronomique anciens] 
   De tous temps, les hommes ont cherché à se repérer. Avant l'avénement de l'électronique et des GPS, de multiples instruments ont pu exister, par exemple : 
   \begin{center}
   \textcolor{B1}{\small
      \begin{tabular}{*{5}{>{\centering\arraybackslash}p{3cm}}}
           \includegraphics[height=2cm]{\currentpath/images/sphere} 
         & \includegraphics[height=2cm]{\currentpath/images/astrolabe} 
         & \includegraphics[height=2cm]{\currentpath/images/boussole}  
         & \includegraphics[height=2cm]{\currentpath/images/octant} 
         & \includegraphics[height=2cm]{\currentpath/images/sextant} \\
        Sphère armillaire : & Astrolabe : & Boussole : & Octant : & Sextant : \\
        modélisation de la sphère céleste & représentation plane de la sphère armillaire & indique le nord magnétique & mesure la hauteur des corps célestes (\udeg{45}) & mesure la hauteur des corps célestes (\udeg{60}) \\
        & & & & \\
        Antiquité & Antiquité & {\small XIII}\up{e} & {\small XVIII}\up{e} & {\small XVIII}\up{e} \\
     \end{tabular}}
   \end{center}   
   \begin{cadre}[B2][J4]
      \begin{center}
         \hrefVideo{https://www.yout-ube.com/watch?v=E0KvuFx0Mr8}{\bf Du kamal au GPS 1} \hfill \hrefVideo{https://www.yout-ube.com/watch?v=Jv21tvyZokk}{\bf Du kamal au GPS 2}
         
         \smallskip
         Chaîne Youtube du {\it Musée national de la marine}
      \end{center}
   \end{cadre}
\end{debat}
% \end{changemargin}