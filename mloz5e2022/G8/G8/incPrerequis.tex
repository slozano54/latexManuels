\vspace*{-7mm}
\begin{changemargin}{-10mm}{-10mm}
%pre-001
\begin{prerequis}[Connaisances \emoji{red-heart} et compétences \emoji{diamond-suit} du cycle 3]    
   \begin{itemize}        
       \item[\emoji{red-heart}] Vocabulaire associé à ces objets et à leurs propriétés : côté, sommet, angle, hauteur.
       \columnbreak
       \item[\emoji{diamond-suit}] Reconnaître, nommer, décrire des triangles, dont les triangles particuliers (triangle rectangle, triangle isocèle, triangle équilatéral).       
   \end{itemize}
\end{prerequis}
\end{changemargin}
\vspace*{-13mm}
\begin{debat}[La perspective]
   La {\bf perspective cavalière} est un outil qui permet de représenter sur une feuille de papier des objets en volume sans point de fuite.
   Cette représentation était utilisée pour la conception des fortifications militaires. Le \og cavalier \fg{} était un promontoire de terre situé en arrière des fortifications et qui permettait de voir par-dessus la ligne des ouvrages de défense, et donc de voir les ouvrages des assaillants et ainsi d'anticiper leurs plans offensifs. D'autres perspectives sont utilisées notamment pour les arts : le perspective par {\bf point de fuite} et la {\bf perspective isométrique} par exemple.
   \begin{center} 
      {\psset{Decran=20,viewpoint=10 5 10,unit=0.45}
      \begin{pspicture}(-5,-4.5)(5,5)
         \psSolid[fcol=0 (red) 1 (Aquamarine) 2 (Bittersweet) 3 (ForestGreen) 4 (Goldenrod) 13 (GreenYellow) 40 (Mulberry), object=cube,mode=3]
      \end{pspicture}
      \begin{pspicture}(-5,-4)(5,5)
         \psSolid[fcol=0 (gray) 2 (Lavender) 3 (SkyBlue) 11 (LimeGreen) 12 (Brown) 23 (OliveGreen) 22 (Yellow) , object=cylindre,h=4,ngrid=4 10](0,0,-2)
      \end{pspicture}}  
   \end{center}   
    \bigskip
    \begin{cadre}[B2][J4]
       \begin{center}
          \hrefVideo{https://www.yout-ube.com/watch?v=zCIxdOCQiZg}{\bf Dessiner des illusions d'optique 3D}, {\it Simple drawing tutorial}.
       \end{center}
    \end{cadre}
\end{debat}