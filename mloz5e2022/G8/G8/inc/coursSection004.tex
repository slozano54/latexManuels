\section{Volumes}
    \begin{propriete}[Volume du prisme]
        \begin{minipage}{0.2\linewidth}
            \begin{Geometrie}[CoinBG={u*(-10,-10)},TypeTrace="Espace"]
                Initialisation(1500,30,30,100);
                color A,B,C,D,E,F;
                D=(0.75,0,0);
                F=(0.75,1/2,0);
                E=(0,1/2,0);
                A-D=(0,0,0.7);
                C-F=A-D;
                B-E=A-D;
                NbS:=6;
                Sommet1:=A;
                Sommet2:=B;
                Sommet3:=C;
                Sommet4:=D;
                Sommet5:=E;
                Sommet6:=F;
                NF:=5;
                Fc[100]:=3;Fc[101]:=1;Fc[102]:=3;Fc[103]:=2;
                Fc[200]:=3;Fc[201]:=4;Fc[202]:=5;Fc[203]:=6;
                Fc[300]:=4;Fc[301]:=1;Fc[302]:=2;Fc[303]:=5;Fc[304]:=4;
                Fc[400]:=4;Fc[401]:=1;Fc[402]:=4;Fc[403]:=6;Fc[404]:=3;
                Fc[500]:=4;Fc[501]:=3;Fc[502]:=6;Fc[503]:=5;Fc[504]:=2;
                DessineObjet;
                label.lft(TEX("A"),Projette(A));
                label.top(TEX("B"),Projette(B));
                label.lrt(TEX("C"),Projette(C));
                label.rt(TEX("E"),Projette(E));
                label.bot(TEX("F"),Projette(F));
                label.llft(TEX("D"),Projette(D));
                trace appelation(D,A,3mm,TEX("hauteur"));
            \end{Geometrie}
        \end{minipage}
        \hfill
        \begin{minipage}{0.75\linewidth}
            $$V_{prisme}=\text{Aire du polygone de base}\times \text{hauteur}$$
        \end{minipage}
    \end{propriete}
    \begin{propriete}[Volume du cylindre]
        \begin{minipage}{0.2\linewidth}
            \begin{Geometrie}[CoinBG={u*(-10,-10)},TypeTrace="Espace"]
                Initialisation(1500,0,20,100);
                color O,O',A,A',B,B',C,C';
                O=(0,0,0);
                O'-O=(0,0,1);
                A-O=(0,1/2,0);
                A'-A=O'-O;
                C=symetrie(A,O);
                C'-C=O'-O;
                B-O=(-1/2,0,0);
                B'-B=O'-O;
                path cc,cd;
                cc=cercles(O,A,O,A,B);
                cd=cercles(O',A',O',A',B');
                trace cd;
                trace segment(C,C');
                trace segment(A,A');
                trace (subpath(0,length cc/2) of cc) dashed evenly;
                trace subpath(length cc/2,length cc) of cc;
                trace segment(O,A) dashed evenly;
                trace segment(O',A') ;
                trace segment(O,O') dashed evenly;
                trace codeperp(A,O,O',5);
                trace codeperp(A',O',O,5);
                trace cotationmil(C,C',3mm,20,TEX("hauteur"));
                trace cotationmil(O,O',3mm,20,TEX("hauteur"));
                trace cotation(Projette(O),Projette(A),-8mm,-3mm,TEX("rayon"));
                marque_p:="croix";
                pointe(O);
                pointe(O');
                marque_p:="non";
            \end{Geometrie}
        \end{minipage}
        \hfill
        \begin{minipage}{0.75\linewidth}
            $$V_{prisme}=\text{Aire du disque de base}\times \text{hauteur}$$
        \end{minipage}
    \end{propriete}
    \begin{remarque}
        Dans les deux cas le volume est égale à l'aire de la base multipliée par la hauteur.
    \end{remarque}
