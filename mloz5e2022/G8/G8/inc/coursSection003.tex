\section{Les patrons}
\vspace*{-10mm}
    \begin{methode*1}[Patron ( ou développement ) du prisme droit]
        \exercice
        Construire le patron d'un prisme à base triangulaire.
        \vspace*{-5mm}
        \begin{multicols}{3}
            \begin{itemize}
                \item \textbf{Étape 1}\\
                Construire une des bases puis une des faces latérales rectangulaires.
                \columnbreak
                \item \textbf{Étape 2}\\
                Construire la seconde base, symétrique de la première.
                \columnbreak
                \item \textbf{Étape 3}\\
                Construire enfin les deux faces latérales restantes, ce sont des rectangles.
            \end{itemize}
        \end{multicols}
        \vspace*{-7mm}
        \correction
        Illustration étape par étape
        \begin{multicols}{3}
            \begin{itemize}
                \item \textbf{Étape 1}\\                
                \begin{Geometrie}
                    pair A,B,C,D,E;
                    A=u*(1,3);
                    B-A=u*(1,-1.5);
                    E-A=u*(1.5,1);
                    C=0.5[B,rotation(E,B,-90)];
                    D=0.5[E,rotation(B,E,90)];
                    % étape1
                    trace polygone(A,B,C,D,E);
                    trace segment(B,E);
                    trace appelation(A,E,2mm,TEX("\tiny\Lg[cm]{4}"));
                    trace appelation(B,A,2mm,TEX("\tiny\Lg[cm]{5}"));
                    trace appelation(B,E,2mm,TEX("\tiny\Lg[cm]{6}"));
                    trace appelation(E,D,2mm,TEX("\tiny\Lg[cm]{3}"));
                    marque_s:=0.3*marque_s;
                    trace Codelongueur(A,E,3);
                    trace Codelongueur(A,B,2);
                    trace Codelongueur(E,B,4);
                \end{Geometrie}
                \columnbreak
                \item \textbf{Étape 2}\\
                \begin{Geometrie}
                    pair A,B,C,D,E;
                    A=u*(1,3);
                    B-A=u*(1,-1.5);
                    E-A=u*(1.5,1);
                    C=0.5[B,rotation(E,B,-90)];
                    D=0.5[E,rotation(B,E,90)];
                    % étape1
                    trace polygone(A,B,C,D,E);
                    trace segment(B,E);
                    trace appelation(A,E,2mm,TEX("\tiny\Lg[cm]{4}"));
                    trace appelation(B,A,2mm,TEX("\tiny\Lg[cm]{5}"));
                    trace appelation(B,E,2mm,TEX("\tiny\Lg[cm]{6}"));
                    trace appelation(E,D,2mm,TEX("\tiny\Lg[cm]{3}"));
                    marque_s:=0.3*marque_s;
                    trace Codelongueur(A,E,3);
                    trace Codelongueur(A,B,2);
                    trace Codelongueur(E,B,4);
                    % étape2
                    pair A',axe[];
                    axe0=iso(E,D);
                    axe1=iso(B,C);
                    A'=symetrie(A,axe0,axe1);
                    trace chemin(D,A',C);
                    draw axe1--axe0 + 0.5(axe0-axe1) dashed dashpattern(on6bp off3bp on1.5bp off3bp) withcolor red;
                    draw axe0--axe1 + 0.5(axe1-axe0) dashed dashpattern(on6bp off3bp on1.5bp off3bp) withcolor red;
                    trace Codelongueur(A',D,3);
                    trace Codelongueur(A',C,2);
                    trace Codelongueur(C,D,4);
                \end{Geometrie}
                \columnbreak
                \item \textbf{Étape 3}
                
                \vspace*{-15mm}
                \begin{Geometrie}
                    pair A,B,C,D,E;
                    A=u*(1,5);
                    B-A=u*(1,-1.5);
                    E-A=u*(1.5,1);
                    C=0.5[B,rotation(E,B,-90)];
                    D=0.5[E,rotation(B,E,90)];
                    % étape1
                    trace polygone(A,B,C,D,E);
                    trace segment(B,E);
                    trace appelation(A,E,2mm,TEX("\tiny\Lg[cm]{4}"));
                    trace appelation(B,A,2mm,TEX("\tiny\Lg[cm]{5}"));
                    trace appelation(B,E,2mm,TEX("\tiny\Lg[cm]{6}"));
                    trace appelation(E,D,2mm,TEX("\tiny\Lg[cm]{3}"));
                    marque_s:=0.3*marque_s;
                    trace Codelongueur(A,E,3);
                    trace Codelongueur(A,B,2);
                    trace Codelongueur(E,B,4);
                    % étape2
                    pair A',axe[];
                    axe0=iso(E,D);
                    axe1=iso(B,C);
                    A'=symetrie(A,axe0,axe1);
                    trace chemin(D,A',C);
                    % draw axe1--axe0 + 0.5(axe0-axe1) dashed dashpattern(on6bp off3bp on1.5bp off3bp);
                    % draw axe0--axe1 + 0.5(axe1-axe0) dashed dashpattern(on6bp off3bp on1.5bp off3bp);
                    trace Codelongueur(A',D,3);
                    trace Codelongueur(A',C,2);
                    trace Codelongueur(C,D,4);
                    %étape3
                    pair F[];
                    F0=droite(B,E) intersectionpoint cercles(E,u*sqrt(3.25));
                    F1=droite(C,D) intersectionpoint cercles(D,u*sqrt(3.25));
                    trace chemin(E,F0,F1,D);
                    trace Codelongueur(E,F0,F1,D,3);
                    F2=droite(B,E) intersectionpoint subpath(4,6) of cercles(B,u*sqrt(3.25));
                    F3=droite(C,D) intersectionpoint subpath(4,6) of cercles(C,u*sqrt(3.25));
                    trace chemin(B,F2,F3,C);
                    trace Codelongueur(B,F2,C,F3,2);
                    trace Codelongueur(F0,F1,E,D,B,C,F2,F3,1);
                \end{Geometrie}
            \end{itemize}
        \end{multicols}
        \vspace*{-15mm}
    \end{methode*1}
    \begin{methode*1}[Patron ( ou développement ) du cylindre]
        \exercice
        Construire le patron d'un prisme à base triangulaire.
        \vspace*{-5mm}
        \begin{multicols}{3}
            \begin{itemize}
                \item \textbf{Étape 1}\\
                Construire une des bases qui est un disque. Avec un rayon de \Lg[cm]{3}, le périmètre se calcule ainsi : $2\pi \times \Lg[cm]{3}$ soit environ \Lg[cm]{18,84}.                \columnbreak
                \item \textbf{Étape 2}\\
                Tracer la surface latérale qui est un rectangle dont l'un des côté est la hauteur du cylindre et l'autre côté est le périmètre du disque de base.
                \columnbreak
                \item \textbf{Étape 3}\\
                Compléter en traçant la seconde base qui est un disque superposable au premier. 
            \end{itemize}
        \end{multicols}
        \vspace*{-7mm}
        \correction
        Illustration étape par étape
        \begin{multicols}{3}
            \begin{itemize}
                \item \textbf{Étape 1}
                
                \smallskip                
                \begin{Geometrie}
                    pair O,A[];
                    O=u*(3,3);
                    numeric rayon;
                    rayon:=0.8;
                    path co;
                    co=cercles(O,u*rayon);
                    % étape1
                    trace co;
                    A0=pointarc(co,135);                    
                    trace cotation(A0,O,0mm,2mm,TEX("\tiny\Lg[cm]{3}"));
                    % etape2
                    A1=pointarc(co,0);
                    A2=A1 shifted (0,-2*rayon);
                    A3=A1 shifted (0,(2*pi-2)*rayon);
                    A4-A3=(2u,0);
                    A5-A2=(2u,0);
                    trace polygone(A2,A3,A4,A5) withcolor white;
                    %algo du peintre
                    trace co;
                \end{Geometrie}
                \columnbreak
                \item \textbf{Étape 2}
                
                \medskip
                \hspace*{-15mm}
                \begin{Geometrie}
                    pair O,A[];
                    O=u*(1,3);
                    numeric rayon;
                    rayon:=0.8;
                    path co;
                    co=cercles(O,u*rayon);
                    % étape1
                    trace co;
                    A0=pointarc(co,135);
                    trace cotation(A0,O,0mm,2mm,TEX("\tiny\Lg[cm]{3}"));                    
                    % etape2
                    A1=pointarc(co,0);
                    A2=A1 shifted (0,-2*rayon);
                    A3=A1 shifted (0,(2*pi-2)*rayon);
                    A4-A3=(2u,0);
                    A5-A2=(2u,0);
                    trace polygone(A2,A3,A4,A5);
                    trace cotation(A2,A5,-3mm,-2mm,TEX("\tiny hauteur"));
                    trace cotation(A5,A4,-3mm,-2mm,TEX("\tiny\Lg[cm]{18.84}"));
                \end{Geometrie}
                \columnbreak
                \item \textbf{Étape 3}
                
                \medskip
                \hspace*{-15mm}
                \begin{Geometrie}
                    pair O,A[];
                    O=u*(1,3);
                    numeric rayon;
                    rayon:=0.8;
                    path co;
                    co=cercles(O,u*rayon);
                    % étape1
                    trace co;
                    A0=pointarc(co,135);
                    trace cotation(A0,O,0mm,2mm,TEX("\tiny\Lg[cm]{3}"));                    
                    % etape2
                    A1=pointarc(co,0);
                    A2=A1 shifted (0,-2*rayon);
                    A3=A1 shifted (0,(2*pi-2)*rayon);
                    A4-A3=(2u,0);
                    A5-A2=(2u,0);
                    trace polygone(A2,A3,A4,A5);
                    trace cotation(A2,A5,-3mm,-2mm,TEX("\tiny hauteur"));
                    trace cotation(A5,A4,-23mm,-2mm,TEX("\tiny\Lg[cm]{18.84}"));
                    %etape3
                    A6=iso(A4,A5) shifted (rayon,rayon);
                    path cAsix;
                    cAsix=cercles(A6,rayon);
                    trace cAsix;
                    A7=pointarc(cAsix,45);
                    trace cotation(A6,A7,0mm,2mm,TEX("\tiny\Lg[cm]{3}"));
                \end{Geometrie}
            \end{itemize}
        \end{multicols}
        \vspace*{-10mm}
    \end{methode*1}
 
