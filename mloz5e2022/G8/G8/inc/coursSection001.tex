\section{Rappels}
    \begin{propriete}[Représentation en perspective cavalière \admise]
        Lorsque l'on dessine un objet en perspective :
        \begin{itemize}
            \item le parallélisme est conservé;
            \item les égalités de longueurs situées sur une même droite ou sur deux droites parallèles sont respectées.
        \end{itemize}
    \end{propriete}
    {\renewcommand{\StringDEFINITION}{Tu dois savoir}
        \begin{definition}
            Pour le \textbf{pavé droit} (ou parallélépipède rectangle) et le \textbf{cube} :
            \begin{itemize}
                \item faire une représentation en perspective,
                \item faire un patron,
                \item le construire,
                \item calculer son volume.
            \end{itemize}
        \end{definition}
    }
