\opset{voperator=bottom,decimalsepsymbol={,},strikedecimalsepsymbol=\rlap{,}\rule[-1pt]{3pt}{0.4pt}}

\section{Division Euclidienne}
\begin{definition}[Vocabulaire]
\begin{itemize}
\item Un \colorbox{red!30}{\textbf{quotient}} c'est le résultat d'une division.
\item Un \colorbox{green!30}{\textbf{diviseur}} c'est un nombre par lequel on divise.
\item Un \colorbox{blue!30}{\textbf{dividende}} c'est un nombre que l'on divise.
\item Un \colorbox{orange}{\textbf{reste}} c'est ce qu'il reste après partage !
\item Une \textbf{division euclidienne} c'est une division où le quotient, le diviseur, le dividende et le reste sont des nombres entiers.
\end{itemize}
\end{definition}

\begin{propriete}[\admise]
	Si dans une division euclidienne, on multiplie le diviseur par le quotient et qu'on ajoute le reste
	alors on retrouve le dividende
\end{propriete}
\vfill
%la macro  \OPoval est dans commandes.tex
\begin{exemple*1}
	\phantom{rrr}

	\begin{minipage}{0.5\linewidth}
		\begin{itemize}
			\item Partage de 84\euro{} entre 3 personnes.
			\item Chacun aura \colorbox{red!30}{28}\euro{} et il ne reste \colorbox{orange}{rien}.
			\item $\colorbox{green!30}{3}\times \colorbox{red!30}{28}+ \colorbox{orange}{0}=\colorbox{blue!30}{84}$
		\end{itemize}
	\end{minipage}
	\begin{minipage}{0.5\linewidth}
		\begin{center}
			\opidiv[displayintermediary=all,
			operandstyle.1=\color{blue} ,
			operandstyle.1.1=\blue\OPoval{A}{0.5},
			operandstyle.2=\color{green} ,
			operandstyle.2.1=\color{green}\OPoval{B}{0.5},
			resultstyle=\red ,
			resultstyle.1=\red\OPoval{C}{0.8},
			remainderstyle.2=\colorbox{orange},
			remainderstyle.2=\colorbox{orange},
			remainderstyle.2=\OPoval{D}{0.8}]{84}{3}\qquad
			\begin{minipage}[b]{2cm}
				\pnode(0,0.2em){F}{\colorbox{green!30}{\textbf{diviseur}}}
				\ncarc{<-}{F}{B}\par
				\pnode(0,0.2em){E}{\colorbox{blue!30}{\textbf{dividende}}}
				\ncarc{->}{A}{E}\par
				\pnode(0,0.2em){G}{\colorbox{red!30}{\textbf{quotient}}}
				\ncarc{<-}{G}{C}
				\pnode(0,0.2em){H}{\colorbox{orange}{\textbf{reste}}}
				\ncarc{<-}{H}{D}
			\end{minipage}
		\end{center}
	\end{minipage}	
\end{exemple*1}
\vfill
\begin{exemple*1}
	\phantom{rrr}
	
	\begin{minipage}{0.5\linewidth}
		\begin{itemize}
			\item Partage de 85\euro{} entre 3 personnes.
			\item Chacun aura \colorbox{red!30}{28}\euro{} et il reste \colorbox{orange}{1}\euro{}.
			\item $\colorbox{green!30}{3}\times \colorbox{red!30}{28}+ \colorbox{orange}{1}=\colorbox{blue!30}{85}$
		\end{itemize}
	\end{minipage}
	\begin{minipage}{0.5\linewidth}
		\begin{center}
			\opidiv[displayintermediary=all,
			operandstyle.1=\blue,
			operandstyle.1.1=\blue\OPoval{A}{0.5},
			operandstyle.2=\green,
			operandstyle.2.1=\green\OPoval{B}{0.5},
			resultstyle=\red,
			resultstyle.1=\red\OPoval{C}{0.8},
			remainderstyle.2=\colorbox{orange},
			remainderstyle.2=\colorbox{orange},
			remainderstyle.2=\OPoval{D}{0.8}]{85}{3}\qquad
			\begin{minipage}[b]{2cm}
				\pnode(0,0.2em){F}{\colorbox{green!30}{\textbf{diviseur}}}
				\ncarc{<-}{F}{B}	\par
				\pnode(0,0.2em){E}{\colorbox{blue!30}{\textbf{dividende}}}
				\ncarc{->}{A}{E}\par
				\pnode(0,0.2em){G}{\colorbox{red!30}{\textbf{quotient}}}
				\ncarc{<-}{G}{C}
				\pnode(0,0.2em){H}{\colorbox{orange}{\textbf{reste}}}
				\ncarc{<-}{H}{D}
			\end{minipage}
		\end{center}
	\end{minipage}
\end{exemple*1}
\vfill
\clearpage
\vspace*{-15mm}
\begin{exemple*1}
	\begin{multicols}2
	La division de $\num{42332}$ par $7$.

	\medskip
	\begin{center}
		\opidiv[displayintermediary=all]{42332}{7}
	\end{center}
	\columnbreak
	La division de $\num{52659}$ par $13$.

	\medskip
	\begin{center}
		\opidiv[displayintermediary=all]{52659}{13}.
	\end{center}
\end{multicols}
\end{exemple*1}

\begin{propriete}[\admise]
La \textbf{division euclidienne} d'un nombre entier $\colorbox{blue!30}{a}$ par un nombre entier $\colorbox{green!30}{b}$ non nul 
permet d'obtenir le couple $(\colorbox{red!30}{q};\colorbox{orange}{r})$ de nombres entiers tels que
$$\colorbox{blue!30}{a}=\colorbox{green!30}{b}\times \colorbox{red!30}{q}+ \colorbox{orange}{r}$$
\end{propriete}

\begin{exemple*1}
$\colorbox{blue!30}{417}=\colorbox{green!30}{19}\times \colorbox{red!30}{21}+ \colorbox{orange}{18}$
\end{exemple*1}

\begin{definition}
On considère deux nombres entiers $a$ et $b$, $b$ ne valant pas zéro\\
On dit que :
\begin{itemize}
\item $b$ est un \textbf{diviseur} de $a$.
\item $b$ \textbf{divise} $a$.
\item ou que $a$ est \textbf{divisible} par $b$.
\item ou encore que $a$ est un \textbf{multiple} de $b$.
\end{itemize}
lorsque le reste de la division euclidienne de $a$ par $b$ vaut $0$.
\end{definition}

\begin{exemple}
Dire si $325$ est un multiple de $25$.
\correction 
$325=25\times13$ donc $325$ est un multiple de $5$
\end{exemple}
\begin{remarque}
	$325$ est aussi un multiple de $13$.
\end{remarque}

\begin{exemple}
	Dire si $399$ est divisible par $19$.
	\correction 
	$399=19\times21$ donc $399$ est divisible par $19$.
\end{exemple}
\begin{remarque}
	$19$ et $21$ sont des diviseurs de $399$.
\end{remarque}

\begin{exemple*1}
	Avec une division posée

	\begin{multicols}2
		\opidiv[displayintermediary=all]{84}{3}
		\columnbreak
		\begin{itemize}
			\item 84 est un \textbf{multiple} de 3.
			\item 84 est \textbf{divisible} par 3.
			\item 3 est un \textbf{diviseur} de 84.
		\end{itemize}
	\end{multicols}

	\begin{multicols}2
		\opidiv[displayintermediary=all]{27}{4}
		\columnbreak
		\begin{itemize}
			\item 4 n'est pas un \textbf{diviseur} de 27.
			\item 27 n'est pas un \textbf{multiple} de 4.
			\item 27 n'est pas \textbf{divisible} par 4.
		\end{itemize}
	\end{multicols}	
\end{exemple*1}