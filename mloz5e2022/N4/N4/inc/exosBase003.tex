\begin{exercice*}
   Résoudre les problèmes suivants :
   \begin{enumerate}
      \item \num{6 798} supporters d'un club de rugby doivent faire un déplacement en car pour soutenir leur équipe. Chaque car dispose de 55 places. \\
        Combien de cars faut-il réserver ?
      \item Des stylos sont conditionnés par boîte de 40. Ashiderdene a \num{2 647} stylos. \\
         Combien lui en manque-t-il pour avoir des boîtes entièrement remplies ?
      \item Trois amis participent à une chasse au trésor et trouvent \num{1 419} pièces en chocolat. \\
      Si le partage est équitable, combien de pièces en chocolat auront-ils chacun ? \\
      Otmane arrive et leur rappelle que c'est lui qui leur a prêté sa boussole. Il exige donc d'avoir la même part que chacun des trois autres plus les pièces restantes. \\
      Combien de pièces recevra-t-il ?
   \end{enumerate}
\end{exercice*}

\begin{corrige}
   \begin{enumerate}
      \item On effectue la division : {\small \opidiv[displayintermediary=all,voperation=top]{6798}{55}} \\
         En prenant 123 cars, il restera 33 personnes, il faudra donc réserver {\red 124 cars}.
      \item On effectue la division : {\small \opidiv[displayintermediary=all,voperation=top]{2647}{40}} \\
         Il lui reste 7 stylos pour une boite de 40, il faut donc ajouter {\red 33 stylos} pour compléter la boite.
   \end{enumerate}
   \Coupe
   \begin{enumerate}
    \setcounter{enumi}{2}
      \item On effectue la division : {\small \opidiv[displayintermediary=all,voperation=top]{1419}{3}} \\
         Chaque ami aura donc {\red 473 pièces en chocolat} et il n'en restera	pas.
   \end{enumerate}
   Lorsque Otmane arrive, on fait {\small \opidiv[displayintermediary=all,voperation=top]{1419}{4}} \\
   Il recevra $(354+3)$ pièces en chocolat, soit {\red 357}. \\
\end{corrige}