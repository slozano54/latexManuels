\vspace*{-7mm}
\begin{changemargin}{-10mm}{-10mm}
%pre-001
\begin{prerequis}[Connaisances \emoji{red-heart} et compétences \emoji{diamond-suit} du cycle 3]    
   \begin{itemize}        
       \item[\emoji{red-heart}] Vocabulaire associé à ces objets et à leurs propriétés : côté, sommet, angle, hauteur.
       \columnbreak
       \item[\emoji{diamond-suit}] Reconnaître, nommer, décrire des triangles, dont les triangles particuliers (triangle rectangle, triangle isocèle, triangle équilatéral).       
   \end{itemize}
\end{prerequis}
\vspace*{-7mm}
%pre-002
\begin{prerequis}[Connaisances \emoji{red-heart} et compétences \emoji{diamond-suit} du cycle 4]    
    \begin{itemize}        
        \item[\emoji{diamond-suit}] Mener des calculs impliquant des grandeurs mesurables, exprimer les résultats dans des les unités adaptées.
        \item[\emoji{diamond-suit}] Exprimer et vérifier la cohérence des résultats du point de vue des unités.
    \end{itemize}
\end{prerequis}
\end{changemargin}
\vspace*{-7mm}
\begin{debat}[La division euclidienne]
    \begin{minipage}{0.6\linewidth} 
    \begin{changemargin}{-15mm}{-15mm}
        Le nom de {\bf division euclidienne} est un hommage rendu à {\it Euclide} (300 av. J.-C.), mathématicien grec qui en explique le principe par soustractions successives dans son \oe uvre {\it Les éléments}. Mais elle apparait très tôt dans l'histoire des mathématiques, par exemple dans les mathématiques égyptiennes, babyloniennes et chinoises.
    \end{changemargin}
    \end{minipage}
    \hspace*{10mm}
    \begin{minipage}{0.3\linewidth}     
        \begin{center}
            {\psset{unit=1}
           \begin{pspicture}(0,1.25)(4,4.25)
              \psline[linewidth=1mm](2,1)(2,4)
              \psline[linewidth=1mm](2,3)(4,3)
              \textcolor{B1}{\it\large
              \rput(0.8,3.5){dividende}
              \rput(3,3.5){diviseur}
              \rput(3,2.5){quotient}
              \rput(3,2){\small (euclidien)}
              \rput(1,1.5){reste}}
           \end{pspicture}}
        \end{center}    
    \end{minipage}
    \smallskip    
    \begin{cadre}[B2][J4]
       \begin{center}
          \hrefVideo{https://www.yout-ube.com/watch?v=VWS9NyXbEyY&t=18s}{\bf Division euclidienne avec matériel multibase}
          
          Chaîne YouTube {\it Méthode Heuristique}.
       \end{center}
    \end{cadre}
 \end{debat}