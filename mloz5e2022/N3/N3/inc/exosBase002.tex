\begin{exercice}
   Si on note $z$ l'âge en années de Rose aujourd'hui, comment peut-on noter :
   \begin{enumerate}
      \item L'âge qu'elle aura dans deux ans ?
      \item Le triple de l'âge qu'elle avait il y a quatre ans ?
      \item La moitié de l'âge qu'elle aura dans cinq ans ?
      \item Son année de naissance si on est en 2022 ?
   \end{enumerate}
\end{exercice}

\begin{corrige}
   \ \\ [-5mm]
   \begin{enumerate}
      \item $red z+2$
      \item $red 3(z-4)$
      \item $red \dfrac12(z+5) =\dfrac{z+5}{2}$ \smallskip
      \item $red 2021-z$
   \end{enumerate}
\end{corrige}