\begin{exercice}
    Si $x$ représente un nombre, comment peut-on écrire les expressions suivantes :
    \begin{enumerate}
       \item Le double de $x$.
       \item Le tiers de $x$.
       \item La somme de $x$ et de 13.
       \item La différence de $x$ et de 7.
       \item Le triple de la somme de 2 et de $x$.
       \item Le tiers de la différence de 16 et $x$.
    \end{enumerate}
    \hrefMathalea{https://coopmaths.fr/mathalea.html?ex=5L10-1,s=1,s2=false,s3=true,n=7&v=l}
 \end{exercice}
 
 \begin{corrige}
    \ \\ [-5mm]
    \begin{enumerate}
       \item $2\times x =\red2x$ \smallskip
       \item $\dfrac13\times x =\red\dfrac13x =\dfrac{x}{3}$ \smallskip
       \item $\red x+13$
       \item $\red x-7$
       \item $3\times(2+x) =\red3(2+x)$ \smallskip
       \item $\dfrac13\times(16-x) =\red\dfrac13(16-x) =\dfrac{16-x}{3}$
    \end{enumerate}
 \end{corrige}

