\begin{activite}[Des variables aux inconnues]
   \vspace*{-7mm}
    {\bf Objectifs :} voir l'effet des variables dans un programme de calcul créé avec Scratch ; produire une expression littérale.
    On considère le programme de calcul ci-dessous dans lequel $x$, Etape 1, Etape 2 et Résultat sont quatre variables.
    \begin{center}
       \begin{Scratch}[Echelle=0.75]
          Place Drapeau ;
          Place Demander("Choisis un nombre");
          Place MettreVar("x",OvalCap("réponse"));
          Place DireT("Je multiplie le nombre par 6.","2");
          Place MettreVar("Etape 1",OpMul("6",OvalVar("x")));
          Place DireT("J'ajoute 10 au résultat.","2");
          Place MettreVar("Etape 2",OpAdd(OvalVar("Etape 1"),"10"));
          Place DireT("Je divise le résultat par 2.","2");
          Place MettreVar("Résultat",OpDiv(OvalVar("Etape 2"),"2"));
          Place Dire(OpRegrouper("J'obtiens finalement",OvalVar("Résultat")));
       \end{Scratch}
    \end{center}
    \begin{enumerate}
    {\psset{unit=0.9}
       \item Julien a fait fonctionner ce programme en choisissant le nombre 5.\\
       Vérifier que ce qui est dit à la fin est : \og J’obtiens finalement 20 \fg.
       
       Pour cela, remplir le diagramme suivant : \\
          \begin{pspicture}(-4,-0.2)(8,1.8)
             \psframe(0,0.5)(1,1.5)
             \rput(0.5,0){$x$}
             \psline[arrowsize=2mm]{->}(1.1,1)(2.4,1)
             \psframe(2.5,0.5)(3.5,1.5)
             \rput(3,0){étape 1}
             \psline[arrowsize=2mm]{->}(3.6,1)(4.9,1)
             \psframe(5,0.5)(6,1.5)
             \rput(5.5,0){étape 2}
             \psline[arrowsize=2mm]{->}(6.1,1)(7.4,1)
             \psframe(7.5,0.5)(8.5,1.5)
             \rput(8,0){résultat}
          \end{pspicture}         
       \item Que dit le programme si Zakarie le fait fonctionner en choisissant au départ le nombre 7 ? \\
           \begin{pspicture}(-4,-0.2)(8,1.8)
             \psframe(0,0.5)(1,1.5)
             \rput(0.5,0){$x$}
             \psline[arrowsize=2mm]{->}(1.1,1)(2.4,1)
             \psframe(2.5,0.5)(3.5,1.5)
             \rput(3,0){étape 1}
             \psline[arrowsize=2mm]{->}(3.6,1)(4.9,1)
             \psframe(5,0.5)(6,1.5)
             \rput(5.5,0){étape 2}
             \psline[arrowsize=2mm]{->}(6.1,1)(7.4,1)
             \psframe(7.5,0.5)(8.5,1.5)
             \rput(8,0){résultat}
          \end{pspicture}         
       \item Titouan fait fonctionner le programme, ce qui est dit à la fin est : \og J’obtiens finalement 8 \fg.
       
       Quel nombre a-t-il choisi au départ ? \\
       \begin{pspicture}(-4,-0.2)(8,1.8)
             \psframe(0,0.5)(1,1.5)
             \rput(0.5,0){$x$}
             \psline[arrowsize=2mm]{<-}(1.1,1)(2.4,1)
             \psframe(2.5,0.5)(3.5,1.5)
             \rput(3,0){étape 1}
             \psline[arrowsize=2mm]{<-}(3.6,1)(4.9,1)
             \psframe(5,0.5)(6,1.5)
             \rput(5.5,0){étape 2}
             \psline[arrowsize=2mm]{<-}(6.1,1)(7.4,1)
             \psframe(7.5,0.5)(8.5,1.5)
             \rput(8,0){résultat}
          \end{pspicture} 
       \item Si l’on appelle $x$ le nombre choisi au départ, écrire en fonction de $x$ l’expression obtenue à la fin du programme. 
       \par \smallskip \dotfill
      }
    \end{enumerate}
    \vfill\hfill{\footnotesize\it Cette activité est une adaptation du DNB 2017 de Pondichery.}
 \end{activite} 