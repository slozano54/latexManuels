\begin{exercice}
    Sahya construit un carré $ABCD$ de \ucm{5} de côté.
    \begin{enumerate}
       \item Calculer le périmètre $\mathcal{P}_1$ et l'aire $\mathcal{A}_1$ de $ABCD$.
       \item On augmente ses côtés de $k$ cm. Exprimer, en fonction de $k$ :
       \begin{itemize}
          \item la longueur $L$ du nouveau côté ;
          \item le nouveau périmètre $\mathcal{P}_2$ de ce carré ;
          \item la nouvelle aire $\mathcal{A}_2$ de ce carré.
       \end{itemize}
       \item Grâce aux expressions trouvées en 2, donner le périmètre et l'aire si on augmente le côté de 2 cm.
    \end{enumerate}
    {\psset{unit=1.2}
    \begin{pspicture}(-2,-1)(3,3.5)
       \pstGeonode[PosAngle={-135,-45,45,135},PointSymbol=none](0,0){A}(2,0){B}(2,2){C}(0,2){D}
       \psframe(0,0)(2,2)
       \psframe(0,0)(0.2,0.2)
       \psframe(0,0)(3,3)
       \psline{<->}(0,-0.5)(2,-0.5)
       \rput(1,-0.75){\small \ucm{5}}
       \psline{<->}(2,-0.5)(3,-0.5)
       \rput(2.5,-0.75){\small k}
    \end{pspicture}}
 \end{exercice}
 
 \begin{corrige}
    \ \\ [-5mm]
    \begin{enumerate}
       \item $\mathcal{P}_1 =4\times\ucm{5} =\red\ucm{20}$ ; $\mathcal{A}_1 =(\ucm{5})^2 =\red\ucmq{25}$
       \item Les longueurs sont en \ucm{} et les aires en \ucmq{}.
       \begin{itemize}
          \item La longueur $L$ du nouveau côté est $\red5+k$.
          \item Le nouveau périmètre vaut $\mathcal{P}_2 =\red4(5+k)$
          \item La nouvelle aire vaut $\mathcal{A}_2 =\red(5+k)^2$
       \end{itemize}
       \item $\mathcal{P}_2 =4(5+2) \, \ucm{} = 4\times7 \, \ucm{} =\red\ucm{28}$ et \\
       \quad\, $\mathcal{A}_2 =((5+2)\,\ucm{})^2 =(\ucm{7})^2 =\red\ucmq{49}$
    \end{enumerate}
 \end{corrige}