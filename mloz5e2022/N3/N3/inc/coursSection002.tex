\section{Produire une expression littérale}

On utilise notamment des expressions littérales pour produire des  formules générales, qui pourront être utilisées quelle que soit la valeur des variables.

\begin{exemple}[0.55]
\ \\ [-10mm]
   \begin{itemize}
      \item Matéo a quatre ans de plus que Noé. \\
         Exprimer l'âge de Matéo par rapport à celui de Noé.
      \item Un rectangle a pour largeur $\ell$ et pour longueur $L$. \\
         Donner l'expression de son aire et de son périmètre.
   \end{itemize}
   \correction
      \ \\ [-10mm]
      \begin{itemize}
         \item En appelant \og $\red x$ \fg{} l'âge de Noé, l'âge de Matéo peut s'écrire \og ${\red x}+4$ \fg.
         \item $\mathcal{A} =\ell\times L$. \\
         $\mathcal{P} =2\times(\ell+L)$.
      \end{itemize}
\end{exemple}

\medskip

\begin{remarque}
   la circonférence d'un disque s'écrit $2\pi R$ (\og deux pierres !!! \fg) où :
   \begin{itemize}
      \item $\pi$ est une lettre grecque qui désigne le nombre 3,14\dots, qui est fixe ;
      \item $R$ est une lettre qui désigne le rayon du disque, ce nombre est variable et dépend du disque.
   \end{itemize}
\end{remarque}
