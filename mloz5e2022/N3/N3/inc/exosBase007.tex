\begin{exercice*}
   Voici un programme :
   \myProgCalcul{$\leadsto$}{Programme de calcul}{%
      \ProgCalcul[Enonce,ThemePerso]{%                
         Choisis un nombre,
         Retire-lui 5,
         Multiplie le résultat par 3,
      }
   } 
   \begin{enumerate}
      \item Faire fonctionner le programme avec trois nombres de son choix supérieurs ou égaux à 5.
      \item Quel nombre faut-il choisir pour obtenir 6 ?
      \item Soit $x$ le nombre de départ, donner l'expression finale en fonction de $x$.
   \end{enumerate}
   \hrefMathalea{https://coopmaths.fr/mathalea.html?ex=5L10-2,s=true,n=2,i=0&v=l}
\end{exercice*}

\begin{corrige}
   \ \\ [-5mm]
   \begin{enumerate}
      \item \ProgCalcul{10,-5 *3}. Avec 10, on obtient {\red 15} \dots
      \item \ProgCalcul[Direct=false]{7,-5 *3}. On obtient 6 avec {\red 7}.
      \item \ProgCalcul[SansCalcul]{x,-5 *3,x-5 3(x-5)}.
   \end{enumerate}
\end{corrige}