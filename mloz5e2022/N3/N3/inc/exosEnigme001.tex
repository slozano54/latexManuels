% Les enigmes ne sont pas numérotées par défaut donc il faut ajouter manuellement la numérotation
% si on veut mettre plusieurs enigmes
\refstepcounter{exercice}
% \numeroteEnigme
\def\croix{\psframe(0,-1)(1,2) \psframe(-1,-0)(2,1)}
\def\carre{\psframe(0,0)(1,1)}

\begin{enigme}[Défis !!!]

\vfill

\parbox{1.75cm}{\includegraphics[width=2cm]{\currentpath/images/defi.pdf} \\ [2mm] \textcolor{orange}{\bf\large \, Défi 1}}
\qquad
\parbox{14.45cm}{
   Rayan et Aya ont choisi un nombre (entier positif). \\
   Rayan le multiplie par 5 et ajoute 35. \\
   Aya le multiplie par 2 et ajoute 146. \\
   Ils trouvent le même nombre à la fin. \\
   Quel nombre ont-ils choisi ?}

\vfill

\parbox{1.75cm}{\includegraphics[width=2cm]{\currentpath/images/defi.pdf} \\ [2mm] \textcolor{orange}{\bf\large \, Défi 2}}
\qquad
\parbox{14cm}{
   Avec des petits carrés tous identiques, on construit un pattern selon le modèle évolutif ci-dessous : \\
   {\psset{unit=0.8}
   \begin{pspicture}(-3.5,-4.5)(4,3.75)
      \croix
      \rput(0.5,-2){Rang 1}
   \end{pspicture}
   \begin{pspicture}(-5,-4.5)(5,3.75)
      \croix
      \rput(2,0){\carre}
      \rput(0,2){\carre}
      \rput(-2,0){\carre}
      \rput(0,-2){\carre}
      \rput(0.5,-3){Rang 2}
   \end{pspicture}
   
   \begin{pspicture}(-8,-4.5)(4,1)
      \croix
      \rput(2,0){\carre}
      \rput(3,0){\carre}
      \rput(0,2){\carre}
      \rput(0,3){\carre}
      \rput(-2,0){\carre}
      \rput(-3,0){\carre}
      \rput(0,-2){\carre}
      \rput(0,-3){\carre}
      \rput(0.5,-4){Rang 3}
   \end{pspicture}}
   \begin{enumerate}
      \item Dessiner l’élément du rang suivant et expliquer la règle.
      \item Déterminer le nombre de petits carrés des éléments du rang 5, du rang 10, du rang 17.
      \item Déterminer le nombre de petits carrés de l’élément du rang 100 et donner un moyen de calculer rapidement le nombre de petits carrés d’un élément à n’importe quel rang.
     \item Existe-t-il un élément qui contient 532 petits carrés ? Un élément qui contient 813 petits carrés ?
  \end{enumerate}}
   
   \vfill\hfill {\it\small Source : La résolution de problèmes mathématiques au collège, MENJS, 2021}
 \end{enigme}  

% Pour le corrigé, il faut décrémenter le compteur, sinon il est incrémenté deux fois
\addtocounter{exercice}{-1}
\begin{corrige}
   \phantom{rrr}

   \hspace*{-7.5mm} \textcolor{orange}{\bf Défi 1} \\
   On peut modéliser ce défi par un schéma en barres. Soit $x$, le nombre choisi. \\
   \ModeleBarre[Separation=2]{DeepSkyBlue 1 "$x$" 1 "$x$" 1 "$x$" 1 "$x$" 1 "$x$" PowderBlue 2 "35"}{DeepSkyBlue 1 "$x$" 1 "$x$" SkyBlue 5 "146"}
   \ModeleBarre[Separation=3]{DeepSkyBlue 1 "$x$" 1 "$x$" 1 "$x$" PowderBlue 2 "35"}{ SkyBlue 3 "111" PowderBlue 2 "35"}
   \ModeleBarre{DeepSkyBlue 1 "$x$" 1 "$x$" 1 "$x$"}{SkyBlue 3 "111"} donc, $x =111\div3 ={\red 37}$.
   
   \Coupe
   \hspace*{-7.5mm} \textcolor{orange}{\bf Défi 2} \\
   \begin{enumerate}
      \item Pour le rang 4, il suffit {\red d'ajouter 1 carré à chacune des extrémités}.
      \item Au rang 1, on a 5 carrés. Au rang 2, $5+4 =9$ carrés. Au rang 3, $9+4 =13$ carrés.  Au rang 4, $13+4 =17$ carrés. {\red Au rang 5, $17+4 =21$ carrés}. \\
         En continuant ainsi, on aura {\red 41 carrés au rang 10 et 69 carrés au rang 17}.
         \item Soit $n$ le rang demandé, {\red le nombre de carrés au rang $n$ est de $1+4\times n$}. \\
            Au rang 100, cela fait donc $1+4\times100 =401$.
         \item En enlevant 1 carré, on doit obtenir un multiple de 4. Or, $532-1 =531$ n'est pas un multiple de 4. \\
            $813-1 =812 =4\times203$ est un multiple de 4. \\
            {\red Il y a donc 813 carrés au rang 203}.     
   \end{enumerate}
\end{corrige}