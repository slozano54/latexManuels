\begin{exercice*}
    Résoudre les défis suivants.
    \begin{enumerate}
       \item
          \begin{enumerate}
             \item Choisir deux nombres entiers consécutifs, calculer leur somme et comparer avec les résultats de la classe.
             \item Quelle conjecture peut-on émettre ?
          \end{enumerate}
       \item On choisit un entier $n$.
          \begin{enumerate}
             \item Comment s'écrit le nombre suivant $n$ ?
             \item Donner l'expression de la somme de deux nombres consécutifs en fonction de $n$.
             \item Démontrer la conjecture.
          \end{enumerate}
       \item Montrer que la somme de trois nombres consécutifs est multiple de 3.
    \end{enumerate}
    \phantom{rrr}\hfill\footnotesize\it D'après Sésamath, le manuel de cycle 4. Magnard 2016
 \end{exercice*}
 
 \begin{corrige}
 \ \\ [-5mm]
    \begin{enumerate}
       \item
          \begin{enumerate}
             \item {\red $16+17 =33$}.
             \item On peut conjecturer que  {\red la somme de deux nombres consécutifs est un nombre impair}.
          \end{enumerate}
       \setcounter{enumi}{1}
       \item
          \begin{enumerate}
             \item Le nombre suivant $n$ s'écrit {\red $n+1$}.
             \item $(n)+(n+1) =n+n+1 ={\red 2n+1}$.
             \item $2n+1$ est un nombre pair ($2n$) auquel on ajoute 1, il impair. \\
                {\red La conjecture est vraie}.
          \end{enumerate}
       \setcounter{enumi}{2}
       \item Soit $n$ un nombre quelconque, le suivant s'écrit $n+1$ et celui d'après $n+2$. \\
          La somme de trois nombres consécutifs s'écrit alors : \\
          $(n)+(n+1)+(n+2) =n+n+1+n+2 =3n+3$. \\
          $3n+3$ est la somme de deux nombres multiples de 3, {\red c'est donc un multiple de 3}.
    \end{enumerate}
 \end{corrige}   
 