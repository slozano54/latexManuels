\section{Simplifier une expression littérale}

\begin{convention}[Multiplication implicite]
    Pour simplifier une expression littérale, on peut supprimer le symbole de multiplication \og $\times$ \fg :
    \begin{itemize}
        \item Devant une lettre.
        \item Devant une parenthèse.
    \end{itemize}
\end{convention}
 
\begin{convention}
   \begin{itemize}
      \item On utilise la notation $2a$ pour $a+a$ ou $a\times2$ ou encore $2\times a$. On dit \og deux a \fg.
      \item On utilise la notation $ab$ pour $a\times b$. On dit \og ab \fg.
      \item On utilise la notation $a^2$ pour $a\times a$. On dit \og a au carré \fg.
      \item On utilise la notation $a^3$ pour $a\times a\times a$. On dit \og a au cube \fg. \\ [-8mm]
   \end{itemize}
\end{convention}

\begin{remarques}
    \begin{itemize}
        \item On écrit une expression comportant un nombre et une \og lettre \fg{} avec le nombre précédé de la \og lettre \fg.
        \item Attention, on ne peut pas supprimer le symbole de multiplication entre deux nombres connus. $2\times 7$ ne peut pas s'\'ecrire $27$ car $2\times 7$ vaut $14$.
    \end{itemize}
\end{remarques}

\begin{exemple}[0.5]
   Simplifier les exepressions : \\
   $A =5\times y-2\times x$. \\
   $B =x\times x \times x + 7\times y\times y$. \\
   $C =x\times3$
   \correction
      $A =5y-2x$. \\
      $B =x^3+7y^2$. \\
      $C =3x$, est préférable à $C=x3$. Par commutativité de la multiplication, $x\times3 =3\times x =3x$.
\end{exemple}