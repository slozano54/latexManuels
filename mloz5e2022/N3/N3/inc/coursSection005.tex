\section{Tester une égalité}
\begin{definition}
    Lorsque deux quantités sont égales, cela se traduit mathématiquement par \mbox{\textbf{une égalité}}.
    $$\underbrace{\mbox{expression A}}_{\mbox{1\ier{} membre ou membre de gauche}}=\underbrace{\mbox{expression B}}_{\mbox{2\ieme{} membre ou membre de droite}}$$
\end{definition}

\begin{exemple*1}
    \begin{enumerate}
        \item $5a=5b$ 
        \begin{itemize}
            \item $a$ et $b$ sont des nombres inconnus
            \item 5 fois $a$ et égal à 5 fois $b$.
            \item $5a$ est le membre de gauche, $5b$ est le membre de droite.
        \end{itemize}
        \item $7c+2=7c+3$
    \end{enumerate}
\end{exemple*1}

\begin{remarque}
    \begin{itemize}
        \item Une lettre désigne toujours le même nombre inconnu.
        \item On peut chercher à déterminer pour quelle(s) valeur(s) des nombres inconnus l'égalité est vraie.
    \end{itemize}
\end{remarque}

\begin{exemple*1}
    \begin{enumerate}
    \item Tester l'égalité $5a=5b$ lorsque $a=0$ et $b=0$
    \item Tester la même égalité, $5a=5b$ lorsque $a=2$ et $b=3$
    \end{enumerate}
\correction
    \begin{enumerate}
        \item 
        \begin{itemize}
            \item Dans ce cas le membre de gauche vaut : $5\times a=5\times 0=0$
            \item  et le membre de droite : $5\times b=5\times 0=0$
            \item On constate alors que \psshadowbox{l'égalité $5a=5b$ est vraie lorsque $a=0$ et $b=0$}.
        \end{itemize}
        \item 
        \begin{itemize}
            \item Dans ce cas le membre de gauche vaut : $5\times a=5\times 2=10$
            \item et le membre de droite : $5\times b=5\times3=15$
            \item On constate alors que \psshadowbox{l'égalité $5a=5b$ est fausse lorsque $a=2$ et $b=3$}.
        \end{itemize}
    \end{enumerate}
\end{exemple*1}