\section{Fractions irréductibles}
\begin{definition}[Fraction irreductible]
    Une fraction est dite \textbf{irréductible} lorsque son numérateur et son dénominateur n'ont pas d'autre diviseur commun que 1.
\end{definition}

\begin{propriete}[\admise]
    Toute fraction admet une unique écriture sous forme de fraction irréductible
\end{propriete}

\begin{exemple}
    Démontrer que la fraction $\dfrac23$ est irréductible.
    \correction
    $2$ et $3$ n'ont pas d'autre diviseur commun que 1 donc la fraction $\dfrac23$ est irréductible.
\end{exemple}
\begin{exemple}
    Démontrer que la fraction $\dfrac{322}{\num{1078}}$ n'est pas irréductible.
    \correction    
    $14$ divise $322$ et $\num{1078}$, donc on peut simplifier $\dfrac{322}{\num{1078}}$ par $14$.

    \medskip    
    $\dfrac{322}{\num{1078}}=\dfrac{23\times14}{77\times14}=\dfrac{23}{77}$
\end{exemple}
