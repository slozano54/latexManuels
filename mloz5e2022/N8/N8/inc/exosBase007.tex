\begin{exercice*}
   On considère un nombre entier naturel $n$. \\
   On note $S$ la somme de tous ses diviseurs stricts (c'est-à-dire ses diviseurs autres que lui-même). 
   \begin{itemize}
      \item $n$ est dit parfait lorsque $S=n$.
      \item $n$ est dit déficient lorsque $S<n$.
      \item $n$ est dit abondant lorsque $S>n$.
   \end{itemize}
   Par exemple, 8 a comme diviseurs stricts 1; 2 et 4. \\
   $S =1+2+4 =7< 8$. Donc, 8 est déficient.
   \begin{enumerate}
      \item Vérifier que 28 et 496 sont des nombre parfait.
      \item Trouver le plus petit nombre déficient, le plus petit nombre parfait et le plus petit nombre abondant.
      \item Quelle est la nature des nombres 7; 11 et 29 ?
      \item Quelle est la nature d'un nombre premier ?
   \end{enumerate}
\end{exercice*}
\begin{corrige}
   \ \\ [-5mm]
   \begin{enumerate}
      \item -- Les diviseurs stricts de 28 sont 1, 2, 4, 7 et 14. \\
         $S =1+2+4+7+14 =28$ donc, {\color{red} 28 est parfait}. \\
         -- Les diviseurs stricts de 496 sont 1, 2, 4, 8, 16, 31, 62, 124 et 248. Donc, {\color{red} 496 est un nombre parfait} car
         $S =1+2+4+8+16+31+62+124+248 =496$.
      \item \textcolor{G1}{$\bullet$} Diviseur de 1 : aucun. $S =0$, {\color{red} 1 est déficient}.
      \begin{itemize}
         \item Diviseur de 2 : 1. $S =1$ donc, 2 est déficient.
         \item Diviseur de 3 : 1. $S =1$ donc, 3 est déficient.
         \item Diviseurs de 4 : 1 ; 2. \\
            $S =1+2 =3$ donc, 4 est déficient.
         \item Diviseur de 5 : 1. $S =1$ donc, 5 est déficient.
         \item Diviseurs de 6 : 1 ; 2 ; 3. \\
            $S =1+2+3 =6$ donc, {\color{red} 6 est parfait}.
         \item Diviseur de 7 : 1. $S =1$ donc, 7 est déficient.
         \item Diviseurs de 8 : 1 ; 2 ; 4. \\
            $S =1+2+4 =7$ donc, 8 est déficient.
         \item Diviseurs de 9 : 1 ; 3. \\
            $S =1+3 =4$ donc, 9 est déficient.
         \item Diviseurs de 10 : 1 ; 2 ; 5. \\
            $S =1+2+5 =8$ donc, 10 est déficient.
         \item Diviseur de 11 : 1. $S =1$ donc, 11 est déficient.
         \item Diviseurs de 12 : 1 ; 2 ; 3 ; 4 ; 6. \\
            $S =1+2+3+4+6 =16$ donc, {\color{red} 12 est abondant}.
      \end{itemize}
      \item Le seul diviseur strict de 7, 11 et 29 est 1 donc, {\color{red} 7, 11 et 29 sont déficients}
      \item Le seul diviseur strict d'un nombre premier est 1 donc, {\color{red} les nombres premiers sont déficients}.
   \end{enumerate}
\end{corrige}
