% Les enigmes ne sont pas numérotées par défaut donc il faut ajouter manuellement la numérotation
% si on veut mettre plusieurs enigmes
% \refstepcounter{exercice}
% \numeroteEnigme
\begin{changemargin}{-10mm}{-10mm}
    \vspace*{-15mm}
    \begin{enigme}[Le jeu de Juniper-Green]
        Ce jeu mathématique se joue à deux. Il a été créé par {\it Richard Porteous}, enseignant à l'école de Juniper Green.
        
        \partie[règles du jeu]
           Deux joueurs jouent sur une grille de $N$ nombres suivant les règles suivantes :
           \begin{itemize}
              \item Règle 1 : le premier joueur choisit un nombre pair entre 1 et $N$ et le barre sur la grille.
              \item Règle 2 : chacun son tour, les deux joueurs choisissent un nombre parmi les multiples ou les diviseurs du nombre choisi précédemment par son adversaire et inférieur à $N$ puis le barre.
              \item Règle 3 : un nombre ne peut être joué qu'une seule fois.
           \end{itemize}
           Le joueur qui ne peut plus jouer a perdu. \\
           
        \partie[avec des grilles de 20 nombres]
           En binôme, jouer quelques parties sur une grille de 20 nombres. Sous la grille, noter la suite de nombres obtenue.

           {\renewcommand{\arraystretch}{1.2}
           \begin{minipage}{0.25\linewidth}
                \begin{tabular}{|*{5}{>{\centering\arraybackslash}m{0.3cm}|}}
                    \hline
                    1 & 2 & 3 & 4 & 5 \\
                        \hline
                    6 & 7 & 8 & 9 & 10 \\
                    \hline
                    11 & 12 & 13 & 14 & 15 \\
                    \hline
                    16 & 17 & 18 & 19 & 20 \\
                    \hline
                \end{tabular}

                \vspace*{7mm}
                \pointilles[0.8\linewidth]
            \end{minipage}
            \begin{minipage}{0.25\linewidth}
                \begin{tabular}{|*{5}{>{\centering\arraybackslash}m{0.3cm}|}}
                    \hline
                    1 & 2 & 3 & 4 & 5 \\
                    \hline
                    6 & 7 & 8 & 9 & 10 \\
                    \hline
                    11 & 12 & 13 & 14 & 15 \\
                    \hline
                    16 & 17 & 18 & 19 & 20 \\
                    \hline
                \end{tabular}     

                \vspace*{7mm}
                \pointilles[0.8\linewidth]
            \end{minipage}
            \begin{minipage}{0.25\linewidth}
                \begin{tabular}{|*{5}{>{\centering\arraybackslash}m{0.3cm}|}}
                    \hline
                    1 & 2 & 3 & 4 & 5 \\
                    \hline
                    6 & 7 & 8 & 9 & 10 \\
                    \hline
                    11 & 12 & 13 & 14 & 15 \\
                    \hline
                    16 & 17 & 18 & 19 & 20 \\
                    \hline
                \end{tabular}

                \vspace*{7mm}
                \pointilles[0.8\linewidth]
            \end{minipage}
            \begin{minipage}{0.25\linewidth}
                \begin{tabular}{|*{5}{>{\centering\arraybackslash}m{0.3cm}|}}
                    \hline
                    1 & 2 & 3 & 4 & 5 \\
                    \hline
                    6 & 7 & 8 & 9 & 10 \\
                    \hline
                    11 & 12 & 13 & 14 & 15 \\
                    \hline
                    16 & 17 & 18 & 19 & 20 \\
                    \hline
                \end{tabular}

                \vspace*{7mm}
                \pointilles[0.8\linewidth]
            \end{minipage}
            }

            Trouver une suite minimale : \pointilles  \\ [3mm]
            Trouver une suite maximale : \pointilles  \\
              
        \partie[avec des grilles de 100 nombres] 
           Jouer avec ces grilles de 100 nombres. \\
              \begin{center}
                 {\renewcommand{\arraystretch}{1.2}
                 \begin{tabular}{|*{10}{>{\centering\arraybackslash}m{0.3cm}|}}
                    \hline
                    1 & 2 & 3 & 4 & 5 & 6 & 7 & 8 & 9 & 10 \\
                    \hline
                    11 & 12 & 13 & 14 & 15 & 16 & 17 & 18 & 19 & 20 \\
                    \hline
                    21 & 22 & 23 & 24 & 25 & 26 & 27 & 28 & 29 & 30 \\
                    \hline
                    31 & 32 & 33 & 34 & 35 & 36 & 37 & 38 & 39 & 40 \\
                    \hline
                    41 & 42 & 43 & 44 & 45 & 46 & 47 & 48 & 49 & 50 \\
                    \hline
                    51 & 52 & 53 & 54 & 55 & 56 & 57 & 58 & 59 & 60 \\
                    \hline
                    61 & 62 & 63 & 64 & 65 & 66 & 67 & 68 & 69 & 70 \\
                    \hline
                    71 & 72 & 73 & 74 & 75 & 76 & 77 & 78 & 79 & 80 \\
                    \hline
                    81 & 82 & 83 & 84 & 85 & 86 & 87 & 88 & 89 & 90 \\
                    \hline
                    91 & 92 & 93 & 94 & 95 & 96 & 97 & 98 & 99 & 100 \\
                    \hline
                 \end{tabular}
                 \qquad
                 \begin{tabular}{|*{10}{>{\centering\arraybackslash}m{0.3cm}|}}
                    \hline
                    1 & 2 & 3 & 4 & 5 & 6 & 7 & 8 & 9 & 10 \\
                    \hline
                    11 & 12 & 13 & 14 & 15 & 16 & 17 & 18 & 19 & 20 \\
                    \hline
                    21 & 22 & 23 & 24 & 25 & 26 & 27 & 28 & 29 & 30 \\
                    \hline
                    31 & 32 & 33 & 34 & 35 & 36 & 37 & 38 & 39 & 40 \\
                    \hline
                    41 & 42 & 43 & 44 & 45 & 46 & 47 & 48 & 49 & 50 \\
                    \hline
                    51 & 52 & 53 & 54 & 55 & 56 & 57 & 58 & 59 & 60 \\
                    \hline
                    61 & 62 & 63 & 64 & 65 & 66 & 67 & 68 & 69 & 70 \\
                    \hline
                    71 & 72 & 73 & 74 & 75 & 76 & 77 & 78 & 79 & 80 \\
                    \hline
                    81 & 82 & 83 & 84 & 85 & 86 & 87 & 88 & 89 & 90 \\
                    \hline
                    91 & 92 & 93 & 94 & 95 & 96 & 97 & 98 & 99 & 100 \\
                    \hline
                 \end{tabular}}
              \end{center}
     \end{enigme}
\end{changemargin}
% Pour le corrigé, il faut décrémenter le compteur, sinon il est incrémenté deux fois
% \addtocounter{exercice}{-1}
 \begin{corrige}
    Exemple de suite minimale : {\color{red} 12 - 1 - 17}. \\
     Exemple de suite maximale : {\color{red} 20 - 10 - 5 - 15 - 3 - 9 - 18 - 6 - 12 - 4 - 16 - 8 - 2 - 14 - 7 - 1 - 19}. 
\end{corrige}