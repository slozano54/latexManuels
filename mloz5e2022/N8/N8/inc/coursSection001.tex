\section{Critères de divisibilité}
\begin{propriete}[\admise]	
	\begin{itemize}
		\item Un nombre est \textbf{divisible par 2} s'il se termine par 2, 4, 6, 8 ou 0.
		\item Un nombre est \textbf{divisible par 3} si la somme de ses chiffres est un multiple de 3.
		\item Un nombre est \textbf{divisible par 4} si ses deux derniers chiffres forment un multiple de 4.
		\item Un nombre est \textbf{divisible par 5} s'il se termine par 0 ou 5.
		\item Un nombre est \textbf{divisible par 9} si la somme de ses chiffres est un multiple de 9.
		\item Un nombre est \textbf{divisible par 10} s'il se termine par 0.
	\end{itemize}
\end{propriete}
\renewcommand{\StringPREUVE}{\textsc{Idée de la preuve du critère de divisibilité par 2}}
\begin{preuve}

	On considère un nombre à 5 chiffres qui se termine par 0;2;4;6 ou 8.

	Un raisonnement analogue pourrait être mené pour davantage de chiffres.
	
	Notons-le $ab\;cde$ où a, b, c, d et e sont des chiffres et e est un chiffre pair, 0;2;4;6 ou 8
	
	Nous avons donc l'égalité suivante, $ab\;cde = ab\;cd0 + e$ or e est pair donc $e = 2\times la\;moiti\acute{e}\;de\;e$

	Et puisque $ab\;cd0$ est un multiple de 10, il est pair aussi donc $ab\;cd0 = 2\times la\;moiti\acute{e}\;de\;ab\;cd0$	
	
	D'où $ab\;cde = 2\times la\;moiti\acute{e}\;de\;ab\;cd0 + 2\times la\;moiti\acute{e}\;de\;e$
		
	Utilisons la distributivité pour factoriser cette expression

	$ab\;cde = 2\times (la\;moiti\acute{e}\;de\;ab\;cd0 + la\;moiti\acute{e}\;de\;e)$
	
	Nous avons donc écrit $ab\;cde$ comme un multiple de 2 $\square$	
\end{preuve}
\renewcommand{\StringPREUVE}{PREUVE}

\begin{exemple}[0.5]
	Utiliser les critères ci-dessus sur $\num{10974}$.
	\correction
	\vspace*{-5mm}
	\begin{itemize}
		\item $\num{10974}$ se termine par $2$ donc $\num{10974}$ est divisible par $2$.
		\item La somme des chiffres de $\num{10974}$ vaut $21$ donc $\num{10974}$ est un multiple de $3$.
		\item La somme des chiffres de $\num{10974}$ vaut $21$ donc $\num{10974}$ n'a pas pour diviseur $9$.
		\item $\num{10974}$ ne se termine ni par $0$ ni par $5$ donc $5$ ne divise pas $\num{10974}$.
		\item \dots		
	\end{itemize}
\end{exemple}
\begin{propriete}[Pour la culture générale \emoji{smile} \admise]	
	\begin{itemize}
		\item Un nombre est \textbf{divisible par 7} si la somme de son nombre de dizaines et de cinq fois son chiffre des unités l'est.
		\item Un nombre est \textbf{divisible par 11} si la différence entre la somme de ses chiffres de rangs pairs et
	la somme de ses chiffres de rangs impairs est nulle ou égale à un multiple de 11.
	\end{itemize}
\end{propriete}

\begin{methode}[Critère de divisibilité par 11]
	\exercice 
	Tester le critère de divisibilité par $11$ sur :
	\begin{itemize}
		\item $\num{16456}$
		\item $\num{81829}$
		\item $\num{81839}$
	\end{itemize}
	\correction
	\vspace*{-5mm}
	\begin{itemize}
		\item $\blue{1}\red{6}\;\blue{4}\red{5}\blue{6}$ est divisible par 11 car ($\blue{1}+\blue{4}+\blue{6}$)-($\red{6}+\red{5}$)$=0$
		\item $\blue{8}\red{1}\;\blue{8}\red{2}\blue{9}$ est divisible par 11 car ($\blue{8}+\blue{8}+\blue{9}$)-($\red{1}+\red{2}$)$=22$ est un multiple de 11.
		\item $\blue{8}\red{1}\;\blue{8}\red{3}\blue{9}$ est divisible par 11 car ($\blue{8}+\blue{8}+\blue{9}$)-($\red{1}+\red{3}$)$=21$ n'est pas un multiple de 11.
	\end{itemize}
\end{methode}

\begin{methode}[Lister tous les diviseurs d'un nombre]
	À l'aide des critères de divisibilité ou des tables de multiplication, on peut lister tous les diviseurs d'un nombre
	en écrivant tous les produits qui donnent ce nombre.

	En organisant l'énumération de façon a toujours écrire le plus petit facteur en premier et en l'incrémentant petit à petit,
	on aura un signal de fin !
	\exercice
	Lister tous les diviseurs de $40$.
	\correction
	\vspace*{-5mm}
	\begin{itemize}
		\item $40 = 1 \times 40$ 
		\item $40 = 2 \times 20$
		\item $40$ n'est pas divisible par $3$
		\item $40 = 4 \times 10$
		\item $40 = 5 \times 8$
		\item $40$ n'est pas divisible par $6$
		\item $40$ n'est pas divisible par $7$
		\item $40$ est divisible par $8$ mais $40 = 8 \times 5$ or nous avons fait le choix de toujours mettre le plus petit facteur en premier donc $40 = 5 \times 8$ que nous avons déjà écrit, l'algorithme est donc fini.
	\end{itemize}

	La liste des diviseurs de 40 est donc : $1 ; 2 ; 4 ; 5 ; 8 ; 10 ; 20 ; 40$
\end{methode}