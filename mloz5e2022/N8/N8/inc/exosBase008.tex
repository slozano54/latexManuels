\begin{exercice*}
   Les chrysodes traduisent des propriétés relatives à la division euclidienne et aux nombres premiers. Ils s'obtiennent à partir d'un cercle gradué.

   \begin{minipage}{0.7\linewidth}
      \begin{enumerate}
         \item Choisir un nombre de 1 à 6. Le multiplier par 3, puis calculer le reste de la division par 7. Tracer le corde d'extrémités le point correspondant au nombre choisi et le point correspondant au reste obtenu.
         \item À partir de ce reste, recommencer plusieurs fois. Que constate-ton ?
      \end{enumerate}  
   \end{minipage}
   \begin{minipage}{0.5\linewidth}
      \begin{center}
         \begin{Geometrie}
            pair O,A[];
            path co;
            O=u*(3,3);
            co=cercles(O,1.5u);
            A0=pointarc(co,90);
            A1=pointarc(co,90-360/7);
            A2=pointarc(co,90-2*360/7);
            A3=pointarc(co,90-3*360/7);
            A4=pointarc(co,90-4*360/7);
            A5=pointarc(co,90-5*360/7);
            A6=pointarc(co,90-6*360/7);
            trace co;
            marque_p:="plein";
            pointe(A0,A1,A2,A3,A4,A5,A6);
            label.top(TEX("0"),A0);
            label.urt(TEX("1"),A1);
            label.lrt(TEX("2"),A2);
            label.lrt(TEX("3"),A3);
            label.llft(TEX("4"),A4);
            label.llft(TEX("5"),A5);
            label.ulft(TEX("6"),A6);
         \end{Geometrie}
      \end{center}
   \end{minipage}
\end{exercice*}
\begin{corrige}
   Les chrysodes traduisent des propriétés relatives à la division euclidienne et aux nombres premiers. Ils s'obtiennent à partir d'un cercle gradué.

   \begin{enumerate}
      \item Choisir un nombre de 1 à 6. Le multiplier par 3, puis calculer le reste de la division par 7. Tracer le corde d'extrémités le point correspondant au nombre choisi et le point correspondant au reste obtenu.
      \item À partir de ce reste, recommencer plusieurs fois. Que constate-ton ?
      
      {\color{red}On remarque que tous les points, sauf 0,  sont atteints et que la ligne brisée est fermée.}
   \end{enumerate}
   \begin{minipage}{\linewidth}
      \begin{center}
         \begin{Geometrie}
            pair O,A[];
            path co;
            O=u*(3,3);
            co=cercles(O,1.5u);
            A0=pointarc(co,90);
            A1=pointarc(co,90-360/7);
            A2=pointarc(co,90-2*360/7);
            A3=pointarc(co,90-3*360/7);
            A4=pointarc(co,90-4*360/7);
            A5=pointarc(co,90-5*360/7);
            A6=pointarc(co,90-6*360/7);
            trace co;
            marque_p:="plein";
            pointe(A0,A1,A2,A3,A4,A5,A6);
            label.top(TEX("0"),A0);
            label.urt(TEX("1"),A1);
            label.lrt(TEX("2"),A2);
            label.lrt(TEX("3"),A3);
            label.llft(TEX("4"),A4);
            label.llft(TEX("5"),A5);
            label.ulft(TEX("6"),A6);
            trace polygone(A1,A3,A2,A6,A4,A5) withcolor red;
         \end{Geometrie}
      \end{center}
   \end{minipage}
\end{corrige}
