\vspace*{-5mm}
\begin{changemargin}{-7mm}{-7mm}
    %pre-001
    \begin{prerequis}[Connaisances \emoji{red-heart} et compétences \emoji{diamond-suit} du cycle 3]    
   \begin{itemize}        
       \item[\emoji{red-heart}] Vocabulaire associé à ces objets et à leurs propriétés : côté, sommet, angle, hauteur.
       \columnbreak
       \item[\emoji{diamond-suit}] Reconnaître, nommer, décrire des triangles, dont les triangles particuliers (triangle rectangle, triangle isocèle, triangle équilatéral).       
   \end{itemize}
\end{prerequis}
    \begin{debat}[La cryptologie]
        La {\bf cryptologie} est un art ancien et une science nouvelle : un art ancien car Jules César l’utilisait déjà ; une science nouvelle parce que ce n’est un thème de recherche scientifique que depuis les années 1970. Ce mot vient du grec {\it krypton} - caché et {\it logos} - science et signifie science du secret. Elle englobe la {\bf cryptographie} (l’écriture secrète) et la {\bf cryptanalyse} (l’analyse de cette dernière). Actuellement, on utilise en cryptographie des méthodes basées sur la difficulté de trouver la décomposition d'un nombre en produit de facteurs premiers.
        \begin{center}
           \begin{pspicture}(0,0.5)(2.5,3)
              \psset{fillstyle=solid}
              \psframe[fillcolor=A1,framearc=0.2,linecolor=A1](0.5,0.5)(2,2)
              \pscircle[fillcolor=white,linecolor=white](1.25,1.5){0.15}
              \psline[linewidth=0.15,linecolor=white]{c-c}(1.25,0.9)(1.25,1.35)
              \psarc[linewidth=0.25](1.25,2){0.5}{0}{180}         
           \end{pspicture}
           \begin{pspicture}(0,0.5)(2.5,3)
              \psset{fillstyle=solid}
               \psarc[linewidth=0.25](1.2,2.1){0.5}{45}{225} 
              \psframe[fillcolor=B1,framearc=0.2,linecolor=B1](0.5,0.5)(2,2)
               \pscircle[fillcolor=white,linecolor=white](1.25,1.5){0.15}
              \psline[linewidth=0.15,linecolor=white]{c-c}(1.25,0.9)(1.25,1.35)       
           \end{pspicture}
           \begin{pspicture}(0,0.5)(2.5,3)
              \psset{fillstyle=solid}
              \psframe[fillcolor=A1,framearc=0.2,linecolor=A1](0.5,0.5)(2,2)
              \pscircle[fillcolor=white,linecolor=white](1.25,1.5){0.15}
              \psline[linewidth=0.15,linecolor=white]{c-c}(1.25,0.9)(1.25,1.35)
              \psarc[linewidth=0.25](1.25,2){0.5}{0}{180}         
           \end{pspicture}
        \end{center}
        \bigskip
        \begin{cadre}[B2][F4]
           \begin{center}
              \hrefVideo{https://www.yout-ube.com/watch?v=4jPtEsDS-qI}{\bf Les nombres premiers}, épisode de la série {\it Petits contes mathématiques}.
           \end{center}
        \end{cadre}
    \end{debat}
\end{changemargin}