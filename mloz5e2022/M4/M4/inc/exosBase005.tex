\begin{exercice*}
   Déterminer la longueur inconnue.
   \begin{enumerate}
      \item Parallélogramme de base \Lg[cm]{8} et d'aire \Aire[cm]{24}. Calculer la hauteur.
      \item Parallélogramme de hauteur \Lg[cm]{30} et d'aire \udmq{2,1}. Calculer la mesure de la base relative à cette hauteur.
   \end{enumerate}
\end{exercice*}
\begin{corrige}
   \ \\ [-5mm]
   \begin{enumerate}
      \item $\mathcal{A} =\Lg[cm]{8}\times h =\Aire[cm]{24}$ soit $h =\dfrac{\Aire[cm]{24}}{\Lg[cm]{8}} =\Lg[cm]{3}$. \\ [1mm]
         La {\color{red} hauteur} du parallélogramme mesure {\color{red} \Lg[cm]{3}}. \smallskip
      \item $\mathcal{A} =b\times\Lg[cm]{30} =\Aire[cm]{210}$ ; $b =\dfrac{\Aire[cm]{210}}{\Lg[cm]{30}} =\Lg[cm]{7}$. \\ [1mm]
      La {\color{red} base} du parallélogramme mesure {\color{red} 7 cm}.
   \end{enumerate}
\end{corrige}
