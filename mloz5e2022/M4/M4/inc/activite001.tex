\vspace*{-15mm}
\begin{activite}[Aire d'un parallélogramme]
    {\bf Objectif :} calculer l'aire d'un rectangle ; déterminer la formule de l'aire d'un parallélogramme. \\

    Pour cette activité, il faut se munir d'une paire de ciseau et de colle.
        \begin{enumerate}
            \item Sur le cahier, tracer une droite horizontale sur toute la largeur de la page en laissant \Lg[cm]{4} au dessus puis dessiner un rectangle de \Lg[cm]{5} sur \Lg[cm]{3} sur la ligne comme ceci :
                \begin{center}
                    \begin{pspicture}(0,-0.5)(16,2.3)
                    \psline[linewidth=1mm](0,0)(16,0)
                    \psframe(1,0)(4,1.8)
                    \rput(2.5,1.55){\footnotesize \Lg[cm]{5}}
                    \rput{-90}(3.75,0.9){\footnotesize \Lg[cm]{3}}
                    \end{pspicture}
                \end{center}
            \item Découper les trois rectangles tracés en bas de page dont une longueur est en gras (la base). \\
                Ils mesurent tous \Lg[cm]{5} par \Lg[cm]{3}.
            \item Transformer les trois rectangles en trois parallélogrammes en donnant un seul coup de ciseaux en ligne droite en partant de la base (côté en gras) et en coupant le bord supérieur.
            \item Coller les deux morceaux obtenus côte à côte sur le cahier en apposant le côté gras sur la droite tracée. \\
                Quelle est la nature de chaque quadrilatère obtenu ? \par \medskip
                \pointilles \par \medskip
            \item Que peut-on dire de l'aire des quatre quadrilatères obtenus ? \par \medskip
                \pointilles \par \medskip
            \item À partir de la formule de l'aire du rectangle, déterminer comment calculer l'aire d'un parallélogramme. \par \medskip
                \pointilles \par \medskip
                \pointilles \par \medskip
                \pointilles \medskip
        \end{enumerate}
        \vspace*{-10mm}
        \begin{changemargin}{-5mm}{-25mm}
            \begin{center}
                \begin{pspicture}(0,-0.5)(5.5,4)
                    \psframe(0,0)(5,3)
                    \psline[linewidth=1mm](0,0)(5,0)
                \end{pspicture}
                \begin{pspicture}(0,-0.5)(5.5,4)
                    \psframe(0,0)(5,3)
                    \psline[linewidth=1mm](0,0)(5,0)
                \end{pspicture}
                \begin{pspicture}(0,-0.5)(5,4)
                    \psframe(0,0)(5,3)
                    \psline[linewidth=1mm](0,0)(5,0)
                \end{pspicture}
            \end{center}        
        \end{changemargin}        
 
    \hfill {\footnotesize\it Inspiré de : \href{http://www.gem-math.be/spip.php?article14}{\og Aire des parallélogrammes \fg}, Groupe d'Enseignement Mathématique, Belgique.}
 \end{activite}