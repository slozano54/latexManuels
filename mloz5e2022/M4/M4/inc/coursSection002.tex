\section{Aire du parallélogramme}

\begin{propriete}
   L'aire du parallélogramme se calcule grâce à la formule :
   $$\mathcal{A} =\text{base}\times\text{hauteur} =b\times h$$
\end{propriete}

\begin{remarque}
   attention à ne pas confondre avec l'aire du rectangle qui est longueur $\times$ largeur.
\end{remarque}

\begin{exemple}[0.3]
   Déterminer l'aire du parallélogramme suivant : \\
   \psset{unit=0.9}
   \begin{pspicture}(1,-3)(5,3)
      \rput{-90}(2,2.5){\pspolygon(0,0)(4,0)(5,2.5)(1,2.5)}
   \end{pspicture}
   \correction
      \begin{minipage}{4.5cm}
         \begin{itemize}
            \item On définit une base ;
            \item on trace une hauteur relative à cette base ;
            \item on mesure la base ;
            \item on mesure la hauteur ;
            \item on applique la formule : \\
            $\mathcal{A} =b\times h$ \\
            $\mathcal{A} =\ucm{4}\times\ucm{2,5}$ \\
            $\mathcal{A} =\ucmq{10}$.
         \end{itemize}
      \end{minipage}
      \begin{minipage}{4cm}
        \psset{unit=0.9}
         \begin{pspicture}(0.5,-3)(5,3)
            \rput{-90}(2,2.5){
            \pspolygon(0,0)(4,0)(5,2.5)(1,2.5)
            \psset{linecolor=B1}
            \color{B1}
            \psline{<->}(0,-0.3)(4,-0.3)
            \rput(2,-0.6){$b =\ucm{4}$}
            \psline(1,0)(1,2.5)
            \psframe(1,0)(1.3,0.3)
            \rput{90}(1.3,1.3){$h =\ucm{2,5}$}
           }
         \end{pspicture}
      \end{minipage}
\end{exemple}