% Les enigmes ne sont pas numérotées par défaut donc il faut ajouter manuellement la numérotation
% si on veut mettre plusieurs enigmes
% \refstepcounter{exercice}
% \numeroteEnigme
\vspace*{-20mm}
\begin{enigme}[La formule de Pick]
   On travaille dans un réseau pointé à maille carrée. On note $u.\ell.$ l'unité de longueur et $u.a.$ l'unité d'aire. \\
   On appelle polygone de Pick, un polygone non aplati construit sur un tel réseau et dont chacun des sommets est un point du réseau. On considère les figures $FORMULE$ et $PICK$ suivantes :
   \begin{center}
      \small
      {\psset{unit=0.5}
      \begin{pspicture}(6,0)(28,11.5)
         \pstGeonode[fillstyle=solid,fillcolor=lightgray!30,CurveType=polygon,PosAngle={45,135,-135,-50,-30,45}](9,10){R}(7,6){O}(7,1){F}(12,1){E}(16,3){L}(16,6){U}(12,6){M}
         \psframe[fillstyle=solid,fillcolor=lightgray!30](22,9)(23,10)
         \pstGeonode[fillstyle=solid,fillcolor=lightgray!30,CurveType=polygon,PosAngle={-150,-45,30,135}](18,1){P}(24,1){I}(27,5){C}(21,5){K}
         \psgrid[griddots=1,gridlabels=0,subgriddiv=1,gridwidth=0.5mm](6,0)(28,11)
         \rput(22.5,8.5){1 $u.a.$}
         \psline{<->}(18,9)(19,9)
         \rput(18.5,8.5){1 $u.\ell.$}      
      \end{pspicture}}
   \end{center}
   
   \partie[avec les formules classiques]
     \ \\ [-10mm]
      \begin{enumerate}
         \item Calculer l'aire du parallélogramme $PICK$ en unité d'aire. \par \bigskip
            \pointilles  \\
         \item Calculer l'aire du polygone $FORMULE$, en unité d'aire en détaillant les étapes du raisonnement. \par \bigskip
            \pointilles  \par \bigskip
            \pointilles \bigskip
      \end{enumerate}
   
   \partie[avec la formule de Pick]
      La formule de Pick permet de calculer l'aire $\mathcal{A}$ d'un polygone de Pick, à partir du nombre $i$ de points du réseau strictement intérieurs à ce polygone et du nombre $b$ de points du réseau sur le bord du polygone : \fbox{$\mathcal{A} =i+\dfrac{b}{2}-1$}. 
      \begin{enumerate}
      \setcounter{enumi}{2}
         \item Appliquer cette formule au parallélogramme $PICK$. \\ [3mm]
            $i =\pointilles  \quad b =\pointilles $ \quad donc, $\mathcal{A} = \pointilles $ \\
          \item Appliquer cette formule au polygone $FORMULE$. \\ [3mm]
            $i =\pointilles  \quad b =\pointilles $ \quad donc, $\mathcal{A} = \pointilles $ \\
          \item Appliquer la formule de Pick aux trois polygones de Pick $MOFE$, $MOR$ et $MULE$. \\
            Vérifier que la somme des résultats obtenus est égale à l'aire totale de la figure. \par \bigskip
            \pointilles \par \bigskip
            \pointilles \par \bigskip
            \pointilles 
      \end{enumerate}
\end{enigme}

% Pour le corrigé, il faut décrémenter le compteur, sinon il est incrémenté deux fois
% \addtocounter{exercice}{-1}
\begin{corrige}
   \begin{enumerate}
      \item $\mathcal{A}_{PICK} =6\,u.\ell.\times4\,u\ell. ={\color{red} 24\,u.a.}$
      \item $\mathcal{A}(MOFE) =5\,u.\ell.\times5\,u.\ell. =25\,u.a.$ \\ [1mm] 
         $\mathcal{A}(MOR) =\dfrac{5\,u.\ell.\times4\,u.\ell.}{2} =10\,u.a.$ \\ [1mm]
         $\mathcal{A}(MULE) =4\,u.\ell.\times3\,u.\ell.+\dfrac{4\,u.\ell.\times2\,u.\ell.}{2} =16\,u.a.$ \\ [1mm]
         En sommant, on obtient : $\mathcal{A}(FORMULE)$ \\
         $=25\,u.a.+10\,u.a.+16\,u.a. ={\color{red} 51\,u.a.}$ \smallskip
      \item {\color{red} $i =18$} et {\color{red} $b =14$} donc, $\mathcal{A} =18+\dfrac{14}{2}-1 ={\color{red} 24}$. \smallskip
      \item {\color{red} $i =41$} et {\color{red} $b =22$} donc, $\mathcal{A} =41+\dfrac{22}{2}-1 ={\color{red} 51}$. \smallskip
      \item On décompose comme à la question 2)
         \begin{itemize}
            \item MOFE : $i =16$, $b =20$, $\mathcal{A} =16+\dfrac{20}{2}-1 =25$. \smallskip
            \item MOR : $i =7$ et $b =8$, $\mathcal{A} =7+\dfrac{8}{2}-1 =10$. \smallskip
            \item MULE : $i =10$ et $b =14$, $\mathcal{A} =10+\dfrac{14}{2}-1 =16$. \smallskip
            \item FORMULE : $\mathcal{A} =25+10+16 =51$. \\
         {\color{red} La somme des résultats obtenus est égale au résultat trouvé à la question 2).}
      \end{itemize}
   \end{enumerate}
\end{corrige}


   
