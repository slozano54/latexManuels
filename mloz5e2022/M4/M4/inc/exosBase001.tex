\begin{exercice*}
   Calculer l'aire puis le périmètre :
   \begin{enumerate}
      \item d’un rectangle de longueur \Lg[m]{30} et de largeur \Lg[m]{20} ;
      \item d’un carré de côté \Lg[cm]{6} ;
      \item du parallélogramme suivant :
      \begin{center}
      {\small
      \psset{unit=1}
         \begin{pspicture}(1,0)(6,2.7)
            \pspolygon(0,2)(3,0)(6,0)(3,2)
            \psline(3,0)(3,2)
            \psframe(3,0)(3.3,0.3)
            \psframe(3,2)(2.7,1.7)
            \rput(1.5,2.3){\Lg[cm]{12}}
            \rput{90}(2.7,1){\Lg[cm]{5}}
            \rput{-35}(4.5,1.35){\Lg[cm]{13}}  
         \end{pspicture}}
      \end{center}
   \end{enumerate}
\end{exercice*}
\begin{corrige}
   \ \\ [-5mm]
   \begin{enumerate}
      \item Aire du rectangle : $\Lg[m]{30}\times\Lg[m]{20} =\color{red} \Aire[m]{600}$. \\
        Périmètre du rectangle : \\
        $2\times(\Lg[m]{30}+\Lg[m]{20}) =\color{red} \Lg[m]{100}$.
      \item Aire du carré : $\Lg[cm]{6}\times\Lg[cm]{6} =\color{red} \Aire[cm]{36}$. \\
        Périmètre du carré : $4\times\Lg[cm]{6} =\color{red} \Lg[cm]{24}$.
     \item Aire du parallélogramme : $\Lg[cm]{12}\times\Lg[cm]{5} =\color{red} \Aire[m]{60}$. \\
        Périmètre : $2\times(\Lg[cm]{12}+\Lg[cm]{13}) =\color{red} \Lg[m]{50}$.
   \end{enumerate}
\end{corrige}
