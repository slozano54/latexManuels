\vspace*{-13mm}
%pre-001
\begin{prerequis}[Connaisances \emoji{red-heart} et compétences \emoji{diamond-suit} du cycle 3]    
   \begin{itemize}        
       \item[\emoji{red-heart}] Vocabulaire associé à ces objets et à leurs propriétés : côté, sommet, angle, hauteur.
       \columnbreak
       \item[\emoji{diamond-suit}] Reconnaître, nommer, décrire des triangles, dont les triangles particuliers (triangle rectangle, triangle isocèle, triangle équilatéral).       
   \end{itemize}
\end{prerequis}
\begin{debat}[Des aires en images]
    Le parallélogramme peut être vu comme un rectangle que l'on aurait \og étiré \fg{} par un coin, son {\bf aire} correspond donc à l'aire un rectangle.
    \begin{center}
       \begin{pspicture}(0,-0.5)(15.5,2.5)
          \rput(0.1,1.7){\Huge \ding{43}}
          \psframe[fillstyle=solid,fillcolor=A3,linecolor=A3](0.5,0)(3.5,2)
          \rput(5.55,1.7){\Huge \ding{43}}
          \pspolygon[fillstyle=solid,fillcolor=A3,linecolor=A3](5,0)(8,0)(9,2)(6,2)
          \psframe[linestyle=dashed,linecolor=B1](5,0)(8,2)
          \rput(10.95,1.7){\Huge \ding{43}}
          \pspolygon[fillstyle=solid,fillcolor=A3,linecolor=A3](9.5,0)(12.5,0)(14.5,2)(11.5,2)
          \psframe[linestyle=dashed,linecolor=B1](9.5,0)(12.5,2)
       \end{pspicture}
    \end{center}
    \begin{cadre}[B2][J4]
       \begin{center}
          \hrefVideo{https://www.yout-ube.com/watch?v=5_PiZrfLghQ}{\bf Aire de figures simples}, chaîne YouTube de {\it Science silencieuse}.
       \end{center}
    \end{cadre}
 \end{debat}