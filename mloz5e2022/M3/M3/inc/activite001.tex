\begin{changemargin}{-5mm}{-15mm}
\begin{activite}[Des pavés cachés]
    {\bf Objectifs :} calculer le volume d'un pavé droit, d'un assemblage de solides ; résoudre un problème dans le domaine des grandeurs et mesures.

    {\bf Matériel à disposition :} des cubes à emboiter, des briques types kapla ou Lego, des boites, des morceaux de sucres.
    \partie[On me voit assez bien !]
        \begin{minipage}{0.7\linewidth}
            Ma petite sœur empile des cubes les uns sur les autres, tous de même forme. Voici sa construction. \\
            Combien a-t-elle empilé de cubes ? 
            \vspace*{15mm}
        \end{minipage}
        \hfill
        \begin{minipage}{0.25\linewidth}
            \VueCubes[%
            Seul,%
            Angle=60,%
            Largeur=5,%
            Profondeur=5,%
            Creation,%            
            ]{%
            3,4,5,5,5,%
            2,4,4,5,5,%
            1,3,4,4,5,%
            1,2,3,4,4,%
            1,1,1,2,3%
            }
        \end{minipage}
        
    \partie[On me voit un peu moins !]
        \begin{minipage}{0.6\linewidth}
            Mon grand-père veut construire la même jardinière que sur la photo ci-contre.

            De combien de briques aura-t-il besoin ?
            \vspace*{20mm}
        \end{minipage}
        \hfill
        \begin{minipage}{0.35\linewidth}
            \includegraphics[width=6.5cm]{\currentpath/images/jardiniere}
        \end{minipage}
        
    \partie[On ne me voit plus !]
        L’entreprise Sucromania fabrique du sucre en morceaux. Elle souhaite conditionner ses morceaux dans un emballage parallélépipédique de \Lg[mm]{280} de long, \Lg[mm]{140} de large et \Lg[mm]{70} de hauteur.

        Sachant qu’un sucre a la forme d’un pavé droit de \Lg[mm]{14} de long, \Lg[mm]{14} de large et \Lg[mm]{10} de hauteur, combien de sucres peut-elle mettre au maximum afin d’optimiser son emballage ?
        \vspace*{40mm}

        \hfill {\it\footnotesize Source : d'après \href{https://eduscol.education.fr/document/13132/download}{\og La résolution de problèmes mathématiques au collège}, MENJS, p.136}
        \vspace*{-40mm}
 \end{activite}
\end{changemargin}