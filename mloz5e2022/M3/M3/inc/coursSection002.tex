\section{Deux nouveaux solides : Le prisme et le cylindre}
\begin{definition}[Le prisme droit]
   \begin{minipage}{0.25\linewidth}
      \begin{Geometrie}[CoinBG={u*(-10,-10)},CoinHD={u*(10,10)},TypeTrace="Espace"]
         Initialisation(5,30,30,70);
         color A,B,C,D,E,F;
         D=(0.75,0,0);
         F=(0.75,1/2,0);
         E=(0,1/2,0);
         A-D=(0,0,1);
         C-F=A-D;
         B-E=A-D;
         NbS:=6;
         Sommet1:=A;
         Sommet2:=B;
         Sommet3:=C;
         Sommet4:=D;
         Sommet5:=E;
         Sommet6:=F;
         NF:=5;
         Fc[100]:=3;Fc[101]:=1;Fc[102]:=3;Fc[103]:=2;
         Fc[200]:=3;Fc[201]:=4;Fc[202]:=5;Fc[203]:=6;
         Fc[300]:=4;Fc[301]:=1;Fc[302]:=2;Fc[303]:=5;Fc[304]:=4;
         Fc[400]:=4;Fc[401]:=1;Fc[402]:=4;Fc[403]:=6;Fc[404]:=3;
         Fc[500]:=4;Fc[501]:=3;Fc[502]:=6;Fc[503]:=5;Fc[504]:=2;
         DessineObjet;
         label.ulft(btex A etex,Projette(A));
         label.top(btex B etex,Projette(B));
         label.lrt(btex C etex,Projette(C));
         label.rt(btex E etex,Projette(E));
         label.bot(btex F etex,Projette(F));
         label.llft(btex D etex,Projette(D));
         % trace appelation(D,A,3mm,btex hauteur etex);
         trace cotationmil(D,A,5mm,20,btex hauteur etex);
      \end{Geometrie} 
   \end{minipage}
   \hfill
   \begin{minipage}{0.7\linewidth}
      \begin{itemize}
         \item  ses bases sont parallèles, ce sont des polygones superposables;
         \item  les autres faces (faces latérales) sont des rectangles;
         \item  les arêtes qui relient les deux bases ont toutes la même longueur; cette longueur est la hauteur du prisme droit.
      \end{itemize}
   \end{minipage}
\end{definition}

\begin{definition}[Le cylindre]
   \begin{minipage}{0.25\linewidth}
      \begin{Geometrie}[CoinBG={u*(-10,-10)},CoinHD={u*(10,10)},TypeTrace="Espace"]
         Initialisation(5,0,20,70);
         color O,O',A,A',B,B',C,C';
         O=(0,0,0);
         O'-O=(0,0,1);
         A-O=(0,1/2,0);
         A'-A=O'-O;
         C=symetrie(A,O);
         C'-C=O'-O;
         B-O=(-1/2,0,0);
         B'-B=O'-O;
         path cc,cd;
         cc=cercles(O,A,O,A,B);
         cd=cercles(O',A',O',A',B');
         trace cd;
         trace segment(C,C');
         trace segment(A,A');
         trace (subpath(0,length cc/2) of cc) dashed evenly;
         trace subpath(length cc/2,length cc) of cc;
         trace segment(O,A) dashed evenly;
         trace segment(O',A') ;
         trace segment(O,O') dashed evenly;
         trace codeperp(A,O,O',5);
         trace codeperp(A',O',O,5);
         trace cotationmil(A,A',-3mm,20,btex hauteur etex);
         % trace cotationmil(O,O',3mm,10,btex hauteur etex);
         marque_p:="croix";
         pointe(O);
         pointe(O');
         marque_p:="non";
      \end{Geometrie}  
   \end{minipage}
   \hfill
   \begin{minipage}{0.7\linewidth}
      \begin{itemize}
         \item  ses bases sont parallèles, ce sont des disques superposables;
         \item  le segment qui joint les centres des disques de base est perpendiculaire aux bases, sa longueur est la hauteur du cylindre.
      \end{itemize}
   \end{minipage}
\end{definition}