\section{Volume par dénombrement}
\begin{definition}
   Le \textbf{volume} est une grandeur physique qui mesure l'espace occupé par celui-ci.
\end{definition}

\begin{exemple*1}
%    {\psset{unit=0.5}
   Ces trois objets n'ont pas la même forme mais occupent la même quantité d'espace, ils ont donc le même volume.
   \begin{center}
    \hfill
    \VueCubes[%
        Seul,%
        Angle=60,%
        Largeur=1,%
        Profondeur=3,%
        Creation,%            
    ]{2,2,2}
    \hfill
    \VueCubes[%
        Seul,%
        Angle=60,%
        Largeur=1,%
        Profondeur=4,%
        Creation,%
        CouleurCube=LightSalmon%            
    ]{2,1,1,2}
    \hfill
    \VueCubes[%
        Seul,%
        Angle=60,%
        Largeur=1,%
        Profondeur=3,%
        Creation,%
        CouleurCube=LightGreen%            
    ]{3,2,1}
    \hfill\phantom{r}
   \end{center}
   Si l'unité de volume est un cube 
    \VueCubes[%
            Seul,%
            Angle=60,%
            Largeur=1,%
            Profondeur=1,%
            Creation,%
            CouleurCube=LightGray%
        ]{1}
   \hfill le volume de ces trois solides est de 6 unités de volume.
%    }
\end{exemple*1}

Il existe deux unités en dimension 3 : les unités de volumes en \og cube \fg{} et les unités de capacité en \og litre \fg. 

\begin{definition}
   \begin{itemize}
      \item Lorsque l'unité de volume est un cube de \Lg[m]{1} d'arête, cela représente \Vol[m]{1}.
      \item Le {\bf litre} (L) est une unité de capacité valant \Vol[dm]{1}. On a alors $\Capa[L]{1} =\Vol[dm]{1}$ et $\Capa[L]{1000} =\Vol[m]{1}$.
   \end{itemize}
   \vspace*{-4mm}
\end{definition}

Pour effectuer un changement d'unité de volume, on reprend les même préfixes que pour les changements de longueur, et on impose pour chacun d'eux trois colonnes au tableau.

\Tableau[Cube,Capacite,FlechesH]{39621/12}
\vspace*{-8mm}

\begin{exemple*1}
    \Vol[dam]{0,0039621} = \Vol[m]{3,9621} = \Vol[dm]{3962,1} = \Capa[L]{3962,1} = \Vol[cm]{3962100}.
\end{exemple*1}