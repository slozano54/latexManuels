\begin{exercice*} %6
   Pour un chantier, un maçon doit construire quatre colonnes en béton de forme cylindrique, de \Lg[cm]{50} de rayon et de \um{4} de hauteur.
   \begin{enumerate}
      \item Quel est le volume total des colonnes ?
      \item Pour \Vol[m]{1} de béton, il faut \Masse[kg]{400} de ciment, \Capa[L]{460} de sable, \Capa[L]{780} de gravillons et \Capa[L]{200} d'eau.
         Donner la quantité de ciment, de sable, de gravillons et d'eau nécessaire pour les quatre colonnes.
   \end{enumerate}
\end{exercice*}

\begin{corrige}
   \begin{enumerate}
      \item $V =4\times(\pi\times(\um{0,5})^2\times\um{4}) \approx \Vol[m]{12,57}$. \\
         \textcolor{red}{Le volume des colonnes est d'environ \Vol[m]{12,57}}.
      \item Les valeurs sont pour \Vol[m]{1} donc, il suffit de multiplier toutes les quantités par 12,57 : \\
         Il faut $12,57\times\Masse[kg]{400} =$ \textcolor{red}{\Masse[kg]{5028} de ciment} ; \\
         $12,57\times\Capa[L]{460} \approx$ \textcolor{red}{\Capa[L]{5 782} de sable} ; \\
         $12,57\times\Capa[L]{780} \approx$ \textcolor{red}{\Capa[L]{9 805} de gravillons} ; \\
         $12,57\times\Capa[L]{200} =$ \textcolor{red}{\Capa[L]{2 514} d'eau}.
   \end{enumerate}
\end{corrige}
