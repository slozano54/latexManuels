%pre-001
\begin{prerequis}[Connaisances \emoji{red-heart} et compétences \emoji{diamond-suit} du cycle 3]    
   \begin{itemize}        
       \item[\emoji{red-heart}] Vocabulaire associé à ces objets et à leurs propriétés : côté, sommet, angle, hauteur.
       \columnbreak
       \item[\emoji{diamond-suit}] Reconnaître, nommer, décrire des triangles, dont les triangles particuliers (triangle rectangle, triangle isocèle, triangle équilatéral).       
   \end{itemize}
\end{prerequis}
%pre-002
\begin{prerequis}[Connaisances \emoji{red-heart} et compétences \emoji{diamond-suit} du cycle 4]    
    \begin{itemize}        
        \item[\emoji{diamond-suit}] Mener des calculs impliquant des grandeurs mesurables, exprimer les résultats dans des les unités adaptées.
        \item[\emoji{diamond-suit}] Exprimer et vérifier la cohérence des résultats du point de vue des unités.
    \end{itemize}
\end{prerequis}
\vfill
\begin{debat}[Débat : repère, ou repaire ?] 
   Ces deux mots ont la même origine, le nom latin {\it rapatrirare}, \og rentrer chez soi, rentrer dans sa patrie \fg. \\
   {\bf Repaire} a assez vite pris le sens de \og gîte d'animaux sauvages \fg. Au moyen-âge, l'écriture du nom n'était pas encore stabilisée et s'écrivait également {\bf repère}, que l'on rattacha à tort au nom latin {\it reperire}, \og retrouver \fg. Finalement, repère se spécialise pour désigner une marque permettant de retrouver quelque chose. \\ [2mm]
    Revenons aux mathématiques, comment se repérer, selon le type d'objet où l'on se trouve :
   \begin{center}
      {\psset{unit=0.8,linecolor=B1}
      \begin{pspicture}(0,-0.5)(15,3.3)
         \psline(0,0)(3,3)
         \rput(1.5,-0.7){\small Sur une droite ?}
         \psframe(4,0)(7,3)
         \rput(5.5,-0.7){\small Sur un plan ?}
         \psframe(8,0)(10,2) 
         \psline(10,0)(11,1)(11,3)(9,3)(8,2)
         \psline(10,2)(11,3)
         \rput(9.5,-0.7){\small Dans l'espace ?}
         \pscircle(13.5,1.5){1.5}
         \psellipticarc(13.5,1.5)(1.5,0.6){180}{0}
         \rput(13.5,-0.7){\small Sur une sphère ?}
      \end{pspicture}}    
   \end{center}
   \bigskip
   \begin{cadre}[B2][J4]
      \begin{center}
         Vidéo : \href{https://www.yout-ube.com/watch?v=Tu2kRuZcWRI}{\bf C'est quoi un jeu en 4D (et 1D) ?}, chaîne YouTube {\it Trash Bandicoot}.
      \end{center}
   \end{cadre}
\end{debat}
