% Les enigmes ne sont pas numérotées par défaut donc il faut ajouter manuellement la numérotation
% si on veut mettre plusieurs enigmes
\refstepcounter{exercice}
% \numeroteEnigme
\begin{enigme}[Déformations]
    \partie[dessin]
       \begin{minipage}{10cm}
          \begin{enumerate}
             \item Placer ces points dans le repérage ci-contre. \\ [-9mm]
                \begin{multicols}{4}
                    \begin{itemize}
                        \item[]$A(-3;2)$
                        \item[] $B(-3;-5)$
                        \item[] $C(3;-5)$
                        \item[] $D(3;2)$
                        \item[] $E(4;1)$
                        \item[] $F(4;2)$
                        \item[] $G(0;6)$
                        \item[] $H(-4;2)$
                        \item[] $I(-4;1)$
                        \item[] $J(0;5)$
                        \item[] $K(-1;-5)$
                        \item[] $L(-1;-2)$
                        \item[] $M(1;-2)$
                        \item[] $N(1;-5)$
                        \item[] $O(-2;0)$
                        \item[] $P(-2;2)$
                        \item[] $Q(-1;2)$
                        \item[] $R(-1;0)$
                        \item[] $S(1;-1)$
                        \item[] $T(1;1)$
                        \item[] $U(2;1)$
                        \item[] $V(2;-1)$
                    \end{itemize}
                \end{multicols}
                \vspace*{-4mm}
          \item Relier les points suivants : \\
             ABCDEFGHIJD \\
             KLMN \\
             OPQRO \\
             STUVS
          \end{enumerate}
       \end{minipage}
       \qquad
       \begin{minipage}{6cm}
          {\psset{unit=0.5}
          \begin{pspicture}(-6,-6)(5,6)
             \psgrid[subgriddiv=0,gridcolor=lightgray,gridlabels=0](0,0)(-5,-6)(5,7)
             \psaxes[labels=none]{->}%
 (0,0)(-5,-6)(5,7)
             \rput(-0.4,-0.4){\scriptsize 0}
             \rput(1,-0.5){\scriptsize 1}
             \rput(-0.5,1){\scriptsize 1}
          \end{pspicture}}
       \end{minipage}
       \bigskip
       
    \partie[déformations]
       Tracer le dessin dans les repères suivants : que se passe-t-il ? \\ [2mm]
       \begin{minipage}{6cm}
          {\psset{xunit=0.5}
          \begin{pspicture}(-6,-5)(5,7)
             \psgrid[subgriddiv=0,gridcolor=lightgray,gridlabels=0](0,0)(-5,-6)(5,7)
             \psaxes[labels=none]{->}%
 (0,0)(-5,-6)(5,7)
             \rput(-0.4,-0.4){\scriptsize 0}
             \rput(1,-0.5){\scriptsize 1}
             \rput(-0.5,1){\scriptsize 1}
          \end{pspicture}}
       \end{minipage}
       \begin{minipage}{11cm}
          {\psset{yunit=0.5}
          \begin{pspicture}(-6,-6)(5,7)
             \psgrid[subgriddiv=0,gridcolor=lightgray,gridlabels=0](0,0)(-5,-6)(5,7)
             \psaxes[labels=none]{->}%
 (0,0)(-5,-6)(5,7)
             \rput(-0.4,-0.4){\scriptsize 0}
             \rput(1,-0.5){\scriptsize 1}
             \rput(-0.5,1){\scriptsize 1}
          \end{pspicture}} \\
          \pstilt{55}{
          {\psset{xunit=0.7,yunit=0.6}
          \begin{pspicture}(-5,-5)(5,8)
             \psgrid[subgriddiv=0,gridcolor=lightgray,gridlabels=0](0,0)(-5,-6)(5,7)
             \psaxes[labels=none]{->}%
 (0,0)(-5,-6)(5,7)
             \rput(-0.4,-0.4){\scriptsize 0}
             \rput(1,-0.5){\scriptsize 1}
             \rput(-0.5,1){\scriptsize 1}
          \end{pspicture}}}
       \end{minipage}
 \end{enigme}
 
% Pour le corrigé, il faut décrémenter le compteur, sinon il est incrémenté deux fois
% \addtocounter{exercice}{-1}
\begin{corrige}
   \setcounter{partie}{0} % Pour s'assurer que le compteur de \partie est à zéro dans les corrigés
   \phantom{rrr}

   \begin{minipage}{0.4\linewidth}
    \partie[dessin]
       {\psset{unit=0.5}
          \begin{pspicture}(-6,-6)(5,8)
             \psgrid[subgriddiv=0,gridcolor=lightgray,gridlabels=0](0,0)(-5,-6)(5,7)
             \psaxes[labels=none]{->}%
 (0,0)(-5,-6)(5,7)
             \rput(-0.4,-0.4){\scriptsize 0}
             \rput(1,-0.5){\scriptsize 1}
             \rput(-0.5,1){\scriptsize 1}
             \psset{linecolor=blue}
             \psline(-3,2)(-3,-5)(3,-5)(3,2)(4,1)(4,2)(0,6)(-4,2)(-4,1)(0,5)(3,2) 
             \psframe(-1,-5)(1,-2)
             \psframe(-2,0)(-1,2)
             \psframe(1,-1)(2,1)
          \end{pspicture}}
         %  \bigskip
   \end{minipage}
   \begin{minipage}{0.5\linewidth}          
    \partie[déformations]
       {\psset{xunit=0.5}
          \begin{pspicture}(-6,-5)(5,7.5)
             \psgrid[subgriddiv=0,gridcolor=lightgray,gridlabels=0](0,0)(-5,-6)(5,7)
             \psaxes[labels=none]{->}%
 (0,0)(-5,-6)(5,7)
             \rput(-0.4,-0.4){\scriptsize 0}
             \rput(1,-0.5){\scriptsize 1}
             \rput(-0.5,1){\scriptsize 1}
             \psset{linecolor=blue}
             \psline(-3,2)(-3,-5)(3,-5)(3,2)(4,1)(4,2)(0,6)(-4,2)(-4,1)(0,5)(3,2) 
             \psframe(-1,-5)(1,-2)
             \psframe(-2,0)(-1,2)
             \psframe(1,-1)(2,1)
          \end{pspicture}}
   \end{minipage}

 \Coupe
       {\psset{yunit=0.5}
          \begin{pspicture}(-4.05,-6)(5,10.55)
             \psgrid[subgriddiv=0,gridcolor=lightgray,gridlabels=0](0,0)(-5,-6)(5,7)
             \psaxes[labels=none]{->}%
 (0,0)(-5,-6)(5,7)
             \rput(-0.4,-0.4){\scriptsize 0}
             \rput(1,-0.5){\scriptsize 1}
             \rput(-0.5,1){\scriptsize 1}
             \psset{linecolor=blue}
             \psline(-3,2)(-3,-5)(3,-5)(3,2)(4,1)(4,2)(0,6)(-4,2)(-4,1)(0,5)(3,2) 
             \psframe(-1,-5)(1,-2)
             \psframe(-2,0)(-1,2)
             \psframe(1,-1)(2,1)
          \end{pspicture}}

      \pstilt{55}{
       {\psset{xunit=0.7,yunit=0.6}
          \begin{pspicture}(-2,-5)(5,10.5)
             \psgrid[subgriddiv=0,gridcolor=lightgray,gridlabels=0](0,0)(-5,-6)(5,7)
             \psaxes[labels=none]{->}%
               (0,0)(-5,-6)(5,7)
             \rput(-0.4,-0.4){\scriptsize 0}
             \rput(1,-0.5){\scriptsize 1}
             \rput(-0.5,1){\scriptsize 1}
             \psset{linecolor=blue}
             \psline(-3,2)(-3,-5)(3,-5)(3,2)(4,1)(4,2)(0,6)(-4,2)(-4,1)(0,5)(3,2) 
             \psframe(-1,-5)(1,-2)
             \psframe(-2,0)(-1,2)
             \psframe(1,-1)(2,1)
          \end{pspicture}}}
 \end{corrige}
