\begin{exercice}
   On considère les points de coordonnées : \\
   {\renewcommand{\arraystretch}{1.2}
   \begin{tabular}{p{1.6cm}p{1.6cm}p{1.6cm}p{1.6cm}}
      $M(-9\,;-5)$ & $N(-4\,;0)$ & $O'(2,5\,;7)$ & $P(5\,;3)$ \\
      $Q(-1\,;-1)$ & $R(2\,;-3)$ & $S(5\,;-2)$ & $T(-6,5\,;-2)$ \\
      $U(-1\,;-4)$ & $V(2\,;0)$ & $W(-6,5\,;4)$ & $X(-9\,,0)$ \\
      $Y(-4\,;-5)$ & $Z(-6,5\,;-1)$ & & \\
   \end{tabular}}
   \begin{enumerate}
      \item Créer un repère orthogonal en prenant un centimètre pour une unité qui puisse contenir tous les points.
      \item Placer les points dans le repère.
      \item Relier dans l'ordre les points suivants :
      \begin{itemize}
         \item $W-X-M-Y-N-W-O'-P-S-Y$.
         \item $U-Q-V-R$.
         \item $X-N-P$.
         \item Tracer le cercle de centre $T$ passant par $Z$.
      \end{itemize}
      Imane reconnait un dessin familier. Quel est-il ?
   \end{enumerate}
\end{exercice}

\begin{corrige}
   Imane reconnait le dessin d'{\blue une maison}. \\
   {\psset{unit=0.7,PointSymbol=none}
   \begin{pspicture}(-10,-5.6)(6,8.3)
   \psgrid[gridlabels=0,subgriddiv=0,gridcolor=lightgray](-10,-6)(6,8)
      \psaxes[labels=none,ticks=none]{->}(0,0)(-10,-6)(6,8)
      \psline(1,-0.2)(1,0.2)
      \psline(-0.2,1)(0.2,1)
      \footnotesize
      \rput(-0.4,-00.4){$O$}
      \rput(1,0.5){1}
      \rput(-0.5,1){1}
      \psset{linecolor=blue}
      \pstGeonode[PosAngle={90,135,-135,-45,50}](-6.5,4){W}(-9,0){X}(-9,-5){M}(-4,-5){Y}(-4,0){N}
      \pstLineAB{W}{X}
      \pstLineAB{X}{M}
      \pstLineAB{M}{Y}
      \pstLineAB{Y}{N}
      \pstLineAB{N}{W}
      \pstGeonode[PosAngle=45](2.5,7){O'}(5,3){P}(5,-2){S}
      \pstLineAB{W}{O'}
      \pstLineAB{O'}{P}
      \pstLineAB{P}{S}
      \pstLineAB{S}{Y}
      \pstGeonode[PosAngle={-90,135,45,-45},CurveType=polyline](-1,-4){U}(-1,-1){Q}(2,0){V}(2,-3){R}
      \pstLineAB{U}{Q}
      \pstLineAB{Q}{V}
      \pstLineAB{V}{R}
      \pstLineAB{X}{N}
      \pstLineAB{N}{P}
      \pstGeonode(-6.5,-2){T}(-6.1,-1){Z}
      \psdot(-6.5,-2)
      \pstCircleOA{T}{Z}      
   \end{pspicture}}

   \smallskip
\end{corrige}