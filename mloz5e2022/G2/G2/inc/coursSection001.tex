\section{La droite graduée (rappels)}

\begin{definition}
   Pour graduer une droite, il faut choisir une {\bf origine} qui correspond au \og 0 \fg{} et une {\bf unité} qui sera reportée de manière régulière. \\
   Sur une droite graduée, un point est repéré par son {\bf abscisse}.
\end{definition}
      
\begin{center}
   \begin{pspicture}(-3,-1.3)(6.6,1)
      \psaxes[yAxis=false]{->}(0,0)(-3,0)(5.1,0)
      \psline[linecolor=gray]{<-}(0,0)(-0.5,0.5)
      \rput(-1,0.7){\textcolor{gray}{origine}}
      \psline[linecolor=gray]{<->}(0,0.3)(1,0.3)
      \rput(0.5,0.6){\textcolor{gray}{unité}}
      \rput(3,0.4){\textcolor{A1}{A}}
      \rput(3,-0.9){\textcolor{A1}{l'abscisse du}}
      \rput(3,-1.3){\textcolor{A1}{point A est 3}}
      \rput(3,-1.7){\textcolor{A1}{on note A(3)}}
      \rput(6.6,0){\textbf{sens croissant}}
   \end{pspicture}
\end{center}