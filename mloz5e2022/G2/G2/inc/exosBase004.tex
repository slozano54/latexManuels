\begin{exercice}
   L'image suivante représente la position obtenue au déclenchement du bloc \og Départ \fg{} d'un programme. \\
   L'arrière-plan est constitué de points espacés de 40 unités. Le chat a pour coordonnées $(-120\,;\,-80)$. \\
   Le but du jeu est de positionner le chat sur la balle représentée par le petit disque.
   \begin{center}
   {\psset{unit=0.35}
   \begin{pspicture}(-6,-3.3)(6,4.3)
      \multido{\n=-6+1}{13}{
      \multido{\na=-4+1}{9}{\psdots[dotscale=0.5](\n,\na)}}
      \psaxes[Dx=10,Dy=10]{->}(0,0)(-6,-4)(6,4)
      \uput[dl](0,0){O}
      \psdots[dotscale=2](4,3)
      \rput(-3,-2){Chat}
   \end{pspicture}}
   \end{center}
   \begin{enumerate}
      \item Quelles sont les coordonnées du centre de la balle représentée dans cette position?
      \item Dans cette question, le chat est dans la position obtenue au déclenchement du bloc départ. \\
      Voici le script du lutin \og chat \fg{} qui se déplace. \\ [2mm]
      \begin{Scratch}[Echelle=0.75]
         Place Drapeau;
         Place Bloc("Départ");
      \end{Scratch}
      \quad
      \begin{Scratch}[Echelle=0.75]
         Place QPresse("n'importe laquelle");
         Place Si(TestCapToucheObjet("Balle"));
            Place DireT("Je t'ai attrapée","2");
            Place Bloc("Départ");
         Place FinBlocSi;
      \end{Scratch} \\ [2mm]
      \hspace*{-5mm}   
      \begin{Scratch}[Echelle=0.65]
         Place QPresse("flèche droite");
         Place AjouterVar("80","x");
      \end{Scratch}
      \begin{Scratch}[Echelle=0.65]
         Place QPresse("flèche haut");
         Place AjouterVar("80","y");
      \end{Scratch} \\ [2mm]
      \hspace*{-5mm}
      \begin{Scratch}[Echelle=0.65]
         Place QPresse("flèche gauche");
         Place AjouterVar("-40","x");
      \end{Scratch}
      \begin{Scratch}[Echelle=0.65]
         Place QPresse("flèche bas");
         Place AjouterVar("-40","y");
      \end{Scratch} 
      \begin{enumerate}
         \item Expliquer pourquoi le chat ne revient pas à sa position de départ si le joueur appuie sur la touche $\to$ puis sur la touche $\gets$.
         \item Le joueur appuie sur la succession de touches suivante : $\to$ $\to$  $\uparrow$ $\gets$ $\downarrow$. Quelles sont les coordonnées $x$ et $y$ du chat après ce déplacement ? Justifier.
         \item Parmi les propositions ci-dessous, laquelle permet au chat d'atteindre la balle ? \medskip
      \end{enumerate}
      \hspace*{-5mm}
      {\small
      \renewcommand{\arraystretch}{1.2}
      \begin{tabular}{|c|c|c|}
         \hline
         déplacement 1 & déplacement 2 & déplacement 3\\ \hline
         $\to \to \to \to \to \to \to \uparrow\uparrow\uparrow\uparrow\uparrow$
         &
         $\to\to\to \uparrow\uparrow\uparrow \to \downarrow \gets$
         &
         $\uparrow \to \uparrow \to \uparrow \to \to \downarrow \downarrow$ \\
         \hline
      \end{tabular}}
      \smallskip
      \item Que se passe-t-il quand le chat atteint la balle ?
   \end{enumerate}
\end{exercice}

\begin{corrige}
   \ \\ [-5mm]
   \begin{enumerate}
      \item La balle est située à quatre espaces vers la droite et trois espaces vers le haut, soit $4\times40$ unités = 160 unités et $3\times40$ unités = 120 unités. \\
      Ses coordonnées sont donc {\blue (160\,;\,120)}.
      \item \\
      \begin{enumerate}
         \item La touche $\to$ ajoute 80 à l'abscisse $x$ ; la touche $\gets$ ajoute $-40$ à l'abscisse $x$ ; donc, la succession $\to \, \gets$ ajoute $80+(-40) =40$ à l'abscisse $x$. \\
            Le chat a donc \og avancé \fg{} de 40 unités vers la droite,  {\blue il ne revient pas à sa position de départ}. \vspace*{11.5cm}
         \item On résume dans un tableau les déplacements : \\ \smallskip
            {\hautab{1.3}
            \begin{tabular}{|*{7}{c|}}
               \hline
               & départ & $\to$ & $\to$ & $\uparrow$ & $\gets$ & $\downarrow$ \\
               \hline
               $x$ & $-120$ & $-40$ & 40 & 40 & 0 & 0 \\
               \hline 
               $y$ & $-80$ & $-80$ & $-80$ & 0 & 0 & $-40$ \\
               \hline
            \end{tabular}} \\ \smallskip
         Les coordonnées du chat après ces cinq déplacements sont {\blue $(0\,;\,-40)$}.
         \item Seul le {\blue déplacement 2} permet au chat d'attraper la balle.
      \end{enumerate}
      \setcounter{enumi}{2}
      \item Quand le chat atteint la balle, {\blue il dit \og Je t'ai attrapée \fg{} pendant 2 sec.} puis retourne au départ. 
   \end{enumerate}
\end{corrige}
