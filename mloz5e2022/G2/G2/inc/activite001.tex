\begin{activite}[Dessin gradué]
    \vspace*{-5mm}
    {\bf Objectif :} créer un dessin en utilisant un repérage particulier. 

       Pour découvrir le dessin codé, il faut placer les points A, B, C\dots{} selon les indications du tableau ci-dessous. Par exemple, le point A est sur la première ligne et son abscisse est 8. \\
       Une fois tous les points placés les relier en suivant les instructions données.
       \begin{center}
        \let\originalTextwidth\textwidth
        \setlength{\linewidth}{25cm}
        \hspace*{-20mm}
          \begin{tabular}{|*{3}{>{\centering\arraybackslash}p{0.5cm}|}}
             \hline
             \cellcolor{lightgray}{\!\!\!\small Ligne} & \cellcolor{lightgray}{\!\!\!\small Point} & \cellcolor{lightgray}{\!\small Abs.} \\
             \hline
             (1) & A & 8 \\
             \hline
             (1) & B & 9 \\
             \hline
             (1) & C & 12 \\
             \hline
             (2) & D & 17 \\
             \hline
             (2) & E & 18 \\
             \hline
             (2) & F& 19 \\
             \hline
             (2) & G & 20 \\
             \hline
             (2) & H & 21 \\
             \hline
             (2) & I & 22 \\
             \hline
             (2) & J & 23 \\
             \hline
             (3) & K & 57 \\
             \hline
          \end{tabular}
          \hfill
          \begin{tabular}{|*{3}{>{\centering\arraybackslash}p{0.5cm}|}}
             \hline
             \cellcolor{lightgray}{\!\!\!\small Ligne} & \cellcolor{lightgray}{\!\!\!\small Point} & \cellcolor{lightgray}{\!\small Abs.} \\
             \hline
             (3) &  L & 58 \\
             \hline
             (3) & M & 59 \\
             \hline
             (3) & N & 63 \\
             \hline
             (3) & O & 64 \\
             \hline
             (4) & P & 23 \\
             \hline
             (4) & Q & 24 \\
             \hline
             (4) & R & 25 \\
             \hline
             (4) & S & 28 \\
             \hline
             (4) & T & 29 \\
             \hline
             (5) & U & 22 \\
             \hline
             (5) & V & 24 \\
             \hline
          \end{tabular}
          \hfill
          \begin{tabular}{|*{3}{>{\centering\arraybackslash}p{0.5cm}|}}
             \hline
             \cellcolor{lightgray}{\!\!\!\small Ligne} & \cellcolor{lightgray}{\!\!\!\small Point} & \cellcolor{lightgray}{\!\small Abs.} \\
             \hline
             (5) & W & 26 \\
             \hline
             (6) & X & 36 \\
             \hline
             (6) & Y & 44 \\
             \hline
             (7) & Z & 6 \\
             \hline
             (7) & A' & 14 \\
             \hline
             (7) & B' & 18 \\
             \hline
             (7) & C' & 22 \\
             \hline
             (8) & D' & 15 \\
             \hline
             (8) & E' & 18 \\
             \hline
             (8) & F' & 27 \\
             \hline
             (9) & G' & 103 \\
             \hline
          \end{tabular}
          \hfill
          \begin{tabular}{|*{3}{>{\centering\arraybackslash}p{0.5cm}|}}
             \hline
             \cellcolor{lightgray}{\!\!\!\small Ligne} & \cellcolor{lightgray}{\!\!\!\small Point} & \cellcolor{lightgray}{\!\small Abs.} \\
             \hline
             (9) & H' & 107 \\
             \hline
             (9) & I' & 108 \\
             \hline
             (10) & J' & 2 \\
             \hline
              (10) & K' & 4 \\
             \hline
             (10) & L' & 16 \\
             \hline
             (11) & M' & 50 \\
             \hline
             (11) & N' & 80 \\
             \hline
             (12) & O' & 32 \\
             \hline
             (12) & P' & 44 \\
             \hline
             (13) & Q' & 0,1 \\
             \hline
             (13) & R' & 0,2 \\
             \hline
          \end{tabular}
          \hfill
          \begin{tabular}{|*{3}{>{\centering\arraybackslash}p{0.5cm}|}}
             \hline
             \cellcolor{lightgray}{\!\!\!\small Ligne} & \cellcolor{lightgray}{\!\!\!\small Point} & \cellcolor{lightgray}{\!\small Abs.} \\
             \hline
             (13) & S' & 0,3 \\
             \hline
             (13) & T' & 0,5 \\
             \hline
             (13) & U' & 0,6 \\
             \hline
             (13) & V' & 0,7 \\
             \hline
             (13) & W' & 0,8 \\
             \hline
             (13) & X' & 0,9 \\
             \hline
             (14) & Y' & $-15$ \\
             \hline
             (14) & Z' & $-14$ \\
             \hline
             (14) & A'' & $-11$ \\
             \hline
             (14) & B'' & $-7$ \\
             \hline
             (14) & C'' & $-6$ \\
             \hline
          \end{tabular}
       \end{center}
       \begin{minipage}{4cm}
          Tracer les lignes brisées \\
          suivantes : \\ [3mm]
          FELMPKDACOTWVY \\
          C'P’C"B"V’O’W’X’B’XA’  \\ [3mm]
          GB \\ [3mm]
          HJNI \\ [3mm]
          ST \\ [3mm]
          QRUC’ \\ [3mm]
          D’E’L’N’U’T’ \\ [3mm]
          F’I’H’ \\ [3mm]
          U’V’ \\ [3mm]
          B"A"S’M’G’K’R’Z’Y’Q’J’ZK \\ [3mm]
          ZG’
       \end{minipage}
       \qquad
       \begin{minipage}{10cm}
             \DessinGradue[LignesIdentiques=false,Echelle=1,EcartVertical=0.8]
             {0/15/15,
             10/25/15,
             50/65/15,
             15/30/15,
             0/30/30, %5
             20/50/15,
             0/30/15,
             0/45/15,
             100/115/15,
             0/30/15, %10
             0/150/15,
             0/60/30,
             0/1.5/15,
             -15/0/15}
             {1/A/8,1/B/9,1/C/12,
             2/D/7,2/E/8,2/F/9,2/G/10,2/H/11,2/I/12,2/J/13,
             3/K/7,3/L/8,3/M/9,3/N/13,3/O/14,
             4/P/8,4/Q/9,4/R/10,4/S /13,4/T/14,
             5/U/22,5/V/24,5/W/26,
             6/X/8,6/ Y/12,
             7/Z/3,7/A'/7,7/B'/9,7/C'/11,
             8/D'/5 ,8/E'/6,8/F'/9,
             9/G'/3,9/H'/7,9/I'/8,
             10/J'/1,10/K'/2,10/L'/8,
             11/M'/5,11/N'/8,
             12/O'/16,12/P'/22,
             13/Q'/1,13/R'/2,13/S'/3,13/ T'/5,13/U'/6,13/V'/7,13/W'/8,13/X'/9,
             14/Y'/0,14/Z'/1,14/A''/4,14/B''/8,14/C''/9}
             {chemin(F,E,L,M,P,K,D,A,C,O,T,W,V,Y,C',P',C '',B'',V',O',W',X',B',X,A'),
             chemin(G,B),
             chemin(H,J,N,I),
             chemin(S,T),
             chemin(Q,R,U ,C'),
             chemin(D',E',L',N',U',T'),
             chemin(F ',I',H'),
             chemin(U',V'),
             chemin(B'',A'',S',M',G',K',R',Z',Y',Q',J',Z,K),
             chemin(Z, G')}
       \end{minipage} \smallskip
    \vfill\hfill{\it\small Activité inspirée de la brochure APMEP n°169 : \og Jeux 7 \fg}
 \end{activite}