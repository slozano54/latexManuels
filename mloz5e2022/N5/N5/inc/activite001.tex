\begin{activite}[Des briques et des fractions]
    \begin{changemargin}{-10mm}{-15mm}
        \vspace*{-5mm}
        {\bf Objectifs :} utiliser des fractions pour exprimer une proportion ; produire des fractions égales, ranger des fractions. \\
        \vspace*{-5mm}
        \partie[les Lego®]
            \begin{minipage}{0.65\linewidth}
                On choisit la brique de Lego® classique $u$ ci-contre que l'on prend comme unité et les onze briques $a$ à $k$. On considère que le volume d'une brique est proportionnel au nombre de \og boutons \fg{} présents sur le dessus.
            \end{minipage}
            \hfill
            \begin{minipage}{0.3\linewidth}
                $u$ : \includegraphics[width=3.15cm]{\currentpath/images/lego_4_2a}
            \end{minipage} \\ [10mm]
                $a$ : \includegraphics[width=5.22cm]{\currentpath/images/lego_8_2a} \qquad 
                $b$ : \includegraphics[width=3.6cm]{ \currentpath/images/lego_6_2a} \qquad
                $c$ : \includegraphics[width=1.35cm]{\currentpath/images/lego_2_2a} \qquad
                $d$ : \includegraphics[width=1.8cm]{ \currentpath/images/lego_3_2a} \\ [5mm]
                $e$ : \includegraphics[width=6.3cm]{ \currentpath/images/lego_10_2a} \qquad
                $f$ : \includegraphics[width=0.9cm]{ \currentpath/images/lego_1_1a} \qquad
                $g$ : \includegraphics[width=1.35cm]{\currentpath/images/lego_2_1a} \qquad
                $h$ : \includegraphics[width=2.7cm]{ \currentpath/images/lego_4_1a} \\ [5mm]
                $i$ : \includegraphics[width=5.4cm]{ \currentpath/images/lego_8_1a} \qquad
                $j$ : \includegraphics[width=1.98cm]{\currentpath/images/lego_3_1a} \qquad
                $k$ : \includegraphics[width=4.05cm]{\currentpath/images/lego_6_1a} \\
                
        \partie[les fractions]
            \begin{enumerate}
                \item Compléter les égalités suivantes à l'aide de nombres fractionnaires : \\ [-3mm]
                \begin{multicols}{6}
                    $a = \pointilles u$ \\ [7mm]
                    $b = \pointilles u$ \\ [7mm]
                    $c = \pointilles u$ \\ [7mm]
                    $d = \pointilles u$ \\ [7mm]
                    $e = \pointilles u$ \\ [7mm]
                    $f = \pointilles u$ \\ [7mm]
                    $g = \pointilles u$ \\ [7mm]
                    $h = \pointilles u$ \\ [7mm]
                    $i = \pointilles u$ \\ [7mm]
                    $j = \pointilles u$ \\ [7mm]
                    $k = \pointilles u$ \\ [7mm]
                \end{multicols} \medskip
                \item Compléter les égalités suivantes à l'aide de fractions dont le dénominateur est 8 : \\ [-3mm]
                \begin{multicols}{6}
                    $a = \dfrac{\qquad}{8} u$ \\ [7mm]
                    $b = \dfrac{\qquad}{8} u$ \\ [7mm]
                    $c = \dfrac{\qquad}{8} u$ \\ [7mm]
                    $d = \dfrac{\qquad}{8} u$ \\ [7mm]
                    $e = \dfrac{\qquad}{8} u$ \\ [7mm]
                    $f = \dfrac{\qquad}{8} u$ \\ [7mm]
                    $g = \dfrac{\qquad}{8} u$ \\ [7mm]
                    $h = \dfrac{\qquad}{8} u$ \\ [7mm]
                    $i = \dfrac{\qquad}{8} u$ \\ [7mm]
                    $j = \dfrac{\qquad}{8} u$ \\ [7mm]
                    $k = \dfrac{\qquad}{8} u$ \\ [7mm]
                \end{multicols}
                \item Classer les Lego® dans l'ordre croissant de leur volume : \par \medskip
                \pointilles
        \end{enumerate}
    \end{changemargin}
    \vspace*{-50mm}
\end{activite}