\begin{exercice*}
   Résoudre les deux problèmes suivants.
   \begin{enumerate}
      \item La collecte : \Prix{20} ont été collectés par 3 élèves lors de la vente de gâteaux. Jim en  a collecté le quart, Paul trois huitièmes et Jane le reste. \\
         Sachant qu’une part de gâteau coûtait 50 centimes, combien de parts de gâteaux ont-ils vendues chacun ?
       \item Économies : Je dépense quatre septièmes de mes économies pour acheter un manteau et le tiers du reste pour une paire de chaussettes. J’ai maintenant \Prix{9.52}. \\
          Combien avais-je d’économies au départ ?
   \end{enumerate}
\end{exercice*}
\begin{corrige}
   \begin{enumerate}
      \item Une part de gâteau coûte \Prix{0.50}. \\
         S'ils ont collecté \Prix{20}, cela signifie qu'ils ont vendus au total 40 parts de gâteau. \\
         Modélisons la situation par un graphique en barre : \\ [1mm]
         \ModeleBarre{Turquoise 8 {"\Prix{20} = 40 parts de gâteaux"}}{LightSkyBlue -2 "Jim $\frac14$" PaleTurquoise -3 "Paul $\frac38$" PowderBlue -3 "Jane"} \\
         On observe que 8 briques unité correspondent à 40 parts de gâteaux. \\
         Une brique correspond donc à 5 parts. \\
         {\red Jim a vendu 10 parts de gâteaux, Paul et Jane en ont vendu 15 chacun}.
      \item Modélisons la situation par un graphique en barre : \\ [1mm]
         \ModeleBarre{Turquoise 7 {"Mes économies"}}{LightSkyBlue -4 "manteau" PaleTurquoise -1 "chaussettes" PowderBlue -2 "\Prix{9.52}"} \\
         \Prix{9.52} correspondent à 2 briques unité. \\
         Une brique est donc égale à \Prix{4.76} ($9,52\div2$) ; \\
         7 briques équivalent à \Prix{33.32} ($7\times4,76$). \\
         {\red Au départ, j'avais \Prix{33.32}.}
   \end{enumerate}
\end{corrige}