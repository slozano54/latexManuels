\begin{exercice*}
   Écrire la fraction qui représente la partie colorée de chaque figure puis la simplifier si possible.
   \begin{center}
      a \Fraction[Regulier,Cotes=4,Rayon=6mm,Couleur=Cyan,Reponse]{1/4} \quad b \Fraction[Regulier,Cotes=4,Rayon=6mm,Couleur=Cyan,Reponse]{3/4} \quad c \Fraction[Regulier,Cotes=4,Rayon=6mm,Couleur=Cyan,Reponse]{2/4} \quad d\Fraction[Regulier,Cotes=4,Rayon=6mm,Couleur=Cyan,Reponse]{4/4} \medskip
      
       e \Fraction[Rayon=6mm,Couleur=IndianRed,Reponse]{10/12} \quad f \Fraction[Rayon=6mm,Couleur=IndianRed,Reponse]{3/12} \quad g \Fraction[Rayon=6mm,Couleur=IndianRed,Reponse]{2/12} \quad h \Fraction[Rayon=6mm,Couleur=IndianRed,Reponse]{6/12} \smallskip
       
       i \Fraction[Triangle,Longueur=12mm,Couleur=LimeGreen,Reponse]{3/9} \quad j \Fraction[Triangle,Longueur=12mm,Couleur=LimeGreen,Reponse]{7/9} \quad k \Fraction[Triangle,Longueur=12mm,Couleur=LimeGreen,Reponse]{6/9} \quad l \Fraction[Triangle,Longueur=12mm,Couleur=LimeGreen,Reponse]{8/9} \medskip
       
      {\psset{unit=0.6}
      \begin{pspicture}(-1,-1.5)(2.2,1)
         \psgrid[subgriddiv=2,subgridcolor=black,subgridwidth=0.8pt,gridlabels=0](-1,-1)(1,1)
         \psframe[fillstyle=solid,fillcolor=J1](-1,0)(0,1)
         \rput(-1.4,-0.9){m}
      \end{pspicture}
      \begin{pspicture}(-1,-1.5)(2.2,1)
         \psgrid[subgriddiv=2,subgridcolor=black,subgridwidth=0.8pt,gridlabels=0](-1,-1)(1,1)
         \psframe[fillstyle=solid,fillcolor=J1](-1,0)(0,1)
         \psframe[fillstyle=solid,fillcolor=J1](0,0.5)(0.5,0)
         \psframe[fillstyle=solid,fillcolor=J1](-1,-0.5)(-0.5,0)
         \rput(-1.4,-0.9){n}
      \end{pspicture}
      \begin{pspicture}(-1,-1.5)(2.2,1)
         \psgrid[subgriddiv=2,subgridcolor=black,subgridwidth=0.8pt,gridlabels=0](-1,-1)(1,1)
         \psframe[fillstyle=solid,fillcolor=J1](-1,-1)(1,0)
         \psframe[fillstyle=solid,fillcolor=J1](-1,0)(0,1)
         \psframe[fillstyle=solid,fillcolor=J1](0,0)(0.5,0.5)
         \rput(-1.4,-0.9){p}
      \end{pspicture}
      \begin{pspicture}(-1,-1.5)(1,1)
         \psgrid[subgriddiv=2,subgridcolor=black,subgridwidth=0.8pt,gridlabels=0](-1,-1)(1,1)
         \psframe[fillstyle=solid,fillcolor=J1](-1,-1)(0.5,0)
         \psframe[fillstyle=solid,fillcolor=J1](-1,0)(-0.5,1)
         \psframe[fillstyle=solid,fillcolor=J1](0.5,-0.5)(1,0)
         \rput(-1.4,-0.9){q}
      \end{pspicture}}   
   \end{center}
\end{exercice*}
\begin{corrige}
   \phantom{rrr}

   \begin{multicols}{2}
      \begin{itemize}
         \item a $=\red \dfrac14$ \medskip
         \item b $=\red \dfrac34$ \medskip
         \item c $=\red \dfrac24 =\dfrac12$ \medskip
         \item d $=\red \dfrac44 =1$ \medskip
         \item e $=\red \dfrac{10}{12} =\dfrac56$ \medskip
         \item f $=\red \dfrac{3}{12} =\dfrac14$
         \item g $=\red \dfrac{2}{12} =\dfrac16$
         \item h $=\red \dfrac{6}{12} =\dfrac12$
         \item i $=\red \dfrac39 =\dfrac13$
         \item j $=\red \dfrac79$
         \item k $=\red \dfrac69 =\dfrac23$
         \item l $=\red \dfrac89$
         \item m $=\red \dfrac{4}{16} =\dfrac14$
         \item n $=\red \dfrac{6}{16} =\dfrac38$
         \item p $=\red \dfrac{13}{16}$
         \item q $=\red \dfrac{9}{16}$
      \end{itemize}
   \end{multicols}
\end{corrige}