\vspace*{-7mm}
\begin{changemargin}{-10mm}{-10mm}
%pre-001
\begin{prerequis}[Connaisances \emoji{red-heart} et compétences \emoji{diamond-suit} du cycle 3]    
   \begin{itemize}        
       \item[\emoji{red-heart}] Vocabulaire associé à ces objets et à leurs propriétés : côté, sommet, angle, hauteur.
       \columnbreak
       \item[\emoji{diamond-suit}] Reconnaître, nommer, décrire des triangles, dont les triangles particuliers (triangle rectangle, triangle isocèle, triangle équilatéral).       
   \end{itemize}
\end{prerequis}
\vspace*{-3mm}
%pre-002
\begin{prerequis}[Connaisances \emoji{red-heart} et compétences \emoji{diamond-suit} du cycle 4]    
    \begin{itemize}        
        \item[\emoji{diamond-suit}] Mener des calculs impliquant des grandeurs mesurables, exprimer les résultats dans des les unités adaptées.
        \item[\emoji{diamond-suit}] Exprimer et vérifier la cohérence des résultats du point de vue des unités.
    \end{itemize}
\end{prerequis}
\end{changemargin}
\vspace*{-13mm}
\begin{debat}[Les fractions, ces nombres rompus !] 
    Tout nombre peut s'écrire de différentes façons : ils ont des habillages différents mais ont la même valeur. La façon dont on les écrit permet de pouvoir les comparer.
    Par exemple, la {\bf fraction} $\dfrac68$ possède, entre autres, les représentations suivantes :
    \begin{center}
       {\psset{unit=0.5}
       \begin{pspicture}(-4,-3.7)(4,3.7)  
          \textcolor{B1}{\large
          \multido{\n=5+60,\i=55+60}{6}{\psline(3;\n)(0,0)(3;\i)}
          \multido{\n=30+60,\i=-60+60,\r=120+60}{6}{\psarc(2.719;\n){1.268}{\i}{\r}}
          \pscircle[fillstyle=solid,fillcolor=yellow](0,0){1.3}
          \rput(0,0){\bf $\dfrac68$}
          \rput(2.7;30){75\,\%}
          \rput(2.7;-30){0,75}
          \rput(2.7;150){$\dfrac34$}
          \rput(2.7;-150){$\dfrac{75}{100}$}
          \rput(2.7;90){\pswedge[fillstyle=solid,fillcolor=B3](0,0){0.8}{0}{-90}
                               \pscircle(0,0){0.8}
                               \multido{\n=0+45}{8}{\psline(0,0)(0.8;\n)}}
           \rput(2.7;-90){\psline(-1,0)(1,0)  
           \rput(-1,-0.4){\footnotesize 0}
           \rput(1,-0.4){\footnotesize 1}
           \psline[linecolor=B1,linewidth=1mm](-1,0)(0.5,0) \multido{\r=-1+0.25}{9}{\rput(\r,0){|}}}}
       \end{pspicture}}
    \end{center}
    \bigskip
    \begin{cadre}[B2][J4]
       \begin{center}
          \hrefVideo{https://www.yout-ube.com/watch?v=eawBr43xWf8}{\bf Les fractions}, site Internet {\it Le blob}, série {\it Petits contes mathématiques}.
       \end{center}
    \end{cadre}
 \end{debat}