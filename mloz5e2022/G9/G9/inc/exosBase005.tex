\begin{exercice*}
   \begin{enumerate}
   \item Tracer un triangle $YES$ quelconque.
      \item Placer :
      \begin{itemize}
         \item le milieu $O$ du côté $[ES]$ ;
         \item le milieu $U$ du côté $[YS]$ ;
         \item le milieu $I$ du côté $[YE]$.
      \end{itemize}
      \item Tracer le triangle $OUI$ puis ses hauteurs.
      \item Placer le point $T$ orthocentre du triangle $OUI$.
      \item Trace le cercle de centre $T$ et de rayon
$[TY]$.
      \item Quelle conjecture peut-on écrire ?
   \end{enumerate}
\end{exercice*}

\begin{corrige}
   \ \\ [-5mm]
   \begin{Geometrie}[CoinBG={(0,-2u)}]
      pair Y,E,S,O,U,I,T;
      Y=u*(0.5,0.5);
      E=u*(6.5,1);
      S=u*(4.3,4.5);
      O=iso(E,S);
      U=iso(Y,S);
      I=iso(Y,E);
      T=droite(O,projection(O,U,I)) intersectionpoint droite(U,projection(U,O,I));
      %
      trace polygone(Y,E,S);
      trace polygone(O,U,I);
      drawoptions(withcolor red);
      trace segment(O,projection(O,U,I)) dashed evenly;
      trace codeperp(O,projection(O,U,I),U,5); 
      trace segment(U,projection(U,I,O)) dashed evenly;
      trace codeperp(U,projection(U,I,O),I,5); 
      trace segment(I,projection(I,O,U)) dashed evenly;
      trace codeperp(I,projection(I,O,U),O,5);
      marque_s:=0.3*marque_s;
      trace Codelongueur(Y,iso(Y,E),iso(Y,E),E,1); 
      trace Codelongueur(E,iso(E,S),iso(E,S),S,2); 
      trace Codelongueur(S,iso(S,Y),iso(S,Y),Y,3);
      trace cercles(T,Y);
      drawoptions(withcolor black);
      label.llft(TEX("Y"),Y);
      label.lrt(TEX("E"),E);
      label.top(TEX("S"),S);
      label.urt(TEX("O"),O);
      label.ulft(TEX("U"),U);
      label.bot(TEX("I"),I);
      label.urt(TEX("T"),T);
   \end{Geometrie}

   On peut conjecturer que {\red le centre du cercle circonscrit au triangle $YES$ est l'orthocentre du triangle $OUI$}.
\end{corrige}
