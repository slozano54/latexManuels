\begin{exercice*}
   Rose avait tracé un triangle $AVU$ au crayon et les médiatrices de deux des côtés au stylo. Son voisin Natan a effacé le triangle mais a laissé le point $A$ et les deux médiatrices. Reconstruire le triangle de Rose.
   \begin{center}
      \begin{Geometrie}[CoinHD={(6u,5.5u)}]
         pair A;
         A=u*(1,3.3);
         marque_p:="croix";
         u:=0.5*u;
         pointe(A);
         u:=2*u;
         label.top(TEX("A"),A);
         trace droite((0,0.5u),(6u,4u));
         trace droite((4u,0),(2.5u,4u));
      \end{Geometrie}
   \end{center}
\end{exercice*}

\begin{corrige}
   Il suffit de construire les points $U$ et $V$ symétriques du point $A$ par rapport aux deux droites déjà tracées.
   \begin{center}
      \begin{Geometrie}[CoinHD={(6u,5.5u)}]
         pair A;
         A=u*(1,3.3);
         marque_p:="croix";
         u:=0.5*u;
         pointe(A);
         u:=2*u;
         label.top(TEX("A"),A);
         trace droite((0,0.5u),(6u,4u));
         trace droite((4u,0),(2.5u,4u));
         drawoptions(withcolor red);
         pair U,V,M[];
         V=symetrie(A,(0,0.5u),(6u,4u));
         U=symetrie(A,(4u,0),(2.5u,4u));
         trace polygone(A,U,V);
         M0=iso(A,U);
         M1=iso(A,V);
         marque_s:=0.3*marque_s;
         trace Codelongueur(A,M0,M0,U,1);
         trace Codelongueur(A,M1,M1,V,2);
         trace codeperp((4u,0),M0,A,5);
         trace codeperp(A,M1,(0,0.5u),5);
         label.top(TEX("U"),U);
         label.bot(TEX("V"),V);
      \end{Geometrie}
   \end{center}
\end{corrige}