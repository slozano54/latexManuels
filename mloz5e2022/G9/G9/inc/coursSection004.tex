\begin{changemargin}{0mm}{-15mm}
    \section{Hauteurs du triangle}
    \begin{definition}
        Dans un triangle, la hauteur \textbf{issue d'un sommet} est la droite qui passe par ce sommet et qui est perpendiculaire \textbf{au côté opposé}.
    \end{definition}
    \begin{center}
        \begin{Geometrie}[CoinHD={(6u,5u)}]
            pair A,B,C;
            pair H,x;
            path d,d';
            A=u*(0.5,2);
            B-A=u*(1,2);
            C-A=u*(4,-1);
            d=perpendiculaire(B,C,A);
            d'=droite(B,C);
            H=d intersectionpoint d';
            x=symetrie(A,H);
            trace polygone(A,B,C) withpen pencircle scaled 1.5bp;
            trace d;
            trace codeperp(B,H,A,5);
            trace appelation(A,x,-2mm,btex hauteur issue de A etex);
            label.lft(TEX("A"),A);
            label.top(TEX("B"),B);
            label.lrt(TEX("C"),C);
            label.top(TEX("x"),x);
        \end{Geometrie}
        \hspace*{0.1\linewidth}
        \begin{Geometrie}[CoinHD={(5u,5u)}]
            pair E,F,G;
            pair H;
            path d,d';
            E=u*(0.5,0.5);
            F-E=u*(1.5,3.5);
            G-E=u*(4,3.5);
            trace  polygone(E,F,G) withpen pencircle scaled 1.5bp;
            d=droite(F,G);
            d'=perpendiculaire(F,G,E);
            H= d intersectionpoint d';
            trace d dashed evenly;
            trace d';
            trace codeperp(E,H,G,5);
            trace appelation(E,H,2mm,btex hauteur issue de E etex);
            label.top(TEX("F"),F);
            label.top(TEX("G"),G);
            label.lrt(TEX("E"),E);
        \end{Geometrie}
    \end{center}
    \begin{remarque}
        On dit aussi que \textbf{$(Ax)$} est la hauteur \textbf{relative} au côté \textbf{$[BC]$}.
    \end{remarque}
    \begin{propriete}[\admise]
        Si on trace les trois hauterus d'un triangle, alors elles sont concourantes (en un point qui s'appelle \textbf{orthocentre}).
    \end{propriete}
    
    \begin{Geometrie}[CoinHD={(10u,7u)},CoinBG={(0,-0.5u)}]
        pair A,B,C;
        pair H[];
        A=u*(0.5,2);
        B-A=u*(2,4);
        C-A=u*(8,-2);
        H0=projection(A,B,C);
        H1=projection(B,C,A);
        H2=projection(C,A,B);
        H3=droite(A,H0) intersectionpoint droite(B,H1);
        trace polygone(A,B,C) withpen pencircle scaled 1.5bp;
        trace droite(A,H0) dashed evenly;
        trace codeperp(B,H0,A,5);
        trace droite(B,H1) dashed evenly;
        trace codeperp(C,H1,B,5);
        trace droite(C,H2) dashed evenly;
        trace codeperp(A,H2,C,5);
        label.ulft(TEX("A"),A);
        label.ulft(TEX("B"),B);
        label.bot(TEX("C"),C);
        label.rt(TEX("Orthocentre"),H3 shifted (0.1u,0));
    \end{Geometrie}
    \hspace*{0.1\linewidth}
    \begin{Geometrie}[CoinHD={(5u,7u)}]
        pair E,F,G;
        pair H[];
        E=u*(0.5,0.5);
        F-E=u*(1.5,3.5);
        G-E=u*(4,3.5);
        H0=projection(E,F,G);
        H1=projection(F,G,E);
        H2=projection(G,E,F);
        H3=droite(E,H0) intersectionpoint droite(F,H1);
        trace droite(F,G);
        trace droite(E,F);
        trace  polygone(E,F,G) withpen pencircle scaled 1.5bp;
        trace droite(E,H0) dashed evenly;
        trace codeperp(F,H0,E,5);
        trace droite(F,H1) dashed evenly;
        trace codeperp(G,H1,F,5);
        trace droite(G,H2) dashed evenly;
        trace codeperp(E,H2,G,5);
        label.top(TEX("F"),F);
        label.top(TEX("G"),G);
        label.lrt(TEX("E"),E);
        label.urt(TEX("Orthocentre"),H3 shifted (0.1u,0));
    \end{Geometrie}
    
\end{changemargin}

