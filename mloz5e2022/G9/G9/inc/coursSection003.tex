\begin{changemargin}{0mm}{-15mm}
    \section{Médianes du triangle}
    \begin{minipage}{0.6\linewidth}
        \begin{propriete}
            Dans un triangle, une \textbf{médiane} est une droite qui passe par un sommet et qui coupe le triangle en deux parties d'aires égales.
        \end{propriete}
    \end{minipage}
    \begin{minipage}{0.4\linewidth}
        \begin{Geometrie}[CoinHD={(6.5u,4u)}]
            pair A,B,C,I;
            B=u*(0.5,0.5);
            A-B=u*(1,3);
            C-B=u*(5,0);
            I=iso(B,C);
            trace polygone(A,B,C);
            marque_s:=0.3*marque_s;
            trace Codelongueur(B,I,I,C,2);
            trace droite(A,I) dashed evenly withcolor red;
            label.urt(TEX("A"),A);
            label.llft(TEX("B"),B);
            label.lrt(TEX("C"),C);
            label.bot(TEX("I"),I);
        \end{Geometrie}
    \end{minipage}
    \begin{propriete}[\admise]
        Si dans un triangle, une droite passe par un sommet et par le milieu du côté opposé à ce sommet, alors c'est une des médianes du triangle
    \end{propriete}
    \begin{preuve}

        \begin{minipage}{0.6\linewidth}
            Il s'agit de montrer que les triangles $ABI$ et $ACI$ ont la même aire.\\
            $\text{Aire}_{ABI}=\dfrac{BI\times AH}{2}$ et $\text{Aire}_{ACI}=\dfrac{CI\times AH}{2}$\\
            or $I$ est le mileiu de $[BC]$ donc $BI=CI$ \\
            d'où $\text{Aire}_{ABI}=\dfrac{BI\times AH}{2}=\dfrac{CI\times AH}{2}=\text{Aire}_{ACI}$ $\square$
        \end{minipage}
        \begin{minipage}{0.4\linewidth}
            \begin{Geometrie}[CoinHD={(6.5u,4u)}]
                pair A,B,C,I,H;
                B=u*(0.5,0.5);
                A-B=u*(1,3);
                C-B=u*(5,0);
                I=iso(B,C);
                H=projection(A,B,C);
                trace polygone(A,B,C);
                marque_s:=0.3*marque_s;
                trace Codelongueur(B,I,I,C,2);
                trace droite(A,I) dashed evenly withcolor red;
                trace droite(A,H) dashed evenly withcolor DarkGreen;
                trace codeperp(A,H,B,5);
                label.urt(TEX("A"),A);
                label.llft(TEX("B"),B);
                label.lrt(TEX("C"),C);
                label.bot(TEX("I"),I);
            \end{Geometrie}
        \end{minipage}
    \end{preuve}
    \begin{minipage}{0.6\linewidth}
        \begin{propriete}[Hors programme \admise]
            Si dans un triangle, on trace les trois médianes, alors elles sont concourantes en un point appelé \textbf{centre de gravité} du triangle, situé aux $\dfrac23$ de chaque médiane en partant des sommets.
        \end{propriete}
    \end{minipage}
    \begin{minipage}{0.4\linewidth}
        \begin{Geometrie}[CoinHD={(6.5u,4u)}]
            pair A[],B[],C[],G;
            A0=u*(0.5,0.5);
            B0-A0=u*(5,1);
            C0-A0=u*(1.5,3);
            A1=iso(B0,C0);
            B1=iso(A0,C0);
            C1=iso(A0,B0);
            G=iso(A0,B0,C0);
            trace polygone(A0,B0,C0);
            marque_s:=0.3*marque_s;
            trace Codelongueur(A0,C1,C1,B0,2);
            trace Codelongueur(B0,A1,A1,C0,1);
            trace Codelongueur(C0,B1,B1,A0,4);
            trace droite(A0,A1) dashed evenly withcolor red;
            trace droite(B0,B1) dashed evenly withcolor red;
            trace droite(C0,C1) dashed evenly withcolor red;
            label.lft(TEX("A"),A0);
            label.bot(TEX("B"),B0);
            label.urt(TEX("C"),C0);
            label.top(TEX("A'"),A1);
            label.ulft(TEX("B'"),B1);
            label.lrt(TEX("C'"),C1);
            label.bot(TEX("G"),G);
        \end{Geometrie}
    \end{minipage}
\end{changemargin}

