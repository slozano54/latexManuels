\begin{changemargin}{0mm}{-15mm}
    \section{Bissectrices du triangle}

    \begin{definition}
        La \textbf{bissectrice} d'un angle est la demi-droite dont l'origine est le sommet de l'angle et qui partage cet angle en deux angles adjacents de même mesure.
    \end{definition}
    \begin{propriete}[\admise]
        L'axe de symétrie d'un triangle isocèle est aussi la bissectrice de l'angle au sommet principal.
    \end{propriete}
    \begin{propriete}[\admise]
        Les angles à la base d'un triangle isocèle sont égaux.
    \end{propriete}
    \begin{propriete}[\admise]
        Les trois angles d'un triangle équilatéral sont égaux.
    \end{propriete}
    \begin{Geometrie}[CoinHD={(5u,6u)}]
        pair O,x,y,z;
        path cO;
        O=u*(0.5,3);
        cO=cercles(O,2u);
        x=pointarc(cO,20);
        y=pointarc(cO,-20);
        z=pointarc(cO,0);
        trace droite(O,x);
        trace droite(O,y);
        trace droite(O,z) dashed dashpattern(on6 off3 on3 off3) withcolor red;
        marque_a:=1.2*marque_a;
        trace marqueangle(z,O,x,2);
        trace marqueangle(y,O,z,2);
        label.top(TEX("bissectrice"),z shifted(u,0));
        label.top(TEX("x"),1.2[O,x]);
        label.bot(TEX("y"),1.2[O,y]);
        label.top(TEX("O"),O);
    \end{Geometrie}
    \hspace*{10mm}
    \begin{Geometrie}[CoinHD={(6u,6u)}]
        pair O,x,y,z;
        path cO;
        O=u*(2.5,5.5);
        cO=cercles(O,4.5u);
        x=pointarc(cO,-70);
        y=pointarc(cO,250);
        z=iso(x,y);
        trace polygone(O,x,y);
        trace droite(O,z) dashed dashpattern(on6 off3 on3 off3) withcolor red;
        trace codeperp(O,z,y,5) withcolor red;
        marque_s:=marque_s/2;
        trace Codelongueur(y,z,z,x,4);
        trace Codelongueur(y,O,O,x,3);
        trace marqueangle(y,O,z,2);
        trace marqueangle(z,O,x,2);
    \end{Geometrie}
    \hspace*{10mm}
    \begin{Geometrie}[CoinHD={(6u,6u)}]
        pair O,x,y,z;
        path cO;
        O=u*(2.5,5.5);
        cO=cercles(O,4u);
        x=pointarc(cO,-75);
        y=pointarc(cO,255);
        z=iso(x,y);
        trace polygone(O,x,y);
        marque_s:=marque_s/2;
        trace Codelongueur(y,O,O,x,3);
        trace marqueangle(x,y,O,4);
        trace marqueangle(O,x,y,4);
    \end{Geometrie}
    \hspace*{10mm}
    \begin{Geometrie}[CoinHD={(6u,6u)}]
        pair A,B,C;
        A=u*(1,1);
        B-A=u*(4,2);
        C=rotation(B,A,60);
        marque_s:=marque_s/2;
        trace Codelongueur(A,B,B,C,C,A,2);
        trace marqueangle(C,B,A,4);
        trace marqueangle(B,A,C,4);
        trace marqueangle(A,C,B,4);
        trace polygone(A,B,C);
    \end{Geometrie}

    \begin{remarque}
        Tu dois aussi connaître deux méthodes pour construire la bissectrice d'un angle :
        \begin{itemize}
            \item avec le rapporteur.
            \item avec le compas.
        \end{itemize}
        \begin{center}
            \textbf{Avec le compas}

            \begin{Geometrie}[CoinHD={u*(7,4)}]
                trace feuillet;
                pair A,B,C,D[],E[];
                path cA;
                A=u*(0.5,2);
                B-A=u*(2,0);
                C-A=u*(5,0);
                cA=cercles(A,B);
                D0=pointarc(cA,20);
                D1=pointarc(cA,45);
                E0=pointarc(cA,-20);
                E1=pointarc(cA,-45);
                trace demidroite(A,D0);
                trace demidroite(A,E0);
                trace (subpath(7,9) of cA) dashed evenly;
                trace compas(A,D1,1);
            \end{Geometrie}
            \begin{Geometrie}[CoinHD={u*(7,4)}]
                trace feuillet;
                pair A,B,C[],D,E;
                path cA,cD;
                A=u*(0.5,2);
                B-A=u*(2,0);
                C0-A=u*(5,0);
                cA=cercles(A,B);
                D=pointarc(cA,20);
                E=pointarc(cA,-20);
                cD=cercles(D,C0);
                C1=pointarc(cD,-5);
                trace demidroite(A,D);
                trace demidroite(A,E);
                trace (subpath(7,9) of cA) dashed evenly;
                trace coupdecompas(D,C0,10) dashed evenly;
                trace coupdecompas(E,C0,10) dashed evenly;
                trace demidroite(A,C0) dashed dashpattern (on6 off3 on3 off3) withcolor red;
                trace compas(D,C1,1);
            \end{Geometrie}
        \end{center}
    \end{remarque}

    \begin{propriete}[Bissectrices du triangle \admise]
        Si on trace les bissectrices des trois angles d'un triangle, alors elles sont concourantes (en un point qui est le centre de son cercle inscrit).
    \end{propriete}

\end{changemargin}


