\begin{exercice*}
   Construction de hauteurs.
   \begin{enumerate}
      \item Construire un triangle $BLE$ puis tracer :
      \begin{itemize}
         \item en bleu, la hauteur issue du sommet $E$ ;
         \item en noir, la hauteur issue du sommet $B$ ;
         \item en rouge, la hauteur relative à $[BE]$.
      \end{itemize}
       \item Quelle remarque peut-on faire ?
   \end{enumerate} 
\end{exercice*}

\begin{corrige}
   \ \\ [-5mm]
   \begin{Geometrie}
      pair B,L,E;
      B=u*(0.5,1);
      L=u*(6.5,2);
      E=u*(4.5,5);
      trace polygone(B,L,E);
      trace segment(E,projection(E,B,L)) dashed evenly withcolor blue;
      trace codeperp(L,projection(E,B,L),E,5) withcolor blue;
      trace segment(B,projection(B,L,E)) dashed evenly;
      trace codeperp(E,projection(B,L,E),B,5);
      trace segment(L,projection(L,E,B)) dashed evenly withcolor red;
      trace codeperp(B,projection(L,E,B),L,5) withcolor red;
      label.llft(TEX("B"),B);
      label.lrt(TEX("L"),L);
      label.top(TEX("E"),E);
   \end{Geometrie}

   On remarque que {\red les hauteurs sont concourantes} en un point appelé l'orthocentre du triangle.
\end{corrige}
