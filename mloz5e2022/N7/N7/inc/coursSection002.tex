\section{Soustraction}

\begin{definition}[Nombres opposés]
    Deux nombres relatifs sont dits \textbf{opposés} quand leur somme vaut zéro.\\
    On note $-a$ l'opposé du nombre $a$.
\end{definition}

\begin{remarque}
    $-a$ peut être positif ! Par exemple lorsque $a$ vaut $-4$.
\end{remarque}

\begin{propriete}[\admise]
    \vspace*{3mm}
    \begin{itemize}
        \item Si un nombre relatif est {\bfseries positif} alors son opposé est {\bfseries négatif}.
        \item Si un nombre relatif est {\bfseries négatif} alors son opposé est {\bfseries positif}.
    \end{itemize}
\end{propriete}

\begin{exemples*1}
    \begin{changemargin}{0mm}{-10mm}
        \begin{enumerate}
            \item $\num{-5.28}$ est l'opposé de $+\num{5.28}$ mais $+\num{5.28}$ est aussi l'opposé de $\num{-5.28}$ en particulier $-(\num{-5.28})=\num{5.28}$
            \item si $a=+\num{2.14}$ alors $-a=\num{-2.14}$ et si $a=\num{-7.81}$ alors $-a=+\num{7.81}$
        \end{enumerate}
    \end{changemargin}
    \vspace*{-10mm}
\end{exemples*1}

\begin{propriete}[Vision géométrique \admise]
    Si deux nombres relatifs sont opposés alors ils correspondent à des points symétriques par rapport à l'origine.
\end{propriete}

\begin{exemples*1}
    \phantom{rrr}
    
    \bigskip
    \begin{changemargin}{0mm}{-15mm}    
        \begin{minipage}{0.55\linewidth}
        $-7$ et $+7$ sont opposés.
        
        \smallskip
        \pspicture(-4,-.5)(4,1)
            \psline{->}(-4,.5)(4,.5)
            \rput(0,0.5){+}
            \uput[90](0,0.5){O} 
            \uput[-90](0,0.5){0} 
            \rput(-3.5,.5){+}
            \uput[90](-3.5,0.5){A} 
            \uput[-90](-3.5,0.5){-7} 
            \rput(3.5,.5){+}
            \uput[90](3.5,0.5){B} 
            \uput[-90](3.5,0.5){+7} 
            \psline[linecolor=red]{<->}(-3.5,-0.2)(0,-0.2)
            \uput[-90](-1.5,-0.2){7 unités} 
            \psline[linecolor=red]{<->}(0,-0.2)(3.5,-0.2)
            \uput[-90](1.5,-0.2){7 unités} 
        \endpspicture
        
        \bigskip
        $A$ et $B$ sont symétriques par rapport à $O$.
        \end{minipage}        
        \hfill
        \begin{minipage}{0.55\linewidth}
        $\dfrac13$ et $-\dfrac13$ sont opposés.
        
        \smallskip
        \pspicture(-4,-.5)(4,1)
            \psline{->}(-4,.5)(4,.5)
            \multirput(-3,0.5)(1,0){7}{+}
            \uput[90](0,0.5){O} 
            \uput[-90](0,0.5){0} 
            \uput[-90](3,0.2){+1} 
            \uput[90](-1,0.5){C} 
            \uput[-90](-1,0.5){$-\frac13$} 
            \uput[90](1,0.5){D} 
            \uput[-90](1,0.5){$\frac13$} 
            \psline[linecolor=red]{<->}(-1,-0.2)(0,-0.2)
            \psline[linecolor=red]{<->}(0,-0.2)(1,-0.2)
        \endpspicture

        \smallskip
        $C$ et $D$ sont symétriques par rapport à $O$.
        \end{minipage}
    \end{changemargin}
\end{exemples*1}

\begin{propriete}[Soustraction \admise]
    Si on soustrait un nombre relatif alors cela revient à additionner son nombre opposé.
\end{propriete}

\begin{exemples*1}
    \begin{itemize}
        \item l'opposé de $+\num{8.2}$ est $\num{-8.2}$ donc soustraire $+\num{8.2}$ revient à ajouter $\num{-8.2}$.
        $$(+14)-(+\num{8.2})=(+14)+(-\num{8.2})=+\num{5.8}$$
        \item l'opposé de $\num{-8.2}$ est $+\num{8.2}$ donc soustraire $\num{-8.2}$ revient à ajouter $+\num{8.2}$.
        $$(\num{-17.2})-(\num{-8.2})=(\num{-17.2})+(+\num{8.2})=-9$$
        \item $+\dfrac54 -(+\dfrac34)=+\dfrac54 +(-\dfrac34)=-\dfrac14$
    \end{itemize}
\end{exemples*1}