\begin{exercice*}
   Dans le monde entier, les heures locales sont fixées par rapport à l'heure universelle (UT). \\
   Paris est à UT, New York est à UT $-\uh{6}$ et New Delhi est à UT $+\uh{4}\,30$.
   \begin{enumerate}
      \item Lucie, qui est à Montpellier, appelle à New York à \uh{20} et téléphone pendant trois quarts d'heure. Quelle heure est-il à New York à la fin de l'appel ?
      \item Après ce coup de téléphone, Lucie peut-elle raisonnablement appeler à New Delhi ?
   \end{enumerate}
\end{exercice*}

\begin{corrige}
   Montpellier est à la même UT que Paris. \\
   \begin{enumerate}
      \item \uh{20}\,\umin{00} \quad $\xrightarrow{+\umin{45}}$ \quad \uh{20}\,\umin{45}. \\
         Donc, Lucie termine son appel à \uh{20}\,45. \\ [1mm]
         \uh{20}\,\umin{45} \quad $\xrightarrow{-\uh{6}}$ \quad \uh{14}\,\umin{45}. \\
         À cette heure, il est {\red \uh{14}\,45} à New-York.
      \item Pour New Delhi, il faut ajouter \uh{4}\,30 à l'heure de Paris, on peut décomposer ainsi : \\ [1mm]
         \uh{20}\,\umin{45} \quad $\xrightarrow{+\uh{4}}$ \quad \uh{24}\,\umin{45} = \uh{0}\,\umin{45}\\ [1mm]
         \uh{0}\,\umin{45} \quad $\xrightarrow{+\umin{30}}$ \quad \uh{1}\,\umin{15} \\ [1mm]
         Il est \uh{1}\,15 du matin donc, {\red il n'est pas raisonnable d'appeler en Inde !}
   \end{enumerate}
\end{corrige}

