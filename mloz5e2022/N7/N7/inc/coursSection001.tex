\section{Addition}

\begin{definition}[Distance à zéro]
    La \textbf{distance à zéro} d'un nombre relatif, c'est la distance qui le sépare de zéro!\\
    Une distance est \textbf{toujours positive}.
\end{definition}

\begin{propriete}[Addition \admise]
    Si deux nombres relatifs sont de \textbf{même signe} et qu'ils sont \textbf{positifs} alors la somme est \textbf{positive} et on calcule sa distance à zéro en additionnant les distances à zéro.
\end{propriete}

\begin{propriete}[Addition \admise]
    Si deux nombres relatifs sont de \textbf{même signe} et qu'ils sont \textbf{négatifs} alors la somme est \textbf{négative} et on calcule sa distance à zéro en additionnant les distances à zéro.
\end{propriete}

\begin{propriete}[Addition \admise]
    Si deux nombres relatifs sont de \textbf{signes contraires} alors la somme est du \textbf{signe du plus éloigné de zéro} et on calcule sa distance à zéro en calculant la différence positive des distances à zéro.
\end{propriete}

\begin{exemples*1}
    \begin{itemize}        
        \item[] $A=(\textcolor{red}{+}\num{17.7})+ (\textcolor{red}{+}\num{1.5})=\textcolor{red}{+}(\num{17.7}+\num{1.5})=\psshadowbox{\textcolor{red}{+}\num{19.2}}$
        \item[] $B=(\textcolor{DarkGreen}{-}\num{23.6})+ (\textcolor{DarkGreen}{-}\num{7.2})=\textcolor{DarkGreen}{-}(\num{23.6}+\num{7.2})=\psshadowbox{\textcolor{DarkGreen}{-}\num{30.8}}$
        \item[] $C=(\textcolor{red}{+}\num{14.3})+ (\textcolor{DarkGreen}{-}\num{4.36})=\textcolor{red}{+}(\num{14.3}-\num{4.36})=\psshadowbox{\textcolor{red}{+}\num{9.94}}$
        
        \smallskip
        Pour le C, plus loin de zéro est le nombre positif donc la somme est positive.        
        \item[] $D=(\textcolor{DarkGreen}{-}\num{11.2})+ (\textcolor{red}{+}\num{7.6})=\textcolor{DarkGreen}{-}(\num{11.2}-\num{7.6} )=\psshadowbox{\textcolor{DarkGreen}{-}\num{3.6}}$
        
        \smallskip
        Pour le D, le plus loin de zéro est le nombre négatif donc la somme est négative.
        
        \smallskip
        \begin{minipage}{0.4\linewidth}
        \item[] $E=(+\num{14.9})+(\num{-5.1})+(\num{1.75})$
        \item[] $E=(+\num{9.8})+(\num{1.75})$
        \smallskip
        \item[] \psshadowbox{$E=+\num{11.55}$}
        \end{minipage}
        \begin{minipage}{0.25\linewidth}
        \item[] $F=(-\dfrac38)+(-\dfrac78)$
        \item[] $F=-\dfrac{10}{8}$
        \smallskip
        \item[] \psshadowbox{$F=-\dfrac{5}{4}$}
        \end{minipage}
        \begin{minipage}{0.25\linewidth}
        \item[] $G=(-\dfrac37)+(+\dfrac{5}{14})$
        \item[] $G=(-\dfrac{6}{14})+(+\dfrac{5}{14})$
        \smallskip
        \item[] \psshadowbox{$G=-\dfrac{1}{14}$}
        \end{minipage}
    \end{itemize}
\end{exemples*1}
