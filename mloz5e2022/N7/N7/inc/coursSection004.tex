\section{Distance entre deux points}
% \proprNum{(admise)}{On veut calculer la distance entre deux points $A$ et $B$ d'une droite graduée}{
% \begin{mylist}
% \item On compare les abscisses de $A$ et de $B$.
% \item On calcule la différence entre la plus grande abscisse et de la plus petite (dans cet ordre).
% \end{mylist}
% }

% \Remarques{
% \begin{mylist}
% \item[$\leadsto$] Une distance est TOUJOURS POSITIVE.
% \item[$\leadsto$] La distance entre $A$ et $B$ se note $AB$ ou $BA$.
% \end{mylist}
% }

% \Exemples{}{
% Soient les points $A(+2,7)$, $B(+3,1)$, $C(-4,2)$, $D(-11,7)$ sur un axe gradué en cm.
% \begin{itemize}
% \begin{minipage}{6cm}
% \item[] $AB=(+3,1)-(+2,7)$
% \item[] $AB=3,1-2,7$
% \item[] \psshadowbox{$AB=0,4$cm}
% \end{minipage}
% \begin{minipage}{6cm}
% \item[] $CD=(-4,2)-(-11,7)$
% \item[] $CD=-4,2+11,7$
% \item[] \psshadowbox{$CD=......$cm}
% \end{minipage}
% \begin{minipage}{6cm}
% \item[] $BC=(+3,1)-(-4,2)$
% \item[] $BC=3,1+4,2$
% \item[] \psshadowbox{$BC=......$cm}
% \end{minipage}
% \end{itemize}
% }
