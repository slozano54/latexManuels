\section{Distance entre deux points}
\begin{propriete}[\admise]
    Si on veut calculer la distance entre deux points $A$ et $B$ d'une droite graduée alors
    \begin{itemize}
        \item On compare les abscisses de $A$ et de $B$.
        \item On calcule la différence entre la plus grande abscisse et de la plus petite (dans cet ordre).
    \end{itemize}
\end{propriete}

\begin{remarques}
    \begin{itemize}
        \item Une distance est {\bfseries toujours positive}.
        \item La distance entre $A$ et $B$ se note $AB$ ou $BA$.
    \end{itemize}
\end{remarques}

\begin{exemples*1}
    Soient les points $A(+\num{2.7})$, $B(+\num{3.1})$, $C(-\num{4.2})$, $D(-\num{11.7})$ sur un axe gradué en cm.

    \bigskip
    \begin{itemize}
        \begin{minipage}{0.3\linewidth}
        \item[] $AB=(+\num{3.1})-(+\num{2.7})$
        \item[] $AB=\num{3.1}-\num{2.7}$
        \item[] \psshadowbox{$AB=\Lg{0.4}$}
        \end{minipage}
        \begin{minipage}{0.3\linewidth}
        \item[] $CD=(-\num{4.2})-(-\num{11.7})$
        \item[] $CD=-\num{4.2}+\num{11.7}$
        \item[] \psshadowbox{$CD=......$\Lg[cm]{}}
        \end{minipage}
        \begin{minipage}{0.3\linewidth}
        \item[] $BC=(+\num{3.1})-(-\num{4.2})$
        \item[] $BC=\num{3.1}+\num{4.2}$
        \item[] \psshadowbox{$BC=......$\Lg[cm]{}}
        \end{minipage}
    \end{itemize}
\end{exemples*1}
