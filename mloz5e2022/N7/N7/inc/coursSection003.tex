\section{\'{E}critures simplifiées}
% \proprNumBis{Conventions}{
% Pour simplifier les écritures des additions et des soustractions des nombres relatifs.
% \begin{enumerate}
% \item On ne mettra plus les parenthèses autours des nombres relatifs
% \begin{center}
% $(+4)-(-8)+(-5)+(+2)$ s'écrira $+4--8+-5++2$
% \end{center}
% \item On n'écrira plus le signe + devant le premier terme lorsqu'il est positif.
% \begin{center}
% $+4--8+-5++2$ devient $4--8+-5++2$
% \end{center}
% \item Lorsque deux signes se suivent, on appliquera la règle des signes:
% 	\begin{enumerate}
% 	\item \textbf{L'opposé d'un nombre négatif est un nombre positif} \\ donc au lieu de - - on écrira +.
% 	\item \textbf{L'opposé d'un nombre positif est un nombre négatif} \\ donc au lieu de -+ on écrira -.
% 	\item \textbf{Ajouter un nombre négatif revient à soustraire son opposé} \\ donc au lieu de +- on écrira -.
% 	\item \textbf{Le symbole + est facultatif devant un nombre positif} \\ donc au lieu de ++ on écrira +.
% 	\end{enumerate}
% \begin{center}
% $4--8+-5++2$ s'écrira $4+8-5+2$
% \end{center}
% \end{enumerate}
% \begin{center}
% \psshadowbox{\underline{Bilan} : L'écriture simplifiée de $(+4)-(-8)+(-5)+(+2)$ est $4+8-5+2$.}
% \end{center}
% }
% \Exemples{Simplifie les expressions suivantes}{
% \begin{itemize}
% \item[]$A=(+4)+(-14)-(+3)+(-3)-(-4)+(+5)$
% \item[]$A=+4+-14-+3+-3--4++5$  \textbf{(règle 1)}
% \item[]$A=4+-14-+3+-3--4++5$  \textbf{(règle 2)}
% \item[]\psshadowbox{$A=4-14-3-3+4+5$} \textbf{(règle 3)}
% \vspace{0.25cm}
% \item[]$B=(+3,5)-(+50,7)+(+60,2)-(-65,7)+(-99,9)$
% \item[]$C=(-\dfrac35)+(+\dfrac5{10})+(-\dfrac{16}{78})-(+\dfrac84)$
% \end{itemize}
% }
