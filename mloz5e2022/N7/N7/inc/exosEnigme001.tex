% Les enigmes ne sont pas numérotées par défaut donc il faut ajouter manuellement la numérotation
% si on veut mettre plusieurs enigmes
% \refstepcounter{exercice}
% \numeroteEnigme
\begin{enigme}[Décryptage]
    \bigskip
    Décrypter les codes suivants utilisé par l'agent Zérozérossette sachant que chaque symbole correspond à un nombre entier relatif et que la somme de chaque ligne et de chaque colonne est indiquée en bout de celle-ci. \\
    \medskip
    \begin{center}
       {\renewcommand{\arraystretch}{1.5}
       \begin{tabular}{*{7}{>{\centering\arraybackslash}p{0.3cm}}}
          \Large\ding{101} & $+$ & \Large\ding{101} & $+$ & \Large\ding{40} & $=$ & \large 16 \\
          $+$ & & $+$ & & $+$ & & \\
          \Large\ding{168} & + & \Large\ding{168} & + & \Large\ding{168} & $=$ & \large 9 \\
          $+$ & & $+$ & & $+$ & & \\
          \Large\ding{40} & + & \Large\ding{52} & + & \Large\ding{168} & $=$ & \large 18 \\
          $=$ & & $=$ & & $=$ & & \\
          \large 14 & & \large 17 & & \large 12 & & \\
       \end{tabular}
       \hspace*{4cm}
       \begin{tabular}{*{7}{>{\centering\arraybackslash}p{0.3cm}}}
          \Large\ding{101} & $+$ & \Large\ding{101} & $+$ & \Large\ding{101} & $=$ & \large $-3$ \\
          $+$ & & $+$ & & $+$ & & \\
          \Large\ding{40} & + & \Large\ding{168} & + & \Large\ding{101} & $=$ & \large $-3$ \\
          $+$ & & $+$ & & $+$ & & \\
          \Large\ding{52} & + & \Large\ding{52} & + & \Large\ding{168} & $=$ & \large 6 \\
          $=$ & & $=$ & & $=$ & & \\
          \large 6 & & \large 0 & & \large $-6$ & & \\
       \end{tabular}
 
       \ \\ [10mm]
       
       \begin{tabular}{|p{1.8cm}|p{1.8cm}|}
          \hline
          \Large\ding{101} = & \Large\ding{40} = \\
          \hline
          \Large\ding{168} = & \Large\ding{52} = \\
          \hline
       \end{tabular}
       \hspace*{5cm}
       \vspace*{1cm}
       \begin{tabular}{|p{1.8cm}|p{1.8cm}|}
          \hline
          \Large\ding{101} = & \Large\ding{40} = \\
          \hline
          \Large\ding{168} = & \Large\ding{52} = \\
          \hline
       \end{tabular}
 
       \ \\ [5mm]
       
       \begin{tabular}{*{9}{>{\centering\arraybackslash}p{0.3cm}}}
          \Large\ding{101} & $+$ & \Large\ding{40} & $+$ & \Large\ding{40} & $+$ & \Large\ding{101} & $=$ & \large 2 \\
          $+$ & & $+$ & & $+$ & & $+$ & \\
          \Large\ding{40} & $+$ & \Large\ding{40} & $+$ & \Large\ding{40} & $+$ & \Large\ding{168} & $=$ & \large 9 \\
          $+$ & & $+$ & & $+$ & & $+$ & \\
          \Large\ding{40} & + & \Large\ding{168} & + & \Large\ding{52} & $+$ & \Large\ding{36} & $=$ & \large $-7$ \\
          $=$ & & $=$ & & $=$ & & $=$ & \\
          \large 3 & & \large 7 & & \large $-1$ & & \large $-5$ & & \\ 
       \end{tabular}
       \hspace*{2cm}
       \begin{tabular}{*{9}{>{\centering\arraybackslash}p{0.3cm}}}
          \Large\ding{52} & $+$ & \Large\ding{52} & $+$ & \Large\ding{168} & $+$ & \Large\ding{168} & $=$ & \large \!\!$-158$ \\
          $+$ & & $+$ & & $+$ & & $+$ & \\
          \Large\ding{40} & $+$ & \Large\ding{52} & $+$ & \Large\ding{36} & $+$ & \Large\ding{36} & $=$ & \large \!$-19$ \\
          $+$ & & $+$ & & $+$ & & $+$ & \\
          \Large\ding{101} & + & \Large\ding{52} & + & \Large\ding{40} & $+$ & \Large\ding{52} & $=$ & \large \!$-86$ \\
          $=$ & & $=$ & & $=$ & & $=$ & \\
          \large \!$-32$ & & \large \!\!$-162$ & & \large \!$-37$ & & \large \!$-32$ & & \\ 
       \end{tabular}
 
      \ \\ [10mm]
      
       \begin{tabular}{|p{1.8cm}|p{1.8cm}|p{1.8cm}|}
          \hline
          \Large\ding{101} = & \Large\ding{40} = & \Large\ding{168} = \\
          \hline
          \Large\ding{36} = & \Large\ding{52} = \\
          \cline{1-2}
       \end{tabular}
       \hspace*{2cm}
       \begin{tabular}{|p{1.8cm}|p{1.8cm}|p{1.8cm}|}
          \hline
          \Large\ding{101} = & \Large\ding{40} = & \Large\ding{168} = \\
          \hline
          \Large\ding{36} = & \Large\ding{52} = \\
          \cline{1-2}
       \end{tabular}}
    \end{center}
 \end{enigme}
% Pour le corrigé, il faut décrémenter le compteur, sinon il est incrémenté deux fois
% \addtocounter{exercice}{-1}

\begin{corrige}
    {\renewcommand{\arraystretch}{2}
    Premier code. \\ [2mm]
    \begin{tabular}{|p{1.8cm}|p{1.8cm}|}
       \hline
       \large\ding{101} = \textcolor{red}{5} & \large\ding{40} = \textcolor{red}{6} \\
       \hline
       \large\ding{168} = \textcolor{red}{3} & \large\ding{52} = \textcolor{red}{9} \\
       \hline
    \end{tabular}
    }
    \Coupe
    {\renewcommand{\arraystretch}{2}  
    \ \\ [5mm]
    Deuxième code. \\ [2mm]
    \begin{tabular}{|p{1.8cm}|p{1.8cm}|}
       \hline
       \large\ding{101} = \textcolor{red}{$-1$} & \large\ding{40} = \textcolor{red}{2} \\
       \hline
       \large\ding{168} = \textcolor{red}{$-4$} & \large\ding{52} = \textcolor{red}{5}\\
       \hline
    \end{tabular}
    \ \\ [5mm]
    Troisième code. \\ [2mm]
    \begin{tabular}{|p{1.8cm}|p{1.8cm}|p{1.8cm}|}
       \hline
       \large\ding{101} = \textcolor{red}{$-1$} & \large\ding{40} = \textcolor{red}{2} & \large\ding{168} = \textcolor{red}{3} \\
       \hline
       \large\ding{36} = \textcolor{red}{$-7$} & \large\ding{52} = \textcolor{red}{$-5$} \\
       \cline{1-2}
    \end{tabular}
    \ \\ [5mm]
    Quatrième code. \\ [2mm]
    \begin{tabular}{|p{1.8cm}|p{1.85cm}|p{1.8cm}|}
       \hline
       \large\ding{101} = \textcolor{red}{81} & \large\ding{40} = \textcolor{red}{$-59$} & \large\ding{168} = \textcolor{red}{$-25$} \\
       \hline
       \large\ding{36} = \textcolor{red}{47} & \large\ding{52} = \textcolor{red}{$-54$} \\
       \cline{1-2}
    \end{tabular}}
 \end{corrige} 