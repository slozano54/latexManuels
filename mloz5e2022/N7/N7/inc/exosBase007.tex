\begin{exercice*}
   Dans un QCM de dix questions, une réponse juste rapporte 4 points, une absence de réponse 0 point et une mauvaise réponse enlève 3 points.
   \begin{enumerate}
      \item Yassin a 2 bonnes réponses et 8 mauvaises. Quelle est sa note ?
      \item Quelle est la plus mauvaise note qu'il est possible d'obtenir à ce QCM ? La meilleure note ?
      \item Chamse-Dine a obtenu 14 points. Donner une combinaison possible pour obtenir ce résultat.
   \end{enumerate}
\end{exercice*}

\begin{corrige}
   \ \\ [-5mm]
   \begin{enumerate}
      \item 2 bonnes réponses donnent  : \\
         $2\times4\text{ points} =8\text{ points}$ ; \\
         8 mauvaises réponses enlèvent : \\
         $8\times3\text{ points} =24\text{ points}$. \\
         Or, $+8-24 =-(24-8) =-16$ donc, la note de Mohamed-Amine est de {\blue $-16$ points}.
      \item La plus mauvaise note est obtenue lorsque l'on donne 10 mauvaises réponses, soit : \\
         $-(10\times3\text{ points}) =\blue -30\text{ points}$ ; \\
         La meilleure note est obtenue lorsque l'on donne 10 bonnes réponses, soit  : \\
         $+(10\times4\text{ points}) =\blue +40\text{ points}$. \\
      \item Emma peut, par exemple, avoir donné :
         \begin{itemize}
            \item {\blue 5 bonnes réponses}, soit +20 points ;
            \item {\blue 2 mauvaises réponses}, soit $-6$ points ; 
            \item {\blue 3 questions sans réponse}, soit 0 point.
         \end{itemize}
      Cela donne bien : $+20-6+0 =14$.
   \end{enumerate}
\end{corrige}

