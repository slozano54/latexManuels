\vspace*{-5mm}
%pre-001
\begin{prerequis}[Connaisances \emoji{red-heart} et compétences \emoji{diamond-suit} du cycle 3]    
   \begin{itemize}        
       \item[\emoji{red-heart}] Vocabulaire associé à ces objets et à leurs propriétés : côté, sommet, angle, hauteur.
       \columnbreak
       \item[\emoji{diamond-suit}] Reconnaître, nommer, décrire des triangles, dont les triangles particuliers (triangle rectangle, triangle isocèle, triangle équilatéral).       
   \end{itemize}
\end{prerequis}
\begin{debat}[Débat : Brahmagupta et l'invention du 0]
   {\bf Brahmagupta} est un mathématicien indien né en 598. Dans l'un de ses ouvrages, le {\it Brahma Sphuta Siddhanta}, il présente les règles d'arithmétique qui concernant les nombres positifs (qu'il appelle les biens) et les nombres négatifs (qu'il appelle les dettes) par des calculs de pertes et de profits. Il définit ainsi le zéro comme la différence d’un nombre par lui-même. Par exemple, voilà comment il exprime les opérations usuelles :
   \begin{itemize}
      \item zéro soustrait d’une dette est une dette ;
      \item zéro soustrait d’un bien est un bien ;
      \item zéro soustrait de zéro est zéro ;
      \item une dette soustraite de zéro est un bien ;
      \item un bien soustrait de zéro est une dette.
   \end{itemize}
   \begin{center}
      {\psset{unit=0.8}
      \textcolor{B1}{
      \begin{pspicture}(0,1)(11,5.5)
         \psset{fillstyle=solid}
         \psellipse[fillcolor=A1!50](2,4.3)(1.3,0.8)
         \rput(2,4.6){\it indien}
         \rput(2,4){\bf sunya}
         \psline{->}(3.5,4.3)(4.5,4.3)
         \psellipse[fillcolor=A1!40](6,4.3)(1.3,0.8)
         \rput(6,4.6){\it arabe}
         \rput(6,4){\bf sifr}
         \psline{->}(7.5,4.3)(9,3.9) %
         \psellipse[fillcolor=A1!30](9.5,3)(1.3,0.8)
         \rput(9.5,3.3){\it latin}
         \rput(9.5,2.7){\bf zephirum}
         \psline{->}(9,2.1)(7.5,1.7)
         \psellipse[fillcolor=A1!20](6,1.7)(1.3,0.8)
         \rput(6,2){\it italien}
         \rput(6,1.4){\bf zephiro}
         \psellipse[fillcolor=A1!10](2,1.7)(1.3,0.8)
         \rput(2,2){\it français}
         \rput(2,1.4){\bf zéro}
         \psline{<-}(3.5,1.7)(4.5,1.7)    
      \end{pspicture}}}
   \end{center}
   \begin{cadre}[B2][J4]
      \begin{center}
         \hrefVideo{https://leblob.fr/fondamental/les-nombres-negatifs}{\bf Les nombres négatifs}, épisode de la série {\it Petits contes mathématiques}.
      \end{center}
   \end{cadre}
\end{debat}