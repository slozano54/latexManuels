\begin{exercice*} %5
   Parmi les deux méthodes suivantes pour calculer $33\times103$, quelle est la plus rapide ?
   \begin{enumerate}
      \item Poser le calcul en colonnes.
      \item Décomposer le nombre 103 comme la somme de deux nombres simples, puis développer l'expression $33\times(\pointilles[5mm]\,+ \pointilles[5mm]\,)$ obtenue. Que remarque-t-on ?
   \end{enumerate}
\end{exercice*}

\begin{corrige}
   \ \\ [-5mm]
   \begin{enumerate}
      \item On trouve $33\times103 =\red \num{3399}$
      \item $\Circled{33}\times103 =\Circled{33}\times(100+3) =\Circled{33}\times100+\Circled{33}\times3$ \\
         \hspace*{18mm} $=\num{3300}+99 =\red\num{3399}$. \\
{\red La deuxième méthode est plus simple pour calculer le résultat de tête}.
   \end{enumerate}
\end{corrige}
