\begin{exercice*} %3
   On considère l'expression suivante : \\
   $A=97\times27+3\times27$
   \begin{enumerate}
      \item En respectant les priorités opératoires, effectuer le calcul de $A$ sans calculatrice.
      \item Factoriser $A$ puis calculer sa valeur toujours sans calculatrice. Que constate-t-on ?
      \item Calculer sans calculatrice $B =47\times\num{1215}-47\times215$.
   \end{enumerate}
\end{exercice*}  

\begin{corrige}
   \ \\ [-5mm]
   \begin{enumerate}
      \item $A=97\times27+3\times27 =\num{2619}+81 =\red\num{2700}$.
      \item $A=97\times\Circled{27}+3\times\Circled{27} =(97+3)\times\Circled{27}$ \\
         \quad\, $=100\times27 =\red\num{2700}$. \\
         {\red La méthode  est plus simple et plus rapide}. 
      \item $B =\num{1215}\times\Circled{47}-\Circled{47}\times215$ \\
         \quad\, $=\Circled{47}\times(\num{1215}-215) =47\times\num{1000} =\red \num{47000}$
   \end{enumerate}
\end{corrige}  
