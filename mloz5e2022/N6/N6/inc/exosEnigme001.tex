% Les enigmes ne sont pas numérotées par défaut donc il faut ajouter manuellement la numérotation
% si on veut mettre plusieurs enigmes
% \refstepcounter{exercice}
% \numeroteEnigme
\begin{changemargin}{-10mm}{-10mm}
    \begin{enigme}[Un diamant littérale]
        \begin{minipage}{9cm}
           Découper les douze pièces du puzzle et les assembler de telle sorte que les expressions face à face soient égales. Coller le puzzle sur votre cahier. \\ [2mm]
           La forme à obtenir est celle ci-contre.
        \end{minipage}
        \hfill
        \begin{minipage}{5cm}
           {\psset{unit=0.5}
           \begin{pspicture}(0,1)(6,7)      
              \rput{-60}(2,4){\tri{}{}{}}
              \rput(2,4){\tri{}{}{}}
              \rput{60}(2,4){\car{}{}{}{}}
              \rput{150}(2,4){\tri{}{}{}}
              \rput{-150}(2,4){\car{}{}{}{}}
              \rput{-150}(3,2.27){\tri{}{}{}}
              \rput{-90}(3,2.27){\tri{}{}{}}
              \rput{-120}(4,4){\car{}{}{}{}}
              \rput{-30}(4,4){\tri{}{}{}}
              \rput{30}(4,4){\car{}{}{}{}}
              \rput{30}(3,5.73){\tri{}{}{}}
              \rput{90}(3,5.73){\tri{}{}{}}
           \end{pspicture}}
        \end{minipage}
        \begin{center}
           {\psset{unit=2.5,linewidth=0.6mm}
           \large
           \begin{pspicture}(0,-0.5)(6,6.5)
              \rput(0,0){\car{4x-8x+6x}{14x+35x}{13x}{}} %6
              \rput(2,0){\car{42x}{12x}{}{8x-4x}} %3
              \rput(4,0){\car{25x}{}{20x+8x-5x}{77x}} %5
              \rput(0,2){\tri{}{2x}{20x+5x}} %12
              \rput{60}(2,2){\tri{154x}{7x+70x}{x}} %2
              \rput(2,2){\tri{}{15x}{16x-4x}} %9
              \rput{60}(4,2){\tri{7x+35x}{14x-35x+22x}{49x}} %1
              \rput(4,2){\tri{}{0}{2x+15x-4x}} %7
              \rput(0,3.73){\car{14x+140x}{20x}{}{10x+5x}} %4
              \rput(2,3.73){\tri{4x}{2x-2x}{}} %8
              \rput{60}(4,3.73){\tri{30x}{10x+10x}{}} %10
              \rput(4,3.73){\tri{}{23x}{10x+20x}} %11
           \end{pspicture}}
        \end{center}
     \end{enigme}
     
     \vfill \hfill{\it\footnotesize Source : \href{https://www.monclasseurdemaths.fr/profs/puzzles/}{monclasseurdemaths.fr}}
\end{changemargin}
% Pour le corrigé, il faut décrémenter le compteur, sinon il est incrémenté deux fois
\addtocounter{exercice}{-1}
\begin{corrige}
    {\psset{unit=2.5}
          \begin{pspicture}(-0.107,0)(6,8)      
             \rput{-60}(2,4){\tri{7x+70x}{x}{154x}} %2
             \rput(2,4){\tri{14x-35x+22x}{49x}{7x+35x}} %1
             \rput{60}(2,4){\car{14x+35x}{13x}{}{4x-8x+6x}} %6
             \rput{150}(2,4){\tri{2x}{20x+5x}{}} %12
             \rput{-150}(2,4){\car{25x}{}{20x+8x-5x}{77x}} %5
             \rput{-150}(3,2.27){\tri{23x}{10x+20x}{}} %11
             \rput{-90}(3,2.27){\tri{30x}{10x+10x}{}} %10
             \rput{-120}(4,4){\car{14x+140x}{20x}{}{10x+5x}} %4
             \rput{-30}(4,4){\tri{15x}{16x-4x}{}} %9
             \rput{30}(4,4){\car{12x}{}{8x-4x}{42x}} %3
             \rput{30}(3,5.73){\tri{4x}{2x-2x}{}} %8
             \rput{90}(3,5.73){\tri{0}{2x+15x-4x}{}} %7
          \end{pspicture}}   
    \end{corrige}