\begin{changemargin}{-7mm}{-7mm}
    \vspace*{-7mm}
    %pre-001
    \begin{prerequis}[Connaisances \emoji{red-heart} et compétences \emoji{diamond-suit} du cycle 3]    
   \begin{itemize}        
       \item[\emoji{red-heart}] Vocabulaire associé à ces objets et à leurs propriétés : côté, sommet, angle, hauteur.
       \columnbreak
       \item[\emoji{diamond-suit}] Reconnaître, nommer, décrire des triangles, dont les triangles particuliers (triangle rectangle, triangle isocèle, triangle équilatéral).       
   \end{itemize}
\end{prerequis}
    \vspace*{-3mm}
    %pre-002
    \begin{prerequis}[Connaisances \emoji{red-heart} et compétences \emoji{diamond-suit} du cycle 4]    
    \begin{itemize}        
        \item[\emoji{diamond-suit}] Mener des calculs impliquant des grandeurs mesurables, exprimer les résultats dans des les unités adaptées.
        \item[\emoji{diamond-suit}] Exprimer et vérifier la cohérence des résultats du point de vue des unités.
    \end{itemize}
\end{prerequis}
    \vspace*{-3mm}
    \begin{debat}[Polysémie du facteur]
        \vspace*{-7mm}
        La {\bf polysémie} est la caractéristique d'un mot ou d'une expression qui a plusieurs sens ou significations différentes. En mathématiques, on utilise régulièrement des mots qui n'ont pas forcément le même sens qu'en français par exemple. \\
        Le mot {\bf facteur} ne déroge pas à cette règle : étymologiquement, il vient du latin {\it factir}, celui qui fait. Le facteur que nous utilisons en mathématiques désigne un terme d'un produit, et le facteur que nous connaissons le mieux est certainement la personne distribuant le courrier. À l'origine, le facteur est un fabriquant d'instruments de musique. Enfin, le terme facteur s'utilise aussi en économie ou en biologie pour mentionner un élément important qui concourt à un résultat.
        \begin{center}
            \scalebox{0.7}{
                \begin{pspicture}(0,0)(3,2)
                    \psframe[fillstyle=solid,fillcolor=A1!15](0,0)(3,2)
                    \pspolygon[fillstyle=solid,fillcolor=A1!10](0,2)(1.5,0.8)(3,2)
                \end{pspicture}
            }
        \end{center}
        \begin{cadre}[B2][J4]
           \begin{center}
              \hrefVideo{https://www.yout-ube.com/watch?v=g73sqrZZlQo}{\bf Comprendre la simple distributivité}, chaîne YouTube de {\it Jean-Yves Labouche}.
           \end{center}
        \end{cadre}
     \end{debat}
\end{changemargin}