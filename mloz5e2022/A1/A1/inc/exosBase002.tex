\begin{exercice}
    Tracer les figures obtenues lorsque l'on exécute les programmes suivants avec scratch. Pour chaque cas, donner la nature de la figure obtenue. \\
    {\it On représentera l'unité (un pas )par \umm{1} sur le cahier.} \\ [2mm]
          Programme 1 \hspace*{2.2cm} Programme 2 \\ [1mm]
          \begin{Scratch}[Echelle=0.7]
             Place Drapeau;
             Place PoserStylo;
             Place Repeter("4");
                Place Tournerd("90");
                Place Avancer("50");
             Place FinBlocRepeter;
          \end{Scratch}
          \qquad
          \begin{Scratch}[Echelle=0.7]
             Place Drapeau;
             Place PoserStylo;
             Place Repeter("3");
                Place Avancer("40");
                Place Tournerg("120");
             Place FinBlocRepeter;
          \end{Scratch}
          \\ [3mm]
          Programme 3 \hspace*{2.2cm} Programme 4 \\ [1mm]
          \begin{Scratch}[Echelle=0.7]
             Place Drapeau;
             Place PoserStylo;
             Place Repeter("2");
                Place Avancer("20");
                Place Tournerd("90");
                Place Avancer("80");
                Place Tournerd("90");
             Place FinBlocRepeter;
          \end{Scratch} 
          \qquad
          \begin{Scratch}[Echelle=0.7]
             Place Drapeau;
             Place PoserStylo;
             Place Repeter("2");
                Place Avancer("30");
                Place Tournerd("50");
                Place Avancer("30");
                Place Tournerd("130");
             Place FinBlocRepeter;
          \end{Scratch}

          \hrefMathalea{https://coopmaths.fr/mathalea.html?ex=6I12,s=9,s2=1,i=0&v=l}
          
          \hrefMathalea{https://coopmaths.fr/mathalea.html?ex=5I11,s=1,s2=1,s3=4,s4=false,n=1,cd=1&v=l}          
 \end{exercice}
 
 \begin{corrige}
   \begin{itemize}
      \def\item{}
    \item Le programme 1 donne un {\blue carré} de côté \ucm{5}, le programme 2 un {\blue triangle équilatéral} de côté \ucm{4}, le programme 3 un {\blue losange} de côté \ucm{3} et le programme 4 un {\blue rectangle} de longueur \ucm{8} et de largeur \ucm{2}. \\
    {\psset{linecolor=blue,unit=0.8cm}
    \begin{pspicture}(0,0)(7,7.3)                                                                              
       \psgrid[gridlabels=0,subgriddiv=0,gridcolor=lightgray](0,0)(7,7)
       \psdot[linewidth=0.7mm](1,6)
       \psframe(1,1)(6,6)    
    \end{pspicture}}
   \end{itemize}
  \Coupe
  \begin{itemize}
   \def\item{}
    \item {\psset{linecolor=blue,unit=0.8cm}
    \begin{pspicture}(0,0)(8,9)                                                                              
       \psgrid[gridlabels=0,subgriddiv=0,gridcolor=lightgray](0,0)(8,9)    
       \psdot[linewidth=0.7mm](0,4)
       \pspolygon(0,4)(4,4)(2,7.46)
       \psdot[linewidth=0.7mm](6,8)
       \psframe(6,0)(8,8)
       \psdot[linewidth=0.7mm](0,3)
       \pspolygon(0,3)(3,3)(4.93,0.7)(1.93,0.7)
    \end{pspicture}}
   \end{itemize}
 \end{corrige}
 