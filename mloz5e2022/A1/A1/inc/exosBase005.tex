\begin{exercice}
   % https://coopmaths.fr/mathalea.html?ex=dnb_2021_09_metropole_4&serie=VbC4&z=1&v=exEtChoix
   Voici trois programmes réalisés avec l'application Scratch.

   \newcommand{\echelleAunMlozCinq}{0.8}
   \small{
      \begin{center}
         \begin{tabularx}{\linewidth}{|*{3}{>{\centering \arraybackslash}X|}}\hline
         Programme 1 &Programme 2 &Programme  3\rule[-3mm]{0mm}{6mm}\\
         \begin{Scratch}[Numerotation,Echelle=\echelleAunMlozCinq]
            Place Drapeau;
            Place PoserStylo;
            Place Repeter("3");
               Place Avancer("100");
               Place Tournerg("120");
            Place FinBlocRepeter;
            Place Avancer("50");
            Place Repeter("4");
               Place Avancer("?");
               Place Tournerg("90");
            Place FinBlocRepeter;
         \end{Scratch}
         &
         \begin{Scratch}[Numerotation,Echelle=\echelleAunMlozCinq]
            Place Drapeau;
            Place PoserStylo;
            Place Repeter("3");
               Place Avancer("100");
               Place Tournerg("120");
            Place FinBlocRepeter;
            Place Avancer("100");
            Place Repeter("4");
               Place Avancer("?");
               Place Tournerg("90");
            Place FinBlocRepeter;
         \end{Scratch}
         &
         \begin{Scratch}[Numerotation,Echelle=\echelleAunMlozCinq]
            Place Drapeau;
            Place PoserStylo;
            Place Repeter("3");
               Place Avancer("100");
               Place Tournerg("120");
            Place FinBlocRepeter;
            Place Tournerg("60");
            Place Repeter("4");
               Place Avancer("?");
               Place Tournerg("90");
            Place FinBlocRepeter;
         \end{Scratch}
         \\ \hline
         \end{tabularx}
       \end{center}
   }
   \begin{enumerate}
   \item Ils donnent les trois figures suivantes constituées de triangles et de quadrilatères \textbf{identiques}.

   \begin{center}
   \begin{tabularx}{\linewidth}{|*{3}{>{\centering \arraybackslash}X|}}\hline
   Figure A &Figure B& Figure C\\
   \psset{unit=1cm}
   \begin{pspicture}(-3.2,-1)(1.3,2.5)
   %\psgrid
   \pspolygon(1.1;-30)(1.1;90)(1.1;210)
   \rput{60}(-0.98,-0.58){\psframe(1.95,1.95)}
   \end{pspicture}&\psset{unit=1cm}
   \begin{pspicture}(-1.3,-1)(3.2,2.5)
   %\psgrid
   \pspolygon(1.1;-30)(1.1;90)(1.1;210)
   \psframe(0,-0.55)(1.95,1.4)
   \end{pspicture}&\psset{unit=1cm}
   \begin{pspicture}(-1.3,-1)(3.2,2.5)
   %\psgrid
   \pspolygon(1.1;-30)(1.1;90)(1.1;210)
   \psframe(0.96,-0.55)(2.91,1.4)
   \end{pspicture}\\ \hline
   \end{tabularx}\end{center}


      \begin{enumerate}
         \item Indiquer la nature du triangle et du quadrilatère sur chaque figure. Aucune justification n'est attendue.
         \item Déterminer la valeur manquante à la ligne 8 dans ces 3 programmes.
         \item Indiquer, pour chaque figure, le numéro du programme qui permet de l'obtenir.
      \end{enumerate}
   \item  
      \begin{enumerate}
         \item Maintenant nous allons modifier les programmes précédents pour construire d'autres figures pour lesquelles
         le périmètre du quadrilatère est égal au périmètre du triangle.
         Indiquer la valeur du pas à choisir à la ligne 8 de chaque programme.
         \item Représenter la figure A obtenue avec cette nouvelle valeur, en prenant 1 cm pour 25 pas.
      \end{enumerate}
   \end{enumerate}

\end{exercice}
 
\begin{corrige}
   \begin{enumerate}
      \item 
         \begin{enumerate}
            \item Dans chaque cas le triangle est équilatéral et le quadrilatère est un carré.
            \item Avancer de 100 pas.
            \item Programme 1 : figure B ; \\Programme 2 : figure C ;\\ Programme 3 : figure A.
         \end{enumerate}
      \item  
         \begin{enumerate}
            \item Si $c$ est la longueur du côté du carré et $t$ la longueur du côté du triangle, on doit avoir $4c = 3t$.
            
            Donc si $t = 100$, alors $4c = 300$, soit $c =  75$.
      
            Il faut donc écrire à la ligne 8  : avancer de 75 pas.
            \item ~
            
               \psset{unit=1cm}
               \begin{pspicture}(5,3)
                  \pspolygon(1.8,0)(4.8,0)(3.3,2.6)
                  \rput{60}(1.8,0){\psframe(2.25,2.25)}
               \end{pspicture}
         \end{enumerate}
   \end{enumerate}
\end{corrige}
 