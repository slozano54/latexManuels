\begin{exercice*}
   Le tableau suivant résume les résultats obtenus par une classe lors d'une évaluation.
   \begin{center}
   {\small
      \Stat[Tableau,CouleurTab=FondTableaux,Largeur=1.5mm,Donnee=N,Effectif=E,Stretch=1.2]{3/1,5/2,6/1,7/3,8/3,9/5,10/6,11/4,12/2,13/1,14/2,17/2,18/1}}
   \end{center}
   \begin{enumerate}
      \item Combien y a-t-il d'élèves dans cette classe ?
      \item Compléter le tableau par les fréquences.
      \item Quel est le pourcentage d'élèves ayant obtenu une note inférieure ou égale à 8 ?
      \item Déterminer la moyenne de la série de notes.
      \item Cette évaluation était la quatrième de la période. \\
         Toutes les évaluations ont le même coefficient et jusqu'alors Bastien avait 9 de moyenne ; après ce devoir, il a 9,5 de moyenne. Quelle note a-t-il obtenue à ce devoir ? 
   \end{enumerate}
\end{exercice*}

\begin{corrige}
   \ \\ [-5mm]
   \begin{enumerate}
      \item $1+2+1+3+3+5+6+4+2+1+2+2+1 =33$. Donc, {\red il y a 33 élèves dans cette classe}.
      \item Tableau des fréquences en \%. \\ \smallskip
      {\small
      \renewcommand{\arraystretch}{1.3}
      \begin{lctableau}{\linewidth}{14}
         \hline
         N & 3 & 5 & 6 & 7 & 8 & 9 & \!10 & \!11 & \!12 & \!13 & \!14 & \!17 & \!18 \\
         \hline
         E & 1 & 2 & 1 & 3 & 3 & 5 & 6 & 4 & 2 & 1 & 2 & 2 & 1 \\
         \hline
         F & 3 & 6 & 3 & 9 & 9 & \!\!15 & \!\!18 & \!\!12 & 6 & 3 & 6 & 6 & 3 \\
         \hline
     \end{lctableau}}
      \item $p =\dfrac{1+2+1+3+3}{33}\times100 =\dfrac{10}{33}\times100 \approx30,3$. \\ [1.5mm]
      Environ {\red 30,3\,\% des élèves} ont obtenu une note inférieure ou égale à 8. \smallskip
      \item $\overline{m} =(1\times3+2\times5+1\times6+3\times7+3\times8+5\times9+6\times10+4\times11+2\times12+1\times13+2\times14+2\times17+1\times18)\div33 =330\div33 =10$. \\
      {\red La moyenne est de 10}.
      \item La moyenne après le 4\up{e} devoir est de 9,5 donc, la somme de ses notes est de $4\times9,5 =38$. \\
         La somme des trois premiers devoirs était de $3\times9 =27$.  Or, $38-27 =11$ donc  : \\
         {\red Bastien a obtenu 11} au dernier devoir.
  \end{enumerate}
\end{corrige}
