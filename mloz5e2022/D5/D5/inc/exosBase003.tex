\begin{exercice*}
   Quatre-vingts archers d'un club de tir à l'arc A ont participé à un championnat. Le nombre de points obtenus par chaque archer du club est donné par le diagramme ci-dessous.
   \begin{center}
      {\footnotesize      
      \Stat[%
         Graphique,Grille,Unitex=0.65,Unitey=0.25,Pasx1,%
         EpaisseurBatons=5,Donnee={},Effectif=Nombre d'archers,%
         LectureFine,Tiret,CouleurDefaut=Gray%
      ]{0/0,1/0,2/5,3/9,5/8,6/12,7/14,8/6,9/8,10/18} \\ [-2mm]
   \hspace*{50mm} Nombre de points}
   \end{center}
   \vspace*{-10mm}
   \begin{enumerate}
      \item 
      \begin{enumerate}
         \item Combien d'archers ont gagné exactement six points lors de ce championnat ?
         \item Combien d'archers ont gagné trois points ou plus lors de ce championnat ?
      \end{enumerate}
      \item Le club de tir à l'arc B a aussi participé à ce championnat. Voici quelques données :
      \begin{itemize}
         \item Le score moyen des archers lors du championnat est 7 points.
         \item Le score moyen des dix meilleurs archers lors du championnat est 9,9 points. \\ [-10mm]
      \end{itemize}
      \begin{enumerate}
         \item Comparer les résultats des deux clubs selon leurs scores moyens.
         \item Comparer les résultats des deux clubs selon les scores de leurs dix meilleurs archers.
      \end{enumerate}
   \end{enumerate}
\end{exercice*}

\begin{corrige}
   \ \\ [-5mm]
   \begin{enumerate}
      \item
      \begin{enumerate}
         \item {\red 12 d'archers} ont gagné six points.
         \item 5 archers ont obtenu 2 points. Or, $80-5 =75$ \\
            donc, {\red 75 ont gagné trois points ou plus}.
      \end{enumerate}
      \setcounter{enumi}{1}
      \item 
      \begin{enumerate}
         \item Score moyen des archers du club A : \\
            $\overline{m} =(5\times2+9\times3+8\times5+12\times6+14\times7+6\times8+8\times9+18\times10)\div80 =547\div80 =6,8375$. \\
            Or, le score moyen du club B est de 7 points donc, c'est {\red le club B} qui a le score moyen le plus élevé.
         \item Les 10 meilleurs archers du club A ont marqué 10 points alors que la moyenne des 10 meilleurs archers du club B est de 9,9 points donc, c'est {\red le club A} qui possède les dix meilleurs résultats.
      \end{enumerate}
   \end{enumerate}
\end{corrige}
