\begin{exercice*}
   Ce tableau présente les températures moyennes mensuelles à Tours en 2019.
   \begin{center}      
      \Stat[Tableau,Qualitatif,CouleurTab=FondTableaux,Donnee=Mois,Effectif=Température (\Temp{}),Largeur=6mm] {J/4.4,F/7.8,M/9.6,A/11.2,M/13.4,J/19.4}

      \Stat[Tableau,Qualitatif,CouleurTab=FondTableaux,Donnee=Mois,Effectif=Température (\Temp{}),Largeur=6mm] {J/22.6,A/20.5,S/17.9,O/14.4,N/8.2,D/7.8}
   \end{center}
   \begin{enumerate}
      \item D'après le tableau, quelle a été la température moyenne à Tours en novembre 2019 ?
      \item Vérifier que la température moyenne annuelle est \udegc{13,1}.
      \item La température moyenne annuelle à Tours en 2009 était de \udegc{11,9}. Le pourcentage d'augmentation entre 2009 et 2019, arrondi à l'unité, est-il de :  7\,\% ; 10\,\%  ou 13\,\%  ? Justifier la réponse.
   \end{enumerate}
\end{exercice*}

\begin{corrige}
   \ \\ [-5mm]
      \begin{enumerate}
      \item La température moyenne à Tours en novembre 2019 a été de {\red \udegc{8,2}}.
      \item $(4,4+7,8+9,6+11,2+13,4+19,4+22,6+20,5+17,9+ 14,4+8,2+7,8)\div12$ \\
         $=157,2\div12 =13,1$. {\red La température moyenne annuelle à Tours en 2019 était de \udegc{13,1}}. \smallskip
         \item \textcolor{G1}{$\bullet$} Calcul avec 7\,\% : $\dfrac{7}{100}\times\udegc{11,9} =\udegc{0,833}$. \\ [1mm]
            $\udegc{11,9}+\udegc{0,833} =\udegc{12,733} \neq\udegc{13,1}$. \\ [1mm]
            \textcolor{G1}{$\bullet$} Calcul avec  10\,\% : $\dfrac{10}{100}\times\udegc{11,9} =\udegc{1,19}$. \\ [1mm]
            $\udegc{11,9}+\udegc{1,19} =\udegc{13,09} \approx\udegc{13,1}$. \\
           {\red Le pourcentage d'augmentation de la température entre 2009 et 2019 a été d'environ 10\,\%}.
   \end{enumerate}
\end{corrige}
