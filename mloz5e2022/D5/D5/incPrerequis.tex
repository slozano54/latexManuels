\vspace*{-8mm}
%pre-001
\begin{prerequis}[Connaisances \emoji{red-heart} et compétences \emoji{diamond-suit} du cycle 3]    
   \begin{itemize}        
       \item[\emoji{red-heart}] Vocabulaire associé à ces objets et à leurs propriétés : côté, sommet, angle, hauteur.
       \columnbreak
       \item[\emoji{diamond-suit}] Reconnaître, nommer, décrire des triangles, dont les triangles particuliers (triangle rectangle, triangle isocèle, triangle équilatéral).       
   \end{itemize}
\end{prerequis}
\vspace*{-4mm}
%pre-002
\begin{prerequis}[Connaisances \emoji{red-heart} et compétences \emoji{diamond-suit} du cycle 4]    
    \begin{itemize}        
        \item[\emoji{diamond-suit}] Mener des calculs impliquant des grandeurs mesurables, exprimer les résultats dans des les unités adaptées.
        \item[\emoji{diamond-suit}] Exprimer et vérifier la cohérence des résultats du point de vue des unités.
    \end{itemize}
\end{prerequis}
\begin{debat}[Les statistiques]
    \begin{changemargin}{-10mm}{-20mm}
        \vspace*{-3mm}
    Les premiers textes écrits retrouvés sur les {\bf statistiques} étaient des recensements de bétail. On attribue souvent l'introduction du terme {\bf statistique} au professeur {\it Achenwall}, qui aurait, en 1746, créé le mot {\it Statistik}, dérivé de l'allemand {\it Staatskunde}. \\
    Même si cette branche des mathématiques est récente, elle fait partie des notions les plus utilisées actuellement et les plus gros consommateurs de statistiques sont les assureurs (risques d'accidents, de maladie des assurés), les médecins (épidémiologie), les démographes (populations et leur dynamique), les économistes (emploi, conjoncture économique), les météorologues\dots
    \vspace*{-5mm}
    \begin{center}
       \begin{Geometrie}[CoinBG={(-u,-u)}]
            drawoptions(withcolor Crimson);
            label(TEX("\rotatebox{10}{\small maximum}"),u*(0,0));
            label(TEX("\rotatebox{20}{\small moyenne}"),u*(1.5,1));
            label(TEX("\rotatebox{-10}{\small minimum}"),u*(3,0));
            label(TEX("\rotatebox{-20}{\small fréquence}"),u*(4.5,1));
            label(TEX("\rotatebox{30}{\small médiane}"),u*(0.5,2));
            label(TEX("\rotatebox{-30}{\small mode}"),u*(2,3));
            label(TEX("\rotatebox{40}{\small écart-type}"),u*(3.5,2));
            label(TEX("\rotatebox{-25}{\small histogramme}"),u*(5,3));
            label(TEX("\small quartiles") ,u*(-0.5,3));
       \end{Geometrie}
    \end{center}
    \begin{cadre}[B2][J4]
       \begin{center}
          \hrefVideo{https://www.yout-ube.com/watch?v=aOX0pIwBCvw}{\bf Chocolat, corrélation et moustache de chat}, chaîne {\it La statistique expliquée à mon chat}.
       \end{center}
    \end{cadre}
    \end{changemargin}
 \end{debat}