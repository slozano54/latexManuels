\begin{exercice}
    Colorier les thermomètres jusqu'à la graduation de la température donnée.\par
    \Reperage[Thermometre,Pasx=10,ValeurUnitex=10]{-17/,-15/,15/}
    \Reperage[Thermometre,Pasx=5,ValeurUnitex=1]{-6/,-10/,6/}
    \Reperage[Thermometre,Pasx=10,ValeurUnitex=1]{-5/,-20/,12/}
    \Reperage[Thermometre,Pasx=10,ValeurUnitex=5]{12/,-15/,17/} \\
    \hspace*{7mm} \Temp{-17} \hfill \Temp{-1,2} \hfill \Temp{-0,5} \hfill \Temp{6}
 \end{exercice}
 
 \begin{corrige}
    On a les hauteurs de mercure suivantes : \\ [2mm]
    \hspace*{-15mm}
    \Reperage[Thermometre,Pasx=10,ValeurUnitex=10,Mercure]{-17/,-15/,15/}
    \Reperage[Thermometre,Pasx=5,ValeurUnitex=1,Mercure]{-6/,-10/,6/}
    \Reperage[Thermometre,Pasx=10,ValeurUnitex=1,Mercure]{-5/,-20/,12/}
    \Reperage[Thermometre,Pasx=10,ValeurUnitex=5,Mercure]{12/,-15/,17/} \\
 \end{corrige}