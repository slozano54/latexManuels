% Les enigmes ne sont pas numérotées par défaut donc il faut ajouter manuellement la numérotation
% si on veut mettre plusieurs enigmes
% \refstepcounter{exercice}
% \numeroteEnigme
\begin{enigme}[Nombres croisés]
    Compléter cette grille de nombres croisés à l'aide de chiffres et de signes \og $+$ \fg{} ou \og $-$ \fg{} grâce aux indications données.
    \begin{center}
    {\renewcommand{\arraystretch}{1.95}
    \begin{tabular}{|*{7}{>{\centering\arraybackslash}p{0.57cm}|}}
       \hline
       & \cellcolor{J2}{\textcolor{B1}{a}} & \cellcolor{J2}{\textcolor{B1}{b}} & \cellcolor{J2}{\textcolor{B1}{c}} & \cellcolor{J2}{\textcolor{B1}{d}} & \cellcolor{J2}{\textcolor{B1}{e}} & \cellcolor{J2}{\textcolor{B1}{f}} \\
       \hline
       \cellcolor{J2}{\textcolor{A1}{1}} & & \cellcolor{black!70} & & & \cellcolor{black!70} & \\
       \hline
       \cellcolor{J2}{\textcolor{A1}{2}} & & & & \cellcolor{black!70} & & \\
       \hline
       \cellcolor{J2}{\textcolor{A1}{3}} & & \cellcolor{black!70} & \cellcolor{black!70} & & & \cellcolor{black!70} \\
       \hline
       \cellcolor{J2}{\textcolor{A1}{4}} & \cellcolor{black!70} & & & \cellcolor{black!70} & \cellcolor{black!70} & \\
       \hline
       \cellcolor{J2}{\textcolor{A1}{5}} & & & \cellcolor{black!70} & & & \\
       \hline
       \cellcolor{J2}{\textcolor{A1}{6}} & & \cellcolor{black!70} & & & & \cellcolor{black!70} \\
       \hline
    \end{tabular}}
    \end{center}
    \bigskip
    \begin{multicols}{2}
        {\bf Horizontalement} \\
        \begin{enumerate}
        \item Valeur du plus grand chiffre. \\
            Opposé de l'entier compris entre $-12,2$ et $-13,9$. \\
            Les nombres négatifs sont précédés de ce signe. \\
        \item Résultat du calcul $8\times20-(12+28)$. \\
            Nombre entier compris entre $-1,8$ et $-0,2$. \\
        \item Opposé de l’opposé de $+8$. \\
            Nombre entier supérieur à $73,01$ et inférieur $74,99$. \\
        \item Sur une droite graduée de $3$ en $3$, je suis placé à trois graduations à gauche de l'origine. \\
            Signe de l’opposé d'un nombre positif. \\
        \item Nombre entier le plus proche $-1,4$. \\
            Nombre entier inférieur à $-15,154$ et supérieur à $-16,98$. \\
        \item Diviseur commun à $12$ ; $24$ et $33$. \\
            Mon chiffre des centaines est le double de mon chiffre des dizaines qui est lui-même le double de mon chiffre des unités. \\ [1mm]
        \end{enumerate}
        
        {\bf Verticalement} \\
        \begin{enumerate}
        \item[\textcolor{B1}{\bf a)}] Résultat du calcul $9\times(100+2)$. \\
            Nombre relatif inférieur à zéro et se trouvant à 5 unités du nombre $+2$. \\
        \item[\textcolor{B1}{\bf b)}] J'ai la même distance à zéro que le nombre $-2$. \\
            Nombre opposé de la moitié de 2. \\
        \item[\textcolor{B1}{\bf c)}] Le chiffre des unités est l'abscisse de l'origine et le chiffre des dizaines est le premier nombre entier positif non nul. \\
            Opposé de l'entier compris entre $-9,12$ et $-8,93$. \\
            Nombre relatif se situant après zéro et se trouvant à $11$ unités du nombre $-7$. \\
        \item[\textcolor{B1}{\bf d)}] Distance à zéro de l'opposé de $-\dfrac{33}{11}$. \\ [1mm]
            Opposé de $-42\div6$. \\
            Nombre négatif se trouvant à deux unités de l'origine. \\
        \item[\textcolor{B1}{\bf e)}] Nombre se trouvant à 8 unités de $-12$. \\ [1mm]
            Distance à zéro de $+\dfrac{22}{2}$. \\
        \item[\textcolor{B1}{\bf f)}] Opposé de $+1$. \\
            Nombre entier le plus proche et supérieur à $-6,98$.
        \end{enumerate}
    \end{multicols}
 \end{enigme}
 

% Pour le corrigé, il faut décrémenter le compteur, sinon il est incrémenté deux fois
% \addtocounter{exercice}{-1}
\begin{corrige}
    \smallskip
    \begin{center}
       {\renewcommand{\arraystretch}{1.95}
       \begin{tabular}{|*{7}{>{\centering\arraybackslash}p{0.57}|}}
          \hline
          & \cellcolor{J2}{a} & \cellcolor{J2}{b} & \cellcolor{J2}{c} & \cellcolor{J2}{d} & \cellcolor{J2}{e} & \cellcolor{J2}{f} \\
          \hline
          \cellcolor{J2}{1} & \textcolor{blue}{9} & \cellcolor{black!70} & \textcolor{blue}{1} & \textcolor{blue}{3} & \cellcolor{black!70} & \textcolor{blue}{$-$} \\
          \hline
          \cellcolor{J2}{2} & \textcolor{blue}{1} & \textcolor{blue}{2} & \textcolor{blue}{0} & \cellcolor{black!70} & \textcolor{blue}{$-$} & \textcolor{blue}{1} \\
          \hline
          \cellcolor{J2}{3} & \textcolor{blue}{8} & \cellcolor{black!70} & \cellcolor{black!70} & \textcolor{blue}{7} & \textcolor{blue}{4} & \cellcolor{black!70} \\
          \hline
          \cellcolor{J2}{4} & \cellcolor{black!70} & \textcolor{blue}{$-$} & \textcolor{blue}{9} & \cellcolor{black!70} & \cellcolor{black!70} & \textcolor{blue}{$-$} \\
          \hline
          \cellcolor{J2}{5} & \textcolor{blue}{$-$} & \textcolor{blue}{1} & \cellcolor{black!70} & \textcolor{blue}{$-$} & \textcolor{blue}{1} & \textcolor{blue}{6} \\
          \hline
          \cellcolor{J2}{6} & \textcolor{blue}{3} & \cellcolor{black!70} & \textcolor{blue}{4} & \textcolor{blue}{2} & \textcolor{blue}{1} & \cellcolor{black!70} \\
          \hline
       \end{tabular}}
       \end{center}
\end{corrige}