\begin{exercice*}   
   Pour découvrir le dessin codé, il faut placer les points A, B, C\dots{} selon les indications des tableaux.
   Par exemple, le point A est sur la première ligne et son abscisse est 8. \\
   Une fois tous les points placés les relier en suivant les instructions données.
   \begin{center}
      \begin{minipage}{0.4\linewidth}
         \begin{tabular}{|*{3}{>{\centering\arraybackslash}p{0.5cm}|}}
            \hline
            \cellcolor{lightgray}{\!\!\!\small Ligne} & \cellcolor{lightgray}{\!\!\!\small Point} & \cellcolor{lightgray}{\!\small Abs.} \\
            \hline
            (1) & A & -18 \\
            \hline
            (3) & B & \num{0.1}\\
            \hline
            (3) & C &  \num{0.5}\\
            \hline
            (3) & D &  \num{0.9}\\
            \hline
            (4) & E &  -20\\
            \hline
            (4) & F &  20\\
            \hline
            (5) & G &  -1\\
            \hline
            (5) & H &  -\num{0.9}\\
            \hline
         \end{tabular}
      \end{minipage}
      \begin{minipage}{0.4\linewidth}
         \begin{tabular}{|*{3}{>{\centering\arraybackslash}p{0.5cm}|}}
            \hline
            \cellcolor{lightgray}{\!\!\!\small Ligne} & \cellcolor{lightgray}{\!\!\!\small Point} & \cellcolor{lightgray}{\!\small Abs.} \\
            \hline
            (5) & I &  -\num{0.1}\\
            \hline
            (5) & J &  0\\
            \hline
            (6) & K &  -20\\
            \hline
            (6) & L &  0\\
            \hline
            (7) &  M &  100\\
            \hline
            (7) &  N &  200\\
            \hline
            (7) &  O &  800\\
            \hline
            (7) &  P &  900\\
            \hline  
         \end{tabular}
      \end{minipage}
   \end{center}
   \begin{center}
   %  \let\originalTextwidth\textwidth
   %  \setlength{\linewidth}{25cm}    
    \begin{minipage}{0.4\linewidth}
      \begin{tabular}{|*{3}{>{\centering\arraybackslash}p{0.5cm}|}}
         \hline
         \cellcolor{lightgray}{\!\!\!\small Ligne} & \cellcolor{lightgray}{\!\!\!\small Point} & \cellcolor{lightgray}{\!\small Abs.} \\
         \hline
         (8) &  Q &  -\num{0.5}\\
         \hline
         (8) &  R &  -\num{0.3}\\
         \hline
         (8) &  S &  -\num{0.2}\\
         \hline
         (8) & T &  \num{0.2}\\
         \hline
         (8) & U &  \num{0.3}\\
         \hline
         (8) & V &  \num{0.5}\\
         \hline
         (9) & W &  \num{3.14}\\
         \hline
         (9) & X &  \num{3.16}\\
         \hline
         (10) & Y &  \num{2.1}\\
         \hline
         (10) & Z &  \num{2.3}\\
         \hline
         (10) & A' &  \num{2.4}\\
         \hline
      \end{tabular}   
   \end{minipage}   
   \begin{minipage}{0.4\linewidth}
      \begin{tabular}{|*{3}{>{\centering\arraybackslash}p{0.5cm}|}}
         \hline
         \cellcolor{lightgray}{\!\!\!\small Ligne} & \cellcolor{lightgray}{\!\!\!\small Point} & \cellcolor{lightgray}{\!\small Abs.} \\
         \hline
         (10) & B' &  \num{2.6}\\
         \hline
         (10) & C' &  \num{2.7}\\
         \hline
         (10) & D' &  \num{2.9}\\
         \hline
         (11) & E' &  115\\
         \hline
         (11) &  F' &  120\\
         \hline
         (11) &  G' &  125\\
         \hline
         (11) &  H' &  130\\
         \hline
         (11) &  I' &  135\\
         \hline
         (12) &  J' &  \num{69.4}\\
         \hline
         (12) &  K' &  \num{69.6}\\
         \hline
         \multicolumn{3}{c}{}\\
      \end{tabular}   
   \end{minipage}

   \begin{minipage}{1\linewidth}
      Tracer les lignes brisées suivantes : \\ [3mm]
      AC ; CFOXWNEC\\
      GBE; FDJ; KHN; OIL\\
      QMS; TPV; YRW; XUD'\\
      WZE'J'K'I'C'X\\
      F'H'B'G'A'F'
   \end{minipage}
\end{center}

   \DessinGradue[LignesIdentiques=false,Echelle=0.7]{%
   -33/-3/10,%
   0/1/10,%
   0/1/10,%
   -50/50/10,%
   -1/0/10,%
   -20/0/10,%
   0/1000/10,%
   -0.5/0.5/10,%
   3.1/3.2/10,%
   2/3/10,%
   100/150/10,%
   69/70/10}{%
   1/A/5,
   2/B/5,
   3/C/1,
   3/D/5,
   3/E/9,
   4/F/3,
   4/G/7,
   5/H/0,
   5/I/1,
   5/J/9,
   5/K/10,
   6/L/0,
   6/M/10,
   7/N/1,
   7/O/2,
   7/P/8,
   7/Q/9,
   8/R/0,
   8/S/2,
   8/T/3,
   8/U/7,
   8/V/8,
   8/W/10,
   9/X/4,
   9/Y/6,
   10/Z/1,
   10/A'/3,
   10/B'/4,
   10/C'/6,
   10/D'/7,
   10/E'/9,
   11/F'/3,
   11/G'/4,
   11/H'/5,
   11/I'/6,
   11/J'/7,
   12/K'/4,
   12/L'/6
 }{%
   chemin(A,D),
   polygone(B',G',H'),
   polygone(C',H',I'),
   chemin(C,F),
   chemin(C,H),
   chemin(I,O),
   chemin(I,L),
   chemin(N,T),
   chemin(N,R),
   chemin(S,X),
   chemin(S,Z),
   chemin(G,E),
   chemin(E,K),
   chemin(P,J),
   chemin(J,M),
   chemin(U,Q),
   chemin(Q,W),
   chemin(Y,V),
   chemin(V,E'),
   polygone(D,F,O,X,Y,P,G),
   polygone(X,Y,D',J',L',K',F',A')
 }
 \vspace*{-20mm}
 \end{exercice*}
 
 \begin{corrige}
    \ \\ [-4mm]
    \DessinGradue[LignesIdentiques=false,Echelle=0.6,Solution]{%
    -33/-3/10,%
    0/1/10,%
    0/1/10,%
    -50/50/10,%
    -1/0/10,%
    -20/0/10,%
    0/1000/10,%
    -0.5/0.5/10,%
    3.1/3.2/10,%
    2/3/10,%
    100/150/10,%
    69/70/10}{%
    1/A/5,
    2/B/5,
    3/C/1,
    3/D/5,
    3/E/9,
    4/F/3,
    4/G/7,
    5/H/0,
    5/I/1,
    5/J/9,
    5/K/10,
    6/L/0,
    6/M/10,
    7/N/1,
    7/O/2,
    7/P/8,
    7/Q/9,
    8/R/0,
    8/S/2,
    8/T/3,
    8/U/7,
    8/V/8,
    8/W/10,
    9/X/4,
    9/Y/6,
    10/Z/1,
    10/A'/3,
    10/B'/4,
    10/C'/6,
    10/D'/7,
    10/E'/9,
    11/F'/3,
    11/G'/4,
    11/H'/5,
    11/I'/6,
    11/J'/7,
    12/K'/4,
    12/L'/6
  }{%
    chemin(A,D),
    polygone(B',G',H'),
    polygone(C',H',I'),
    chemin(C,F),
    chemin(C,H),
    chemin(I,O),
    chemin(I,L),
    chemin(N,T),
    chemin(N,R),
    chemin(S,X),
    chemin(S,Z),
    chemin(G,E),
    chemin(E,K),
    chemin(P,J),
    chemin(J,M),
    chemin(U,Q),
    chemin(Q,W),
    chemin(Y,V),
    chemin(V,E'),
    polygone(D,F,O,X,Y,P,G),
    polygone(X,Y,D',J',L',K',F',A')
  }
 \end{corrige}