\begin{exercice}
   Rédiger l'énoncé
   Pour découvrir le dessin codé, il faut placer les points A, B, C\dots{} selon les indications du tableau ci-dessous. Par exemple, le point A est sur la première ligne et son abscisse est 8. \\
   Une fois tous les points placés les relier en suivant les instructions données.
   \begin{center}
   %  \let\originalTextwidth\textwidth
   %  \setlength{\linewidth}{25cm}    
    \begin{minipage}{0.4\linewidth}
      \begin{tabular}{|*{3}{>{\centering\arraybackslash}p{0.5cm}|}}
         \hline
         \cellcolor{lightgray}{\!\!\!\small Ligne} & \cellcolor{lightgray}{\!\!\!\small Point} & \cellcolor{lightgray}{\!\small Abs.} \\
         \hline
         (1) & A & 18 \\
         \hline
         % (1) & B & 9 \\
         % \hline
         % (1) & C & 12 \\
         % \hline
         % (2) & D & 17 \\
         % \hline
         % (2) & E & 18 \\
         % \hline
         % (2) & F& 19 \\
         % \hline
         % (2) & G & 20 \\
         % \hline
         % (2) & H & 21 \\
         % \hline
         % (2) & I & 22 \\
         % \hline
         % (2) & J & 23 \\
         % \hline
         % (3) & K & 57 \\
         % \hline
      \end{tabular}   
   \end{minipage}   
   \begin{minipage}{0.5\linewidth}
      Tracer les lignes brisées \\
      suivantes : \\ [3mm]
      % FELMPKDACOTWVY \\
      % C'P’C"B"V’O’W’X’B’XA’  \\ [3mm]
      % GB \\ [3mm]
      % HJNI \\ [3mm]
      % ST \\ [3mm]
      % QRUC’ \\ [3mm]
      % D’E’L’N’U’T’ \\ [3mm]
      % F’I’H’ \\ [3mm]
      % U’V’ \\ [3mm]
      % B"A"S’M’G’K’R’Z’Y’Q’J’ZK \\ [3mm]
      % ZG’
   \end{minipage}
\end{center}

   \DessinGradue[LignesIdentiques=false,Echelle=0.6,Solution]{%
   3/33/10,%
   0/1/10,%
   0/1/10,%
   1/100/10,%
   1/2/10,%
   10/30/10,%
   0/1000/10,%
   0/1/10,%
   3.1/3.2/10,%
   2/3/10,%
   100/150/10,%
   69/70/10}{%
   1/A/5,
   2/B/5,
   3/C/1,
   3/D/5,
   3/E/9,
   4/F/3,
   4/G/7,
   5/H/0,
   5/I/1,
   5/J/9,
   5/K/10,
   6/L/0,
   6/M/10,
   7/N/1,
   7/O/2,
   7/P/8,
   7/Q/9,
   8/R/0,
   8/S/2,
   8/T/3,
   8/U/7,
   8/V/8,
   8/W/10,
   9/X/4,
   9/Y/6,
   10/Z/1,
   10/A'/3,
   10/B'/4,
   10/C'/6,
   10/D'/7,
   10/E'/9,
   11/F'/3,
   11/G'/4,
   11/H'/5,
   11/I'/6,
   11/J'/7,
   12/K'/4,
   12/L'/6
 }{%
   chemin(A,D),
   polygone(B',G',H'),
   polygone(C',H',I'),
   chemin(C,F),
   chemin(C,H),
   chemin(I,O),
   chemin(I,L),
   chemin(N,T),
   chemin(N,R),
   chemin(S,X),
   chemin(S,Z),
   chemin(G,E),
   chemin(E,K),
   chemin(P,J),
   chemin(J,M),
   chemin(U,Q),
   chemin(Q,W),
   chemin(Y,V),
   chemin(V,E'),
   polygone(D,F,O,X,Y,P,G),
   polygone(X,Y,D',J',L',K',F',A')
 }
 \end{exercice}
 
 \begin{corrige}
    \ \\ [-5mm]

 \end{corrige}