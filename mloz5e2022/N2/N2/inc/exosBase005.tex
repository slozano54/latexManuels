\begin{exercice}
    Reproduire l'axe chronologique ci-dessous puis placer le plus précisément possible ces évènements : \\ [2mm]
    \Reperage[Unitex=0.75,ValeurUnitex=100]{4/B,-5/A}
    \begin{itemize}
       \item T : le temple de Jérusalem est détruit en 70 après Jésus-Christ ;
       \item J : Jules César naît en 100 avant J.-C. ;
       \item C : Constantin crée Constantinople en 324 ;
       \item A : Alexandre le Grand meurt en 324 avant J.-C.
    \end{itemize}
 \end{exercice}
 
 \begin{corrige}
    Droite graduée complétée : \\ [2mm]
    \hspace*{-10mm} \Reperage[Unitex=0.75,ValeurUnitex=100,AffichageNom,AffichageAbs=1]{-5/,-1/J,0.7/T,3.24/C,-3.24/A,4/}
 \end{corrige}