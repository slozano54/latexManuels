\begin{exercice}
    Quelle est la température indiquée par chacun des thermomètres ? \\ [3mm]
    \hspace*{-5mm}
        \Reperage[Thermometre,Pasx=10,ValeurUnitex=5,Mercure]{4/A,-16/,16/}
        \Reperage[Thermometre,Pasx=10,ValeurUnitex=10,Mercure]{-12/A,-15/,17/}
        \Reperage[Thermometre,Pasx=5,Mercure]{-2/A,-10/,6/}
        \Reperage[ Thermometre,Pasx=5,ValeurUnitex=0.5,Mercure]{7/A,-5/,11/}
 \end{exercice}
 
 \begin{corrige}
    On peut lire les températures suivantes : \\
    \red \udegc{2} \hfill \udegc{-12} \hfill \udegc{-0,4} \hfill \udegc{0,7}
 \end{corrige}