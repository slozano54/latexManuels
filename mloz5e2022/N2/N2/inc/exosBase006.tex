\begin{exercice*}
    Construire une droite graduée dont l'origine est au milieu du cahier et l'unité vaut \ucm{1} puis répondre aux questions suivantes.
    \begin{enumerate}
       \item Sur la droite graduée, placer les points : \\
          A($+8$), B($-2$), C($+3$), D($-5$) et E($+2$).
       \item En examinant la position des points A, B, C, D et E sur cette droite graduée, comparer : \\ [-9mm]
          \begin{multicols}{3}
             $+2$ et $-2$ \\ \smallskip
             $-2$ et $-5$ \\ \smallskip
             $+2$ et $-5$ \\
             $+8$ et $-2$ \\
             $+3$ et $+8$ \\
             $-5$ et $+3$
          \end{multicols}
          \ \\ [-15mm]
       \item Ranger dans l'ordre croissant : $+8 ; -2 ; +3 ; -5$ et $+2$.
    \end{enumerate}
 \end{exercice*}
 
 \begin{corrige}
    \ \\ [-5mm]
    \begin{enumerate}
       \item Droite graduée complétée (à l'échelle 1/2) : \\ [2mm]
       \hspace*{-10mm} \Reperage[Unitex=0.5,AffichageNom,AffichageAbs=1]{-6/,-5/D,-2/B,2/E,3/C,8/A,9/}
       \item On peut comparer directement par lecture graphique :
          \begin{multicols}{3}
             $+2 \; {\red >} \, -2$ \\ \smallskip
             $-2  \; {\red >} \, -5$ \\ \smallskip
             $+2 \; {\red >} \, -5$ \\
             $+8 \; {\red >} \, -2$ \\
             $+3 \; {\red <} \, +8$ \\
             $-5 \; {\red <} \, +3$
          \end{multicols}
          \vspace*{-3mm}
       \item \red$-5<-2<+2<+3<+8$.
    \end{enumerate}
 \end{corrige}