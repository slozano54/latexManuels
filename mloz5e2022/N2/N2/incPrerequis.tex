\vspace*{-7mm}
%pre-001
\begin{prerequis}[Connaisances \emoji{red-heart} et compétences \emoji{diamond-suit} du cycle 3]    
   \begin{itemize}        
       \item[\emoji{red-heart}] Vocabulaire associé à ces objets et à leurs propriétés : côté, sommet, angle, hauteur.
       \columnbreak
       \item[\emoji{diamond-suit}] Reconnaître, nommer, décrire des triangles, dont les triangles particuliers (triangle rectangle, triangle isocèle, triangle équilatéral).       
   \end{itemize}
\end{prerequis}
\vspace*{-2mm}
%pre-002
\begin{prerequis}[Connaisances \emoji{red-heart} et compétences \emoji{diamond-suit} du cycle 4]    
    \begin{itemize}        
        \item[\emoji{diamond-suit}] Mener des calculs impliquant des grandeurs mesurables, exprimer les résultats dans des les unités adaptées.
        \item[\emoji{diamond-suit}] Exprimer et vérifier la cohérence des résultats du point de vue des unités.
    \end{itemize}
\end{prerequis}
\def\thermo{\pspolygon[linearc=0.1,fillstyle=solid](-0.1,6.8)(-0.1,0.6)(-0.2,0.4)(-0.2,-0.1)(0.2,-0.1)(0.2,0.4)(0.1,0.6)(0.1,6.8) \pspolygon[linecolor=B1, linearc=0.1, fillstyle=solid, fillcolor=B1](0,5)(0,0.5)(-0.1,0.3)(-0.1,0)(0.1,0)(0.1,0.3)(0,0.5) \pscircle(0,6.7){0.05}}
            
\begin{debat}[Les unités de mesure de température]
   Il existe trois échelles principales de température :
   \begin{itemize}
      \item l'échelle Farenheit, créée en 1720 par le scientifique allemand {\bf Gabriel Farenheit} et allant de \udeg{32}F à \udeg{212}F ;
      \item l'échelle Celsius, créée en 1741 par le physicien suédois {\bf Anders Celsius}  dans laquelle \udeg{0}C correspond au point de congélation de l'eau et \udeg{100}C à son point d'ébullition ;
      \item l'échelle de Kelvin, créée à la fin du {\small XIX}\up{e} siècle par {\bf Lord Kelvin} pour laquelle le point 0 correspond au zéro absolu, c'est-à-dire à la plus basse température existante.
   \end{itemize}
   \begin{center}
      {\psset{yunit=0.35}
      \begin{pspicture}(0,0)(8,7)
         \textcolor{B1}{
         \rput(1,0){\thermo}
         \rput[l](1.2,1){\udeg{-459}F}
         \rput[l](1.2,6){\udeg{212}F}
         \rput[l](1.2,4.66){\udeg{32}F}
         \rput(3.5,0){\thermo}
         \rput[l](3.7,1){\udeg{-273}C} 
         \rput[l](3.7,4.66){\udeg{0}C}
         \rput[l](3.7,6){\udeg{100}C}
         \rput(6,0){\thermo}
         \rput[l](6.2,1){\udeg{0}K} 
         \rput[l](6.2,4.66){\udeg{273}K}  
         \rput[l](6.2,6){\udeg{373}K} }          
      \end{pspicture}}
   \end{center}   
   \begin{cadre}[B2][J4]
      \begin{center}
         Vidéo : \href{https://www.yout-ube.com/watch?v=nzirDkQN99M&rel=0}{\bf Celsius et Farenheit}, chaîne YouTube {\it Ma deuxième école}, série {\it Culture G}.
      \end{center}
   \end{cadre}
\end{debat}