\begin{exercice*}
   Un quadrilatère $ABCD$ est appelé isocerfvolant en $A$ si l'angle $\widehat{BAD}$ est droit et si la droite $(AC)$ est un axe de symétrie.
   \begin{enumerate}
      \item
      \begin{enumerate}
         \item Construire un quadrilatère ABCD qui soit un isocerfvolant en A.
         \item Construire un quadrilatère qui admette un axe de symétrie et qui ne soit pas un isocerfvolant.
      \end{enumerate}
      \item
      \begin{enumerate}
         \item Dans un isocerfvolant, montrer que $\widehat{DAC} =\widehat{BAC}$
         \item En déduire la mesure de l'angle $\widehat{DAC}$
         \item Quelle est la position relative de $(BD)$ et $(AC)$ ?
      \end{enumerate}
      \item Les affirmations suivantes sont-elles vraies ou fausses ? Justifier les réponses.
      \begin{enumerate}
         \item Tous les carrés sont des isocerfvolants.
         \item Tous les rectangles sont des isocerfvolants.
      \end{enumerate}
   \end{enumerate}
\end{exercice*}

\begin{corrige}
   \ \\ [-5mm]
   \begin{enumerate}
      \item 
      \begin{enumerate}
         \item Quadrilatère $ABCD$ isocerfvolant en $A$ : \\
            \begin{pspicture}(-1.5,-1.5)(6,1.7)
               \pstGeonode[PointSymbol=none,CurveType=polygon,PosAngle={135,-90,45,90}]{A}(1,-1){B}(4,0){C}(1,1){D}
               \pstLineAB[linecolor=red,nodesep=-1]{A}{C}
               \pstRightAngle[linecolor=B1]{D}{A}{B}
               \pstLineAB[]{A}{B}
               \pstLineAB[]{B}{C}
               \pstLineAB[]{C}{D}
               \pstLineAB[]{D}{A}
            \end{pspicture}
         \item Quadrilatère $ABCD$ non isocervolant ayant un axe de symétrie : \\
            \begin{pspicture}(-1.5,-2.8)(6,2.8)
               \pstGeonode[PointSymbol=none,CurveType=polygon,PosAngle={135,-90,45,90}]{A}(1,-2){B}(4,0){C}(1,2){D}
               \pstLineAB[linecolor=red,nodesep=-1]{A}{C}
               \pstLineAB[]{A}{B}
               \pstLineAB[]{B}{C}
               \pstLineAB[]{C}{D}
               \pstLineAB[]{D}{A}
            \end{pspicture}
      \end{enumerate}
      \setcounter{enumi}{1}
      \item 
      \begin{enumerate}     
         \item $(AC)$ est un axe de symétrie de la figure $ABCD$, donc par conservation des angles, $\red \widehat{DAC} =\widehat{BAC}$.
         \item $\widehat{DAB} =\udeg{90}$, donc $\widehat{DAC}$ vaut la moitié, c'est-à-dire $\red \udeg{45}$.
         \item $B$ est le symétrique de $D$ par rapport à $(AC)$ donc, {\red les droites $(BD)$ et $(AC)$ sont perpendiculaires}.
      \end{enumerate}
   \setcounter{enumi}{2}
   \item
      \begin{enumerate}
         \item {\red Un carré est un isocervolant} car ses diagonales sont des axes de symétrie et tous ses angles mesurent \udeg{90}.
         \item {\red Un rectangle n'est pas un isocervolant} car ses diagonales ne sont pas des axes de symétrie.
      \end{enumerate}
   \end{enumerate}
\end{corrige}