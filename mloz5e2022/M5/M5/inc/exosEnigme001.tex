% Les enigmes ne sont pas numérotées par défaut donc il faut ajouter manuellement la numérotation
% si on veut mettre plusieurs enigmes
% \refstepcounter{exercice}
% \numeroteEnigme
\begin{enigme}[Entrelacs chinois]

\medskip

   \begin{minipage}{12cm}
      Dans le rectangle quadrillé ci-dessous, les axes de symétrie ont été tracés en rouge, ils se coupent au point O. Le but est de reproduire l'entrelacs ci-contre, formé de trois rubans entrelacés, sachant que :
      \begin{itemize}
          \item l'épaisseur de ces rubans est de 1 carreau ;
            \item les dimensions de la figure complète sont de 17 carreaux sur 19 carreaux ;
            \item tous les segments sont portés par des lignes du quadrillages ;
            \item les cercles ont pour centre commun le point O ;
            \item les extrémités de leurs diamètres verticaux sont sur des lignes horizontales ;
            \item cette figure admet un centre de symétrie mais pas d'axe de symétrie.
         \end{itemize}
      \end{minipage}
      \qquad
      \begin{minipage}{4.5cm}
         \includegraphics[width=4cm]{\currentpath/images/entrelacs}
      \end{minipage}
   \begin{center}
      {\psset{unit=0.9}
      \begin{pspicture}(0,0.5)(17,19.3)
         \psgrid[subgriddiv=0,gridlabels=0,gridcolor=lightgray](0,0)(17,19)
         \psline[linecolor=red](8.5,0)(8.5,19)
         \psline[linecolor=red](0,9.5)(17,9.5)
       \end{pspicture}}
   \end{center}
\end{enigme}

% Pour le corrigé, il faut décrémenter le compteur, sinon il est incrémenté deux fois
% \addtocounter{exercice}{-1}
\begin{corrige}
    {\red
    \dots
    }
 \end{corrige}