\begin{exercice*}
   Soit $ABC$ un triangle isocèle en $A$ tel que \\
   $BC=\ucm{3}$ et $BA=\ucm{4}$.
   \begin{enumerate}
      \item Construire le triangle $ABC$.
      \item Construire le symétrique de $ABC$ par apport à $A$ \\
         ($D$ est le symétrique de $B$ et $E$ celui de $C$).
      \item Construire le milieu $I$ de $[BC]$ et $J$ celui de $[DE]$.
      \item Démontrer que les trois points $J, A$ et $I$ sont alignés. Que représente la droite $(IJ)$ pour les segments $[BC]$ et $[DE]$ ?
   \end{enumerate}
\end{exercice*}

\begin{corrige}
   \ \\ [-5mm]
   \begin{pspicture}(-2,-1)(5,8.2)
      \psset{PointSymbol=none}
      \pstGeonode[PosAngle={-135,-45,0,45,90}]{B}(3,0){C}(1.5,3.71){A}(1.5,0){I}(1.5,7.42){J}
      \pstLineAB{B}{C}
      \pstSymO[PosAngle={45,135}]{A}{B,C}[D,E]
      \pstSegmentMark{A}{B}
      \pstSegmentMark{A}{C}
      \pstLabelAB[offset=-3mm]{B}{C}{\ucm{3}}
      \pstLabelAB{B}{A}{\ucm{4}}
      \pstSegmentMark[SegmentSymbol=pstslash]{B}{I}
      \pstSegmentMark[SegmentSymbol=pstslash]{C}{I}
      \pstSegmentMark[SegmentSymbol=pstslash]{E}{J}
      \pstSegmentMark[SegmentSymbol=pstslash]{D}{J}
      \psset{linecolor=B1}
      \pstSegmentMark{A}{D}
      \pstSegmentMark{A}{E}
      \pstLineAB{E}{D}
      \pstLineAB[linecolor=red]{I}{J}
   \end{pspicture} \\
   Démonstration de la question 4 : \\
   \begin{itemize}
      \item Le triangle $ABC$ est isocèle en $A$ et $I$ est le milieu de $[BC]$ donc, la droite $(AI)$ est la hauteur issue de $A$ dans le triangle $ABC$, mais aussi la médiatrice du segment $[BC]$. On a alors $(AI) \perp (BC)$. \smallskip
      \item Le triangle $ADE$ est le symétrique du triangle $ABC$ par la symétrie centrale de centre $A$, par conservation des angles, on a $(AJ)\perp(DE)$. \smallskip
      \item De plus, la droite $(BC)$ est transformée en la droite $(DE)$ qui lui est parallèle donc, on a $\left.\begin{array}{c} (AI) \perp (BC)\,\\\,(AJ)\perp(DE) \\ (BC) // (DE) \end{array}\right\}$ soit $(AI) // (AJ)$. \smallskip
      \item Par conséquent, {\red les points $J$, $A$ et $I$ sont alignés.}
      \item La droite $(IJ)$ est la {\red médiatrice} des segments $[BC]$ et $[DE]$.
   \end{itemize}
\end{corrige}
