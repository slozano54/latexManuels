%pre-001
\begin{prerequis}[Connaisances \emoji{red-heart} et compétences \emoji{diamond-suit} du cycle 3]    
   \begin{itemize}        
       \item[\emoji{red-heart}] Vocabulaire associé à ces objets et à leurs propriétés : côté, sommet, angle, hauteur.
       \columnbreak
       \item[\emoji{diamond-suit}] Reconnaître, nommer, décrire des triangles, dont les triangles particuliers (triangle rectangle, triangle isocèle, triangle équilatéral).       
   \end{itemize}
\end{prerequis}
%pre-002
\begin{prerequis}[Connaisances \emoji{red-heart} et compétences \emoji{diamond-suit} du cycle 4]    
    \begin{itemize}        
        \item[\emoji{diamond-suit}] Mener des calculs impliquant des grandeurs mesurables, exprimer les résultats dans des les unités adaptées.
        \item[\emoji{diamond-suit}] Exprimer et vérifier la cohérence des résultats du point de vue des unités.
    \end{itemize}
\end{prerequis}
\vfill
\begin{debat}[histoire des probabilités]
   \hspace*{-10mm}
   \begin{minipage}{0.8\linewidth} 
      C’est en cherchant à résoudre des problèmes posés par les jeux de hasard que les mathématiciens donnent naissance aux {\bf probabilités}. Lors de fouilles archéologiques, on a trouvé des indices montrant que les jeux de hasard se pratiquaient déjà 5\,000 ans av. J.-C. (on utilisait des osselets). Les premiers dés connus ont été mis à jour à {\it Tepe Gawra}, au nord de l’Irak, et datent du troisième millénaire av. J.-C. Le jeu de cartes était également pratiqué dans divers pays depuis des époques reculées. Les cartes actuelles apparaissent en France au {\small XIV}\up{e} siècle et leur utilisation donne très vite lieu à des jeux d’argent. On attribue souvent la réelle naissance à la fin du {\small XVII}\up{e} siècle ce qui en fait une branche des mathématiques relativement récente.
   \end{minipage}
   \begin{minipage}{0.3\linewidth}
         \scalebox{0.9}{
         \begin{pspicture}(1,0)(8,3)
            \rput(2,1){\psdice[linecolor=G1]{6}}
            \rput(5,2){\psdice[linecolor=A1]{5}}
            \rput(4,0.5){\psdice[linecolor=B1]{4}}
            \rput(7,2.5){\psdice[linecolor=J1]{3}}
            \rput(6,0.75){\psdice[linecolor=gray]{2}}
            \rput(3,2.5){\psdice[linecolor=H1]{1}}
         \end{pspicture}
         }
   \end{minipage}
   \vfill   
   \begin{cadre}[B2][J4]
      \begin{center}
         \hrefVideo{https://leblob.fr/fondamental/les-probabilites}{\bf Les probabilités}, {\it Petits contes mathématiques}.
      \end{center}
   \end{cadre}
\end{debat}
