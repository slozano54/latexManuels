\begin{exercice*}
    Pour chacun des événements suivants, indiquer s'il relève du hasard et si oui, le placer sur l'échelle ci-dessous.
    \begin{center}
       \begin{pspicture}(0,-0.5)(8,0.5)
          \psline{->}(0.3,0)(7.7,0)
          \rput(4,0){|}
          \footnotesize
          \rput{90}(0.1,-0.2){impossible}
          \rput{90}(7.9,0){certain}
          \rput(1.2,0.3){improbable}
          \rput(3,0.3){peu probable}
          \rput(4.8,0.3){probable}
          \rput(6.5,0.3){très probable}
       \end{pspicture}
    \end{center}
    \begin{enumerate}
       \item Obtenir pile au jeu de pile ou face \dotfill
       \item La fête nationale aura lieu le 14 juillet \dotfill
       \item Un élève aura des basquettes demain \dotfill
       \item Obtenir 6 avec un dé à six faces \dotfill
       \item Trouver la bonne combinaison au loto \dotfill
       \item Demain il fera beau \dotfill
    \end{enumerate}
    \hrefMathaleaVIII{86db6}
\end{exercice*}
\begin{corrige}
    %\setcounter{partie}{0} % Pour s'assurer que le compteur de \partie est à zéro dans les corrigés
    %\phantom{rrr}
    \ \\ [-5mm]
    \begin{enumerate}
       \item Obtenir pile au jeu de pile ou face : {\red hasard}.
       \item La fête nationale aura lieu le 14 juillet.
       \item Un élève aura des basquettes demain : {\red hasard}.
       \item Obtenir 6 avec un dé à six faces : {\red hasard}.
       \item Trouver la bonne combinaison au loto : {\red hasard}.
       \item Demain il fera beau : {\red hasard}.
    \end{enumerate}
    \begin{pspicture}(0.7,-0.9)(8,0.9)
       \psline{->}(0.3,0)(7.7,0)
       \rput(4,0){|}
       \footnotesize
       \rput{90}(0.1,0){impossible}
       \rput{90}(7.9,0){certain}
       \rput(1.2,0.3){improbable}
       \rput(3,0.3){peu probable}
       \rput(4.8,0.3){probable}
       \rput(6.5,0.3){très probable}
       \rput(5,-0.4){\red 1}
       \rput(7.7,-0.4){\red 2}
       \rput(6,-0.4){\red 3}
       \rput(3,-0.4){\red 4}
       \rput(1,-0.4){\red 5}
       \rput(3.9,-0.4){\red 6}
    \end{pspicture}
\end{corrige}

