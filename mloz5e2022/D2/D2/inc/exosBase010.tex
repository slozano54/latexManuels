\begin{exercice*}
  Dans le frigo, il y a 21 desserts lactés. 2 sont au chocolat, 7 sont à la vanille, 2 sont au café, 6 sont à la pistache et 4 sont au caramel.
  
  David choisit au hasard l'un d'entre eux. Il regarde le parfum.
  \begin{enumerate}
    \item Est-ce que c'est une expérience aléatoire ? Pourquoi ?
    \item Quelles sont les issues ?
    \item Quelles issues réalisent l'événement \og{} Tomber sur l'un des desserts lactés au caramel ou au café \fg{} ?
    \item Quelles issues ne réalisent pas l'événement \og{} Tomber sur l'un des desserts lactés à la pistache ou au caramel \fg{} ?
  \end{enumerate}  
\end{exercice*}
\begin{corrige}
\begin{enumerate}
  \item On ne connait pas le parfum du dessert lacté sur lequel on va tomber.  
  On ne peut pas prévoir à l'avance le résultat et on peut recommencer l'expérience dans les mêmes conditions \textcolor{red}{c'est donc bien une expérience aléatoire}.
  \item Les issues de cette expérience sont :
  \begin{itemize}
  \item \textcolor{red}{chocolat} ;
  \item \textcolor{red}{vanille} ;
  \item \textcolor{red}{café} ;
  \item \textcolor{red}{pistache} ;
  \item \textcolor{red}{caramel}.
  \end{itemize}
  \item Issues réalisant l'événement \og{} Tomber sur l'un des desserts lactés au caramel ou au café \fg{} :
  \begin{itemize}
    \item \textcolor{red}{caramel} ;
    \item \textcolor{red}{café}.
  \end{itemize}
  \item Issues ne réalisant pas l'événement \og{} Tomber sur l'un des desserts lactés au caramel ou au café \fg{} :
  \begin{itemize} 
          \item \textcolor{red}{chocolat} ;
          \item \textcolor{red}{vanille } ;
          \item \textcolor{red}{pistache}.
  \end{itemize}
\end{enumerate}
\end{corrige}