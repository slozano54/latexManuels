\begin{exercice*}
    Une roue de loterie est partagée en huit secteurs identiques numérotés de 1 à 8. Calculer la probabilité de chaque événement et les placer sur l'échelle.
    \begin{center}
       \begin{pspicture}(0.7,-0.5)(8,1)
          \psline{->}(0.3,0)(7.7,0)
          \rput(4,0){|}
          \footnotesize
          \rput{90}(0.1,-0.2){impossible}
          \rput{90}(7.9,0){certain}
          \rput(1.2,0.3){improbable}
          \rput(3,0.3){peu probable}
          \rput(4.8,0.3){probable}
          \rput(6.5,0.3){très probable}
       \end{pspicture}
    \end{center}
    \begin{enumerate}
       \item \og Obtenir 2. \fg
       \item \og Obtenir un multiple de 2. \fg
       \item \og Obtenir un nombre supérieur à 4. \fg
       \item \og Obtenir un nombre positif. \fg
       \item \og Obtenir un nombre impair. \fg
       \item \og Obtenir un multiple de 13. \fg
    \end{enumerate}
\end{exercice*}
\begin{corrige}
    %\setcounter{partie}{0} % Pour s'assurer que le compteur de \partie est à zéro dans les corrigés
    %\phantom{rrr}
    \ \\ [-5mm]
    \begin{enumerate}
       \item Obtenir 2 : $\red P =\dfrac18$
       \item Obtenir un multiple de 2 : $\red P =\dfrac48$
       \item Obtenir un nombre supérieur à 5 : $\red P =\dfrac38$
       \item Obtenir un nombre positif : $\red P =\dfrac88 =1$
       \item Obtenir un nombre impair : $\red P =\dfrac48$
       \item Obtenir un multiple de 13 : $\red P =\dfrac08 =0$
    \end{enumerate}
    \begin{pspicture}(0.7,-1)(8,0.9)
       \psline{->}(0.3,0)(7.7,0)
       \rput(4,0){|}
       \footnotesize
       \rput{90}(0.1,0){impossible}
       \rput{90}(7.9,0){certain}
       \rput(1.2,0.3){improbable}
       \rput(3,0.3){peu probable}
       \rput(4.8,0.3){probable}
       \rput(6.5,0.3){très probable}
       \rput(3,-0.4){\red 1}
       \rput(5,-0.4){\red 2 et 5}
       \rput(4,-0.4){\red 3}
       \rput(7.5,-0.4){\red 4}
       \rput(0.5,-0.4){\red 6}
    \end{pspicture}
\end{corrige}    