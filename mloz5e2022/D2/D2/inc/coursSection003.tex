\section{Quelques événements particuliers}
\begin{definition}
    \begin{itemize}
        \item L'\textbf{événement "A et B"} se produit lorsque les événements A et B ont lieux simultanément.
        \item L'\textbf{événement "A ou B"} se produit lorsque l'événement A ou l'événement  B ou les deux ont lieux.
        \item Deux événements sont dits \textbf{incompatibles} s'ils ne peuvent se produire en même temps.
        \item Deux événements sont dits \textbf{contraires} si la réalisation de l'un correspond à la non réalistation de l'autre.
        \item On note $\mathbf{\overline{A}}$ l'événement contraire de l'événement $\mathbf{A}$
    \end{itemize}
\end{definition}

\begin{exemples*1}
    \titreExemple{On lance un dé à 6 faces}

    \begin{itemize}
        \item P l'événement \textbf{"obtenir un nombre pair"}
        \item S \textbf{"obtenir un nombre supérieur à 3"}
        \item (P et S) est donc l'événement \textbf{"obtenir un nombre pair supérieur ou égal à 3"}.
        \item I l'événement \textbf{"sortir un nombre inférieur ou égal à 4"}
        \item (P ou I) est donc l'événement \textbf{"obtenir un nombre pair inférieur ou égal à 4"} c'est à dire \textbf{"obtenir 2 ou 4"}  
        \item \textbf{A} l'événement \textbf{"obtenir 1 ou 2"}
        \begin{list}{$\gtrdot$}{}
            \item $\overline{A}$ par un phrase sans négation \textbf{"obtenir 3,4,5 ou 6"}.
            \item On peut calculer la probabilité de $\overline{A}$ de deux manières qui donnent le même résultat :
            \begin{enumerate}
                \item 4 issues favorables sur 6 issues possibles donc $p(\overline{A})=\dfrac{4}{6}$
                \item $\overline{A}$ est l'événement contraire de $A$ or $p(A)=\dfrac{2}{6}$ \\
                donc $p(\overline{A})=1-p(A)=1-\dfrac{2}{6}=\dfrac{6}{6}-\dfrac{2}{6}=\dfrac{4}{6}$
            \end{enumerate}
        \end{list}
    \end{itemize}
\end{exemples*1}
