\begin{exercice*}
    Trois personnes, Ali, Ben et Charles, ont chacune un sac contenant des billes. Chacune tire au hasard une bille de son sac dont le contenu est le suivant : \\ [2mm]
    {\renewcommand{\arraystretch}{1.3}
    \begin{ltableau}{\linewidth}{3}
       \hline
       Sac d'Ali & Sac de Ben & Sac de Charles \\
       \hline
       10 billes rouges & 97 billes rouges & 5 billes rouges \\
       30 billes noires & 3 billes noires & \\
       \hline
    \end{ltableau}
    }
 Laquelle de ces trois personnes a-t-elle la plus grande probabilité de tirer une bille rouge ? Justifier.
\end{exercice*}
\begin{corrige}
    %\setcounter{partie}{0} % Pour s'assurer que le compteur de \partie est à zéro dans les corrigés
    %\phantom{rrr}
    probabilité de tirer une bille rouge :
    \begin{itemize}
       \item Pour Ali : $P =\dfrac{10}{40} =0,25$. \smallskip
       \item Pour Ben : $P =\dfrac{97}{100} =0,97$. \smallskip
       \item Pour Charles : $P =\dfrac{5}{5} =1$. \smallskip
    \end{itemize}
   {\red Charles a la plus grande probabilité d'obtenir une bille rouge}, ce qui est logique puisqu'il n'a QUE des billes rouges. \\
\end{corrige}    