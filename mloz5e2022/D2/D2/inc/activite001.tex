\begin{activite}[Imposible, probable o seguro ?]
    {\bf Objectifs} : placer un événement sur une échelle de probabilité. 
    \partie[traduction]
        Traduire en français les six vignettes de cette illustration.
        \begin{center}
        \includegraphics[scale=0.6]{\currentpath/images/BD}
        \end{center}
        \ \\ [-19mm]
        \begin{doublespacing}
        \begin{itemize}
            \item Vignette 1 : \makebox[0.7\linewidth]{\dotfill}
            \item Vignette 2 : \makebox[0.7\linewidth]{\dotfill}
            \item Vignette 3 : \makebox[0.7\linewidth]{\dotfill}
            \item Vignette 4 : \makebox[0.7\linewidth]{\dotfill}
            \item Vignette 5 : \makebox[0.7\linewidth]{\dotfill}
            \item Vignette 6 : \makebox[0.7\linewidth]{\dotfill}
        \end{itemize}
        \end{doublespacing}
        \medskip
    \partie[exploitation]
        Classer ces vignettes sur l'échelle ci-dessous en indiquant le numéro de la vignette en dessous de l'échelle.
        \begin{center}
        \begin{pspicture}(0,-0.3)(12,0.5)
            \psline{->}(0,0)(12,0)
            \footnotesize
            \rput(0,0.3){imposible}
            \rput(4,0.3){poco probable}
            \rput(8,0.3){bastante probable}
            \rput(12,0.3){seguro}
        \end{pspicture}
        \end{center}
    \vfill\hfill {\it\footnotesize Source : Une initiation aux probabilités par le jeu, IREM de Rouen.}
\end{activite}
 