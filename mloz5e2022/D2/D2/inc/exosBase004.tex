\begin{exercice*}
    On dispose d’un dé à six faces numérotées de 1 à 6 et d’un dé à quatre faces avec des sommets numérotés de 1 à 4, parfaitement équilibrés. On lance les deux dés.
    \begin{enumerate}
       \item Avec quel dé la probabilité d’obtenir un 3 est-elle la plus
 grande ?
       \item Avec quel dé la probabilité d’obtenir un multiple de 3 est-elle la plus grande ?
    \end{enumerate}
\end{exercice*}
\begin{corrige}
    %\setcounter{partie}{0} % Pour s'assurer que le compteur de \partie est à zéro dans les corrigés
    %\phantom{rrr}
    \ \\ [-5mm]
    \begin{enumerate}
       \item Dé à six faces : $P =\dfrac16$ (obtenir 3) ; \\ [1mm]
       dé à quatre faces : $P =\dfrac14$ (obtenir 3). \\ [1mm]
       C'est avec le {\red dé à quatre faces} que la probabilité d'obtenir un 3 est la plus grande. \smallskip
       \item Dé à six faces : $ P =\dfrac26$ (obtenir 3 ou 6) ; \\ [1mm]
       dé à quatre faces : $P =\dfrac14$ (obtenir 3). \\ [1mm] 
       C'est avec le {\red dé à six faces} que la probabilité d'obtenir un multiple de 3 est la plus grande.
    \end{enumerate}
\end{corrige}    