\begin{exercice*}
 Dans un tas de jetons de poker, il y a 18 jetons.
 4 sont rouges, 4 sont verts, 4 sont bleus, 3 sont noirs et 3 sont jaunes.
 
 Benoît choisit au hasard l'un d'entre eux.
 
 \begin{enumerate}
  \item Quelle est la probabilité que son choix tombe sur l'un des jetons jaunes ?
  \item Quelle est la probabilité que son choix tombe sur l'un des jetons verts ?
  \item Quelle est la probabilité que son choix ne tombe pas sur l'un des jetons rouges ?
  \item Quelle est la probabilité que son choix tombe sur l'un des jetons jaunes ou verts ?
 \end{enumerate}  
\end{exercice*}
\begin{corrige}
  \begin{enumerate}
    \item Il y a 3 jetons jaunes sur 18 jetons possibles.
    
    La probabilité que son choix tombe sur l'un des jetons jaunes est donc de 3 chances sur 18, soit une probabilité de \textcolor{red}{$\dfrac{3}{18}$}.
    \item Il y a 4 jetons verts sur 18 jetons possibles.
    
    La probabilité que son choix tombe sur l'un des jetons verts est donc de 4 chances sur 18, soit une probabilité de \textcolor{red}{$\dfrac{4}{18}$}.
    \item Il y a 4 jetons rouges sur 18 jetons possibles, donc 14 jetons ne sont pas rouges.
    
    La probabilité que son choix ne tombe pas sur l'un des jetons rouges est donc de 14 chances sur 18, soit une probabilité de \textcolor{red}{$\dfrac{14}{18}$}.
    \item Il y a 3 jetons jaunes et 4 jetons verts, soit 7 jetons jaunes ou verts sur 18 jetons possibles.
    
    La probabilité que son choix tombe sur l'un des jetons jaunes ou verts est donc de 7 chances sur 18, soit une probabilité de \textcolor{red}{$\dfrac{7}{18}$}.
  \end{enumerate}   
\end{corrige}