% Les enigmes ne sont pas numérotées par défaut donc il faut ajouter manuellement la numérotation
% si on veut mettre plusieurs enigmes
\refstepcounter{exercice}
% \numeroteEnigme
\begin{enigme}[Vers la loi des grands nombres\dots]

    {\renewcommand{\arraystretch}{1.6}
    \begin{enumerate}
       \item Compléter le tableau suivant : \\ [1mm] 
       \begin{Ltableau}{0.95\linewidth}{7}{c}
          \hline
          Numéro sur la face visible du dé & 1 & 2 & 3 & 4 & 5 & 6 \\
          \hline
          Probabilité d'obtenir cette face (fraction) & & & & & & \\
          \hline
          Probabilité d'obtenir cette face (décimal) & & & & & & \\
          \hline
       \end{Ltableau} \bigskip
       Que remarque-t-on ? \dotfill \\ [3mm]
       \hspace*{-5mm}
       {\bf Le travail s'effectue maintenant en binôme, vous avez à votre disposition un dé classique à six faces.} \\
       \item Lancer 10 fois le dé et noter les résultats obtenus dans le tableau suivant : \\ [1mm]
       \begin{Ltableau}{0.95\linewidth}{7}{c}
          \hline
          Numéro sur la face visible du dé & 1 & 2 & 3 & 4 & 5 & 6 \\
          \hline
          Nombre de fois où cette face est obtenue & & & & & & \\
          \hline
          Probabilité d'obtenir cette face (fraction) & & & & & & \\
          \hline
          Probabilité d'obtenir cette face (décimal) & & & & & & \\
          \hline
       \end{Ltableau} \bigskip
       Que remarque-t-on ? \dotfill \\
       \item Lancer 100 fois le dé et noter les résultats obtenus dans le tableau suivant : \\ [1mm]
       \begin{Ltableau}{0.95\linewidth}{7}{c}
          \hline
          Numéro sur la face visible du dé & 1 & 2 & 3 & 4 & 5 & 6 \\
          \hline
          Nombre de fois où cette face est obtenue & & & & & & \\
          \hline
          Probabilité d'obtenir cette face (fraction) & & & & & & \\
          \hline
          Probabilité d'obtenir cette face (décimal) & & & & & & \\
          \hline
       \end{Ltableau} \bigskip
       Que remarque-t-on ? \dotfill \\
       \item Répertorier les résultats de la classe entière et noter les résultats obtenus dans le tableau suivant : \\ [1mm]
       \begin{Ltableau}{0.95\linewidth}{7}{c}
          \hline
          Numéro sur la face visible du dé & 1 & 2 & 3 & 4 & 5 & 6 \\
          \hline
          Nombre de fois où cette face est obtenue & & & & & & \\
          \hline
          Probabilité d'obtenir cette face (fraction) & & & & & & \\
          \hline
          Probabilité d'obtenir cette face (décimal) & & & & & & \\
          \hline
       \end{Ltableau} \bigskip
       Que remarque-t-on ? \dotfill
    \end{enumerate}}
\end{enigme}

% Pour le corrigé, il faut décrémenter le compteur, sinon il est incrémenté deux fois
\addtocounter{exercice}{-1}
\begin{corrige}
    Pour le premier tableau, on obtient : \\
    \begin{ltableau}{\linewidth}{7}
       \hline
       N° & 1 & 2 & 3 & 4 & 5 & 6 \\
       \hline
       \small Proba & 1/6 & 1/6 & 1/6 & 1/6 & 1/6 & 1/6 \\
       \hline
       \small Proba & $0,17$ & $0,17$ & $0,17$ & $0,17$ & $0,17$ & $0,17$ \\
       \hline
    \end{ltableau}
    Les probabilités sont, en théorie, toutes identiques.
\end{corrige}