\begin{exercice*} %5
   Au collège de Ginan, le foyer prend en charge 25\,\% du prix des voyages scolaires alors que dans celui de Ludovico, le foyer donne \Prix{54} pour un voyage de \Prix{180} et l'aide est proportionnelle au coût du voyage.
   \begin{enumerate}
      \item Si Ginan participe à un voyage qui coûte \Prix{230}, quel montant est pris en charge par son foyer ?
      \item En proportion, dans quel collège le foyer participe-t-il le plus au financement des voyages ?
   \end{enumerate}
\end{exercice*}

\begin{corrige}
   \ \\ [-5mm]
   \begin{enumerate}
      \item $\dfrac{25}{100}\times\Prix{230} =\Prix{57,5}$. \\ [2mm]
         Sur \Prix{230}, {\red \Prix{57,5} sont pris en charge par le foyer}.
      \item Au collège de Ginan, le pourcentage pris en charge est de $\dfrac{\Prix{54}}{\Prix{180}}\times100 =30\%$. \\ [2mm]
      C'est au {\red collège de Ginan que la proportion prise en charge par le foyer est la plus élevée}.
   \end{enumerate}
\end{corrige}
