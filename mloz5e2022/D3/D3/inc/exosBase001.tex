\begin{exercice*} %1
   Ces situations sont-elles proportionnelles ? \\
   Justifier par un contre-exemple ou une preuve.
   \begin{enumerate}
      \item Taille en mètre en fonction de l'âge ?
      \item Périmètre du carré en fonction de son côté ?
      \item Aire du carré en fonction de son côté ?
      \item Distance parcourue à vélo à vitesse constante en fonction du temps.
   \end{enumerate}
\end{exercice*}

\begin{corrige}
   \ \\ [-5mm]
   \begin{enumerate}
      \item Taille en mètre en fonction de l'âge : {\red non}. \\
         Par exemple, si un enfant mesure \Lg[cm]{75} à 1 an, il ne mesurera pas $\Lg[cm]{750} =\Lg[m]{7,5}$ à 10 ans.
      \item Périmètre du carré en fonction de son côté : {\red oui}. \\
         $\mathcal{P} =4\times c$, 4 étant une constante.
      \item Aire du carré en fonction de son côté  : {\red non}. \\
         Par exemple, un carré de côté \Lg[cm]{1} a une aire de \Aire[cm]{1} et un carré de côté \Lg[cm]{2} a une aire de \Aire[cm]{4}.
      \item Distance parcourue à vitesse constante : {\red oui}. \\
         $d =v\times t$, la vitesse étant constante.
   \end{enumerate}
\end{corrige}
