\section{Reconnaître une situation de proportionnalité}

Pour reconnaître des grandeurs proportionnelles, on peut vérifier qu'il existe un coefficient de proportionnalité entre elles.
\vspace*{-5mm}

\begin{exemple}
\ \\ [-10mm]
   \begin{itemize}
      \item Le périmètre d'un cercle est-il proportionnel à son rayon ?
      \item L'aire d'un disque est-elle proportionnelle à son rayon ?
   \end{itemize}
   \correction
    \ \\ [-10mm]
       \begin{itemize}
         \item $p =$\fbox{$2\times\pi$}$\times r$. Le coefficient $2\times\pi$ est constant, le périmètre est donc proportionnel à son rayon.
         \item  $A =\pi\times r^2 =$\fbox{$\pi\times r$}$\times r$. La variable $\pi\times r$ varie en fonction de $r$, l'aire n'est donc pas proportionnelle à son rayon.
      \end{itemize}
  \end{exemple}