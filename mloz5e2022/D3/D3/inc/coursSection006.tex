\section{Déterminer un pourcentage}
\begin{exemple*1}
   \textit{On se demande quel est le pourcentage d'élèves de $5^{\grave{e}me}$ qui sont demi-pensionnaires. Aujourd'hui sur les \ldots \ldots \ldots présents, il y a \ldots \ldots \ldots demi-pensionnaires.}

   "Si dans la classe il y avait eu $100$ présents alors combien y aurait-il de demi-pensionnaires?"

   C'est une situation de proportionnalité.

   \begin{center}
      \begin{tabular}{|c|c|c|}
         \hline
         Nombre d'élèves présents&$100$&$n_{pr\acute{e}sents}$\\
         \hline
         Nombre de demi-pensionnaires&$n$&$n_{DP}$\\
         \hline
      \end{tabular}
   \end{center}

   Les rapports $\dfrac{n}{100}$ et $\dfrac{n_{DP}}{n_{pr\acute{e}sents}}$ sont égaux d'où
   $n=\dfrac{n_{DP}}{n_{pr\acute{e}sents}}\times 100$
\end{exemple*1}
