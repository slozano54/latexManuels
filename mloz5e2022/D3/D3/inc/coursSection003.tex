\section{Tableau de proportionnalité}
\begin{definition}
   Lorsque, dans un tableau, le \textbf{rapport, ou quotient} de deux valeurs correspondantes est \textbf{constant} c'est à dire qu'il \textbf{vaut toujours la même chose}, c'est un \textbf{tableau de proportionnalité}.
\end{definition}
\begin{exemple*1}
   \begin{center}
      \Propor[Stretch=1.25,%
         Math,%
         GrandeurA=1\up{ere} ligne : $x$,%
         GrandeurB=2\up{eme} ligne : $y$,%
      ]{0.5/1.5,1/3,1.5/4.5,2.5/7.5}
   \end{center}
   \FlechesPD{1}{2}{$\times\num{3}$}
   \FlechesPG{2}{1}{$\div\num{3}$}

   \smallskip
   $\dfrac{\num{1.5}}{\num{.5}} =\dfrac{3}{1}=\dfrac{\num{4.5}}{\num{1.5}}=\dfrac{\num{7.5}}{\num{2.5}}$   
   Ce nombre est \textbf{constant} et \textbf{plus grand que 1}.
   C'est le \textcolor{mygreen}{\textbf{coefficient de proportionnalité}} du tableau.

   Il permet d'exprimer $y$ en fonction de $x$ : $y=\textcolor{mygreen}{\mathbf{3}}\times x$.
\end{exemple*1}

\begin{remarque}
   $\dfrac{0,5}{1,5} =\dfrac{1}{3} =\dfrac{1,5}{4,5}=\dfrac{2,5}{7,5}=\dfrac13$ est aussi constant mais inférieur à 1.

   \smallskip
   C'est \textbf{l'autre coefficient du tableau}.
\end{remarque}
