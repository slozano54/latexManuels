\begin{exercice*} %6
   Selene fait pour ses amis deux verres contenant des boissons au sirop.
      \begin{enumerate}
         \item Le premier verre a une contenance de \Capa[cL]{20} et il y a 3,5\,\% de sirop. Combien cela fait-il de sirop ?
         \item Le deuxième verre contient \Capa[cL]{10} de boisson dont 5\,\% de sirop. Combien cela fait-il de sirop ?
         \item Sachant qu'au départ, il y avait \Capa[mL]{15} de sirop, combien lui reste-t-il de sirop après ces deux verres ?
      \end{enumerate}
\end{exercice*}
   
\begin{corrige}
\ \\ [-5mm]
   \begin{enumerate}
      \item 1\up{er} verre : $\dfrac{3,5}{100}\times\Capa[cL]{20} =\Capa[cL]{0,7}$. \\ [1mm]
         {\red Le premier verre contient \Capa[cL]{0,7} de sirop}. \smallskip
      \item 2\up{e} verre : $\dfrac{5}{100}\times\Capa[cL]{10} =\Capa[cL]{0,5}$. \\ [1mm]
         {\red Le deuxième verre contient \Capa[cL]{0,7} de sirop}.
      \item Dans les verres, il y a $\Capa[cL]{0,7}+\Capa[cL]{0,5} =\Capa[cL]{1,2}$ de sirop. \\
         Au départ, Selene avait $\Capa[mL]{15} = \Capa[cL]{1,5}$, {\red il lui restera donc \Capa[cL]{0,3} de sirop}.
   \end{enumerate}
\end{corrige}

