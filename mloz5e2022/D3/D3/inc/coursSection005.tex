\section{Application aux pourcentages}
\begin{propriete}[\admise]
   Si on veut prendre "$t\%$" d'un nombre alors on le multiplie par $\dfrac{t}{100}$.
\end{propriete}

\begin{exemple*1}
   \begin{itemize}
      \item $35\%$ des élèves d'un collège de $560$ élèves sont demi-pensionnaires, donc $560\times \dfrac{35}{100}=196$ élèves.
      \item $100 \%$ d'une classe de $5^{eme}$ de $23$ élèves est externe, c'est à dire $23\times \dfrac{100}{100}= 23$ élèves!
      \item \textit{C'est Mardi gras et aujourd'hui on a $40\%$ sur les crêpes car la boulangère est de bonne humeur. D'ordinaire la crêpe coûte $2$ \Prix{}.}
      \begin{enumerate}
         \item \`{A} combien s'élève la réduction?
         \item Combien payera t'on alors une crêpe?
      \end{enumerate}

      \smallskip
      \begin{tabular}{|c|c|c|}
         \hline
         Montant de la réduction&$R$&$40$ \Prix{}\\
         \hline
         Prix payé sans réduction&$2$ \Prix{}&$100$ \Prix{}\\
         \hline
      \end{tabular}

      \smallskip
      C'est un tableau de proportionnalité, donc les rapports $\dfrac{R}{2}$ et $\dfrac{40}{100}$ sont égaux,

      d'où $\dfrac{R}{2}=\dfrac{40}{100}$ d'où en les multipliant par 2, $\dfrac{R}{2} \times 2=\dfrac{40}{100} \times 2$ d'où $R=\dfrac{40}{100} \times 2$
   \end{itemize}
\end{exemple*1}

