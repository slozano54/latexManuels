\section{Procédures de proportionnalité (rappels)}

\begin{center}
   \begin{pspicture}(0,3.2)(16,8.5)
      %1
      \psframe*[fillstyle=solid,linecolor=A3](1.5,6.8)(8,8.6)
      \rput(4.75,7.8){\parbox{6cm}{\centering {\bf Linéarité additive} \\ 12 stylos = 4 stylos + 4 stylos + 4 stylos \\ coûtent \Prix{10} +\Prix{10} + \Prix{10} = \Prix{30}}}
      %2
      \psframe*[fillstyle=solid,linecolor=A3](9,6.8)(13.5,8.6)
      \rput(11.25,7.8){\parbox{4cm}{\centering {\bf Linéarité multiplicative} \\ 12 stylos = 3$\times$4 stylos \\ coûtent $3\times\Prix{10} =\Prix{30}$}}
      %3
      \psframe*[fillstyle=solid,linecolor=A3!50](1.5,2.3)(7,5.3)
      \rput(4.25,3.8){\parbox{5cm}{\centering {\bf Passage par l'unité} \\ 1 stylo coûte 4 fois moins cher : \\ $\Prix{10}\div4 =\Prix{2,5}$ \\ 12 stylos coûtent 12 fois plus cher : \\ $12\times\Prix{2,5} =\Prix{30}$}}
      %4
      \psframe*[fillstyle=solid,linecolor=A3!50](9,2.3)(15.5,5.3)
      \rput(12.25,3.8){\parbox{6cm}{\centering {\bf Coefficient de proportionnalité} \\ $4\times\fbox{2,5} =10$ \\ le coefficient de proportionnalité vaut 2,5 \\ $12\times\fbox{2,5} =30$ \\ 12 stylos coûtent \Prix{30}}}
      %5
      \psellipse[fillstyle=solid,fillcolor=B3](8,6)(2.7,1)
      \rput(8,6){\parbox{4.5cm}{\centering \bf Si 4 stylos coûtent 10 \Prix{} \\ combien coûtent 12 stylos ?}}
   \end{pspicture}
\end{center}