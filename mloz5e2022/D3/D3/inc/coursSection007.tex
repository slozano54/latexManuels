\section{Application aux échelles}
\begin{definition}
   Les longueurs réelles et les longueurs sur le plan sont dans la même unité.

   \textbf{L'échelle} du plan c'est \textbf{le} coefficient de proportionnalité qui permet de passer des longueurs réelles aux longueurs sur le plan
\end{definition}

\begin{exemple*1}
   \begin{itemize}
      \item Pour l'échelle de la salle,
      \begin{tabular}{|c|c|}
      \hline
      Longueurs réelles (en cm)&\\
      \hline
      Longueurs sur le plan (en cm)&\\
      \hline
      \end{tabular}
      \par\vspace{0.25cm}
      \item Pour une carte routière,
      \begin{tabular}{|c|c|}
      \hline
      Longueurs réelles (en km)&\\
      \hline
      Longueurs sur la carte (en km)&\\
      \hline
      \end{tabular}
   \end{itemize}
\end{exemple*1}
