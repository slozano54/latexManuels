\begin{changemargin}{-10mm}{-15mm}
\begin{activite}[Le puzzle de Brousseau]
    {\bf Objectifs :} mettre en \oe uvre un ou des moyens pour résoudre un problème d'agrandissement ; reproduire une figure géométrique en respectant des mesures ; rendre compte d'un travail en groupe.

    \partie[présentation du puzzle]
       Ci-dessous se trouve un puzzle composé de quatre pièces A, B, C et D dont les mesures sont indiquées sur la figure. \\
       \begin{center}
        \scalebox{0.9}{
          \begin{pspicture}(-1,-1)(12,11.5)
             \psframe(0,0)(11,11)
             \psline(4,0)(4,9)(11,2)
             \psline(6,11)(0,5)
             \psline[linestyle=dashed,linecolor=gray](11,9)(4,9)
             \rput(2,9){\bf\large A}
             \rput(8.5,7.5){\bf\large B}
             \rput(2,4){\bf\large C}
             \rput(7,3){\bf\large D}
             \rput{90}(11.5,6.5){\ucm{9}}
             \rput{90}(11.5,1){\ucm{2}}
             \rput(8.5,11.5){\ucm{5}}
             \rput(3,11.5){\ucm{6}}
             \rput{90}(-0.5,8){\ucm{6}}
             \rput{90}(-0.5,2.5){\ucm{5}}
             \rput(2,-0.5){\ucm{4}}
             \rput(7.5,-0.5){\ucm{7}}
             \rput{90}(3.5,4){\ucm{9}}
             \rput(8,9.5){\ucm{7}}
          \end{pspicture}       
        }
        \end{center}
       \partie[travail demandé]
          Par groupes, vous allez devoir refaire le même puzzle mais en plus grand : il faudra s'accorder sur la procédure à adopter pour agrandir les éléments du puzzle, se répartir la construction des pièces en faisant les calculs individuellement puis assembler les morceaux pour reconstituer le puzzle agrandi. \\
          Le compte-rendu de vos recherches sera présenté sous la forme d’une affiche par groupe.
          \begin{center}
             {\bf C'est parti\dots{} le segment de 4 cm devra mesurer 5 cm sur votre puzzle agrandi.}
          \end{center}
          \bigskip
 \end{activite}
\end{changemargin}