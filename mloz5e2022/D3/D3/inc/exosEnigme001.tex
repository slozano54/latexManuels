% Les enigmes ne sont pas numérotées par défaut donc il faut ajouter manuellement la numérotation
% si on veut mettre plusieurs enigmes
% \refstepcounter{exercice}
% \numeroteEnigme
\begin{enigme}[Empreinte carbone]
   \partie[une infographie]
     Voici un extrait d’un document paru dans le quotidien {\it Le Monde} (infographie d’Hélène Pasquier), élaboré à partir de l’Ademe, et intitulé \og Rapport des inégalités du monde 2022 \fg, d’après le ministère de la transition écologique et le Haut conseil pour le climat. \\ [1mm] 
   \includegraphics[width=17cm]{\currentpath/images/carbone} \\ [-3mm]
   
   \partie[la tâche proposée]
      En groupe, trouver une méthode pour vérifier si cette infographie est bien réalisée, en expliquant la méthode. Récapituler les recherches sur une affiche.
\end{enigme}
\vfill\hfill {\it\small Source : d'après une idée de Claire Lommé sur son site \href{https://clairelommeblog.wordpress.com/category/maths-et-societe}{Pierre carrée.}}

% Pour le corrigé, il faut décrémenter le compteur, sinon il est incrémenté deux fois
% \addtocounter{exercice}{-1}
\begin{corrige}
   Pour voir si cette infographie est bien réalisée, il faut vérifier que l'aire de chaque \og bulle \fg{} rectangulaire est proportionnelle au nombre de kilos équivalents CO$_{2}$ émis. \smallskip
   
   {\bf 1) On peut commencer par vérifier globalement, par domaine}. \\
   Dans un tableau, on récapitule les données dont on a besoin : longueur de la bulle, hauteur de la bulle, ce qui nous permet de déterminer son aire. Ensuite, on vérifie qu'il y a proportionnalité entre l'aire et l'émission de CO$_{2}$ en calculant, par exemple, le rapport entre les deux. \medskip
   
   {\renewcommand{\arraystretch}{1.5}
   \small
   \begin{LCtableau}{\linewidth}{6}{p{2cm}}
      \hline
      & long. & haut. & aire & carb. & rapp. \\
      \hline
      Transport & 8,3 & 6,9 & 57,27 & 2650 & 46,3 \\
      \hline
      Alimentation & 8,3 & 6,1 & 50,63 & 2350 & 46,4 \\
      \hline
      Habitat & 4,4 & 9,4 & 41,36 & 1900 & 45,9 \\
      \hline
      Consommation & 3,6 & 9,4 & 33,84 & 1600 & 47,3 \\
      \hline
      Dépenses pub. & 8,1 & 3,2 & 25,92 & 1400 & 46,7 \\
      \hline
   \end{LCtableau}}
   
   Les rapports sont situés autour de 46-47, on peut donc en déduire que les bulles de domaines sont bien proportionnées (les différences sont dues aux approximations de lecture des grandeurs). \smallskip
   
   {\bf 2) On peut ensuite par vérifier si, par domaine, les bulles détails sont bien réalisées}. \\
   On crée le même type de tableau, par exemple pour le domaine des transports : \medskip
   
   {\renewcommand{\arraystretch}{1.5}
   \small
   \begin{LCtableau}{\linewidth}{6}{p{2cm}}
      \hline
      & long. & haut. & aire & carb. & rapp. \\
      \hline
      Voiture & 6,3 & 6,9 & 43,47 & 2030 & 46,7 \\
      \hline
      Avion & 1,9 & 4,8 & 9,12 & 430 & 47,1 \\
      \hline
      Autres & 1,9 & 2,1 & 3,99 & 190 & 47,6 \\
      \hline
   \end{LCtableau}}
   
   Les rapports sont situés autour de 47 également, on peut donc en déduire que les bulles de détails sont bien proportionnées. \smallskip
   
   {\red En conclusion, on peut légitimement émettre l'hypothèse que le reste de l'infographie est construire sous le même modèle, et donc que cette infographie a les bonnes proportions}.

\end{corrige}