\begin{exercice*} %4
   Un robinet qui fuit laisse échapper de façon continue trois litres d’eau en deux heures.
   \begin{enumerate}
      \item Quelle quantité d’eau se sera écoulée au bout d’une demi-journée ?
      \item Quel temps s’est écoulé pour laisser s’échapper \Capa[L]{51} ?
      \item L’eau est facturée \Prix{0,0031} le litre. \\
         Quel sera le montant de la facture au bout d’un an ?
   \end{enumerate}
\end{exercice*}

\begin{corrige}
   \ \\ [-5mm]
   \begin{enumerate}
      \item Une demi-journée dure \Temps{;;;12}, c'est-à-dire $6\times\Temps{;;;2}$. \\
         La quantité d'eau écoulée sera de $6\times\Capa[L]{3} =\red \Capa[L]{18}$.
      \item $\Capa[L]{51} =17\times\Capa[L]{3}$. \\
         Il s'est donc écoulé $17\times\Temps{;;;2} =\red \Temps{;;;34}$.
      \item Un an, c'est 365 jours, soit 730 demi-journées. \\
         Il s'écoulera donc $730\times\Capa[L]{18} =\Capa[L]{13140}$ en un an. \\
         À \Prix{0,0031} le litre, cela fait $\num{13140}\times\Prix{0,0031} \approx\red \Prix{41}$.
   \end{enumerate}
\end{corrige}
