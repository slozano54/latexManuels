\begin{exercice*} %3
   Rayan a pesé ses beignets et a trouvé que deux beignets pèsent \Masse[g]{300} et trois beignets pèsent \Masse[g]{450}.
   \begin{enumerate}
      \item Combien pèsent cinq beignets ?
      \item Combien pèsent six beignets ?
      \item Combien pèsent quatorze beignets ?
   \end{enumerate}
\end{exercice*}

\begin{corrige}
   \ \\ [-5mm]\begin{enumerate}
      \item 5 beignets $=$ 2 beignets $+$ 3 beignets. \\
         Or, $\Masse[g]{300}+\Masse[g]{450} =\Masse[g]{750}$. \\
         Donc, {\red 5 beignets pèsent \Masse[g]{750}}.
      \item 6 beignets $=2\times3$ beignets.
         Or, $2\times\Masse[g]{450} = \Masse[g]{900}$. \\
         Donc, {\red 6 beignets pèsent \Masse[g]{900}}.
      \item 14 beignets $=7\times2$ beignets. \\
         Or, $7\times\Masse[g]{300} =\Masse[g]{2100}$. \\
         Donc, {\red 14 beignets pèsent \Masse[g]{2100}}.
   \end{enumerate}
\end{corrige}
