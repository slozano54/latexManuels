\begin{exercice*}
   Répondre aux petits défis suivants :
   \begin{enumerate}
      \item Combien faut-il ajouter à $\dfrac23$ pour obtenir 1 ? \smallskip
      \item Combien faut-il soustraire à $\dfrac{15}4$ pour obtenir 3 ? \smallskip
      \item Combien faut-il ajouter à $\dfrac{13}{10}$ pour obtenir 2 ? \smallskip
      \item Combien faut-il soustraire à $\dfrac{52}{17}$ pour obtenir 3 ?
   \end{enumerate}
\end{exercice*}

\begin{corrige}
   \ \\ [-5mm]
   \begin{enumerate}
      \item Il faut ajouter $\red \dfrac13$ à $\dfrac23$ pour obtenir 1. \bigskip
      \item $\dfrac{15}{4} =3+\dfrac34$ donc, \\ [1mm]
         il faut soustraire $\red \dfrac34$ à $\dfrac{15}{4}$ pour obtenir 3. \bigskip
      \item $\dfrac{13}{10} =1+\dfrac{3}{10}$ donc, \\ [1mm]
         il faut ajouter $\red \dfrac{7}{10}$ à $\dfrac{13}{10}$ pour obtenir 2. \bigskip
      \item $\dfrac{52}{17} =3+\dfrac{1}{17}$ donc, \\ [1mm]
         il faut soustraire $\red \dfrac{1}{17}$ à $\dfrac{52}{17}$ pour obtenir 3. \bigskip
   \end{enumerate}
\end{corrige}