
\begin{exercice*}
   Un adulte passe en moyenne 25\% de son temps à travailler, $\dfrac13$ à dormir, $\dfrac{1}{12}$ à gérer le quotidien et $\dfrac{3}{36}$ à manger.
   \begin{enumerate}
      \item Sur une journée de \uh{24}, combien de temps un adulte passe-t-il sur chacune de ces activités ?
      \item Quelle pourcentage de son temps lui reste-t-il pour ses loisirs ?
   \end{enumerate}
\end{exercice*}

\begin{corrige}
   \ \\ [-5mm]
   \begin{enumerate}
      \item Sur une journée, un adulte passe : \smallskip
      \begin{itemize}
         \item 25\,\% correspond à une faction de $\dfrac14$. \\ [2mm]
            \quad $\dfrac14\times\uh{24} =$ {\red \uh{6} pour travailler} ; \\ [2mm]
         \item $\dfrac13\times\uh{24} =$ {\red \uh{8} pour dormir} ; \\ [2mm]
         \item $\dfrac1{12}\times\uh{24} =$ {\red \uh{2} pour gérer le quotidien} ; \\ [2mm]
         \item $\dfrac{3}{36}\times\uh{24} =$ {\red \uh{2} pour manger.} \\ [3mm]
      \end{itemize}
      \item On calcule le nombre d'heures restantes : \\
          $\uh{24}-\uh{6}-\uh{8}-\uh{2}-\uh{2} =\uh{6}$. \\ [2mm]
         Or, $\dfrac{\uh{6}}{\uh{24}} =\dfrac{1\times\uh{6}}{4\times\uh{6}} =\dfrac14 =25\,\%$. \\ [2mm]
         {\red Il lui reste donc 25\,\% de son temps pour les loisirs}. \bigskip
   \end{enumerate} 
\end{corrige}