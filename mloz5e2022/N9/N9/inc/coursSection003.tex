\begin{changemargin}{0mm}{-15mm}
    \section{Additionner et soustraire des fractions} %%%

   \begin{methode*2*2}[Additionner et soustraire deux fractions de même dénominateur]
      Pour additionner ou soustraire deux fractions ayant le même dénominateur, il suffit d'additionner ou de soustraire les numérateurs tout en gardant le dénominateur commun.\\
      Si $a, b$ et $k$ sont des nombres, $k$ étant non nul.\\ [-2mm]
      $${a\over {\red k}}+{b\over {\red k}}={a+b\over {\red k}} \qquad \text{et} \qquad {a\over {\red k}}-{b\over {\red k}}={a-b\over {\red k}}$$
      \exercice
         Additionner $\dfrac36$ et $\dfrac26$. 
      \correction
         $\dfrac36+\dfrac26 =\dfrac{3+2}{6} =\dfrac56$ \\
         {\psset{unit=0.5}
         \begin{pspicture}(-7.7,-1.35)(1.75,1.25)
            \pscircle(0,0){1}
            \pswedge[fillstyle=solid,fillcolor=B3](0,0){1}{0}{180}
            \multido{\n=0+60}{6}{\psline(0,0)(1;\n)}
            \rput(1.5,0){$+$}
         \end{pspicture}
         \begin{pspicture}(-1,-1.35)(1.75,1.25)
            \pscircle(0,0){1}
            \pswedge[fillstyle=solid,fillcolor=B3](0,0){1}{180}{300}
            \multido{\n=0+60}{6}{\psline(0,0)(1;\n)}
            \rput(1.5,0){$=$}
         \end{pspicture}
         \begin{pspicture}(-1,-1.35)(1,1.25)
            \pscircle(0,0){1}
            \pswedge[fillstyle=solid,fillcolor=B3](0,0){1}{0}{300}
            \multido{\n=0+60}{6}{\psline(0,0)(1;\n)}
         \end{pspicture}}
      \exercice
         Soustraire $\dfrac26$ de $\dfrac36$.
      \correction
         $\dfrac36-\dfrac26 =\dfrac{3-2}{6} =\dfrac16$ \\
         {\psset{unit=0.5}
            \begin{pspicture}(-7.7,-1.35)(1.75,1.25)
               \pscircle(0,0){1}
               \pswedge[fillstyle=solid,fillcolor=B3](0,0){1}{0}{180}
               \multido{\n=0+60}{6}{\psline(0,0)(1;\n)}
               \rput(1.5,0){$-$}
            \end{pspicture}
            \begin{pspicture}(-1,-1.35)(1.75,1.25)
               \pscircle(0,0){1}
               \pswedge[fillstyle=solid,fillcolor=B2](0,0){1}{60}{180}
               \multido{\n=0+60}{6}{\psline(0,0)(1;\n)}
               \rput(1.5,0){$=$}
            \end{pspicture}
            \begin{pspicture}(-1,-1.35)(1,1.25)
               \pscircle(0,0){1}
               \pswedge[fillstyle=solid,fillcolor=B3](0,0){1}{0}{60}
               \multido{\n=0+60}{6}{\psline(0,0)(1;\n)}
            \end{pspicture}}
   \end{methode*2*2}
    
   \begin{exemples*1}
      \begin{multicols}{2}
         \begin{itemize}
            \item $\dfrac{5}{\textcolor{red}{3}}+\dfrac{7}{\textcolor{red}{3}} = \dfrac{5+7}{\textcolor{red}{3}} = \psshadowbox{\dfrac{12}{3}}$
            \item $\dfrac{7,5}{\textcolor{red}{4}}-\dfrac{5,2}{\textcolor{red}{4}} = \dfrac{7,5+5,2}{\textcolor{red}{4}} = \psshadowbox{\dfrac{2,3}{4}}$
            \item $\dfrac{37}{\textcolor{red}{12}}-\dfrac{50}{\textcolor{red}{12}}+ \dfrac{10}{\textcolor{red}{12}}= \dfrac{37-50+10}{\textcolor{red}{12}} = \psshadowbox{\dfrac{-3}{12}}$
         \end{itemize}
      \end{multicols}
      \vspace*{-10mm}
   \end{exemples*1}

   \begin{methode*2*2}[Additionner et soustraire deux fractions de dénominateurs différents]
      Pour additionner ou soustraire deux fractions ayant des dénominateurs différents, on cherche un dénominateur commun pour les écrire avec le même dénominateur.
      \exercice
         Additionner $\dfrac23$ et $\dfrac56$.
      \correction
      \begin{align*}
         \dfrac23+\dfrac56 &=\dfrac{{\red2}\times2}{{\red2}\times3}+\dfrac56\\
         \dfrac23+\dfrac56 &=\dfrac46+\dfrac56 =\dfrac96.
      \end{align*}
         {\psset{unit=0.5}
         \begin{pspicture}(-1,-1.25)(1.45,1.5)
            \pscircle(0,0){1}
            \pswedge[fillstyle=solid,fillcolor=B3](0,0){1}{-60}{180}
            \multido{\n=180+120}{3}{\psline(0,0)(1;\n)}
            \rput(1.3,0){$+$}
         \end{pspicture}
         \begin{pspicture}(-1,-1.25)(1.45,1.5)
            \pscircle(0,0){1}
            \pswedge[fillstyle=solid,fillcolor=B3](0,0){1}{0}{300}
            \multido{\n=0+60}{6}{\psline(0,0)(1;\n)}
            \rput(1.3,0){$=$}
         \end{pspicture}
         \begin{pspicture}(-1,-1.25)(1.45,1.5)
            \pscircle(0,0){1}
            \pswedge[fillstyle=solid,fillcolor=B3](0,0){1}{-60}{180}
            \multido{\n=180+60}{6}{\psline[linestyle=dashed](0,0)(1;\n)}
            \multido{\n=180+120}{3}{\psline(0,0)(1;\n)}
            \rput(1.3,0){$+$}
         \end{pspicture}
         \begin{pspicture}(-1,-1.25)(1.45,1.5)
            \pscircle(0,0){1}
            \pswedge[fillstyle=solid,fillcolor=B3](0,0){1}{0}{300}
            \multido{\n=0+60}{6}{\psline(0,0)(1;\n)}
            \rput(1.3,0){$=$}
         \end{pspicture}
         \begin{pspicture}(-1,-1.25)(1,1.5)
            \pscircle[fillstyle=solid,fillcolor=B3](0,0){1}
            \multido{\n=0+60}{6}{\psline(0,0)(1;\n)}
         \end{pspicture}
         \begin{pspicture}(-0.9,-1.25)(1,1.5)
            \pscircle(0,0){1}
            \pswedge[fillstyle=solid,fillcolor=B3](0,0){1}{0}{180}
            \multido{\n=0+60}{6}{\psline(0,0)(1;\n)}
         \end{pspicture}}
      \exercice
         Soustraire $\dfrac23$ de $\dfrac{11}{12}$.
      \correction
      \begin{align*}
         \dfrac{11}{12}-\dfrac23 &=\dfrac{11}{12}-\dfrac{{\red4}\times2}{{\red4}\times3}\\
         \dfrac{11}{12}-\dfrac23 &=\dfrac{11}{12}-\dfrac{8}{12} =\dfrac3{12}.
      \end{align*}
         {\psset{unit=0.5}
         \begin{pspicture}(-1,-1.25)(1.55,1.5)
            \pscircle(0,0){1}
            \pswedge[fillstyle=solid,fillcolor=B3](0,0){1}{-150}{180}
            \multido{\n=0+30}{12}{\psline(0,0)(1;\n)}
            \rput(1.35,0){$-$}
         \end{pspicture}
         \begin{pspicture}(-1,-1.25)(1.55,1.5)
            \pscircle(0,0){1}
            \pswedge[fillstyle=solid,fillcolor=B2](0,0){1}{-60}{180}
            \multido{\n=180+120}{3}{\psline(0,0)(1;\n)}
            \rput(1.35,0){$=$}
         \end{pspicture}
         \begin{pspicture}(-1,-1.25)(1.55,1.5)
            \pscircle(0,0){1}
            \pswedge[fillstyle=solid,fillcolor=B3](0,0){1}{-150}{180}
            \multido{\n=0+30}{12}{\psline(0,0)(1;\n)}
            \rput(1.35,0){$-$}
         \end{pspicture}
         \begin{pspicture}(-1,-1.25)(1.55,1.5)
            \pscircle(0,0){1}
            \pswedge[fillstyle=solid,fillcolor=B2](0,0){1}{-60}{180}
            \multido{\n=0+30}{12}{\psline[linestyle=dashed](0,0)(1;\n)}
            \multido{\n=180+120}{3}{\psline(0,0)(1;\n)}
            \rput(1.35,0){$=$}
         \end{pspicture}
         \begin{pspicture}(-1,-1.25)(1,1.5)
            \pscircle(0,0){1}
            \pswedge[fillstyle=solid,fillcolor=B3](0,0){1}{-150}{-60}
            \multido{\n=0+30}{12}{\psline(0,0)(1;\n)}
         \end{pspicture}}
   \end{methode*2*2}
    
   \begin{remarque}
      Donc pour additionner ou soustraire des parts, il faut qu'elles soient égales.
   \end{remarque}

   \begin{exemples*1}
      $A=\dfrac{3}{4}+\dfrac{\num{4.7}}{6}$

      \medskip
      \begin{itemize}
      \item \textit{On cherche d'abord un nombre non nul dans la table de 4 ET de 6.}
      \begin{itemize}
      \item[] \underline{Table de 4} : 0 4 8 \psframebox{12} 16 20 \psframebox{24} 28 32 \psframebox{36} 40 44 \psframebox{48} 52  $\ldots$.
      \item[] \underline{Table de 6} : 0 6 \psframebox{12} 18 \psframebox{24} 30 \psframebox{36} 42 \psframebox{48}  $\ldots$.\\
      \textit{Il en suffit d'un !}
      \end{itemize}
      \item \textit{On transforme les écritures si nécessaire avec la propriéeté fondamentale.}
      
      \medskip
      $\dfrac{3}{4}=\dfrac{3\times \textcolor{red}{3}}{4\times \textcolor{red}{3}}=\dfrac{9}{12}$ et $\dfrac{4,7}{6}=\dfrac{4,7\times \textcolor{red}{2}}{6\times \textcolor{red}{2}}=\dfrac{9,4}{12}$
      \end{itemize}
      \begin{minipage}{0.3\linewidth}
         \begin{align*}
            A=&\dfrac{9}{12}+\dfrac{\num{9.4}}{12}\\
            A=&\psshadowbox{\dfrac{\num{18.4}}{12}}
         \end{align*}
      \end{minipage}
      \begin{minipage}{0.3\linewidth}
         \begin{align*}
            B=&\dfrac{25}{49}-\dfrac{2}{7}\\
            B=&\dfrac{25}{49}-\dfrac{2\times \textcolor{red}{7}}{7\times \textcolor{red}{7}}\\
            B=&\dfrac{25-2\times 7}{49}\\
            B=&\psshadowbox{\dfrac{11}{49}}
         \end{align*}
      \end{minipage}
      \begin{minipage}{0.3\linewidth}
         \begin{align*}
            C=&\dfrac{\num{2.1}}{3}-\dfrac{1}{6}+\dfrac{3}{18}\\
            C=&\dfrac{\num{2.1}\times \textcolor{red}{6}}{3\times \textcolor{red}{6}}-\dfrac{1\times \textcolor{red}{3}}{6\times \textcolor{red}{3}}+\dfrac{3}{18}\\
            C=&\dfrac{\num{12.6}}{18}-\dfrac{3}{18}+\dfrac{3}{18}\\
            C=&\dfrac{\num{12.6}-3+3}{18}\\
            C=&\psshadowbox{\dfrac{\num{12.6}}{18}}
         \end{align*}
      \end{minipage}
   \end{exemples*1}

   \begin{remarque}
      \titreRemarque{\emoji{bomb} Attention !!!}

      $\dfrac{\pnode(0,0.2em){A}{9}}{\pnode(0,0.2em){C}{2}}+\dfrac{10}{8}=\dfrac{\pnode(0.4,0.2em){D}{9+10}}{\pnode(0.4,0.2em){B}{2+8}}$\ncarc{-}{A}{B} \ncarc{-}{C}{D}

      \smallskip
      En effet, $\dfrac{9}{2}=\num{4.5}$ et $\dfrac{10}{8}=\num{1.25}$ donc $\dfrac{9}{2}+\dfrac{10}{8}=\num{5.75}$ et $\dfrac{9+10}{2+8}=\dfrac{19}{10}=\num{1.9}$ !!!
   \end{remarque}
\end{changemargin}
 
