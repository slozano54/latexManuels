\begin{changemargin}{0mm}{-15mm}
    % \opset{voperator=bottom,decimalsepsymbol={,},strikedecimalsepsymbol=\rlap{,}\rule[-1pt]{3pt}{0.4pt}}
    \section{Rappels} %%%
    \begin{remarque}
        La notation fractionnaire sert notamment à écrire le quotient des divisions "interminables" par exemple le quotient de $5\div 7$ ne peut s'écrire sous forme décimale.
    \end{remarque}
    
    \vspace*{-7mm}
    \begin{minipage}[c]{0.2\linewidth}
        \pnode(2.7,0.2em){C}{\colorbox{red!30}{NUMÉRATEUR}}\\	
        \pnode(3.1,0.2em){D}{\colorbox{blue!30}{DÉNOMINATEUR}}
    \end{minipage}
    $\dfrac{\OPoval{A}{0,1}{\colorbox{red!30}{5}}}{\OPoval{B}{0,1}{\colorbox{blue!30}{7}}}$\qquad
    \ncarc[arcangle=-15]{->}{A}{C}
    \ncarc{->}{B}{D}
    \begin{minipage}[c]{0.4\linewidth}
    se lit "cinq septièmes"\\
    C'est une écriture fractionnaire du quotient    
    \end{minipage}
    $\OPoval{G}{0,1}{\colorbox{red!30}{5}} \div \OPoval{H}{0,1}{\colorbox{blue!30}{7}}$\qquad
    \begin{minipage}[c]{0.3\linewidth}
        \pnode(0,0.2em){E}{\colorbox{red!30}{DIVIDENDE}}
        \par\vspace{2.5cm}
        \pnode(0,0.2em){F}{\colorbox{blue!30}{DIVISEUR}}
    \end{minipage}
    \ncarc[arcangle=-15]{<-}{E}{G}
    \ncarc{<-}{F}{H}
    \vspace*{-10mm}

    \begin{definition}[Nombre fraction]
        Soit $a$ et $b$ deux nombres ($b\neq0$).\\
        Le {\bf quotient} $\dfrac{a}{b}$ est le nombre qui, lorsqu'il est multiplié par $b$, donne $a$, donc  $\dfrac{a}{b}\times b = a$.\\
        Ce quotient écrit sous forme d'une fraction est le résultat d'une division : $\dfrac{a}{b} =a\div b$.
    \end{definition}

    \begin{propriete}[Quotients égaux \admise]
        Le quotient de deux nombres relatifs ne change pas lorsque l'on multiplie, ou on divise, ces deux nombres par un même nombre
        relatif non nul (différent de 0).
        $$\frac{a}{b}=\frac{a\times c}{b\times c}\qquad \text{ou} \qquad \frac{a}{b}=\frac{a\div c}{b\div c}\qquad avec \quad  (b\not=0; c\not=0)$$
    \end{propriete}
\end{changemargin}