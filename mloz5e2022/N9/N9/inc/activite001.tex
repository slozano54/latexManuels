\begin{activite}[Décomposer une fraction]
    \begin{changemargin}{-10mm}{-15mm}
           {\bf Objectifs :} représenter des fractions ; écrire une fraction sous la forme de la somme ou de la différence d'un entier et d'une fraction inférieure à 1.

           \partie[des toasts en entrée]
                Pour faire des toasts à ses amis, Abdelmajid coupe des tranches de pain de mie en quatre, avant de les garnir. \\
                Natan mange onze de ces petits toasts. Quelle fraction de grande tranche de pain de mie Natan a-t-il mangé ?
                \begin{center}
                    {\psset{unit=0.9}
                    \begin{pspicture}(0,0)(3,1.5)
                    \psframe[fillstyle=solid,fillcolor=J2](0,0)(2,1.5)
                    \psline(1,0)(1,1.5)
                    \psline(0,0.75)(2,0.75)
                    \end{pspicture}
                    \begin{pspicture}[fillstyle=solid,fillcolor=J2](0,0)(3,1.5)
                    \psframe(0,0)(2,1.5)
                    \psline(1,0)(1,1.5)
                    \psline(0,0.75)(2,0.75)
                    \end{pspicture}
                    \begin{pspicture}(0,0)(3,1.5)
                    \psframe(0,0)(2,1.5)
                    \pspolygon[fillstyle=solid,fillcolor=J2](0,0)(1,0)(1,0.75)(2,0.75)(2,1.5)(0,1.5)
                    \psline(1,0)(1,1.5)
                    \psline(0,0.75)(2,0.75)
                    \end{pspicture}
                    \begin{pspicture}(0,0)(3,1.5)
                    \psframe(0,0)(2,1.5)
                    \psline(1,0)(1,1.5)
                    \psline(0,0.75)(2,0.75)
                    \end{pspicture}
                    \begin{pspicture}(0,0)(3,1.5)
                    \psframe(0,0)(2,1.5)
                    \psline(1,0)(1,1.5)
                    \psline(0,0.75)(2,0.75)
                    \end{pspicture}}
                \end{center}
                Il a mangé 11 fois $\dfrac14$ de grande tranche, donc $\dfrac{11}4$ de grande tranche, ce qui fait 2 tranches et $\dfrac34$ de tranche, ou \\ [1mm]
        3 tranches moins $\dfrac14$ de tranche. On peut écrire : $\dfrac{11}4 =2+\dfrac34 =3-\dfrac14$. \\ [1mm]
                Marcel, lui, a mangé 17 petits toasts. Colorier ce que cela représente ci-dessous.
                \begin{center}
                    {\psset{unit=0.9}
                    \begin{pspicture}(0,0)(3,1.5)
                    \psframe(0,0)(2,1.5)
                    \psline(1,0)(1,1.5)
                    \psline(0,0.75)(2,0.75)
                    \end{pspicture}
                    \begin{pspicture}(0,0)(3,1.5)
                    \psframe(0,0)(2,1.5)
                    \psline(1,0)(1,1.5)
                    \psline(0,0.75)(2,0.75)
                    \end{pspicture}
                    \begin{pspicture}(0,0)(3,1.5)
                    \psframe(0,0)(2,1.5)
                    \pspolygon(0,0)(1,0)(1,0.75)(2,0.75)(2,1.5)(0,1.5)
                    \psline(1,0)(1,1.5)
                    \psline(0,0.75)(2,0.75)
                    \end{pspicture}
                    \begin{pspicture}(0,0)(3,1.5)
                    \psframe(0,0)(2,1.5)
                    \psline(1,0)(1,1.5)
                    \psline(0,0.75)(2,0.75)
                    \end{pspicture}
                    \begin{pspicture}(0,0)(3,1.5)
                    \psframe(0,0)(2,1.5)
                    \psline(1,0)(1,1.5)
                    \psline(0,0.75)(2,0.75)
                    \end{pspicture}}
                \end{center}
                \smallskip
                Compléter l'égalité  : \hspace{30mm} $\dfrac{17}{4} = \pointilles[1cm] + \dfrac{\pointilles[1cm]}{\pointilles[1cm]} =\pointilles[1cm] - \dfrac{\pointilles[1cm]}{\pointilles[1cm]}$ \\

            \partie[des \og flam \fg{} en plat principal]
                Pour poursuivre, Abdelmajid propose à ses amis de manger des flammekueches (flam). \\
                \begin{minipage}{7.75cm}
                    \begin{pspicture}(-1.2,-1)(1,1)
                    \pscircle[fillstyle=solid,fillcolor=J2](0,0){0.7}
                    \multido{\n=0+72}{5}{\psline(0,0)(0.7;\n)}
                    \end{pspicture}
                    \begin{pspicture}(-0.7,-1)(1,1)
                    \pscircle(0,0){0.7}
                    \pswedge[fillstyle=solid,fillcolor=J2](0,0){0.7}{72}{288}
                    \multido{\n=0+72}{5}{\psline(0,0)(0.7;\n)}
                    \end{pspicture}
                    \begin{pspicture}(-0.7,-1)(1,1)
                    \pscircle(0,0){0.7}
                    \multido{\n=0+72}{5}{\psline(0,0)(0.7;\n)}
                    \end{pspicture}
                    \begin{pspicture}(-0.7,-1)(1,1)
                    \pscircle(0,0){0.7}
                    \multido{\n=0+72}{5}{\psline(0,0)(0.7;\n)}
                    \end{pspicture} \\
                    Une part de flam représente $\dfrac{\pointilles[1cm]}{\pointilles[1cm]}$ de flam. \\ [3mm]
                    $\dfrac{\pointilles[1cm]}{\pointilles[1cm]}$ de flam est colorié. \\ [3mm]
                    Donc, $\dfrac{\pointilles[1cm]}{\pointilles[1cm]} ={\pointilles[1cm]} + \dfrac{\pointilles[1cm]}{\pointilles[1cm]} ={\pointilles[1cm]} - \dfrac{\pointilles[1cm]}{\pointilles[1cm]}$ \\ [2mm]
                \end{minipage}
                \qquad
                \begin{minipage}{7.75cm}
                    \begin{pspicture}(-1.2,-1)(1,1)
                    \pscircle(0,0){0.7}
                    \multido{\n=0+45}{8}{\psline(0,0)(0.7;\n)}
                    \end{pspicture}
                    \begin{pspicture}(-0.7,-1)(1,0.8)
                    \pscircle(0,0){0.7}
                    \multido{\n=0+45}{8}{\psline(0,0)(0.7;\n)}
                    \end{pspicture}
                    \begin{pspicture}(-0.7,-1)(1,1)
                    \pscircle(0,0){0.7}
                    \multido{\n=0+45}{8}{\psline(0,0)(0.7;\n)}
                    \end{pspicture}
                    \begin{pspicture}(-0.7,-1)(1,1)
                    \pscircle(0,0){0.7}
                    \multido{\n=0+45}{8}{\psline(0,0)(0.7;\n)}
                    \end{pspicture} \\
                    Une part de flam représente $\dfrac{\pointilles[1cm]}{\pointilles[1cm]}$ de flam. \\
                    Colorier $\dfrac{13}{8}$ de flam. \\ [1mm]
                    Donc, $\dfrac{13}{8} ={\pointilles[1cm]} + \dfrac{\pointilles[1cm]}{\pointilles[1cm]} ={\pointilles[1cm]} - \dfrac{\pointilles[1cm]}{\pointilles[1cm]}$ \\ [2mm]
                \end{minipage}

            \partie[des éclairs en dessert]
                Kymia a mangé $\dfrac{73}9$ de mini-éclairs au chocolat, trouver une manière de décomposer cette fraction en la somme et
                la différence d'un nombre entier et d'une fraction inférieure à 1. \\
        
        \vfill\hfill{\footnotesize{\it Inspiré de \href{http://ww2.ac-poitiers.fr/dsden86-pedagogie/sites/dsden86-pedagogie/IMG/pdf/groupe4_c13.pdf}{\og Écrire une fraction sous la forme d'un entier et d'une fraction inférieure à 1 \fg}, académie de Poitiers.}}
    \end{changemargin}
\end{activite}