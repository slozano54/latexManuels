\begin{exercice*}
   Dans cet exercice, l'unité est le carré suivant (le tangram), constitué de sept pièces.
   
   {\psset{unit=1.5}
   \begin{pspicture}(-0.5,-0.3)(4,4.3)
      \psgrid[subgriddiv=1,gridcolor=gray!80,gridlabels=0](4,4)
      \psset{linewidth=0.5mm}
      \psframe(0,0)(4,4)
      \psline(0,0)(4,4)
      \psline(3,1)(0,4)
      \psline(2,0)(4,2)
      \psline(2,0)(1,1)
      \psline(3,1)(3,3)
      \rput(0.8,2){\large 5}
      \rput(2,3.2){\large 6}
      \rput(1,0.4){\large 2}
      \rput(2,1){\large 1}
      \rput(2.6,2){\large 3}
      \rput(3.5,2.5){\large 4}
      \rput(3.5,0.6){\large 7}
   \end{pspicture}}
   \begin{enumerate}
      \item Quelle fraction du tangram représente chacune des pièces de 1 à 7 ?
      \item Avec quelles pièces du tangram (deux minimum) peut-on faire la fraction $\dfrac18$ ? \smallskip
      \item Avec quelles pièces (deux minimum) peut-on faire la fraction $\dfrac{3}{16}$ ? Donner trois possibilités. \smallskip
      \item Avec quelles pièces (deux minimum) peut-on faire la fraction $\dfrac14$ ? Donner quatre possibilités.
   \end{enumerate}
\end{exercice*}

\begin{corrige}
\ \\ [-5mm]
   \begin{enumerate}
      \item Les pièces 2 et 3 représentent {\red $\dfrac{1}{16}$} du tangram. \\ \smallskip
         Les pièces 1, 4 et 7 représentent {\red $\dfrac18$} du tangram. \\ \smallskip
         Les pièces 5 et 6 représentent {\red $\dfrac14$} du tangram. \medskip
      \item On peut faire $\dfrac18$ avec les pièces {\red 2 et 3}. \medskip
      \item On peut faire $\dfrac{3}{16}$ avec les pièces \\ [1mm]
         {\red 1 et 2} ou {\red 1 et 3} ou encore {\red 3 et 4}. \medskip
      \item On peut faire $\dfrac14$ avec les pièces \\ [1mm]
         {\red 4 et 7} ou {\red 1, 2 et 3} ou {\red 2, 3 et 4} encore {\red 1 et 4}.
   \end{enumerate}
\end{corrige}