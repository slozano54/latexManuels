% ============================================================================================
% ======= Compatibilité avec ProfCollege
% ============================================================================================
% Ces deux options sont passées par la classe sesamanuel.cls
% \PassOptionsToPackage{table}{xcolor}
% \PassOptionsToPackage{svgnames}{xcolor}

% ============================================================================================
% ======= Générer les correction dans un dossier spécifique
% ============================================================================================
\renewcommand\PrefixeCorrection{corrections/}

% ============================================================================================
% Factorisation des titres et couleurs de thèmes
% ============================================================================================
\colorlet{IColor}{gray}
\colorlet{NColor}{J1}
\colorlet{GColor}{PartieGeometrie}
\colorlet{DColor}{PartieStatistique}%H1
\colorlet{MColor}{A1}
\colorlet{AColor}{G1}
\colorlet{preAndPostFace}{C1}

% ============================================================================================
% ======= Thèmes de base prédéfinis
% \themaG ; \themaF ; \themaS
%
% ======= Thèmes personnalisés
%
% \NewThema{N}{n}{titre}{Titre}{TITRE}{couleur entete et ...}{couleur pied de page et ...}
%
% ============================================================================================
% Modifier ./tocs/incIntroduction.tex en conséquence si on change pour autre chose ici
\NewThema{I}{i}{introduction}{Introduction}{INTRODUCTION}{IColor}{IColor!50}

% Modifier ./tocs/incNumerique.tex en conséquence si on change pour autre chose ici
\NewThema{N}{n}{nombres \&\\~calculs}{Nombres \&\\~calculs}{NOMBRES \&\\~CALCULS}{NColor}{NColor!50}

% Pour la géométrie on garde le théme d'origine, donc renewcommand ou défintion d'un nouveau theme
% au besoin
% Modifier ./tocs/incGeometrie.tex en conséquence

% Modifier ./tocs/incGestionDeDonnees.tex en conséquence si on change pour autre chose ici
\NewThema{D}{d}{organisation \&\\~gestion de données}{Organisation \&\\~gestion de données}{ORGANISATION \&\\~GESTION DE DONNÉES}{DColor}{DColor!50}

% Modifier ./tocs/incGrandeursEtMesures.tex en conséquence si on change pour autre chose ici
\NewThema{M}{m}{grandeurs \&\\~mesures}{Grandeurs \&\\~mesures}{GRANDEURS \&\\~MESURES}{MColor}{MColor!50}

% Modifier ./tocs/incAlgorithmiqueEtProgrammation.tex en conséquence si on change pour autre chose ici
\NewThema{A}{a}{algorithmique \&\\~programmation}{Algorithmique \&\\~programmation}{ALGORITHMIQUE \&\\~PROGRAMMATION}{AColor}{AColor!50}

% ============================================================================================
% ======= Comme il y a des thèmes personnalisés
% ======= Il faut redefinir la commande \ListeMethodesThemes{}
% ============================================================================================

\renewcommand\ListeMethodesThemes{{i}{I},{n}{N},{g}{G},{d}{D},{m}{M},{a}{A}}

% ============================================================================================
% ======= Quelques styles supplémentaires
% ============================================================================================
\fancypagestyle{firstCover}{%   1ere de couverture
    \fancyhf{}%                 On initialise headers and footers à rien du tout !        
    \renewcommand{\headrulewidth}{0pt}%trait horizontal pour l'en-tête
    %\renewcommand{\footrulewidth}{0.4pt}%trait horizontal pour le pied de page    
}
\fancypagestyle{backCover}{%   4ere de couverture
    \fancyhf{}%                 On initialise headers and footers à rien du tout !        
    %\renewcommand{\headrulewidth}{0pt}%trait horizontal pour l'en-tête
    %\renewcommand{\footrulewidth}{0.4pt}%trait horizontal pour le pied de page
}

% ============================================================================================
% ======= Quelques redéfinitions et ajouts de commandes sesamanuel
% ============================================================================================

\renewcommand*\StringManuel{Des ressources num\'eriques pour pr\'eparer
  le chapitre sur
  \href{https://mathslozano.fr}{\textcolor{U4}{\LogoURLManuelFont https://mathslozano.fr}}}

% Ajout d'un logo
\newcommand*\NewLogoFont{\fontsize{9}{9}\sffamily\bfseries}  
\newcommand*\StringNEWLOGO{NEWLOGO}
\newcommand*\newLogo{%
  \psframebox[framesep=1pt,linewidth=\LogoLineWidth,
              linecolor=TiceLineColor, fillstyle=solid,
              fillcolor=TiceBkgColor, framearc=0.6]{%
    \NewLogoFont
    \textcolor{TiceTextColor}{\StringNEWLOGO}%
  }
}

\newcommand*\StringMATHALEA{MATHALEA}
\colorlet{mathaleaLineColor}{J3}
\colorlet{mathaleaBkgColor}{J1}
\colorlet{mathaleaTextColor}{Blanc}
\newcommand*\mathaleaLogo{%
  \psframebox[framesep=1pt,linewidth=\LogoLineWidth,
              linecolor=mathaleaLineColor, fillstyle=solid,
              fillcolor=mathaleaBkgColor, framearc=0.6]{%
    \NewLogoFont
    \textcolor{mathaleaTextColor}{\StringMATHALEA}%
  }
}
% Un environnement pour mathalea
\DeclareActivityLike{mathalea}{BESOIN DE RÉVISER}{J3}{J1}{H1}

% Un environnement pour les points de vue historique
\DeclareActivityLike{histoire}{INTERLUDE HISTORIQUE}{J3}{J1}{H1}

% Pour avoir un nouvel environnement de type activité
\DeclareActivityLike{nouveau}{DÉCOUVERTE}{lime}{orange}{red}

% On factorise le titre du glossaire de propriétés
\newcommand{\StringGlossaireProprietes}{Glossaire de propriétés}

