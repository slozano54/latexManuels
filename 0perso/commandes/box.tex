% Box perso, on peut l'utiliser pour les programmes de calcul
% #1 --> Contenu
% #2 --> Titre
\newtcolorbox{myBoxText}[2][]{
    enhanced,
    boxsep=1mm,
    bottom=.75mm,
    boxrule=2pt,
    text width=0.77\linewidth,
    % text width=\linewidth,
    colframe=gray,
    colback=gray!20,
    colbacktitle=white,
    fonttitle=\bfseries\color{black},
    halign upper=center,
    attach boxed title to top center={yshift=-2mm},
    title={#2},#1
}

% Box perso, on peut l'utiliser pour les programmes de calcul
% #1 --> Contenu
% #2 --> Titre
\newtcolorbox{myBox}[2][]{
    enhanced,
    boxsep=1mm,
    bottom=.75mm,
    boxrule=2pt,
    % text width=0.75\linewidth,
    text width=\linewidth,
    colframe=gray,
    colback=gray!20,
    colbacktitle=white,
    fonttitle=\bfseries\color{black},
    halign upper=center,
    attach boxed title to top center={yshift=-2mm},
    title={#2},#1
}

% Patch pour pouvoir redefinir un compteur via \setlist
% Necessaire pour la commande \ProgCalcul du paquet profcollege
\let\enumerateold\enumerate
\let\endenumerateold\endenumerate
% #1 --> titre
\newcommand{\myTCBset}[1]{
    \tcbset{ProgCalcul/.style={%
    enhanced,
    boxsep=1mm,
    bottom=.75mm,
    boxrule=2pt,
    text width=0.75\linewidth,
    colframe=gray,
    colback=gray!20,
    colbacktitle=white,
    fonttitle=\bfseries\color{black},
    halign upper=center,
    attach boxed title to top center={yshift=-2mm},
    title={#1},
    }%
    }%   
}
% Un programme de calcul encadré
% #1 --> label 
% #2 --> titre
% #3 --> Commande \ProgCalcul du paquet ProfCollege
\newcommand{\myProgCalcul}[3]{% label, titre, commande \ProgCalcul
    \begingroup
    \let\enumerate\enumerateold
    \let\endenumerate\endenumerateold
    \setlist[enumerate]{label=#1}
    \myTCBset{#2}
    #3
    \endgroup
}