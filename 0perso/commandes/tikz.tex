% Pavage en L dans un quadrillage 8x8
\newcommand{\pavageL}{%
    \foreach \a in {0,-2,-4}{%
        \draw[shift={(\a,-\a)}] (14,0) -- (14,2) -- (16,2) -- (16,4) -- (12,4) -- (12,0) -- cycle;
        \draw[shift={(\a,-\a)}] (15,2) -- (15,3) -- (13,3) -- (13,1) -- (14,1);
        \draw[shift={(\a,-\a)}] (12,2) -- (13,2);
        \draw[shift={(\a,-\a)}] (14,3) -- (14,4);
        \draw[shift={(\a,-\a)}] (14,0) -- (15,0) -- (15,1) -- (16,1) -- (16,2);
    }
    \foreach \a in {0}{%
        \draw[shift={(\a,-\a)},rotate around={90:(12,0)}] (14,0) -- (14,2) -- (16,2) -- (16,4) -- (12,4) -- (12,0) -- cycle;
        \draw[shift={(\a,-\a)},rotate around={90:(12,0)}] (15,2) -- (15,3) -- (13,3) -- (13,1) -- (14,1);
        \draw[shift={(\a,-\a)},rotate around={90:(12,0)}] (12,2) -- (13,2);
        \draw[shift={(\a,-\a)},rotate around={90:(12,0)}] (14,3) -- (14,4);
        \draw[shift={(\a,-\a)},rotate around={90:(12,0)}] (14,0) -- (15,0) -- (15,1) -- (16,1) -- (16,2);
    }
    \foreach \a in {0}{%
        \draw[shift={(\a,-\a)},rotate around={-90:(16,4)}] (14,0) -- (14,2) -- (16,2) -- (16,4) -- (12,4) -- (12,0) -- cycle;
        \draw[shift={(\a,-\a)},rotate around={-90:(16,4)}] (15,2) -- (15,3) -- (13,3) -- (13,1) -- (14,1);
        \draw[shift={(\a,-\a)},rotate around={-90:(16,4)}] (12,2) -- (13,2);
        \draw[shift={(\a,-\a)},rotate around={-90:(16,4)}] (14,3) -- (14,4);
        \draw[shift={(\a,-\a)},rotate around={-90:(16,4)}] (14,0) -- (15,0) -- (15,1) -- (16,1) -- (16,2);
    }
}

% Image d'une maille définie par des points nommés A,B,C,D,E,F par homothetie de centre O
% #1 -> ratio
% #2 -> couleur
\newcommand{\imageHomothetyHexaGone}[2]{%
    \tkzDefPointBy[homothety=center O ratio #1](A); \tkzGetPoint{Ab};
    \tkzDefPointBy[homothety=center O ratio #1](B); \tkzGetPoint{Bb};
    \tkzDefPointBy[homothety=center O ratio #1](C); \tkzGetPoint{Cb};
    \tkzDefPointBy[homothety=center O ratio #1](D); \tkzGetPoint{Db};
    \tkzDefPointBy[homothety=center O ratio #1](E); \tkzGetPoint{Eb};
    \tkzDefPointBy[homothety=center O ratio #1](F); \tkzGetPoint{Fb};
    \draw[color=#2,fill=#2,fill opacity=0.5] (Ab) -- (Bb) -- (Cb) -- (Db) -- (Eb) -- (Fb) -- cycle;
}

% Pavage en losanges dans un quadrillage mxn
% #1 -> longueur du pavage en abscisse
% #2 -> longueur du pavage en ordonnée
\newcommand{\pavageLosange}[2]{%
  \foreach \i in {1,...,#2}{%
    \begin{scope}[shift={(\i-1,\i*1.73-1.73)}]
        \foreach \j in {1,...,#1}{%
            \draw[shift={(\j*2-2,0)}] (0,0) -- (2, 0) -- (3, 1.73) -- (1, 1.73) -- cycle;
        }
    \end{scope}
  }
}

% Image d'une maille définie par des points nommés A,B,C,D,E,F par homothetie de centre O
% #1 -> ratio
% #2 -> couleur
\newcommand{\imageHomothetyParallelogramme}[2]{%
    \tkzDefPointBy[homothety=center O ratio #1](A); \tkzGetPoint{Ab};
    \tkzDefPointBy[homothety=center O ratio #1](B); \tkzGetPoint{Bb};
    \tkzDefPointBy[homothety=center O ratio #1](C); \tkzGetPoint{Cb};
    \tkzDefPointBy[homothety=center O ratio #1](D); \tkzGetPoint{Db};
    \draw[color=#2,fill=#2,fill opacity=0.5] (Ab) -- (Bb) -- (Cb) -- (Db) -- cycle;
}

% Axe Gradué
% #1 -> abscisse de O
% #2 -> abscisse de M
\newcommand{\axeHomothety}[2]{%
    \draw[help lines, color=black!30] (0,0) grid (18,2);            
    % Axe
    \draw (0,1)--(18,1);
    \foreach \x in {0,...,18} \draw (\x,0.9) -- (\x,1.1);
    % Points
    \coordinate (O) at (#1,1);
    \coordinate (M) at (#2,1);    
    % Marques
    \tkzDrawPoints[shape=cross out, size=5pt](O,M);
    \tkzLabelPoints[above](O,M);    
}

% Image d'un point nommé M par homothetie de centre O
% #1 -> ratio
% #2 -> couleur
\newcommand{\imageHomothetyPoint}[2][black]{%
    \tkzDefPointBy[homothety=center O ratio #2](M); \tkzGetPoint{M1};            
    % Marques
    \tkzDrawPoints[shape=cross out, size=5pt,color=#1](M1);    
    \tkzLabelPoint[above,color=#1](M1){$M_1$};
}

% Quadrillage à mailles carrées mxn
% #1 -> couleur : facultive black!30 par défaut
% #2 -> nombre de carreaux en abscisse
% #3 -> nombre de carreaux en ordonnée
\newcommand{\quadrilageMailleCarree}[3][black!30]{%
    \draw[help lines, color=#1] (0,0) grid (#2,#3);
}

% Papier millimétré
% #1 -> couleur : facultative, brown par défaut
\newcommand{\papierMillimetre}[1][brown]{
    \begin{pgfonlayer}{background}
        \draw[step=1mm,ultra thin,#1!30] (current bounding box.south west) grid (current bounding box.north east);
        \draw[step=5mm,very thin,#1!50] (current bounding box.south west) grid (current bounding box.north east);
        \draw[step=1cm,thin,#1!70] (current bounding box.south west) grid (current bounding box.north east);
        \draw[step=5cm,thick,#1!90] (current bounding box.south west) grid (current bounding box.north east);
    \end{pgfonlayer}
}

% Machine maths
% #1 --> nom de la fonction
% #2 --> Texte procédé de calcul sur deux lignes
% #3 --> Texte antécédent sur deux lignes
% #4 --> Texte image sur deux lignes
\newcommand{\machineMaths}[4]{
    \begin{tikzpicture}[line cap=round,line join=round,>=triangle 45,x=1cm,y=1cm]
        % Avant cadre
        \fill [line width=3pt,color=J1] (-6,2.5) -- (-6.5,1.5) -- (-5.5,2.5) -- (-6.5,3.5) -- cycle;
        \draw [line width=3pt,color=J1] (-5,2.9)-- (-5,2.1);
        \draw [line width=3pt,color=J1] (-4,2.5)-- (-5,2.5);                
        % Cadre
        \draw [line width=3pt,color=J1] (-4,4)-- (2,4) -- (2,1) -- (-4,1) -- cycle;
        % Après cadre
        \draw [->,line width=3pt,color=J1] (2,2.5) -- (3,2.5);
        \fill [line width=3pt,color=J1] (3.5,2.5) -- (3,1.5) -- (4,2.5) -- (3,3.5) -- cycle;
        % Textes
        \node[text width=3cm,text centered, scale=1.8] at (-1,3.5){\textbf{machine #1}};
        \node[text width=3cm,text centered, scale=1.5] at (-1,2.8){\textbf{---}};
        \node[text width=3cm,text centered, scale=1.5] at (-1,2){#2};
        \node[text width=3cm,text centered, scale=1.5] at (-8,2.5) {#3};
        \node[text width=3cm,text centered, scale=1.5] at (6,2.5) {#4};
    \end{tikzpicture}
}

% Figure avec carrés périphériques théorème de Pythagore

\newcommand{\figCarresPythagore}[7]{%
    % #1 scale
    % #2 PetiteLongueur
    % #3 MoyenneLongueur
    % #4 GrandeLongueur
    % #5 PetiteAire
    % #6 MoyenneAire
    % #7 GrandeAire
    \begin{tikzpicture}[scale=#1]
        % On crée le triangle
        \coordinate (A) at (1,1);
        \coordinate (B) at (6,1);
        \tkzInterCC[R](A,4)(B,3);
        \tkzGetFirstPoint{C};
        \draw[fill=gray!30] (A)--(B)--(C)--cycle;
        \tkzMarkRightAngles[size=0.3](A,C,B);
        % On construit les carrés sur les côtés
        \tkzDefSquare(B,A)\tkzGetPoints{E}{F};
        \tkzDrawPolygon(B,A,E,F);
        \tkzDefSquare(C,B)\tkzGetPoints{G}{H};
        \tkzDrawPolygon(C,B,G,H);
        \tkzDefSquare(A,C)\tkzGetPoints{I}{J};
        \tkzDrawPolygon(A,C,I,J);
        % On ajoute les labels de longueurs
        \tkzLabelSegment[auto,sloped,below](A,B){#4};
        \tkzLabelSegment[auto,sloped](B,C){#2};
        \tkzLabelSegment[auto,sloped](C,A){#3};
        % On ajoute les aires
        \tkzLabelSegment[yshift=3mm](A,F){#7};
        \tkzLabelSegment[yshift=5mm](B,H){#5};
        \tkzLabelSegment[xshift=-3mm,yshift=3mm](C,J){#6};
    \end{tikzpicture}
}

% Spirale d'arichmede et décimales de pi
% #1 -> facteur d'échelle, par défaut 2
% http://trucsmaths.free.fr/images/pi/pi2400.htm
\newcommand{\spiraleArchimedePi}[1][2]{
    \begin{tikzpicture}[decoration={text effects along path,
        text={{\red 3},141 592 653 589 793 238 462 643 383 279 502 884 197 169 399 375 105 820 974 944 592 307 816 406 286 208 998 628 034 825 342 117 067 982 148 086 513 282 306 647 093 844 609 550 582 231 725 359 408 128 481 117 450 284 102 701 938 521 105 559 644 622 948 954 930 381 964 428 810 975 665 933 446 128 475 648 233 786 783 165 271 201 909 145 648 566 923 460 348 610 454 326 648 213 393 607 260 249 141 273 724 587 006 606 315 588 174 881 520 920 962 829 254 091 715 364 367 892 590 360 011 330 530 548 820 466 521 384 146 951 941 511 609 433 057 270 365 759 591 953 092 186 117 381 932 611 793 105 118 548 074 462 379 962 749 567 351 885 752 724 891 227 938 183 011 949 129 833 673 362 440 656 643 086 021 394 946 395 224 737 190 702 179 860 943 702 770 539 217 176 293 176 752 384 674 818 467 669 405 132 000 568 127 145 263 560 827 785 771 342 757 789 609 173 637 178 721 468 440 901 224 953 430 146 549 585 371 050 792 279 689 258 923 542 019 956 112 129 021 960 864 034 418 159 \ },
        text effects/.cd,
        %repeat text,
        %character count=\m, character total=\n,
        %characters={text along path, scale=0.5+\m/\n/2}}]
        characters={text along path}},
        scale=#1]
        \path [draw=gray, ultra thin, postaction=decorate]
        (180:2) \foreach \a in {0,...,12}{ arc (180-\a*90:90-\a*90:1.5-\a/10) };
        \coordinate[label=below:{\fontsize{60}{60}\selectfont\bfseries $\pi$}] (pi) at (-0.3,0.3);
    \end{tikzpicture}
}

% Grid isometrique
% #1 -> couleur de la grid, black!30 par défaut
% #2 -> nombre d'unités en abscisse
% #3 -> nombre de double-unités en ordonnée
\newcommand{\isometricGrid}[3][black!30]{%
    \def\nx{#2}
    \pgfmathsetmacro\nxx{\nx-1}
    \def\ny{#3} 

    \draw[color=#1] (0,0) -- (\nx +1,0);
    \foreach \j in {0,...,\ny} {
        \foreach \i in {0,...,\nx} {
            \draw[color=#1](0:\i)++(60:\j) ++(120:\j)--++(60:2)--++(-1,0)--++(-60:1)--++(-60:1);
        }
        \foreach \i in {0,...,\nxx} {
            \draw[color=#1](0:\i+0.5)++(90:{(1+2*\j)*sin(60)})--++(1,0);
        }
    }
}

% Grid pointée isometrique
% #1 -> nombre d'unités en abscisse
% #2 -> nombre de double-unités en ordonnée
\newcommand{\isometricPointGrid}[2]{%
    \def\nx{#1}
    \pgfmathsetmacro\nxx{\nx-1}
    \def\ny{#2} 

    \foreach \j in {0,...,\ny} {
        \foreach \i in {0,...,\nx} {
            \fill(0:\i)++(60:\j) ++(120:\j)  circle (0.05) node {} ++(60:2) circle (0.05) node {} ++(-1,0) circle (0.05) node {} ++(-60:1) circle (0.05) node {} ++(-60:1) circle (0.05) node {};
        }
        \foreach \i in {0,...,\nxx} {
            \fill(0:\i+0.5)++(90:{(1+2*\j)*sin(60)})++(1,0);
        }
    }
}

%========= Translations
\newcommand{\coursVecteurFigUn}[1]{
    %line width=1pt,line cap=round,line join=round,>=triangle 45,x=1cm,y=1cm,
    \begin{tikzpicture}[scale=#1,>=triangle 45]
        \coordinate (A) at (-6,-6);
        \coordinate (C) at (-3,-4);
        \coordinate (E) at (-2,-1);
        \coordinate (G) at (-5,-3);
        \coordinate (B) at (1,-5);
        \tkzDefPointBy[translation= from A to B](G);
        \tkzGetPoint{H};
        \tkzDefPointBy[translation= from A to B](E);
        \tkzGetPoint{F};
        \tkzDefPointBy[translation= from A to B](C);
        \tkzGetPoint{D};
        \draw (A)--(C)--(E)--(G)--(A);
        \draw (B)--(D)--(F)--(H)--(B);
        \draw [->,color=orange] (A) -- (B);
        \draw [->,color=orange] (G) -- (H);
        \draw [->,color=orange] (E) -- (F);
        \draw [->,color=orange] (C) -- (D);
        \tkzDefMidPoint(A,B);
        \tkzGetPoint{M};
        \tkzLabelPoint[below](M){$\overrightarrow{AB}$}        
        \tkzDrawPoints[shape=cross out,size=2pt](A,B,C,D,E,F,G,H);
        \tkzLabelPoints[below](A,B,C,D);
        \tkzLabelPoints[above](E,F,G,H);
    \end{tikzpicture}
}
