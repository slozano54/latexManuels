% Piece de curvica - commande de Nathalie DAVAL
% #1 --> 
\def\curvica#1{
\begin{pspicture}(-1,-0.25)(3,2.5)
   #1
   \psset{linewidth=0.4mm,linestyle=dotted} %Curvica vierge
   \psframe(0,0)(2,2)
   \psarc(1,4){2.24}{-116.6}{-63.4}
   \psarc(1,0){2.24}{63.4}{116.6}
   \psarc(1,2){2.24}{-116.6}{-63.4}
   \psarc(1,-2){2.24}{63.4}{116.6}
   \psarc(4,1){2.24}{153.4}{-153.4}
   \psarc(0,1){2.24}{-26.6}{26.6}
   \psarc(2,1){2.24}{153.4}{-153.4}
   \psarc(-2,1){2.24}{-26.6}{26.6}
\end{pspicture}
}

%Pieces de curvica Bis pour 5e - commande de Nathalie DAVAL
\newcommand{\vg}{\psarc(3,0){2.24}{153.4}{-153.4}}
\newcommand{\vm}{\psline(1,-1)(1,1)}
\newcommand{\vd}{\psarc(-1,0){2.24}{-26.6}{26.6}}
\newcommand{\hb}{\psarc(0,3){2.24}{-116.6}{-63.4}}
\newcommand{\hm}{\psline(-1,1)(1,1)}
\newcommand{\hh}{\psarc(0,-1){2.24}{63.4}{116.6}}
\newcommand{\curvicaMDeux}[5]{\rput(#1,#2){#3} \rput(#1,#2){#4} \rput(#1,#2){\large #5}} 

% Hexagone regulier - commande de Nathalie DAVAL
% #1 --> rayon
% #2 --> couleur
\newcommand{\hexa}[2]{\pspolygon[fillstyle=solid,fillcolor=#2](#1;0)(#1;60)(#1;120)(#1;180)(#1;-120)(#1;-60)}

%pièces du tangram  - commande de Nathalie DAVAL
\def\pt{\psset{unit=4.24}\pspolygon(0,0)(1,0)(1,1)} 
\def\mt{\psset{unit=6}\pspolygon(0,0)(1,0)(1,1)} 
\def\gt{\psset{unit=8.49}\pspolygon(0,0)(1,0)(1,1)} 
\def\ca{\psset{unit=4.24}\psframe(0,0)(1,1)}
\def\pa{\psset{unit=3}\pspolygon(0,0)(-2,0)(-3,1)(-1,1)}
\def\pas{\psset{unit=3}\pspolygon(0,0)(-2,0)(-3,-1)(-1,-1)}

% Pour la distributivité commandes de Nathalie DAVAL
\newcommand{\tri}[3]{\pspolygon(0,0)(2,0)(2;60) \rput(1,0.25){$#1$} \rput{-120}(0.7,0.8){$#2$} \rput{120}(1.3,0.8){$#3$}}
\newcommand{\car}[4]{\psframe(0,0)(2,2) \rput(1,0.2){$#1$} \rput{90}(1.8,1){$#2$} \rput{180}(1,1.8){$#3$} \rput{-90}(0.2,1){$#4$}}

% Geometrie dans l'espace - commande de NAthalie DAVAL
\newcommand{\coneND}{\pspolygon[fillstyle=solid,fillcolor=white](0,0)(0.9,0)(0.45,1.7)}
\newcommand{\bouleND}{\pscircle[fillstyle=solid,fillcolor=white](0,0.45){0.45}}
\newcommand{\cubeND}{\psframe[fillstyle=solid,fillcolor=white](0,0)(1.15,0.9)\psline(0.75,0)(0.75,0.9)}
\newcommand{\cubegND}{\psframe[fillstyle=solid,fillcolor=white](0,0)(1.15,0.9)\psline(0.4,0)(0.4,0.9)}

% Cube isometrique - commande de Nathalie DAVAL
%Cube
\newcommand{\cubiso}[2]{
   \rput(#1,#2) 
      {\psset{fillstyle=solid}
       \pspolygon[fillcolor=purple!80](0,0)(2;90)(2;30)(2;-30)
       \pspolygon[fillcolor=lightgray](2;-30)(3.464;0)(4;30)(2;30)
       \pspolygon[fillcolor=teal](2;90)(2;30)(4;30)(3.464;60)}
}
%Cube v2
\newcommand{\cubisoBis}[2]{
   \rput(#1,#2)
   {\pspolygon[fillstyle=solid,fillcolor=red](0,0)(0,2)(-2,3)(-2,1) \pspolygon[fillstyle=solid,fillcolor=lightgray](0,0)(2,1)(2,3)(0,2) \pspolygon[fillstyle=solid,fillcolor=black](0,2)(2,3)(0,4)(-2,3)}
}
%==========================================
% Probabilités
%==========================================
% Commande de Nathalie DAVAL
% #1 --> 
\newcommand{\sac}[1]{\begin{pspicture}(-0.5,-0.5)(2,2)
    \psline(0,0)(0,1.5)
    \psline(1.5,0)(1.5,1.5)
    \psellipse(0.75,1.5)(0.75,0.25)
    \psellipticarc(0.75,0)(0.75,0.4){180}{0}
    #1
 \end{pspicture}}
