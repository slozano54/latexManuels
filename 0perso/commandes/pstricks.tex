% Piece de curvica - commande de Nathalie DAVAL
% #1 --> 
\def\curvica#1{
\begin{pspicture}(-1,-0.25)(3,2.5)
   #1
   \psset{linewidth=0.4mm,linestyle=dotted} %Curvica vierge
   \psframe(0,0)(2,2)
   \psarc(1,4){2.24}{-116.6}{-63.4}
   \psarc(1,0){2.24}{63.4}{116.6}
   \psarc(1,2){2.24}{-116.6}{-63.4}
   \psarc(1,-2){2.24}{63.4}{116.6}
   \psarc(4,1){2.24}{153.4}{-153.4}
   \psarc(0,1){2.24}{-26.6}{26.6}
   \psarc(2,1){2.24}{153.4}{-153.4}
   \psarc(-2,1){2.24}{-26.6}{26.6}
\end{pspicture}
}

% Hexagone regulier - commande de Nathalie DAVAL
% #1 --> rayon
% #2 --> couleur
\newcommand{\hexa}[2]{\pspolygon[fillstyle=solid,fillcolor=#2](#1;0)(#1;60)(#1;120)(#1;180)(#1;-120)(#1;-60)}

%==========================================
% Probabilités
%==========================================
% Commande de Nathalie DAVAL
% #1 --> 
\newcommand{\sac}[1]{\begin{pspicture}(-0.5,-0.5)(2,2)
    \psline(0,0)(0,1.5)
    \psline(1.5,0)(1.5,1.5)
    \psellipse(0.75,1.5)(0.75,0.25)
    \psellipticarc(0.75,0)(0.75,0.4){180}{0}
    #1
 \end{pspicture}}
