% 1ere commande avec simplekv
% Reseau de paralléogrammes
\setKVdefault[Reseau]{%
    Lignes=5,%
    Colonnes=3,%
    Decalagex=0.2,%
    Decalagey=0.5,%
    Couleur=Grey,
    TracesAAjouter=false
}%
\defKV[Reseau]{Traces=\setKV[Reseau]{TracesAAjouter}}

\def\MPReseau#1#2#3{%
  \ifluatex%
  \mplibforcehmode%
  \begin{mplibcode}%
  	  %%% Maille parallélogramme  
  	  vardef parallelo(expr num)=
        save Loz;
        picture Loz;
        path parallelog;
        parallelog:=(0,0)*u--(1,0)*u--(1 + \useKV[Reseau]{Decalagex},\useKV[Reseau]{Decalagey})*u--(\useKV[Reseau]{Decalagex},\useKV[Reseau]{Decalagey})*u--cycle;
        Loz=image(
            trace parallelog withcolor #3;
        );
        Loz
      enddef;
      %%% Numérotation
      vardef numerotation(expr num)=
        save Loz;
        picture Loz;
        Loz=image(
            pair centreparalello;
            centreparalello=1/2[(0,0)*u,(1+\useKV[Reseau]{Decalagex},\useKV[Reseau]{Decalagey})*u];
            label(TEX(decimal(num)), centreparalello);
        );
        Loz
      enddef;
      %%% Placer un point dans le réseau
      vardef ppreseau(expr q,r)=
        save point;
      	pair point;
      	point = (0,0) shifted ((1*q*u+r*\useKV[Reseau]{Decalagex}*u,0)+r*(0,\useKV[Reseau]{Decalagey}*u));
		point
	  enddef;
	  %%% Construction de la figure
      for l=0 upto #1-1:
        for k=0 upto #2-1:
            trace parallelo(#1*l+k) shifted(u*(\useKV[Reseau]{Decalagex}*l+k,\useKV[Reseau]{Decalagey}*l));
        endfor;
      endfor;
      if \useKV[Reseau]{TracesAAjouter}:
        \useKV[Reseau]{Traces};
      fi;
      for l=0 upto #1-1:
        for k=0 upto #2-1:
            trace numerotation(#1*l+k) shifted(u*(\useKV[Reseau]{Decalagex}*l+k,\useKV[Reseau]{Decalagey}*l));
        endfor;
      endfor;
  \end{mplibcode}
  \else%
  \begin{mpost}[mpsettings={
  		def traces=
  		begingroup
          \useKV[Reseau]{Traces}
        endgroup
        enddef;
        decalagex:=\useKV[Reseau]{Decalagex};
        decalagey:=\useKV[Reseau]{Decalagey};
  }]%
  	%%% Maille parallélogramme    	  
  	vardef parallelo(expr num)=
        save Loz;
        picture Loz;
        path parallelog;
        parallelog:=(0,0)*u--(1,0)*u--(1+decalagex,decalagey)*u--(decalagex,decalagey)*u--cycle;
        Loz=image(
            trace parallelog withcolor #3;
        );
        Loz
    enddef;
    %%% Numérotation
    vardef numerotation(expr num)=
        save Loz;
        picture Loz;
        Loz=image(
        trace parallelog withcolor #3;
        pair centreparalello;
        centreparalello=1/2[(0,0)*u,(1+decalagex,decalagey)*u];
        label(LATEX(decimal(num)), centreparalello);
        );
        Loz
    enddef;
    %%% Placer un point dans le réseau
    vardef ppreseau(expr q,r)=
        save point;
        pair point;
        point = (0,0) shifted ((1*q*u+r*decalagex*u,0)+r*(0,decalagey*u));
        point
    enddef;
    %%% Construction de la figure
    for l=0 upto #1-1:
        for k=0 upto #2:
            trace parallelo(#1*l+k) shifted(u*(decalagex*l+k,decalagey*l));
        endfor;
    endfor;
    traces;
    for l=0 upto #1-1:
        for k=0 upto #2:
            trace numerotation(#1*l+k) shifted(u*(decalagex*l+k,decalagey*l));
        endfor;
    endfor;
  \end{mpost}%
  \fi%
}
\NewDocumentCommand\Reseau{o}{%
  \useKVdefault[Reseau]% On revient aux valeurs par défaut, équivalent à restoreKV[Reseau]
  \setKV[Reseau]{#1}% On lit les arguments optionnels
  \xdef\ReseauLignes{\useKV[Reseau]{Lignes}}%
  \xdef\ReseauColonnes{\useKV[Reseau]{Colonnes}}%
  \xdef\ReseauCouleur{\useKV[Reseau]{Couleur}}%
  \MPReseau{\ReseauLignes}{\ReseauColonnes}{\ReseauCouleur}
}