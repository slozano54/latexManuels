\newcommand{\myUrlColorStyle}[1]{\textcolor{blue}{\underline{\bfseries #1}}}
% Pour le lien vers un exo mathaleaV2
% #1 --> Texte par défaut
% Par défaut #1 est le smiley yeux etoilés + smiley link + texte mais on peut le remplacer par la ref mathalea
% #2 --> url
\newcommand{\hrefMathalea}[2][\emoji{star-struck} \emoji{link} \myUrlColorStyle{S'entraîner sur le site} \mathaleaLogo]{
    \qrcode[hyperlink,height=0.4in]{#2}\hspace*{2mm}\href{#2}{#1}
}

% On définit le séparateur pour le paquet listofitems
\setsepchar{,}%

% Une précommande pour concaténer tous les éléments d'une liste
% #1 --> list des uuid séparés par des virgules
\newcommand{\createListOfUuid}[1]{%
	\readlist*\myTmpList{#1}%
	\xdef\listOfUuid{uuid=\myTmpList[1]}%
	\ifthenelse{\myTmpListlen > 1}{%
		\foreach \i in {2,...,\myTmpListlen}{%
			\expandafter\xdef\expandafter\listOfUuid\expandafter{\listOfUuid\&uuid=\myTmpList[\i]}%
		}%
		}{%
			\xdef\listOfUuid{uuid=\myTmpList[1]}%
		}%
}

% Pour le lien vers un exo mathaleaV3
% #1 --> Texte par défaut
% Par défaut #1 est le smiley yeux etoilés + smiley link + texte mais on peut le remplacer par la ref mathalea
% #2 --> liste d'uuid de l'exercice
% #3 --> vue
% #4 --> parametrage de la vue eleve
\newcommand{\hrefMathaleaVIII}[4][\emoji{star-struck} \emoji{link} \myUrlColorStyle{S'entraîner sur le site} \mathaleaLogo]{
	\createListOfUuid{#2}    
    \qrcode[hyperlink,height=0.4in]{https://coopmaths.fr/alea/?\listOfUuid&v=#3&es=#4}\hspace*{2mm}\href{https://coopmaths.fr/alea/?\listOfUuid&v=#3&es=#4}{#1}
}

% Pour la géométrie, voir des constructions etc ...
% #1 --> emojis par defaut
% Par défaut #1 est emoji equerre + emoji regle + le smiley link mais cela peut être modifié
% #2 --> url
% #3 --> texte
\newcommand{\hrefConstruction}[3][\emoji{triangular-ruler} \emoji{straight-ruler} \emoji{link}]{
    \qrcode[hyperlink,height=0.4in]{#2}\hspace*{2mm}\href{#2}{#1 \myUrlColorStyle{#3}} 
}

% Pour les liens vers youtube
% #1 --> emojis par défaut
% Par défaut #1 est le movie-camera + link mais on peut le remplacer par la ref mathalea
% #2 --> url
% #3 --> texte
\newcommand{\hrefVideo}[3][\emoji{movie-camera} \emoji{link}]{
    \qrcode[hyperlink,height=0.4in]{#2}\hspace*{2mm}\href{#2}{#1 \myUrlColorStyle{#3}}
}

% Pour les liens de révision
% #1 --> emojis par defaut
% Par défaut #1 est emoji check-mark-button + link mais cela peut être modifié
% #2 --> url
% #3 --> texte
\newcommand{\hrefRevision}[3][\emoji{check-mark-button} \emoji{link}]{
    \qrcode[hyperlink,height=0.4in]{#2}\hspace*{2mm}\href{#2}{#1 \myUrlColorStyle{#3}}
}

% Pour les liens lambda
% #1 --> emojis par défaut
% Par défaut #1 est le movie-camera + link mais on peut le remplacer par la ref mathalea
% #2 --> url
% #3 --> texte
\newcommand{\hrefLien}[3][\emoji{link}]{
    \qrcode[hyperlink,height=0.4in]{#2}\hspace*{2mm}\href{#2}{#1 \myUrlColorStyle{#3}}
}