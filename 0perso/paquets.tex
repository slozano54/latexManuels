\usepackage{sesamanuelTIKZ}
\usepackage{ProfCollege}

% Pour la gestion des fontes maths notamment en mode LuaLaTeX.
\usepackage{unicode-math}
% Par exemple, une fonte sans serif pour les briques Scratch.
\newfontfamily\myfontScratch[]{FreeSans}
% Support emoji /!\ uniquement en mode lualatex
\usepackage{emoji}

% Pour travailler sur les listes enumeratives
\usepackage{enumitem}

% Pour générer du faux texte
\usepackage{lipsum}

% Pour les espaces interlignes
\usepackage{setspace}

% Pour fixer les Warnings : "Underfull \hbox (badness 10000)"
\usepackage{parskip} 

% Pour les hyperliens, notamment des sommaires cliquables
% \usepackage{hyperref}
\usepackage[luatex]{hyperref}
\hypersetup{
    colorlinks=true,% On active la couleur pour les liens. Couleur par défaut rouge
    linkcolor=black,% On définit la couleur pour les liens internes
    % filecolor=magenta,% On définit la couleur pour les liens vers les fichiers locaux      
    urlcolor=black,% On définit la couleur pour les liens vers des sites web
    % pdftitle={Puissance Quatre},% On définit un titre pour le document pdf
    % pdfpagemode=FullScreen,% On fixe l'affichage par défaut à plein écran
    }

% Extensions tikz supplémentaires pour marquage des angles
\usetikzlibrary{angles,quotes}

% Pour la géométrie euclidienne avec tikz
\usepackage{tkz-euclide}
% Pour la géométrie euclidienne avec pstricks
\usepackage{pst-eucl}

% Pour les lettres qualigrafié géométrie
\setmathfont{XITS Math}

% Pour des soulignés spéciaux
\usepackage{ulem}

% Pour entourer du texte
\usepackage{circledsteps}

% Écriture cursive
\usepackage{frcursive}

% Pour générer des qrcode
\usepackage{qrcode}

% Pour la numération égyptienne
\usepackage{hieroglf}