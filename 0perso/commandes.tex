% ============================================================================================
% ======= 1ere et 4eme de couverture
% ============================================================================================

\newcommand{\myAuthorName}{Sébastien LOZANO}

\newcommand{\myAuthorSchoolName}{Collège Jean Lurçat 54390 FROUARD}

\newcounter{postCurrentSchoolYear}
\setcounter{postCurrentSchoolYear}{\the\year}
\addtocounter{postCurrentSchoolYear}{1}
\newcommand{\currentSchoolYear}{Année  \the\year ~- \thepostCurrentSchoolYear}

\newcommand{\mySite}{\href{https://mathslozano.fr}{https://mathslozano.fr}}

\newcommand{\myManualName}{Master Manuel \LaTeX de \currentNiveau}

\newcommand{\myMessage}{Nom, Lieu de travail et Année courante, sont à modifier dans le fichier
./config.tex  du manuel concerné \par Ce message est à supprimer en supprimant l'appel à la commande
\textbackslash myMessage !
}

% ============================================================================================
% ======= Sommaire
% ============================================================================================

% Pour pouvoir séparer la numérotation des chapitres en fonction des parties
% Important pour que les liens cliquables du sommaire renvoient au bon endroit
\counterwithin*{chapter}{part}

% En prévision de l'application d'un style particulier
% pour les parties dans le sommaire
% #1 --> couleur
% #2 --> texte
\newcommand{\myTocFrame}[2]{%
    \addtocontents{toc}{
        \vspace{1cm}
        \begin{cadre}[#1][#1!50]
            \begin{center}
                #2
            \end{center}
        \end{cadre}
    }
}

% ============================================================================================
% Factorisation de commandes
% ============================================================================================

%==========================================
% Gestion enigmes/recreation
%==========================================
% Pour pouvoir numéroter les énigmes quand on en met plusieurs
% dans la section \recreation
\newcommand{\numeroteEnigme}{
    \begin{pspicture}(0,0)(\ExerciceNumFrameWidth,\ExerciceNumFrameHeight)
        \psframe*[linewidth=0pt,
                  linecolor=LibreExerciceNumFrameColor]
                 (0,-\ExerciceNumFrameDepth)
                 (\ExerciceNumFrameWidth,\ExerciceNumFrameHeight)
      \rput[B](\dimexpr\ExerciceNumFrameWidth/2,0){%
        \textcolor{LibreExerciceNumColor}{\ExerciceNumFont \theexercice}%
      }
  \end{pspicture}  
}

%==========================================
% Gestion Glossaire
%==========================================
% ======= Il n'y a qu'un glossaire, est-ce judicieux de regrouper ici ?
% ======= Je pense que oui, question de "séparatisme" !
% Pour uniformiser les titres de l'environnement acquis
% #1 --> titre
\newcommand{\titreConnaissancesAcquis}[1]{%
  \textcolor{G1}{\textbf{#1}} :
}

% On factorise les titres des sections pour le glossaire de propriété
% #1 --> Titre principal
% #2 --> couleur
% #3 --> sous-titre
\newcommand{\sectionsGlossaireProprietes}[3]{
    \section{#1 \textcolor{#2}{#3}}
}
\newcommand{\titreSectionGlossairePropUn}{
    \sectionsGlossaireProprietes{Section 1}{A1}{texte en couleur différente}
}
\newcommand{\titreSectionGlossairePropDeux}{
    \sectionsGlossaireProprietes{Section 2}{A1}{texte en couleur différente}
}

%==========================================
% Gestion des propriétés
%==========================================

% Pour écrire (admise) pour les propriétés
\newcommand{\admise}{(admise)}

% Pour écrire (admis) pour les théorèmes
\newcommand{\admis}{(admis)}

%==========================================
% Gestion des liens
%==========================================
% Pour le lien vers un exo interactif mathalea
% #1 --> Texte par défaut
% Par défaut #1 est le smiley yeux etoilés + smiley link + texte mais on peut le remplacer par la ref mathalea
% #2 --> url
\newcommand{\hrefMathalea}[2][\emoji{star-struck} \emoji{link} S'entraîner sur le site \mathaleaLogo]{
    \href{#2}{#1} 
}

% Pour la géométrie, voir des constructions etc ...
% #1 --> emojis par dafaut
% Par défaut #1 est emoji equerre + emoji regle + le smiley link mais cela peut être modifié
% #2 --> url
% #3 --> texte
\newcommand{\hrefConstruction}[3][\emoji{triangular-ruler} \emoji{straight-ruler} \emoji{link}]{
    \href{#2}{#1 #3} 
}

%==========================================
% Gestion des boxs
%==========================================

% Box perso, on peut l'utiliser pour les programmes de calcul
% #1 --> Contenu
% #2 --> Titre
\newtcolorbox{myBox}[2][]{
    enhanced,
    boxsep=1mm,
    bottom=.75mm,
    boxrule=2pt,
    text width=0.75\linewidth,
    colframe=gray,
    colback=gray!20,
    colbacktitle=white,
    fonttitle=\bfseries\color{black},
    halign upper=center,
    attach boxed title to top center={yshift=-2mm},
    title={#2},#1
}

% Patch pour pouvoir redefinir un compteur via \setlist
% Necessaire pour la commande \ProgCalcul du paquet profcollege
\let\enumerateold\enumerate
\let\endenumerateold\endenumerate
% #1 --> titre
\newcommand{\myTCBset}[1]{
    \tcbset{ProgCalcul/.style={%
    enhanced,
    boxsep=1mm,
    bottom=.75mm,
    boxrule=2pt,
    text width=0.75\linewidth,
    colframe=gray,
    colback=gray!20,
    colbacktitle=white,
    fonttitle=\bfseries\color{black},
    halign upper=center,
    attach boxed title to top center={yshift=-2mm},
    title={#1},
    }%
    }%   
}
% Un programme de calcul encadré
% #1 --> label 
% #2 --> titre
% #3 --> Commande \ProgCalcul du paquet ProfCollege
\newcommand{\myProgCalcul}[3]{% label, titre, commande \ProgCalcul
    \begingroup
    \let\enumerate\enumerateold
    \let\endenumerate\endenumerateold
    \setlist[enumerate]{label=#1}
    \myTCBset{#2}
    #3
    \endgroup
}

%==========================================
% Tableaux
%==========================================
% Pour l'environnement tabularx
\newcolumntype{C}{>{\centering}X}

%==========================================
% Couleurs
%==========================================
\definecolor{mygreen}{rgb}{0.0, 0.5, 0.0}

%==========================================
% Notations 
%==========================================

%==========================================
% Gestion des exemples 
%==========================================
\newcommand{\titreExemple}[1]{
    {\color{red}\bfseries #1}
}
%==========================================
% Textes répétitifs
%==========================================
%====Pour renvoyer vers des compléments numériques
% #1 --> Paramètre à passer, singulier ou pluriel en minuscules
\newcommand{\infoComplementsNumeriques}[1]{
    \ifthenelse{\equal{#1}{singulier}}{\emoji{face-with-monocle} \textbf{Complément numérique}}{%
        \ifthenelse{\equal{#1}{pluriel}}{\emoji{face-with-monocle} \textbf{Compléments numériques}}{%
            \textbf{#1}
        }
    }
}
%==========================================
% Gestion des crédits
%==========================================
% Pour avoir une note de bas de page sans marque en utilisant \footnotetext !
\makeatletter
    \def\blfootnote{\gdef\@thefnmark{}\@footnotetext}
\makeatother

% InstrumentPoche
\newcommand{\creditInstrumentPoche}{
    \blfootnote{Source : Animations InstrumentPoche association Sésamath}
}

% Geogebra
% #1 --> Auteur
\newcommand{\creditGeogebra}[1]{
    \blfootnote{Source : D'après une appliquette Geogebra de #1}
}

% Cahiers Iparcours
% #1 --> Année
% #2 --> Niveau
\newcommand{\creditCahiersIparcours}[2]{
    \blfootnote{Source : Fiches tirées du cahier iParcours #1 de #2}
}

% Manuels Iparcours
% #1 --> Niveau
\newcommand{\creditManuelsIparcours}[1]{
    \blfootnote{Source : Exercices tirés du manuel iParcours de #1}
}

% Cahiers Sesamath
% #1 --> Année
% #1 --> Niveau
\newcommand{\creditCahiersSesamath}[2]{
    \blfootnote{Source : Fiches tirées du cahier Sésamath #1 de #2}
}

% Manuels Sesamath
% #1 --> Niveau
\newcommand{\creditManuelsSesamath}[1]{
    \blfootnote{Source : Exercices tirés du manuel Sésamath de #1}
}

% Libre
% #1 --> texte
\newcommand{\creditLibre}[1]{
    \blfootnote{Source : #1}
}

% Macro pour entourer les opérandes
% #1 --> 
% #2 --> 
% #3 --> 
\newcommand{\OPoval}[3]{\dimen1=#2\opcolumnwidth \ovalnode{#1}{\kern\dimen1 #3\kern\dimen1}}

%==========================================
% Instruments de géométrie
%==========================================
% Equerre
% #1 --> 
% #2 --> 
% #3 --> 
% #4 --> 
\newcommand{\equerre}[4]
   {\scalebox{#4}
      {\rput{#3}(#1,#2)      
         {\pspolygon[linecolor=B1](0,0)(1,0)(0,1.8)
          \pspolygon[linecolor=B1](0.2,0.2)(0.65,0.2)(0.2,1)
          \multido{\r=0+0.1}{17}{\psline[linecolor=red,linewidth=0.01](0,\r)(0.075,\r)}
          }
       }      
   }

%==========================================
% TiKz 
%==========================================
% Pavage en L dans un quadrillage 8x8
\newcommand{\pavageL}{%
    \foreach \a in {0,-2,-4}{%
        \draw[shift={(\a,-\a)}] (14,0) -- (14,2) -- (16,2) -- (16,4) -- (12,4) -- (12,0) -- cycle;
        \draw[shift={(\a,-\a)}] (15,2) -- (15,3) -- (13,3) -- (13,1) -- (14,1);
        \draw[shift={(\a,-\a)}] (12,2) -- (13,2);
        \draw[shift={(\a,-\a)}] (14,3) -- (14,4);
        \draw[shift={(\a,-\a)}] (14,0) -- (15,0) -- (15,1) -- (16,1) -- (16,2);
    }
    \foreach \a in {0}{%
        \draw[shift={(\a,-\a)},rotate around={90:(12,0)}] (14,0) -- (14,2) -- (16,2) -- (16,4) -- (12,4) -- (12,0) -- cycle;
        \draw[shift={(\a,-\a)},rotate around={90:(12,0)}] (15,2) -- (15,3) -- (13,3) -- (13,1) -- (14,1);
        \draw[shift={(\a,-\a)},rotate around={90:(12,0)}] (12,2) -- (13,2);
        \draw[shift={(\a,-\a)},rotate around={90:(12,0)}] (14,3) -- (14,4);
        \draw[shift={(\a,-\a)},rotate around={90:(12,0)}] (14,0) -- (15,0) -- (15,1) -- (16,1) -- (16,2);
    }
    \foreach \a in {0}{%
        \draw[shift={(\a,-\a)},rotate around={-90:(16,4)}] (14,0) -- (14,2) -- (16,2) -- (16,4) -- (12,4) -- (12,0) -- cycle;
        \draw[shift={(\a,-\a)},rotate around={-90:(16,4)}] (15,2) -- (15,3) -- (13,3) -- (13,1) -- (14,1);
        \draw[shift={(\a,-\a)},rotate around={-90:(16,4)}] (12,2) -- (13,2);
        \draw[shift={(\a,-\a)},rotate around={-90:(16,4)}] (14,3) -- (14,4);
        \draw[shift={(\a,-\a)},rotate around={-90:(16,4)}] (14,0) -- (15,0) -- (15,1) -- (16,1) -- (16,2);
    }
}

% Image d'une maille définie par A,B,C,D,E,F par homothetie de centre O
% #1 -> ratio
% #2 -> couleur
\newcommand{\imageHomothetyHexaGone}[2]{
    \tkzDefPointBy[homothety=center O ratio #1](A); \tkzGetPoint{Ab};
    \tkzDefPointBy[homothety=center O ratio #1](B); \tkzGetPoint{Bb};
    \tkzDefPointBy[homothety=center O ratio #1](C); \tkzGetPoint{Cb};
    \tkzDefPointBy[homothety=center O ratio #1](D); \tkzGetPoint{Db};
    \tkzDefPointBy[homothety=center O ratio #1](E); \tkzGetPoint{Eb};
    \tkzDefPointBy[homothety=center O ratio #1](F); \tkzGetPoint{Fb};
    \draw[color=#2,fill=#2,fill opacity=0.5] (Ab) -- (Bb) -- (Cb) -- (Db) -- (Eb) -- (Fb) -- cycle;
}
