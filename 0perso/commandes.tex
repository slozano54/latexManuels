% ============================================================================================
% ======= 1ere et 4eme de couverture
% ============================================================================================

\newcommand{\myAuthorName}{Sébastien LOZANO}

\newcommand{\myAuthorSchoolName}{Collège Jean Lurçat 54390 FROUARD}

\newcounter{postCurrentSchoolYear}
\setcounter{postCurrentSchoolYear}{\the\year}
\addtocounter{postCurrentSchoolYear}{1}
\newcommand{\currentSchoolYear}{Année  \the\year ~- \thepostCurrentSchoolYear}

\newcommand{\mySite}{\href{https://mathslozano.fr}{https://mathslozano.fr}}

\newcommand{\myManualName}{Master Manuel \LaTeX de \currentNiveau}

\newcommand{\myMessage}{Nom, Lieu de travail et Année courante, sont à modifier dans le fichier
./config.tex  du manuel concerné \par Ce message est à supprimer en supprimant l'appel à la commande
\textbackslash myMessage !
}

% ============================================================================================
% ======= Sommaire
% ============================================================================================

% Pour pouvoir séparer la numérotation des chapitres en fonction des parties
% Important pour que les liens cliquables du sommaire renvoient au bon endroit
\counterwithin*{chapter}{part}

% En prévision de l'application d'un style particulier
% pour les parties dans le sommaire
% #1 --> couleur
% #2 --> texte
\newcommand{\myTocFrame}[2]{%
    \addtocontents{toc}{
        \vspace{1cm}
        \begin{cadre}[#1][#1!50]
            \begin{center}
                #2
            \end{center}
        \end{cadre}
    }
}

% ============================================================================================
% Factorisation de commandes
% ============================================================================================

%==========================================
% Gestion enigmes/recreation
%==========================================
% Pour pouvoir numéroter les énigmes quand on en met plusieurs
% dans la section \recreation
\newcommand{\numeroteEnigme}{
    \begin{pspicture}(0,0)(\ExerciceNumFrameWidth,\ExerciceNumFrameHeight)
        \psframe*[linewidth=0pt,
                  linecolor=LibreExerciceNumFrameColor]
                 (0,-\ExerciceNumFrameDepth)
                 (\ExerciceNumFrameWidth,\ExerciceNumFrameHeight)
      \rput[B](\dimexpr\ExerciceNumFrameWidth/2,0){%
        \textcolor{LibreExerciceNumColor}{\ExerciceNumFont \theexercice}%
      }
  \end{pspicture}  
}

%==========================================
% Gestion Glossaire
%==========================================
% ======= Il n'y a qu'un glossaire, est-ce judicieux de regrouper ici ?
% ======= Je pense que oui, question de "séparatisme" !
% Pour uniformiser les titres de l'environnement acquis
% #1 --> titre
\newcommand{\titreConnaissancesAcquis}[1]{%
  \textcolor{G1}{\textbf{#1}} :
}

% On factorise les titres des sections pour le glossaire de propriété
% #1 --> Titre principal
% #2 --> couleur
% #3 --> sous-titre
\newcommand{\sectionsGlossaireProprietes}[3]{
    \section{#1 \textcolor{#2}{#3}}
}
\newcommand{\titreSectionGlossairePropUn}{
    \sectionsGlossaireProprietes{Section 1}{A1}{texte en couleur différente}
}
\newcommand{\titreSectionGlossairePropDeux}{
    \sectionsGlossaireProprietes{Section 2}{A1}{texte en couleur différente}
}

%==========================================
% Gestion des propriétés
%==========================================

% Pour écrire (admise) pour les propriétés
\newcommand{\admise}{(admise)}

\newcommand{\admises}{(admises)}

% Pour écrire (admis) pour les théorèmes
\newcommand{\admis}{(admis)}

%==========================================
% Gestion des liens
%==========================================
\newcommand{\myUrlColorStyle}[1]{\textcolor{blue}{\underline{\bfseries #1}}}
% Pour le lien vers un exo interactif mathalea
% #1 --> Texte par défaut
% Par défaut #1 est le smiley yeux etoilés + smiley link + texte mais on peut le remplacer par la ref mathalea
% #2 --> url
\newcommand{\hrefMathalea}[2][\emoji{star-struck} \emoji{link} \myUrlColorStyle{S'entraîner sur le site} \mathaleaLogo]{
    \qrcode[hyperlink,height=0.4in]{#2}\hspace*{2mm}\href{#2}{#1}
}

% Pour la géométrie, voir des constructions etc ...
% #1 --> emojis par defaut
% Par défaut #1 est emoji equerre + emoji regle + le smiley link mais cela peut être modifié
% #2 --> url
% #3 --> texte
\newcommand{\hrefConstruction}[3][\emoji{triangular-ruler} \emoji{straight-ruler} \emoji{link}]{
    \qrcode[hyperlink,height=0.4in]{#2}\hspace*{2mm}\href{#2}{#1 \myUrlColorStyle{#3}} 
}

% Pour les liens vers youtube
% #1 --> emojis par défaut
% Par défaut #1 est le movie-camera + link mais on peut le remplacer par la ref mathalea
% #2 --> url
% #3 --> texte
\newcommand{\hrefVideo}[3][\emoji{movie-camera} \emoji{link}]{
    \qrcode[hyperlink,height=0.4in]{#2}\hspace*{2mm}\href{#2}{#1 \myUrlColorStyle{#3}}
}

% Pour les liens de révision
% #1 --> emojis par defaut
% Par défaut #1 est emoji check-mark-button + link mais cela peut être modifié
% #2 --> url
% #3 --> texte
\newcommand{\hrefRevision}[3][\emoji{check-mark-button} \emoji{link}]{
    \qrcode[hyperlink,height=0.4in]{#2}\hspace*{2mm}\href{#2}{#1 \myUrlColorStyle{#3}}
}

% Pour les liens lambda
% #1 --> emojis par défaut
% Par défaut #1 est le movie-camera + link mais on peut le remplacer par la ref mathalea
% #2 --> url
% #3 --> texte
\newcommand{\hrefLien}[3][\emoji{link}]{
    \qrcode[hyperlink,height=0.4in]{#2}\hspace*{2mm}\href{#2}{#1 \myUrlColorStyle{#3}}
}
%==========================================
% Gestion des boxs
%==========================================

% Box perso, on peut l'utiliser pour les programmes de calcul
% #1 --> Contenu
% #2 --> Titre
\newtcolorbox{myBox}[2][]{
    enhanced,
    boxsep=1mm,
    bottom=.75mm,
    boxrule=2pt,
    text width=0.75\linewidth,
    colframe=gray,
    colback=gray!20,
    colbacktitle=white,
    fonttitle=\bfseries\color{black},
    halign upper=center,
    attach boxed title to top center={yshift=-2mm},
    title={#2},#1
}

% Patch pour pouvoir redefinir un compteur via \setlist
% Necessaire pour la commande \ProgCalcul du paquet profcollege
\let\enumerateold\enumerate
\let\endenumerateold\endenumerate
% #1 --> titre
\newcommand{\myTCBset}[1]{
    \tcbset{ProgCalcul/.style={%
    enhanced,
    boxsep=1mm,
    bottom=.75mm,
    boxrule=2pt,
    text width=0.75\linewidth,
    colframe=gray,
    colback=gray!20,
    colbacktitle=white,
    fonttitle=\bfseries\color{black},
    halign upper=center,
    attach boxed title to top center={yshift=-2mm},
    title={#1},
    }%
    }%   
}
% Un programme de calcul encadré
% #1 --> label 
% #2 --> titre
% #3 --> Commande \ProgCalcul du paquet ProfCollege
\newcommand{\myProgCalcul}[3]{% label, titre, commande \ProgCalcul
    \begingroup
    \let\enumerate\enumerateold
    \let\endenumerate\endenumerateold
    \setlist[enumerate]{label=#1}
    \myTCBset{#2}
    #3
    \endgroup
}

%==========================================
% Tableaux
%==========================================
% Pour l'environnement tabularx
\newcolumntype{C}{>{\centering}X}

%==========================================
% Couleurs
%==========================================
\definecolor{mygreen}{rgb}{0.0, 0.5, 0.0}

%==========================================
% Notations 
%==========================================

%==========================================
% Gestion des exemples, remarques, preuves
%==========================================
\newcommand{\titreExemple}[1]{
    {\color{red}\bfseries #1}
}
\newcommand{\titreRemarque}[1]{
    {\color{RemTitleColor}\bfseries #1}
}

\newcommand{\titrePreuve}[1]{
    {\color{ProofTitleColor}\bfseries #1}
}

%==========================================
% Textes répétitifs
%==========================================
%====Pour renvoyer vers des compléments numériques
% #1 --> Paramètre à passer, singulier ou pluriel en minuscules
\newcommand{\infoComplementsNumeriques}[1]{
    \ifthenelse{\equal{#1}{singulier}}{\emoji{face-with-monocle} \textbf{Complément numérique}}{%
        \ifthenelse{\equal{#1}{pluriel}}{\emoji{face-with-monocle} \textbf{Compléments numériques}}{%
            \textbf{#1}
        }
    }
}
%==========================================
% Gestion des crédits
%==========================================
% Pour avoir une note de bas de page sans marque en utilisant \footnotetext !
\makeatletter
    \def\blfootnote{\gdef\@thefnmark{}\@footnotetext}
\makeatother

% InstrumentPoche
\newcommand{\creditInstrumentPoche}{
    \blfootnote{Source : Animations InstrumentPoche association Sésamath}
}

% Geogebra
% #1 --> Auteur
\newcommand{\creditGeogebra}[1]{
    \blfootnote{Source : D'après une appliquette Geogebra de #1}
}

% Cahiers Iparcours
% #1 --> Année
% #2 --> Niveau
\newcommand{\creditCahiersIparcours}[2]{
    \blfootnote{Source : Fiches tirées du cahier iParcours #1 de #2}
}

% Manuels Iparcours
% #1 --> Niveau
\newcommand{\creditManuelsIparcours}[1]{
    \blfootnote{Source : Exercices tirés du manuel iParcours de #1}
}

% Cahiers Sesamath
% #1 --> Année
% #1 --> Niveau
\newcommand{\creditCahiersSesamath}[2]{
    \blfootnote{Source : Fiches tirées du cahier Sésamath #1 de #2}
}

% Manuels Sesamath
% #1 --> Niveau
\newcommand{\creditManuelsSesamath}[1]{
    \blfootnote{Source : Exercices tirés du manuel Sésamath de #1}
}

% Libre
% #1 --> texte
\newcommand{\creditLibre}[1]{
    \blfootnote{Source : #1}
}

% Macro pour entourer les opérandes
% #1 --> 
% #2 --> 
% #3 --> 
\newcommand{\OPoval}[3]{\dimen1=#2\opcolumnwidth \ovalnode{#1}{\kern\dimen1 #3\kern\dimen1}}

%==========================================
% Instruments de géométrie
%==========================================
% Equerre - commande de Nathalie DAVAL
% #1 --> 
% #2 --> 
% #3 --> 
% #4 --> 
\newcommand{\equerre}[4]
   {\scalebox{#4}
      {\rput{#3}(#1,#2)      
         {\pspolygon[linecolor=B1](0,0)(1,0)(0,1.8)
          \pspolygon[linecolor=B1](0.2,0.2)(0.65,0.2)(0.2,1)
          \multido{\r=0+0.1}{17}{\psline[linecolor=red,linewidth=0.01](0,\r)(0.075,\r)}
          }
       }      
   }
% COMPAS - commande de Nathalie DAVAL
% #1 --> 
% #2 --> 
% #3 --> 
% #4 --> 
% #5 --> 
\newcommand{\compas}[5]%
{\psset{fillstyle=solid,fillcolor=gray!20}
\scalebox{#4}
    {\rput{#3}(#1,#2)      
        {\psframe(-0.1,3.6)(0.1,4)
        \rput{-#5}(-0.1,3){\psframe(-0.1,-2.4)(0.1,0.1)
                                    \psline[fillcolor=gray!60](-0.05,-2.4)(0,-3)(0.05,-2.4)}
        \rput{#5}(0.1,3){\psframe(-0.1,-2.4)(0.1,0.1)
                                \pspolygon[fillcolor=gray](-0.05,-2.4)(-0.05,-2.95)(0.05,-2.7)(0.05,-2.4)}
        \psframe[fillcolor=gray!40](-0.25,3)(0.25,3.6)
        \pscircle(0,3.3){0.1}
        }
    }      
}

% Piece de curvica - commande de Nathalie DAVAL
% #1 --> 
\def\curvica#1{
\begin{pspicture}(-1,-0.25)(3,2.5)
   #1
   \psset{linewidth=0.4mm,linestyle=dotted} %Curvica vierge
   \psframe(0,0)(2,2)
   \psarc(1,4){2.24}{-116.6}{-63.4}
   \psarc(1,0){2.24}{63.4}{116.6}
   \psarc(1,2){2.24}{-116.6}{-63.4}
   \psarc(1,-2){2.24}{63.4}{116.6}
   \psarc(4,1){2.24}{153.4}{-153.4}
   \psarc(0,1){2.24}{-26.6}{26.6}
   \psarc(2,1){2.24}{153.4}{-153.4}
   \psarc(-2,1){2.24}{-26.6}{26.6}
\end{pspicture}
}
%==========================================
% TiKz 
%==========================================
% Pavage en L dans un quadrillage 8x8
\newcommand{\pavageL}{%
    \foreach \a in {0,-2,-4}{%
        \draw[shift={(\a,-\a)}] (14,0) -- (14,2) -- (16,2) -- (16,4) -- (12,4) -- (12,0) -- cycle;
        \draw[shift={(\a,-\a)}] (15,2) -- (15,3) -- (13,3) -- (13,1) -- (14,1);
        \draw[shift={(\a,-\a)}] (12,2) -- (13,2);
        \draw[shift={(\a,-\a)}] (14,3) -- (14,4);
        \draw[shift={(\a,-\a)}] (14,0) -- (15,0) -- (15,1) -- (16,1) -- (16,2);
    }
    \foreach \a in {0}{%
        \draw[shift={(\a,-\a)},rotate around={90:(12,0)}] (14,0) -- (14,2) -- (16,2) -- (16,4) -- (12,4) -- (12,0) -- cycle;
        \draw[shift={(\a,-\a)},rotate around={90:(12,0)}] (15,2) -- (15,3) -- (13,3) -- (13,1) -- (14,1);
        \draw[shift={(\a,-\a)},rotate around={90:(12,0)}] (12,2) -- (13,2);
        \draw[shift={(\a,-\a)},rotate around={90:(12,0)}] (14,3) -- (14,4);
        \draw[shift={(\a,-\a)},rotate around={90:(12,0)}] (14,0) -- (15,0) -- (15,1) -- (16,1) -- (16,2);
    }
    \foreach \a in {0}{%
        \draw[shift={(\a,-\a)},rotate around={-90:(16,4)}] (14,0) -- (14,2) -- (16,2) -- (16,4) -- (12,4) -- (12,0) -- cycle;
        \draw[shift={(\a,-\a)},rotate around={-90:(16,4)}] (15,2) -- (15,3) -- (13,3) -- (13,1) -- (14,1);
        \draw[shift={(\a,-\a)},rotate around={-90:(16,4)}] (12,2) -- (13,2);
        \draw[shift={(\a,-\a)},rotate around={-90:(16,4)}] (14,3) -- (14,4);
        \draw[shift={(\a,-\a)},rotate around={-90:(16,4)}] (14,0) -- (15,0) -- (15,1) -- (16,1) -- (16,2);
    }
}

% Image d'une maille définie par des points nommés A,B,C,D,E,F par homothetie de centre O
% #1 -> ratio
% #2 -> couleur
\newcommand{\imageHomothetyHexaGone}[2]{%
    \tkzDefPointBy[homothety=center O ratio #1](A); \tkzGetPoint{Ab};
    \tkzDefPointBy[homothety=center O ratio #1](B); \tkzGetPoint{Bb};
    \tkzDefPointBy[homothety=center O ratio #1](C); \tkzGetPoint{Cb};
    \tkzDefPointBy[homothety=center O ratio #1](D); \tkzGetPoint{Db};
    \tkzDefPointBy[homothety=center O ratio #1](E); \tkzGetPoint{Eb};
    \tkzDefPointBy[homothety=center O ratio #1](F); \tkzGetPoint{Fb};
    \draw[color=#2,fill=#2,fill opacity=0.5] (Ab) -- (Bb) -- (Cb) -- (Db) -- (Eb) -- (Fb) -- cycle;
}

% Pavage en losanges dans un quadrillage mxn
% #1 -> longueur du pavage en abscisse
% #2 -> longueur du pavage en ordonnée
\newcommand{\pavageLosange}[2]{%
  \foreach \i in {1,...,#2}{%
    \begin{scope}[shift={(\i-1,\i*1.73-1.73)}]
        \foreach \j in {1,...,#1}{%
            \draw[shift={(\j*2-2,0)}] (0,0) -- (2, 0) -- (3, 1.73) -- (1, 1.73) -- cycle;
        }
    \end{scope}
  }
}

% Image d'une maille définie par des points nommés A,B,C,D,E,F par homothetie de centre O
% #1 -> ratio
% #2 -> couleur
\newcommand{\imageHomothetyParallelogramme}[2]{%
    \tkzDefPointBy[homothety=center O ratio #1](A); \tkzGetPoint{Ab};
    \tkzDefPointBy[homothety=center O ratio #1](B); \tkzGetPoint{Bb};
    \tkzDefPointBy[homothety=center O ratio #1](C); \tkzGetPoint{Cb};
    \tkzDefPointBy[homothety=center O ratio #1](D); \tkzGetPoint{Db};
    \draw[color=#2,fill=#2,fill opacity=0.5] (Ab) -- (Bb) -- (Cb) -- (Db) -- cycle;
}

% Axe Gradué
% #1 -> abscisse de O
% #2 -> abscisse de M
\newcommand{\axeHomothety}[2]{%
    \draw[help lines, color=black!30] (0,0) grid (18,2);            
    % Axe
    \draw (0,1)--(18,1);
    \foreach \x in {0,...,18} \draw (\x,0.9) -- (\x,1.1);
    % Points
    \coordinate (O) at (#1,1);
    \coordinate (M) at (#2,1);    
    % Marques
    \tkzDrawPoints[shape=cross out, size=5pt](O,M);
    \tkzLabelPoints[above](O,M);    
}

% Image d'un point nommé M par homothetie de centre O
% #1 -> ratio
% #2 -> couleur
\newcommand{\imageHomothetyPoint}[2][black]{%
    \tkzDefPointBy[homothety=center O ratio #2](M); \tkzGetPoint{M1};            
    % Marques
    \tkzDrawPoints[shape=cross out, size=5pt,color=#1](M1);    
    \tkzLabelPoint[above,color=#1](M1){$M_1$};
}

% Quadrillage à mailles carrées mxn
% #1 -> couleur : facultive black!30 par défaut
% #2 -> nombre de carreaux en abscisse
% #3 -> nombre de carreaux en ordonnée
\newcommand{\quadrilageMailleCarree}[3][black!30]{%
    \draw[help lines, color=#1] (0,0) grid (#2,#3);
}

% Papier millimétré
% #1 -> couleur : facultative, brown par défaut
\newcommand{\papierMillimetre}[1][brown]{
    \begin{pgfonlayer}{background}
        \draw[step=1mm,ultra thin,#1!30] (current bounding box.south west) grid (current bounding box.north east);
        \draw[step=5mm,very thin,#1!50] (current bounding box.south west) grid (current bounding box.north east);
        \draw[step=1cm,thin,#1!70] (current bounding box.south west) grid (current bounding box.north east);
        \draw[step=5cm,thick,#1!90] (current bounding box.south west) grid (current bounding box.north east);
    \end{pgfonlayer}
}

% Machine maths
% #1 --> nom de la fonction
% #2 --> Texte procédé de calcul sur deux lignes
% #3 --> Texte antécédent sur deux lignes
% #4 --> Texte image sur deux lignes
\newcommand{\machineMaths}[4]{
    \begin{tikzpicture}[line cap=round,line join=round,>=triangle 45,x=1cm,y=1cm]
        % Avant cadre
        \fill [line width=3pt,color=J1] (-6,2.5) -- (-6.5,1.5) -- (-5.5,2.5) -- (-6.5,3.5) -- cycle;
        \draw [line width=3pt,color=J1] (-5,2.9)-- (-5,2.1);
        \draw [line width=3pt,color=J1] (-4,2.5)-- (-5,2.5);                
        % Cadre
        \draw [line width=3pt,color=J1] (-4,4)-- (2,4) -- (2,1) -- (-4,1) -- cycle;
        % Après cadre
        \draw [->,line width=3pt,color=J1] (2,2.5) -- (3,2.5);
        \fill [line width=3pt,color=J1] (3.5,2.5) -- (3,1.5) -- (4,2.5) -- (3,3.5) -- cycle;
        % Textes
        \node[text width=3cm,text centered, scale=1.8] at (-1,3.5){\textbf{machine #1}};
        \node[text width=3cm,text centered, scale=1.5] at (-1,2.8){\textbf{---}};
        \node[text width=3cm,text centered, scale=1.5] at (-1,2){#2};
        \node[text width=3cm,text centered, scale=1.5] at (-8,2.5) {#3};
        \node[text width=3cm,text centered, scale=1.5] at (6,2.5) {#4};
    \end{tikzpicture}
}

% Figure avec carrés périphériques théorème de Pythagore

\newcommand{\figCarresPythagore}[7]{%
    % #1 scale
    % #2 PetiteLongueur
    % #3 MoyenneLongueur
    % #4 GrandeLongueur
    % #5 PetiteAire
    % #6 MoyenneAire
    % #7 GrandeAire
    \begin{tikzpicture}[scale=#1]
        % On crée le triangle
        \coordinate (A) at (1,1);
        \coordinate (B) at (6,1);
        \tkzInterCC[R](A,4)(B,3);
        \tkzGetFirstPoint{C};
        \draw[fill=gray!30] (A)--(B)--(C)--cycle;
        \tkzMarkRightAngles[size=0.3](A,C,B);
        % On construit les carrés sur les côtés
        \tkzDefSquare(B,A)\tkzGetPoints{E}{F};
        \tkzDrawPolygon(B,A,E,F);
        \tkzDefSquare(C,B)\tkzGetPoints{G}{H};
        \tkzDrawPolygon(C,B,G,H);
        \tkzDefSquare(A,C)\tkzGetPoints{I}{J};
        \tkzDrawPolygon(A,C,I,J);
        % On ajoute les labels de longueurs
        \tkzLabelSegment[auto,sloped,below](A,B){#4};
        \tkzLabelSegment[auto,sloped](B,C){#2};
        \tkzLabelSegment[auto,sloped](C,A){#3};
        % On ajoute les aires
        \tkzLabelSegment[yshift=3mm](A,F){#7};
        \tkzLabelSegment[yshift=5mm](B,H){#5};
        \tkzLabelSegment[xshift=-3mm,yshift=3mm](C,J){#6};
    \end{tikzpicture}
}