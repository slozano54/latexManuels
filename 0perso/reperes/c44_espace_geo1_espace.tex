{\tiny
\renewcommand{\arraystretch}{1.5}
\begin{tabular}{|p{0.31\linewidth}|p{0.31\linewidth}|p{0.31\linewidth}|}
\hline
\multicolumn{3}{|>{\centering\arraybackslash}p{\linewidth}|}{\cellcolor{lightgray}{\textsc{\textbf{Espace et géométrie}}}}
\\\hline 
\multicolumn{3}{|>{\centering\arraybackslash}p{\linewidth}|}{
\textbf{Représenter l'espace}
}
\\\hline 
\multicolumn{1}{|>{\centering\arraybackslash}p{0.31\linewidth}|}{$\mathbf{5^{\grave{e}me}}$}
&
\multicolumn{1}{|>{\centering\arraybackslash}p{0.31\linewidth}|}{$\mathbf{4^{\grave{e}me}}$}
&
\multicolumn{1}{|>{\centering\arraybackslash}p{0.31\linewidth}|}{$\mathbf{3^{\grave{e}me}}$}
\\\hline
Le repérage se fait sur une droite graduée ou dans le
plan muni d’un repère orthogonal.\par\vspace{0.25cm}
Dans la continuité de ce qui a été travaillé au cycle 3,
la reconnaissance de solides (pavé droit, cube,
cylindre, pyramide, cône, boule) s’effectue à partir
d’un objet réel, d’une image, d’une représentation en
perspective cavalière ou sur un logiciel de géométrie
dynamique.\par\vspace{0.25cm}
Les élèves construisent et mettent en relation une
représentation en perspective cavalière et un patron
d’un pavé droit ou d’un cylindre.
&
Le repérage se fait dans un pavé droit (abscisse,
ordonnée, altitude). Les élèves produisent et
mettent en relation une représentation en
perspective cavalière et un patron d’une pyramide
ou d’un cône.
&
Le repérage s’étend à la sphère (latitude, longitude).
Un logiciel de géométrie est utilisé pour visualiser
des solides et leurs sections planes. Les élèves
produisent et mettent en relation différentes
représentations des solides étudiés (patrons,
représentation en perspective cavalière, vues de
face, de dessus, en coupe).
\\\hline
\end{tabular}
\renewcommand{\arraystretch}{1}
}