%{\tiny
%\renewcommand{\arraystretch}{1.5}
%\begin{tabular}{|p{2.5cm}|p{6cm}|p{4cm}|p{4cm}|}
%\cline{2-4}
%\multicolumn{1}{p{2.5cm}|}{}&\multicolumn{1}{p{6cm}}{}&\multicolumn{1}{>{\centering\arraybackslash}p{4cm}}%{\textbf{Cycle 4 - Calcul littéral}}&\multicolumn{1}{p{4cm}|}{}\\\hline 
%Thème&$5^{\grave{e}me}$&$4^{\grave{e}me}$&$3^{\grave{e}me}$\\\hline
%\textbf{Expressions} \par \textbf{littérales}
%&
{\tiny
\renewcommand{\arraystretch}{1.5}
\begin{tabular}{|p{0.31\linewidth}|p{0.31\linewidth}|p{0.31\linewidth}|}
\hline
\multicolumn{3}{|>{\centering\arraybackslash}p{\linewidth}|}{\cellcolor{lightgray}{\textsc{\textbf{Nombres et calculs}}}}\\\hline 
\multicolumn{3}{|>{\centering\arraybackslash}p{\linewidth}|}{
\textbf{Calcul littéral}
}
\\\hline 
\multicolumn{3}{|p{\linewidth}|}{
\textbf{Expressions littérales}
}\\\hline 
\multicolumn{1}{|>{\centering\arraybackslash}p{0.31\linewidth}|}{$\mathbf{5^{\grave{e}me}}$}
&
\multicolumn{1}{|>{\centering\arraybackslash}p{0.31\linewidth}|}{$\mathbf{4^{\grave{e}me}}$}
&
\multicolumn{1}{|>{\centering\arraybackslash}p{0.31\linewidth}|}{$\mathbf{3^{\grave{e}me}}$}\\\hline
Les expressions littérales sont introduites à travers
des formules mettant en jeu des grandeurs ou
traduisant des programmes de calcul. L’usage de la
lettre permet d’exprimer un résultat général (par
exemple qu’un entier naturel est pair ou impair) ou
de démontrer une propriété générale (par exemple
que la somme de trois entiers consécutifs est un
multiple de 3). Les notations du calcul littéral (par
exemple $2a$ pour $a \times 2$ ou $2 \times a$, $ab$ pour $a \times b$) sont
progressivement utilisées, en lien avec les
propriétés de la multiplication.\par
Les élèves substituent une valeur numérique à une
lettre pour calculer la valeur d’une expression
littérale.
&
Le travail sur les formules est poursuivi,
parallèlement à la présentation de la notion
d’identité (égalité vraie pour toute valeur des
indéterminées).\par 
La notion de solution d’une équation est formalisée.
&
Le travail sur les expressions littérales est
consolidé avec des transformations d’expressions,
des programmes de calcul, des mises en équations,
des fonctions...
\\\hline
\end{tabular}
\renewcommand{\arraystretch}{1}
}