{\tiny
\renewcommand{\arraystretch}{1.5}
\begin{tabular}{|p{0.31\linewidth}|p{0.31\linewidth}|p{0.31\linewidth}|}
\hline
\multicolumn{3}{|>{\centering\arraybackslash}p{\linewidth}|}{\cellcolor{lightgray}{\textsc{\textbf{Grandeurs et mesures}}}}
\\\hline 
\multicolumn{3}{|>{\centering\arraybackslash}p{\linewidth}|}{
\textbf{Calculs sur des grandeurs mesurables}
}
\\\hline 
\multicolumn{1}{|>{\centering\arraybackslash}p{0.31\linewidth}|}{$\mathbf{5^{\grave{e}me}}$}
&
\multicolumn{1}{|>{\centering\arraybackslash}p{0.31\linewidth}|}{$\mathbf{4^{\grave{e}me}}$}
&
\multicolumn{1}{|>{\centering\arraybackslash}p{0.31\linewidth}|}{$\mathbf{3^{\grave{e}me}}$}
\\\hline
La connaissance des formules donnant les aires du
rectangle, du triangle et du disque, ainsi que le
volume du pavé droit est entretenue à travers la
résolution de problèmes. Elle est enrichie par celles
de l’aire du parallélogramme, du volume du prisme
et du cylindre. La correspondance entre unités de
volume et de contenance est faite. Les calculs
portent aussi sur des durées et des horaires, en
prenant appui sur des contextes issus d’autres
disciplines ou de la vie quotidienne.\par\vspace{0.25cm}
Les élèves sont sensibilisés au contrôle de la
cohérence des résultats du point de vue des unités.
&
Le lexique des formules s’étend au volume des
pyramides et du cône. Le lien est fait entre le
volume d’une pyramide (respectivement d’un cône)
et celui du prisme droit (respectivement du
cylindre) construit sur sa base et ayant même
hauteur. Des grandeurs produits (par exemple
trafic, énergie) et des grandeurs quotients (par
exemple vitesse, débit, concentration, masse
volumique) sont introduites à travers la résolution
de problèmes. Les conversions d’unités sont
travaillées.\par\vspace{0.25cm}
Les élèves sont sensibilisés au contrôle de la
cohérence des résultats du point de vue des unités
des grandeurs composées.
&
La formule donnant le volume d’une boule est
utilisée.\par\vspace{0.25cm}
Le travail sur les grandeurs mesurables et les
unités est poursuivi.\par\vspace{0.25cm}
Il est possible de réinvestir le calcul avec les
puissances de 10 pour les conversions d’unités.\par\vspace{0.25cm}
Par exemple, à partir de : $1$ $m$ = $10^2$ $cm$, il vient\par\vspace{0.25cm}
$1$ $m^3$ = $(1 m)^3$ = $(10^2 cm)^3$ = $10^6 cm^3$\par\vspace{0.25cm}
ou, à partir de : $1 dm$ = $10^{-1} m$, il vient\par\vspace{0.25cm}
$1$ $dm^3$ = $(10^{-1} m)^3$ = $10^{-3} m^3$.
\\\hline
\end{tabular}
\renewcommand{\arraystretch}{1}
}