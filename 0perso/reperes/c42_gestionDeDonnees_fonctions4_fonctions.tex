{\tiny
\renewcommand{\arraystretch}{1.5}
\begin{tabular}{|p{0.31\linewidth}|p{0.31\linewidth}|p{0.31\linewidth}|}
\hline
\multicolumn{3}{|>{\centering\arraybackslash}p{\linewidth}|}{\cellcolor{lightgray}{\textsc{\textbf{Organisation et gestion de données, fonction}}}}\\\hline 
\multicolumn{3}{|>{\centering\arraybackslash}p{\linewidth}|}{
\textbf{Fonctions}
}\\\hline 
\multicolumn{1}{|>{\centering\arraybackslash}p{0.31\linewidth}|}{$\mathbf{5^{\grave{e}me}}$}
&
\multicolumn{1}{|>{\centering\arraybackslash}p{0.31\linewidth}|}{$\mathbf{4^{\grave{e}me}}$}
&
\multicolumn{1}{|>{\centering\arraybackslash}p{0.31\linewidth}|}{$\mathbf{3^{\grave{e}me}}$}\\\hline
La dépendance de deux grandeurs est traduite par un tableau de valeurs ou une formule.
&
La dépendance de deux grandeurs est traduite par un tableau de valeurs, une formule, un graphique. Les représentations graphiques permettent de déterminer des images et des antécédents, qui sont interprétés en fonction du contexte.\par
\textit{La notation et le vocabulaire fonctionnels ne sont pas formalisés en $4^{\grave{e}me}$.}
&
Les notions de variable, de fonction, d’antécédent, d’image sont formalisées et les notations fonctionnelles sont utilisées. Un travail est mené sur le passage d’un mode de représentation d’une fonction (graphique, symbolique, tableau de valeurs) à un autre. Les fonctions affines et linéaires sont présentées par leurs expressions algébriques et leurs représentations graphiques. Les fonctions sont utilisées pour modéliser des phénomènes continus et résoudre des problèmes.
\\\hline
\end{tabular}
\renewcommand{\arraystretch}{1}
}