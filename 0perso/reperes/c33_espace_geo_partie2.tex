{\tiny
\renewcommand{\arraystretch}{1.5}
\begin{tabular}{|p{0.31\linewidth}|p{0.31\linewidth}|p{0.31\linewidth}|}
\hline
\multicolumn{3}{|>{\centering\arraybackslash}p{\linewidth}|}{\cellcolor{lightgray}{\textsc{\textbf{Espace et Géométrie}}}}\\\hline 
\multicolumn{3}{|>{\centering\arraybackslash}p{\linewidth}|}{\textbf{Le raisonnement}}\\\hline 
\multicolumn{1}{|>{\centering\arraybackslash}p{0.31\linewidth}|}{\textbf{CM1}}
&
\multicolumn{1}{|>{\centering\arraybackslash}p{0.31\linewidth}|}{\textbf{CM2}}
&
\multicolumn{1}{|>{\centering\arraybackslash}p{0.31\linewidth}|}{$\mathbf{6^{\grave{e}me}}$}\\\hline
\multicolumn{3}{|>{\centering\arraybackslash}p{\linewidth}|}{
La dimension perceptive, l’usage des instruments et les propriétés élémentaires des figures sont articulés tout au long du cycle.
}\\\hline 
\multicolumn{2}{|p{11cm}|}{
Le raisonnement peut prendre appui sur différents types de codage :
\begin{mylist}
\item signe ajouté aux traits constituant la figure (signe de l’angle droit, mesure, coloriage...) ;
\item qualité particulière du trait lui-même (couleur, épaisseur, pointillés, trait à main levée...) ;
\item élément de la figure qui traduit une propriété implicite (appartenance ou non appartenance,
égalité...) ;
\item nature du support de la figure (quadrillage, papier à réseau pointé, papier millimétré).
\end{mylist}
}
&\multicolumn{1}{p{0.31\linewidth}|}{
Tout le long de l’année se poursuit le travail entrepris au CM2 visant à faire évoluer la perception qu’ont les élèves des activités géométriques (passer de l’observation et du mesurage au codage et au raisonnement).\par
Les élèves utilisent les propriétés relatives aux droites parallèles ou perpendiculaires pour valider la méthode de construction d’une parallèle à la règle et à l’équerre, et établir des relations de perpendicularité ou de parallélisme entre deux droites.\par 
}\\
\cline{1-2}
Un vocabulaire spécifique est employé dès le début du cycle pour désigner des objets, des relations et des propriétés.
&
On amène progressivement les élèves à dépasser la dimension perceptive et instrumentée des propriétés des figures planes pour tendre vers le raisonnement hypothético-déductif.\par 
Il s'agit de conduire sans formalisme des raisonnements simples utilisant les propriétés des figures usuelles ou de la symétrie axiale.
&
Ils complètent leurs acquis sur les propriétés des côtés des figures par celles sur les diagonales et les angles.\par 
Dès que l’étude de la symétrie est suffisamment avancée, ils utilisent les propriétés de conservation
de longueur, d’angle, d’aire et de parallélisme pour justifier une procédure de la construction de la
figure symétrique ou pour répondre à des problèmes de longueur, d’angle, d’aire ou de
parallélisme sans recours à une vérification instrumentée.
\\ 
\hline
\end{tabular}
\renewcommand{\arraystretch}{1}
}