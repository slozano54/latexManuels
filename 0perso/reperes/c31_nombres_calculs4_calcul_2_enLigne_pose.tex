{\tiny
\renewcommand{\arraystretch}{1.5}
\begin{tabular}{|p{0.31\linewidth}|p{0.31\linewidth}|p{0.31\linewidth}|}
\hline
\multicolumn{3}{|>{\centering\arraybackslash}p{\linewidth}|}{\cellcolor{lightgray}{\textsc{\textbf{Nombres et calculs}}}}\\\hline 
\multicolumn{3}{|>{\centering\arraybackslash}p{\linewidth}|}{
\textbf{Calcul}
}
\\\hline 
\multicolumn{1}{|>{\centering\arraybackslash}p{0.31\linewidth}|}{\textbf{CM1}}
&
\multicolumn{1}{|>{\centering\arraybackslash}p{0.31\linewidth}|}{\textbf{CM2}}
&
\multicolumn{1}{|>{\centering\arraybackslash}p{0.31\linewidth}|}{$\mathbf{6^{\grave{e}me}}$}
\\\hline
\multicolumn{3}{|p{\linewidth}|}{
Calcul en ligne}
\\\hline
\multicolumn{2}{|p{0.6\linewidth}|}{
Les connaissances et compétences mises en œuvre pour le calcul en ligne sont les mêmes que pour le calcul
mental, le support de l’écrit permettant d’alléger la mémoire de travail et ainsi de traiter des calculs portant sur 
un registre numérique étendu.
}
&
Dans des calculs simples, confrontés à des
 problématiques de priorités opératoires, par exemple
en relation avec l’utilisation de calculatrices, les élèves
utilisent des parenthèses.
\\\hline
\multicolumn{3}{|p{\linewidth}|}{
Calcul posé}
\\\hline
Dès la \textbf{période 1}, les élèves renforcent leur maîtrise
des algorithmes appris au cycle 2 (addition,
soustraction et multiplication de deux nombres
entiers).\par\vspace{0.25cm}
En \textbf{période 2}, ils étendent aux nombres décimaux
les algorithmes de l’addition et de la soustraction.\par\vspace{0.25cm}
En \textbf{période 3} ils apprennent l’algorithme de la
division euclidienne de deux nombres entiers.
&
Les élèves apprennent les algorithmes :
\begin{mylist}
\item de la multiplication d’un nombre décimal par un
nombre entier (dès la \textbf{période 1}, en relation avec
le calcul de l’aire du rectangle) ;
\item de la division de deux nombres entiers (quotient
décimal ou non : par exemple, 10 : 4 ou 10 : 3),
dès la \textbf{période 2} ;
\item de la division d’un nombre décimal par un
nombre entier dès la \textbf{période 3}.
\end{mylist}
&
Tout au long de l’année, au travers de situations
variées, les élèves entretiennent leurs acquis de CM
sur les algorithmes opératoires.\par\vspace{0.25cm}
Au plus tard en \textbf{période 3}, ils apprennent l’algorithme
de la multiplication de deux nombres décimaux.
\\\hline
\end{tabular}
\renewcommand{\arraystretch}{1}
}