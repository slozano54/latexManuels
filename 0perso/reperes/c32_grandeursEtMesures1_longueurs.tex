{\tiny
\renewcommand{\arraystretch}{1.5}
\begin{tabular}{|p{0.31\linewidth}|p{0.31\linewidth}|p{0.31\linewidth}|}
\hline
\multicolumn{3}{|>{\centering\arraybackslash}p{\linewidth}|}{\cellcolor{lightgray}{\textsc{\textbf{Grandeurs et mesures}}}}
\\\hline 
\multicolumn{3}{|>{\centering\arraybackslash}p{\linewidth}|}{
\textbf{Les longueurs}
}
\\\hline 
\multicolumn{1}{|>{\centering\arraybackslash}p{0.31\linewidth}|}{\textbf{CM1}}
&
\multicolumn{1}{|>{\centering\arraybackslash}p{0.31\linewidth}|}{\textbf{CM2}}
&
\multicolumn{1}{|>{\centering\arraybackslash}p{0.31\linewidth}|}{$\mathbf{6^{\grave{e}me}}$}
\\\hline
\multicolumn{3}{|p{\linewidth}|}{
L’étude d’une grandeur nécessite des activités ayant pour but de définir la grandeur (comparaison directe ou indirecte, ou recours à la mesure), d’explorer les
unités du système international d’unités correspondant, de faire usage des instruments de mesure de cette grandeur, de calculer des mesures avec ou sans
formule. Toutefois, selon la grandeur ou selon la fréquentation de celle-ci au cours du cycle précédent, les comparaisons directes ou indirectes de grandeurs
(longueur, masse et durée) ne seront pas reprises systématiquement. Tout au long du cycle et en relation avec l’apprentissage des nombres décimaux, les
élèves font le lien entre les unités de numération et les unités de mesure (par exemple : dixième --> dm, dg, dL ; centième --> cm, cg, cL, centimes d’euros).
}
\\\hline 
Les élèves comparent des périmètres sans avoir
recours à la mesure, mesurent des périmètres par
report d’unités et de fractions d’unités ou par report
des longueurs des côtés sur un segment de droite
avec le compas ; ils calculent le périmètre d’un
polygone en ajoutant les longueurs de ses côtés
(avec des entiers et fractions puis avec des
décimaux à deux décimales).
&
Ils établissent les formules du périmètre du carré et
du rectangle. Ils les utilisent tout en continuant à
calculer des périmètres de polygones variés en
ajoutant les longueurs de leurs côtés.
&
Selon l’avancement du thème « nombres et calcul »,
les élèves réinvestissent leurs acquis de CM pour
calculer des périmètres simples ou complexes.\par\vspace{0.25cm}
Ils apprennent la formule de la longueur d’un cercle et
l’utilisent après consolidation du produit d’un entier
par un décimal, dans un premier temps, puis du
produit de deux décimaux.
\\\hline
\end{tabular}
\renewcommand{\arraystretch}{1}
}