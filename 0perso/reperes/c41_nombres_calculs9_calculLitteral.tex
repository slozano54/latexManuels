{\tiny
\renewcommand{\arraystretch}{1.5}
\begin{tabular}{|p{0.31\linewidth}|p{0.31\linewidth}|p{0.31\linewidth}|}
\hline
\multicolumn{3}{|>{\centering\arraybackslash}p{\linewidth}|}{\cellcolor{lightgray}{\textsc{\textbf{Nombres et calculs}}}}
\\\hline 
\multicolumn{3}{|>{\centering\arraybackslash}p{\linewidth}|}{
\textbf{Calcul littéral}
}
\\\hline 
\multicolumn{3}{|p{\linewidth}|}{
\textbf{Expressions littérales}
}\\\hline 
\multicolumn{1}{|>{\centering\arraybackslash}p{0.31\linewidth}|}{$\mathbf{5^{\grave{e}me}}$}
&
\multicolumn{1}{|>{\centering\arraybackslash}p{0.31\linewidth}|}{$\mathbf{4^{\grave{e}me}}$}
&
\multicolumn{1}{|>{\centering\arraybackslash}p{0.31\linewidth}|}{$\mathbf{3^{\grave{e}me}}$}\\\hline
Les expressions littérales sont introduites à travers
des formules mettant en jeu des grandeurs ou
traduisant des programmes de calcul. L’usage de la
lettre permet d’exprimer un résultat général (par
exemple qu’un entier naturel est pair ou impair) ou
de démontrer une propriété générale (par exemple
que la somme de trois entiers consécutifs est un
multiple de 3). Les notations du calcul littéral (par
exemple $2a$ pour $a \times 2$ ou $2 \times a$, $ab$ pour $a \times b$) sont
progressivement utilisées, en lien avec les
propriétés de la multiplication.\par
Les élèves substituent une valeur numérique à une
lettre pour calculer la valeur d’une expression
littérale.
&
Le travail sur les formules est poursuivi,
parallèlement à la présentation de la notion
d’identité (égalité vraie pour toute valeur des
indéterminées).\par 
La notion de solution d’une équation est formalisée.
&
Le travail sur les expressions littérales est
consolidé avec des transformations d’expressions,
des programmes de calcul, des mises en équations,
des fonctions...
\\\hline 
\multicolumn{3}{|p{\linewidth}|}{
\textbf{Distributivité}
}
\\\hline 
\multicolumn{1}{|>{\centering\arraybackslash}p{0.31\linewidth}|}{$\mathbf{5^{\grave{e}me}}$}
&
\multicolumn{1}{|>{\centering\arraybackslash}p{0.31\linewidth}|}{$\mathbf{4^{\grave{e}me}}$}
&
\multicolumn{1}{|>{\centering\arraybackslash}p{0.31\linewidth}|}{$\mathbf{3^{\grave{e}me}}$}
\\\hline
Tôt dans l’année, sans attendre la maîtrise des
opérations sur des nombres relatifs, la propriété de
distributivité simple est utilisée pour réduire une
expression littérale de la forme $ax + bx$, où $a$ et $b$
sont des nombres décimaux.\par
Le lien est fait avec des procédures de calcul
numérique déjà rencontrées au cycle 3 (calculs du
type $12 \times 50$ ; $37 \times 99$ ; $3 \times 23 + 7 \times 23$).
&
La structure d’une expression littérale (somme ou
produit) est étudiée. La propriété de distributivité
simple est formalisée et est utilisée pour
développer un produit, factoriser une somme,
réduire une expression littérale.
&
La double distributivité est abordée.
Le lien est fait avec la simple distributivité. Il est
possible de démontrer l’identité
$(a + b)(c + d) = ac + ad + bc + bd$ en posant
$k = a + b$ et en utilisant la simple distributivité.
\\\hline 
\multicolumn{3}{|p{\linewidth}|}{
\textbf{Équations}
}\\\hline 
\multicolumn{1}{|>{\centering\arraybackslash}p{0.31\linewidth}|}{$\mathbf{5^{\grave{e}me}}$}
&
\multicolumn{1}{|>{\centering\arraybackslash}p{0.31\linewidth}|}{$\mathbf{4^{\grave{e}me}}$}
&
\multicolumn{1}{|>{\centering\arraybackslash}p{0.31\linewidth}|}{$\mathbf{3^{\grave{e}me}}$}\\\hline
Les élèves sont amenés à tester si une égalité où
figure une lettre est vraie lorsqu’on lui attribue une
valeur numérique.\par 
Les élèves testent des égalités par essais erreurs, à
la main ou à l’aide d’une calculatrice ou d’un
tableur, des valeurs numériques dans des
expressions littérales, ce qui constitue une
première approche de la notion de solution d’une
équation, sans formalisation à ce stade.
&
Les notions d’inconnue et de solution d’une
équation sont abordées. Elles permettent d’aborder
la mise en équation d’un problème et la résolution
algébrique d’une équation du premier degré.\par 
\textit{Les équations sont travaillées tout au long de l’année
tableur, des valeurs numériques dans des
par un choix progressif des coefficients de l’équation.}
&
La factorisation d’une expression du type $a^2 - b^2$ permet de résoudre des équations produits se
ramenant au premier degré (notamment des équations du type $x^2 = a$ en lien avec la racine
carrée).
\textit{Aucune virtuosité calculatoire n’est attendue dans
les développements et les factorisations.}
\\\hline
\end{tabular}
\renewcommand{\arraystretch}{1}
}