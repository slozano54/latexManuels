{\tiny
\renewcommand{\arraystretch}{1.5}
\begin{tabular}{|p{0.31\linewidth}|p{0.31\linewidth}|p{0.31\linewidth}|}
\hline
\multicolumn{3}{|>{\centering\arraybackslash}p{\linewidth}|}{\cellcolor{lightgray}{\textsc{\textbf{Organisation et gestion de données, fonction}}}}
\\\hline 
\multicolumn{3}{|>{\centering\arraybackslash}p{\linewidth}|}{
\textbf{Proportionnalité}
}
\\\hline 
\multicolumn{1}{|>{\centering\arraybackslash}p{0.31\linewidth}|}{$\mathbf{5^{\grave{e}me}}$}
&
\multicolumn{1}{|>{\centering\arraybackslash}p{0.31\linewidth}|}{$\mathbf{4^{\grave{e}me}}$}
&
\multicolumn{1}{|>{\centering\arraybackslash}p{0.31\linewidth}|}{$\mathbf{3^{\grave{e}me}}$}
\\\hline
Les élèves sont confrontés à des situations
relevant ou non de la proportionnalité. Des
procédures variées (linéarité, passage par l’unité,
coefficient de proportionnalité), déjà étudiées au
cycle 3, permettent de résoudre des problèmes
relevant de la proportionnalité.
&
Le calcul d’une quatrième proportionnelle est
systématisé et les points de vue se diversifient
avec l’utilisation de représentations graphiques, du
calcul littéral et de problèmes de géométrie
relevant de la proportionnalité (configuration de
Thalès dans le cas des triangles emboîtés,
agrandissement et réduction).
&
Le lien est fait entre taux d’évolution et coefficient
multiplicateur, ainsi qu’entre la proportionnalité et
les fonctions linéaires. Le champ des problèmes de
géométrie relevant de la proportionnalité est élargi
(homothéties, triangles semblables, configurations
de Thalès).
\\\hline
\end{tabular}
\renewcommand{\arraystretch}{1}
}