{\tiny
\renewcommand{\arraystretch}{1.5}
\begin{tabular}{|p{0.31\linewidth}|p{0.31\linewidth}|p{0.31\linewidth}|}
\hline
\multicolumn{3}{|>{\centering\arraybackslash}p{\linewidth}|}{\cellcolor{lightgray}{\textsc{\textbf{Nombres et calculs}}}}
\\\hline 
\multicolumn{3}{|>{\centering\arraybackslash}p{\linewidth}|}{
\textbf{Calcul littéral}
}
\\\hline 
\multicolumn{3}{|p{\linewidth}|}{
\textbf{Distributivité}
}
\\\hline 
\multicolumn{1}{|>{\centering\arraybackslash}p{0.31\linewidth}|}{$\mathbf{5^{\grave{e}me}}$}
&
\multicolumn{1}{|>{\centering\arraybackslash}p{0.31\linewidth}|}{$\mathbf{4^{\grave{e}me}}$}
&
\multicolumn{1}{|>{\centering\arraybackslash}p{0.31\linewidth}|}{$\mathbf{3^{\grave{e}me}}$}
\\\hline
Tôt dans l’année, sans attendre la maîtrise des
opérations sur des nombres relatifs, la propriété de
distributivité simple est utilisée pour réduire une
expression littérale de la forme $ax + bx$, où $a$ et $b$
sont des nombres décimaux.\par
Le lien est fait avec des procédures de calcul
numérique déjà rencontrées au cycle 3 (calculs du
type $12 \times 50$ ; $37 \times 99$ ; $3 \times 23 + 7 \times 23$).
&
La structure d’une expression littérale (somme ou
produit) est étudiée. La propriété de distributivité
simple est formalisée et est utilisée pour
développer un produit, factoriser une somme,
réduire une expression littérale.
&
La double distributivité est abordée.
Le lien est fait avec la simple distributivité. Il est
possible de démontrer l’identité
$(a + b)(c + d) = ac + ad + bc + bd$ en posant
$k = a + b$ et en utilisant la simple distributivité.
\\\hline
\end{tabular}
\renewcommand{\arraystretch}{1}
}