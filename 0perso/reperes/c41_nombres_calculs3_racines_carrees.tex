{\tiny
\renewcommand{\arraystretch}{1.5}
\begin{tabular}{|p{0.31\linewidth}|p{0.31\linewidth}|p{0.31\linewidth}|}
\hline
\multicolumn{3}{|>{\centering\arraybackslash}p{\linewidth}|}{\cellcolor{lightgray}{\textsc{\textbf{Nombres et calculs}}}}\\\hline 
\multicolumn{3}{|>{\centering\arraybackslash}p{\linewidth}|}{
\textbf{Racine carrée}
}\\\hline 
\multicolumn{1}{|>{\centering\arraybackslash}p{0.31\linewidth}|}{$\mathbf{5^{\grave{e}me}}$}
&
\multicolumn{1}{|>{\centering\arraybackslash}p{0.31\linewidth}|}{$\mathbf{4^{\grave{e}me}}$}
&
\multicolumn{1}{|>{\centering\arraybackslash}p{0.31\linewidth}|}{$\mathbf{3^{\grave{e}me}}$}\\\hline
&
La racine carrée est introduite, en lien avec des
situations géométriques (théorème de Pythagore,
agrandissement des aires) et à l’appui de la
connaissance des carrés parfaits de 1 à 144 et de
l’utilisation de la calculatrice.
&
La racine carrée est utilisée dans le cadre de la
résolution de problèmes.\par\vspace{0.25cm} 
\textit{Aucune connaissance n’est attendue sur les
propriétés algébriques des racines carrées.
}
\\\hline
\end{tabular}
\renewcommand{\arraystretch}{1}
}