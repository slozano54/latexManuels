{\tiny
\renewcommand{\arraystretch}{1.5}
\begin{tabular}{|p{0.31\linewidth}|p{0.31\linewidth}|p{0.31\linewidth}|}
\hline
\multicolumn{3}{|>{\centering\arraybackslash}p{\linewidth}|}{\cellcolor{lightgray}{\textsc{\textbf{Grandeurs et mesures}}}}
\\\hline 
\multicolumn{3}{|>{\centering\arraybackslash}p{\linewidth}|}{
\textbf{Proportionnalité}
}
\\\hline 
\multicolumn{1}{|>{\centering\arraybackslash}p{0.31\linewidth}|}{\textbf{CM1}}
&
\multicolumn{1}{|>{\centering\arraybackslash}p{0.31\linewidth}|}{\textbf{CM2}}
&
\multicolumn{1}{|>{\centering\arraybackslash}p{0.31\linewidth}|}{$\mathbf{6^{\grave{e}me}}$}
\\\hline
Les élèves commencent à identifier et à résoudre
des problèmes de proportionnalité portant sur des 
grandeurs.
&
Des situations très simples impliquant des
échelles et des vitesses constantes peuvent être 
rencontrées.
&
Sur des situations très simples en relation avec l’utilisation
d’un rapporteur, les élèves construisent des représentations
de données sous la forme de diagrammes circulaires ou
semi-circulaires.
\\\hline
\end{tabular}
\renewcommand{\arraystretch}{1}
}