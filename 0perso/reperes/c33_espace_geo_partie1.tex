{\tiny
\renewcommand{\arraystretch}{1.5}
\begin{tabular}{|p{0.31\linewidth}|p{0.31\linewidth}|p{0.31\linewidth}|}
\hline
\multicolumn{3}{|>{\centering\arraybackslash}p{\linewidth}|}{\cellcolor{lightgray}{\textsc{\textbf{Espace et Géométrie}}}}\\
\hline 
\multicolumn{1}{|>{\centering\arraybackslash}p{0.31\linewidth}|}{\textbf{CM1}}
&
\multicolumn{1}{|>{\centering\arraybackslash}p{0.31\linewidth}|}{\textbf{CM2}}
&
\multicolumn{1}{|>{\centering\arraybackslash}p{0.31\linewidth}|}{$\mathbf{6^{\grave{e}me}}$}\\\hline
\multicolumn{3}{|>{\centering\arraybackslash}p{\linewidth}|}{
\textit{Il est possible,lors de la résolution de problèmes, d’aller avec certains élèves ou toute la classe au-delà des repères de progression identifiés pour chaque niveau.}
}\\\hline 
\multicolumn{3}{|>{\centering\arraybackslash}p{\linewidth}|}{\textbf{Les apprentissages spatiaux}}\\\hline 
\multicolumn{3}{|>{\centering\arraybackslash}p{\linewidth}|}{
Dans la continuité du cycle 2 et tout au long du cycle, les apprentissages spatiaux, en une, deux ou trois dimensions, se réalisent à partir de problèmes de
repérage de déplacement d’objets, d’élaboration de représentation dans des espaces réels, matérialisés (plans, cartes...) ou numériques.
}\\\hline 
\multicolumn{3}{|>{\centering\arraybackslash}p{\linewidth}|}{\textbf{Initiation à la programmation}}\\\hline 
\multicolumn{2}{|p{11cm}|}{
Au CM1 puis au CM2, les élèves apprennent à programmer le déplacement d’un personnage sur un écran.\par
Ils commencent par compléter de tels programmes, puis ils apprennent à corriger un programme erroné. Enfin, ils créent eux-mêmes des programmes permettant d’obtenir des déplacements d’objets ou de personnages.\par 
Les instructions correspondent à des déplacements absolus (liés à l’environnement: \og aller vers l’ouest \fg , \og aller vers la fenêtre \fg ) ou relatifs (liés au personnage: \og tourner d’un quart de tour à gauche \fg).
}
&\multicolumn{1}{p{0.31\linewidth}|}{
La construction de figures géométriques de simples à plus complexes, permet d’amener les élèves vers la répétition d’instructions.\par
Ils peuvent commencer à programmer, seuls ou en équipe, des saynètes impliquant un ou plusieurs personnages interagissant ou se déplaçant simultanément ou successivement.
}\\\hline 
\multicolumn{3}{|>{\centering\arraybackslash}p{\linewidth}|}{\textbf{Les apprentissages géométriques}}\\\hline 
Les élèves tracent avec l’équerre la droite perpendiculaire à une droite donnée en un point donné de cette droite.\par
Ils tracent un carré ou un rectangle de dimensions données.\par
Ils tracent un cercle de centre et de rayon donnés, un triangle rectangle de dimensions données.\par 
Ils apprennent à reconnaître et à nommer une boule, un cylindre, un cône, un cube, un pavé droit, un prisme droit,une pyramide.\par 
Ils apprennent à construire un patron d’un cube de dimension donnée.
&
Les élèves apprennent à reconnaître et nommer un triangle isocèle, un triangle équilatéral, un losange,ainsi qu’à les décrire à partir des propriétés de leurs côtés.\par 
Ils tracent avec l’équerre la droite perpendiculaire à une droite donnée passant par un point donné qui peut être extérieur à la droite.\par 
Ils tracent la droite parallèle à une droite donnée passant par un point donné.\par 
Ils apprennent à construire, pour un cube de dimension donnée, des patrons différents.\par 
Ils apprennent à reconnaître, parmi un ensemble de patrons et de faux patrons donnés, ceux qui correspondent à un solide donné: cube, pavé droit, pyramide.
&
Les élèves sont confrontés à la nécessité de représenter une figure à main levée avant d’en faire un tracé instrumenté. C’est l’occasion d’instaurer le codage de la figure à main levée (au fur et à mesure, égalités de longueurs, perpendicularité, égalité d’angles).\par 
Les figures étudiées sont de plus en plus complexes et les élèves les construisent à partir d’un programme de construction. Ils utilisent selon les cas les figures à main levée, les constructions aux instruments et l’utilisation d’un logiciel de géométrie dynamique.\par 
Ils définissent et différencient le cercle et le disque.Ils réalisent des patrons de pavés droits. Ils travaillent sur des assemblages de solides simples.
\\\hline
\end{tabular}
\renewcommand{\arraystretch}{1}
}