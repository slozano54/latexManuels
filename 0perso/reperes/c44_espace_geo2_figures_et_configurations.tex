{\tiny
\renewcommand{\arraystretch}{1.5}
\begin{tabular}{|p{0.31\linewidth}|p{0.31\linewidth}|p{0.31\linewidth}|}
\hline
\multicolumn{3}{|>{\centering\arraybackslash}p{\linewidth}|}{\cellcolor{lightgray}{\textsc{\textbf{Espace et géométrie}}}}\\\hline 
\multicolumn{3}{|>{\centering\arraybackslash}p{\linewidth}|}{
\textbf{Géométrie Plane}
}\\\hline 
\multicolumn{3}{|p{\linewidth}|}{
\textbf{Figures et configurations}
}\\\hline
\multicolumn{1}{|>{\centering\arraybackslash}p{0.31\linewidth}|}{$\mathbf{5^{\grave{e}me}}$}
&
\multicolumn{1}{|>{\centering\arraybackslash}p{0.31\linewidth}|}{$\mathbf{4^{\grave{e}me}}$}
&
\multicolumn{1}{|>{\centering\arraybackslash}p{0.31\linewidth}|}{$\mathbf{3^{\grave{e}me}}$}\\\hline 
La caractérisation angulaire du parallélisme (angles
alternes-internes et angles correspondants) est
énoncée. La valeur de la somme des angles d’un
triangle peut être démontrée et est utilisée. L’inégalité
triangulaire est énoncée. Le lien est fait entre
l’inégalité triangulaire et la construction d’un triangle à
partir de la donnée de trois longueurs. Des
constructions de triangles à partir de la mesure d’une
longueur et de deux angles ou d’un angle et de deux
longueurs sont proposées.\par 
Le parallélogramme est défini à partir de l’une de ses
propriétés : parallélisme des couples de côtés
opposés ou intersection des diagonales. L’autre
propriété est démontrée et devient une propriété
caractéristique. Il est alors montré que les côtés
opposés d’un parallélogramme sont deux à deux de
même longueur grâce aux propriétés de la symétrie.\par
Les propriétés relatives aux côtés et aux diagonales
d’un parallélogramme sont mises en œuvre pour
effectuer des constructions et mener des
raisonnements.\par
Les élèves consolident le travail sur les codages de
figures : interprétation d’une figure codée ou
réalisation d’un codage.\par
Les élèves découvrent de nouvelles droites
remarquables du triangle : les hauteurs. Ils
poursuivent le travail engagé au cycle 3 sur la
médiatrice dans le cadre de résolution de problèmes.
Ils peuvent par exemple être amenés à démontrer que
les médiatrices d’un triangle sont concourantes.
&
Les cas d’égalité des triangles sont présentés et
utilisés pour résoudre des problèmes. Le lien est
fait avec la construction d’un triangle de mesures
données (trois longueurs, une longueur et deux
angles, deux longueurs et un angle). Le théorème
de Thalès et sa réciproque dans la configuration
des triangles emboîtés sont énoncés et utilisés,
ainsi que le théorème de Pythagore (plusieurs
démonstrations possibles) et sa réciproque. La
définition du cosinus d’un angle d’un triangle
rectangle découle, grâce au théorème de Thalès,
de l’indépendance du rapport des longueurs le
définissant.\par 
\textit{Une progressivité dans l’apprentissage de la
recherche de preuve est aménagée, de manière à
encourager les élèves dans l’exercice de la
démonstration. Aucun formalisme excessif n’est
exigé dans la rédaction.
}
&
Une définition et une caractérisation des triangles
semblables sont données. Le théorème de Thalès
et sa réciproque dans la configuration du papillon
sont énoncés et utilisés (démonstration possible,
utilisant une symétrie centrale pour se ramener à la
configuration étudiée en quatrième). Les lignes
trigonométriques (cosinus, sinus, tangente) dans le
triangle rectangle sont utilisées pour calculer des
longueurs ou des angles.\par
Deux triangles semblables peuvent être définis par
la proportionnalité des mesures de leurs côtés. Une
caractérisation angulaire de cette définition peut
être donnée et démontrée à partir d’un cas d’égalité
des triangles et d’une caractérisation angulaire du
parallélisme.
\\\hline 
\end{tabular}
\renewcommand{\arraystretch}{1}
}