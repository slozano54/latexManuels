{\tiny
\renewcommand{\arraystretch}{1.5}
\begin{tabular}{|p{0.31\linewidth}|p{0.31\linewidth}|p{0.31\linewidth}|}
\hline
\multicolumn{3}{|>{\centering\arraybackslash}p{\linewidth}|}{\cellcolor{lightgray}{\textsc{\textbf{Espace et Géométrie}}}}\\\hline 
\multicolumn{3}{|>{\centering\arraybackslash}p{\linewidth}|}{
\textbf{Le vocabulaire et les notations}
}
\\\hline 
\multicolumn{1}{|>{\centering\arraybackslash}p{0.31\linewidth}|}{\textbf{CM1}}
&
\multicolumn{1}{|>{\centering\arraybackslash}p{0.31\linewidth}|}{\textbf{CM2}}
&
\multicolumn{1}{|>{\centering\arraybackslash}p{0.31\linewidth}|}{$\mathbf{6^{\grave{e}me}}$}
\\\hline
\multicolumn{3}{|>{\centering\arraybackslash}p{\linewidth}|}{
Tout au long du cycle, les notations $(AB)$, $[AB)$, $[AB]$, $AB$, sont toujours précédées du nom de l’objet qu’elles désignent: droite $(AB)$, demi-droite $[AB)$, segment $[AB]$, longueur $AB$. Les élèves apprennent à utiliser le symbole d’appartenance $(\in)$ d’un point à une droite, une demi-droite ou un segment.\par
Le vocabulaire et les notations nouvelles ($\in$, $[AB]$, $(AB)$, $[AB)$, $AB$, $\widehat{AOB}$)  sont introduits au fur et à mesure de leur utilité, et non au départ d’un apprentissage.
}\\\hline 
Le vocabulaire utilisé est le même qu’en fin de cycle 2: côté, sommet, angle, angle droit, face, arête, milieu, droite, segment.\par 
Les élèves commencent à rencontrer la notation \og segment $[AB]$ \fg  pour désigner le segment d’extrémités $A$ et $B$ mais cette notation n’est pas exigible; pour les droites, on parle de la droite \og qui passe par les points $A$ et $B$ \fg , ou de \og la droite $(d)$ \fg .
&
Les élèves commencent à rencontrer la notation \og   droite $(AB)$ \fg , et nomment les angles par leur sommet: par exemple,\og   l’angle $\widehat{A}$ \fg 
&
Les élèves utilisent la notation $AB$ pour désigner la longueur d’un segment qu’ils différencient de la notation du segment $[AB]$.\par
Dès que l’on utilise les objets concernés, les élèves utilisent aussi la notation \og   angle $\widehat{ABC}$ \fg , ainsi que la notation courante pour les demi-droites.\par
Les élèves apprennent à rédiger un programme de construction en utilisant le vocabulaire et les notations appropriés pour des figures simples au départ puis pour des figures plus complexes au fil des périodes suivantes.
\\\hline
\multicolumn{3}{|>{\centering\arraybackslash}p{\linewidth}|}{\textbf{Les instruments}}\\\hline 
Tout au long de l’année, les élèves utilisent la règle graduée ou non graduée ainsi que des bandes de papier à bord droit pour reporter des longueurs.\par 
Ils utilisent l’équerre pour repérer ou construire un angle droit.\par
Ils utilisent aussi d’autres gabarits d’angle ainsi que du papier calque.\par
Ils utilisent le compas pour tracer un cercle, connaissant son centre et un point du cercle ou son centre et la longueur d’un rayon, ou bien pour reporter une longueur.
&
Le travail sur les angles se poursuit, notamment sur des fractions simples de l’angle droit (ex: un \og   demi angle droit \fg , \og   un tiers d’angle droit \fg , \og   l’angle plat comme la somme de deux angles droits \fg ).\par 
Les élèves doivent comprendre que la mesure d’un angle (\og   l’ouverture \fg  formée par les deux demi-droites) ne change pas lorsque l’on prolonge ces demi-droites.
&
Les élèves se servent des instruments (règle, équerre, compas) pour reproduire des figures simples, notamment un triangle de dimensions données. Cette utilisation est souvent combinée à des tracés préalables codés à main levée.\par 
Ils utilisent le rapporteur pour mesurer et construire des angles.\par 
Dès que le cercle a été défini, puis que la propriété caractéristique de la médiatrice d’un segment est connue, les élèves peuvent enrichir leurs procédures de construction à la règle et au compas.
\\\hline
\end{tabular}
\renewcommand{\arraystretch}{1}
}