{\tiny
\renewcommand{\arraystretch}{1.5}
\begin{tabular}{|p{0.31\linewidth}|p{0.31\linewidth}|p{0.31\linewidth}|}
\hline
\multicolumn{3}{|>{\centering\arraybackslash}p{\linewidth}|}{\cellcolor{lightgray}{\textsc{\textbf{Nombres et calculs}}}}
\\\hline 
\multicolumn{3}{|>{\centering\arraybackslash}p{\linewidth}|}{
\textbf{La résolution de problèmes}
}
\\\hline 
\multicolumn{1}{|>{\centering\arraybackslash}p{0.31\linewidth}|}{\textbf{CM1}}
&
\multicolumn{1}{|>{\centering\arraybackslash}p{0.31\linewidth}|}{\textbf{CM2}}
&
\multicolumn{1}{|>{\centering\arraybackslash}p{0.31\linewidth}|}{$\mathbf{6^{\grave{e}me}}$}
\\\hline
%\\\hline 
\multicolumn{3}{|p{\linewidth}|}{
Dès le début du cycle, les problèmes proposés relèvent des quatre opérations.
La progressivité sur la résolution de problèmes combine notamment :
\begin{mylist}
\item les nombres mis en jeu : entiers (tout au long du cycle) puis décimaux dès le CM1 sur des nombres très simples ;
\item le nombre d’étapes que l’élève doit mettre en œuvre pour leur résolution ;
\item les supports proposés pour la prise d’informations : texte, tableau, représentations graphiques.
La communication de la démarche prend différentes formes : langage naturel, schémas, opérations.
\end{mylist}
}
\\\hline 
\multicolumn{3}{|p{\linewidth}|}{
Problèmes relevant de la proportionnalité
}
\\\hline 
Le recours aux propriétés de linéarité (multiplicative
et additive) est privilégié. Ces propriétés doivent
être explicitées ; elles peuvent être
institutionnalisées de façon non formelle à l’aide
d’exemples verbalisés (« Si j’ai deux fois, trois fois...
plus d’invités, il me faudra deux fois, trois fois... plus
d’ingrédients » ; « Je dispose de briques de masses
identiques. Si je connais la masse de 7 briques et
celle de 3 briques alors je peux connaître la masse
de 10 briques en faisant la somme des deux
masses »). Dès la \textbf{période 1}, des situations de
proportionnalité peuvent être proposées
(recettes...). L'institutionnalisation des propriétés
se fait progressivement à partir de la \textbf{période 2}.
&
Dès la \textbf{période 1}, le passage par l’unité vient
enrichir la palette des procédures utilisées lorsque
cela s’avère pertinent.\par\vspace{0.25cm}
À partir de la \textbf{période 3}, le symbole \% est introduit
dans des cas simples, en lien avec les fractions d’une 
quantité (50 \% pour la moitié ; 25 \% pour le quart ;
75 \% pour les trois quarts ; 10 \% pour le dixième).
&
Tout au long de l’année, les procédures déjà
étudiées en CM sont remobilisées et enrichies par
l’utilisation explicite du coefficient de
proportionnalité lorsque cela s’avère pertinent.\par\vspace{0.25cm}
Dès la \textbf{période 2}, en relation avec le travail effectué
en CM, les élèves appliquent un pourcentage
simple (en relation avec les fractions simples de
quantité : 10 \%, 25 \%, 50 \%, 75 \%).\par\vspace{0.25cm}
Dès la \textbf{période 3}, ils apprennent à appliquer un
pourcentage dans des registres variés.
\\\hline
\end{tabular}
\renewcommand{\arraystretch}{1}
}