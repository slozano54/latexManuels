{\tiny
\renewcommand{\arraystretch}{1.5}
\begin{tabular}{|p{0.31\linewidth}|p{0.31\linewidth}|p{0.31\linewidth}|}
\hline
\multicolumn{3}{|>{\centering\arraybackslash}p{\linewidth}|}{\cellcolor{lightgray}{\textsc{\textbf{Algorithmique et programmation}}}}
\\\hline 
\multicolumn{3}{|>{\centering\arraybackslash}p{\linewidth}|}{
\textbf{Écrire, mettre au point, exécuter un programme}
}
\\\hline 
\multicolumn{3}{|>{\centering\arraybackslash}p{\linewidth}|}{
Les repères qui suivent indiquent une progressivité dans le \textbf{niveau de complexité des activités} relevant de ce thème. Certains élèves sont capables de réaliser des activités de troisième niveau dès le début du cycle.
}
\\\hline 

\multicolumn{1}{|>{\centering\arraybackslash}p{0.31\linewidth}|}{$\mathbf{1^{er}\;niveau}$}
&
\multicolumn{1}{|>{\centering\arraybackslash}p{0.31\linewidth}|}{$\mathbf{2^{e}\;niveau}$}
&
\multicolumn{1}{|>{\centering\arraybackslash}p{0.31\linewidth}|}{$\mathbf{3^{e}\;niveau}$}
\\\hline
À un premier niveau, les élèves mettent en ordre
et/ou complètent des blocs Scratch fournis par le
professeur pour construire un programme simple.\par\vspace{0.25cm}
L’utilisation progressive des instructions
conditionnelles et/ou de la boucle \og répéter ... fois \fg)
permet d’écrire des scripts de déplacement, de
construction géométrique ou de programme de
calcul.
&
À un deuxième niveau, les connaissances et les
compétences en algorithmique et en
programmation s’élargissent par :
\begin{mylist}
\item l’écriture d’une séquence d’instructions
(condition \og  si ... alors \fg  et boucle
\og répéter ... fois \fg) ;
\item l’écriture de programmes déclenchés par des
événements extérieurs ;
\item l’intégration d’une variable dans un programme
de déplacement, de construction géométrique,
de calcul ou de simulation d’une expérience
aléatoire.
\end{mylist}
&
À un troisième niveau, l’utilisation simultanée de
boucles \og  répéter ... fois \fg , et \og  répéter jusqu’à ... \fg  et
d’instructions conditionnelles permet de réaliser
des figures, des calculs et des déplacements plus
complexes. L’écriture de plusieurs scripts
fonctionnant en parallèle permet de gérer les
interactions et de créer des jeux.
La décomposition d’un problème en sous-
problèmes et la traduction d’un sous-problème par
la création d’un bloc-utilisateur contribuent au
développement des compétences visées.
\\\hline
\end{tabular}
\renewcommand{\arraystretch}{1}
}