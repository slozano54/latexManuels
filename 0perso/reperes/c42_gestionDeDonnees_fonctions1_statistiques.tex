{\tiny
\renewcommand{\arraystretch}{1.5}
\begin{tabular}{|p{0.31\linewidth}|p{0.31\linewidth}|p{0.31\linewidth}|}
\hline
\multicolumn{3}{|>{\centering\arraybackslash}p{\linewidth}|}{\cellcolor{lightgray}{\textsc{\textbf{Organisation et gestion de données, fonction}}}}
\\\hline 
\multicolumn{3}{|>{\centering\arraybackslash}p{\linewidth}|}{
\textbf{Statistiques}
}
\\\hline 
\multicolumn{1}{|>{\centering\arraybackslash}p{0.31\linewidth}|}{$\mathbf{5^{\grave{e}me}}$}
&
\multicolumn{1}{|>{\centering\arraybackslash}p{0.31\linewidth}|}{$\mathbf{4^{\grave{e}me}}$}
&
\multicolumn{1}{|>{\centering\arraybackslash}p{0.31\linewidth}|}{$\mathbf{3^{\grave{e}me}}$}
\\\hline
Le traitement de données statistiques se prête à
des calculs d’effectifs, de fréquences et de
moyennes. Selon les situations, la représentation
de données statistiques sous forme de tableaux, de
diagrammes ou de graphiques est réalisée à la
main ou à l’aide d’un tableur-grapheur. Les calculs
et les représentations donnent lieu à des
interprétations.
&
Un nouvel indicateur de position est introduit : la
médiane.\par\vspace{0.25cm}
Le travail sur les représentations graphiques, le
calcul, en particulier celui des effectifs et des
fréquences, et l’interprétation des indicateurs de
position est poursuivi.
&
Un indicateur de dispersion est introduit : l’étendue.\par\vspace{0.25cm}
Le travail sur les représentations graphiques, le
calcul, en particulier celui des effectifs et des
fréquences, et l’interprétation des indicateurs de
position est consolidé.\par\vspace{0.25cm}
Un nouveau type de diagramme est introduit : les
histogrammes pour des classes de même
amplitude.
\\\hline
\end{tabular}
\renewcommand{\arraystretch}{1}
}