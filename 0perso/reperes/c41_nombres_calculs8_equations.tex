{\tiny
\renewcommand{\arraystretch}{1.5}
\begin{tabular}{|p{0.31\linewidth}|p{0.31\linewidth}|p{0.31\linewidth}|}
\hline
\multicolumn{3}{|>{\centering\arraybackslash}p{\linewidth}|}{\cellcolor{lightgray}{\textsc{\textbf{Nombres et calculs}}}}
\\\hline 
\multicolumn{3}{|>{\centering\arraybackslash}p{\linewidth}|}{
\textbf{Calcul littéral}
}
\\\hline 
\multicolumn{3}{|p{\linewidth}|}{
\textbf{Équations}
}\\\hline 
\multicolumn{1}{|>{\centering\arraybackslash}p{0.31\linewidth}|}{$\mathbf{5^{\grave{e}me}}$}
&
\multicolumn{1}{|>{\centering\arraybackslash}p{0.31\linewidth}|}{$\mathbf{4^{\grave{e}me}}$}
&
\multicolumn{1}{|>{\centering\arraybackslash}p{0.31\linewidth}|}{$\mathbf{3^{\grave{e}me}}$}\\\hline
Les élèves sont amenés à tester si une égalité où
figure une lettre est vraie lorsqu’on lui attribue une
valeur numérique.\par 
Les élèves testent des égalités par essais erreurs, à
la main ou à l’aide d’une calculatrice ou d’un
tableur, des valeurs numériques dans des
expressions littérales, ce qui constitue une
première approche de la notion de solution d’une
équation, sans formalisation à ce stade.
&
Les notions d’inconnue et de solution d’une
équation sont abordées. Elles permettent d’aborder
la mise en équation d’un problème et la résolution
algébrique d’une équation du premier degré.\par 
\textit{Les équations sont travaillées tout au long de l’année
tableur, des valeurs numériques dans des
par un choix progressif des coefficients de l’équation.}
&
La factorisation d’une expression du type $a^2 - b^2$ permet de résoudre des équations produits se
ramenant au premier degré (notamment des équations du type $x^2 = a$ en lien avec la racine
carrée).
\textit{Aucune virtuosité calculatoire n’est attendue dans
les développements et les factorisations.}
\\\hline
\end{tabular}
\renewcommand{\arraystretch}{1}
}