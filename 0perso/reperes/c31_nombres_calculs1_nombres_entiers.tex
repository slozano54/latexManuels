{\tiny
\renewcommand{\arraystretch}{1.5}
\begin{tabular}{|p{0.31\linewidth}|p{0.31\linewidth}|p{0.31\linewidth}|}
\hline
\multicolumn{3}{|>{\centering\arraybackslash}p{\linewidth}|}{\cellcolor{lightgray}{\textsc{\textbf{Nombres et calculs}}}}
\\\hline 
\multicolumn{3}{|>{\centering\arraybackslash}p{\linewidth}|}{
\textbf{Les nombres entiers}
}
\\\hline 
\multicolumn{1}{|>{\centering\arraybackslash}p{0.31\linewidth}|}{\textbf{CM1}}
&
\multicolumn{1}{|>{\centering\arraybackslash}p{0.31\linewidth}|}{\textbf{CM2}}
&
\multicolumn{1}{|>{\centering\arraybackslash}p{0.31\linewidth}|}{$\mathbf{6^{\grave{e}me}}$}
\\\hline
Les élèves apprennent à utiliser et à représenter les Le répertoire est étendu jusqu’au milliard.
grands nombres entiers jusqu’au million. Il s'agit
d'abord de consolider les connaissances (écritures,
représentations...).
&
Le répertoire est étendu jusqu’au milliard.
&
En \textbf{période 1}, dans un premier temps, les principes de
la numération décimale de position sur les entiers
sont repris jusqu’au million, puis au milliard comme
en CM, et mobilisés sur les situations les plus variées
possibles, notamment en relation avec d’autres
disciplines.
\\\hline
\multicolumn{3}{|>{\centering\arraybackslash}p{\linewidth}|}{
La valeur positionnelle des chiffres doit constamment être mise en lien avec des activités de groupements et d’échanges.
}
\\\hline 
\end{tabular}
\renewcommand{\arraystretch}{1}
}