{\tiny
\renewcommand{\arraystretch}{1.5}
\begin{tabular}{|p{0.31\linewidth}|p{0.31\linewidth}|p{0.31\linewidth}|}
\hline
\multicolumn{3}{|>{\centering\arraybackslash}p{\linewidth}|}{\cellcolor{lightgray}{\textsc{\textbf{Grandeurs et mesures}}}}
\\\hline 
\multicolumn{3}{|>{\centering\arraybackslash}p{\linewidth}|}{
\textbf{Les contenances et les volumes}
}
\\\hline 
\multicolumn{1}{|>{\centering\arraybackslash}p{0.31\linewidth}|}{\textbf{CM1}}
&
\multicolumn{1}{|>{\centering\arraybackslash}p{0.31\linewidth}|}{\textbf{CM2}}
&
\multicolumn{1}{|>{\centering\arraybackslash}p{0.31\linewidth}|}{$\mathbf{6^{\grave{e}me}}$}
\\\hline
Les élèves comparent des contenances sans les
mesurer, puis en les mesurant. Ils découvrent et
apprennent qu’un litre est la contenance d’un cube
de 10 cm d’arête. Ils font des analogies avec les
autres unités de mesure à l’appui des préfixes.
&
Ils poursuivent ce travail en utilisant de
nouvelles unités de contenance : dL, cL et mL.
&
Ils relient les unités de volume et de contenance
(1 L = 1 $dm^3$ ; 1 000 L = 1 $m^3$ ). Ils utilisent les unités de
volume : $cm^3$ , $dm^3$ , $m^3$ et leurs relations.\par\vspace{0.25cm}
Ils calculent le volume d’un cube ou d’un pavé droit en
utilisant une formule.
\\\hline
\end{tabular}
\renewcommand{\arraystretch}{1}
}