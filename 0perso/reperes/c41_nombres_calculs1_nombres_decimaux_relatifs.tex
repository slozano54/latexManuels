{\tiny
\renewcommand{\arraystretch}{1.5}
\begin{tabular}{|p{0.31\linewidth}|p{0.31\linewidth}|p{0.31\linewidth}|}
\hline
\multicolumn{3}{|>{\centering\arraybackslash}p{\linewidth}|}{\cellcolor{lightgray}{\textsc{\textbf{Nombres et calculs}}}}\\\hline 
\multicolumn{3}{|>{\centering\arraybackslash}p{\linewidth}|}{
\textbf{Nombres décimaux relatifs}
}\\\hline 
\multicolumn{1}{|>{\centering\arraybackslash}p{0.31\linewidth}|}{$\mathbf{5^{\grave{e}me}}$}
&
\multicolumn{1}{|>{\centering\arraybackslash}p{0.31\linewidth}|}{$\mathbf{4^{\grave{e}me}}$}
&
\multicolumn{1}{|>{\centering\arraybackslash}p{0.31\linewidth}|}{$\mathbf{3^{\grave{e}me}}$}\\\hline
Le travail mené au cycle 3 sur l’enchaînement des opérations, les Le produit et le quotient de décimaux relatifs Le travail est consolidé notamment lors des
comparaisons et le repérage sur une droite graduée de nombres sont abordés.
résolutions de problèmes.
décimaux positifs est poursuivi. Les nombres relatifs (d’abord
entiers, puis décimaux) sont construits pour rendre possibles
toutes les soustractions. La notion d’opposé est introduite,
l’addition et la soustraction sont étendues aux nombres
décimaux (positifs ou négatifs). Il est possible de mettre en
évidence que soustraire un nombre revient à additionner son
opposé, en s’appuyant sur des exemples à valeur générique du
type : $3,1 - (-2) = 3,1 + 0 - (-2) = 3,1 + 2 + (-2) - (-2)$, donc
$3,1 - (-2) = 3,1 + 2 + 0 = 3,1 + 2 = 5,1$
&
Le produit et le quotient de décimaux relatifs sont abordés.
&
Le travail est consolidé notamment lors des résolutions de problèmes.
\\\hline
\end{tabular}
\renewcommand{\arraystretch}{1}
}