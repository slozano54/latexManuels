{\tiny
\renewcommand{\arraystretch}{1.5}
\begin{tabular}{|p{0.31\linewidth}|p{0.31\linewidth}|p{0.31\linewidth}|}
\hline
\multicolumn{3}{|>{\centering\arraybackslash}p{\linewidth}|}{\cellcolor{lightgray}{\textsc{\textbf{Espace et Géométrie}}}}\\\hline 
\multicolumn{3}{|>{\centering\arraybackslash}p{\linewidth}|}{\textbf{La proportionnalité}}\\\hline 
&
Les élèves agrandissent ou réduisent une figure dans un rapport simple donné (par exemple $\times \dfrac{1}{2}$, $\times 2$, $\times 3$).
&
Les élèves agrandissent ou réduisent une figure dans un rapport plus complexe qu’au CM2 (par exemple $\dfrac{3}{2}$ ou $\dfrac{3}{4}$); ils reproduisent une figure à une échelle donnée et complètent un agrandissement ou une réduction d’une figure donnée à partir de la connaissance d’une des mesures agrandie ou réduite.
\\\hline
\end{tabular}
\renewcommand{\arraystretch}{1}
}