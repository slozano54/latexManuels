{\tiny
\renewcommand{\arraystretch}{1.5}
\begin{tabular}{|p{0.31\linewidth}|p{0.31\linewidth}|p{0.31\linewidth}|}
\hline
\multicolumn{3}{|>{\centering\arraybackslash}p{\linewidth}|}{\cellcolor{lightgray}{\textsc{\textbf{Grandeurs et mesures}}}}
\\\hline 
\multicolumn{3}{|>{\centering\arraybackslash}p{\linewidth}|}{
\textbf{Les durées}
}
\\\hline 
\multicolumn{1}{|>{\centering\arraybackslash}p{0.31\linewidth}|}{\textbf{CM1}}
&
\multicolumn{1}{|>{\centering\arraybackslash}p{0.31\linewidth}|}{\textbf{CM2}}
&
\multicolumn{1}{|>{\centering\arraybackslash}p{0.31\linewidth}|}{$\mathbf{6^{\grave{e}me}}$}
\\\hline
Tout au long de l’année, les élèves consolident la
lecture de l’heure et l’utilisation des unités de
mesure des durées et de leurs relations ; des
conversions peuvent être nécessaires
(siècle/années ; semaine/jours ; heure/minutes ;
minute/secondes).\par\vspace{0.25cm}
Ils les réinvestissent dans la résolution de
problèmes de deux types : calcul d’une durée
connaissant deux instants et calcul d’un instant
connaissant un instant et une durée.
&
Tout au long de l’année, les élèves poursuivent le
travail d’appropriation des relations entre les unités
de mesure des durées.\par\vspace{0.25cm}
Des conversions nécessitant l’interprétation d’un reste 
peuvent être demandées (transformer des heures en heures en 
jours, avec un reste en heures ou des secondes en
minutes, avec un reste en secondes).
&
Selon les situations, les élèves utilisent leurs acquis
de CM sur les durées.\par\vspace{0.25cm}
Des conversions nécessitant deux étapes de
traitement peuvent être demandées (transformer des
heures en semaines, jours et heures ; transformer des
secondes en heures, minutes et secondes).
\\\hline
\end{tabular}
\renewcommand{\arraystretch}{1}
}