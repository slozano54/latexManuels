{\tiny
\renewcommand{\arraystretch}{1.5}
\begin{tabular}{|p{0.31\linewidth}|p{0.31\linewidth}|p{0.31\linewidth}|}
\hline
\multicolumn{3}{|>{\centering\arraybackslash}p{\linewidth}|}{\cellcolor{lightgray}{\textsc{\textbf{Nombres et calculs}}}}\\\hline 
\multicolumn{3}{|>{\centering\arraybackslash}p{\linewidth}|}{
\textbf{Fraction, nombres rationnels}
}\\\hline 
\multicolumn{1}{|>{\centering\arraybackslash}p{0.31\linewidth}|}{$\mathbf{5^{\grave{e}me}}$}
&
\multicolumn{1}{|>{\centering\arraybackslash}p{0.31\linewidth}|}{$\mathbf{4^{\grave{e}me}}$}
&
\multicolumn{1}{|>{\centering\arraybackslash}p{0.31\linewidth}|}{$\mathbf{3^{\grave{e}me}}$}\\\hline
La conception d’une fraction en tant que nombre, déjà abordée
en sixième, est consolidée. Les élèves sont amenés à
reconnaître et à produire des fractions égales (sans privilégier de
méthode en particulier), à comparer, additionner et soustraire
des fractions dont les dénominateurs sont égaux ou multiples
l’un de l’autre.\par 
Au moins une des propriétés suivantes est démontrée, à partir
de la définition d’un quotient :
\begin{mylist}
\item $\dfrac{ab}{ac}=\dfrac{b}{c}$
\item $a\dfrac{b}{c}=\dfrac{ab}{c}$
\item $\dfrac{a}{c}+\dfrac{b}{c}=\dfrac{a+b}{c}$
\item $\dfrac{a}{c}-\dfrac{b}{c}=\dfrac{a-b}{c}$
\end{mylist}
Il est possible, à ce niveau, de se limiter à des exemples à valeur
générique. Cependant, le professeur veille à spécifier que la
vérification d’une propriété, même sur plusieurs exemples, n’en
constitue pas une démonstration.\par 
$\dfrac{2}{3}=\dfrac{10}{15}$\par
On commence par $\dfrac{2}{3}\times 15$ :\par 
$\dfrac{2}{3}\times 15=\dfrac{2}{3}\times 3\times 5$.\par 
La définition du quotient permet de simplifier par 3, puisque $\dfrac{2}{3}$
est le nombre qui, multiplié par 3, donne 2.\par
Donc $\dfrac{2}{3}\times 15=2\times 5=10$.\par 
Par définition du quotient, il vient donc $\dfrac{2}{3}=\dfrac{10}{15}$, puisque
$\dfrac{2}{3}$ multiplié par 15 donne 10.\par\vspace{0.5cm}
&
Un nombre rationnel est défini comme
quotient d’un entier relatif par un entier relatif
non nul, ce qui renvoie à la notion de
fraction.\par 
Le quotient de deux nombres décimaux peut
ne pas être un nombre décimal.\par 
La notion d’inverse est introduite, les
opérations entre fractions sont étendues à la
multiplication et la division. Les élèves sont
conduits à comparer des nombres
rationnels, à en utiliser différentes
représentations et à passer de l’une à l’autre.\par
Une ou plusieurs démonstrations de calculs
fractionnaires sont présentées. Le recours au
calcul littéral vient compléter pour tout ou
partie des élèves l’utilisation d’exemples
à valeurs génériques.
&
La notion de fraction irréductible est
abordée, en lien avec celles de multiple et de
diviseur qui sont travaillées tout au long du
cycle.
\\\hline
\end{tabular}
\renewcommand{\arraystretch}{1}
}