{\tiny
\renewcommand{\arraystretch}{1.5}
\begin{tabular}{|p{0.31\linewidth}|p{0.31\linewidth}|p{0.31\linewidth}|}
\hline
\multicolumn{3}{|>{\centering\arraybackslash}p{\linewidth}|}{\cellcolor{lightgray}{\textsc{\textbf{Grandeurs et mesures}}}}
\\\hline 
\multicolumn{3}{|>{\centering\arraybackslash}p{\linewidth}|}{
\textbf{Les angles}
}
\\\hline 
\multicolumn{1}{|>{\centering\arraybackslash}p{0.31\linewidth}|}{\textbf{CM1}}
&
\multicolumn{1}{|>{\centering\arraybackslash}p{0.31\linewidth}|}{\textbf{CM2}}
&
\multicolumn{1}{|>{\centering\arraybackslash}p{0.31\linewidth}|}{$\mathbf{6^{\grave{e}me}}$}
\\\hline
\multicolumn{2}{|p{0.6\linewidth}|}{
Dès le CM1, les élèves apprennent à repérer les angles d’une figure plane, puis à comparer ces
angles par superposition (utilisation du papier calque) ou en utilisant un gabarit.\par\vspace{0.25cm}
Ils estiment, puis vérifient en utilisant l’équerre, qu’un angle est droit, aigu ou obtus.
}
&
Avant d’utiliser le rapporteur, les élèves poursuivent le
travail entrepris au CM en attribuant des mesures en
degrés à des multiples ou sous-multiples de l’angle droit
de mesure 90° (par exemple, on pourra considérer que la
diagonale d’un carré partage l’angle droit en deux angles
égaux de 45°).\par\vspace{0.25cm}
Les élèves apprennent à utiliser un rapporteur pour mesurer
un angle en degrés ou construire un angle de mesure donnée
en degrés.
\\\hline
\end{tabular}
\renewcommand{\arraystretch}{1}
}