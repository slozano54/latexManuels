{\tiny
\renewcommand{\arraystretch}{1.5}
\begin{tabular}{|p{0.31\linewidth}|p{0.31\linewidth}|p{0.31\linewidth}|}
\hline
\multicolumn{3}{|>{\centering\arraybackslash}p{\linewidth}|}{\cellcolor{lightgray}{\textsc{\textbf{Nombres et calculs}}}}\\\hline 
\multicolumn{3}{|>{\centering\arraybackslash}p{\linewidth}|}{
\textbf{Puissances}
}\\\hline 
\multicolumn{1}{|>{\centering\arraybackslash}p{0.31\linewidth}|}{$\mathbf{5^{\grave{e}me}}$}
&
\multicolumn{1}{|>{\centering\arraybackslash}p{0.31\linewidth}|}{$\mathbf{4^{\grave{e}me}}$}
&
\multicolumn{1}{|>{\centering\arraybackslash}p{0.31\linewidth}|}{$\mathbf{3^{\grave{e}me}}$}\\\hline
&
Les puissances de 10 sont d’abord introduites avec
des exposants positifs, puis négatifs, afin de définir
les préfixes de nano à giga et la notation
scientifique. Celle-ci est utilisée pour comparer des
nombres et déterminer des ordres de grandeurs, en
lien d’autres disciplines. Les puissances de base
quelconque d’exposants positifs sont introduites
pour simplifier l’écriture de produits.\par\vspace{0.25cm}
\textit{La connaissance des formules générales sur les
produits ou quotients de puissances de 10 n’est pas un
attendu du programme : la mise en œuvre des calculs
sur les puissances découle de leur définition.}
&
Les puissances de base quelconque d’exposants
négatifs sont introduites et utilisées pour simplifier
des quotients.\par \vspace{0.25cm}
\textit{La connaissance des formules générales sur les
produits ou quotients de puissances n’est pas un
attendu du programme : la mise en œuvre des calculs
sur les puissances découle de leur définition.
}
\\\hline
\end{tabular}
\renewcommand{\arraystretch}{1}
}