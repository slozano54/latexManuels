\documentclass[12pt,french,a4paper]{article}
\usepackage{ProfCollege}
%===================================================
% Modifictaion pour la compilation en mode luaLaTeX
%===================================================
% Font
\usepackage[warnings-off={mathtools-colon,mathtools-overbracket}]{unicode-math}
\setmainfont{TeX Gyre Schola}
\setmathfont{TeX Gyre Schola Math}
% \usepackage[math-style=french,Scale=0.98]{fourier-otf}
% \newfontfamily\myfontScratch[Scale=0.725]{DejaVu Sans}


% Pour gérer la géométrie de la page.
\usepackage[a4paper,margin=1cm,noheadfoot]{geometry}
\setlength{\parindent}{0pt}
\pagestyle{empty}

% titre
\title{\textcolor{red}{\sc -Repères annuels de progression-\\-Cycle 3-}}
\author{{\sc S.Lozano}}
\date{19 septembre 2020 - mise à jour \today}

% Pour utiliser les usages français grâce au <french> de l'option de classe.
\usepackage{babel}

%Liste avec petite fléches zarbi 
\newenvironment{mylist}{\begin{list}{$\leadsto$}{}}{\end{list}}

\begin{document}
\maketitle
\thispagestyle{empty}
\abstract{Factorisation des repères annuels de progression pour les chapitres. Synthèse globale.}

\section*{Nombres et calculs}
{\tiny
\renewcommand{\arraystretch}{1.5}
\begin{tabular}{|p{0.31\linewidth}|p{0.31\linewidth}|p{0.31\linewidth}|}
\hline
\multicolumn{3}{|>{\centering\arraybackslash}p{\linewidth}|}{\cellcolor{lightgray}{\textsc{\textbf{Nombres et calculs}}}}
\\\hline 
\multicolumn{3}{|>{\centering\arraybackslash}p{\linewidth}|}{
\textbf{Les nombres entiers}
}
\\\hline 
\multicolumn{1}{|>{\centering\arraybackslash}p{0.31\linewidth}|}{\textbf{CM1}}
&
\multicolumn{1}{|>{\centering\arraybackslash}p{0.31\linewidth}|}{\textbf{CM2}}
&
\multicolumn{1}{|>{\centering\arraybackslash}p{0.31\linewidth}|}{$\mathbf{6^{\grave{e}me}}$}
\\\hline
Les élèves apprennent à utiliser et à représenter les Le répertoire est étendu jusqu’au milliard.
grands nombres entiers jusqu’au million. Il s'agit
d'abord de consolider les connaissances (écritures,
représentations...).
&
Le répertoire est étendu jusqu’au milliard.
&
En \textbf{période 1}, dans un premier temps, les principes de
la numération décimale de position sur les entiers
sont repris jusqu’au million, puis au milliard comme
en CM, et mobilisés sur les situations les plus variées
possibles, notamment en relation avec d’autres
disciplines.
\\\hline
\multicolumn{3}{|>{\centering\arraybackslash}p{\linewidth}|}{
La valeur positionnelle des chiffres doit constamment être mise en lien avec des activités de groupements et d’échanges.
}
\\\hline 
\end{tabular}
\renewcommand{\arraystretch}{1}
}
\par\smallskip
{\tiny
\renewcommand{\arraystretch}{1.5}
\begin{tabular}{|p{0.31\linewidth}|p{0.31\linewidth}|p{0.31\linewidth}|}
\hline
\multicolumn{3}{|>{\centering\arraybackslash}p{\linewidth}|}{\cellcolor{lightgray}{\textsc{\textbf{Nombres et calculs}}}}
\\\hline 
\multicolumn{3}{|>{\centering\arraybackslash}p{\linewidth}|}{
\textbf{Fractions}
}
\\\hline 
\multicolumn{1}{|>{\centering\arraybackslash}p{0.31\linewidth}|}{\textbf{CM1}}
&
\multicolumn{1}{|>{\centering\arraybackslash}p{0.31\linewidth}|}{\textbf{CM2}}
&
\multicolumn{1}{|>{\centering\arraybackslash}p{0.31\linewidth}|}{$\mathbf{6^{\grave{e}me}}$}
\\\hline
Dès la \textbf{période 1} les élèves utilisent d’abord les
fractions simples (comme $\dfrac{2}{3}$,$\dfrac{1}{4}$ ,$\dfrac{5}{2}$ ) dans le cadre
de partage de grandeurs. Ils travaillent des
fractions inférieures et des fractions supérieures à
1.\par\vspace{0.25cm}
Dès la \textbf{période 2}, les fractions décimales sont
régulièrement mobilisées : elles acquièrent le statut
de nombre et sont positionnées sur une droite
graduée. Les élèves comparent des fractions de
même dénominateur. Ils ajoutent des fractions
décimales de même dénominateur. Ils apprennent à
écrire des fractions décimales sous forme de somme
d’un nombre entier et d’une fraction décimale
inférieure à 1.
&
Dès la \textbf{période 1}, dans la continuité du CM1, les
élèves étendent le registre des fractions qu’ils
manipulent (en particulier $\dfrac{1}{1\;000}$) ; ils apprennent à
écrire des fractions sous forme de somme d’un
nombre entier et d’une fraction inférieure à 1.
&
En \textbf{période 1}, sont réactivées les fractions comme
opérateurs de partage vues en CM, puis les
fractions décimales en relation avec les nombres
décimaux (par exemple à partir de mesures de
longueurs) ; les élèves ajoutent des fractions
décimales de même dénominateur.\par\vspace{0.25cm}
En \textbf{période 2} l’addition est étendue à des fractions
de même dénominateur (inférieur ou égal à 5 et en
privilégiant la vocalisation : deux cinquièmes plus
un cinquième égale trois cinquièmes).\par\vspace{0.25cm}
En \textbf{période 3}, les élèves apprennent que $\dfrac{a}{b}$
est le nombre qui, multiplié par b, donne a (définition du
quotient de a par b).
\\\hline
\end{tabular}
\renewcommand{\arraystretch}{1}
}
\par\smallskip
{\tiny
\renewcommand{\arraystretch}{1.5}
\begin{tabular}{|p{0.31\linewidth}|p{0.31\linewidth}|p{0.31\linewidth}|}
\hline
\multicolumn{3}{|>{\centering\arraybackslash}p{\linewidth}|}{\cellcolor{lightgray}{\textsc{\textbf{Nombres et calculs}}}}
\\\hline 
\multicolumn{3}{|>{\centering\arraybackslash}p{\linewidth}|}{
\textbf{Nombres décimaux}
}
\\\hline 
\multicolumn{1}{|>{\centering\arraybackslash}p{0.31\linewidth}|}{\textbf{CM1}}
&
\multicolumn{1}{|>{\centering\arraybackslash}p{0.31\linewidth}|}{\textbf{CM2}}
&
\multicolumn{1}{|>{\centering\arraybackslash}p{0.31\linewidth}|}{$\mathbf{6^{\grave{e}me}}$}
\\\hline 
\multicolumn{3}{|>{\centering\arraybackslash}p{\linewidth}|}{
Tout au long du cycle, les désignations orale et écrite des nombres décimaux basées sur les unités de numération contribuent à l’acquisition du sens des
nombres décimaux (par exemple pour $3,12$ : « trois unités et douze centièmes » ou « trois unités, un dixième et deux centièmes » ou « trois cent douze
centièmes »).
}
\\\hline
À partir de la \textbf{période 2}, les élèves apprennent à
utiliser les nombres décimaux ayant au plus deux
décimales en veillant à mettre en relation fractions
décimales et écritures à virgule\par\vspace{0.25cm}
(ex : $3,12 = 3 +\dfrac{12}{100}$).
Ils connaissent des écritures décimales de fractions simples 
($\dfrac{1}{2} = 0,5 = \dfrac{5}{10}$ ; $\dfrac{1}{4}=\dfrac{25}{100}=0,25$;
la moitié d’un entier sur des petits nombres).
&
Dès la \textbf{période 1}, les élèves rencontrent et utilisent
des nombres décimaux ayant une, deux ou trois
décimales.\par\vspace{0.25cm}
Ils connaissent des écritures décimales de fractions
simples ($\dfrac{1}{5} = 0,2 = \dfrac{2}{10}$ ; $\dfrac{3}{4} = \dfrac{75}{100}=0,75$ ; la moitié d’un entier).
&
Dès la \textbf{période 1}, dans le prolongement des acquis du
CM, on travaille sur les décimaux jusqu’à trois
décimales. La quatrième décimale sera introduite en
\textbf{période 2} au travers des diverses activités.
\\\hline
\end{tabular}
\renewcommand{\arraystretch}{1}
}
\par\smallskip
\input{./c31_nombres_calculs4_calcul}
\par\smallskip
{\tiny
\renewcommand{\arraystretch}{1.5}
\begin{tabular}{|p{0.31\linewidth}|p{0.31\linewidth}|p{0.31\linewidth}|}
\hline
\multicolumn{3}{|>{\centering\arraybackslash}p{\linewidth}|}{\cellcolor{lightgray}{\textsc{\textbf{Nombres et calculs}}}}
\\\hline 
\multicolumn{3}{|>{\centering\arraybackslash}p{\linewidth}|}{
\textbf{La résolution de problèmes}
}
\\\hline 
\multicolumn{1}{|>{\centering\arraybackslash}p{0.31\linewidth}|}{\textbf{CM1}}
&
\multicolumn{1}{|>{\centering\arraybackslash}p{0.31\linewidth}|}{\textbf{CM2}}
&
\multicolumn{1}{|>{\centering\arraybackslash}p{0.31\linewidth}|}{$\mathbf{6^{\grave{e}me}}$}
\\\hline
%\\\hline 
\multicolumn{3}{|p{\linewidth}|}{
Dès le début du cycle, les problèmes proposés relèvent des quatre opérations.
La progressivité sur la résolution de problèmes combine notamment :
\begin{mylist}
\item les nombres mis en jeu : entiers (tout au long du cycle) puis décimaux dès le CM1 sur des nombres très simples ;
\item le nombre d’étapes que l’élève doit mettre en œuvre pour leur résolution ;
\item les supports proposés pour la prise d’informations : texte, tableau, représentations graphiques.
La communication de la démarche prend différentes formes : langage naturel, schémas, opérations.
\end{mylist}
}
\\\hline 
\multicolumn{3}{|p{\linewidth}|}{
Problèmes relevant de la proportionnalité
}
\\\hline 
Le recours aux propriétés de linéarité (multiplicative
et additive) est privilégié. Ces propriétés doivent
être explicitées ; elles peuvent être
institutionnalisées de façon non formelle à l’aide
d’exemples verbalisés (« Si j’ai deux fois, trois fois...
plus d’invités, il me faudra deux fois, trois fois... plus
d’ingrédients » ; « Je dispose de briques de masses
identiques. Si je connais la masse de 7 briques et
celle de 3 briques alors je peux connaître la masse
de 10 briques en faisant la somme des deux
masses »). Dès la \textbf{période 1}, des situations de
proportionnalité peuvent être proposées
(recettes...). L'institutionnalisation des propriétés
se fait progressivement à partir de la \textbf{période 2}.
&
Dès la \textbf{période 1}, le passage par l’unité vient
enrichir la palette des procédures utilisées lorsque
cela s’avère pertinent.\par\vspace{0.25cm}
À partir de la \textbf{période 3}, le symbole \% est introduit
dans des cas simples, en lien avec les fractions d’une 
quantité (50 \% pour la moitié ; 25 \% pour le quart ;
75 \% pour les trois quarts ; 10 \% pour le dixième).
&
Tout au long de l’année, les procédures déjà
étudiées en CM sont remobilisées et enrichies par
l’utilisation explicite du coefficient de
proportionnalité lorsque cela s’avère pertinent.\par\vspace{0.25cm}
Dès la \textbf{période 2}, en relation avec le travail effectué
en CM, les élèves appliquent un pourcentage
simple (en relation avec les fractions simples de
quantité : 10 \%, 25 \%, 50 \%, 75 \%).\par\vspace{0.25cm}
Dès la \textbf{période 3}, ils apprennent à appliquer un
pourcentage dans des registres variés.
\\\hline
\end{tabular}
\renewcommand{\arraystretch}{1}
}

\section*{Grandeurs et mesures}
{\tiny
\renewcommand{\arraystretch}{1.5}
\begin{tabular}{|p{0.31\linewidth}|p{0.31\linewidth}|p{0.31\linewidth}|}
\hline
\multicolumn{3}{|>{\centering\arraybackslash}p{\linewidth}|}{\cellcolor{lightgray}{\textsc{\textbf{Grandeurs et mesures}}}}
\\\hline 
\multicolumn{3}{|>{\centering\arraybackslash}p{\linewidth}|}{
\textbf{Les longueurs}
}
\\\hline 
\multicolumn{1}{|>{\centering\arraybackslash}p{0.31\linewidth}|}{\textbf{CM1}}
&
\multicolumn{1}{|>{\centering\arraybackslash}p{0.31\linewidth}|}{\textbf{CM2}}
&
\multicolumn{1}{|>{\centering\arraybackslash}p{0.31\linewidth}|}{$\mathbf{6^{\grave{e}me}}$}
\\\hline
\multicolumn{3}{|p{\linewidth}|}{
L’étude d’une grandeur nécessite des activités ayant pour but de définir la grandeur (comparaison directe ou indirecte, ou recours à la mesure), d’explorer les
unités du système international d’unités correspondant, de faire usage des instruments de mesure de cette grandeur, de calculer des mesures avec ou sans
formule. Toutefois, selon la grandeur ou selon la fréquentation de celle-ci au cours du cycle précédent, les comparaisons directes ou indirectes de grandeurs
(longueur, masse et durée) ne seront pas reprises systématiquement. Tout au long du cycle et en relation avec l’apprentissage des nombres décimaux, les
élèves font le lien entre les unités de numération et les unités de mesure (par exemple : dixième --> dm, dg, dL ; centième --> cm, cg, cL, centimes d’euros).
}
\\\hline 
Les élèves comparent des périmètres sans avoir
recours à la mesure, mesurent des périmètres par
report d’unités et de fractions d’unités ou par report
des longueurs des côtés sur un segment de droite
avec le compas ; ils calculent le périmètre d’un
polygone en ajoutant les longueurs de ses côtés
(avec des entiers et fractions puis avec des
décimaux à deux décimales).
&
Ils établissent les formules du périmètre du carré et
du rectangle. Ils les utilisent tout en continuant à
calculer des périmètres de polygones variés en
ajoutant les longueurs de leurs côtés.
&
Selon l’avancement du thème « nombres et calcul »,
les élèves réinvestissent leurs acquis de CM pour
calculer des périmètres simples ou complexes.\par\vspace{0.25cm}
Ils apprennent la formule de la longueur d’un cercle et
l’utilisent après consolidation du produit d’un entier
par un décimal, dans un premier temps, puis du
produit de deux décimaux.
\\\hline
\end{tabular}
\renewcommand{\arraystretch}{1}
}
\par\smallskip
{\tiny
\renewcommand{\arraystretch}{1.5}
\begin{tabular}{|p{0.31\linewidth}|p{0.31\linewidth}|p{0.31\linewidth}|}
\hline
\multicolumn{3}{|>{\centering\arraybackslash}p{\linewidth}|}{\cellcolor{lightgray}{\textsc{\textbf{Grandeurs et mesures}}}}
\\\hline 
\multicolumn{3}{|>{\centering\arraybackslash}p{\linewidth}|}{
\textbf{Les durées}
}
\\\hline 
\multicolumn{1}{|>{\centering\arraybackslash}p{0.31\linewidth}|}{\textbf{CM1}}
&
\multicolumn{1}{|>{\centering\arraybackslash}p{0.31\linewidth}|}{\textbf{CM2}}
&
\multicolumn{1}{|>{\centering\arraybackslash}p{0.31\linewidth}|}{$\mathbf{6^{\grave{e}me}}$}
\\\hline
Tout au long de l’année, les élèves consolident la
lecture de l’heure et l’utilisation des unités de
mesure des durées et de leurs relations ; des
conversions peuvent être nécessaires
(siècle/années ; semaine/jours ; heure/minutes ;
minute/secondes).\par\vspace{0.25cm}
Ils les réinvestissent dans la résolution de
problèmes de deux types : calcul d’une durée
connaissant deux instants et calcul d’un instant
connaissant un instant et une durée.
&
Tout au long de l’année, les élèves poursuivent le
travail d’appropriation des relations entre les unités
de mesure des durées.\par\vspace{0.25cm}
Des conversions nécessitant l’interprétation d’un reste 
peuvent être demandées (transformer des heures en heures en 
jours, avec un reste en heures ou des secondes en
minutes, avec un reste en secondes).
&
Selon les situations, les élèves utilisent leurs acquis
de CM sur les durées.\par\vspace{0.25cm}
Des conversions nécessitant deux étapes de
traitement peuvent être demandées (transformer des
heures en semaines, jours et heures ; transformer des
secondes en heures, minutes et secondes).
\\\hline
\end{tabular}
\renewcommand{\arraystretch}{1}
}
\par\smallskip
{\tiny
\renewcommand{\arraystretch}{1.5}
\begin{tabular}{|p{0.31\linewidth}|p{0.31\linewidth}|p{0.31\linewidth}|}
\hline
\multicolumn{3}{|>{\centering\arraybackslash}p{\linewidth}|}{\cellcolor{lightgray}{\textsc{\textbf{Grandeurs et mesures}}}}
\\\hline 
\multicolumn{3}{|>{\centering\arraybackslash}p{\linewidth}|}{
\textbf{Les aires}
}
\\\hline 
\multicolumn{1}{|>{\centering\arraybackslash}p{0.31\linewidth}|}{\textbf{CM1}}
&
\multicolumn{1}{|>{\centering\arraybackslash}p{0.31\linewidth}|}{\textbf{CM2}}
&
\multicolumn{1}{|>{\centering\arraybackslash}p{0.31\linewidth}|}{$\mathbf{6^{\grave{e}me}}$}
\\\hline
Les élèves comparent des surfaces selon leur aire
par estimation visuelle, par superposition ou
découpage et recollement. Ils estiment des aires,
ou les déterminent, en faisant appel à une aire de
référence.
Le lien est fait chaque fois que possible avec le
travail sur les fractions.
&
L’utilisation d’une unité de référence est
systématique. Cette unité peut être une maille
d’un réseau quadrillé adapté, le $cm^2$ , le $dm^2$ ou
le $m^2$.\par\vspace{0.25cm}
Les élèves apprennent à utiliser les formules
d’aire du carré, du rectangle et du triangle
rectangle.
&
En relation avec le travail sur la quatrième décimale, les
élèves utilisent les multiples et sous-multiples du $m^2$ et les
relations qui les lient. Ils utilisent la formule pour calculer
l’aire d’un triangle quelconque lorsque les données sont
exprimées avec des nombres entiers.\par\vspace{0.25cm}
Après avoir consolidé le produit de décimaux, ils utilisent les
formules pour calculer l’aire d’un triangle quelconque et celle
d’un disque.
\\\hline
\end{tabular}
\renewcommand{\arraystretch}{1}
}
\par\smallskip
{\tiny
\renewcommand{\arraystretch}{1.5}
\begin{tabular}{|p{0.31\linewidth}|p{0.31\linewidth}|p{0.31\linewidth}|}
\hline
\multicolumn{3}{|>{\centering\arraybackslash}p{\linewidth}|}{\cellcolor{lightgray}{\textsc{\textbf{Grandeurs et mesures}}}}
\\\hline 
\multicolumn{3}{|>{\centering\arraybackslash}p{\linewidth}|}{
\textbf{Les contenances et les volumes}
}
\\\hline 
\multicolumn{1}{|>{\centering\arraybackslash}p{0.31\linewidth}|}{\textbf{CM1}}
&
\multicolumn{1}{|>{\centering\arraybackslash}p{0.31\linewidth}|}{\textbf{CM2}}
&
\multicolumn{1}{|>{\centering\arraybackslash}p{0.31\linewidth}|}{$\mathbf{6^{\grave{e}me}}$}
\\\hline
Les élèves comparent des contenances sans les
mesurer, puis en les mesurant. Ils découvrent et
apprennent qu’un litre est la contenance d’un cube
de 10 cm d’arête. Ils font des analogies avec les
autres unités de mesure à l’appui des préfixes.
&
Ils poursuivent ce travail en utilisant de
nouvelles unités de contenance : dL, cL et mL.
&
Ils relient les unités de volume et de contenance
(1 L = 1 $dm^3$ ; 1 000 L = 1 $m^3$ ). Ils utilisent les unités de
volume : $cm^3$ , $dm^3$ , $m^3$ et leurs relations.\par\vspace{0.25cm}
Ils calculent le volume d’un cube ou d’un pavé droit en
utilisant une formule.
\\\hline
\end{tabular}
\renewcommand{\arraystretch}{1}
}
\par\smallskip
{\tiny
\renewcommand{\arraystretch}{1.5}
\begin{tabular}{|p{0.31\linewidth}|p{0.31\linewidth}|p{0.31\linewidth}|}
\hline
\multicolumn{3}{|>{\centering\arraybackslash}p{\linewidth}|}{\cellcolor{lightgray}{\textsc{\textbf{Grandeurs et mesures}}}}
\\\hline 
\multicolumn{3}{|>{\centering\arraybackslash}p{\linewidth}|}{
\textbf{Les angles}
}
\\\hline 
\multicolumn{1}{|>{\centering\arraybackslash}p{0.31\linewidth}|}{\textbf{CM1}}
&
\multicolumn{1}{|>{\centering\arraybackslash}p{0.31\linewidth}|}{\textbf{CM2}}
&
\multicolumn{1}{|>{\centering\arraybackslash}p{0.31\linewidth}|}{$\mathbf{6^{\grave{e}me}}$}
\\\hline
\multicolumn{2}{|p{0.6\linewidth}|}{
Dès le CM1, les élèves apprennent à repérer les angles d’une figure plane, puis à comparer ces
angles par superposition (utilisation du papier calque) ou en utilisant un gabarit.\par\vspace{0.25cm}
Ils estiment, puis vérifient en utilisant l’équerre, qu’un angle est droit, aigu ou obtus.
}
&
Avant d’utiliser le rapporteur, les élèves poursuivent le
travail entrepris au CM en attribuant des mesures en
degrés à des multiples ou sous-multiples de l’angle droit
de mesure 90° (par exemple, on pourra considérer que la
diagonale d’un carré partage l’angle droit en deux angles
égaux de 45°).\par\vspace{0.25cm}
Les élèves apprennent à utiliser un rapporteur pour mesurer
un angle en degrés ou construire un angle de mesure donnée
en degrés.
\\\hline
\end{tabular}
\renewcommand{\arraystretch}{1}
}
\par\smallskip
{\tiny
\renewcommand{\arraystretch}{1.5}
\begin{tabular}{|p{0.31\linewidth}|p{0.31\linewidth}|p{0.31\linewidth}|}
\hline
\multicolumn{3}{|>{\centering\arraybackslash}p{\linewidth}|}{\cellcolor{lightgray}{\textsc{\textbf{Grandeurs et mesures}}}}
\\\hline 
\multicolumn{3}{|>{\centering\arraybackslash}p{\linewidth}|}{
\textbf{Proportionnalité}
}
\\\hline 
\multicolumn{1}{|>{\centering\arraybackslash}p{0.31\linewidth}|}{\textbf{CM1}}
&
\multicolumn{1}{|>{\centering\arraybackslash}p{0.31\linewidth}|}{\textbf{CM2}}
&
\multicolumn{1}{|>{\centering\arraybackslash}p{0.31\linewidth}|}{$\mathbf{6^{\grave{e}me}}$}
\\\hline
Les élèves commencent à identifier et à résoudre
des problèmes de proportionnalité portant sur des 
grandeurs.
&
Des situations très simples impliquant des
échelles et des vitesses constantes peuvent être 
rencontrées.
&
Sur des situations très simples en relation avec l’utilisation
d’un rapporteur, les élèves construisent des représentations
de données sous la forme de diagrammes circulaires ou
semi-circulaires.
\\\hline
\end{tabular}
\renewcommand{\arraystretch}{1}
}

\section*{Espace et géométrie}
{\tiny
\renewcommand{\arraystretch}{1.5}
\begin{tabular}{|p{0.31\linewidth}|p{0.31\linewidth}|p{0.31\linewidth}|}
\hline
\multicolumn{3}{|>{\centering\arraybackslash}p{\linewidth}|}{\cellcolor{lightgray}{\textsc{\textbf{Espace et Géométrie}}}}\\
\hline 
\multicolumn{1}{|>{\centering\arraybackslash}p{0.31\linewidth}|}{\textbf{CM1}}
&
\multicolumn{1}{|>{\centering\arraybackslash}p{0.31\linewidth}|}{\textbf{CM2}}
&
\multicolumn{1}{|>{\centering\arraybackslash}p{0.31\linewidth}|}{$\mathbf{6^{\grave{e}me}}$}\\\hline
\multicolumn{3}{|>{\centering\arraybackslash}p{\linewidth}|}{
\textit{Il est possible,lors de la résolution de problèmes, d’aller avec certains élèves ou toute la classe au-delà des repères de progression identifiés pour chaque niveau.}
}\\\hline 
\multicolumn{3}{|>{\centering\arraybackslash}p{\linewidth}|}{\textbf{Les apprentissages spatiaux}}\\\hline 
\multicolumn{3}{|>{\centering\arraybackslash}p{\linewidth}|}{
Dans la continuité du cycle 2 et tout au long du cycle, les apprentissages spatiaux, en une, deux ou trois dimensions, se réalisent à partir de problèmes de
repérage de déplacement d’objets, d’élaboration de représentation dans des espaces réels, matérialisés (plans, cartes...) ou numériques.
}\\\hline 
\multicolumn{3}{|>{\centering\arraybackslash}p{\linewidth}|}{\textbf{Initiation à la programmation}}\\\hline 
\multicolumn{2}{|p{11cm}|}{
Au CM1 puis au CM2, les élèves apprennent à programmer le déplacement d’un personnage sur un écran.\par
Ils commencent par compléter de tels programmes, puis ils apprennent à corriger un programme erroné. Enfin, ils créent eux-mêmes des programmes permettant d’obtenir des déplacements d’objets ou de personnages.\par 
Les instructions correspondent à des déplacements absolus (liés à l’environnement: \og aller vers l’ouest \fg , \og aller vers la fenêtre \fg ) ou relatifs (liés au personnage: \og tourner d’un quart de tour à gauche \fg).
}
&\multicolumn{1}{p{0.31\linewidth}|}{
La construction de figures géométriques de simples à plus complexes, permet d’amener les élèves vers la répétition d’instructions.\par
Ils peuvent commencer à programmer, seuls ou en équipe, des saynètes impliquant un ou plusieurs personnages interagissant ou se déplaçant simultanément ou successivement.
}\\\hline 
\multicolumn{3}{|>{\centering\arraybackslash}p{\linewidth}|}{\textbf{Les apprentissages géométriques}}\\\hline 
Les élèves tracent avec l’équerre la droite perpendiculaire à une droite donnée en un point donné de cette droite.\par
Ils tracent un carré ou un rectangle de dimensions données.\par
Ils tracent un cercle de centre et de rayon donnés, un triangle rectangle de dimensions données.\par 
Ils apprennent à reconnaître et à nommer une boule, un cylindre, un cône, un cube, un pavé droit, un prisme droit,une pyramide.\par 
Ils apprennent à construire un patron d’un cube de dimension donnée.
&
Les élèves apprennent à reconnaître et nommer un triangle isocèle, un triangle équilatéral, un losange,ainsi qu’à les décrire à partir des propriétés de leurs côtés.\par 
Ils tracent avec l’équerre la droite perpendiculaire à une droite donnée passant par un point donné qui peut être extérieur à la droite.\par 
Ils tracent la droite parallèle à une droite donnée passant par un point donné.\par 
Ils apprennent à construire, pour un cube de dimension donnée, des patrons différents.\par 
Ils apprennent à reconnaître, parmi un ensemble de patrons et de faux patrons donnés, ceux qui correspondent à un solide donné: cube, pavé droit, pyramide.
&
Les élèves sont confrontés à la nécessité de représenter une figure à main levée avant d’en faire un tracé instrumenté. C’est l’occasion d’instaurer le codage de la figure à main levée (au fur et à mesure, égalités de longueurs, perpendicularité, égalité d’angles).\par 
Les figures étudiées sont de plus en plus complexes et les élèves les construisent à partir d’un programme de construction. Ils utilisent selon les cas les figures à main levée, les constructions aux instruments et l’utilisation d’un logiciel de géométrie dynamique.\par 
Ils définissent et différencient le cercle et le disque.Ils réalisent des patrons de pavés droits. Ils travaillent sur des assemblages de solides simples.
\\\hline
\end{tabular}
\renewcommand{\arraystretch}{1}
}
\par\smallskip
{\tiny
\renewcommand{\arraystretch}{1.5}
\begin{tabular}{|p{0.31\linewidth}|p{0.31\linewidth}|p{0.31\linewidth}|}
\hline
\multicolumn{3}{|>{\centering\arraybackslash}p{\linewidth}|}{\cellcolor{lightgray}{\textsc{\textbf{Espace et Géométrie}}}}\\\hline 
\multicolumn{3}{|>{\centering\arraybackslash}p{\linewidth}|}{\textbf{Le raisonnement}}\\\hline 
\multicolumn{1}{|>{\centering\arraybackslash}p{0.31\linewidth}|}{\textbf{CM1}}
&
\multicolumn{1}{|>{\centering\arraybackslash}p{0.31\linewidth}|}{\textbf{CM2}}
&
\multicolumn{1}{|>{\centering\arraybackslash}p{0.31\linewidth}|}{$\mathbf{6^{\grave{e}me}}$}\\\hline
\multicolumn{3}{|>{\centering\arraybackslash}p{\linewidth}|}{
La dimension perceptive, l’usage des instruments et les propriétés élémentaires des figures sont articulés tout au long du cycle.
}\\\hline 
\multicolumn{2}{|p{11cm}|}{
Le raisonnement peut prendre appui sur différents types de codage :
\begin{mylist}
\item signe ajouté aux traits constituant la figure (signe de l’angle droit, mesure, coloriage...) ;
\item qualité particulière du trait lui-même (couleur, épaisseur, pointillés, trait à main levée...) ;
\item élément de la figure qui traduit une propriété implicite (appartenance ou non appartenance,
égalité...) ;
\item nature du support de la figure (quadrillage, papier à réseau pointé, papier millimétré).
\end{mylist}
}
&\multicolumn{1}{p{0.31\linewidth}|}{
Tout le long de l’année se poursuit le travail entrepris au CM2 visant à faire évoluer la perception qu’ont les élèves des activités géométriques (passer de l’observation et du mesurage au codage et au raisonnement).\par
Les élèves utilisent les propriétés relatives aux droites parallèles ou perpendiculaires pour valider la méthode de construction d’une parallèle à la règle et à l’équerre, et établir des relations de perpendicularité ou de parallélisme entre deux droites.\par 
}\\
\cline{1-2}
Un vocabulaire spécifique est employé dès le début du cycle pour désigner des objets, des relations et des propriétés.
&
On amène progressivement les élèves à dépasser la dimension perceptive et instrumentée des propriétés des figures planes pour tendre vers le raisonnement hypothético-déductif.\par 
Il s'agit de conduire sans formalisme des raisonnements simples utilisant les propriétés des figures usuelles ou de la symétrie axiale.
&
Ils complètent leurs acquis sur les propriétés des côtés des figures par celles sur les diagonales et les angles.\par 
Dès que l’étude de la symétrie est suffisamment avancée, ils utilisent les propriétés de conservation
de longueur, d’angle, d’aire et de parallélisme pour justifier une procédure de la construction de la
figure symétrique ou pour répondre à des problèmes de longueur, d’angle, d’aire ou de
parallélisme sans recours à une vérification instrumentée.
\\ 
\hline
\end{tabular}
\renewcommand{\arraystretch}{1}
}
\par\smallskip
{\tiny
\renewcommand{\arraystretch}{1.5}
\begin{tabular}{|p{0.31\linewidth}|p{0.31\linewidth}|p{0.31\linewidth}|}
\hline
\multicolumn{3}{|>{\centering\arraybackslash}p{\linewidth}|}{\cellcolor{lightgray}{\textsc{\textbf{Espace et Géométrie}}}}\\\hline 
\multicolumn{3}{|>{\centering\arraybackslash}p{\linewidth}|}{
\textbf{Le vocabulaire et les notations}
}
\\\hline 
\multicolumn{1}{|>{\centering\arraybackslash}p{0.31\linewidth}|}{\textbf{CM1}}
&
\multicolumn{1}{|>{\centering\arraybackslash}p{0.31\linewidth}|}{\textbf{CM2}}
&
\multicolumn{1}{|>{\centering\arraybackslash}p{0.31\linewidth}|}{$\mathbf{6^{\grave{e}me}}$}
\\\hline
\multicolumn{3}{|>{\centering\arraybackslash}p{\linewidth}|}{
Tout au long du cycle, les notations $(AB)$, $[AB)$, $[AB]$, $AB$, sont toujours précédées du nom de l’objet qu’elles désignent: droite $(AB)$, demi-droite $[AB)$, segment $[AB]$, longueur $AB$. Les élèves apprennent à utiliser le symbole d’appartenance $(\in)$ d’un point à une droite, une demi-droite ou un segment.\par
Le vocabulaire et les notations nouvelles ($\in$, $[AB]$, $(AB)$, $[AB)$, $AB$, $\widehat{AOB}$)  sont introduits au fur et à mesure de leur utilité, et non au départ d’un apprentissage.
}\\\hline 
Le vocabulaire utilisé est le même qu’en fin de cycle 2: côté, sommet, angle, angle droit, face, arête, milieu, droite, segment.\par 
Les élèves commencent à rencontrer la notation \og segment $[AB]$ \fg  pour désigner le segment d’extrémités $A$ et $B$ mais cette notation n’est pas exigible; pour les droites, on parle de la droite \og qui passe par les points $A$ et $B$ \fg , ou de \og la droite $(d)$ \fg .
&
Les élèves commencent à rencontrer la notation \og   droite $(AB)$ \fg , et nomment les angles par leur sommet: par exemple,\og   l’angle $\widehat{A}$ \fg 
&
Les élèves utilisent la notation $AB$ pour désigner la longueur d’un segment qu’ils différencient de la notation du segment $[AB]$.\par
Dès que l’on utilise les objets concernés, les élèves utilisent aussi la notation \og   angle $\widehat{ABC}$ \fg , ainsi que la notation courante pour les demi-droites.\par
Les élèves apprennent à rédiger un programme de construction en utilisant le vocabulaire et les notations appropriés pour des figures simples au départ puis pour des figures plus complexes au fil des périodes suivantes.
\\\hline
\multicolumn{3}{|>{\centering\arraybackslash}p{\linewidth}|}{\textbf{Les instruments}}\\\hline 
Tout au long de l’année, les élèves utilisent la règle graduée ou non graduée ainsi que des bandes de papier à bord droit pour reporter des longueurs.\par 
Ils utilisent l’équerre pour repérer ou construire un angle droit.\par
Ils utilisent aussi d’autres gabarits d’angle ainsi que du papier calque.\par
Ils utilisent le compas pour tracer un cercle, connaissant son centre et un point du cercle ou son centre et la longueur d’un rayon, ou bien pour reporter une longueur.
&
Le travail sur les angles se poursuit, notamment sur des fractions simples de l’angle droit (ex: un \og   demi angle droit \fg , \og   un tiers d’angle droit \fg , \og   l’angle plat comme la somme de deux angles droits \fg ).\par 
Les élèves doivent comprendre que la mesure d’un angle (\og   l’ouverture \fg  formée par les deux demi-droites) ne change pas lorsque l’on prolonge ces demi-droites.
&
Les élèves se servent des instruments (règle, équerre, compas) pour reproduire des figures simples, notamment un triangle de dimensions données. Cette utilisation est souvent combinée à des tracés préalables codés à main levée.\par 
Ils utilisent le rapporteur pour mesurer et construire des angles.\par 
Dès que le cercle a été défini, puis que la propriété caractéristique de la médiatrice d’un segment est connue, les élèves peuvent enrichir leurs procédures de construction à la règle et au compas.
\\\hline
\end{tabular}
\renewcommand{\arraystretch}{1}
}
\par\smallskip
{\tiny
\renewcommand{\arraystretch}{1.5}
\begin{tabular}{|p{0.31\linewidth}|p{0.31\linewidth}|p{0.31\linewidth}|}
\hline
\multicolumn{3}{|>{\centering\arraybackslash}p{\linewidth}|}{\cellcolor{lightgray}{\textsc{\textbf{Espace et Géométrie}}}}\\\hline 
\multicolumn{3}{|>{\centering\arraybackslash}p{\linewidth}|}{
\textbf{La symétrie axiale}
}
\\\hline 
\multicolumn{1}{|>{\centering\arraybackslash}p{0.31\linewidth}|}{\textbf{CM1}}
&
\multicolumn{1}{|>{\centering\arraybackslash}p{0.31\linewidth}|}{\textbf{CM2}}
&
\multicolumn{1}{|>{\centering\arraybackslash}p{0.31\linewidth}|}{$\mathbf{6^{\grave{e}me}}$}
\\\hline
\multicolumn{3}{|p{\linewidth}|}{
Reconnaître si une figure présente un axe de symétrie: on conjecture visuellement l’axe à trouver et on valide cette conjecture en utilisant du papier calque, des découpages, des pliages.\par
Compléter une figure pour qu'elle devienne symétrique par rapport à un axe donné.
\begin{mylist}
\item Symétrie axiale.
\item Figure symétrique, axe de symétrie d’une figure, figures symétriques par rapport à un axe.
\item Propriétés conservées par symétrie axiale.
\end{mylist}
}\\\hline 
Les élèves reconnaissent qu’une figure admet un (ou plusieurs) axe de symétrie, visuellement et/ou par pliage ou en utilisant du papier calque. Ils complètent une figure par symétrie ou construisent le symétrique d’une figure donnée par rapport à un axe donné, par pliage et piquage ou en utilisant du papier calque.
&
Ils observent que deux points sont symétriques par rapport à une droite donnée lorsque le segment qui les joint coupe cette droite perpendiculairement en son milieu.\par 
Ils construisent, à l’équerre et à la règle graduée, le symétrique d’un point, d’un segment, d’une figure par rapport à une droite.
&
Les élèves consolident leurs acquis du CM sur la symétrie axiale et font émerger l’image mentale de la médiatrice d’une part et certaines conservations par symétrie d’autre part.\par 
Ils donnent du sens aux procédures utilisées en CM2 pour la construction de symétriques à la règle et à l’équerre.\par 
À cette occasion:
\begin{mylist}
\item la médiatrice d’un segment est définie et les élèves apprennent à la construire à la règle et à l’équerre;
\item ils étudient les propriétés de conservation de la symétrie axiale.
\end{mylist}
En lien avec les propriétés de la symétrie axiale, ils connaissent la propriété caractéristique de la médiatrice d’un segment et l’utilisent à la fois pour tracer à la règle non graduée et au compas:
\begin{mylist}
\item la médiatrice d’un segment donné;
\item la figure symétrique d’une figure donnée par rapport à une droite donnée.
\end{mylist}
\\\hline
\end{tabular}
\renewcommand{\arraystretch}{1}
}
\end{document}