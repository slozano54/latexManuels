{\tiny
\renewcommand{\arraystretch}{1.5}
\begin{tabular}{|p{0.31\linewidth}|p{0.31\linewidth}|p{0.31\linewidth}|}
\hline
\multicolumn{3}{|>{\centering\arraybackslash}p{\linewidth}|}{\cellcolor{lightgray}{\textsc{\textbf{Nombres et calculs}}}}
\\\hline 
\multicolumn{3}{|>{\centering\arraybackslash}p{\linewidth}|}{
\textbf{Fractions}
}
\\\hline 
\multicolumn{1}{|>{\centering\arraybackslash}p{0.31\linewidth}|}{\textbf{CM1}}
&
\multicolumn{1}{|>{\centering\arraybackslash}p{0.31\linewidth}|}{\textbf{CM2}}
&
\multicolumn{1}{|>{\centering\arraybackslash}p{0.31\linewidth}|}{$\mathbf{6^{\grave{e}me}}$}
\\\hline
Dès la \textbf{période 1} les élèves utilisent d’abord les
fractions simples (comme $\dfrac{2}{3}$,$\dfrac{1}{4}$ ,$\dfrac{5}{2}$ ) dans le cadre
de partage de grandeurs. Ils travaillent des
fractions inférieures et des fractions supérieures à
1.\par\vspace{0.25cm}
Dès la \textbf{période 2}, les fractions décimales sont
régulièrement mobilisées : elles acquièrent le statut
de nombre et sont positionnées sur une droite
graduée. Les élèves comparent des fractions de
même dénominateur. Ils ajoutent des fractions
décimales de même dénominateur. Ils apprennent à
écrire des fractions décimales sous forme de somme
d’un nombre entier et d’une fraction décimale
inférieure à 1.
&
Dès la \textbf{période 1}, dans la continuité du CM1, les
élèves étendent le registre des fractions qu’ils
manipulent (en particulier $\dfrac{1}{1\;000}$) ; ils apprennent à
écrire des fractions sous forme de somme d’un
nombre entier et d’une fraction inférieure à 1.
&
En \textbf{période 1}, sont réactivées les fractions comme
opérateurs de partage vues en CM, puis les
fractions décimales en relation avec les nombres
décimaux (par exemple à partir de mesures de
longueurs) ; les élèves ajoutent des fractions
décimales de même dénominateur.\par\vspace{0.25cm}
En \textbf{période 2} l’addition est étendue à des fractions
de même dénominateur (inférieur ou égal à 5 et en
privilégiant la vocalisation : deux cinquièmes plus
un cinquième égale trois cinquièmes).\par\vspace{0.25cm}
En \textbf{période 3}, les élèves apprennent que $\dfrac{a}{b}$
est le nombre qui, multiplié par b, donne a (définition du
quotient de a par b).
\\\hline
\end{tabular}
\renewcommand{\arraystretch}{1}
}