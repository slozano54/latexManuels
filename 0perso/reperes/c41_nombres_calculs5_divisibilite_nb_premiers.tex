{\tiny
\renewcommand{\arraystretch}{1.5}
\begin{tabular}{|p{0.31\linewidth}|p{0.31\linewidth}|p{0.31\linewidth}|}
\hline
\multicolumn{3}{|>{\centering\arraybackslash}p{\linewidth}|}{\cellcolor{lightgray}{\textsc{\textbf{Nombres et calculs}}}}
\\\hline 
\multicolumn{3}{|>{\centering\arraybackslash}p{\linewidth}|}{
\textbf{Divisibilité, nombres premiers}
}
\\\hline 
\multicolumn{1}{|>{\centering\arraybackslash}p{0.31\linewidth}|}{$\mathbf{5^{\grave{e}me}}$}
&
\multicolumn{1}{|>{\centering\arraybackslash}p{0.31\linewidth}|}{$\mathbf{4^{\grave{e}me}}$}
&
\multicolumn{1}{|>{\centering\arraybackslash}p{0.31\linewidth}|}{$\mathbf{3^{\grave{e}me}}$}
\\\hline
\multicolumn{3}{|>{\centering\arraybackslash}p{\linewidth}|}{
Tout au long du cycle, les élèves sont amenés à modéliser et résoudre des problèmes mettant en jeu la divisibilité et les nombres premiers.
}
\\\hline 
Le travail sur les multiples et les diviseurs, déjà
abordé au cycle 3, est poursuivi. Il est enrichi par
l’introduction de la notion de nombre premier. Les
élèves se familiarisent avec la liste des nombres
premiers inférieurs ou égaux à 30. Ceux-ci sont
utilisés pour la décomposition en produit de
facteurs premiers. Cette décomposition est utilisée
pour reconnaître et produire des fractions égales.
&
Les élèves déterminent la liste des nombres
premiers inférieurs ou égaux à 100 et l’utilisent
pour décomposer des nombres en facteurs
premiers, reconnaître et produire des fractions
égales, simplifier des fractions.
&
La notion de fraction irréductible est introduite.
L’utilisation d’un tableur, d’un logiciel de
programmation ou d’une calculatrice permet
d’étendre la procédure de décomposition en facteurs
premiers.
\\\hline
\end{tabular}
\renewcommand{\arraystretch}{1}
}