{\tiny
\renewcommand{\arraystretch}{1.5}
\begin{tabular}{|p{0.31\linewidth}|p{0.31\linewidth}|p{0.31\linewidth}|}
\hline
\multicolumn{3}{|>{\centering\arraybackslash}p{\linewidth}|}{\cellcolor{lightgray}{\textsc{\textbf{Espace et Géométrie}}}}\\\hline 
\multicolumn{3}{|>{\centering\arraybackslash}p{\linewidth}|}{
\textbf{La symétrie axiale}
}
\\\hline 
\multicolumn{1}{|>{\centering\arraybackslash}p{0.31\linewidth}|}{\textbf{CM1}}
&
\multicolumn{1}{|>{\centering\arraybackslash}p{0.31\linewidth}|}{\textbf{CM2}}
&
\multicolumn{1}{|>{\centering\arraybackslash}p{0.31\linewidth}|}{$\mathbf{6^{\grave{e}me}}$}
\\\hline
\multicolumn{3}{|p{\linewidth}|}{
Reconnaître si une figure présente un axe de symétrie: on conjecture visuellement l’axe à trouver et on valide cette conjecture en utilisant du papier calque, des découpages, des pliages.\par
Compléter une figure pour qu'elle devienne symétrique par rapport à un axe donné.
\begin{mylist}
\item Symétrie axiale.
\item Figure symétrique, axe de symétrie d’une figure, figures symétriques par rapport à un axe.
\item Propriétés conservées par symétrie axiale.
\end{mylist}
}\\\hline 
Les élèves reconnaissent qu’une figure admet un (ou plusieurs) axe de symétrie, visuellement et/ou par pliage ou en utilisant du papier calque. Ils complètent une figure par symétrie ou construisent le symétrique d’une figure donnée par rapport à un axe donné, par pliage et piquage ou en utilisant du papier calque.
&
Ils observent que deux points sont symétriques par rapport à une droite donnée lorsque le segment qui les joint coupe cette droite perpendiculairement en son milieu.\par 
Ils construisent, à l’équerre et à la règle graduée, le symétrique d’un point, d’un segment, d’une figure par rapport à une droite.
&
Les élèves consolident leurs acquis du CM sur la symétrie axiale et font émerger l’image mentale de la médiatrice d’une part et certaines conservations par symétrie d’autre part.\par 
Ils donnent du sens aux procédures utilisées en CM2 pour la construction de symétriques à la règle et à l’équerre.\par 
À cette occasion:
\begin{mylist}
\item la médiatrice d’un segment est définie et les élèves apprennent à la construire à la règle et à l’équerre;
\item ils étudient les propriétés de conservation de la symétrie axiale.
\end{mylist}
En lien avec les propriétés de la symétrie axiale, ils connaissent la propriété caractéristique de la médiatrice d’un segment et l’utilisent à la fois pour tracer à la règle non graduée et au compas:
\begin{mylist}
\item la médiatrice d’un segment donné;
\item la figure symétrique d’une figure donnée par rapport à une droite donnée.
\end{mylist}
\\\hline
\end{tabular}
\renewcommand{\arraystretch}{1}
}