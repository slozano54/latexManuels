{\tiny
\renewcommand{\arraystretch}{1.5}
\begin{tabular}{|p{0.31\linewidth}|p{0.31\linewidth}|p{0.31\linewidth}|}
\hline
\multicolumn{3}{|>{\centering\arraybackslash}p{\linewidth}|}{\cellcolor{lightgray}{\textsc{\textbf{Grandeurs et mesures}}}}
\\\hline 
\multicolumn{3}{|>{\centering\arraybackslash}p{\linewidth}|}{
\textbf{Effet des transformations sur des grandeurs géométriques}
}
\\\hline 
\multicolumn{1}{|>{\centering\arraybackslash}p{0.31\linewidth}|}{$\mathbf{5^{\grave{e}me}}$}
&
\multicolumn{1}{|>{\centering\arraybackslash}p{0.31\linewidth}|}{$\mathbf{4^{\grave{e}me}}$}
&
\multicolumn{1}{|>{\centering\arraybackslash}p{0.31\linewidth}|}{$\mathbf{3^{\grave{e}me}}$}
\\\hline
Les élèves connaissent et utilisent l’effet des
symétries axiale et centrale sur les longueurs, les
aires, les angles.
&
Les élèves connaissent et utilisent l’effet d’un
agrandissement ou d’une réduction sur les
longueurs, les aires et les volumes. Ils le travaillent
en lien avec la proportionnalité.
&
Les élèves connaissent et utilisent l’effet des
transformations au programme (symétries,
translations, rotations, homothéties) sur les
longueurs, les angles, les aires et les volumes.\par\vspace{0.25cm}
Le lien est fait entre la proportionnalité et certaines
configurations ou transformations géométriques
(triangles semblables, homothéties).
\\\hline
\end{tabular}
\renewcommand{\arraystretch}{1}
}