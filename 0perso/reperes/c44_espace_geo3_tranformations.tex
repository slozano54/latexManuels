{\tiny
\renewcommand{\arraystretch}{1.5}
\begin{tabular}{|p{0.31\linewidth}|p{0.31\linewidth}|p{0.31\linewidth}|}
\hline
\multicolumn{3}{|>{\centering\arraybackslash}p{\linewidth}|}{\cellcolor{lightgray}{\textsc{\textbf{Espace et géométrie}}}}\\\hline 
\multicolumn{3}{|>{\centering\arraybackslash}p{\linewidth}|}{
\textbf{Géométrie Plane}
}\\\hline 
\multicolumn{3}{|p{\linewidth}|}{
\textbf{Transformations}
}\\\hline 
\multicolumn{1}{|>{\centering\arraybackslash}p{0.31\linewidth}|}{$\mathbf{5^{\grave{e}me}}$}
&
\multicolumn{1}{|>{\centering\arraybackslash}p{0.31\linewidth}|}{$\mathbf{4^{\grave{e}me}}$}
&
\multicolumn{1}{|>{\centering\arraybackslash}p{0.31\linewidth}|}{$\mathbf{3^{\grave{e}me}}$}\\\hline
Les élèves transforment (à la main ou à l'aide d'un logiciel) une figure par symétrie centrale. Cela permet de découvrir les propriétés de la symétrie centrale (conservation de l'alignement, du parallélisme, des longueurs, des angles) qui sont ensuite admises et utilisées. Le lien est fait entre la symétrie centrale et le parallélogramme. Les élèves identifient des symétries axiales ou centrales dans des frises, des pavages, des rosaces.
&
Les élèves sont amenés à transformer (à la main ou à l'aide d'un logiciel) une figure par translation. Ils identifient des translations dans des frises ou des pavages;le lien est alors fait entre translation et parallélogramme. La définition ponctuelle d'une translation ne figure pas au programme.Toutefois, par commodité, la translation transformant le point A en le point B pourra être nommée translation de vecteur AB, mais aucune connaissance n'est attendue sur l'objet vecteur.
&
Les élèves transforment (à la main ou à l'aide d'un logiciel) une figure par rotation et par homothétie (de rapport positif ou négatif). Le lien est fait entre angle et rotation, entre le théorème de Thalès et les homothéties. Les élèves identifient des transformations dans des frises, des pavages, des rosaces.\par
\textit{Les définitions ponctuelles d'une translation, d'une rotation et d'une homothétie ne figurent pas au programme. Pour faire le lien entre les transformations et les configurations du programme, il est possible d'identifier à la main ou à l'aide d'un logiciel de géométrie, l'effet, sur un triangle donné, de l'enchaînement d'une translation, d'une rotation et d'une homothétie, voire d'une symétrie axiale et réciproquement, pour deux triangles semblables donnés, chercher des transformations transformant l'un en l'autre.}
\\\hline
\end{tabular}
\renewcommand{\arraystretch}{1}
}