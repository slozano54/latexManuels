{\tiny
\renewcommand{\arraystretch}{1.5}
\begin{tabular}{|p{0.31\linewidth}|p{0.31\linewidth}|p{0.31\linewidth}|}
\hline
\multicolumn{3}{|>{\centering\arraybackslash}p{\linewidth}|}{\cellcolor{lightgray}{\textsc{\textbf{Nombres et calculs}}}}\\\hline 
\multicolumn{3}{|>{\centering\arraybackslash}p{\linewidth}|}{
\textbf{Calcul}
}
\\\hline 
\multicolumn{1}{|>{\centering\arraybackslash}p{0.31\linewidth}|}{\textbf{CM1}}
&
\multicolumn{1}{|>{\centering\arraybackslash}p{0.31\linewidth}|}{\textbf{CM2}}
&
\multicolumn{1}{|>{\centering\arraybackslash}p{0.31\linewidth}|}{$\mathbf{6^{\grave{e}me}}$}
\\\hline 
\multicolumn{3}{|p{\linewidth}|}{
Tout au long du cycle, la pratique régulière du calcul conforte et consolide la mémorisation des tables de multiplication jusqu’à 9 dont la maîtrise est attendue
en fin de cycle 2.
}
\\\hline
\multicolumn{3}{|p{\linewidth}|}{
Calcul mental}
\\\hline
Dans la continuité du travail conduit au cycle 2, les 
élèves mémorisent les quatre premiers multiples de 
25 et de 50.\par\vspace{0.25cm}
À partir de la \textbf{période 3}, ils apprennent à multiplier
et à diviser par 10 des nombres décimaux ; ils
apprennent à rechercher le complément au nombre 
entier supérieur.
&
Dès le début de l’année, les élèves apprennent à
diviser un nombre décimal (entier ou non) par 100.\par\vspace{0.25cm}
En \textbf{période 3} les élèves apprennent à multiplier un
nombre décimal (entier ou non) par 5 et par 50.\par\vspace{0.25cm}
Au plus tard en \textbf{période 4}, ils apprennent les critères
de divisibilité par 3 et par 9.
&
Dès la \textbf{période 1}, dans le prolongement des acquis
du CM, on réactive la multiplication et la division
par 10, 100, 1 000.\par\vspace{0.25cm}
À partir de la \textbf{période 2}, les élèves apprennent à
multiplier un nombre entier puis décimal par 0,1 et
par 0,5 (différentes stratégies sont envisagées
selon les situations).
\\\hline
Tout au long de l’année, ils stabilisent leur
connaissance des propriétés des opérations
(ex : 12 + 199 = 199 + 12 ; 5 × 21 = 21 × 5 ;
45 × 21 = 45 × 20 + 45 × 1 ; 6 × 18 = 6 × 20 - 6 × 2).\par\vspace{0.25cm}
À partir de la \textbf{période 3}, ils apprennent les critères
de divisibilité par 2, 5 et 10.\par\vspace{0.25cm}
En \textbf{période 4 ou 5}, ils apprennent à multiplier par
1 000 un nombre décimal.
&
Tout au long de l’année, ils étendent l’utilisation des
principales propriétés des opérations à des calculs
rendus plus complexes par la nature des nombres
en jeu, leur taille ou leur nombre (exemples :
1,2 + 27,9 + 0,8 = 27,9 + 2 ; 3,2 × 25 × 4 = 3,2 × 100).\par\vspace{0.25cm}
Ils étendent l’utilisation des principales propriétés
des opérations (notamment la commutativité de la
multiplication) à des calculs rendus plus complexes
par la nature des nombres en jeu, leur taille, ou leur
nombre (exemple : 1,2 + 27,9 + 0,8 = 27,9 + 2 ;
3,2 × 10 = 10 ×3,2 ; 3,2 × 25 × 4 = 3,2 × 100).
&
Tout au long de l’année, ils stabilisent la
connaissance des propriétés des opérations et les
procédures déjà utilisées à l’école élémentaire, et
utilisent la propriété de distributivité simple dans
les deux sens (par exemple :
23 × 12 = 23 × 10 + 23 × 2 et
23 × 7 + 23 × 3 = 23 × 10).
\\\hline
\end{tabular}
\renewcommand{\arraystretch}{1}
}