{\tiny
\renewcommand{\arraystretch}{1.5}
\begin{tabular}{|p{0.31\linewidth}|p{0.31\linewidth}|p{0.31\linewidth}|}
\hline
\multicolumn{3}{|>{\centering\arraybackslash}p{\linewidth}|}{\cellcolor{lightgray}{\textsc{\textbf{Espace et Géométrie}}}}\\\hline 
\hline 
\multicolumn{1}{|>{\centering\arraybackslash}p{0.31\linewidth}|}{\textbf{CM1}}
&
\multicolumn{1}{|>{\centering\arraybackslash}p{0.31\linewidth}|}{\textbf{CM2}}
&
\multicolumn{1}{|>{\centering\arraybackslash}p{0.31\linewidth}|}{$\mathbf{6^{\grave{e}me}}$}\\\hline
\multicolumn{3}{|>{\centering\arraybackslash}p{\linewidth}|}{\textbf{Les apprentissages géométriques}}\\\hline 
Les élèves tracent avec l’équerre la droite perpendiculaire à une droite donnée en un point donné de cette droite.\par
Ils tracent un carré ou un rectangle de dimensions données.\par
Ils tracent un cercle de centre et de rayon donnés, un triangle rectangle de dimensions données.\par 
Ils apprennent à reconnaître et à nommer une boule, un cylindre, un cône, un cube, un pavé droit, un prisme droit,une pyramide.\par 
Ils apprennent à construire un patron d’un cube de dimension donnée.
&
Les élèves apprennent à reconnaître et nommer un triangle isocèle, un triangle équilatéral, un losange,ainsi qu’à les décrire à partir des propriétés de leurs côtés.\par 
Ils tracent avec l’équerre la droite perpendiculaire à une droite donnée passant par un point donné qui peut être extérieur à la droite.\par 
Ils tracent la droite parallèle à une droite donnée passant par un point donné.\par 
Ils apprennent à construire, pour un cube de dimension donnée, des patrons différents.\par 
Ils apprennent à reconnaître, parmi un ensemble de patrons et de faux patrons donnés, ceux qui correspondent à un solide donné: cube, pavé droit, pyramide.
&
Les élèves sont confrontés à la nécessité de représenter une figure à main levée avant d’en faire un tracé instrumenté. C’est l’occasion d’instaurer le codage de la figure à main levée (au fur et à mesure, égalités de longueurs, perpendicularité, égalité d’angles).\par 
Les figures étudiées sont de plus en plus complexes et les élèves les construisent à partir d’un programme de construction. Ils utilisent selon les cas les figures à main levée, les constructions aux instruments et l’utilisation d’un logiciel de géométrie dynamique.\par 
Ils définissent et différencient le cercle et le disque.Ils réalisent des patrons de pavés droits. Ils travaillent sur des assemblages de solides simples.
\\\hline
\multicolumn{3}{|>{\centering\arraybackslash}p{\linewidth}|}{\textbf{Le raisonnement}}\\\hline 
\multicolumn{3}{|>{\centering\arraybackslash}p{\linewidth}|}{
La dimension perceptive, l’usage des instruments et les propriétés élémentaires des figures sont articulés tout au long du cycle.
}\\\hline 
\multicolumn{2}{|p{11cm}|}{
Le raisonnement peut prendre appui sur différents types de codage :
\begin{mylist}
\item signe ajouté aux traits constituant la figure (signe de l’angle droit, mesure, coloriage...) ;
\item qualité particulière du trait lui-même (couleur, épaisseur, pointillés, trait à main levée...) ;
\item élément de la figure qui traduit une propriété implicite (appartenance ou non appartenance,
égalité...) ;
\item nature du support de la figure (quadrillage, papier à réseau pointé, papier millimétré).
\end{mylist}
}
&\multicolumn{1}{p{0.31\linewidth}|}{
Tout le long de l’année se poursuit le travail entrepris au CM2 visant à faire évoluer la perception qu’ont les élèves des activités géométriques (passer de l’observation et du mesurage au codage et au raisonnement).\par
Les élèves utilisent les propriétés relatives aux droites parallèles ou perpendiculaires pour valider la méthode de construction d’une parallèle à la règle et à l’équerre, et établir des relations de perpendicularité ou de parallélisme entre deux droites.\par 
}\\
\cline{1-2}
Un vocabulaire spécifique est employé dès le début du cycle pour désigner des objets, des relations et des propriétés.
&
On amène progressivement les élèves à dépasser la dimension perceptive et instrumentée des propriétés des figures planes pour tendre vers le raisonnement hypothético-déductif.\par 
Il s'agit de conduire sans formalisme des raisonnements simples utilisant les propriétés des figures usuelles ou de la symétrie axiale.
&
Ils complètent leurs acquis sur les propriétés des côtés des figures par celles sur les diagonales et les angles.\par 
Dès que l’étude de la symétrie est suffisamment avancée, ils utilisent les propriétés de conservation
de longueur, d’angle, d’aire et de parallélisme pour justifier une procédure de la construction de la
figure symétrique ou pour répondre à des problèmes de longueur, d’angle, d’aire ou de
parallélisme sans recours à une vérification instrumentée.
\\ 
\hline
\end{tabular}
\renewcommand{\arraystretch}{1}
}