{\tiny
\renewcommand{\arraystretch}{1.5}
\begin{tabular}{|p{0.31\linewidth}|p{0.31\linewidth}|p{0.31\linewidth}|}
\hline
\multicolumn{3}{|>{\centering\arraybackslash}p{\linewidth}|}{\cellcolor{lightgray}{\textsc{\textbf{Nombres et calculs}}}}
\\\hline 
\multicolumn{3}{|>{\centering\arraybackslash}p{\linewidth}|}{
\textbf{Nombres décimaux}
}
\\\hline 
\multicolumn{1}{|>{\centering\arraybackslash}p{0.31\linewidth}|}{\textbf{CM1}}
&
\multicolumn{1}{|>{\centering\arraybackslash}p{0.31\linewidth}|}{\textbf{CM2}}
&
\multicolumn{1}{|>{\centering\arraybackslash}p{0.31\linewidth}|}{$\mathbf{6^{\grave{e}me}}$}
\\\hline 
\multicolumn{3}{|>{\centering\arraybackslash}p{\linewidth}|}{
Tout au long du cycle, les désignations orale et écrite des nombres décimaux basées sur les unités de numération contribuent à l’acquisition du sens des
nombres décimaux (par exemple pour $3,12$ : « trois unités et douze centièmes » ou « trois unités, un dixième et deux centièmes » ou « trois cent douze
centièmes »).
}
\\\hline
À partir de la \textbf{période 2}, les élèves apprennent à
utiliser les nombres décimaux ayant au plus deux
décimales en veillant à mettre en relation fractions
décimales et écritures à virgule\par\vspace{0.25cm}
(ex : $3,12 = 3 +\dfrac{12}{100}$).
Ils connaissent des écritures décimales de fractions simples 
($\dfrac{1}{2} = 0,5 = \dfrac{5}{10}$ ; $\dfrac{1}{4}=\dfrac{25}{100}=0,25$;
la moitié d’un entier sur des petits nombres).
&
Dès la \textbf{période 1}, les élèves rencontrent et utilisent
des nombres décimaux ayant une, deux ou trois
décimales.\par\vspace{0.25cm}
Ils connaissent des écritures décimales de fractions
simples ($\dfrac{1}{5} = 0,2 = \dfrac{2}{10}$ ; $\dfrac{3}{4} = \dfrac{75}{100}=0,75$ ; la moitié d’un entier).
&
Dès la \textbf{période 1}, dans le prolongement des acquis du
CM, on travaille sur les décimaux jusqu’à trois
décimales. La quatrième décimale sera introduite en
\textbf{période 2} au travers des diverses activités.
\\\hline
\end{tabular}
\renewcommand{\arraystretch}{1}
}