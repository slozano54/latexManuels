{\tiny
\renewcommand{\arraystretch}{1.5}
\begin{tabular}{|p{0.31\linewidth}|p{0.31\linewidth}|p{0.31\linewidth}|}
\hline
\multicolumn{3}{|>{\centering\arraybackslash}p{\linewidth}|}{\cellcolor{lightgray}{\textsc{\textbf{Grandeurs et mesures}}}}
\\\hline 
\multicolumn{3}{|>{\centering\arraybackslash}p{\linewidth}|}{
\textbf{Les aires}
}
\\\hline 
\multicolumn{1}{|>{\centering\arraybackslash}p{0.31\linewidth}|}{\textbf{CM1}}
&
\multicolumn{1}{|>{\centering\arraybackslash}p{0.31\linewidth}|}{\textbf{CM2}}
&
\multicolumn{1}{|>{\centering\arraybackslash}p{0.31\linewidth}|}{$\mathbf{6^{\grave{e}me}}$}
\\\hline
Les élèves comparent des surfaces selon leur aire
par estimation visuelle, par superposition ou
découpage et recollement. Ils estiment des aires,
ou les déterminent, en faisant appel à une aire de
référence.
Le lien est fait chaque fois que possible avec le
travail sur les fractions.
&
L’utilisation d’une unité de référence est
systématique. Cette unité peut être une maille
d’un réseau quadrillé adapté, le $cm^2$ , le $dm^2$ ou
le $m^2$.\par\vspace{0.25cm}
Les élèves apprennent à utiliser les formules
d’aire du carré, du rectangle et du triangle
rectangle.
&
En relation avec le travail sur la quatrième décimale, les
élèves utilisent les multiples et sous-multiples du $m^2$ et les
relations qui les lient. Ils utilisent la formule pour calculer
l’aire d’un triangle quelconque lorsque les données sont
exprimées avec des nombres entiers.\par\vspace{0.25cm}
Après avoir consolidé le produit de décimaux, ils utilisent les
formules pour calculer l’aire d’un triangle quelconque et celle
d’un disque.
\\\hline
\end{tabular}
\renewcommand{\arraystretch}{1}
}