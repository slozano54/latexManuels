{\tiny
\renewcommand{\arraystretch}{1.5}
\begin{tabular}{|p{0.31\linewidth}|p{0.31\linewidth}|p{0.31\linewidth}|}
\hline
\multicolumn{3}{|>{\centering\arraybackslash}p{\linewidth}|}{\cellcolor{lightgray}{\textsc{\textbf{Organisation et gestion de données, fonction}}}}
\\\hline 
\multicolumn{3}{|>{\centering\arraybackslash}p{\linewidth}|}{
\textbf{Probabilités}
}
\\\hline 
\multicolumn{1}{|>{\centering\arraybackslash}p{0.31\linewidth}|}{$\mathbf{5^{\grave{e}me}}$}
&
\multicolumn{1}{|>{\centering\arraybackslash}p{0.31\linewidth}|}{$\mathbf{4^{\grave{e}me}}$}
&
\multicolumn{1}{|>{\centering\arraybackslash}p{0.31\linewidth}|}{$\mathbf{3^{\grave{e}me}}$}
\\\hline
Les élèves appréhendent le hasard à travers des
expériences concrètes : pile ou face, dé, roue de
loterie, urne...\par\vspace{0.25cm}
Le vocabulaire relatif aux probabilités (expérience
aléatoire, issue, événement, probabilité) est utilisé.
Le placement d’un événement sur une échelle de
probabilités et la détermination de probabilités
dans des situations très simples d’équiprobabilité
contribuent à une familiarisation avec la
modélisation mathématique du hasard.\par\vspace{0.25cm}
Pour exprimer une probabilité, on accepte des
formulations du type \og 2 chances sur 5 \fg.
&
Les calculs de probabilités concernent des
situations simples, mais ne relevant pas
nécessairement du modèle équiprobable. Le lien
est fait entre les probabilités de deux événements
contraires.
&
Le constat de la stabilisation des fréquences
s’appuie sur la simulation d’expériences aléatoires
à une épreuve à l’aide d’un tableur ou d’un logiciel
de programmation. Les calculs de probabilités, à
partir de dénombrements, s’appliquent à des
contextes simples faisant prioritairement intervenir
une seule épreuve. Dans des cas très simples, il est
cependant possible d’introduire des expériences à
deux épreuves. Les dénombrements s’appuient
alors uniquement sur des tableaux à double entrée,
la notion d’arbre ne figurant pas au programme.\par\vspace{0.25cm}
Les élèves simulent une expérience aléatoire à
l’aide d’un tableur ou d’un logiciel de
programmation.
\\\hline
\end{tabular}
\renewcommand{\arraystretch}{1}
}