\PassOptionsToPackage{table}{xcolor}
\PassOptionsToPackage{svgnames}{xcolor}

\documentclass[nocrop]{sesamanuel}
\renewcommand\PrefixeCorrection{fixSolutionLabyNombreCorrections/}
\usepackage{etex}
\usepackage{ProfCollege}

\begin{document}

\themaG
\chapter{Fix Sol LabyNombre}


\exercicesbase

%\begin{colonne*exercice}

  \begin{exercice*}[Exercice 1]
    \LabyNombre[Multiple=7,Longueur=12,Largeur=8,XDepart=2,YDepart=2,XArrivee=11,YArrivee=6,Murs]
  \end{exercice*}
  
  \begin{corrige}
  Corrigé 1

  \LabyNombre[Solution,Multiple=7,Longueur=12,Largeur=8,XDepart=2,YDepart=2,XArrivee=11,YArrivee=6,Murs]  
  \end{corrige}  

  \begin{exercice*}[Exercice 2]
    \LabyNombre[Multiple=10,Longueur=12,Largeur=8,XDepart=2,YDepart=2,XArrivee=10,YArrivee=6]
  \end{exercice*}
  
  \begin{corrige}
  Corrigé 2

  \LabyNombre[Multiple=10,Longueur=12,Largeur=8,XDepart=2,YDepart=2,XArrivee=10,YArrivee=6,Solution]
  \end{corrige}  

%\end{colonne*exercice}


% La commande \Recreation permet d'inclure un saut de page
\Recreation

\begin{enigme}
  Enigme 

  \LabyNombre[Multiple=7,Longueur=12,Largeur=8,XDepart=2,YDepart=2,XArrivee=11,YArrivee=6,Murs]
\end{enigme}

\begin{corrige}
   Test corrigé énigme

   \LabyNombre[Solution,Multiple=7,Longueur=12,Largeur=8,XDepart=2,YDepart=2,XArrivee=11,YArrivee=6,Murs]  
\end{corrige}

\AfficheCorriges[1]


\end{document}
