    %\setcounter{partie}{0} % Pour s'assurer que le compteur de \partie est à zéro dans les corrigés
    % \phantom{rrr}
    \begin{enumerate}
        \item Écrire tous les multiples de $6$ inférieurs à $90$.

        \Circled{$0$} ; $6$ ; $12$ ; $18$ ; $24$ ; \Circled{$30$} ; $36$ ; $42$ ; $48$ ; $54$ ; \Circled{$60$} ; $66$ ; $72$ ; $78$ ; $84$ ; \Circled{$90$}.
        \item Écrire tous les multiples de $15$ inférieurs à $90$.

        \Circled{$0$} ; $15$ ; \Circled{$30$} ; $45$ ; \Circled{$60$} ; $75$ ; \Circled{$90$}.
        \item Entourer les multiples communs à $6$ et $15$.

        Cf plus haut.
        \item En déduire leur plus petit multiple commun non nul.

        Leur plus petit multiple commun à $6$ et $15$ vaut donc $30$.
    \end{enumerate}
