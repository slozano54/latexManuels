    %\setcounter{partie}{0} % Pour s'assurer que le compteur de \partie est à zéro dans les corrigés
    % \phantom{rrr}
    \begin{enumerate}
        \item Dire sous quelle forme s'écrit un multiple de $24$.

        $24\times n$ où $n$ est un nombre entier quelconque.
        \item Démontrer qu'un multiple de $24$ est également un multiple de $4$.

        Un multiple de $24$, s'écrit $24\times n$ or $24\times n = 4\times (6\times n)$ et $6\times n$ est un entier si $n$ l'est.

        donc un mutliple de $24$ est aussi un multiple de $4$.
        \item Démontrer que la somme de deux multiples de $24$ est un multiple de $24$.

        Soient deux multiples de $24$ : $24\times n_1$ et $24\times n_2$ où $n_1$ et $n_2$ sont des entiers.

        $24\times n_1+24\times n_2 = 24\times (n_1+n_2)$ et $n_1+n_2$ est un entier.

        Donc la somme de deux multiples de $24$ est un multiple de $24$.
    \end{enumerate}
