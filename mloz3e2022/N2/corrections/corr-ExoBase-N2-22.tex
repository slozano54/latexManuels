    %\setcounter{partie}{0} % Pour s'assurer que le compteur de \partie est à zéro dans les corrigés
    % \phantom{rrr}    \begin{enumerate}
    Un pâtissier dispose de $450$ morceaux de pommes et de $315$ framboises.
    Il veut préparer le maximum de tartelettes identiques en utilisant tous les fruits.
    \begin{enumerate}
        \item Justifier qu'il peut préparer $15$ tartelettes.

        {\color{red} $450$ et $315$ sont divisibles par $15$ donc il peut faire $15$ tartelettes.}
        \item Justifier qu'il ne peut pas préparer $21$ tartelettes.

        {\color{red} $315$ est divisible par $21$ mais $450$ n'est pas divisible par $21$ donc il ne peut pas faire $21$ tartelettes.}
        \item Trouve les diviseurs communs de $450$ et $315$.

        {\color{red} $450 = 2\times 3^2\times 5^2$ et $315=3^2\times 5\times 7$

        Donc les diviseurs communs à $450$ et $315$ sont : $3$ ; $3^2=9$ ; $3\times 5 = 15$ ; $3^2\times 5 = 45$.
        }
        \item Déterminer le nombre de tartelettes que ce pâtissier va faire.

        {\color{red} Ce pâtissier fera donc $45$ tartelettes.}
    \end{enumerate}
