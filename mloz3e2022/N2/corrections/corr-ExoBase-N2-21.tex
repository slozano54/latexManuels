    %\setcounter{partie}{0} % Pour s'assurer que le compteur de \partie est à zéro dans les corrigés
    % \phantom{rrr}    \begin{enumerate}
        \begin{enumerate}
            \item
            \begin{enumerate}
                \item Liste des premiers multiples de $6$ : \\
                $1\times6 = 6$ ; $2\times6 = 12$ ; $3\times6 = 18$ ; $4\times6 = 24$ ; $5\times6 = 30$ ; \\
                $6\times6 = 36$ ; $7\times6 = 42$ ; $8\times6 = 48$ ; $9\times6 = 54$ ; $10\times6 = \mathbf{{\color[HTML]{f15929}60}}$ ; \\
                $11\times6 = 66$ ; $12\times6 = 72$ ; $13\times6 = 78$ ; $14\times6 = 84$ ; $15\times6 = 90$ ; \\
                \dots \\
                Liste des premiers multiples de $20$ : \\
                $1\times20 = 20$ ; $2\times20 = 40$ ; $3\times20 = \mathbf{{\color[HTML]{f15929}60}}$ ; $4\times20 = 80$ ; $5\times20 = 100$ ; \\
                \dots \\
                \medskip
                \item Le plus petit multiple commun à $6$ et $20$ vaut donc $60$.\\
                Il suffit donc que chaque roue tourne de $60$ dents pour faire un nombre entier de tours et ainsi revenir dans sa position initiale.\\
                En effet, chaque roue doit tourner de façon à ce que le nombre total de dents utilisé soit un multiple de son nombre
                de dents soit au minimum de $60$ dents.\\
                 Cela correspond à $(60\text{ dents})\div (6\text{ dents/tour}) = 10$ tours pour la roue n°$1$.\\
                Cela correspond à $(60\text{ dents})\div (20\text{ dents/tour}) = 3$ tours pour la roue n°$2$.
            \end{enumerate}
            \item Pour un nombre de dents plus élevé, il est plus commode d'utiliser les décompositions en produit de facteurs premiers.\\
            \begin{enumerate}
                \item Décomposition de $72$ en produit de facteurs premiers :  $72 = 2\times2\times2\times3\times3$.\\
                Décomposition de $39$ en produit de facteurs premiers :  $39 = 3\times13$.
               \medskip
                \item Pour retrouver la position initiale,
                chaque roue doit tourner de façon à ce que le nombre total de dents utilisé soit un multiple de son nombre
                de dents.\\
                Soit, grâce aux décompositions précédentes, au minimum de $2\times2\times2\times3\times3\times13 = 936$ dents.\\
                 Cela correspond à $(936\text{ dents})\div (72\text{ dents/tour}) = 13$ tours pour la roue n°$1$.\\
                Cela correspond à $(936\text{ dents})\div (39\text{ dents/tour}) = 24$ tours pour la roue n°$2$.
            \end{enumerate}
            \item Puisque la roue n°$2$, qui a $38$ dents, fait $8$  tours , cela représente $304$ dents.\\
            La roue n°$1$ doit donc aussi tourner de $304$ dents, ceci en $19$  tours .\\
            On obtient donc $(304\text{ dents})\div (19\text{ tours }) = 16 \text{ dents/tour}.$\\
            La roue n°$1$ a donc $16$ dents.
        \end{enumerate}
