    %\setcounter{partie}{0} % Pour s'assurer que le compteur de \partie est à zéro dans les corrigés
    % \phantom{rrr}
    Dans un logiciel de programmation par blocs, l'intruction
    \raisebox{-0.3\totalheight}[0.3\totalheight]{\raisebox{\depth}{
    \begin{Scratch}[Echelle=0.75]
        Place OpModulo(" "," ");
    \end{Scratch}
    }}
    permet de calculer le reste d'une division euclidienne.

    Compléter les pointillés dans le script suivant.

    \begin{Scratch}[Echelle=0.75]
        Place Drapeau;
        Place Demander("Donnez-moi un nombre entier, svp");
        Place Si(TestOpEgal(OpModulo(OvalCap("réponse"),"2"),"0"));
        Place DireT(OpRegrouper(OvalCap("réponse")," est un nombre \textbf{pair}"),"2");
        Place Sinon;
        Place DireT(OpRegrouper(OvalCap("réponse")," est un nombre \textbf{impair}"),"2");
        Place FinBlocSi;
    \end{Scratch}
