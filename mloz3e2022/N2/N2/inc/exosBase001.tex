\begin{exercice*}
    Calculer le nombre $n$ sachant que :
    \begin{enumerate}
            \item Le quotient de la division euclidienne de $71$ par $n$ vaut $5$ et que le reste vaut $6$.
            \item Le quotient de la division euclidienne de $148$ par $19$ vaut $7$ et que le reste vaut $n$.            
    \end{enumerate}
\end{exercice*}
\begin{corrige}
    %\setcounter{partie}{0} % Pour s'assurer que le compteur de \partie est à zéro dans les corrigés
    % \phantom{rrr}    
    Calculer le nombre $n$ sachant que :

    \begin{enumerate}
            \item Le quotient de la division euclidienne de $71$ par $n$ vaut $5$ et que le reste vaut $6$.
            
            $71 = 5\times n + 6$ soit $5\times n + 6 =71$
            
            On peut trouver $n$ :
            \begin{itemize}
                \item En remontant un programme de calcul : 
                
                \ProgCalcul[SansCalcul]{n,*5 +6,5\times n 5\times n+6=71}

                \ProgCalcul[SansCalcul]{5\times n+6=71,-6 /5,5\times n=65 n=65\div 5=13}
                
                \ProgCalcul{71,-6 /5}
                \item En résolvant l'équation :
            \ResolEquation[Lettre=n,Decomposition,Fleches,Decimal,Ecart=1.1]{0}{71}{5}{6}
            \end{itemize}
            \item Le quotient de la division euclidienne de $148$ par $19$ vaut $7$ et que le reste vaut $n$.            
            
            $148 = 7\times 19 + n$ soit $7\times 19 + n = 148$ ou $133+n=148$ ou encore $n+133=148$

            On peut trouver $n$ :
            \begin{itemize}
                \item En remontant un programme de calcul : 
                
                \ProgCalcul[SansCalcul]{n,+133,n+133=148}

                \ProgCalcul[SansCalcul]{n+133=148,-133,n=148-133=15}
                
                \ProgCalcul{148,-133}
                \item En résolvant l'équation :
            \ResolEquation[Lettre=n,Decomposition,Fleches,Ecart=1.2]{0}{148}{1}{133}
            \end{itemize}

    \end{enumerate}
\end{corrige}

