\begin{exercice*}
    Une guirlande électrique est constituée de lumières rouges et vertes.
    Les lumières rouges s'allument toutes les \Temps{;;;;;385} et les vertes toutes les \Temps{;;;;;105}.
    À un instant donné, on voit les lumières rouges et vertes allumées en même temps.
    \begin{enumerate}
        \item Justifier s'il est possible qu'elles s'allument à nouveau ensemble au bout de \Temps{;;;;;770}.
        \item Déterminer au bout de combien de temps ce phénomène se reproduira. Justifier.
    \end{enumerate}    
    \hrefMathalea{https://coopmaths.fr/mathalea.html?ex=3A11-1,s=2,n=1,cd=1&v=l}
\end{exercice*}
\begin{corrige}
    %\setcounter{partie}{0} % Pour s'assurer que le compteur de \partie est à zéro dans les corrigés
    % \phantom{rrr}    \begin{enumerate}
        Une guirlande électrique est constituée de lumières rouges et vertes.
        Les lumières rouges s'allument toutes les \Temps{;;;;;385} et les vertes toutes les \Temps{;;;;;105}.
        À un instant donné, on voit les lumières rouges et vertes allumées en même temps.
    
        \begin{enumerate}
            \item Justifier s'il est possible qu'elles s'allument à nouveau ensemble au bout de \Temps{;;;;;770}.
            
            {\color{red}{$770$ est un multiple de $385$ mais pas de $105$, donc elles ne s'allumeront pas à nouveau ensemble au bout de \Temps{;;;;;770}}}            
            \item Déterminer au bout de combien de temps ce phénomène se reproduira. Justifier.

            \color{red}{
                $385 = 5\times 7\times 11$ et $105=5\times 7\times 3$

                Donc le plus petit multiple commun de $385$ et $105$ vaut $5\times 7\times 11\times 3 = \num{1155}$.

                Les lumières rouges et vertes s'allumeront à nouveau ensemble au bout de \Temps{;;;;;1155}.
            }

        \end{enumerate}                    
\end{corrige}

