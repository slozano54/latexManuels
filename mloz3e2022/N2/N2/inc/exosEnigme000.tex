% Les enigmes ne sont pas numérotées par défaut donc il faut ajouter manuellement la numérotation
% si on veut mettre plusieurs enigmes
% \refstepcounter{exercice}
% \numeroteEnigme
\begin{enigme}
    \begin{Large}
        \textsc{\textbf{Jeu de Juniper Green}}
    \end{Large}

    Le jeu du Juniper green ou jeu des diviseurs et multiples a été créé par Richard Porteous enseignant à l'école de Juniper Green.

    On définit une limite de jeu, par exemple les nombres de 1 à 100.
    
    Lorsque l'on joue à deux, chaque joueur utilise un couleur différente.

    Le premier joueur choisit un nombre et l'entoure avec sa couleur.
    
    Le joueur suivant doit choisir un multiple ou un diviseur du nombre choisi précédemment.
    
    Il est possible aussi de jouer seul. Dans ce cas, il faut réaliser la plus longue suite de nombres.
    
    \textbf{À vos jeux, prêt, partez !!!}
    \begin{center}
        {\renewcommand{\arraystretch}{2}
        \begin{tabularx}{\linewidth}{|*{10}{>{\centering\arraybackslash}X|}}
            \hline	
            $1 $  & $2 $  & $3 $  & $4 $  & $5 $  & $6 $  & $7 $  & $8 $  & $9 $  & $10 $\\
            \hline
            $11$  & $12$  & $13$  & $14$  & $15$  & $16$  & $17$  & $18$  & $19$  & $20 $\\
            \hline
            $21$  & $22$  & $23$  & $24$  & $25$  & $26$  & $27$  & $28$  & $29$  & $30 $\\
            \hline
            $31$  & $32$  & $33$  & $34$  & $35$  & $36$  & $37$  & $38$  & $39$  & $40 $\\
            \hline
            $41$  & $42$  & $43$  & $44$  & $45$  & $46$  & $47$  & $48$  & $49$  & $50 $\\
            \hline
            $51$  & $52$  & $53$  & $54$  & $55$  & $56$  & $57$  & $58$  & $59$  & $60 $\\
            \hline
            $61$  & $62$  & $63$  & $64$  & $65$  & $66$  & $67$  & $68$  & $69$  & $70 $\\
            \hline
            $71$  & $72$  & $73$  & $74$  & $75$  & $76$  & $77$  & $78$  & $79$  & $80 $\\
            \hline
            $81$  & $82$  & $83$  & $84$  & $85$  & $86$  & $87$  & $88$  & $89$  & $90 $\\
            \hline
            $91$  & $92$  & $93$  & $94$  & $95$  & $96$  & $97$  & $98$  & $99$  & $100$ \\
            \hline
        \end{tabularx}
    }
    \end{center}
    
    \hrefLien[\emoji{star-struck}]{http://lozano.maths.free.fr/jeux-serieux/juniperGreen-a2.html}{Jouer à deux en ligne avec les entiers de $1$ à $40$}

    \hrefLien[\emoji{star-struck}]{http://lozano.maths.free.fr/jeux-serieux/juniperGreen-seul.html}{Jouer seul en ligne avec les entiers de $1$ à $40$}
\end{enigme}

% Pour le corrigé, il faut décrémenter le compteur, sinon il est incrémenté deux fois
% \addtocounter{exercice}{-1}
% \begin{corrige}
%     \ldots
% \end{corrige}