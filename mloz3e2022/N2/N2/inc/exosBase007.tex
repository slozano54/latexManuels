\begin{exercice*}
    \begin{enumerate}
        \item En écrivant les dix premiers multiples de $30$ et de $42$, trouver leur plus petit multiple commun.
        \smallskip
        \item Calculer la différence $\dfrac{7}{30}-\dfrac{-3}{42}$.
    \end{enumerate}
\end{exercice*}
\begin{corrige}
    %\setcounter{partie}{0} % Pour s'assurer que le compteur de \partie est à zéro dans les corrigés
    % \phantom{rrr}
    \begin{enumerate}
        \item En écrivant les dix premiers multiples de $30$ et de $42$, trouver leur plus petit multiple commun non nul.
        
        \begin{itemize}
            \item Multiples de $30$ : $0$ ; $30$ ; $60$ ; $90$ ; $120$ ; $150$ ; $180$ ; $\mathbf{210}$ ; $240$ ; $270$ ; $300$.
            \item Multiples de $42$ : $0$ ; $42$ ; $84$ ; $126$ ; $168$ ; $\mathbf{210}$ ; $252$ ; $294$ ; $336$ ; $378$ ; $420$.
        \end{itemize}

        Le plus petit multiple commun est donc $210$.

        \smallskip
        \item Calculer la différence $\dfrac{7}{30}-\dfrac{-3}{42}$.
        
        \begin{spacing}{2}
            $A = \dfrac{7}{30}-\dfrac{-3}{42}$

            $A = \dfrac{7\times 7}{30\times 7}-\dfrac{-3\times 5}{42\times 5}$
            
            $A = \dfrac{49}{210}-\dfrac{-15}{210}$

            $A = \dfrac{64}{210}$

            $A = \dfrac{32}{105}$
        \end{spacing}
    \end{enumerate}

\end{corrige}

