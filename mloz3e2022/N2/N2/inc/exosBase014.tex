\begin{exercice*}
    \begin{enumerate}
        \item Entourer les fractions simplifiables.
        
        \smallskip        
        $\dfrac{4}{6}$ ; $\dfrac{3}{19}$ ; $\dfrac{15}{30}$ ; $\dfrac{1}{82}$ ; $\dfrac{42}{39}$.
        \smallskip
        \item Simplifier celles qui sont simplifiables.
    \end{enumerate}
    
    \hrefMathalea{https://coopmaths.fr/mathalea.html?ex=5N13,s=11,s2=true,n=10,i=0&v=l}
\end{exercice*}
\begin{corrige}
    %\setcounter{partie}{0} % Pour s'assurer que le compteur de \partie est à zéro dans les corrigés
    % \phantom{rrr}
    \begin{enumerate}
        \item Entoure les fractions simplifiables.
        
        \smallskip        
        \Circled{$\dfrac{4}{6}$} ; $\dfrac{3}{19}$ ; \Circled{$\dfrac{15}{30}$} ; $\dfrac{1}{82}$ ; \Circled{$\dfrac{42}{39}$}.
        \smallskip

        \item Simplifier celles qui sont simplifiables.
        
        $\dfrac{4}{6}=\dfrac{4\div 2}{6\div 2}=\dfrac{2}{3}$ ; $\dfrac{15}{30}=\dfrac{\cancel{15}\times 1}{\cancel{15}\times 2}=\dfrac{1}{2}$ ; 
        $\dfrac{42}{39}=\dfrac{14}{13}$ en simplifiant par $3$.
    \end{enumerate}
\end{corrige}

