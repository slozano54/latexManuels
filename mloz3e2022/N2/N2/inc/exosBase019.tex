\begin{exercice*}
    \begin{enumerate}
        \item Écrire la décomposition en produits de facteurs premiers de $231$ et $712$.
        \item En déduire le plus grand diviseur commun de $231$ et $712$.
        \item Déterminer si la fraction $\dfrac{231}{712}$ est irréductible ou non.
    \end{enumerate}    
\end{exercice*}
\begin{corrige}
    %\setcounter{partie}{0} % Pour s'assurer que le compteur de \partie est à zéro dans les corrigés
    % \phantom{rrr}    \begin{enumerate}
        \begin{enumerate}
            \item Écrire la décomposition en produits de facteurs premiers de $231$ et $712$.
            
            \textcolor{red}{
                $231=3\times 7\times 11$ et $712= 2^3\times 89$.
            }      
            \item En déduire le plus grand diviseur commun de $231$ et $712$.
            
            \textcolor{red}{
                $231$ et $712$ n'ont pas de diviseur premier commun. Donc leur seul diviseur commun vaut $1$, c'est donc aussi le plus grand.
            }        
            \item Déterminer si la fraction $\dfrac{231}{712}$ est irréductible ou non.
            
            \textcolor{red}{
                On "peut simplifier" la fraction $\dfrac{231}{712}$ par $1$ mais $\dfrac{231}{712} = \dfrac{231\div 1}{712\div 1} = \dfrac{231}{712}$ !
                Donc la fraction $\dfrac{231}{712}$ n'est pas simplifiable.
            }       
        \end{enumerate}                    
\end{corrige}

