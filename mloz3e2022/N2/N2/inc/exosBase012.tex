\begin{exercice*}
    On donne les décompositions en produits de facteurs premiers de $21$, $35$, $42$ et $75$.

    $21=3\times 7$ ; $35=5\times 7$ ; $42=2\times 3\times 7$ et $75=3\times 5^2$.

    Déterminer le plus petit multiple commun :
    \begin{enumerate}
        \setlength{\columnseprule}{0pt}
        \begin{multicols}{2}        
            \item de $21$ et $35$.
            \item de $21$ et $75$.
            \item de $35$ et $42$.
            \item de $42$ et $75$.
        \end{multicols}
    \end{enumerate}    
\end{exercice*}
\begin{corrige}
    %\setcounter{partie}{0} % Pour s'assurer que le compteur de \partie est à zéro dans les corrigés
    % \phantom{rrr}
    On donne les décompositions en produits de facteurs remiers de $21$, $35$, $42$ et $75$.

    $21=3\times 7$ ; $35=5\times 7$ ; $42=2\times 3\times 7$ et $75=3\times 5^2$.

    Pour déterminer le plus petit multiple commun à deux entiers à partir de leurs décompositions en produits de facteurs premiers,
    il suffit de prendre la réunion de leurs diviseurs premiers avec leur plus grande multiplicité.

    \begin{enumerate}
        \item Le plus petit multiple commun de $21$ et $35$ vaut $3\times 5\times 7 = 105$.
        \item Le plus petit multiple commun de $21$ et $75$ vaut $3\times 5^2\times 7 = 525$.
        \item Le plus petit multiple commun de $35$ et $42$ vaut $2\times 3\times 5\times 7 = 210$.
        \item Le plus petit multiple commun de $42$ et $75$ vaut $2\times 3\times 5^2\times 7 = \num{1050}$.
    \end{enumerate}    
\end{corrige}

