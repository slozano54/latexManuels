\begin{exercice*}
    \begin{enumerate}
        \item Décomposer $110$ en produit de facteurs premiers.
        \item Lister ses diviseurs premiers.
        \item Compléter le tableau ci-dessous.
        
        \smallskip
        % \renewcommand{\arraystretch}{1}
        $\begin{array}{|c|c|c|}
        \hline
        \times & \phantom{plusLarge}2^{0}\phantom{plusLarge} & \phantom{plusLarge}2^{1}\phantom{plusLarge}\\
        \hline
        5^{0}\times11^{0} &  & \\
        \hline
        5^{0}\times11^{1} &  & \\
        \hline
        5^{1}\times11^{0} &  & \\
        \hline
        5^{1}\times11^{1} &  & \\
        \hline
        \end{array}$
        % \renewcommand{\arraystretch}{1}
        \smallskip
        \item En déduire le nombre de diviseurs de $110$.
        \item Conjecturer la façon de déterminer le nombre de
        
        diviseurs de $110$ par le calcul.
        \item Lister les diviseurs de $110$ dans l'ordre croissant.
    \end{enumerate}
\end{exercice*}
\begin{corrige}
    %\setcounter{partie}{0} % Pour s'assurer que le compteur de \partie est à zéro dans les corrigés
    \phantom{rrr}
    \begin{multicols}2
        \begin{enumerate}
            \item Déterminer la décomposition en produit de facteurs premiers de $110$.        
            \Decomposition[All]{110}
            \item Lister ses diviseurs premiers.
            
            Les diviseurs premiers de $110$ sont donc $2$, $5$ et $11$.
            \item Compléter le tableau ci-dessous.
            
            \smallskip
            $\renewcommand{\arraystretch}{1}
            \begin{array}{|c|c|c|}
                \hline
                \times & 2^{0} & 2^{1}\\
                \hline
                5^{0}\times11^{0} & 5^{0}\times11^{0}\times2^{0}=\mathbf{1} & 5^{0}\times11^{0}\times2^{1}=\mathbf{2}\\
                \hline
                5^{0}\times11^{1} & 5^{0}\times11^{1}\times2^{0}=\mathbf{11} & 5^{0}\times11^{1}\times2^{1}=\mathbf{22}\\
                \hline
                5^{1}\times11^{0} & 5^{1}\times11^{0}\times2^{0}=\mathbf{5} & 5^{1}\times11^{0}\times2^{1}=\mathbf{10}\\
                \hline
                5^{1}\times11^{1} & 5^{1}\times11^{1}\times2^{0}=\mathbf{55} & 5^{1}\times11^{1}\times2^{1}=\mathbf{110}\\
                \hline
            \end{array}
            \renewcommand{\arraystretch}{1}$
            \smallskip

            \item En déduire le nombre de diviseurs de $110$.
            
            $110$ a donc $8$ diviseurs.

            \item Conjecturer la façon de déterminer le nombre de diviseurs de $110$ par le calcul.

            $110$ a $(1+1)\times(1+1)\times(1+1) = 2\times2\times2 = 8$ diviseurs.

            Par exemple, dans la décomposition, le facteur $2$ apparaît avec la multiplicité $1$, donc sous les formes : $2^{0}$ ou $2^{1}$ d'où le facteur $(1+1)$.       
            \item Dresser la liste des diviseurs de $110$ dans l'ordre croissant : les $8$ diviseurs de $110$ issus du tableau ci-dessus : $1\text{ ; }2\text{ ; }5\text{ ; }10\text{ ; }11\text{ ; }22\text{ ; }55\text{ ; }110.$
        \end{enumerate}
    \end{multicols}
\end{corrige}

