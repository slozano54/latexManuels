\begin{exercice*}
    Utiliser les critères de divisibilité pour simplifier les fractions suivantes.
    \begin{enumerate}
        % \begin{spacing}{2}
            \setlength{\columnseprule}{0pt}
            \begin{multicols}{4}
                \item $\dfrac{105}{75}$
                \item $\dfrac{225}{45}$
                \item $\dfrac{375}{175}$
                \item $\dfrac{126}{42}$
            \end{multicols}
        % \end{spacing}
    \end{enumerate}    
    \hrefMathalea{https://coopmaths.fr/mathalea.html?ex=4C24,s=1,n=5,i=0&v=l}
    
    \hrefMathalea{https://coopmaths.fr/mathalea.html?ex=5N13,s=9,s2=false,n=10,i=0&v=l}
\end{exercice*}
\begin{corrige}
    %\setcounter{partie}{0} % Pour s'assurer que le compteur de \partie est à zéro dans les corrigés
    % \phantom{rrr}
    Utiliser les critères de divisibilité pour simplifier les fractions suivantes.
    
    \begin{enumerate}
        \item $\dfrac{105}{75}=\dfrac{\cancel{3}\times\cancel{5}\times7}{\cancel{3}\times\cancel{5}\times5}=\dfrac{7}{5}$
        \item $\dfrac{225}{45}=\dfrac{\cancel{3}\times\cancel{3}\times\cancel{5}\times5}{\cancel{3}\times\cancel{3}\times\cancel{5}}=5$
        \item $\dfrac{375}{175}=\dfrac{3\times\cancel{5}\times\cancel{5}\times5}{\cancel{5}\times\cancel{5}\times7}=\dfrac{15}{7}$
        \item $\dfrac{126}{42}=\dfrac{\cancel{2}\times\cancel{3}\times3\times\cancel{7}}{\cancel{2}\times\cancel{3}\times\cancel{7}}=3$
    \end{enumerate}
\end{corrige}

