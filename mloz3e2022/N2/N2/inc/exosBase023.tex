\begin{exercice*}[Fleuriste]
    Un fleuriste dispose de $\num{1756}$ tulipes et de $\num{1317}$ \oe illets.

    Il veut, en utilisant toutes ses fleurs, réaliser un maximum de bouquets contenant tous le même nombre de tulipes et le même nombre d'\oe illets. 
    \begin{enumerate}
        \item Quel est le nombre maximal de bouquets ?
        \item Quel est le nombre de tulipes dans chaque bouquet ?
        \item Quel est le nombre d'\oe illets dans chaque bouquet ?
    \end{enumerate}
\end{exercice*}
\begin{corrige}
    %\setcounter{partie}{0} % Pour s'assurer que le compteur de \partie est à zéro dans les corrigés
    % \phantom{rrr} 
    Un fleuriste dispose de $\num{1756}$ tulipes et de $\num{1317}$ \oe illets.

    Il veut, en utilisant toutes ses fleurs, réaliser un maximum de bouquets contenant tous le même nombre de tulipes et le même nombre d'\oe illets. 
    
    \begin{enumerate}
        \item Quel est le nombre maximal de bouquets ?

        {\color{red} $\num{1756} = 2^2\times 439$ et $\num{1317} = 3\times 439$. 
            
            $439$ est donc le plus grand nombre divisant à la fois $\num{1756}$ et $\num{1317}$. 

            Le nombre maximal de bouquet est donc $439$.
        }
        \item Quel est le nombre de tulipes dans chaque bouquet ?
     
        {\color{red} Le nombre de tulipes dans chaque bouquet est : $4$}
        \item Quel est le nombre d'\oe illets dans chaque bouquet ?
    
        {\color{red} Le nombre d'\oe illets dans chaque bouquet est : $3$}
    \end{enumerate}       
\end{corrige}

