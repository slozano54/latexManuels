\begin{exercice*}
    Déterminer les fonctions affines $f$, $g$ et $h$ telles que :
    \begin{enumerate}
        \item $f(1)=4$ et $f(4)=7$.
        \item $g(2)=-1$ et $g(-1)=2$.
        \item $h(9)=-1$ et $h(18)=-8$.
    \end{enumerate}
\end{exercice*}
\begin{corrige}
    Déterminer les fonctions affines $f$, $g$ et $h$ telles que :
    \begin{enumerate}
        \item $f(1)=4$ et $f(4)=7$.
        
        {\color{red} \FonctionAffine[Retrouve]{4}{7}{1}{4}}
    \end{enumerate}
    \Coupe
    \begin{enumerate}
        \setcounter{enumi}{1}
        \item $g(2)=-1$ et $g(-1)=2$.
        
        {\color{red} %\FonctionAffine[Retrouve,Nom=g]{2}{-1}{-1}{2}
        De même que précédemment, $a=-1$ et $b=1$ donc $g:x\longmapsto -x+1$}
        \item $h(9)=-1$ et $h(18)=-8$.
        
        {\color{red} %\FonctionAffine[Retrouve,Nom=h]{18}{-8}{9}{-1}
        De même que précédemment, $a=-\dfrac79$ et $b=6$ donc $h:x\longmapsto -\dfrac79 x +6$
        }
    \end{enumerate}
\end{corrige}
