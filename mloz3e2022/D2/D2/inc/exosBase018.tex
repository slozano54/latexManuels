\begin{exercice*}
    On veut tracer la représentation graphique $(d_f)$ de la fonction $f : x\longmapsto 3x + 3$.
    \begin{enumerate}
        \item Indiquer les coordonnées du point A de $(d_f)$ d’abscisse 0.\hfill \pointilles[30mm]
        \item Indiquer le nom deson ordonnée. \pointilles
        \item Placer le point A dans le repère ci-dessous.
        \item En utilisant le coefficient de la fonction $f$, placer un deuxième point B de $(d_f)$.
        \item Indiquer ses coordonnées.\hfill\pointilles[30mm]
        \item Tracer la courbe $(d_f)$ representative de $f$.
        
        \scalebox{1}{%
            \Fonction[%
                Trace,%
                Calcul=0,%
                Ymin=-3.5,Ymax=3.5,Ystep=2,%
                Xmin=-2.5,Xmax=2.5,Xstep=1,%
                Origine={(2.5,3.5)},%
                PasGrilleX=0.5,PasGrilleY=0.5,%
                Grille,%
                Traces={%
                    dotlabel.bot(btex \num{1} etex,placepoint(1,0));
                    dotlabel.lft(btex \num{1} etex,placepoint(0,1));
                }
            ]{}
        }
        \item Tracer les courbes $(d_g)$ et $(d_h)$ des fonctions $g$ et $h$ définies par $g(x)=3x$ et $h(x)=3x-4$.
        \item Faire une remarque justifiée.
        
        \pointilles

        \pointilles

        \pointilles
        \item Placer les points $F$, $G$ et $H$ d'abscisse \num{-1} appartenant respectivement à $(d_f)$, $(d_g)$ et $(d_h)$.
        \item Donner les coordonnées de ces points.
        
        \pointilles
    \end{enumerate}
\end{exercice*}
\begin{corrige}
    On veut tracer la représentation graphique $(d_f)$ de la fonction $f : x\longmapsto 3x + 3$.

    \begin{enumerate}
        \item Indiquer les coordonnées du point A de $(d_f)$ d’abscisse 0.\hfill\textcolor{red}{$A(0;3)$}
        \item Indiquer le nom deson ordonnée.
        
        \textcolor{red}{Ordonnée à l'origine.}
        \item Placer le point A dans le repère ci-dessous.
        \item En utilisant le coefficient de la fonction $f$, placer un deuxième point B de $(d_f)$.
        \item Indiquer ses coordonnées.\hfill\textcolor{red}{$B(1;6)$}
        \item Tracer la courbe $(d_f)$ representative de $f$.
        
        \scalebox{1}{%
            \Fonction[%
                Trace,%
                Calcul={3x+3,3x,3x-4},%
                CouleurTrace={red,red,red},%
                Bornea={-3,-3,-3},% les bornes a de l'intervalle [a,b] de tracé
                Borneb={3 ,3 ,3 },% les bornes b de l'intervalle [a,b] de tracé
                LabelC={0.1,0.15,0.9},% les positions du label des courbes
                NomCourbe={\textcolor{red}{$(d_f)$},\textcolor{red}{$(d_g)$},\textcolor{red}{$(d_h)$}},% les noms des représentations graphiques.
                Ymin=-3.5,Ymax=3.5,Ystep=2,%
                Xmin=-2.5,Xmax=2.5,Xstep=1,%
                Origine={(2.5,3.5)},%
                PasGrilleX=0.5,PasGrilleY=0.5,%
                Grille,%
                Traces={%
                    dotlabel.bot(btex \num{1} etex,placepoint(1,0));
                    dotlabel.lft(btex \num{1} etex,placepoint(0,1));
                    drawoptions(withcolor red);
                    dotlabel.rt(btex $B$ etex,placepoint(1,6));
                    dotlabel.lrt(btex $A$ etex,placepoint(0,3));
                    dotlabel.ulft(btex $F$ etex,placepoint(-1,0));
                    dotlabel.ulft(btex $G$ etex,placepoint(-1,-3));
                    dotlabel.ulft(btex $H$ etex,placepoint(-1,-7));
                    drawarrow placepoint(0,3)--placepoint(1,3) dashed evenly;
                    drawarrow placepoint(1,3)--placepoint(1,6) dashed evenly;
                }
            ]{}
        }
        \item Tracer les courbes $(d_g)$ et $(d_h)$ des fonctions $g$ et $h$ définies par $g(x)=3x$ et $h(x)=3x-4$.
        \item Faire une remarque justifiée.
        
        \textcolor{red}{Les trois droites ont le même coefficient directeur, elles sont donc parallèles.}
        \item Placer les points $F$, $G$ et $H$ d'abscisse \num{-1} appartenant respectivement à $(d_f)$, $(d_g)$ et $(d_h)$.
        \item Donner les coordonnées de ces points.
        
        \textcolor{red}{$F(-1;0)$ $G(-1;-3)$ $H(-1;-7)$}
    \end{enumerate}
\end{corrige}
