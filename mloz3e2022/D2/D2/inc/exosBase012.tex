\begin{exercice*}
    Les droites $(d_1)$, $(d_2)$, $(d_3)$ et $(d_4)$ sont les représentations graphiques respectives de quatre fonctions linéaires $f_1$, $f_2$, $f_3$ et $f_4$.

    \hspace*{-5mm}
    \scalebox{0.8}{
    \Fonction[%
        Trace,
        Calcul={3x,-1.5x,-0.5x,0.4x},% les fonctions.
        Bornea={-1,-2.8,-5,-5},% les bornes a de l'intervalle [a,b] de tracé
        Borneb={1.5,2,6,6},% les bornes b de l'intervalle [a,b] de tracé
        LabelC={0.1,0.1,0.1,0.1},% les positions du label des courbes
        NomCourbe={$(d_1)$,$(d_2)$,$(d_3)$,$(d_4)$},% les noms des représentations graphiques.
        Xmin=-4.5,Xmax=4.5,
        Ymin=-2,Ymax=3,
        CouleurTrace={bleu,Crimson,Gray,DarkGreen},
        Origine={(4,2)},
        Grille,PasGrilleX=1,PasGrilleY=1,
        Graduations,
        Traces={%
            draw placepoint(0,-1.5)--placepoint(1,-1.5) dashed evenly;
            draw placepoint(1,0)--placepoint(1,-1.5) dashed evenly;
            label.lrt(btex $A_1$ etex,placepoint(1,3));
            label.llft(btex $A_2$ etex,placepoint(1,-1.5));
            label.bot(btex $A_3$ etex,placepoint(2,-1));
            label.bot(btex $A_4$ etex,placepoint(5,2));
            label.lft(btex \num{-1.5} etex,placepoint(0,-1.5));
            marque_p:="plein";
            pointe(placepoint(1,3));
            pointe(placepoint(1,-1.5));
            pointe(placepoint(2,-1));
            pointe(placepoint(5,2));
        }
    ]{}
    }
    \begin{enumerate}
        \item Déterminer les coordonnées de $A_1$, $A_2$, $A_3$ et $A_4$.
        \item En déduire quatre égalités avec $f_1$, $f_2$, $f_3$ et $f_4$.
        \item En déduire le coefficient de $f_1$, $f_2$, $f_3$ et $f_4$.
        
        {\renewcommand{\arraystretch}{1.8}
        \[\begin{array}{|>{\columncolor{gray!15}}>{\centering\arraybackslash}m{15mm}|*{4}{>{\centering\arraybackslash}m{10mm}|}}%
            \hline
            \text{Fonction}   &$f_1$&$f_2$&$f_3$&$f_4$\\\hline
            \text{Coefficient}&&&&\\\hline
        \end{array}
        \]        
        }
        \item En déduire l'expression de chaque fonction.
    \end{enumerate}
\end{exercice*}
\begin{corrige}
    Les droites $(d_1)$, $(d_2)$, $(d_3)$ et $(d_4)$ sont les représentations graphiques respectives de quatre fonctions linéaires $f_1$, $f_2$, $f_3$ et $f_4$.

    \scalebox{0.8}{
    \Fonction[%
        Trace,
        Calcul={3x,-1.5x,-0.5x,0.4x},% les fonctions.
        Bornea={-1,-2.8,-5,-5},% les bornes a de l'intervalle [a,b] de tracé
        Borneb={1.5,2,6,6},% les bornes b de l'intervalle [a,b] de tracé
        LabelC={0.1,0.1,0.1,0.1},% les positions du label des courbes
        NomCourbe={$(d_1)$,$(d_2)$,$(d_3)$,$(d_4)$},% les noms des représentations graphiques.
        Xmin=-4.5,Xmax=4.5,
        Ymin=-2,Ymax=3,
        CouleurTrace={bleu,Crimson,Gray,DarkGreen},
        Origine={(4,2)},
        Grille,PasGrilleX=1,PasGrilleY=1,
        Graduations,
        Traces={%
            draw placepoint(0,-1.5)--placepoint(1,-1.5) dashed evenly;
            draw placepoint(1,0)--placepoint(1,-1.5) dashed evenly;
            label.lrt(btex $A_1$ etex,placepoint(1,3));
            label.llft(btex $A_2$ etex,placepoint(1,-1.5));
            label.bot(btex $A_3$ etex,placepoint(2,-1));
            label.bot(btex $A_4$ etex,placepoint(5,2));
            label.lft(btex \num{-1.5} etex,placepoint(0,-1.5));
            marque_p:="plein";
            pointe(placepoint(1,3));
            pointe(placepoint(1,-1.5));
            pointe(placepoint(2,-1));
            pointe(placepoint(5,2));
        }
    ]{}
    }
    \begin{enumerate}
        \item Déterminer les coordonnées de $A_1$, $A_2$, $A_3$ et $A_4$.
        
        \textcolor{red}{$A_1 (1;3)$ ; $A_2 (1;\num{-1.5})$ ; $A_3 (2;-1)$ ; $A_4 (5;2)$}
        \item En déduire quatre égalités avec $f_1$, $f_2$, $f_3$ et $f_4$.
        
        \textcolor{red}{$f_1(1)=3$ ; $f_2(1)=\num{-1.5}$ ; $f_3(2)=-1$ ; $f_4(5)=2$}
        \item En déduire le coefficient de $f_1$, $f_2$, $f_3$ et $f_4$.
        
        {\renewcommand{\arraystretch}{1.8}
        \[\begin{array}{|>{\columncolor{gray!15}}>{\centering\arraybackslash}m{15mm}|*{4}{>{\centering\arraybackslash}m{10mm}|}}%
            \hline
            \text{Fonction}   &$f_1$&$f_2$&$f_3$&$f_4$\\\hline
            \text{Coefficient}&\textcolor{red}{3}&\textcolor{red}{\num{-1.5}}&$\textcolor{red}{-\dfrac{1}{2}}$&$\textcolor{red}{\dfrac{2}{5}}$\\\hline
        \end{array}
        \]        
        }
        \item En déduire l'expression de chaque fonction.
        
        \textcolor{red}{$f_1(x)=3x$ ; $f_2(x)=\num{-1.5}x$ ; $f_3(x)=-\dfrac{1}{2}x$ ; $f_4(x)=\dfrac{2}{5}x$}
    \end{enumerate}
\end{corrige}
