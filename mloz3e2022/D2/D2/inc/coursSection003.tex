\section{Fonction linéaire et pourcentage}
\begin{propriete}[\admise]
    Prendre $t\%$ d'un nombre, c'est multiplier ce nombre par $\dfrac{t}{100}$.

    Augmenter un nombre de $t\%$, c'est multiplier ce nombre par $1+\dfrac{t}{100}$.

    Diminuer un nombre de $t\%$, c'est multiplier ce nombre par $1-\dfrac{t}{100}$.
\end{propriete}
\begin{exemple*1}
    15\% de $x$ : $x\times\dfrac{15}{100}$. On lui associe la fonction linéaire $x\mapsto0,15\times x$.

    Diminuer $x$ de 12\% : $x\times\left(1-\dfrac{12}{100}\right)=x\times0,88$. On lui associe la fonction linéaire $x\mapsto0,88\times x$.

    Augmenter $x$ de 3\% : $x\times\left(1+\dfrac3{100}\right)=x\times1,03$. On lui associe la fonction linéaire $x\mapsto1,03\times x$.
\end{exemple*1}
