\section{Fonctions affines : Définition(s)}
\begin{definition}
    Soit $a$ et $b$ deux nombres quelconques \og{}fixes\fg{}.
    
    Définir une fonction affine, c'est associer à chaque nombre $x$, le nombre $ax+b$.
    
    On dit que $ax+b$ est l'image de $x$.
\end{definition}
\begin{definition}[Vocabulaire \& Notation]
% \definNumTitre{Notation}{
    La fonction affine qui à $x$ associe $ax+b$ se note $x\mapsto ax+b$ et se lit \og{}à $x$, on associe $ax+b$\fg{}.
    
    On peut également la noter par une lettre, $f$ par exemple. Dans ce cas, l'image de $x$ se note $f(x)$ et on lit \og{}$f$ de $x$\fg{}.
\end{definition}
\begin{exemples*1}
    $f$ est la fonction affine qui à $x$ associe $2x-1$.
    
    L'image de 3 est $2\times3-1=5$ et on note $f(3)=5$.
    
    \textbf{Cas particuliers :}
    \begin{itemize}
        \item \underline{$b=0$} On obtient $f:x\mapsto ax$, c'est à dire une fonction linéaire.
        \item \underline{$a=0$} On obtient $f:x\mapsto b$, c'est à dire une fonction {\bf constante}.
    \end{itemize}
\end{exemples*1}
\begin{definition}[Vocabulaire \& Notation]
    On dit que $x\mapsto ax$ est la fontion linéaire associée à la fonction affine $x\mapsto ax+b$.
\end{definition}
