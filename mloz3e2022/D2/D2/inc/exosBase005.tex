\begin{exercice*}
    En France, on utilise le degré Celsius (\Temp{}) pour mesurer la température alors qu'aux États-Unis, on utilise le degré Fahrenheit (\Temp[F]{}).
    Pour passer des \Temp{} aux \Temp[F]{}, on multiplie le nombre de départ par \num{1.8} puis on ajoute \num{32}.

    On note $x$ la température en degrés Celsius et $f(x)$ la même température en degrés Fahrenheit.
    \begin{enumerate}
        \item Exprimer $f(x)$ en fonction de $x$.
        \item Dire de quel type de fonction il s'agit.
        \item Calculer $f(10)$ et $f(-30)$.
        \item Déterminer l'antécédent de 41 par la fonction $f$.
    \end{enumerate}
\end{exercice*}
\begin{corrige}
    En France, on utilise le degré Celsius (\Temp{}) pour mesurer la température alors qu'aux États-Unis, on utilise le degré Fahrenheit (\Temp[F]{}).
    Pour passer des \Temp{} aux \Temp[F]{}, on multiplie le nombre de départ par \num{1.8} puis on ajoute \num{32}.

    On note $x$ la température en degrés Celsius et $f(x)$ la même température en degrés Fahrenheit.
    \begin{enumerate}
        \item Exprimer $f(x)$ en fonction de $x$.
        
        \textcolor{red}{$f(x)=x\times\num{1.8}+32=\num{1.8}x+32$}
        \item Dire de quel type de fonction il s'agit.
        
        \textcolor{red}{C'est une fonction affine.}
        \item Calculer $f(10)$ et $f(-30)$.
        
        \textcolor{red}{$f(10)=\num{1.8}\times 10 + 32 = 18 + 32 = 50$}
        
        \textcolor{red}{$f(-30)=\num{1.8}\times (-30) + 32 = -54 + 32 = -22$}
        \item Déterminer l'antécédent de 41 par la fonction $f$.
        
        \textcolor{red}{On cherche $x$ tel que $f(x)=41$ soit $\num{1.8}x+32=41$ soit $\num{1.8}x=41-32=9$ soit $x=\dfrac{9}{\num{1.8}}=5$}
    \end{enumerate}
\end{corrige}
