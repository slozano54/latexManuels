\begin{exercice*}
    TRAP est un trapèze, rectangle en A et en P, tel que :

    $TP=\Lg[cm]{3}$ ; $PA=\Lg[cm]{5}$ et $AR=\Lg[cm]{4}$.
    
    $M$ est un point variable du segment $[PA]$ et on note $x$ la longueur du segment $[PM]$ en \Lg[cm]{}.

    \begin{Geometrie}
        pair R,A,T,P,M;
        P=u*(1,1);
        A-P=u*(5,0);
        T-P=u*(0,3);
        R-P=u*(5,4);
        M-P=u*(1.5,0);
        trace polygone(R,A,P,T);
        trace chemin(T,M,R);
        trace cotationmil(P,A,-7mm,15,btex \Lg[cm]{5} etex);
        trace cotationmil(P,M,-5mm,10,btex $x$ etex);
        trace cotationmil(P,T,5mm,15,btex \Lg[cm]{3} etex);
        trace cotationmil(A,R,-5mm,15,btex \Lg[cm]{4} etex);
        label.llft(btex $P$ etex,P);
        label.bot(btex $M$ etex,M);
        label.lrt(btex $A$ etex,A);
        label.urt(btex $R$ etex,R);
        label.ulft(btex $T$ etex,T);        
    \end{Geometrie}

    \begin{enumerate}
        \item Donner les valeurs entre lesquelles $x$ peut varier.
        \item Montrer que l’aire du triangle $PTM$ est $\num{1,5}x$ et que l’aire du triangle $ARM$ est $10-2x$.
        
        \medskip
        Cette droite est la représentation graphique de la fonction qui, à $x$ , associe l’aire du triangle $ARM$.

        \smallskip
        \hspace*{-5mm}
        \scalebox{0.7}{%
            \Fonction[%
                Trace,%
                Calcul=10/8-2x/5,%
                CouleurTrace=blue,%
                Bornea=0,% les bornes a de l'intervalle [a,b] de tracé
                Borneb=3.125,% les bornes b de l'intervalle [a,b] de tracé
                LabelC=0.8,%
                NomCourbe=\textcolor{blue}{Aire du triangle $ARM$},%
                Xmin=0,Xmax=13,Xstep=0.25,%
                Ymin=0,Ymax=6,Ystep=0.25,%
                Origine={(0.25,0.25)},%
                PasGrilleX=0.25,%
                PasGrilleY=0.25,%
                Grille,%
                LabelX={\textcolor{black}{\footnotesize\textbf{Longueur de $x$ en \Lg[cm]{}}}},%
                LabelY={\textcolor{black}{\footnotesize\textbf{Aire en \Aire[cm]{}}}},%
                Traces={%
                    dotlabel.bot(btex \num{0.2} etex,placepoint(0.125,0));                    
                    dotlabel.bot(btex \num{1} etex,placepoint(0.625,0));
                    dotlabel.bot(btex \num{2} etex,placepoint(1.25,0));
                    dotlabel.bot(btex \num{3} etex,placepoint(1.875,0));
                    dotlabel.bot(btex \num{4} etex,placepoint(2.5,0));
                    dotlabel.bot(btex \num{5} etex,placepoint(3.125,0));
                    %
                    dotlabel.lft(btex \num{1} etex,placepoint(0,0.125));
                    dotlabel.lft(btex \num{5} etex,placepoint(0,0.625));
                    dotlabel.lft(btex \num{10} etex,placepoint(0,1.25));
                }
            ]{}
        }

        Répondre aux questions \theenumi), \setcounter{enumi}{3}\theenumi), \setcounter{enumi}{5}\theenumi)~en utilisant ce
        graphique. Laisser apparents les traits nécessaires.
        \setcounter{enumi}{2}
        \item Déterminer la valeur de $x$ pour laquelle l’aire du triangle $ARM$ est égale à \Aire[cm]{6}.
        \item Déterminer la valeur l’aire du triangle $ARM$ lorsque $x$ est égal à \Lg[cm]{4}.
        \item Sur ce graphique, tracer la droite représentant la fonction $x\longmapsto \num{1,5}x$.
        \item Estimer, à un millimètre près, la valeur de $x$ pour laquelle les triangles $PTM$ et $ARM$ ont la même aire.
        \item Montrer par le calcul que la valeur exacte de $x$ pour laquelle les deux aires sont égales est $\dfrac{100}{35}$.
    \end{enumerate}
\end{exercice*}
\begin{corrige}
    TRAP est un trapèze, rectangle en A et en P, tel que :
    $TP=\Lg[cm]{3}$ ; $PA=\Lg[cm]{5}$ et $AR=\Lg[cm]{4}$.
    
    $M$ est un point variable du segment $[PA]$ et on note $x$ la longueur du segment $[PM]$ en \Lg[cm]{}.
    \scalebox{0.7}{
    \begin{Geometrie}
        pair R,A,T,P,M;
        P=u*(1,1);
        A-P=u*(5,0);
        T-P=u*(0,3);
        R-P=u*(5,4);
        M-P=u*(1.5,0);
        trace polygone(R,A,P,T);
        trace chemin(T,M,R);
        trace cotationmil(P,A,-7mm,15,btex \Lg[cm]{5} etex);
        trace cotationmil(P,M,-5mm,10,btex $x$ etex);
        trace cotationmil(P,T,5mm,15,btex \Lg[cm]{3} etex);
        trace cotationmil(A,R,-5mm,15,btex \Lg[cm]{4} etex);
        label.llft(btex $P$ etex,P);
        label.bot(btex $M$ etex,M);
        label.lrt(btex $A$ etex,A);
        label.urt(btex $R$ etex,R);
        label.ulft(btex $T$ etex,T);        
    \end{Geometrie}
    }

    \begin{enumerate}
        \item Donner les valeurs entre lesquelles $x$ peut varier.
        
        \textcolor{red}{$x$ peut varier entre $0$ et $5$.}
        \item Montrer que l’aire du triangle $PTM$ est $\num{1,5}x$ et que l’aire du triangle $ARM$ est $10-2x$.
        
        {\color{red}%
        $\mathcal{A}_{PTM}=\dfrac12 \times PT \times PM=\dfrac12 \times 3 \times x=\num{1.5} x$.

        $MA = 5-x$ donc $\mathcal{A}_{ARM}=\dfrac12 \times AR \times AM$

        $\mathcal{A}_{ARM}=\dfrac12 \times 4 \times (5-x) = 2\times (5-x)= 10-2x$
        }
        Cette droite est la représentation graphique de la fonction qui, à $x$ , associe l’aire du triangle $ARM$.

        \smallskip
        \hspace*{-10mm}
        \scalebox{0.6}{%
            \Fonction[%
                Trace,%
                Calcul={10/8-2x/5,1.5x/5},%
                CouleurTrace={blue,red},%
                Bornea={0    ,0    },% les bornes a de l'intervalle [a,b] de tracé
                Borneb={3.125,3.125},% les bornes b de l'intervalle [a,b] de tracé
                LabelC={0.8,0.8},%
                NomCourbe={\textcolor{blue}{Aire du triangle $ARM$},\textcolor{white}{r}},%
                Xmin=0,Xmax=13,Xstep=0.25,%
                Ymin=0,Ymax=6,Ystep=0.25,%
                Origine={(0.25,0.25)},%
                PasGrilleX=0.25,%
                PasGrilleY=0.25,%
                Grille,%
                LabelX={\textcolor{black}{\footnotesize\textbf{Longueur de $x$ en \Lg[cm]{}}}},%
                LabelY={\textcolor{black}{\footnotesize\textbf{Aire en \Aire[cm]{}}}},%
                Traces={%
                    dotlabel.bot(btex \num{0.2} etex,placepoint(0.125,0));                    
                    dotlabel.bot(btex \num{1} etex,placepoint(0.625,0));
                    dotlabel.bot(btex \num{2} etex,placepoint(1.25,0));
                    dotlabel.bot(btex \num{3} etex,placepoint(1.875,0));
                    dotlabel.bot(btex \num{4} etex,placepoint(2.5,0));
                    dotlabel.bot(btex \num{5} etex,placepoint(3.125,0));
                    %
                    dotlabel.lft(btex \num{1} etex,placepoint(0,0.125));
                    dotlabel.lft(btex \num{5} etex,placepoint(0,0.625));
                    dotlabel.lft(btex \num{10} etex,placepoint(0,1.25));
                    %
                    drawoptions(withcolor red);
                    drawarrow placepoint(0,0.75)--placepoint(1.25,0.75) dashed evenly;
                    drawarrow placepoint(1.25,0.75)--placepoint(1.25,0) dashed evenly;
                    drawarrow placepoint(0,0.54)--placepoint(1.79,0.54) dashed evenly;
                    drawarrow placepoint(1.79,0.54)--placepoint(1.79,0) dashed evenly;
                    drawarrow placepoint(2.5,0)--placepoint(2.5,0.25) dashed evenly;
                    drawarrow placepoint(2.5,0.25)--placepoint(0,0.25) dashed evenly;
                }
            ]{}
        }
    \end{enumerate}
    \Coupe
    \begin{enumerate}
        \setcounter{enumi}{2}
        Répondre aux questions \theenumi), \setcounter{enumi}{3}\theenumi), \setcounter{enumi}{5}\theenumi)~en utilisant ce
        graphique. Laisser apparents les traits nécessaires.
        \setcounter{enumi}{2}
        \item Déterminer la valeur de $x$ pour laquelle l’aire du triangle $ARM$ est égale à \Aire[cm]{6}.
        
        \textcolor{red}{Pour $x=\Lg[cm]{2}$}
        \item Déterminer la valeur l’aire du triangle $ARM$ lorsque $x$ est égal à \Lg[cm]{4}.
        
        \textcolor{red}{Si $x=\Lg[cm]{4}$ alors l'aire du triangle vaut \Aire[cm]{2}.}
        \item Sur ce graphique, tracer la droite représentant la fonction $x\longmapsto \num{1,5}x$.
        \item Estimer, à un millimètre près, la valeur de $x$ pour laquelle les triangles $PTM$ et $ARM$ ont la même aire.
        
        \textcolor{red}{Pour $x\approx\Lg[cm]{2.8}$}
        \item Montrer par le calcul que la valeur exacte de $x$ pour laquelle les deux aires sont égales est $\dfrac{100}{35}$.
        
        {\color{red} $10-2x=\num{1.5} x$ donc $10=2x +\num{1.5} x$ donc $10=\num{3.5} x$ soit $x=\dfrac{10}{\num{3.5}}=\dfrac{100}{35}$}
    \end{enumerate}
\end{corrige}
