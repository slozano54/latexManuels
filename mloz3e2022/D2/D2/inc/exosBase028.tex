\begin{exercice*}
    Introduit en Australie en 1935 pour lutter contre les insectes rongeant la canne à sucre, le crapaud buffle, qui est venimeux, ravage désormais la faune locale.
    \begin{enumerate}
        \item La taille des 100 spécimens introduits à l'origine était au maximum de \Lg[cm]{14}, mais un spécimen de \Lg[cm]{38} a été capturé en 2007.
        Déterminer le pourcentage d'augmentation de sa taille.
        \item Une estimation donne la population actuelle de crapauds buffles en Australie de l'ordre de 200 millions d'individus.
        Déterminer le pourcentage d'augmentation de leur nombre par rapport à 1935.
    \end{enumerate}
\end{exercice*}
\begin{corrige}
    Introduit en Australie en 1935 pour lutter contre les insectes rongeant la canne à sucre, le crapaud buffle, qui est venimeux, ravage désormais la faune locale.

    \begin{enumerate}
        \item La taille des 100 spécimens introduits à l'origine était au maximum de \Lg[cm]{14}, mais un spécimen de \Lg[cm]{38} a été capturé en 2007.
        Déterminer le pourcentage d'augmentation de sa taille.

        {\color{red} $\Lg[cm]{38}-\Lg[cm]{14}=\Lg[cm]{24}$ et $\Lg[cm]{24}\div\Lg[cm]{14}\approx\num{1.71}$ donc sa taille a augmenté de 171\%.}
    \end{enumerate}
    \Coupe
    \begin{enumerate}
        \setcounter{enumi}{1}
        \item Une estimation donne la population actuelle de crapauds buffles en Australie de l'ordre de 200 millions d'individus.
        Déterminer le pourcentage d'augmentation de leur nombre par rapport à 1935.

        {\color{red} $\num{200 000 000}-100=\num{199 999 900}$, leur nombre a donc augmenté de \num{199 999 900}\% \num{199 999 900} de plus pour 100 à l'origine!}
    \end{enumerate}
\end{corrige}
