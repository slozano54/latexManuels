\begin{exercice*}
    Soit la fonction $f:x\longmapsto -3x+7$
    \begin{enumerate}
        \item Calculer $f(8)$.
        \item Calculer l'image de $0$.
        \item Calculer l'antécédent de $2$.
    \end{enumerate}
\end{exercice*}
\begin{corrige}
    Soit la fonction $f:x\longmapsto -3x+7$
    
    \begin{enumerate}
        \item Calculer $f(8)$.
        
        \textcolor{red}{$f(8)=-3\times 8 + 7=-24+7=-17$.}
        \item Calculer l'image de $0$.
        
        \textcolor{red}{$f(0)=-3\times 0 + 7=0+7=7$.}
        \item Calculer l'antécédent de $2$.
        
        \textcolor{red}{$f(x)=2$ donc $-3x + 7=2$ soit $-3x=-5$ d'où $x=\dfrac{-5}{-3}=\dfrac53$.}

        \textcolor{red}{L'antécédent de $2$ vaut $\dfrac53$.}
    \end{enumerate}
\end{corrige}
