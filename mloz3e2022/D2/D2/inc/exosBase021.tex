\begin{exercice*}
    Soient $f$ et $g$ deux fonctions affines telles que : 
    
    $f(0)=2$ et $f(4)=18$ ; $g(0)=-1$ et $g(4)=13$.
    \begin{enumerate}
        \item Indiquer l'ordonnée à l'origine $(b_f)$ et $(b_g)$ de chaque fonction.
        \item Déterminer les fonctions f et g .
    \end{enumerate}    
\end{exercice*}
\begin{corrige}
    Soient $f$ et $g$ deux fonctions affines telles que $f(0)=2$ et $f(4)=18$ ; $g(0)=-1$ et $g(4)=13$.

    \begin{enumerate}
        \item Indiquer l'ordonnée à l'origine $(b_f)$ et $(b_g)$ de chaque fonction.
        
        \textcolor{red}{Les ordonnées à l'origine respectives sont les images de 0 donc $b_f=2$ et $b_g=-1$.}
        \item Déterminer les fonctions f et g .
        
        {\color{red}%
        En notant $a_f$ le coefficient directeur de $f$, $f(x)=a_f x+2$ d'où $f(4)=4a_f +2=-18$ d'où $a_f=-5$ donc $f(x)=-5x+2$.

        En notant $a_g$ le coefficient directeur de $g$, $g(x)=a_g x-1$ d'où $g(4)=4a_g -1=13$ d'où $a_g=\num{3.5}$ donc $g(x)=\num{3.5}x-1$.
        }
    \end{enumerate}
\end{corrige}
