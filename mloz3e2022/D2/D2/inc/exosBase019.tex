\begin{exercice*}
    On considère les fonctions suivantes.

    $f:x\longmapsto \dfrac23 x -1$ et $g:x\longmapsto -\dfrac13 x +2$

    On appelle $(d_f)$ et $(d_g)$ leur représentation graphique.

    \scalebox{1}{%
        \Fonction[%
            Trace,%
            Calcul=0,%
            Xmin=-3.666,Xmax=3.666,Xstep=1,%
            Ymin=-3,Ymax=3,Ystep=1,%
            Origine={(3.666,3)},%
            PasGrilleX=0.333,%
            PasGrilleY=0.333,%
            Grille,%
            Traces={%
                dotlabel.bot(btex \num{1} etex,placepoint(1,0));
                dotlabel.lft(btex \num{1} etex,placepoint(0,1));
            }
        ]{}
    }
    \begin{enumerate}
        \item Déterminer les coordonnées des points $F_0$ et $G_0$ d'abscisse $0$, respectivement sur $(d_f)$ et $(d_g)$.
        
        \medskip
        \pointilles
        \item Déterminer le coefficient de $f$ et de $g$.
        
        \medskip
        \pointilles
        \item En déduire les coordonnées des points $F_1$ et $G_1$ d'abscisse $1$, respectivement sur $(d_f)$ et $(d_g)$.
        
        \medskip
        \pointilles
        \item Justifier si ces deux points suffisent à tracer précisément chaque courbe ou non.
        
        \medskip
        \pointilles
        \item Déterminer les coordonnées des points $F_{-3}$ et $G_{-3}$ d'abscisse $-3$, respectivement sur $(d_f)$ et $(d_g)$.
        
        \medskip
        \pointilles
        \item Placer ces différents points puis tracer $(d_f)$ et $(d_g)$.
        \item Ces deux droites sont sécantes en un point $I$. Indiquer les coordonnées de ce point $I$.
        
        \medskip
        \pointilles
        \item Résoudre graphiquement l'équation $f(x)=g(x)$. Expliquer ce à quoi cela correspond graphiquement.
        
        \medskip
        \pointilles

        \medskip
        \pointilles

        \medskip
        \pointilles
    \end{enumerate}
\end{exercice*}
\begin{corrige}
    On considère les fonctions suivantes.

    $f:x\longmapsto \dfrac23 x -1$ et $g:x\longmapsto -\dfrac13 x +2$

    On appelle $(d_f)$ et $(d_g)$ leur représentation graphique.

    \scalebox{1}{%
        \Fonction[%
            Trace,%
            Calcul={2x/3-1,-x/3+2},%                
            CouleurTrace={red,red,red},%
            Bornea={-4,-4},% les bornes a de l'intervalle [a,b] de tracé
            Borneb={4 ,4 },% les bornes b de l'intervalle [a,b] de tracé
            Xmin=-3.666,Xmax=3.666,Xstep=1,%
            Ymin=-3,Ymax=3,Ystep=1,%
            Origine={(3.666,3)},%
            PasGrilleX=0.333,%
            PasGrilleY=0.333,%
            Grille,%
            Traces={%
                dotlabel.bot(btex \num{1} etex,placepoint(1,0));
                dotlabel.lft(btex \num{1} etex,placepoint(0,1));      
                drawoptions(withcolor red);
                dotlabel.urt(btex $G_0$ etex,placepoint(0,2));
                dotlabel.lrt(btex $F_0$ etex,placepoint(0,-1));
                dotlabel.urt(btex $G_1$ etex,placepoint(1,1+2/3));
                dotlabel.lrt(btex $F_1$ etex,placepoint(1,-1/3));
                dotlabel.llft(btex $G_{-3}$ etex,placepoint(-3,3));
                dotlabel.ulft(btex $F_{-3}$ etex,placepoint(-3,-3));
                dotlabel.top(btex $I$ etex,placepoint(3,1));
            }
        ]{}
    }
    \begin{enumerate}
        \item Déterminer les coordonnées des points $F_0$ et $G_0$ d'abscisse $0$, respectivement sur $(d_f)$ et $(d_g)$.
        
        \textcolor{red}{$F_0(0;-1)$ et $G_0(0;2)$}
        \item Déterminer le coefficient de $f$ et de $g$.
        
        \textcolor{red}{Pour $(d_f)$ : $\dfrac23$ et pour $(d_g)$ : $-\dfrac13$}
        \item En déduire les coordonnées des points $F_1$ et $G_1$ d'abscisse $1$, respectivement sur $(d_f)$ et $(d_g)$.
        
        \textcolor{red}{$F_1(1;-1+\dfrac23)$ soit $F_1(1;-\dfrac13)$ et $G_1(1;\dfrac53)$}
        \item Justifier si ces deux points suffisent à tracer précisément chaque courbe ou non.
        
        \textcolor{red}{Une droite est complètement connue grâce à deux de ses points, donc on pourra tracer précisément chaque courbe avec deux points sur chaque droite.}
        \item Déterminer les coordonnées des points $F_{-3}$ et $G_{-3}$ d'abscisse $-3$, respectivement sur $(d_f)$ et $(d_g)$.
        
        \textcolor{red}{$F_{-3}(-3;-3)$ et $G_{-3}(-3;3)$}
        \item Placer ces différents points puis tracer $(d_f)$ et $(d_g)$.
        \item Ces deux droites sont sécantes en un point $I$. Indiquer les coordonnées de ce point $I$.
        
        \textcolor{red}{$I(3;1)$}
        \item Résoudre graphiquement l'équation $f(x)=g(x)$. Expliquer ce à quoi cela correspond graphiquement.
        
        \textcolor{red}{$x=3$, il s'agit de l'abscisse du point d'intersection de $(d_f)$ et $(d_g)$.}
    \end{enumerate}
\end{corrige}
