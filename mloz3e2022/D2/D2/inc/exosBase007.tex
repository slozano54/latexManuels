\begin{exercice*}
    Dans chaque cas, indiquer en le justifiant si la fonction est affine.
    \begin{enumerate}
        \item La fonction qui, à un nombre, associe le résultat du programme de calcul suivant :
        
           \myProgCalcul{\textbullet}{Programme de calcul}{%
                \ProgCalcul[Enonce,ThemePerso]{%
                    Choisir un nombre.,%
                    Lui ajouter $1$.,%
                    Multiplier le tout par $3$.,%
                    Annoncer le résultat.%
                }
            } 
        \item La fonction par laquelle la longueur du rayon d'un cercle a pour image le périmètre de ce cercle.
        \item La fonction qui, à la longueur du rayon d'un disque, associe l'aire de ce disque.       
    \end{enumerate}
\end{exercice*}
\begin{corrige}
    Dans chaque cas, indiquer en le justifiant si la fonction est affine.

    \begin{enumerate}
        \item La fonction qui, à un nombre, associe le résultat du programme de calcul suivant :
    \end{enumerate}
           \myProgCalcul{\textbullet}{Programme de calcul}{%
                \ProgCalcul[Enonce,ThemePerso]{%
                    Choisir un nombre.,%
                    Lui ajouter $1$.,%
                    Multiplier le tout par $3$.,%
                    Annoncer le résultat.%
                }
            }
            
        \textcolor{red}{C'est la fonction $f(x)=3(x+1)=3x+3$, elle est donc affine.}  
                  
        \begin{enumerate}
            \setcounter{enumi}{1}
        \item La fonction par laquelle la longueur du rayon d'un cercle a pour image le périmètre de ce cercle.
        
        \textcolor{red}{Si on note $r$ le rayon du cercle, c'est la fonction $P(r)=2\pi r$, c'est une fonction linéaire de coefficient $2\pi$, elle est donc affine.}
        \item La fonction qui, à la longueur du rayon d'un disque, associe l'aire de ce disque.
        
        \textcolor{red}{Si on note $r$ le rayon du disque, c'est la fonction $A(r)=\pi r^2$, ce n'est donc pas une fonction affine.}
    \end{enumerate}
\end{corrige}
