\begin{exercice*}
    Les droites $(d_1)$, $(d_2)$ et $(d_3)$ sont les représentations graphiques respectives de quatre fonctions linéaires $f_1$, $f_2$ et $f_3$.

    \scalebox{1}{
    \Fonction[%
        Trace,
        Calcul={0.5x,-x+5,2x+1},% les fonctions.
        Bornea={-1.5,-0.5,-1},% les bornes a de l'intervalle [a,b] de tracé
        Borneb={7.5,6,2.3},% les bornes b de l'intervalle [a,b] de tracé
        LabelC={0.9,0.95,0.9},% les positions du label des courbes
        NomCourbe={$(d_1)$,$(d_2)$,$(d_3)$},% les noms des représentations graphiques.
        Xmin=-1,Xmax=7,
        Ymin=-0.5,Ymax=5,
        CouleurTrace={bleu,Crimson,Gray},
        Origine={(1,0.5)},
        Grille,PasGrilleX=0.5,PasGrilleY=0.5,        
        Graduations,
        Traces={%
            drawarrow placepoint(0,1)--placepoint(1,1) dashed evenly;
            drawarrow placepoint(1,1)--placepoint(1,3) dashed evenly;
            drawarrow placepoint(2,3)--placepoint(3,3) dashed evenly;
            drawarrow placepoint(3,3)--placepoint(3,2) dashed evenly;
            drawarrow placepoint(4,2)--placepoint(5,2) dashed evenly;
            drawarrow placepoint(5,2)--placepoint(5,2.5) dashed evenly;
        }
    ]{}
    }
    \begin{enumerate}
        \item Indiquer la(les) fonction(s) qui ont un coefficient négatif.        
        \item Indiquer le coefficient de $f_1$, $f_2$ et $f_3$.
        
        {\renewcommand{\arraystretch}{1.8}
        \[\begin{array}{|>{\columncolor{gray!15}}>{\centering\arraybackslash}m{15mm}|*{3}{>{\centering\arraybackslash}m{10mm}|}}%
            \hline
            \text{Fonction}   &$f_1$&$f_2$&$f_3$\\\hline
            \text{Coefficient}&&&\\\hline
        \end{array}
        \]        
        }
        \item Indiquer l'ordonnée à l'origine de chaque droite.
        
        {\renewcommand{\arraystretch}{1.8}
        \[\begin{array}{|>{\columncolor{gray!15}}>{\centering\arraybackslash}m{30mm}|*{3}{>{\centering\arraybackslash}m{10mm}|}}%
            \hline
            \text{Droite}   &$(d_1)$&$(d_2)$&$(d_3$)\\\hline
            \text{Ordonnée à l'origine}&&&\\\hline
        \end{array}
        \]        
        }
        \item En déduire l'expression de chaque fonction.
    \end{enumerate}
\end{exercice*}
\begin{corrige}
    Les droites $(d_1)$, $(d_2)$ et $(d_3)$ sont les représentations graphiques respectives de quatre fonctions linéaires $f_1$, $f_2$ et $f_3$.

    \hspace*{-10mm}\scalebox{0.9}{
    \Fonction[%
        Trace,
        Calcul={0.5x,-x+5,2x+1},% les fonctions.
        Bornea={-1.5,-0.5,-1},% les bornes a de l'intervalle [a,b] de tracé
        Borneb={7.5,6,2.3},% les bornes b de l'intervalle [a,b] de tracé
        LabelC={0.9,0.95,0.9},% les positions du label des courbes
        NomCourbe={$(d_1)$,$(d_2)$,$(d_3)$},% les noms des représentations graphiques.
        Xmin=-1,Xmax=7,
        Ymin=-0.5,Ymax=5,
        CouleurTrace={bleu,Crimson,Gray},
        Origine={(1,0.5)},
        Grille,PasGrilleX=0.5,PasGrilleY=0.5,        
        Graduations,
        Traces={%
            drawarrow placepoint(0,1)--placepoint(1,1) dashed evenly;
            drawarrow placepoint(1,1)--placepoint(1,3) dashed evenly;
            drawarrow placepoint(2,3)--placepoint(3,3) dashed evenly;
            drawarrow placepoint(3,3)--placepoint(3,2) dashed evenly;
            drawarrow placepoint(4,2)--placepoint(5,2) dashed evenly;
            drawarrow placepoint(5,2)--placepoint(5,2.5) dashed evenly;
        }
    ]{}
    }    
    \begin{enumerate}
        \item Indiquer la(les) fonction(s) qui ont un coefficient négatif.          
    \end{enumerate}
    \Coupe
    \textcolor{red}{Pour $f_2$, quand les abscisses augmentent, les images correspondantes diminuent, le coefficient est négatif. }
    \begin{enumerate}
        \setcounter{enumi}{1}        
        \item Indiquer le coefficient de $f_1$, $f_2$ et $f_3$.
        {\renewcommand{\arraystretch}{1.8}
        $\begin{array}{|>{\columncolor{gray!15}}>{\centering\arraybackslash}m{15mm}|*{3}{>{\centering\arraybackslash}m{10mm}|}}%
            \hline
            \text{Fonction}   &$f_1$&$f_2$&$f_3$\\\hline
            \text{Coefficient}&\textcolor{red}{\num{0.5}}&\textcolor{red}{\num{-1}}&\textcolor{red}{\num{2}}\\\hline
        \end{array}$
        }
        \item Indiquer l'ordonnée à l'origine de chaque droite.
        
        {\renewcommand{\arraystretch}{1.8}
        $\begin{array}{|>{\columncolor{gray!15}}>{\centering\arraybackslash}m{30mm}|*{3}{>{\centering\arraybackslash}m{10mm}|}}%
            \hline
            \text{Droite}   &$(d_1)$&$(d_2)$&$(d_3$)\\\hline
            \text{Ordonnée à l'origine}&\textcolor{red}{\num{0}}&\textcolor{red}{\num{5}}&\textcolor{red}{\num{1}}\\\hline
        \end{array}$
        }
        \item En déduire l'expression de chaque fonction.
        
        \textcolor{red}{$f_1(x)=\num{0.5}x$; $f_2(x)=-x+5$ et $f_3(x)=2x+1$}
    \end{enumerate}
\end{corrige}
