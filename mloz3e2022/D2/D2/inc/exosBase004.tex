\begin{exercice*}
    $j$ est une fonction linéaire telle que $j(4)=3$.
    \begin{enumerate}
        \item Est-il possible que $j(8)=-5$ ? Justifier.
        \item Sans déterminer le coefficient de $j$, calculer $j(24)$ et $j(-2)$.
        \item Déterminer le coefficient de $j$.
    \end{enumerate}
\end{exercice*}
\begin{corrige}
    $j$ est une fonction linéaire telle que $j(4)=3$.
    \begin{enumerate}
        \item Est-il possible que $j(8)=-5$ ? Justifier.
        
        \textcolor{red}{$4\times (-2) = -8$ or $3\times (-2) \neq -5$ donc si c'était possible, $j$ ne serait pas linéaire.}
        \item Sans déterminer le coefficient de $j$, calculer $j(24)$ et $j(-2)$.
        
        \textcolor{red}{Une fonction linéaire traduit la proportionnalité donc $j(24)=j(6\times 4)=6\times j(4)=6\times 3=18$
        et $j(-2)=\num{-0.5}\times j(4) = \num{-0.5}\times 3 = \num{-1.5}$}
        \item Déterminer le coefficient de $j$.
        
        \textcolor{red}{$j$ est de la forme $ax$, or $j(4)=3$ donc $4a=3$ et $a=\num{0.75}$. Le coefficient de $j$ est donc \num{0.75}.}
    \end{enumerate}
\end{corrige}
