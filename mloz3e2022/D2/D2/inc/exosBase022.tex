\begin{exercice*}
    $f(x)$ est une fonction affine de la forme $ax+b$ telle que $f(-3)=-10$ et $f(3)=2$.

    On souhaite déterminer l'expression algébrique de $f$ , c'est-à-dire déterminer $a$ et $b$.
    \begin{enumerate}
        \item Calculer le coefficient de $f$ en utilisant la formule 
        
        $$a=\dfrac{f(x_1)-f(x_2)}{x_1-x_2}$$
        
        Puis déterminer l'expression algébrique de $f$.
        \item Vérifier que la fonction trouvée convient.
    \end{enumerate}
\end{exercice*}
\begin{corrige}
    $f(x)$ est une fonction affine de la forme $ax+b$ telle que $f(-3)=-10$ et $f(3)=2$.

    On souhaite déterminer l'expression algébrique de $f$ , c'est-à-dire déterminer $a$ et $b$.
    
    \begin{enumerate}
        \item Calculer le coefficient de $f$ en utilisant la formule $a=\dfrac{f(x_1)-f(x_2)}{x_1-x_2}$ puis déterminer l'expression algébrique de $f$.
        
        {\color{red} 
        %\FonctionAffine[Retrouve]{3}{2}{-3}{-10}
        On sait que f est une fonction affine. Donc elle s’écrit sous la forme : $f (x) = ax + b$
        
        Or, $f (3) = 2$ et $f (-3) = -10$. Par conséquent, d’après la propriété des accroissements :

        $a=\dfrac{f(3)-f(-3)}{3-(-3)}$

        $a=\dfrac{2-(-10)}{6}$

        $a=2$

        La fonction $f$ s’écrit alors sous la forme $f (x) = 2x + b$.
        De plus, comme $f (3) = 2$, alors : $2\times 3+b = 2$ d'où $b=-4$

        La fonction affine $f$ cherchée est donc $f : x \longmapsto 2x - 4$
        
        }
    \end{enumerate}
    \Coupe
    \begin{enumerate}
        \setcounter{enumi}{1}
        \item Vérifier que la fonction trouvée convient.        
        {\color{red}%
        \FonctionAffine[Image]{-3}{2}{-4}{}
        \FonctionAffine[Image]{3}{2}{-4}{}
        }
    \end{enumerate}
\end{corrige}
