\begin{exercice*}
    $f(x)$ est une fonction affine de la forme $ax+b$ telle que $f(-3)=-10$ et $f(3)=2$.

    On souhaite déterminer l'expression algébrique de $f$ , c'est-à-dire déterminer $a$ et $b$.
    \begin{enumerate}
        \item Calculer le coefficient de $f$ en utilisant la formule 
        
        $$a=\dfrac{f(x_1)-f(x_2)}{x_1-x_2}$$
        
        Puis déterminer l'expression algébrique de $f$.
        \item Vérifier que la fonction trouvée convient.
    \end{enumerate}
\end{exercice*}
\begin{corrige}
    $f(x)$ est une fonction affine de la forme $ax+b$ telle que $f(-3)=-10$ et $f(3)=2$.

    On souhaite déterminer l'expression algébrique de $f$ , c'est-à-dire déterminer $a$ et $b$.
    \begin{enumerate}
        \item Calculer le coefficient de $f$ en utilisant la formule $a=\dfrac{f(x_1)-f(x_2)}{x_1-x_2}$ puis déterminer l'expression algébrique de $f$.
        
        {\color{red} \FonctionAffine[Retrouve]{3}{2}{-3}{-10}}
        \item Vérifier que la fonction trouvée convient.
        
        {\color{red}%
        \FonctionAffine[Image]{-3}{2}{-4}{}

        \FonctionAffine[Image]{3}{2}{-4}{}
        }
    \end{enumerate}
\end{corrige}
