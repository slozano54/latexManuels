\section{Représentation graphique d'une fonction linéaire}
\begin{propriete}[\admise]
    Dans un repère, la représentation graphique d'une fonction linéaire de coefficient $a$ est une droite passant par l'origine du repère.
\end{propriete}
\begin{exemple*1}
    Représenter graphiquement la fonction linéaire $f$ de coefficient 0,6.

    $$f:x\mapsto0,6x$$
    \begin{minipage}{0.55\linewidth}
        \psset{unit=0.9}
        \begin{pspicture}(-3,-3)(4,3)
            \psplot[plotpoints=2,linecolor=red]{-3}{4}{0.6 x mul}
            \rput(0,0){\rnode{A}{\pscircle[fillstyle=solid,fillcolor=yellow](0,0){0.1}}}
            \rput(3,1.8){\rnode{C}{\pscircle[fillstyle=solid,fillcolor=yellow](0,0){0.1}}}
            \psgrid[griddots=5,subgriddots=5,gridlabels=0pt]
            \psline{->}(0,-3)(0,3)
            \psline{->}(-3,0)(4,0)
            {\footnotesize
            \rput(1,-0.2){1}
            \rput(-0.2,1){1}
            \rput(3,-0.2){3}
            \rput(-0.25,1.8){1,8}
            }
            \pscircle*(1,0){0.05}
            \pscircle*(0,1){0.05}
            \pscircle*(0,1.8){0.05}
            \pscircle*(3,0){0.05}
            \psline{->}(0,1.8)(3,1.8)
            \psline{->}(3,0)(3,1.8)
        \end{pspicture}
    \end{minipage}
    \begin{minipage}{0.45\linewidth}
        Comme $f$ est une fonction linéaire, sa représentation graphique est une droite qui passe par \rnode{B}{\psframebox{l'origine du repère}}.
        $$\begin{tabular}{|c|c|c|}
        \hline
        $x$&0&3\\
        \hline
        $f(x)$&0&1,8\\
        \hline
        \end{tabular}
        $$
        On prend $x=3$ : son image est $f(3)=0,5\times3=1,8$. Je place \rnode{D}{\psframebox{le point de coordonnées $(3;1,8)$}}.
    \end{minipage}
    \ncline[linecolor=gray,linestyle=dashed,nodesepA=3pt,nodesepB=3pt]{->}{B}{A}
    \ncline[linecolor=gray,linestyle=dashed,nodesepA=3pt,nodesepB=3pt]{->}{D}{C}
\end{exemple*1}
\begin{propriete}[\admise]
    Soit $(d)$ la droite qui représente graphiquement la fonction linéaire de coefficient $a$.
    
    On dit alors que $a$ est {\bf le coefficient directeur} de la droite $(d)$ et que $y=ax$ est une {\bf équation de la droite $(d)$}.
\end{propriete}
\begin{exemple*1}
    Représenter graphiquement la fonction linéaire $g$ de coefficient $-2$.

    $$f:x\mapsto-2x$$
    \begin{minipage}{0.55\linewidth}
        \psset{unit=0.9}
        \begin{pspicture}(-3,-3)(4,3)
            \psplot[plotpoints=2,linecolor=red]{-1.5}{1.5}{-2 x mul}
            \rput(0,0){\rnode{A}{\pscircle[fillstyle=solid,fillcolor=yellow](0,0){0.1}}}
            \rput(1,-2){\rnode{C}{\pscircle[fillstyle=solid,fillcolor=yellow](0,0){0.1}}}
            \psgrid[griddots=5,subgriddots=5,gridlabels=0pt]
            \psline{->}(0,-3)(0,3)
            \psline{->}(-3,0)(4,0)
            {\footnotesize
            \rput(1,-0.2){1}
            \rput(-0.2,1){1}
            \rput(-0.25,-2){$-2$}
            }
            \pscircle*(1,0){0.05}
            \pscircle*(0,1){0.05}
            \pscircle*(0,-2){0.05}
            \psline{->}(0,-2)(1,-2)
            \psline{->}(1,0)(1,-2)
        \end{pspicture}
    \end{minipage}
    \begin{minipage}{0.45\linewidth}    
        Comme $f$ est une fonction linéaire, sa représentation graphique est une droite qui passe par \rnode{B}{\psframebox{l'origine du repère}}.
        $$\begin{tabular}{|c|c|c|}
        \hline
        $x$&0&1\\
        \hline
        $f(x)$&0&-2\\
        \hline
        \end{tabular}
        $$
        On prend $x=1$ : son image est $g(1)=-2\times1=-2$. Je place \rnode{D}{\psframebox{le point de coordonnées $(1;-2)$}}.
    \end{minipage}
    \ncline[linecolor=gray,linestyle=dashed,nodesepA=3pt,nodesepB=3pt]{->}{B}{A}
    \ncline[linecolor=gray,linestyle=dashed,nodesepA=3pt,nodesepB=3pt]{->}{D}{C}    
\end{exemple*1}
