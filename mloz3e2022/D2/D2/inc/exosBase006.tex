\begin{exercice*}
    Parmi ses fonctions :
    \begin{multicols}{2}
    \begin{itemize}
        \item[] $f:x\longmapsto 4x-3$
        \item[] $g:x\longmapsto 5-2x$
        \item[] $h:x\longmapsto \num{4.5}{x}$
        \columnbreak
        \item[] $j:x\longmapsto 3x^2+5$        
        \item[] $k:x\longmapsto -4$
        \item[] $l:x\longmapsto \dfrac{1}{x}$
    \end{itemize}
    \end{multicols}
    Déterminer en le justifiant, celles :
    \begin{enumerate}
        \item qui sont affines,
        \item qui sont linéaires,
        \item qui sont constantes,
        \item qui ne sont pas affines.
    \end{enumerate}
\end{exercice*}
\begin{corrige}
    Parmi ses fonctions :
    \begin{multicols}{2}
    \begin{itemize}
        \def\item{}
        \item $f:x\longmapsto 4x-3$
        \item $g:x\longmapsto 5-2x$
        \columnbreak
        \item $h:x\longmapsto \num{4.5}{x}$
        \item $j:x\longmapsto 3x^2+5$        
        \item $k:x\longmapsto -4$
        \item $l:x\longmapsto \dfrac{1}{x}$
    \end{itemize}
    \end{multicols}
    Déterminer en le justifiant, celles :

    \begin{enumerate}
        \item qui sont affines,
        
        \textcolor{red}{$f$, $g$, $h$ et $k$ sont bien de la forme $ax+b$ donc ce sont des fonctions affines. On pourra préciser les valeurs de $a$ et $b$ pour chaque fonctions.}
        \item qui sont linéaires,
        
        \textcolor{red}{$h$ est de la forme $ax$ avec $a=\num{4.5}$ donc c'est une fonction linéaire.}
        \item qui sont constantes,
        
        \textcolor{red}{$k$ est de la forme $b$ avec $b=-4$ donc c'est une fonction constante.}
        \item qui ne sont pas affines.
        
        \textcolor{red}{$j$ et $l$ ne sont pas de la forme $ax+b$ donc ce ne sont pas des fonctions affines.}
    \end{enumerate}
\end{corrige}
