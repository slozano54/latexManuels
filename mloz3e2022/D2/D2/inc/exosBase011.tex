\begin{exercice*}
    Soit $h$ la fonction affine qui, à un nombre $x$, associe le nombre $7x+3$.
    \begin{enumerate}
        \item Calculer les rapports suivants :
        \begin{enumerate}
            \item $\dfrac{h(3)-h(2)}{3-2}$ \smallskip
            \item $\dfrac{h(5)-h(-1)}{5-(-1)}$ \smallskip
            \item $\dfrac{h(-3)-h(4)}{-3-4}$
        \end{enumerate}
        \item Faire une remarque.
    \end{enumerate}
\end{exercice*}
\begin{corrige}
    Soit $h$ la fonction affine qui, à un nombre $x$, associe le nombre $7x+3$.

    \begin{enumerate}
        \item Calculer les rapports suivants :
        
        \begin{enumerate}
            \item $\dfrac{h(3)-h(2)}{3-2} = \textcolor{red}{\dfrac{24-17}{1}=7}$ \smallskip
            \item $\dfrac{h(5)-h(-1)}{5-(-1)} = \textcolor{red}{\dfrac{38-(-4)}{5+1}=\dfrac{38+4}{6}=\dfrac{42}{6}=7}$ \smallskip
            \item $\dfrac{h(-3)-h(4)}{-3-4} = \textcolor{red}{\dfrac{-18-31}{-7}=\dfrac{-49}{-7}=7}$
        \end{enumerate}
        \item Faire une remarque.
        
        \textcolor{red}{Ces trois rapports sont égaux. Pour une fonction affine, les accroissements traduisent une situation de proportionnalité.}
    \end{enumerate}
\end{corrige}
