\begin{exercice*}
    $f$ est une fonction linéaire de coefficient $-5$.
    \begin{enumerate}
        \item Compléter le tableau de valeurs suivant.
        \[\begin{array}{|>{\columncolor{gray!15}}c|*{7}{>{\centering\arraybackslash}m{7mm}|}}%
            \hline
            x   &\num{-3}&\num{-0.5}&&&\num{5}&&\num{10}\\\hline
            f(x)&&&\num{0.5}&\num{0}&&\num{-18}&\\\hline
        \end{array}
        \]     
        \item Faire un commentaire justifié sur ce tableau.
    \end{enumerate}
\end{exercice*}
\begin{corrige}
    $f$ est une fonction linéaire de coefficient $-5$.
    \begin{enumerate}
        \item Compléter le tableau de valeurs suivant.
        
        \hspace*{-10mm}
        $\begin{array}{|>{\columncolor{gray!15}}c|*{7}{>{\centering\arraybackslash}m{7mm}|}}%
            \hline
            x   &\num{-3}      &\num{-0.5}     &{\red\num{-0.1}}&{\red\num{0}}&\num{5}        &{\red\num{3.6}}&\num{10}\\\hline
            f(x)&{\red\num{15}}&{\red\num{2.5}}&\num{0.5}       &\num{0}      &{\red\num{-25}}&\num{-18}   &{\red\num{-50}}\\\hline
        \end{array}$    
        \item Faire un commentaire justifié sur ce tableau.
        
        \textcolor{red}{On obtient les valeurs de la ligne $f(x)$ sont égales à 5 fois celles de la ligne $x$, c'est donc un tableau de proportionnalité.}
    \end{enumerate}
\end{corrige}
