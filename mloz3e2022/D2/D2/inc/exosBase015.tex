\begin{exercice*}
    Soient les fonctions $f : x\longmapsto 4 x$ et $g : x \longmapsto - 4 x$.
    \begin{enumerate}
        \item Déterminer la nature de leur représentation graphique ? Justifier.
        \item Calculer les coordonnées des points $F$ et $G$ d'abscisse 1 de la courbe de $f$ puis de celle de $g$.
        
        \smallskip
        \begin{minipage}{0.4\linewidth}
            \item Tracer la courbe de $f$.
        
            \begin{center}
                \Fonction[%
                    Trace,%
                    % Vide,%
                    Calcul=0,%
                    Ymin=-2,Ymax=2,Ystep=2,%
                    Xmin=-0.5,Xmax=0.5,Xstep=2,%
                    Origine={(0.5,2)},%
                    PasGrilleX=0.5,PasGrilleY=0.5,%
                    Grille,%
                    Traces={%
                        dotlabel.bot(btex \num{1} etex,placepoint(1,0));
                        dotlabel.lft(btex \num{1} etex,placepoint(0,1));
                    }
                ]{}
            \end{center}    
        \end{minipage}
        \hfill
        \begin{minipage}{0.4\linewidth}
            \item Tracer la courbe de $g$.
            
            \begin{center}
                \Fonction[%
                    Trace,%
                    % Vide,%
                    Calcul=0,%
                    Ymin=-2,Ymax=2,Ystep=2,%
                    Xmin=-0.5,Xmax=0.5,Xstep=2,%
                    Origine={(0.5,2)},%
                    PasGrilleX=0.5,PasGrilleY=0.5,%
                    Grille,%
                    Traces={%
                        dotlabel.bot(btex \num{1} etex,placepoint(1,0));
                        dotlabel.lft(btex \num{1} etex,placepoint(0,1));
                    }
                ]{}
            \end{center}
        \end{minipage}
    \end{enumerate}
\end{exercice*}
\begin{corrige}
    Soient les fonctions $f : x\longmapsto 4 x$ et $g : x \longmapsto - 4 x$.

    \begin{enumerate}
        \item Déterminer la nature de leur représentation graphique ? Justifier.
        
        \textcolor{red}{Les fonctions sont linéaires donc e sont des droites passant par l'origine du repère.}
        \item Calculer les coordonnées des points $F$ et $G$ d'abscisse 1 de la courbe de $f$ puis de celle de $g$.
        
        \textcolor{red}{$F(1;4)$ et $G(1;-4)$}
        
        \smallskip
        \begin{minipage}{0.48\linewidth}
            \item Courbe de $f$.
        
                \Fonction[%
                    Trace,%
                    Calcul=4x,%
                    CouleurTrace=red,%
                    Ymin=-2,Ymax=2,Ystep=2,%
                    Xmin=-0.5,Xmax=0.5,Xstep=2,%
                    Origine={(0.5,2)},%
                    PasGrilleX=0.5,PasGrilleY=0.5,%
                    Grille,%
                    Traces={%
                        dotlabel.bot(btex \num{1} etex,placepoint(1,0));
                        dotlabel.lft(btex \num{1} etex,placepoint(0,1));
                        drawoptions(withcolor red);
                        dotlabel.rt(btex \textcolor{red}{$F$} etex,placepoint(1,4));
                    }
                ]{}
        \end{minipage}
        \hfill
        \begin{minipage}{0.48\linewidth}
            \item Courbe de $g$.
            
                \Fonction[%
                    Trace,%
                    % Vide,%
                    Calcul=-4x,%
                    CouleurTrace=red,%
                    Ymin=-2,Ymax=2,Ystep=2,%
                    Xmin=-0.5,Xmax=0.5,Xstep=2,%
                    Origine={(0.5,2)},%
                    PasGrilleX=0.5,PasGrilleY=0.5,%
                    Grille,%
                    Traces={%
                        dotlabel.bot(btex \num{1} etex,placepoint(1,0));
                        dotlabel.lft(btex \num{1} etex,placepoint(0,1));
                        drawoptions(withcolor red);
                        dotlabel.rt(btex \textcolor{red}{$G$} etex,placepoint(1,-4));
                    }
                ]{}
        \end{minipage}
    \end{enumerate}
\end{corrige}
