\begin{exercice*}
    Le collège Jean Lurçat décide d’acheter un logiciel pour gérer sa bibliothèque. Il y a trois tarifs :
    \begin{itemize}
        \item \textbf{Tarif A} : \Prix{19} ;
        \item \textbf{Tarif B} : \Prix{0.10} par élève ;
        \item \textbf{Tarif C} : \Prix{8} et  \Prix{0.05} par élève.
    \end{itemize}
    \clearpage
    \begin{enumerate}
        \item Compléter le tableau    
        {\renewcommand{\arraystretch}{1.5}
        \[\begin{array}{|>{\centering\arraybackslash}m{25mm}|*{3}{>{\centering\arraybackslash}m{15mm}|}}%
            \hline
            \rowcolor{gray!15}\text{Nombre d'élèves}&$100$&$200$&$300$\\\hline
            \text{\textbf{Tarif A}}&\Prix{19}&&\\\hline
            \text{\textbf{Tarif B}}&&&\Prix{30}\\\hline
            \text{\textbf{Tarif C}}&&\Prix{18}&\\\hline
        \end{array}
        \] 
        }
        \item Si $x$ représente le nombre d'élèves, entourer la fonction qui correspond au tarif C.
        
        $x\longmapsto 8 + 5x$ \hfill $x\longmapsto 8 + \num{0.05}x$ \hfill $x\longmapsto \num{0.05} + 8x$
        \item Indiquer la nature de cette fonction.
        
        \medskip
        \pointilles
        \item Sur le graphique ci-dessous, on a représenté le tarif B. Sur ce même graphique, représenter les tarifs A et C.
        
        \smallskip
        \scalebox{0.8}{%
            \Fonction[%
                Trace,%
                Calcul=x,%
                CouleurTrace=blue,%
                Bornea=0,% les bornes a de l'intervalle [a,b] de tracé
                Borneb=36,% les bornes b de l'intervalle [a,b] de tracé
                LabelC=0.9,%
                NomCourbe=\textcolor{blue}{Tarif B},%
                Xmin=0,Xmax=9,Xstep=4,%
                Ymin=0,Ymax=9,Ystep=4,%
                Origine={(0.5,0.5)},%
                PasGrilleX=0.25,%
                PasGrilleY=0.25,%
                Grille,%
                LabelX={\textcolor{black}{\textbf{Nombre d'élèves}}},%
                LabelY={\rotatebox{90}{\textcolor{black}{\textbf{Prix en euros}}}},%
                Traces={%
                    dotlabel.bot(btex \num{100} etex,placepoint(10,0));
                    dotlabel.lft(btex \num{10} etex,placepoint(0,10));
                    dotlabel.bot(btex \num{200} etex,placepoint(20,0));
                    dotlabel.lft(btex \num{20} etex,placepoint(0,20));
                    dotlabel.bot(btex \num{300} etex,placepoint(30,0));
                    dotlabel.lft(btex \num{30} etex,placepoint(0,30));
                    drawoptions(withcolor blue);
                    draw placepoint(0,32)--placepoint(32,32) dashed evenly;
                    draw placepoint(32,0)--placepoint(32,32) dashed evenly;
                }
            ]{}
        }
        \item Par lecture graphique, indiquer à partir de combien d’élèves le tarif A est plus intéressant que le tarif C.
        
        Faire apparaitre sur le graphique les tracés nécessaires à la lecture.

        \medskip
        \pointilles

        \medskip
        \pointilles

        \medskip
        \pointilles
        \item Au collège Jean Lurçat, il y a 209 élèves. Indiquer le tarif le plus intéressant.
        
        \medskip
        \pointilles

        \medskip
        \pointilles

        \medskip
        \pointilles
    \end{enumerate}
\end{exercice*}
\begin{corrige}
    Le collège Jean Lurçat décide d’acheter un logiciel pour gérer sa bibliothèque. Il y a trois tarifs :
    \begin{itemize}
        \item \textbf{Tarif A} : \Prix{19} ;
        \item \textbf{Tarif B} : \Prix{0.10} par élève ;
        \item \textbf{Tarif C} : \Prix{8} et  \Prix{0.05} par élève.
    \end{itemize}
    \clearpage
    \begin{enumerate}
        \item Compléter le tableau    
        {\renewcommand{\arraystretch}{1.5}
        \[\begin{array}{|>{\centering\arraybackslash}m{25mm}|*{3}{>{\centering\arraybackslash}m{15mm}|}}%
            \hline
            \rowcolor{gray!15}\text{Nombre d'élèves}&$100$&$200$&$300$\\\hline
            \text{\textbf{Tarif A}}&\Prix{19}&\textcolor{red}{\Prix{19}}&\textcolor{red}{\Prix{19}}\\\hline
            \text{\textbf{Tarif B}}&\textcolor{red}{\Prix{10}}&\textcolor{red}{\Prix{20}}&\Prix{30}\\\hline
            \text{\textbf{Tarif C}}&\textcolor{red}{\Prix{13}}&\Prix{18}&\textcolor{red}{\Prix{23}}\\\hline
        \end{array}
        \] 
        }
        \item Si $x$ représente le nombre d'élèves, entourer la fonction qui correspond au tarif C.
        
        $x\longmapsto 8 + 5x$ \hfill \Circled[inner color=red,outer color=red]{$x\longmapsto 8 + \num{0.05}x$} \hfill $x\longmapsto \num{0.05} + 8x$
        \item Indiquer la nature de cette fonction.
        
        \textcolor{red}{C'est une fonction affine.}
        \item Sur le graphique ci-dessous, on a représenté le tarif B. Sur ce même graphique, représenter les tarifs A et C.
        
        \smallskip
        \hspace*{-15mm}
        \scalebox{1}{%
            \Fonction[%
                Trace,%
                Calcul={x,19,0.5x+8},%
                CouleurTrace={blue,red,red},%
                Bornea={0 ,0 ,0 },% les bornes a de l'intervalle [a,b] de tracé
                Borneb={36,36,36},% les bornes b de l'intervalle [a,b] de tracé
                LabelC={0.9,0.2,0.2},%
                NomCourbe={\textcolor{blue}{Tarif B},\textcolor{red}{Tarif A},\textcolor{red}{Tarif C}},%
                Xmin=0,Xmax=9,Xstep=4,%
                Ymin=0,Ymax=9,Ystep=4,%
                Origine={(0.5,0.5)},%
                PasGrilleX=0.25,%
                PasGrilleY=0.25,%
                Grille,%
                LabelX={\textcolor{black}{\textbf{Nombre d'élèves}}},%
                LabelY={\rotatebox{90}{\textcolor{black}{\textbf{Prix en euros}}}},%
                Traces={%
                    dotlabel.bot(btex \num{100} etex,placepoint(10,0));
                    dotlabel.lft(btex \num{10} etex,placepoint(0,10));
                    dotlabel.bot(btex \num{200} etex,placepoint(20,0));
                    dotlabel.lft(btex \num{20} etex,placepoint(0,20));
                    dotlabel.bot(btex \num{300} etex,placepoint(30,0));
                    dotlabel.lft(btex \num{30} etex,placepoint(0,30));
                    drawoptions(withcolor blue);
                    draw placepoint(0,32)--placepoint(32,32) dashed evenly;
                    draw placepoint(32,0)--placepoint(32,32) dashed evenly;
                    drawoptions(withcolor red);
                    drawarrow placepoint(22,19)--placepoint(22,0) dashed evenly;
                }
            ]{}
        }
        \item Par lecture graphique, indiquer à partir de combien d’élèves le tarif A est plus intéressant que le tarif C.
        
        Faire apparaitre sur le graphique les tracés nécessaires à la lecture.

        \textcolor{red}{À partir d'environ 220 élèves, le tarif A est plus avantageux que le tarif C.}
        \item Au collège Jean Lurçat, il y a 209 élèves. Indiquer le tarif le plus intéressant.
        
        {\color{red}%
        Tarif A : \Prix{19} ; Tarif B : $209\times\num{0.1}=\Prix{20.90}$ ; Tarif C : $8+209\times\num{0.05}=\Prix{18.45}$

        Le tarif C est plus avantageux pour 209 élèves.
        }
    \end{enumerate}
\end{corrige}
