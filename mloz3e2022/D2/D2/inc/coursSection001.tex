\section{Fonctions linéaires : Définition(s)}
\begin{definition}
    Soit $a$ un nombre quelconque \og{}fixe\fg{}.
    \\
    Si à chaque nombre $x$ on peut associer son produit par $a$ (c'est à dire $a\times x$) alors on définit {\bf la fonction linéaire} de \underline{coefficient} $a$
    $$\rnode{A}{x}\kern2cm\rnode{B}{ax}$$
    On dit que $ax$ est l'image de $x$.
    \ncarc[arcangle=-15,nodesepA=2pt,nodesepB=2pt]{->}{A}{B}
    \naput{\footnotesize $\times a$}
\end{definition}
\begin{exemple*1}
    L'unité de longueur est le centimètre.\\
    Notons $x$ la longueur du côté d'un carré et $y$ le périmètre de ce carré.

    \medskip
    \begin{minipage}{0.45\linewidth}
        \begin{tabular}{|c|c|c|c|}
        \hline
        $x$&1&0,8&3\\
        \hline
        $y$&4&3,2&12\\
        \hline
        \end{tabular}
    \end{minipage}
    \begin{minipage}{0.45\linewidth}
        On obtient un tableau de proportionnalité : le périmètre d'un carré est proportionnel à son côté et 4 est le coefficient de proportionnalité.
        \\On peut écrire $y=4\times x$ ou $y=4x$.
    \end{minipage}
\end{exemple*1}
\begin{definition}[Vocabulaire \& Notation]
    La fonction linéaire de coefficient $a$ se note $x\mapsto ax$ et se lit \og{}à $x$, on associe $ax$\fg{}.\\
    On peut également la noter par une lettre, $f$ par exemple, alors l'image de $x$ par la fonction linéaire $f$ se note $f(x)$ et on lit \og{}$f$ de $x$\fg{}.
\end{definition}
\begin{exemple*1}
    \begin{itemize}
        \item La fonction linéaire qui à $x$ associe le nombre $4x$ a pour coefficient 4. On écrit
        $$f:x\mapsto4x\mbox{ ou }f(x)=4x$$
        \item Si $f$ est la fonction linéaire de coefficient $-2$ alors l'image de $3$ est $-2\times3=-6$ et on note $f(3)=-6$.
        \item Si $g$ est la fonction linéaire de coefficient 5 alors le nombre $x$ qui a pour image 15 est 3.
    \end{itemize}
\end{exemple*1}

\begin{remarque}
    Une fonction linéaire représente une situation de proportionnalité.
\end{remarque}

