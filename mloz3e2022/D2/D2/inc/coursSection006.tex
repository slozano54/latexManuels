\vspace*{-30mm}
\section{Proportionnalité des accroissements d'une fonction affine}
\begin{propriete}[\admise]
    Soit $f$ une fonction affine $x\mapsto ax+b$.

    Si $x$ varie (c'est à dire augmente ou diminue) d'un nombre $h$, alors son image $f(x)$ varie de $ah$.
\end{propriete}
\begin{methode*1}[Expression algébrique d'une fonction affine - Méthode 1]
    \exercice
    Déterminer la fonction affine $f$ tel que $f(1)=4$ et $f(3)=8$
    \correction
    Une application affine est de la forme $x\mapsto ax+b$.
    $$\begin{tabular}{|c|c|c|}
    \hline
    $x$&1&3\\
    \hline
    $f(x)$&4&8\\
    \hline
    \end{tabular}$$
    \par Donc
    $$\Eqalign{
    4&=2\times a\cr
    a&=2\cr
    }$$
    \par D'où $f$ est de la forme $x\mapsto2x+b$.
    \par Or $f(1)=2\times1+b$ et $f(1)=4$.
    \par D'où
    $$\Eqalign{
    4&=2+b\cr
    b&=2\cr
    }$$
    \par L'application affine cherchée est $x\mapsto2x+2$.
\end{methode*1}
\begin{methode*1}[Expression algébrique d'une fonction affine - Méthode 2]
    \exercice
    Déterminer la fonction affine $f$ tel que $f(1)=4$ et $f(3)=8$
    \correction
    Une application affine est de la forme $x\mapsto ax+b$. Donc

    \begin{minipage}{0.5\linewidth}
    $$\left\{\begin{tabular}{l}
    $4=a\times1+b$\\
    $8=a\times3+b$\\
    \end{tabular}
    \right.
    $$
    $$\left\{\begin{tabular}{l}
    $4=a+b$\\
    $8=3a+b$\\
    \end{tabular}
    \right.
    $$
    $$\left\{\begin{tabular}{l}
    $-4=-a-b$\\
    $8=3a+b$\\
    \end{tabular}
    \right.    
    $$
    \end{minipage}
    \begin{minipage}{0.5\linewidth}
    $$\left\{\begin{tabular}{l}
    $4=2a$\\
    $8=3a+b$\\
    \end{tabular}
    \right.
    $$
    $$\left\{\begin{tabular}{l}
    $a=2$\\
    $8=6+b$\\
    \end{tabular}
    \right.
    $$
    $$\left\{\begin{tabular}{l}
    $a=2$\\
    $b=2$\\
    \end{tabular}
    \right.
    $$
    \end{minipage}
    \par Donc la fonction $f$ est $x\mapsto2x+2$.
\end{methode*1}
