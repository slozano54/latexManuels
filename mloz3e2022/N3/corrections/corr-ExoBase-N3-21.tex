    %\setcounter{partie}{0} % Pour s'assurer que le compteur de \partie est à zéro dans les corrigés
    %\phantom{rrr}
    En informatique, on utilise comme unités de mesure les multiples de l’octet noté o.

    \begin{itemize}
        \def\item{}
        \item[] kilo octet : \Octet[ko]{1} = \Octet[o]{e3};
        \item[] Mega octet : \Octet[Mo]{1} = \Octet[o]{e6};
        \item[] Giga octet : \Octet{1} = \Octet[o]{e9}.
    \end{itemize}

    Sur un disque durde \Octet[To]{1} (Téra octet), déterminer le nombre de vidéos de \Octet{8} (Giga octet)
    que l'on peut stocker.

    {\red
    \Octet[To]{1} = \Octet[o]{e12}; \Octet{8} = \Octet[o]{8e9}; $\dfrac{10^{12}}{8\times 10^9} = 125$

    On peut stocker $125$ vidéos sur le disque dur.
    }
