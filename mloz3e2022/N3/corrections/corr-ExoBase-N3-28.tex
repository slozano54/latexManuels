    %\setcounter{partie}{0} % Pour s'assurer que le compteur de \partie est à zéro dans les corrigés
    %\phantom{rrr}
    Dans le système binaire, utilisé par les ordinateurs par exemple, les nombres sont codés seulement avec
    des 0 et des 1. Pour cela, on utilise des puissances de 2.

    Par exemple, l’écriture décimale de {\bfseries {\red 1}{\blue 0}\textcolor{mygreen}{1}1} est :
    $$\boldsymbol{{\red 1}} \times 2^3 + \boldsymbol{{\blue 0}} \times 2^2 + \boldsymbol{{\textcolor{mygreen}{1}}} \times 2^1 + \boldsymbol{1} \times 2^0$$
    \begin{enumerate}
        \item Déterminer l’écriture décimale de $1001$.

        {\red
        $A= 1 \times 2^3 + 0 \times 2^2 + 0 \times 2^1 +  1\times 2^0$\\
        $A= 1 \times 8 + 0 \times 4 + 0 \times 2 +  1\times 1$\\
        $A= 8+1 = 9$\\
        donc l'écriture décimale de $1001$ est $9$.
        }
        \item Déterminer l’écriture décimale de $11011101$.

        {\red
        $B= 1 \times 2^7 + 1\times 2^6 + 1\times 2^4 + 1\times 2^3 + 1\times 2^2 + 1\times 2^0$\\
        $B= 128+64+16+8+4+41=221$\\
        donc l'écriture décimale de $11011101$ est $221$.
        }
        \item Déterminer l’écriture en binaire du nombre $15$.

        {\red
        $C=8+4+2+1$\\
        $C=1\times 2^3 + 1\times 2^2 + 1\times 2^1 + 1\times 2^0$\\
        donc l'écriture binaire de $15$ est $1111$.
        }
        \item Déterminer l’écriture en binaire du nombre $219$.

        {\red
        $D=128+64+16+8+2+1$\\
        $D=1\times 2^7 + 1\times 2^6 + 1\times 2^4 + 1\times 2^3 + 1\times 2^1 + 1\times 2^0$\\
        donc l'écriture binaire de $219$ est $11011011$.
        }

    \end{enumerate}
