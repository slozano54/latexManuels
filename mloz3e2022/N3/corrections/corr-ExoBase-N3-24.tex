    %\setcounter{partie}{0} % Pour s'assurer que le compteur de \partie est à zéro dans les corrigés
    %\phantom{rrr}
    La lumière parcourt \Lg[km]{300000} par seconde.

    La distance Terre-Soleil est de \num{151.38} millions de \Lg[km]{}.

    \medskip
    Déterminer le temps mis par la lumière pour faire la distance Terre-Soleil.

    {\red $\num{300000}=3\times 10^5$ et $1$ million = $10^6$.

    vitesse = $\dfrac{\text{distance}}{\text{temps}}$ donc temps = $\dfrac{\text{distance}}{\text{vitesse}}$.

    Il faut \Temps{;;;;;504.6} pour faire la distance Terre-Soleil soit environ \Temps{;;;;8;25}.
    }
