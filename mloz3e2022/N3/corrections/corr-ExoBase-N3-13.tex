    %\setcounter{partie}{0} % Pour s'assurer que le compteur de \partie est à zéro dans les corrigés
    %\phantom{rrr}
    Écrire sous la forme d'une puissance d'un nombre.
        \begin{spacing}{2}
            $A=\left((-2)^4\right)^{3}$\\
            $A=\color{red}{\underbrace{\mathbf{\color{red}{((-2)^4)}} \times \mathbf{\color{red}{((-2)^4)}}\times \mathbf{\color{red}{((-2)^4)}}}_{3\thickspace\text{facteurs}}}$\\
            Il y a donc $\mathbf{\color{red}{3}~\color{black}{\times}~\color{blue}{4}}$ facteurs tous égaux à $(-2)$\\
            $A=(-2)^{4\times3} = (-2)^{12}=  2^{12}$\\
            \textbf{Remarque : } Dans ce cas comme les puissances d'exposant pair de deux nombres opposés sont égaux, on peut écrire $ 2^{12}$ à la place de $(-2)^{12}$\\

            \medskip
            $B=\left((-8)^4\right)^{4}$\\
            $B=\color{red}{\underbrace{\mathbf{\color{red}{((-8)^4)}} \times \mathbf{\color{red}{((-8)^4)}}\times \mathbf{\color{red}{((-8)^4)}}\times \mathbf{\color{red}{((-8)^4)}}}_{4\thickspace\text{facteurs}}}$\\
            Il y a donc $\mathbf{\color{red}{4}~\color{black}{\times}~\color{blue}{4}}$ facteurs tous égaux à $(-8)$\\
            $B=(-8)^{4\times4} = (-8)^{16}=  8^{16}$\\
            \textbf{Remarque : } Dans ce cas comme les puissances d'exposant pair de deux nombres opposés sont égaux, on peut écrire $ 8^{16}$ à la place de $(-8)^{16}$\\

            \medskip
            $C=\left(4^3\right)^{3}$\\
            $C=\color{red}{\underbrace{\mathbf{\color{red}{(4^3)}} \times \mathbf{\color{red}{(4^3)}}\times \mathbf{\color{red}{(4^3)}}}_{3\thickspace\text{facteurs}}}$\\
            Il y a donc $\mathbf{\color{red}{3}~\color{black}{\times}~\color{blue}{3}}$ facteurs tous égaux à $4$\\
            $C=4^{3\times3} = 4^{9}$\\

            \medskip
            $D=\left(6^3\right)^{3}$\\
            $D=\color{red}{\underbrace{\mathbf{\color{red}{(6^3)}} \times \mathbf{\color{red}{(6^3)}}\times \mathbf{\color{red}{(6^3)}}}_{3\thickspace\text{facteurs}}}$\\
            Il y a donc $\mathbf{\color{red}{3}~\color{black}{\times}~\color{blue}{3}}$ facteurs tous égaux à $6$\\
            $D=6^{3\times3} = 6^{9}$\\
        \end{spacing}
