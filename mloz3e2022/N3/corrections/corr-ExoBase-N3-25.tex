    %\setcounter{partie}{0} % Pour s'assurer que le compteur de \partie est à zéro dans les corrigés
    %\phantom{rrr}
    L’atome d’oxygène a un diamètre de $650 \times 10^{-13}$ \Lg[m]{}.

    \begin{enumerate}
        \item Exprimer ce rayon en picomètres.

        {\red $1$ picomètre = $1$ pm = \Lg[m]{d{-12}} = \Lg[m]{10d{-13}}

        \Lg[m]{650d{-13}}=$65\times\Lg[m]{10d{-13}}=65$ pm.
        }
        \item Déterminer le nombre d'atomes d’oxygène que l'on peut ranger côte à côte sur une longueur de \Lg[mm]{1}.

        {\red \Lg[mm]{1} = \Lg[m]{d{-3}} donc $\dfrac{10^{-3}}{650\times 10^{-13}}\approx\num{15384615.38}$

        On peut ranger plus de $15$ millions d'atomes sur une longueur de \Lg[mm]{1}.
        }
    \end{enumerate}
