\begin{exercice*}
    Simplifier l'écriture en utilisant la notation puissance.
    \begin{spacing}{2}
        \begin{multicols}{2}
            \begin{enumerate}
                \item $\dfrac{1}{7\times 7\times 7\times 7\times 7}$
                \item $\dfrac{1}{3\times 3\times 3\times 3}$
                \item $\dfrac{1}{(-3)\times (-3)\times (-3)}$
                \item $\dfrac{1}{\num{2.5}\times \num{2.5}\times \num{2.5}\times \num{2.5}\times \num{2.5}}$
            \end{enumerate}
        \end{multicols}
    \end{spacing}
    \hrefMathalea{https://coopmaths.fr/mathalea.html?ex=4C33-0,s=2,s2=2,n=4,i=0&v=l}
\end{exercice*}
\begin{corrige}
    %\setcounter{partie}{0} % Pour s'assurer que le compteur de \partie est à zéro dans les corrigés
    %\phantom{rrr}    
    Simplifier l'écriture en utilisant la notation puissance.
    \begin{spacing}{2}
        \begin{enumerate}
            \item $\dfrac{1}{7\times 7\times 7\times 7\times 7}                                         ={\red \dfrac{1}{7^5}          = 7^{-5}}$
            \item $\dfrac{1}{3\times 3\times 3\times 3}                                                 ={\red \dfrac{1}{3^4}          = 3^{-4}}$
            \item $\dfrac{1}{(-3)\times (-3)\times (-3)}                                                ={\red \dfrac{1}{(-3)^3}       = (-3)^{-3}}$
            \item $\dfrac{1}{\num{2.5}\times \num{2.5}\times \num{2.5}\times \num{2.5}\times \num{2.5}} ={\red \dfrac{1}{\num{2.5}^5}  = \num{2.5}^{-5}}$
        \end{enumerate}
    \end{spacing}
    \vspace*{-10mm}
\end{corrige}

