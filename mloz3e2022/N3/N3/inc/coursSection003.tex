\section{Écriture scientifique d'un nombre décimal}\subitem{}\par

\begin{propriete}[Produit par une puissance de 10]
    n est un entier positif.
    \begin{itemize}
        \item Multiplier par $10^n$ revient à rendre l'unité du nombre devient $10^n$ fois \textbf{plus} forte.
        \item Multiplier par $10^{-n}$ revient à diviser par $10^n$, l'unité devient donc $10^n$ fois \textbf{moins} forte.
    \end{itemize}
\end{propriete}

\begin{exemple*1}   
    $1{\red 7},8 \times 10^4 = 1{\red 7}\underbrace{8\;000}_{4\;rangs}$, les ${\red 7}$ unités deviennent ${\red 7}$ dizaines de milliers. 

    $1{\red 7},8 \times 10^-4 = 0,\underbrace{001{\red 7}}_{4\;rangs}8$, les les ${\red 7}$ unités deviennent ${\red 7}$ dix-millièmes. 
\end{exemple*1}

\begin{definition}[Préfixes pour les sciences]
    \begin{center}
        {\renewcommand{\arraystretch}{1}
        \begin{tabular}{||c||c||c||c||c||c||c||c||}
        \hline 
        \rule[-1ex]{0pt}{2.5ex} Préfixe & giga & méga & kilo & milli & micro & nano & pico \\ 
        \hline 
        \rule[-1ex]{0pt}{2.5ex} Symbole & G & M & k & m & $\mu$ & n & p \\ 
        \hline 
        \rule[-1ex]{0pt}{2.5ex} Puissance de 10 & $10^{9}$ & $10^{6}$ & $10^{3}$ & $10^{-3}$ & $10^{-6}$ & $10^{-9}$ & $10^{-12}$ \\ 
        \hline 
        \end{tabular} 
        }
    \end{center}
\end{definition}

\begin{exemple*1}
    \begin{itemize}
        \item \textbf{En informatique} : Un gigaoctet, noté Go, vaut $10^{3}$ mégaoctets soit $10^{9}$ octets 
        
        c'est à dire 1 milliard d'octets.
        \item \textbf{Dimensions atomiques} : Un nanomètre, noté nm, vaut $10^{-9}$m soit 1 milliardième de mètre.
    \end{itemize}
\end{exemple*1}

\begin{definition}[Écriture scientifique]
\textbf{L'écriture scientifique} d'un nombre décimal est l'écriture de ce nombre sous la forme:
 $$a\times10^p$$
    \begin{itemize}  
        \item $a$ est un nombre décimal ayant un seul chiffre non nul (différent de 0) avant la virgule
        \item $p$ un entier relatif
    \end{itemize}
\end{definition}

\begin{exemple*1}    
    \begin{itemize}
        \item $8,456\times 10^7$ est un nombre écrit en notation scientifique.
        \item L'écriture scientifique de 385 est $3,85\times10^2$.
        \item $4,8\times 3^4$ n'est pas écrit en notation scientifique.
        \item  L'écriture scientifique de $A=35\,48\times10^4$ est $3,548\times10^5$, en effet :
        $$\Eqalign{
        A&=3,548\times10^1\times10^4\cr
        }$$
        $$\psshadowbox{A=3,548\times10^5}$$
    \end{itemize}

    \hrefMathalea[\emoji{star-struck} \emoji{link} Glisse nombre pour les puissances]{https://mathix.org/glisse-nombre/puissance/}    
    \creditLibre{https://mathix.org/linux/}
\end{exemple*1}

\begin{methode*1}[Lorsque l'écriture est \og presque \fg scientifique]
    Comme pour $A$ dans l'exemple précédent.
    \exercice
    Déterminer l'écriture scientifique de $B = 279,31\times10^6$
    \correction
    \begin{itemize}
        \item $B = 279,31\times10^6$
        \item \textbf{On écrit le nombre décimal en notation scientifique} : $B = 2,7931\times 10^2 \times10^6$
        \item \textbf{On rassemble les puissances de 10} : $\psshadowbox{B = 2,7931\times 10^8}$
    \end{itemize}
\end{methode*1}

\begin{methode*1}[Produit des écritures scientifiques]
    \exercice
    Déterminer l'écriture scientifique de $C=7\times 10^7 \times 5 \times 10^4$
    \correction

    $C=7\times 10^7 \times 5 \times 10^4$\\
    $C=7 \times 5 \times 10^7 \times 10^4$\\
    $C=35 \times 10^{11}$\\ 
    $\psshadowbox{C=3,5 \times 10^{12}}$ 
\end{methode*1}

\begin{methode*1}[Quotient des écritures scientifiques]
    \exercice
    Déterminer l'écriture scientifique de $D=\dfrac{7\times 10^7}{5,6 \times 10^4}$
    \correction

    $D=\dfrac{7\times 10^7}{5,6 \times 10^4}$\\
    $D=\dfrac{7}{5,6}\times \dfrac{10^7}{10^4}$\\
    $\psshadowbox{D=1,4\times 10^3}$
\end{methode*1}

\begin{myBox}{\emoji{warning} \emoji{warning} \emoji{warning}}
    Le rapport de deux écritures scientifiques n'a pas toujours d'écriture scientifique.

    \medskip
    {\raggedright
    $E=\dfrac{8\times 10^4}{3 \times 10^2}$ n'a pas d'écriture scientifique car il n'est pas décimal!

    \smallskip
    $E=\dfrac{8}{3}\times \dfrac{10^4}{10^2}$

    \smallskip
    $E=\dfrac{8}{3}\times 10^2$ et $\dfrac{8}{3}$ n'est pas un nombre décimal.

    }
\end{myBox}