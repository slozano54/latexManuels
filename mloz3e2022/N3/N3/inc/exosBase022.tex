\begin{exercice*}
    Un atome est formé d’un noyau et d’électrons. Pour la semaine de la science, Phédra doit réaliser
    une maquette d’un atome. Pour cela, elle représente le noyau par une boule de \Lg{12} de diamètre alors
    que celui-ci mesure $3 \times 10^{-3}$ \Lg{} en réalité.    
    \begin{enumerate}
        \item Déterminer l’échelle de cette maquette.
        \item Écrire le résultat en notation scientifique.
    \end{enumerate}    
\end{exercice*}
\begin{corrige}
    %\setcounter{partie}{0} % Pour s'assurer que le compteur de \partie est à zéro dans les corrigés
    %\phantom{rrr}        
    Un atome est formé d’un noyau et d’électrons. Pour la semaine de la science, Phédra doit réaliser
    une maquette d’un atome. Pour cela, elle représente le noyau par une boule de \Lg{12} de diamètre alors
    que celui-ci mesure $3 \times 10^{-3}$ \Lg{} en réalité.
    
    \begin{enumerate}
        \item Déterminer l’échelle de cette maquette.
        \item Écrire le résultat en notation scientifique.
    \end{enumerate}    

    {\red
    Échelle = $\dfrac{\text{distance représentée}}{\text{distance réelle}}=\dfrac{12}{3\times 10^{-3}}$.
    
    Échelle = $4\times 10^3$.

    L'échelle est de $4\times 10^3$.
    }
\end{corrige}

