\section{Puissances d'un entier relatif}
\begin{definition}
Soit $n$ un nombre entier positif.
$$a^n=\underbrace{a\times a\times a\times\dots\times a}_{n\mbox{ facteurs}}\quad (n\geq 2);\qquad a^1=a;\qquad a^0=1.$$
\begin{itemize}
    \item Le nombre \textbf{n} s'appelle \textbf{exposant}
    \item $a^n$ se lit \textbf{a exposant n} ou \textbf{a puissance n}
    \item On note $a^{-n}$ l'inverse de $a^n$, c'est le nombre qui lorsqu'il est multiplié par $a^n$ donne $1$.
\end{itemize}
\smallskip
$$a^{-n}=\dfrac1{a^n} \text{\hspace*{10mm} et \hspace*{10mm}} a^{-n}\times a^n=a^n\times a^{-n}=1$$
\end{definition}

\begin{exemple*1}    

\begin{minipage}{0.45\linewidth}
$2^5=2\times2\times2\times2\times2=32$ \par\vspace{0.25cm}
$(-5)^3=(-5)\times(-5)\times(-5)=-125$
\end{minipage}
\begin{minipage}{0.45\linewidth}
$5^{-3}=\dfrac1{5^3}=\dfrac1{125}=\num{0.008}$ \par\vspace{0.25cm}
$(-2)^{-4}=\dfrac1{(-2)^4}=\dfrac1{16}=\num{0.0625}$
\end{minipage}
\end{exemple*1}

\begin{methode*1}[Utilisation de la calculatrice]
    Il faut identifier la touche puissance de votre calculatrice, ça peut être : \fbox{$\uparrow$} ou \fbox{$y^x$} ou \fbox{$x^y$} ou \fbox{$\hat{ }$} ou ...
    \exercice
    À l'aide de la calculatrice calculer $(-5)^3$ et $2^5$
    \correction
    {\hfill$(-5)^3=\quad\underbrace{(\,5\,\pm\,)\,\uparrow\,3\,=}_{\mbox{calculatrice}}\quad-125$\hfill$2^5=\quad\underbrace{2\,\uparrow\,5=}_{\mbox{calculatrice}}\quad32$\hfill}

    \medskip
    {\hfill\Calculatrice[Ecran]{"(-5)^3"/"-125"} \hfill \Calculatrice[Ecran]{"2^5"/"32"}\hfill}
\end{methode*1}

\begin{propriete}[Opérations sur les puissances \admise]
    Si $a$ est un entier relatif non nul et si $m$ et $n$ sont des entiers relatifs alors
    $$a^m\times a^n=a^{m+n};\kern1cm\frac{a^m}{a^n}=a^{m-n};\kern1cm\left(a^m\right)^n=a^{m\times n};\kern1cm a^m\times b^m=(a\times b)^m.$$
\end{propriete}


\begin{exemple*1}
    \titreExemple{Mise en pratique}

    Les résultats ci-dessus ne sont pas exigibles mais il faut savoir faire le genre de calculs suivants :
    \begin{enumerate}
        \item $A=\dfrac{(-9)^1}{(-9)^6}= \dfrac{\mathbf{\color{blue}{(-9)}}}{\mathbf{\color{red}{(-9)}} \times \mathbf{\color{red}{(-9)}}\times \mathbf{\color{red}{(-9)}}\times \mathbf{\color{red}{(-9)}}\times \mathbf{\color{red}{(-9)}}\times \mathbf{\color{red}{(-9)}}}$

        \medskip
        Il y a donc $\mathbf{\color{blue}{1}}$ simplifications par $(-9)$ possible.

        \medskip
        $A=\dfrac{\mathbf{\color{blue}{\cancel{(-9)}}}}{\mathbf{\color{red}{\cancel{(-9)}}}\times\mathbf{\color{red}{(-9)}} \times \mathbf{\color{red}{(-9)}}\times \mathbf{\color{red}{(-9)}}\times \mathbf{\color{red}{(-9)}}\times \mathbf{\color{red}{(-9)}}}$

        \medskip
        $A=\dfrac{1}{(-9)^{6-1}}=\dfrac{1}{(-9)^{5}}=\psshadowbox{(-9)^{-5}}$
    \end{enumerate}
    \medskip
    \begin{multicols}{2}        
        \begin{enumerate}
            \setcounter{enumi}{1}
            \item $B=((-6)^2)^{2}=\color{red}{\underbrace{\mathbf{\color{red}{((-6)^2)}} \times \mathbf{\color{red}{((-6)^2)}}}_{2\thickspace\text{facteurs}}}$

            \medskip
            $B=\color{red}{\underbrace{\mathbf{\color{red}{(\color{blue}{\underbrace{\mathbf{\color{blue}{(-6)}} \times \mathbf{\color{blue}{(-6)}}}_{2\thickspace\text{facteurs}}}\color{red})}} \times \mathbf{\color{red}{(\color{blue}{\underbrace{\mathbf{\color{blue}{(-6)}} \times \mathbf{\color{blue}{(-6)}}}_{2\thickspace\text{facteurs}}}\color{red})}}}_{2\times\color{blue}{2}\thickspace\color{black}{\text{facteurs}}}}$

            \medskip
            Il y a donc $\mathbf{\color{red}{2}~\color{black}{\times}~\color{blue}{2}}$ facteurs tous égaux à $(-6)$

            \medskip
            $B=(-6)^{2\times2} = (-6)^{4}= \psshadowbox{ 6^{4}}$

            \medskip
            \item $C=4^4\times 4^1=\mathbf{\color{red}{4}} \times \mathbf{\color{red}{4}}\times \mathbf{\color{red}{4}}\times \mathbf{\color{red}{4}} \times \mathbf{\color{blue}{4}}$

            \medskip
            Il y a donc $\mathbf{\color{red}{4}~\color{black}{+}~\color{blue}{1}}$ facteurs tous égaux à $4$

            \medskip
            $C=4^{4+1} = \psshadowbox{4^{5}}$

            \medskip
            \item $D=5^5\times 8^5$
            
            \medskip
            $D=\mathbf{\color{red}{5}} \times \mathbf{\color{red}{5}}\times \mathbf{\color{red}{5}}\times \mathbf{\color{red}{5}}\times \mathbf{\color{red}{5}} \times \mathbf{\color{blue}{8}} \times \mathbf{\color{blue}{8}}\times \mathbf{\color{blue}{8}}\times \mathbf{\color{blue}{8}}\times \mathbf{\color{blue}{8}}$

            \medskip
            $D=\mathbf{(\color{red}{5}} \times \mathbf{\color{blue}{8}}) \times (\mathbf{\color{red}{5}} \times \mathbf{\color{blue}{8}})\times (\mathbf{\color{red}{5}} \times \mathbf{\color{blue}{8}})\times (\mathbf{\color{red}{5}} \times \mathbf{\color{blue}{8}})\times (\mathbf{\color{red}{5}} \times \mathbf{\color{blue}{8}})$

            \medskip
            $D= (\color{red}{\mathbf{5}} \color{black}{\times} \color{blue}{\mathbf{8}}\color{black}{)^{5}}=\psshadowbox{40^5}$

            \medskip
            \item $E=((-2)^2)^{4}=\color{red}{\underbrace{\mathbf{\color{red}{((-2)^2)}} \times \mathbf{\color{red}{((-2)^2)}}\times \mathbf{\color{red}{((-2)^2)}}\times \mathbf{\color{red}{((-2)^2)}}}_{4\thickspace\text{facteurs}}}$

            \medskip
            $E=\color{red}{\underbrace{\mathbf{\color{red}{(\color{blue}{\underbrace{\mathbf{\color{blue}{(-2)}} \times \mathbf{\color{blue}{(-2)}}}_{2\thickspace\text{facteurs}}}\color{red})}} \times \mathbf{\color{red}{(\color{blue}{\underbrace{\mathbf{\color{blue}{(-2)}} \times \mathbf{\color{blue}{(-2)}}}_{2\thickspace\text{facteurs}}}\color{red})}}\times \mathbf{\color{red}{(\color{blue}{\underbrace{\mathbf{\color{blue}{(-2)}} \times \mathbf{\color{blue}{(-2)}}}_{2\thickspace\text{facteurs}}}\color{red})}}\times \mathbf{\color{red}{(\color{blue}{\underbrace{\mathbf{\color{blue}{(-2)}} \times \mathbf{\color{blue}{(-2)}}}_{2\thickspace\text{facteurs}}}\color{red})}}}_{4\times\color{blue}{2}\thickspace\color{black}{\text{facteurs}}}}$

            \medskip
            Il y a donc $\mathbf{\color{red}{4}~\color{black}{\times}~\color{blue}{2}}$ facteurs tous égaux à $(-2)$

            \medskip
            $E=(-2)^{2\times4} = (-2)^{8}= \psshadowbox{ 2^{8}}$
        \end{enumerate}
    \end{multicols}
\end{exemple*1}