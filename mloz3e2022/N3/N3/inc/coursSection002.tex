\section{Puissances de 10}
\begin{definition}
    Une puissance de 10 se note sous la forme $10^m$ où $m$ est un nombre relatif.\\
    Dans cette écriture, $m$ est appelé \textbf{l'exposant}.
\end{definition}

\begin{definition}[Puissance d'exposant entier positif]
    Soit $m$ un entier positif, alors $10^0=1\mbox{ et }10^1=10$ et pour $m\geq 2$ :
    $$10^m=\underbrace{10\times10\times10\times\ldots\times10}_{\mbox{$m$ facteurs tous égaux à 10}}=1\underbrace{000\ldots0}_{\mbox{$m$ zéros}}$$
\end{definition}

\begin{exemple*1}
    $10^3=10\times10\times10=1\,000\kern1cm10^5=10\times10\times10\times10\times10=100\,000$
\end{exemple*1}

\begin{definition}[Puissance d'exposant entier négatif]
    Soit $m$ un entier positif, alors
    $$10^{-m}=\frac1{10^m}=\underbrace{0,00\ldots0}_{\mbox{$m$ zéros}}1$$
\end{definition}

\begin{exemple*1}
    $10^{-3}=\dfrac1{10^3}=0,001\kern1cm10^{-5}=\dfrac1{10^5}=0,000\,01$
\end{exemple*1}

\begin{center}
\begin{myBox}{\emoji{light-bulb} Lumière sur la formule magique des zéros !}
    \begin{list}{}{}
        \item \textbf{Cas de  l'exposant positif :} Multiplier par 10 fait changer l'unité de rang.\\
        $1 \text{ unité } \times 10 = 1 \text{ dizaine } = 10$ ;\\
        $1 \text{ unité } \times 10 \times 10= 1 \text{ dizaine }\times 10 = 1 \text{ centaine } = 100$;\\
        \ldots
        \item \textbf{Cas de  l'exposant négatif :} Multiplier par 0,1 fait changer l'unité de rang.\\
        $1\text{ unité }\times \num{0.1} = 1 \text{ dixième }$ ;\\
        $1\text{ unité }\times \num{0.1} \times \num{0.1} = 1 \text{ dixième }\times \num{0.1} = 1\text{ centième } = \num{0.01}$;\\
        \ldots
    \end{list}

    \hrefLien{https://mathix.org/glisse-nombre/puissance/}{https://mathix.org/glisse-nombre/puissance/}
    \creditLibre{https://mathix.org/linux/}
\end{myBox}
\end{center}

\begin{propriete}[Formules de calculs \admises]
    Soit $m$ et $n$ deux nombres entiers relatifs.$$10^m\times10^n=10^{m+n};\kern2cm\frac{10^m}{10^n}=10^{m-n};\kern2cm\left(10^m\right)^n=10^{m\times n}$$
\end{propriete}

\begin{exemple*1}
    \titreExemple{Mise en pratique}

    Les résultats ci-dessus ne sont pas exigibles mais il faut savoir faire le genre de calculs suivants :
    \begin{multicols}{2}
        \begin{enumerate}
            \item $A=\dfrac{10^4}{10^1}=\dfrac{\mathbf{\color{red}{10}} \times \mathbf{\color{red}{10}}\times \mathbf{\color{red}{10}}\times \mathbf{\color{red}{10}}}{\mathbf{\color{blue}{10}}}$

            \medskip
            Il y a donc $\mathbf{\color{blue}{1}}$ simplifications par $10$ possible.

            \medskip
            $A=\dfrac{\mathbf{\color{red}{\cancel{10}}}\times\mathbf{\color{red}{10}} \times \mathbf{\color{red}{10}}\times \mathbf{\color{red}{10}}}{\mathbf{\color{blue}{\cancel{10}}}}$

            \medskip
            $A=10^{4-1}=\psshadowbox{10^{3}}$

            \item $B=(10^4)^{2}=\color{red}{\underbrace{\mathbf{\color{red}{(10^4)}} \times \mathbf{\color{red}{(10^4)}}}_{2\thickspace\text{facteurs}}}$

            \medskip
            $B=\color{red}{\underbrace{\mathbf{\color{red}{(\color{blue}{\underbrace{\mathbf{\color{blue}{10}} \times \mathbf{\color{blue}{10}}\times \mathbf{\color{blue}{10}}\times \mathbf{\color{blue}{10}}}_{4\thickspace\text{facteurs}}}\color{red})}} \times \mathbf{\color{red}{(\color{blue}{\underbrace{\mathbf{\color{blue}{10}} \times \mathbf{\color{blue}{10}}\times \mathbf{\color{blue}{10}}\times \mathbf{\color{blue}{10}}}_{4\thickspace\text{facteurs}}}\color{red})}}}_{2\times\color{blue}{4}\thickspace\color{black}{\text{facteurs}}}}$

            \medskip
            Il y a donc $\mathbf{\color{red}{2}~\color{black}{\times}~\color{blue}{4}}$ facteurs tous égaux à $10$

            \medskip
            $B=10^{4\times2} = \psshadowbox{10^{8}}$
        \end{enumerate}
    \end{multicols}
\end{exemple*1}
    
\begin{exemple*1}
    \titreExemple{Mise en pratique - suite}
    
    \begin{enumerate}
        \setcounter{enumi}{2}        
        \item $C=10^7\times 10^8=\mathbf{\color{red}{10}} \times \mathbf{\color{red}{10}}\times \mathbf{\color{red}{10}}\times \mathbf{\color{red}{10}}\times \mathbf{\color{red}{10}}\times \mathbf{\color{red}{10}}\times \mathbf{\color{red}{10}} \times \mathbf{\color{blue}{10}} \times \mathbf{\color{blue}{10}}\times \mathbf{\color{blue}{10}}\times \mathbf{\color{blue}{10}}\times \mathbf{\color{blue}{10}}\times \mathbf{\color{blue}{10}}\times \mathbf{\color{blue}{10}}\times \mathbf{\color{blue}{10}}$
            
        \medskip
        Il y a donc $\mathbf{\color{red}{7}~\color{black}{+}~\color{blue}{8}}$ facteurs tous égaux à $10$

        \medskip
        $C=10^{7+8} = \psshadowbox{10^{15}}$

        \medskip
        \item $D=(10^3)^{4}=\color{red}{\underbrace{\mathbf{\color{red}{(10^3)}} \times \mathbf{\color{red}{(10^3)}}\times \mathbf{\color{red}{(10^3)}}\times \mathbf{\color{red}{(10^3)}}}_{4\thickspace\text{facteurs}}}$
            
        \medskip
        $D=\color{red}{\underbrace{\mathbf{\color{red}{(\color{blue}{\underbrace{\mathbf{\color{blue}{10}} \times \mathbf{\color{blue}{10}}\times \mathbf{\color{blue}{10}}}_{3\thickspace\text{facteurs}}}\color{red})}} \times \mathbf{\color{red}{(\color{blue}{\underbrace{\mathbf{\color{blue}{10}} \times \mathbf{\color{blue}{10}}\times \mathbf{\color{blue}{10}}}_{3\thickspace\text{facteurs}}}\color{red})}}\times \mathbf{\color{red}{(\color{blue}{\underbrace{\mathbf{\color{blue}{10}} \times \mathbf{\color{blue}{10}}\times \mathbf{\color{blue}{10}}}_{3\thickspace\text{facteurs}}}\color{red})}}\times \mathbf{\color{red}{(\color{blue}{\underbrace{\mathbf{\color{blue}{10}} \times \mathbf{\color{blue}{10}}\times \mathbf{\color{blue}{10}}}_{3\thickspace\text{facteurs}}}\color{red})}}}_{4\times\color{blue}{3}\thickspace\color{black}{\text{facteurs}}}}$

        \medskip
        Il y a donc $\mathbf{\color{red}{4}~\color{black}{\times}~\color{blue}{3}}$ facteurs tous égaux à $10$

        \medskip
        $D=10^{3\times4} = \psshadowbox{10^{12}}$

        \medskip
        \item $E=\dfrac{10^4}{10^7}=\dfrac{\mathbf{\color{blue}{10}} \times \mathbf{\color{blue}{10}}\times \mathbf{\color{blue}{10}}\times \mathbf{\color{blue}{10}}}{\mathbf{\color{red}{10}} \times \mathbf{\color{red}{10}}\times \mathbf{\color{red}{10}}\times \mathbf{\color{red}{10}}\times \mathbf{\color{red}{10}}\times \mathbf{\color{red}{10}}\times \mathbf{\color{red}{10}}}$

        \medskip
        Il y a donc $\mathbf{\color{blue}{4}}$ simplifications par $10$ possibles.

        \medskip
        $E=\dfrac{\mathbf{\color{blue}{\cancel{10}}} \times \mathbf{\color{blue}{\cancel{10}}}\times \mathbf{\color{blue}{\cancel{10}}}\times \mathbf{\color{blue}{\cancel{10}}}}{\mathbf{\color{red}{\cancel{10}}} \times \mathbf{\color{red}{\cancel{10}}}\times \mathbf{\color{red}{\cancel{10}}}\times \mathbf{\color{red}{\cancel{10}}}\times\mathbf{\color{red}{10}} \times \mathbf{\color{red}{10}}\times \mathbf{\color{red}{10}}}$

        \medskip
        $E=\dfrac{1}{10^{7-4}}=\dfrac{1}{10^{3}}=\psshadowbox{10^{-3}}$
    \end{enumerate}
\end{exemple*1}
