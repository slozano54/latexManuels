\section{Puissances de 10}
% \definNum{
% Une puissance de 10 se note sous la forme $10^m$ o\`{u} $m$ est un nombre relatif.\\
% Dans cette \'ecriture, $m$ est appel\'e \textbf{l'exposant}.
% }
% \definNum{
% {\bf L'exposant est un entier positif}\par
% Soit $m$ un entier positif, alors $10^0=1\mbox{ et }10^1=10$ et pour $m\geq 2$ :
% $$10^m=\underbrace{10\times10\times10\times\ldots\times10}_{\mbox{$m$ facteurs tous égaux à 10}}=1\underbrace{000\ldots0}_{\mbox{$m$ z\'eros}}$$
% }

% \Exemples{}{
% $10^3=10\times10\times10=1\,000\kern1cm10^5=10\times10\times10\times10\times10=100\,000$
% }

% \definNum{
% {\bf L'exposant est un entier négatif}\par
% Soit $m$ un entier positif, alors
% $$10^{-m}=\frac1{10^m}=\underbrace{0,00\ldots0}_{\mbox{$m$ z\'eros}}1$$
% }

% \Exemples{}{
% $10^{-3}=\dfrac1{10^3}=0,001\kern1cm10^{-5}=\dfrac1{10^5}=0,000\,01$
% }

% \CadreLampe{Pourquoi la formule magique des m zéros ?}{
% \begin{mylist}
% \item \textbf{Cas de  l'exposant positif :} Multiplier par 10 fait changer l'unité de rang.\\
% $1$ unité $\times 10 = 1$ dizaine $= 10$ ; \hfill $1$ unité $\times 10 \times 10= 1$ dizaine $\times 10 = 1$  centaine$=100$ ...
% \item \textbf{Cas de  l'exposant négatif :} Multiplier par 0,1 fait changer l'unité de rang.\\
% $1$ unité $\times 0,1 = 1$ dixième ; \hfill $1$ unité $\times 0,1 \times 0,1= 1$ dixième $\times 0,1 = 1$ centième $=0,01$ ...
% \end{mylist}
% \infoComplementNumerique
% \lienCadre{https://mathix.org/glisse-nombre/puissance/}{https://mathix.org/glisse-nombre/puissance/}
% \creditLibre{https://mathix.org/linux/}
% }

% \proprNumBis{ Formules de calculs (admises)}{Soit $m$ et $n$ deux nombres entiers relatifs.$$10^m\times10^n=10^{m+n}\kern2cm\frac{10^m}{10^n}=10^{m-n}\kern2cm\left(10^m\right)^n=10^{m\times n}$$
% }

% \Exemples{}{
% Les résultats ci-dessus ne sont pas exigibles mais il faut savoir faire le genre de calculs suivants :
% \begin{enumerate}
% \item $A=\dfrac{10^4}{10^1}=\dfrac{\mathbf{\color{red}{10}} \times \mathbf{\color{red}{10}}\times \mathbf{\color{red}{10}}\times \mathbf{\color{red}{10}}}{\mathbf{\color{blue}{10}}}$

% \medskip
% Il y a donc $\mathbf{\color{blue}{1}}$ simplifications par $10$ possible.

% \medskip
% $A=\dfrac{\mathbf{\color{red}{\cancel{10}}}\times\mathbf{\color{red}{10}} \times \mathbf{\color{red}{10}}\times \mathbf{\color{red}{10}}}{\mathbf{\color{blue}{\cancel{10}}}}$

% \medskip
% $A=10^{4-1}=\psshadowbox{10^{3}}$

% \newpage
% \item $B=(10^4)^{2}=\color{red}{\underbrace{\mathbf{\color{red}{(10^4)}} \times \mathbf{\color{red}{(10^4)}}}_{2\thickspace\text{facteurs}}}$

% \medskip
% $B=\color{red}{\underbrace{\mathbf{\color{red}{(\color{blue}{\underbrace{\mathbf{\color{blue}{10}} \times \mathbf{\color{blue}{10}}\times \mathbf{\color{blue}{10}}\times \mathbf{\color{blue}{10}}}_{4\thickspace\text{facteurs}}}\color{red})}} \times \mathbf{\color{red}{(\color{blue}{\underbrace{\mathbf{\color{blue}{10}} \times \mathbf{\color{blue}{10}}\times \mathbf{\color{blue}{10}}\times \mathbf{\color{blue}{10}}}_{4\thickspace\text{facteurs}}}\color{red})}}}_{2\times\color{blue}{4}\thickspace\color{black}{\text{facteurs}}}}$

% \medskip
% Il y a donc $\mathbf{\color{red}{2}~\color{black}{\times}~\color{blue}{4}}$ facteurs tous égaux à $10$

% \medskip
% $B=10^{4\times2} = \psshadowbox{10^{8}}$

% \item $C=10^7\times 10^8=\mathbf{\color{red}{10}} \times \mathbf{\color{red}{10}}\times \mathbf{\color{red}{10}}\times \mathbf{\color{red}{10}}\times \mathbf{\color{red}{10}}\times \mathbf{\color{red}{10}}\times \mathbf{\color{red}{10}} \times \mathbf{\color{blue}{10}} \times \mathbf{\color{blue}{10}}\times \mathbf{\color{blue}{10}}\times \mathbf{\color{blue}{10}}\times \mathbf{\color{blue}{10}}\times \mathbf{\color{blue}{10}}\times \mathbf{\color{blue}{10}}\times \mathbf{\color{blue}{10}}$
	
% \medskip
% Il y a donc $\mathbf{\color{red}{7}~\color{black}{+}~\color{blue}{8}}$ facteurs tous égaux à $10$

% \medskip
% $C=10^{7+8} = \psshadowbox{10^{15}}$

% \item $D=(10^3)^{4}=\color{red}{\underbrace{\mathbf{\color{red}{(10^3)}} \times \mathbf{\color{red}{(10^3)}}\times \mathbf{\color{red}{(10^3)}}\times \mathbf{\color{red}{(10^3)}}}_{4\thickspace\text{facteurs}}}$
	
% \medskip
% $D=\color{red}{\underbrace{\mathbf{\color{red}{(\color{blue}{\underbrace{\mathbf{\color{blue}{10}} \times \mathbf{\color{blue}{10}}\times \mathbf{\color{blue}{10}}}_{3\thickspace\text{facteurs}}}\color{red})}} \times \mathbf{\color{red}{(\color{blue}{\underbrace{\mathbf{\color{blue}{10}} \times \mathbf{\color{blue}{10}}\times \mathbf{\color{blue}{10}}}_{3\thickspace\text{facteurs}}}\color{red})}}\times \mathbf{\color{red}{(\color{blue}{\underbrace{\mathbf{\color{blue}{10}} \times \mathbf{\color{blue}{10}}\times \mathbf{\color{blue}{10}}}_{3\thickspace\text{facteurs}}}\color{red})}}\times \mathbf{\color{red}{(\color{blue}{\underbrace{\mathbf{\color{blue}{10}} \times \mathbf{\color{blue}{10}}\times \mathbf{\color{blue}{10}}}_{3\thickspace\text{facteurs}}}\color{red})}}}_{4\times\color{blue}{3}\thickspace\color{black}{\text{facteurs}}}}$

% \medskip
% Il y a donc $\mathbf{\color{red}{4}~\color{black}{\times}~\color{blue}{3}}$ facteurs tous égaux à $10$

% \medskip
% $D=10^{3\times4} = \psshadowbox{10^{12}}$

% \item $E=\dfrac{10^4}{10^7}=\dfrac{\mathbf{\color{blue}{10}} \times \mathbf{\color{blue}{10}}\times \mathbf{\color{blue}{10}}\times \mathbf{\color{blue}{10}}}{\mathbf{\color{red}{10}} \times \mathbf{\color{red}{10}}\times \mathbf{\color{red}{10}}\times \mathbf{\color{red}{10}}\times \mathbf{\color{red}{10}}\times \mathbf{\color{red}{10}}\times \mathbf{\color{red}{10}}}$

% \medskip
% Il y a donc $\mathbf{\color{blue}{4}}$ simplifications par $10$ possibles.

% \medskip
% $E=\dfrac{\mathbf{\color{blue}{\cancel{10}}} \times \mathbf{\color{blue}{\cancel{10}}}\times \mathbf{\color{blue}{\cancel{10}}}\times \mathbf{\color{blue}{\cancel{10}}}}{\mathbf{\color{red}{\cancel{10}}} \times \mathbf{\color{red}{\cancel{10}}}\times \mathbf{\color{red}{\cancel{10}}}\times \mathbf{\color{red}{\cancel{10}}}\times\mathbf{\color{red}{10}} \times \mathbf{\color{red}{10}}\times \mathbf{\color{red}{10}}}$

% \medskip
% $E=\dfrac{1}{10^{7-4}}=\dfrac{1}{10^{3}}=\psshadowbox{10^{-3}}$
% \end{enumerate}
% }
