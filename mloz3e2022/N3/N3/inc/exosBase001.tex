\begin{exercice*}
    Simplifier les écritures en utilisant la notation puissance.    

    $A=10\times 10 \times 10\times 10\times 10\times 10\times 10\times 10$
    \begin{multicols}{2}
        \begin{itemize}
            \begin{spacing}{1.5}                     
                \item[] $B=2\times 2\times 2\times 2\times 2$
                \item[] $C=\num{0.1}$
                \item[] $D=\num{0.000001}$
                \item[] $E=\num{1000000000}$        
            \end{spacing}
        \end{itemize}  
    \end{multicols}  
\end{exercice*}
\begin{corrige}
    %\setcounter{partie}{0} % Pour s'assurer que le compteur de \partie est à zéro dans les corrigés
    %\phantom{rrr}    
    Écrire chaque expression sous la forme d'une puissance d'un nombre.
    \begin{list}{}{}
        \item[] $A=10\times 10 \times 10\times 10\times 10\times 10\times 10\times 10$
        
        {\red $A=10^8$}
        \item[] $B=2\times 2\times 2\times 2\times 2$
        
        {\red $B=2^5$}
        \item[] $C=\num{0.1}$
        
        {\red $C=10^{-1}$}
        \item[] $D=\num{0.000001}$
        
        {\red $D=10^{-6}$}
        \item[] $E=\num{1000000000}$
        
        {\red $E=10^9$}
    \end{list}
    \vspace*{-10mm}
\end{corrige}

