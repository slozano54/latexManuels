\begin{exercice*}
    \begin{enumerate}
        \item Compléter avec le préfixe correspondant à cette 
        
        puissance de 10.
        \begin{multicols}{3}
            \begin{enumerate}
                \item $10^{6}$
                \item $10^{9}$
                \item $10^{-6}$
            \end{enumerate}
        \end{multicols}
        \item  Compléter avec la puissance de 10 correspondant 
        
        à ce préfixe.   
        \begin{multicols}{3}
            \begin{enumerate}
                \item déca
                \item Mega
                \item Tera
            \end{enumerate}
        \end{multicols}
    \end{enumerate}

    \hrefMathalea{https://coopmaths.fr/mathalea.html?ex=4C30-4,s=1,n=3,i=0&ex=4C30-4,s=2,n=3,i=0&v=l}
\end{exercice*}
\begin{corrige}
    %\setcounter{partie}{0} % Pour s'assurer que le compteur de \partie est à zéro dans les corrigés
    %\phantom{rrr}    
    \begin{enumerate}
        \item Compléter avec le préfixe correspondant à cette 
        
        puissance de 10.    

            \begin{enumerate}
                \item $10^{6}$  {\red c'est un million donc : Mega.}
                \item $10^{9}$  {\red c'est un milliard donc : Giga.}
                \item $10^{-6}$ {\red c'est un millionième donc : micro.}
            \end{enumerate}
            \columnbreak
            \setcounter{enumi}{1}
                \item  Compléter avec la puissance de 10 correspondant à ce préfixe. 

            \begin{enumerate}
                \item déca, {\red c'est dix soit $10^{1}$.}
                \item Mega, {\red c'est un million soit $10^{6}$.}
                \item Tera, {\red c'est mille-milliards soit $10^{12}$.}
            \end{enumerate}
    \end{enumerate}
\end{corrige}

