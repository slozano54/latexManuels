\begin{exercice*}
    Dans un litre de sang, il y a environ 4 500 milliards de globules rouges.
    \begin{enumerate}
        \item Exprimer ce nombre en écriture scientifique.
        \item Le corps d’un adulte est composé en moyenne de 5 litres de sang.
        
        En déduire la quantité moyenne de globules rouges pour un adulte.
    \end{enumerate}    
\end{exercice*}
\begin{corrige}
    %\setcounter{partie}{0} % Pour s'assurer que le compteur de \partie est à zéro dans les corrigés
    %\phantom{rrr}        
    Dans un litre de sang, il y a environ 4 500 milliards de globules rouges.
    \begin{enumerate}
        \item Exprimer ce nombre en écriture scientifique.
        
        {\red $\num{4500}$ millards = $\num{4.5}\times 10^3\times 10^9 = \num{4.5}\times 10^{12}$}
        \item Le corps d’un adulte est composé en moyenne de 5 litres de sang.
        
        En déduire la quantité moyenne de globules rouges pour un adulte.

        {\red Pour un adulte, la quantité moyenne de globules rouges est : $5\times \num{4.5}\times 10^{12} = \num{22.5}\times 10^{12}$}
    \end{enumerate}    
\end{corrige}

