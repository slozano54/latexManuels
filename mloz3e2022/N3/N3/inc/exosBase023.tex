\begin{exercice*}    
    Une année-lumière (symbole al) est égale à la distance parcourue par la lumière dans le vide
    pendant une année. Elle vaut environ $\num{10000}$ milliards de kilomètres. Un « parsec » (symbole pc) vaut
    environ $\num{3.2616}$ al.
    \begin{enumerate}
        \item Exprimer un parsec en kilomètres.
        \item Exprimer un kilomètre en parsec.
        \item L'étoile la plus proche du Soleil, Proxima Centauri, se trouve à $\num{1.316}$ pc de la Terre. Calcule
        le temps mis par la lumière pour nous parvenir de cette étoile.
    \end{enumerate}
    
\end{exercice*}
\begin{corrige}
    %\setcounter{partie}{0} % Pour s'assurer que le compteur de \partie est à zéro dans les corrigés
    %\phantom{rrr}        
    Une année-lumière (symbole al) est égale à la distance parcourue par la lumière dans le vide
    pendant une année. Elle vaut environ $\num{10000}$ milliards de kilomètres. Un « parsec » (symbole pc) vaut
    environ $\num{3.2616}$ al.

    \begin{enumerate}
        \item Exprimer un parsec en kilomètres.
        
        {\red $\num{1000}$ milliards de \Lg[km]{} = $10^4\times 10^9$ \Lg[km]{} = $10^{13}$ \Lg[km]{} 
        donc $1$ pc = $\num{3.2616}$ al = $\num{3.2616}\times 10^{13}$ \Lg[km]{}.        
        }
        \item Exprimer un kilomètre en parsec.
        
        {\red \Lg[km]{1} = $\dfrac{1}{\num{3.2616}\times 10^{13}}$ \Lg[km]{} $\approx 3\times 10^{-14}$ pc
        }
        \item L'étoile la plus proche du Soleil, Proxima Centauri, se trouve à $\num{1.316}$ pc de la Terre. Calcule
        le temps mis par la lumière pour nous parvenir de cette étoile.

        {\red $\num{1.316}$ pc = $\num{1.316}\times \num{3.2616} \times 10^{13}$ \Lg[km]{} $\approx \num{4.29}\times 10^{13}$ \Lg[km]{}

        Il faut environ \num{4.3} ans pour atteindre cette étoile.        
        }
    \end{enumerate}
\end{corrige}

