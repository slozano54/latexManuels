\begin{exercice*}    
    Écrire sous la forme d'une puissance d'un nombre.
    \begin{changemargin}{0mm}{-10mm}
    \begin{multicols}{2}
        \begin{itemize}
            \item[] $A=2^4\times 2^{-3}$ \medskip
            \item[] $B=(-3)^{-4}\times (-3)^{-1}$ \medskip
            \item[] $C=(-4)^{-2}\times (-4)^4$ 
            \columnbreak
            \item[] $D=\left(\dfrac{1}{5}\right)^2\times 5^{-3}$\medskip
            \item[] $E=\left(\dfrac{1}{4}\right)\times 4^{-5}$\medskip
            \item[] $F=\left(\dfrac{2}{3}\right)^{-4}\times \left(\dfrac{3}{2}\right)^5$
        \end{itemize}
    \end{multicols}
    \end{changemargin}
    \hrefMathalea{https://coopmaths.fr/mathalea.html?ex=4C33-1,s=1,n=4,cd=1,i=0&v=l}
\end{exercice*}
\begin{corrige}
    %\setcounter{partie}{0} % Pour s'assurer que le compteur de \partie est à zéro dans les corrigés
    %\phantom{rrr}    
    Écrire sous la forme d'une puissance d'un nombre.
        \begin{spacing}{2}
            $A=2^4\times 2^{-3}$                                                   {\red = $2^{4-3}=2^1=2$} \\
            $B=(-3)^{-4}\times (-3)^{-1}$                                          {\red = $(-3)^{-4-1}=(-3)^{-5}$} \\
            $C=(-4)^{-2}\times (-4)^4$                                             {\red = $(-4)^{-2+4}=(-4)^2$}                                                                        
            $D=\left(\dfrac{1}{5}\right)^2\times 5^{-3}$                           {\red = $5^{-2}\times 5^{-3}=5^{-2-3}=5^{-5}$}  \\
            $E=\left(\dfrac{1}{4}\right)\times 4^{-5}$                             {\red = $4^{-1}\times 4^{-5}=4^{-1-5}=4^{-6}$}  \\
            $F=\left(\dfrac{2}{3}\right)^{-4}\times \left(\dfrac{3}{2}\right)^5$   \\
            {\red $F=\left(\dfrac{3}{2}\right)^4\times \left(\dfrac{3}{2}\right)^5=\left(\dfrac{3}{2}\right)^{4+5}=\left(\dfrac{3}{2}\right)^9$}  
        \end{spacing}
\end{corrige}


