\section{Relations trigonométriques}
% \subsubsection{Propriété fondamentale de la trigonométrie}
% \proprNum{}{dans un triangle rectangle, $x$ désigne la mesure de l'un des angles aigus}{$(\sin x)^2+(\cos x)^2=1$}

% \Remarques[Remarque]{On note également $(\sin x)^2=\sin^2x$ et $(\cos x)^2=\cos^2x$.}

% \Preuve{
% \compo{1}{courstrigo}{1}{
% ABC est un triangle rectangle en A.\par
% $U=(\sin \widehat{ACB})^2+(\cos \widehat{ACB})^2=(\dfrac{AB}{BC})^2+(\dfrac{AC}{BC})^2$\par
% $U=\dfrac{AB^2}{BC^2}+\dfrac{AC^2}{BC^2}=\dfrac{AB^2+AC^2}{BC^2}$\par\vspace{0.25cm}
% avec le théorème de Pythagore\par\vspace{0.25cm}
% $U=\dfrac{BC^2}{BC^2}=1$ $\square$
% }
% }

% \subsubsection{Propriété de la tangente}
% \proprNum{}{dans un triangle rectangle, $x$ désigne la mesure de l'un des angles aigus}{$\tan x=\dfrac{\sin x}{\cos x}$}

% \Preuve{
% \compo{1}{courstrigo}{1}{
% ABC est un triangle rectangle en A.\par
% $\tan(\widehat{ACB})=\dfrac{AB}{AC}$ par définition\par\vspace{0.25cm}
% $\dfrac{\sin (\widehat{ACB})}{\cos (\widehat{ACB})}=\dfrac{\dfrac{AB}{BC}}{\dfrac{AC}{BC}}=\dfrac{AB}{BC}\times \dfrac{BC}{AC}=\dfrac{AB}{AC}$ $\square$
% }
% }

