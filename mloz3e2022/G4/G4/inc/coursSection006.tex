\section{Relations trigonométriques}
\begin{propriete}[Propriété fondamentale de la trigonométrie]
    Si dans un triangle rectangle, $x$ désigne la mesure de l'un des angles aigus alors 
    $$(\sin x)^2+(\cos x)^2=1$$
\end{propriete}

\begin{remarque}
    On note également $(\sin x)^2=\sin^2x$ et $(\cos x)^2=\cos^2x$.
\end{remarque}

\begin{preuve}

    \begin{minipage}{0.4\linewidth}
        \begin{Geometrie}[CoinHD={(5u,4.5u)}]
            u:=0.75*u;            
            pair A,B,C;
            A=u*(1.75,1.75);
            B=u*(1.75,5.25);
            C=u*(6.25,1.75);
            trace polygone(A,B,C);
            remplis codeperp(B,A,C,8)--A--cycle withcolor noir;
            trace codeperp(B,A,C,8);
            trace appelation(A,B,8mm,btex Côté opposé etex);
            trace appelation(A,B,3mm,btex à l'angle $\widehat{ACB}$ etex);
            trace appelation(A,C,-3mm,btex Côté adjacent etex);
            trace appelation(A,C,-8mm,btex à l'angle $\widehat{ACB}$etex);
            trace appelation(B,C,3mm,btex Hypoténuse etex);            
            trace marqueangle(B,C,A,0);
            trace marqueangle(A,B,C,0);
            marque_a:=marque_a*1.1;    
            trace marqueangle(A,B,C,0);
            label.llft(btex A etex,A);
            label.lrt(btex C etex,C);
            label.top(btex B etex,B);
        \end{Geometrie}
    \end{minipage}
    \begin{minipage}{0.55\linewidth}
        ABC est un triangle rectangle en A.\\\bigskip
        $U=(\sin \widehat{ACB})^2+(\cos \widehat{ACB})^2=(\dfrac{AB}{BC})^2+(\dfrac{AC}{BC})^2$\\\bigskip
        $U=\dfrac{AB^2}{BC^2}+\dfrac{AC^2}{BC^2}=\dfrac{AB^2+AC^2}{BC^2}$\\\bigskip
        avec le théorème de Pythagore\\\bigskip
        $U=\dfrac{BC^2}{BC^2}=1$ $\square$
    \end{minipage}
\end{preuve}

\begin{propriete}[Propriété de la tangente]
    Si dans un triangle rectangle, $x$ désigne la mesure de l'un des angles aigus alors 
    $$\tan x=\dfrac{\sin x}{\cos x}$$
\end{propriete}

\begin{preuve}

    \begin{minipage}{0.4\linewidth}
        \begin{Geometrie}[CoinHD={(5u,4.5u)}]
            u:=0.75*u;            
            pair A,B,C;
            A=u*(1.75,1.75);
            B=u*(1.75,5.25);
            C=u*(6.25,1.75);
            trace polygone(A,B,C);
            remplis codeperp(B,A,C,8)--A--cycle withcolor noir;
            trace codeperp(B,A,C,8);
            trace appelation(A,B,8mm,btex Côté opposé etex);
            trace appelation(A,B,3mm,btex à l'angle $\widehat{ACB}$ etex);
            trace appelation(A,C,-3mm,btex Côté adjacent etex);
            trace appelation(A,C,-8mm,btex à l'angle $\widehat{ACB}$etex);
            trace appelation(B,C,3mm,btex Hypoténuse etex);            
            trace marqueangle(B,C,A,0);
            trace marqueangle(A,B,C,0);
            marque_a:=marque_a*1.1;    
            trace marqueangle(A,B,C,0);
            label.llft(btex A etex,A);
            label.lrt(btex C etex,C);
            label.top(btex B etex,B);
        \end{Geometrie}
    \end{minipage}
    \begin{minipage}{0.55\linewidth}
        ABC est un triangle rectangle en A.\\\bigskip
        $\tan(\widehat{ACB})=\dfrac{AB}{AC}$ par définition\\\bigskip
        $\dfrac{\sin (\widehat{ACB})}{\cos (\widehat{ACB})}=\dfrac{\dfrac{AB}{BC}}{\dfrac{AC}{BC}}=\dfrac{AB}{BC}\times \dfrac{BC}{AC}=\dfrac{AB}{AC}$ $\square$
    \end{minipage}
\end{preuve}
