\section{Deux moyens mnémotchniques}

% \CadreLampe{Retenir sans m\'elanger!}{
% En d\'esignant chaque mot par sa premi\`{e}re lettre,\\
% \begin{multicols}2
% \begin{itemize}
% \item[] S représente le Sinus
% \item[] C représente le Cosinus
% \item[] T représente la Tangente
% \item[] O représente le côté Opposé
% \item[] A représente le côté Adjacent
% \item[] H représente l'Hypoténuse
% \end{itemize}
% \end{multicols}
% \par\vspace{0.5cm}
% On peut, par exemple, retenir les relations trigonom\'etriques de la fa\c con suivante :
% \begin{enumerate}
% \item \underline{\`{A} la TAXI} : SOH - CAH - TOA
% \item \underline{\`{A} la familière} : CAH - SOH - TOA ("casse toi !")
% \end{enumerate}
% }