\section{Deux moyens mnémotchniques}
\begin{center}
    \begin{myBox}{Retenir sans mélanger !}
        En désignant chaque mot par sa première lettre,\\
        \begin{multicols}{2}
            S représente le Sinus\\
            C représente le Cosinus\\
            T représente la Tangente\\
            O représente le côté Opposé\\
            A représente le côté Adjacent\\
            H représente l'Hypoténuse\\        
        \end{multicols}
        
        \bigskip
        On peut, par exemple, retenir les relations trigonométriques de la fa\c con suivante :\\
        \underline{À la TAXI} : SOH - CAH - TOA\\
        \underline{À la familière} : CAH - SOH - TOA ("casse toi !")        
    \end{myBox}
\end{center}
