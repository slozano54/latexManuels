\section{Cosinus d'un angle aigu}
% \definNum{Dans un triangle RECTANGLE, le \textbf{cosinus} d'un angle aigu est égal au rapport du côté adjacent à cet angle aigu par l'hypoténuse.}

% \Remarques[Remarque]{Pour tout angle aigu, le cosinus est compris entre 0 et 1 car l'hypot\'enuse est le plus grand c\^{o}t\'e d'un triangle rectangle!}

% \CadreLampe{Cosinus}{
% Dans le triangle $ABC$ rectangle en $A$, on a : 
% $$\cos\widehat{ABC}=\frac{\mbox{côté adjacent à l'angle $\widehat{ABC}$}}{\mbox{hypoténuse}}=\frac{AB}{BC}$$
% $$\cos\widehat{ACB}=\frac{\mbox{côté adjacent à l'angle $\widehat{ACB}$}}{\mbox{hypoténuse}}=\frac{AC}{BC}$$
% }