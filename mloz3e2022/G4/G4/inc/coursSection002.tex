\section{Cosinus d'un angle aigu}
\begin{definition}
    Dans un triangle {\bfseries rectangle}, le \textbf{cosinus} d'un angle aigu est égal au rapport du côté adjacent à cet angle aigu par l'hypoténuse.
\end{definition}

\begin{remarque}
    Pour tout angle aigu, le cosinus est compris entre 0 et 1 car l'hypoténuse est le plus grand côté d'un triangle rectangle!
\end{remarque}

\begin{exemples*1}
    \titreExemple{Cosinus}

    \medskip
    Dans le triangle $ABC$ rectangle en $A$, on a : 
    $$\cos\widehat{ABC}=\frac{\mbox{côté adjacent à l'angle $\widehat{ABC}$}}{\mbox{hypoténuse}}=\frac{AB}{BC}$$
    $$\cos\widehat{ACB}=\frac{\mbox{côté adjacent à l'angle $\widehat{ACB}$}}{\mbox{hypoténuse}}=\frac{AC}{BC}$$
\end{exemples*1}