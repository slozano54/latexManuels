\begin{exercice*}
    $MNP$ est un triangle rectangle en $M$ tel que :\\$PN=\Lg[cm]{5.4}$ et $\widehat{MPN}=\ang{42}$.\\
    Compléter la trame pour calculer la longueur $MP$.
    \begin{center}
        \begin{Geometrie}[CoinHD={(6u,4.5u)}]        
            pair M,N,P;
            M=u*(4,4);
            N-M=u*(1.5,-2.5);
            P=1.2[M,rotation(N,M,-90)];
            trace polygone(M,N,P);            
            trace codeperp(P,M,N,8);
            trace Codeangle(N,P,M,0,btex \ang{42} etex);
            trace appelation(P,N,3mm,btex \Lg[cm]{5.4} etex);
            label.lft(btex P etex,P);
            label.top(btex M etex,M);
            label.lrt(btex N etex,N);
        \end{Geometrie}
    \end{center}
    Dans le triangle $MNP$ \makebox[0.2\linewidth]{\dotfill} en \makebox[0.1\linewidth]{\dotfill}, on connaît :
    \begin{itemize}        
        \item l'angle $\widehat{\phantom{M}\dots\phantom{N}}=\makebox[0.2\linewidth]{\dotfill}$,
        \item l'hypoténuse : $PN=\makebox[0.2\linewidth]{\dotfill}$,
    \end{itemize}
    et on cherche le côté \makebox[0.3\linewidth]{\dotfill} à l'angle $\widehat{MPN}$. On utilise donc \makebox[0.2\linewidth]{\dotfill}.
    \begin{align*}
        \makebox[0.3\linewidth]{\dotfill}&=\frac{\makebox[0.2\linewidth]{\dotfill}}{\makebox[0.2\linewidth]{\dotfill}}\\
        &\\
        \makebox[0.3\linewidth]{\dotfill}&=\makebox[0.2\linewidth]{\dotfill}\\
        &\\
        \intertext{\centering\bf Les produits en croix sont égaux}
        \makebox[0.3\linewidth]{\dotfill}&=\makebox[0.2\linewidth]{\dotfill}\\        
        \makebox[0.3\linewidth]{\dotfill}&\simeq\makebox[0.2\linewidth]{\dotfill}
    \end{align*}
\end{exercice*}
\begin{corrige}
    %\setcounter{partie}{0} % Pour s'assurer que le compteur de \partie est à zéro dans les corrigés
    % \phantom{rrr}    
    $MNP$ est un triangle rectangle en $M$ tel que :\\$PN=\Lg[cm]{5.4}$ et $\widehat{MPN}=\ang{42}$.\\
    Compléter la trame pour calculer la longueur $MP$.
    \begin{center}
        \begin{Geometrie}[CoinHD={(6u,4.5u)}]        
            pair M,N,P;
            M=u*(4,4);
            N-M=u*(1.5,-2.5);
            P=1.2[M,rotation(N,M,-90)];
            trace polygone(M,N,P);            
            trace codeperp(P,M,N,8);
            trace Codeangle(N,P,M,0,btex \ang{42} etex);
            trace appelation(P,N,3mm,btex \Lg[cm]{5.4} etex);
            label.lft(btex P etex,P);
            label.top(btex M etex,M);
            label.lrt(btex N etex,N);
        \end{Geometrie}
    \end{center}
    \Coupe
    Dans le triangle $MNP$ {\red rectangle} en {\red $M$}, on connaît :
    \begin{itemize}        
        \item l'angle {\red $\widehat{MPN}=\ang{42}$},
        \item l'hypoténuse : $PN={\red\Lg[cm]{5.4}}$,
    \end{itemize}
    et on cherche le côté {\red adjacent} à l'angle $\widehat{MPN}$. On utilise donc {\red le cosinus}.
    \begin{align*}
        {\red \cos(\widehat{MPN})}&=\frac{\red MP}{\red PN}\\        
        {\red \cos(\ang{42})}&=\frac{\red MP}{\red \num{5.4}}\\        
        \intertext{\centering\bf Les produits en croix sont égaux}
        {\red MP}&={\red \cos(\ang{42})\times \num{5.4}}\\        
        {\red MP}&\simeq {\red\Lg[cm]{4}}
    \end{align*}
\end{corrige}

