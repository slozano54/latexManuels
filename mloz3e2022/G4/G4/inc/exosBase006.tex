\begin{exercice*}
    À l'aide de la calculatrice, calculer et arrondir au centième, le cosinus, le sinus et la tangente des angles donnés.

    \begin{tabular}{|>{\columncolor{LightGray}}p{0.17\linewidth}|*{5}{@{}>{\vrule width0pt height\dimexpr.4cm-.2pt\relax depth\dimexpr.1cm-.2pt\relax\centering\arraybackslash}p{\dimexpr13mm-.4pt\relax}@{}|}}
        \hline
        Angle&\ang{20}&\ang{30}&\ang{45}&\ang{60}&\ang{83}\\\hline
        Cosinus&&&&&\\\hline
        Sinus&&&&&\\\hline
        Tangente&&&&&\\\hline
    \end{tabular}    
\end{exercice*}
\begin{corrige}
    %\setcounter{partie}{0} % Pour s'assurer que le compteur de \partie est à zéro dans les corrigés
    % \phantom{rrr}    
    À l'aide de la calculatrice, calculer et arrondir au centième, le cosinus, le sinus et la tangente des angles donnés.

    \begin{tabular}{|>{\columncolor{LightGray}}p{0.18\linewidth}|*{5}{@{}>{\vrule width0pt height\dimexpr.4cm-.2pt\relax depth\dimexpr.1cm-.2pt\relax\centering\arraybackslash}p{\dimexpr13mm-.4pt\relax}@{}|}}
        \hline
        Angle&\ang{20}&\ang{30}&\ang{45}&\ang{60}&\ang{83}\\\hline
        Cosinus &{\red \num{0.94}}&{\red \num{0.87}}&{\red \num{0.71}}&{\red \num{0.5} }&{\red \num{0.12}}\\\hline
        Sinus   &{\red \num{0.34}}&{\red \num{0.5} }&{\red \num{0.71}}&{\red \num{0.87}}&{\red \num{0.99}}\\\hline
        Tangente&{\red \num{0.36}}&{\red \num{0.58}}&{\red \num{1}   }&{\red \num{1.73}}&{\red \num{8.14}}\\\hline
    \end{tabular}  
\end{corrige}

