\begin{exercice*}
    Utiliser la couleur demandée pour repasser sur les côtés indiqués.
    \begin{enumerate}
        \item Le côté opposé à l'angle $\widehat{MNO}$ en bleu.
        
        \medskip        
        \begin{Geometrie}[CoinHD={(6u,4.5u)}]        
            pair M,O,N;
            O=u*(3,1);
            N-O=u*(2,1.5);
            M=0.65[O,rotation(N,O,90)];
            trace polygone(M,O,N);            
            trace codeperp(N,O,M,8);        
            trace marqueangle(M,N,O,0);
            label.bot(btex O etex,O);
            label.rt(btex N etex,N);
            label.top(btex M etex,M);
        \end{Geometrie}
        \item L'hypoténuse en rouge, et le côté opposé à l'angle $\widehat{SRT}$ en bleu.
        
        \medskip        
        \begin{Geometrie}[CoinHD={(6u,4.5u)}]        
            pair S,R,T;
            S=u*(5,1);
            R-S=u*(-0.5,2.5);
            T=1.5[S,rotation(R,S,90)];
            trace polygone(S,R,T);            
            trace codeperp(R,S,T,8);        
            trace marqueangle(T,R,S,0);
            label.bot(btex S etex,S);
            label.rt(btex R etex,R);
            label.lft(btex T etex,T);
        \end{Geometrie}
        \item L'hypoténuse en rouge, et le côté adjacent à l'angle $\widehat{WXY}$ en bleu.
        
        \medskip        
        \begin{Geometrie}[CoinHD={(6u,4.5u)}]        
            pair W,X,Y;
            W=u*(1,3);
            X-W=u*(0.2,-2);
            Y=1.5[W,rotation(X,W,90)];
            trace polygone(W,X,Y);            
            trace codeperp(X,W,Y,8);
            label.ulft(btex W etex,W);
            label.llft(btex X etex,X);
            label.urt(btex Y etex,Y);
        \end{Geometrie}
    \end{enumerate}
\end{exercice*}
\begin{corrige}
    %\setcounter{partie}{0} % Pour s'assurer que le compteur de \partie est à zéro dans les corrigés
    % \phantom{rrr}    
    Utiliser la couleur demandée pour repasser sur les côtés indiqués.
    \begin{enumerate}
        \item Le côté opposé à l'angle $\widehat{MNO}$ en bleu.
        
        \medskip        
        \begin{Geometrie}[CoinHD={(6u,4.5u)}]        
            pair M,O,N;
            O=u*(3,1);
            N-O=u*(2,1.5);
            M=0.65[O,rotation(N,O,90)];
            trace polygone(M,O,N);            
            trace codeperp(N,O,M,8);        
            trace marqueangle(M,N,O,0);
            label.bot(btex O etex,O);
            label.rt(btex N etex,N);
            label.top(btex M etex,M);
            % Correction
            trace segment(M,O) withcolor blue withpen pencircle scaled 2bp;
        \end{Geometrie}
        \item L'hypoténuse en rouge, et le côté opposé à l'angle $\widehat{SRT}$ en bleu.
        
        \medskip
        \scalebox{0.7}{        
        \begin{Geometrie}[CoinHD={(6u,4.5u)}]        
            pair S,R,T;
            S=u*(5,1);
            R-S=u*(-0.5,2.5);
            T=1.5[S,rotation(R,S,90)];
            trace polygone(S,R,T);
            trace codeperp(R,S,T,8);        
            trace marqueangle(T,R,S,0);
            label.bot(btex S etex,S);
            label.rt(btex R etex,R);
            label.lft(btex T etex,T);
            % Correction
            trace segment(R,T) withcolor red withpen pencircle scaled 2bp;
            trace segment(S,T) withcolor blue withpen pencircle scaled 2bp;
        \end{Geometrie}
        }
        \item L'hypoténuse en rouge, et le côté adjacent à l'angle $\widehat{WXY}$ en bleu.
        
        \medskip        
        \begin{Geometrie}[CoinHD={(6u,4.5u)}]        
            pair W,X,Y;
            W=u*(1,3);
            X-W=u*(0.2,-2);
            Y=1.5[W,rotation(X,W,90)];
            trace polygone(W,X,Y);            
            trace codeperp(X,W,Y,8);
            label.ulft(btex W etex,W);
            label.llft(btex X etex,X);
            label.urt(btex Y etex,Y);
            % Correction
            trace segment(X,Y) withcolor red withpen pencircle scaled 2bp;
            trace segment(W,X) withcolor blue withpen pencircle scaled 2bp;
        \end{Geometrie}
    \end{enumerate}
\end{corrige}

