\section{Applications}
\begin{remarque}
    Pour les calculs, on utilise le mode \og{}{\bf degré}\fg{} de la calculatrice.
\end{remarque}

% \begin{methode*1}[Titre]
%     texte
%     \exercice
%     texte

%     \Trigo[Cosinus,FigureSeule,Angle=-90,Echelle=6.5mm]{RST}{2.2}{}{40}
%     \correction
%     texte
% \end{methode*1}

% \begin{methode*1}[Cosinus - Calculer la longueur d'un des côtés de l'angle droit]

%     \exercice    
%         {\Trigo[Cosinus,FigureSeule,Angle=-90,Echelle=6.5mm]{RST}{2.2}{}{40}}
%     \correction
%       \dots

% \end{methode*1}

% \compo{3}{courscosinus}{1}{
% Le triangle $RST$ est rectangle en $R$ donc
% $$\Eqalign{
% \cos\widehat{RST}&=\frac{RS}{ST}\cr
% {\color{red}\frac{\psframebox[linestyle=dashed]{\rnode{A}{\color{black}\cos38}}}{\rnode{D}{1}}}\,&=\,\frac{\rnode{B}{RS}}{\rnode{C}{6}}\cr
% }$$
% \begin{center}
% {\bf Les produits en croix sont \'egaux}
% \end{center}
% \ncline[linestyle=dashed,linewidth=0.1mm]{<->}{A}{C}
% \ncline[linestyle=dashed,linewidth=0.1mm]{<->}{B}{D}
% $$\Eqalign{
% RS&=6\times\psframebox[linestyle=dashed]{\cos38}\cr
% RS&\simeq4,73\,cm\cr
% }$$
% }
% }

% \Exemples[Cas n$\degres$ 2 : L'hypot\'enuse]{}{
% \compo{4}{courscosinus}{1}{
% Le triangle $EFG$ est rectangle en $G$ donc
% $$\Eqalign{
% \cos\widehat{EFG}&=\frac{EG}{EF}\cr
% {\color{red}\frac{\psframebox[linestyle=dashed]{\rnode{A}{\color{black}\cos60}}}{\rnode{D}{1}}}\,&=\,\frac{\rnode{B}{6}}{\rnode{C}{EF}}\cr
% }$$
% \begin{center}
% {\bf Les produits en croix sont \'egaux}
% \end{center}
% \ncline[linestyle=dashed,linewidth=0.1mm]{<->}{A}{C}
% \ncline[linestyle=dashed,linewidth=0.1mm]{<->}{B}{D}
% $$\Eqalign{
% EF\times\psframebox[linestyle=dashed]{\cos60}&=6\cr
% EF&=\frac{6}{\psframebox[linestyle=dashed]{\cos60}}\cr
% EF&=12\,cm\cr
% }$$
% }
% }

% \subsubsection{Cosinus - Calculer un angle}
% \Exemples[Exemple]{}{
% \compo{5}{courscosinus}{1}{
% Le triangle $IJK$ est rectangle en $J$ donc
% $$\Eqalign{
% \cos\widehat{KIJ}&=\frac{IJ}{IK}\cr
% \cos\widehat{KIJ}&=\frac{6}{10}\cr
% \widehat{KIJ}&=\cos^{-1}\left(\frac{6}{10}\right)\cr
% \widehat{KIJ}&\simeq53\degres\cr
% }$$
% }
% }

% \subsubsection{Tangente et Sinus - Calculer une longueur}
% \Exemples{}{
% \compo{3}{courscosinus}{1}{
% Dans le triangle $ABC$ rectangle en $C$, on a : 
% $$\Eqalign{
% \tan\widehat{CAB}&=\frac{BC}{AC}\cr
% \tan30&=\frac{BC}{4}\cr
% }$$
% \begin{center}
% {\bf Les produits en croix sont \'egaux}
% \end{center}
% $$\Eqalign{BC&=4\times\tan30\cr
% }$$
% \begin{center}
% \psshadowbox{$BC\simeq2,3\,cm$}
% \end{center}
% }

% \compo{4}{courscosinus}{1}{
% Dans le triangle $RST$ rectangle en $S$, on a :
% $$\Eqalign{
% \sin\widehat{RTS}&=\frac{RS}{RT}\cr
% \sin63&=\frac{4,1}{RT}\cr
% }$$
% \begin{center}
% {\bf Les produits en croix sont \'egaux}
% \end{center}
% $$\Eqalign{
% RT\times\sin63&=4,1\cr
% RT&=\frac{4,1}{\sin63}\cr
% }$$
% \begin{center}
% \psshadowbox{$RT\simeq4,6\,cm$}
% \end{center}
% }
% %\end{multicols}
% }
% \subsubsection{Tangente et Sinus - Calculer un angle}
% \columnseprule.4pt
% \Exemples{}{
% \begin{multicols}{2}
% $$\includegraphics[scale=0.8]{courstrigo.5}$$
% Dans le triangle $LOA$ rectangle en $O$, on a
% $$\Eqalign{
% \sin\widehat{LAO}&=\frac{OL}{AL}\cr
% \sin\widehat{LAO}&=\frac{1,3}{2}\cr
% \widehat{LAO}&=sin^{-1}\left(\frac{1,3}{2}\right)\cr
% }$$
% \begin{center}
% \psshadowbox{$\widehat{LAO}\simeq41\degres$}
% \end{center}
% $$\includegraphics{courstrigo.6}$$
% Dans le triangle $IJK$ rectangle en $I$, on a
% $$\Eqalign{
% \tan\widehat{IJK}&=\frac{IK}{IJ}\cr
% \tan\widehat{IJK}&=\frac{3}{2}\cr
% \widehat{IJK}&=tan^{-1}\left(\frac{3}{2}\right)\cr
% }$$
% \begin{center}
% \psshadowbox{$\widehat{IJK}\simeq56\degres$}
% \end{center}
% \end{multicols}
% \vspace{0.25cm}
% }
% \columnseprule0pt


