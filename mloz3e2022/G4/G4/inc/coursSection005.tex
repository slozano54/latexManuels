\section{Applications}
\begin{remarque}
    Pour les calculs, on utilise le mode \og{}{\bf degré}\fg{} de la calculatrice.
\end{remarque}

\begin{methode}[Cosinus - Calculer la longueur d'un des côtés de l'angle droit]
    \exercice    
    Dans le triangle $RST$, déterminer $RS$.\\
    \begin{Geometrie}[CoinHD={(6u,4.5u)}]        
        pair S,R,T;
        S=u*(1,1);
        R-S=u*(4,0);
        T=0.78[R,rotation(S,R,-90)];
        trace polygone(S,R,T);
        remplis codeperp(S,R,T,8)--R--cycle withcolor noir;
        trace codeperp(S,R,T,8);
        trace appelation(S,T,3mm,btex \Lg[cm]{6} etex);        
        trace cotation(S,R,-2mm,-3mm,btex ? etex);
        marque_a:=marque_a*1.5;
        trace Codeangle(R,S,T,0,btex \ang{38} etex);
        label.llft(btex S etex,S);
        label.lrt(btex R etex,R);
        label.top(btex T etex,T);
    \end{Geometrie}
    \correction
    Le triangle $RST$ est rectangle en $R$ donc
    $$\Eqalign{
    \cos\widehat{RST}&=\frac{RS}{ST}\cr
    {\color{red}\frac{\psframebox[linestyle=dashed]{\rnode{A}{\color{black}\cos38}}}{\rnode{D}{1}}}\,&=\,\frac{\rnode{B}{RS}}{\rnode{C}{6}}\cr
    }$$
    \begin{center}
    {\bf Les produits en croix sont égaux}
    \end{center}
    \ncline[linestyle=dashed,linewidth=0.1mm]{<->}{A}{C}
    \ncline[linestyle=dashed,linewidth=0.1mm]{<->}{B}{D}
    $$\Eqalign{
    RS&=6\times\psframebox[linestyle=dashed]{\cos38}\cr
    RS&\simeq\Lg[cm]{4.73}\cr
    }$$
\end{methode}

\begin{methode}[Cosinus - Calculer la longueur de l'hypoténuse]
    \exercice    
    Dans le triangle $EFG$, déterminer $EF$.\\
    \begin{Geometrie}[CoinHD={(4u,5u)}]        
        pair E,F,G;
        E=u*(1,1);
        G-E=u*(2,0);
        F=2[E,rotation(G,E,60)];
        trace polygone(E,F,G);
        remplis codeperp(E,G,F,8)--G--cycle withcolor noir;
        trace codeperp(E,G,F,8);
        trace appelation(E,G,-3mm,btex \Lg[cm]{6} etex);        
        trace cotation(E,F,2mm,3mm,btex ? etex);
        marque_a:=marque_a*1.5;
        trace Codeangle(G,E,F,0,btex \ang{60} etex);
        label.llft(btex E etex,E);
        label.lrt(btex G etex, G);
        label.top(btex F etex, F);
    \end{Geometrie}
    \correction
    Le triangle $EFG$ est rectangle en $G$ donc
    $$\Eqalign{
    \cos\widehat{EFG}&=\frac{EG}{EF}\cr
    {\color{red}\frac{\psframebox[linestyle=dashed]{\rnode{A}{\color{black}\cos60}}}{\rnode{D}{1}}}\,&=\,\frac{\rnode{B}{6}}{\rnode{C}{EF}}\cr
    }$$
    \begin{center}
    {\bf Les produits en croix sont égaux}
    \end{center}
    \ncline[linestyle=dashed,linewidth=0.1mm]{<->}{A}{C}
    \ncline[linestyle=dashed,linewidth=0.1mm]{<->}{B}{D}
    $$\Eqalign{
    EF\times\psframebox[linestyle=dashed]{\cos60}&=6\cr
    EF&=\frac{6}{\psframebox[linestyle=dashed]{\cos60}}\cr
    EF&=\Lg[cm]{12}\cr
    }$$
\end{methode}

\begin{methode}[Cosinus - Calculer un angle]
    \exercice    
    Dans le triangle $IJK$, déterminer $\widehat{KIJ}$.\\
    \begin{Geometrie}[CoinHD={(6u,4.5u)}]        
        pair I,J,K;
        K=u*(1,1);
        J-K=u*(4,0);
        I=0.78[J,rotation(K,J,-90)];
        trace polygone(K,J,I);
        remplis codeperp(K,J,I,8)--J--cycle withcolor noir;
        trace codeperp(K,J,I,8);
        trace appelation(K,I,3mm,btex \Lg[cm]{10} etex);
        trace appelation(I,J,3mm,btex \Lg[cm]{6} etex);        
        marque_a:=marque_a*1.5;
        trace Codeangle(K,I,J,0,btex ? etex);
        label.llft(btex K etex,K);
        label.lrt(btex J etex,J);
        label.top(btex I etex,I);
    \end{Geometrie}
    \correction
    Le triangle $IJK$ est rectangle en $J$ donc
    $$\Eqalign{
    \cos\widehat{KIJ}&=\frac{IJ}{IK}\cr
    \cos\widehat{KIJ}&=\frac{6}{10}\cr
    \widehat{KIJ}&=\cos^{-1}\left(\frac{6}{10}\right)\cr
    \widehat{KIJ}&\simeq\ang{53}\cr
    }$$
\end{methode}

\begin{methode}[Tangente - Calculer une longueur]
    \exercice
    Dans le triangle $RST$, déterminer $RT$.\\
    \begin{Geometrie}[CoinHD={(6u,4.5u)}]        
        pair S,R,T;
        S=u*(1,1);
        R-S=u*(4,0);
        T=0.78[R,rotation(S,R,-90)];
        trace polygone(S,R,T);
        remplis codeperp(S,R,T,8)--R--cycle withcolor noir;
        trace codeperp(S,R,T,8);
        trace appelation(S,R,-3mm,btex \Lg[cm]{4} etex);             
        marque_a:=marque_a*1.5;
        trace Codeangle(R,S,T,0,btex \ang{30} etex);
        label.llft(btex S etex,S);
        label.lrt(btex R etex,R);
        label.top(btex T etex,T);
    \end{Geometrie}
    \correction
    Dans le triangle $RST$ rectangle en $R$, on a : 
    $$\Eqalign{
    \tan\widehat{RST}&=\frac{RT}{RS}\cr
    \tan30&=\frac{RT}{4}\cr
    }$$
    \begin{center}
        {\bf Les produits en croix sont égaux}
    \end{center}
    $$\Eqalign{RT&=4\times\tan30\cr
    RT&\simeq \Lg[cm]{2.3}\cr
    }$$
\end{methode}

\begin{methode}[Sinus - Calculer une longueur]
    \exercice
    Dans le triangle $EFG$, déterminer $EF$.\\
    \begin{Geometrie}[CoinHD={(4u,5u)}]        
        pair E,F,G;
        E=u*(1,1);
        G-E=u*(2,0);
        F=2[E,rotation(G,E,60)];
        trace polygone(E,F,G);
        remplis codeperp(E,G,F,8)--G--cycle withcolor noir;
        trace codeperp(E,G,F,8);
        trace appelation(E,G,-3mm,btex \Lg[cm]{4.1} etex);        
        trace cotation(E,F,2mm,3mm,btex ? etex);
        marque_a:=marque_a*1.5;
        trace Codeangle(E,F,G,0,btex \ang{63} etex);
        label.llft(btex E etex,E);
        label.lrt(btex G etex, G);
        label.top(btex F etex, F);
    \end{Geometrie}
    \correction
    Dans le triangle $EFG$ rectangle en $G$, on a :
    $$\Eqalign{
    \sin\widehat{EFG}&=\frac{EG}{EF}\cr
    \sin63&=\frac{4,1}{EF}\cr
    }$$
    \begin{center}
    {\bf Les produits en croix sont égaux}
    \end{center}
    $$\Eqalign{
    EF\times\sin63&=4,1\cr
    EF&=\frac{4,1}{\sin63}\cr
    EF&\simeq\Lg[cm]{4.6}\cr
    }$$
\end{methode}

\begin{methode}[Tangente/Sinus - Calculer un angle]
    \exercice
    \begin{enumerate}    
        \item Déterminer $\widehat{LAO}$ dans le triangle $LAO$.
        
        \medskip
        \scalebox{1.2}{
        \begin{Geometrie}[CoinHD={(6u,4.5u)}]        
            pair L,O,A;
            O=u*(3,1);
            A-O=u*(2,1.5);
            L=0.65[O,rotation(A,O,90)];
            trace polygone(L,O,A);
            remplis codeperp(A,O,L,8)--O--cycle withcolor noir;
            trace codeperp(A,O,L,8);
            trace appelation(L,A,3mm,btex \Lg[cm]{2} etex);
            trace appelation(L,O,-3mm,btex \Lg[cm]{1.3} etex);            
            trace Codeangle(L,A,O,0,btex ? etex);
            label.bot(btex O etex,O);
            label.rt(btex A etex,A);
            label.top(btex L etex,L);
        \end{Geometrie}
        }
        \item Dans le triangle $IJK$, déterminer $\widehat{IJK}$.
        
        \medskip
        \scalebox{1.2}{
        \begin{Geometrie}[CoinHD={(6.5u,5u)}]        
            pair I,J,K;
            I=u*(3,3);
            J-I=u*(-1,-2);
            K=1.5[I,rotation(J,I,90)];
            trace polygone(I,J,K);
            remplis codeperp(J,I,K,8)--I--cycle withcolor noir;
            trace codeperp(J,I,K,8);
            trace appelation(J,I,3mm,btex \Lg[cm]{2} etex);
            trace appelation(I,K,3mm,btex \Lg[cm]{3} etex);
            trace Codeangle(K,J,I,0,btex ? etex);
            label.llft(btex J etex,J);
            label.lrt(btex K etex, K);
            label.top(btex I etex, I);
        \end{Geometrie}
        }
    \end{enumerate}
    \correction
    \begin{enumerate}
        \item Dans le triangle $LOA$ rectangle en $O$, on a
        $$\Eqalign{
        \sin\widehat{LAO}&=\frac{OL}{AL}\cr
        \sin\widehat{LAO}&=\frac{1,3}{2}\cr
        \widehat{LAO}&=sin^{-1}\left(\frac{1,3}{2}\right)\cr
        \widehat{LAO}&\simeq\ang{41}\cr
        }$$
        \item Dans le triangle $IJK$ rectangle en $I$, on a
        $$\Eqalign{
        \tan\widehat{IJK}&=\frac{IK}{IJ}\cr
        \tan\widehat{IJK}&=\frac{3}{2}\cr
        \widehat{IJK}&=tan^{-1}\left(\frac{3}{2}\right)\cr
        \widehat{IJK}&\simeq\ang{56}\cr
        }$$
    \end{enumerate}
\end{methode}
