\section{Applications}
\begin{remarque}
    Pour les calculs, on utilise le mode \og{}{\bf degré}\fg{} de la calculatrice.
\end{remarque}

\begin{methode}[Cosinus - Calculer la longueur d'un des côtés de l'angle droit]
    \exercice    
    Dans le triangle $RST$, déterminer $RS$.\\
    \begin{Geometrie}[CoinHD={(6u,4.5u)}]        
        pair S,R,T;
        S=u*(1,1);
        R-S=u*(4,0);
        T=0.78[R,rotation(S,R,-90)];
        trace polygone(S,R,T);
        remplis codeperp(S,R,T,8)--R--cycle withcolor noir;
        trace codeperp(S,R,T,8);
        trace appelation(S,T,3mm,btex \Lg[cm]{6} etex);        
        trace cotation(S,R,-2mm,-3mm,btex ? etex);
        marque_a:=marque_a*1.5;
        trace Codeangle(R,S,T,0,btex \ang{38} etex);
        label.llft(btex S etex,S);
        label.lrt(btex R etex,R);
        label.top(btex T etex,T);
    \end{Geometrie}
    \correction
    Dans le triangle $RST$ rectangle en $R$, on connaît :
    \begin{itemize}
        \item l'angle $\widehat{RST}=\ang{38}$,
        \item la longueur de l'hypoténuse : $ST=\Lg[cm]{6}$,
    \end{itemize}
    et on cherche la longueur du côté adjacent à l'angle $\widehat{RST}$.\\
    On utilise donc le cosinus.
    \begin{align*}
        \cos\widehat{RST}&=\frac{RS}{ST}\\
        {\color{red}\frac{\psframebox[linestyle=dashed]{\rnode{A}{\color{black}\cos38}}}{\rnode{D}{1}}}\,&=\,\frac{\rnode{B}{RS}}{\rnode{C}{6}}\\    
        \intertext{\centering\bf Les produits en croix sont égaux}
        RS&=6\times\psframebox[linestyle=dashed]{\cos38}\\
        RS&\simeq\Lg[cm]{4.73}
    \end{align*}
    \ncline[linestyle=dashed,linewidth=0.1mm]{<->}{A}{C}
    \ncline[linestyle=dashed,linewidth=0.1mm]{<->}{B}{D}
\end{methode}

\begin{methode*1}[Cosinus - Calculer la longueur de l'hypoténuse]
    \exercice    
    Dans le triangle $EFG$, déterminer $EF$.\\
    \begin{Geometrie}[CoinHD={(4u,5u)}]        
        pair E,F,G;
        E=u*(1,1);
        G-E=u*(2,0);
        F=2[E,rotation(G,E,60)];
        trace polygone(E,F,G);
        remplis codeperp(E,G,F,8)--G--cycle withcolor noir;
        trace codeperp(E,G,F,8);
        trace appelation(E,G,-3mm,btex \Lg[cm]{6} etex);        
        trace cotation(E,F,2mm,3mm,btex ? etex);        
        trace Codeangle(G,E,F,0,btex \ang{60} etex);
        label.llft(btex E etex,E);
        label.lrt(btex G etex, G);
        label.top(btex F etex, F);
    \end{Geometrie}
    \correction
    Dans le triangle $EFG$ rectangle en $G$, on connaît :
    \begin{itemize}
        \item l'angle $\widehat{FEG}=\ang{60}$,
        \item le côté adjacent à l'angle $\widehat{FEG}$ : $EG=\Lg[cm]{6}$,
    \end{itemize}
    et on cherche l'hypoténuse. On utilise donc le cosinus.
    \begin{align*}
        \cos\widehat{EFG}&=\frac{EG}{EF}\\
        {\color{red}\frac{\psframebox[linestyle=dashed]{\rnode{A}{\color{black}\cos60}}}{\rnode{D}{1}}}\,&=\,\frac{\rnode{B}{6}}{\rnode{C}{EF}}\\
        \intertext{\centering\bf Les produits en croix sont égaux}
        EF\times\psframebox[linestyle=dashed]{\cos60}&=6\\
        EF&=\frac{6}{\psframebox[linestyle=dashed]{\cos60}}\\
        EF&=\Lg[cm]{12}
    \end{align*}
    \ncline[linestyle=dashed,linewidth=0.1mm]{<->}{A}{C}
    \ncline[linestyle=dashed,linewidth=0.1mm]{<->}{B}{D}
\end{methode*1}

\begin{methode}[Cosinus - Calculer un angle]
    \exercice    
    Dans le triangle $IJK$, déterminer $\widehat{KIJ}$.\\
    \begin{Geometrie}[CoinHD={(6u,4.5u)}]        
        pair I,J,K;
        K=u*(1,1);
        J-K=u*(4,0);
        I=0.78[J,rotation(K,J,-90)];
        trace polygone(K,J,I);
        remplis codeperp(K,J,I,8)--J--cycle withcolor noir;
        trace codeperp(K,J,I,8);
        trace appelation(K,I,3mm,btex \Lg[cm]{10} etex);
        trace appelation(I,J,3mm,btex \Lg[cm]{6} etex);                
        trace Codeangle(K,I,J,0,btex ? etex);
        label.llft(btex K etex,K);
        label.lrt(btex J etex,J);
        label.top(btex I etex,I);
    \end{Geometrie}
    \correction
    Dans le triangle $IJK$ rectangle en $J$, on connaît :
    \begin{itemize}        
        \item l'hypoténuse : $IK=\Lg[cm]{10}$,
        \item le côté adjacent à l'angle $\widehat{KIJ}$ : $EG=\Lg[cm]{6}$,
    \end{itemize}
    et on cherche l'angle $\widehat{KIJ}$. On utilise donc le cosinus.
    \begin{align*}
        \cos\widehat{KIJ}&=\frac{IJ}{IK}\\
        \cos\widehat{KIJ}&=\frac{6}{10}\\
        \widehat{KIJ}&=\cos^{-1}\left(\frac{6}{10}\right)\\
        \widehat{KIJ}&\simeq\ang{53}
    \end{align*}
\end{methode}

\begin{methode*1}[Tangente - Calculer une longueur]
    \exercice
    Dans le triangle $RST$, déterminer $RT$.\\
    \begin{Geometrie}[CoinHD={(6u,4.5u)}]        
        pair S,R,T;
        S=u*(1,1);
        R-S=u*(4,0);
        T=0.78[R,rotation(S,R,-90)];
        trace polygone(S,R,T);
        remplis codeperp(S,R,T,8)--R--cycle withcolor noir;
        trace codeperp(S,R,T,8);
        trace appelation(S,R,-3mm,btex \Lg[cm]{4} etex);  
        trace cotation(T,R,2mm,3mm,btex ? etex);
        marque_a:=marque_a*1.5;
        trace Codeangle(R,S,T,0,btex \ang{30} etex);
        label.llft(btex S etex,S);
        label.lrt(btex R etex,R);
        label.top(btex T etex,T);
    \end{Geometrie}
    \correction
    Dans le triangle $RST$ rectangle en $R$, on connaît :
    \begin{itemize}        
        \item l'angle $\widehat{RST}=\ang{30}$,        
        \item le côté adjacent à l'angle $\widehat{RST}$ : $RS=\Lg[cm]{4}$,
    \end{itemize}
    et on cherche le côté opposé à l'angle $\widehat{RST}$.\\
    On utilise donc la tangente.
    \begin{align*}
        \tan\widehat{RST}&=\frac{RT}{RS}\\
        \tan30&=\frac{RT}{4}\\
        \intertext{\centering\bf Les produits en croix sont égaux}
        RT&=4\times\tan30\\
        RT&\simeq \Lg[cm]{2.3}
    \end{align*}
\end{methode*1}

\begin{methode*1}[Tangente - Calculer un angle]
    \exercice
    Dans le triangle $IJK$, déterminer $\widehat{IJK}$.
        
        \medskip        
        \begin{Geometrie}[CoinHD={(6.5u,5u)}]        
            pair I,J,K;
            I=u*(3,3);
            J-I=u*(-1,-2);
            K=1.5[I,rotation(J,I,90)];
            trace polygone(I,J,K);
            remplis codeperp(J,I,K,8)--I--cycle withcolor noir;
            trace codeperp(J,I,K,8);
            trace appelation(J,I,3mm,btex \Lg[cm]{2} etex);
            trace appelation(I,K,3mm,btex \Lg[cm]{3} etex);
            trace Codeangle(K,J,I,0,btex ? etex);
            label.llft(btex J etex,J);
            label.lrt(btex K etex, K);
            label.top(btex I etex, I);
        \end{Geometrie}
    \correction
     Dans le triangle $IJK$ rectangle en $I$, on a
        \begin{align*}
            \tan\widehat{IJK}&=\frac{IK}{IJ}\\
            \tan\widehat{IJK}&=\frac{3}{2}\\
            \widehat{IJK}&=tan^{-1}\left(\frac{3}{2}\right)\\
            \widehat{IJK}&\simeq\ang{56}
        \end{align*}
\end{methode*1}

\begin{methode}[Sinus - Calculer une longueur]
    \exercice
    Dans le triangle $EFG$, déterminer $EF$.\\
    \begin{Geometrie}[CoinHD={(4u,5u)}]        
        pair E,F,G;
        E=u*(1,1);
        G-E=u*(2,0);
        F=2[E,rotation(G,E,60)];
        trace polygone(E,F,G);
        remplis codeperp(E,G,F,8)--G--cycle withcolor noir;
        trace codeperp(E,G,F,8);
        trace appelation(E,G,-3mm,btex \Lg[cm]{4.1} etex);        
        trace cotation(E,F,2mm,3mm,btex ? etex);
        marque_a:=marque_a*1.5;
        trace Codeangle(E,F,G,0,btex \ang{63} etex);
        label.llft(btex E etex,E);
        label.lrt(btex G etex, G);
        label.top(btex F etex, F);
    \end{Geometrie}
    \correction
    Dans le triangle $EFG$ rectangle en $G$, on connaît :
    \begin{itemize}        
        \item l'angle $\widehat{EFG}=\ang{63}$,        
        \item le côté opposé à l'angle $\widehat{EFG}$ : $EG=\Lg[cm]{4.1}$,
    \end{itemize}
    et on cherche l'hypoténuse. On utilise donc le sinus.
    \begin{align*}
        \sin\widehat{EFG}&=\frac{EG}{EF}\\
        \sin63&=\frac{4,1}{EF}\\
        \intertext{\centering\bf Les produits en croix sont égaux}
        EF\times\sin63&=4,1\\
        EF&=\frac{4,1}{\sin63}\\
        EF&\simeq\Lg[cm]{4.6}
    \end{align*}
\end{methode}

\begin{methode}[Sinus - Calculer un angle]
    \exercice
    Déterminer $\widehat{LAO}$ dans le triangle $LAO$.
        
        \medskip
        \scalebox{1.2}{
        \begin{Geometrie}[CoinHD={(6u,4.5u)}]        
            pair L,O,A;
            O=u*(3,1);
            A-O=u*(2,1.5);
            L=0.65[O,rotation(A,O,90)];
            trace polygone(L,O,A);
            remplis codeperp(A,O,L,8)--O--cycle withcolor noir;
            trace codeperp(A,O,L,8);
            trace appelation(L,A,3mm,btex \Lg[cm]{2} etex);
            trace appelation(L,O,-3mm,btex \Lg[cm]{1.3} etex);            
            trace Codeangle(L,A,O,0,btex ? etex);
            label.bot(btex O etex,O);
            label.rt(btex A etex,A);
            label.top(btex L etex,L);
        \end{Geometrie}
        }
    \correction
    Dans le triangle $LOA$ rectangle en $O$, on a
    \begin{align*}
        \sin\widehat{LAO}&=\frac{OL}{AL}\\
        \sin\widehat{LAO}&=\frac{1,3}{2}\\
        \widehat{LAO}&=sin^{-1}\left(\frac{1,3}{2}\right)\\
        \widehat{LAO}&\simeq\ang{41}
    \end{align*}
\end{methode}
