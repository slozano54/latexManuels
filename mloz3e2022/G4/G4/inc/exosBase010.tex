\begin{exercice*}
    $ABC$ est un triangle rectangle en $A$ tel que\\ $AC=\Lg[cm]{5}$ et $\widehat{ABC}=\ang{35}$. On veut calculer $BC$.
    \begin{itemize}
        \item Faire un croquis avec les données.
        \item Utiliser la trame précédente pour calculer $BC$.
    \end{itemize}
\end{exercice*}
\begin{corrige}
    %\setcounter{partie}{0} % Pour s'assurer que le compteur de \partie est à zéro dans les corrigés
    \phantom{rrr}\\
    \begin{center}
        \begin{Geometrie}[CoinHD={(5u,4.5u)}]
            u:=0.75*u;            
            pair A,B,C;
            A=u*(1.75,1.75);
            B=u*(1.75,5.25);
            C=u*(6.25,1.75);
            trace polygone(A,B,C);            
            trace codeperp(B,A,C,8);
            trace appelation(A,C,-3mm,btex \Lg[cm]{5} etex);            
            trace appelation(B,C,3mm,btex ? etex);
            % marque_a:=marque_a*1.5;    
            trace Codeangle(A,B,C,0,btex \ang{35} etex);
            label.llft(btex A etex,A);
            label.lrt(btex C etex,C);
            label.top(btex B etex,B);
        \end{Geometrie}
    \end{center}
    {\red     
    Dans le triangle $ABC$ rectangle en $A$, on connaît :
    \begin{itemize}        
        \item l'angle $\widehat{ABC}=\ang{35}$,
        \item le côté opposé à l'angle $\widehat{ABC}$ : $AC=\Lg[cm]{5}$,
    \end{itemize}
    et on cherche l'hypoténuse. On utilise donc le sinus.
    \begin{align*}
        \sin(\widehat{ABC})&=\frac{AC}{BC}\\        
        \sin(\ang{35})&=\frac{\num{5}}{BC}\\        
        \intertext{\centering\bf Les produits en croix sont égaux}
        BC\times\sin(\ang{35})&=\num{5}\\
        BC&=\frac{\num{5}}{\sin(\ang{35})}\\
        BC&\simeq \Lg[cm]{8.7}
    \end{align*}
    }
\end{corrige}

