\begin{exercice*}
    $IJK$ est un triangle rectangle en $I$ tel que\\ $IJ=\Lg[cm]{3.2}$ et $JK=\Lg[cm]{5.3}$.\\
    On veut calculer $\widehat{IKJ}$ au degré près.
    \begin{itemize}
        \item Faire un croquis avec les données.
        \item Utiliser la trame précédente pour calculer $\widehat{IKJ}$.
    \end{itemize}
\end{exercice*}
\begin{corrige}
    %\setcounter{partie}{0} % Pour s'assurer que le compteur de \partie est à zéro dans les corrigés
    % \phantom{rrr}    
    $IJK$ est un triangle rectangle en $I$ tel que\\ $IJ=\Lg[cm]{3.2}$ et $JK=\Lg[cm]{5.3}$. On veut calculer $\widehat{IKJ}$ au degré près.
    \begin{itemize}
        \item Faire un croquis avec les données.
        \item Utiliser la trame précédente pour calculer $\widehat{IKJ}$.
    \end{itemize}
    \begin{center}
        \scalebox{0.9}{
        \begin{Geometrie}[CoinHD={(5.5u,4u)},TypeTrace="MainLevee"]
            pair I,J,K;
            I=u*(5,3);
            K-I=u*(-0.2,-1.5);
            J=1.7[I,rotation(K,I,-90)];
            drawoptions(withcolor red);
            trace polygone(I,J,K);
            trace codeperp(J,I,K,8);
            marque_a:=marque_a/1.5;
            trace Codeangle(I,K,J,0, btex ? etex);
            trace appelation(J,I,3mm,btex \Lg[cm]{3.2} etex);
            trace appelation(J,K,-3mm,btex \Lg[cm]{5.3} etex);
            label.rt(btex I etex,I);
            label.ulft(btex J etex,J);
            label.bot(btex  K etex,K);
        \end{Geometrie}
        }
    \end{center}
    {\red 
    Dans le triangle $IJK$ rectangle en $I$, on connaît :
    \begin{itemize}        
        \item l'hypoténuse,
        \item le côté opposé à l'angle $\widehat{\red IKJ}$,
    \end{itemize}
    et on cherche la mesure de l'angle $\widehat{IKJ}$.\\ On utilise donc le sinus.
    \begin{align*}
        \sin(\widehat{IKJ})&=\frac{IJ}{JK}\\        
        \sin(\widehat{IKJ})&=\frac{\num{3.2}}{\num{5.3}}\\                
        \widehat{IKJ}&=\sin^{-1}\left(\frac{\num{3.2}}{\num{5.3}}\right)\\
        \widehat{IKJ}&\simeq \ang{37}
    \end{align*}
    }
\end{corrige}

