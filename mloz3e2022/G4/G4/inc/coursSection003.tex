\section{Sinus d'un angle aigu}
% \definNum{
% Dans un triangle RECTANGLE, le \textbf{sinus} d'un angle aigu est égal au rapport du côté opposé à cet angle aigu par l'hypoténuse.}

% \CadreLampe{Sinus}{
% \compo{1}{courstrigo}{0.8}{
% Dans le triangle $ABC$ rectangle en $A$, on a :
% $$\sin\widehat{ABC}=\frac{\mbox{côté opposé à l'angle $\widehat{ABC}$}}{\mbox{hypoténuse}}=\frac{AC}{BC}$$
% $$\sin\widehat{ACB}=\frac{\mbox{côté opposé à l'angle $\widehat{ACB}$}}{\mbox{hypoténuse}}=\frac{AB}{BC}$$
% }
% }

% \proprNum{}{on calcule le sinus d'un angle aigu dans un triangle rectangle}{sa valeur est comprise entre 0 et 1}

% \Preuve{
% C'est un rapport de longueurs.\par
% Dans un triangle rectangle le plus grand côté est l'hypoténuse !\par
% Ici, c'est aussi le dénominateur du rapport qui est donc nécessairement plus petit que 1.$\square$
% }

% \proprNum{}{deux angles sont complémentaires ( $\sum$, leur somme, vaut 90\degre )}{le sinus de l'un est égal au cosinus de l'autre}

% \Preuve{
% \compo{1}{courstrigo}{0.8}{
% On peut considérer que ce sont les deux angles aigus dans un triangle rectangle.\\
% Dans ce cas, le côté opposé de l'un est aussi le côté adjacent de l'autre.$\square$
% }
% }

% \Remarques[Remarque]{
% $\widehat{ABC}$ et $\widehat{ACB}$ sont complémentaires \textbf{donc} :
% $$sin(\widehat{ABC})=cos(\widehat{ACB})$$
% $$cos(\widehat{ABC})=sin(\widehat{ACB})$$ 
% }