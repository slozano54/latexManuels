\section{Tangente d'un angle aigu}
\begin{definition}
    Dans un triangle {\bfseries rectangle}, la \textbf{tangente} d'un angle aigu est égal au rapport du côté opposé  par le côté adjacent à cet angle aigu.
\end{definition}

\begin{exemples*1}
    Dans le triangle $ABC$ rectangle en $A$, on a : 
    
    \medskip
    \begin{minipage}{0.4\linewidth}
        \begin{Geometrie}[CoinHD={(5u,4.5u)}]
            u:=0.75*u;            
            pair A,B,C;
            A=u*(1.75,1.75);
            B=u*(1.75,5.25);
            C=u*(6.25,1.75);
            trace polygone(A,B,C);
            remplis codeperp(B,A,C,8)--A--cycle withcolor noir;
            trace codeperp(B,A,C,8);
            trace appelation(A,B,8mm,btex Côté opposé etex);
            trace appelation(A,B,3mm,btex à l'angle $\widehat{ACB}$ etex);
            trace appelation(A,C,-3mm,btex Côté adjacent etex);
            trace appelation(A,C,-8mm,btex à l'angle $\widehat{ACB}$etex);
            trace appelation(B,C,3mm,btex Hypoténuse etex);
            marque_a:=marque_a*1.5;    
            trace marqueangle(B,C,A,0);
            label.llft(btex A etex,A);
            label.lrt(btex C etex,C);
            label.top(btex B etex,B);
        \end{Geometrie}
    \end{minipage}
    \begin{minipage}{0.55\linewidth}        
        $$\tan\widehat{ABC}=\frac{\mbox{côté opposé à l'angle $\widehat{ABC}$}}{\mbox{côté adjacent à l'angle $\widehat{ABC}$}}=\frac{AC}{AB}$$
        $$\tan\widehat{ACB}=\frac{\mbox{côté opposé à l'angle $\widehat{ACB}$}}{\mbox{côté adjacent à l'angle $\widehat{ACB}$}}=\frac{AB}{AC}$$
    \end{minipage}
\end{exemples*1}

