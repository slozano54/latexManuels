\begin{exercice*}
    $RST$ est un triangle rectangle en $S$ tel que :\\$RS=\Lg[cm]{4}$ et $ST=\Lg[cm]{7}$.\\
    Compléter la trame pour calculer l'angle $\widehat{SRT}$.
    \begin{center}
        \scalebox{0.9}{
            \begin{Geometrie}[CoinHD={(5.5u,3.5u)}]        
                pair R,S,T;
                S=u*(1,1);
                R-S=u*(0,2);
                T-S=u*(4,0);
                trace polygone(R,S,T);
                trace codeperp(T,S,R,8);
                trace marqueangle(S,R,T,0);
                trace appelation(S,T,-3mm,btex \Lg[cm]{7} etex);
                trace appelation(S,R,3mm,btex \Lg[cm]{4} etex);
                label.ulft(btex R etex,R);
                label.llft(btex S etex,S);
                label.lrt(btex T etex,T);
            \end{Geometrie}
        }
    \end{center}
    Dans le triangle $RST$ \makebox[0.4\linewidth]{\dotfill} en \makebox[0.1\linewidth]{\dotfill}, on connaît :
    \begin{itemize}        
        \item le côté \makebox[0.3\linewidth]{\dotfill} à l'angle $\widehat{SRT}$,
        \item le côté \makebox[0.3\linewidth]{\dotfill} à l'angle $\widehat{\phantom{M}\dots\phantom{N}}$,
    \end{itemize}
    et on cherche la mesure de l'angle $\widehat{\phantom{M}\dots\phantom{N}}$.\\ On utilise donc \makebox[0.2\linewidth]{\dotfill}.
    \begin{align*}
        \makebox[0.3\linewidth]{\dotfill}&=\frac{\makebox[0.2\linewidth]{\dotfill}}{\makebox[0.2\linewidth]{\dotfill}}\\
        &\\
        \makebox[0.3\linewidth]{\dotfill}&=\makebox[0.2\linewidth]{\dotfill}\\
        &\\        
        \makebox[0.3\linewidth]{\dotfill}&=\makebox[0.2\linewidth]{\dotfill}\\        
        \makebox[0.3\linewidth]{\dotfill}&\simeq\makebox[0.2\linewidth]{\dotfill}
    \end{align*}
\end{exercice*}
\begin{corrige}
    %\setcounter{partie}{0} % Pour s'assurer que le compteur de \partie est à zéro dans les corrigés
    % \phantom{rrr}    
    $RST$ est un triangle rectangle en $S$ tel que :\\$RS=\Lg[cm]{4}$ et $ST=\Lg[cm]{7}$.\\
    Compléter la trame pour calculer l'angle $\widehat{SRT}$.
    \begin{center}
        \scalebox{0.9}{
        \begin{Geometrie}[CoinHD={(5.5u,3.5u)}]        
            pair R,S,T;
            S=u*(1,1);
            R-S=u*(0,2);
            T-S=u*(4,0);
            trace polygone(R,S,T);
            trace codeperp(T,S,R,8);
            trace marqueangle(S,R,T,0);
            trace appelation(S,T,-3mm,btex \Lg[cm]{7} etex);
            trace appelation(S,R,3mm,btex \Lg[cm]{4} etex);
            label.ulft(btex R etex,R);
            label.llft(btex S etex,S);
            label.lrt(btex T etex,T);
        \end{Geometrie}
        }
    \end{center}
    Dans le triangle $RST$ {\red rectangle} en {\red $S$}, on connaît :
    \begin{itemize}        
        \item le côté {\red adjacent} à l'angle $\widehat{SRT}$,
        \item le côté {\red opposé} à l'angle $\widehat{\red SRT}$,
    \end{itemize}
    et on cherche la mesure de l'angle $\widehat{\red SRT}$.\\ On utilise donc {\red la tangente}.
    \begin{align*}
        {\red \tan(\widehat{SRT})}&=\frac{\red ST}{\red RS}\\        
        {\red \tan(\widehat{SRT})}&=\frac{\red 7}{\red 4}\\                
        {\red \widehat{SRT}}&={\red \tan^{-1}\left(\frac{7}{4}\right)}\\
        {\red \widehat{SRT}}&\simeq {\red\ang{60.3}}
    \end{align*}
\end{corrige}

