\begin{exercice*}[La mesure du rayon de la Terre vers 200 avant J.-C.]
  Vers 200 Avant J.C., le mathématicien astronome grec Ératosthène avait utilisé les remarques suivantes :
  \begin{description}
  \item[\ding{172}] En un point $A$ de la haute vallée du Nil (Syène, aujourd'hui Assouan), le soleil éclaire un certain jour à midi le fond des puits, il est dont à la verticale du point $A$.
  \item[\ding{173}] Au même instant, en un point $B$ du delta du Nil 
  (Alexandrie) (c'est-à-dire situé sur le même méridien que $A$); le soleil fait avec la verticale un angle $\alpha$ qu'un observateur peut relever.
  \item[\ding{174}] Comme le soleil est très loin, on peut considérer que les droites qui vont de $A$ au soleil et de $B$ au soleil sont pratiquement parallèles.
  \end{description}
  \begin{enumerate}
  \item Explique comment, connaissant la distance de $A$ à $B$ et l'angle $\alpha$, Ératosthène a pu calculer (approximativement) le rayon de la Terre.
  % \[\includegraphics{3trigoexo25.1}\]
  \item Évalue à ton tour le rayon de la Terre sachant que $AB=800$~km et $\alpha=7\degres$.
  \end{enumerate}
  \begin{minipage}{\linewidth}
    \begin{Geometrie}[CoinHD={(22u,10u)}]
      pair A,O,B,P,Q,R,S,T,U;
      O=u*(15,1);
      P-O=u*(-7,0);
      Q=rotation(P,O,-80);
      path cc;
      cc=arccercle(Q,P,O);
      trace cc;
      A=rotation(P,O,-23);
      B=rotation(P,O,-41);
      trace segment(O,A);
      R=6.5/5[O,B];
      trace segment(O,R);
      trace segment(B,R) withpen pencircle scaled1.5bp;
      T-R=u*(-3,2);
      U-T=0.6*(O-A);
      picture rayon;
      rayon=image(
        drawoptions(withcolor orange);
        trace segment(U,iso(T,U));
        drawarrow segment(T,iso(T,U));
        drawoptions();
        );
      trace rayon;
      trace rayon shifted(R-U);
      trace rayon shifted(A-U);
      trace rayon shifted(R-U+0.3*(A-R));
      trace rayon shifted(R-U+0.55*(A-R));
      trace rayon shifted(R-U+0.8*(A-R)+0.05*(T-A));
      S=cc intersectionpoint parallele(T,U,R);
      trace segment(S,R);
      marque_a:=30;
      trace Codeangle(B,R,S,0,btex $\alpha$ etex);
      marque_p:="plein";
      label.bot(btex A etex,A);
      label.lft(btex B etex,B);
      label.top(btex O etex,O);
      trace codeperp(R,B,S,5);
      trace codeperp(O,A,B,5);
      trace appelation(T,U,2mm,btex rayons du soleil etex);
      trace appelation(A,O,2mm,btex Verticale de A etex);
      trace appelation(B,O,2mm,btex Verticale de B etex);
      label.lft(btex Centre de la Terre etex,O);
    \end{Geometrie}
    \vspace*{10mm}
    \creditLibre{melusine.eu.org, d'après {\sl Histoire des Mathématiques pour le collège.} {\sc CEDIC}}
  \end{minipage}
\end{exercice*}
% \begin{corrige}
%   %\setcounter{partie}{0} % Pour s'assurer que le compteur de \partie est à zéro dans les corrigés
%   % \phantom{rrr}    
%   \dots
% \end{corrige}

