\begin{exercice*}
    \phantom{rrr}

    % \vspace*{-40mm}
    \begin{minipage}{0.7\linewidth}
    Indique pour chaque silhouette, si elle correspond à une réduction, à un agrandisssement ou à une déformation de la silhouette ci-contre.
    \end{minipage}
    \begin{minipage}{0.25\linewidth}
        \includegraphics[scale=0.4]{\currentpath/images/silhouetteChat.png}
    \end{minipage}

    \begin{minipage}{1\linewidth}
    \begin{center}
        \begin{multicols}2   
            \begin{enumerate}
                \item \phantom{rrr} 
                
                \includegraphics[scale=0.4]{\currentpath/images/silhouetteChatQ1.png}
                \item \phantom{rrr} 
                
                \includegraphics[scale=0.4]{\currentpath/images/silhouetteChatQ2.png}
                \item \phantom{rrr} 
                
                \includegraphics[scale=0.4]{\currentpath/images/silhouetteChatQ3.png}
                \item \phantom{rrr} 
                
                \includegraphics[scale=0.4]{\currentpath/images/silhouetteChatQ4.png}
            \end{enumerate}
        \end{multicols}
    \end{center}
    \creditLibre{Cahier sésamaths 2021 de $3^e$}
    \end{minipage}
    
\end{exercice*}
\begin{corrige}
    %\setcounter{partie}{0} % Pour s'assurer que le compteur de \partie est à zéro dans les corrigés
    % \phantom{rrr}

    \begin{enumerate}
        \item Déformation.                
        \item Réduction après symétrie axiale ou inversement.
        \item Agrandissement.
        \item Déformation                
    \end{enumerate}
\end{corrige}

