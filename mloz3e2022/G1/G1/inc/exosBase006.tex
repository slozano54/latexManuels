\begin{exercice*}
    \phantom{rrr}

    \vspace*{-10mm}
    \begin{minipage}{0.65\linewidth}
        Sur la figure ci-contre :
        \begin{itemize}
            \item Les points $M$, $A$ et $C$ sont alignés.
            \item Les points $N$, $A$ et $B$ sont alignés.
            \item $(MN)$ et $(BC)$ sont parallèles.
            \item $AM=\Lg{0.6}$.
            \item $BC=\Lg{2.1}$.
            \item $AC=\Lg{1.8}$.
        \end{itemize}
    \end{minipage}
    \begin{minipage}{0.35\linewidth}        
        \Thales[FigurecroiseeSeule,Angle=30]{ABCNM}{AN}{0.6}{MN}{AB}{1.8}{2.1}  
    \end{minipage}

    Calculer $MN$.

    \hrefMathalea{https://coopmaths.fr/mathalea.html?ex=3G20,s=3,n=1,video=j_zZOpLLl9k,cd=1,i=1&v=l}
\end{exercice*}
\begin{corrige}
    %\setcounter{partie}{0} % Pour s'assurer que le compteur de \partie est à zéro dans les corrigés
    \phantom{rrr}

    \Thales[Figurecroisee,Droites,Angle=30,ChoixCalcul=3]{ABCNM}{AN}{0.6}{MN}{AB}{1.8}{2.1}  

\end{corrige}

