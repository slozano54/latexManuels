\begin{exercice*}
    \phantom{rrr}

    \begin{minipage}{0.7\linewidth}        
    Sur la figure ci-contre, on a représenté un cône tel que $SO=\Lg{10}$.

    Un plan parallèle à la base coupe ce cône tel que $SO'=\Lg{7}$.
    \end{minipage}
    \begin{minipage}{0.3\linewidth}        
        \begin{center}
            \textit{La figure n'est pas à l'échelle}

            \begin{tikzpicture}[scale=0.7]
                % \draw[help lines, color=black!30, dashed] (0,0) grid (8,12);        
                \coordinate[label=below:$O$] (O) at (4.5,2);
                \coordinate[label=above:$S$] (S) at (4.47,7.05);
                \coordinate (O1) at (7.07,2);
                \coordinate (O2) at (1.93,2);
                \draw[thick, dashed] (O1) arc(0:180:25.7mm and 6.10mm);
                \draw[thick] (O2) arc(180:360:25.7mm and 6.10mm);
                \draw[thick,dashed] (S)--(H);
                \draw[thick] (S)--(O2);
                \draw[thick] (S)--(O1);
                \tkzDefPointBy[homothety=center S ratio 0.8](O)	\tkzGetPoint{O'}
                \tkzLabelPoints[left](O')
                \tkzDrawPoints[shape=cross out](O,O')
                \tkzDefPointBy[homothety=center S ratio 0.8](O1)	\tkzGetPoint{O1'}
                \tkzDefPointBy[homothety=center S ratio 0.8](O2)	\tkzGetPoint{O2'}
                \draw[thick, dashed] (O1') arc(0:180:20.56mm and 4.88mm);
                \draw[thick] (O2') arc(180:360:20.56mm and 4.88mm);
            \end{tikzpicture}
        \end{center}
    \end{minipage}

    \begin{enumerate}
        \item Sachant que le rayon du disque de base du grand cône vaut \Lg{3.2}.
        
        Calculer la valeur exacte du volume du grand cône.
        \item Déterminer le coefficient de réduction permettant de passer du grand cône au petit cône.
        \item Calculer la valeur exacte du volume du petit cône, puis arrondir au \Vol{}.
    \end{enumerate}

    \hrefMathalea{https://coopmaths.fr/mathalea.html?ex=3G22-1,n=3,i=1&v=l}    
\end{exercice*}
\begin{corrige}
    %\setcounter{partie}{0} % Pour s'assurer que le compteur de \partie est à zéro dans les corrigés
    \phantom{rrr}
    \begin{spacing}2
    \begin{enumerate}
        \item $\mathcal{V}_{\text{grand cône}} = \dfrac{1}{3}\times \mathcal{A}_{\text{base}}\times \text{hauteur} = \dfrac{1}{3}\times (\pi\times \num{3.2}^2) \times 10 = \dfrac{512}{15}\pi~\Vol{} \simeq \Vol{107.23}$.
        \item Si on note $k$ le coefficient de réduction, $k<1$, $k = \dfrac{SO'}{SO} = \dfrac{7}{10} = \num{0.7}$
        \item On sait maintenant que $k=\num{0.7}$, or si les longueurs sont multipliées par un rapport $k$, les volumes, qui sont homogènes à 
        des produits de trois longueurs sont multipliées par $k^3$.

        Donc $\mathcal{V}_{\text{petit cône}} = (\num{0.7})^3\times \mathcal{V}_{\text{grand cône}}$

        D'où $\mathcal{V}_{\text{petit cône}} = \dfrac{343}{\num{1000}}\times \dfrac{512}{15}\pi = \dfrac{\num{21952}}{\num{1875}}\pi~\Vol{} \simeq \Vol{37}$.


        \medskip
        On pourrait vérifier en faisant le calcul suivant :

        $\mathcal{V}_{\text{petit cône}} = \dfrac{1}{3}\times (\pi\times (\num{3.2}\times \num{0.7})^2) \times 7 \simeq \Vol{36.78}$.
    \end{enumerate}
    \end{spacing}

\end{corrige}

