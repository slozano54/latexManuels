\begin{exercice*}
    On considère que ces deux tables à repasser sont posées sur un sol horizontal.

    \begin{minipage}{1\linewidth}
    \begin{center}
        La figure n'est pas à l'échelle.

        \includegraphics[scale=0.4]{\currentpath/images/tablesARepasser.png}    
    \end{center}
    \creditLibre{Cahier iparcours 2022 de $3^e$}
    \end{minipage}    

    \begin{enumerate}
        \item Déterminer si le plateau rose est horizontal?
        \item Déterminer si le plateau vert est horizontal?
    \end{enumerate}
\end{exercice*}
\begin{corrige}
    %\setcounter{partie}{0} % Pour s'assurer que le compteur de \partie est à zéro dans les corrigés
    \phantom{rrr}

    \begin{center}
        La figure n'est pas à l'échelle.

        \includegraphics[scale=0.4]{\currentpath/images/tablesARepasserCorr.png}    
    \end{center}

    \begin{enumerate}
        \item \textbf{Plateau rose}
        
        On compare les rapports $\dfrac{\num{21.6}}{52}$ et $\dfrac{25}{60}$.

        \medskip
        $\num{21.6}\times 60 = \num{1296}$ et $52\times 25 = \num{1300}$

        Les produits en croix ne sont pas égaux donc les rapports non plus.

        Si le plateau était parallèle au sol, les rapports seraient égaux d'après le 
        théorème de Thalès or ce n'est pas le cas donc le plateau rose n'est pas horizontal.
        \item \textbf{Plateau vert}

        On compare les rapports $\dfrac{18}{48}$ et $\dfrac{24}{64}$.

        \medskip
        Ces rapports sont tous les deux égaux à $\dfrac{3}{8}$

        De plus les points $A$, $E$ et $D$ d'une part et $B$, $E$ et $C$ d'autre part sont alignés dans le même ordre.

        D'après la réciproque du théorème de Thalès le plateau vert $(AB)$ est parallèle au sol $(CD)$ qui est horizontal donc le plateau vert est horizontal.

    \end{enumerate}
\end{corrige}



