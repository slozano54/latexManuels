\begin{exercice*}
    La figure ci-dessous représente un schéma du fonctionnement d'un appareil photo argentique.

    Un objet $[AB]$ situé à une distance $d$ de l'objectif $O$ a une image $[A'B']$ située à une distance $d'$ de $O$.

    \medskip
    Démontrer l'égalité $\dfrac{d}{d'}=\dfrac{AB}{A'B'}$.

    \begin{minipage}{1\linewidth}
    \begin{center}
        La figure n'est pas à l'échelle.

        \includegraphics[scale=0.4]{\currentpath/images/schemaAppareilPhoto.png}    
    \end{center}
    \creditLibre{Cahier iparcours 2022 de $3^e$}
    \end{minipage}    
\end{exercice*}
\begin{corrige}
    %\setcounter{partie}{0} % Pour s'assurer que le compteur de \partie est à zéro dans les corrigés
    % \phantom{rrr}
    Les droites $(AB)$ et $(A'B')$ sont perpendiculaires à la même droite $(AA')$, elles sont donc parallèles entre elles.

    \begin{itemize}
        \item Les points $A$, $O$ et $A'$ d'une part et $B$, $O$ et $B'$ d'autre part sont alignés.
        \item $(AB)$ et $(A'B')$ sont parallèles.
    \end{itemize}

    Le théorème de Thalès permet donc d'écrire $\dfrac{OA}{OA'}=\dfrac{OB}{OB'}=\dfrac{AB}{A'B'}$ d'où $\dfrac{d}{d'}=\dfrac{AB}{A'B'}$.
\end{corrige}

