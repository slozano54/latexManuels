\begin{exercice*}
    L'aire de la base d'un cylindre est de \Aire{51}.
    On considère une réduction de ce cylindre de rapport \num{0.6}.
    \begin{enumerate}
        \item Déterminer l'aire de la base du cylindre réduit.
        \item Déterminer le rayon du cylindre réduit.
    \end{enumerate}
\end{exercice*}
\begin{corrige}
    %\setcounter{partie}{0} % Pour s'assurer que le compteur de \partie est à zéro dans les corrigés
    % \phantom{rrr}
    \begin{enumerate}
        \item Lorsque les longueurs sont multipliées par un rapport $k$, les aires, qui sont homogènes à 
    des produits de deux longueurs sont donc multipliées par $k^2$.

    Ici, $k=\num{0.6}$, donc $\mathcal{A}_{\text{base du cylindre réduit}} = k^2\times \mathcal{A}_{\text{base du cylindre}} = \num{0.6}^2\times 51 = \num{18.36}$

    Donc l'aire de la base du cylindre réduit vaut environ \Aire{18.36}.
        \item L'aire d'un disque valant $\pi \times \text{Rayon}^2$, $R_\text{base du cylindre réduit}=\sqrt{\num{18.36}\div\pi} \simeq \num{2.4}$

    Donc le rayon de la base du cylindre réduit vaut environ \Lg{2.4}.
    \end{enumerate}
\end{corrige}

