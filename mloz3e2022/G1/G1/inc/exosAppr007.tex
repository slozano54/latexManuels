\begin{exercice*}
    \phantom{rrr}
    
    % \begin{minipage}{0.65\linewidth}
        Sur la figure ci-dessous :
        \begin{itemize}
            \item Les points $O$, $U$ et $Z$ sont alignés.
            \item Les points $O$, $X$ et $T$ sont alignés.
            \item Les graduations sont régulières.
        \end{itemize}

        Montrer que les droites $(XU)$ et $(ZT)$ sont parallèles.
    % \end{minipage}

    % \begin{minipage}{0.35\linewidth}        
        \begin{tikzpicture}[scale=0.8]        
            % \draw[help lines, color=black!30, dashed] (0,0) grid (8,12);        
            \coordinate[label=below:$O$] (O) at (1,1);
            \coordinate[label=below:$U$] (U) at (3,1);
            \coordinate[label=below:$Z$] (Z) at (7,1);
            \coordinate[label=above:$X$] (X) at (3.5,2);                        
            \tkzDefPointBy[homothety=center O ratio 2](X)	\tkzGetPoint{X1};
            \tkzDefPointBy[homothety=center O ratio 3](X)	\tkzGetPoint{T};
            \tkzDrawPoints[shape= cross out](X,X1,T);
            \tkzDrawLine(O,T);
            \tkzDrawLine(O,Z);
            \foreach \x in {1,1.5,...,7} \draw (\x,0.95) -- (\x,1.05);
        \end{tikzpicture}
    % \end{minipage}
\end{exercice*}
\begin{corrige}
    %\setcounter{partie}{0} % Pour s'assurer que le compteur de \partie est à zéro dans les corrigés
    \phantom{rrr}

    \begin{tikzpicture}[scale=0.8]        
        % \draw[help lines, color=black!30, dashed] (0,0) grid (8,12);        
        \coordinate[label=below:$O$] (O) at (1,1);
        \coordinate[label=below:$U$] (U) at (3,1);
        \coordinate[label=below:$Z$] (Z) at (7,1);
        \coordinate[label=above:$X$] (X) at (3.5,2);                        
        \tkzDefPointBy[homothety=center O ratio 2](X)	\tkzGetPoint{X1};
        \tkzDefPointBy[homothety=center O ratio 3](X)	\tkzGetPoint{T};
        \tkzLabelPoints[above left](T);
        \tkzDrawPoints[shape= cross out](X,X1,T);
        \tkzDrawLine(O,T);
        \tkzDrawLine(O,Z);
        \tkzDrawLines[red, add = 1 and 1](U,X);
        \tkzDrawLines[red, add = 0.3 and 0.3](Z,T);
        \foreach \x in {1,1.5,...,7} \draw (\x,0.95) -- (\x,1.05);
    \end{tikzpicture}
    \begin{itemize}
        \item Les points $O$, $U$ et $Z$ sont alignés dans le même ordre que
        les points $O$, $X$ et $T$.

        \medskip
        \item Les graduations sont régulières donc $\dfrac{OU}{OZ}=\dfrac{4}{12}=\dfrac{1}{3}$ et $\dfrac{OX}{OT}=\dfrac{1}{3}$
        
        donc $\dfrac{OU}{OZ}=\dfrac{OX}{OT}$
    \end{itemize}

    D'après la réciproque du théorème de Thalès, les droites $(XU)$ et $(ZT)$ sont parallèles.
\end{corrige}



