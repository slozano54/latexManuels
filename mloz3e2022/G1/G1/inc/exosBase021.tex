\begin{exercice*}
    On réduit une figure dont l'aire vaut \Aire{124}. L'aire le la figure obtenue vaut \Aire{89.59}.

    \medskip
    Déterminer le rapport de réduction.
\end{exercice*}
\begin{corrige}
    %\setcounter{partie}{0} % Pour s'assurer que le compteur de \partie est à zéro dans les corrigés
    % \phantom{rrr}
    Lorsque les longueurs sont multipliées par un rapport $k$, les aires, qui sont homogènes à 
    des produits de deux longueurs sont donc multipliées par $k^2$.

    En notant le rapport $k$, on peut écrire : $124\times k^2 = \num{89.59}$.

    Donc $k^2 = \num{89.59}\div 124 = \num{0.7225}$, $k$ étant un nombre positif, $k=\sqrt{\num{0.7225}} = \num{0.85}$

    \medskip
    Le rapport de réduction vaut $\num{0.85}$.
\end{corrige}

