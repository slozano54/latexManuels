\begin{exercice*}
    Dans tout l'exercice, les points $A$, $P$ et $B$ sont alignés ainsi que les points $A$, $R$ et $C$.

    Pour chaque figure :
    \begin{itemize}
        \item Expliquer pourquoi on peut appliquer le théorème de Thalès.
        \item Écrire alors les rapports égaux.
    \end{itemize}

    \begin{enumerate}
        \item \phantom{rrr}
        
        \begin{tikzpicture}[scale=0.6]
            % \draw[help lines, color=black!30, dashed] (0,0) grid (8,12);        
            \coordinate[label=above left:$C$] (C) at (1,4);
            \coordinate[label=above left:$B$] (B) at (1,1);            
            \coordinate[label=below:$A$] (A) at (5,4);
            \tkzDefPointBy[homothety=center A ratio 0.4](C)	\tkzGetPoint{R};
            \tkzDefPointBy[homothety=center A ratio 0.4](B)	\tkzGetPoint{P};
            \tkzDefPointBy[homothety=center B ratio 1.2](C)	\tkzGetPoint{C1};
            \tkzDefPointBy[homothety=center P ratio 1.5](R)	\tkzGetPoint{R1};
            \tkzLabelPoints[above left](R);
            \tkzLabelPoints[below right](P);
            \tkzDrawLine(C,A);
            \tkzDrawLine(B,A);
            \tkzDrawLine(C,B);
            \tkzDrawLine[add=0.5 and 0.5](R,P);
            \tkzMarkRightAngle[fill=gray!20](A,R,R1);
            \tkzMarkRightAngle[fill=gray!20](A,C,C1);
        \end{tikzpicture}
        \item \phantom{rrr}
        
        \begin{tikzpicture}[scale=0.6]
            % \draw[help lines, color=black!30, dashed] (0,0) grid (8,12);        
            \coordinate[label=below left:$B$] (B) at (1,4.3);
            \coordinate[label=above:$A$] (A) at (3,4);            
            \coordinate[label=right:$C$] (C) at (1.5,3);
            \tkzDefPointBy[homothety=center A ratio -1.5](B) \tkzGetPoint{P};
            \tkzDefPointBy[homothety=center A ratio -1.5](C)	\tkzGetPoint{R};
            \tkzLabelPoints[right](R);
            \tkzLabelPoints[below left](P);
            \tkzDrawLine[add=0.5 and 0.5](B,P);
            \tkzDrawLine[add=0.5 and 0.5](C,R);
            \tkzDrawLine[add=1 and 1](B,C);
            \tkzDrawLine[add=1 and 1](R,P);
            \tkzDefPointBy[homothety=center C ratio 2](B)	\tkzGetPoint{s};
            \tkzLabelPoints[right](s);
            \tkzDefPointBy[homothety=center B ratio 1.8](C)	\tkzGetPoint{z};
            \tkzLabelPoints[right](z);
            \tkzMarkAngle[size=0.35,mark=||](P,B,s);
            \tkzDefPointBy[homothety=center P ratio 2](R)	\tkzGetPoint{t};
            \tkzLabelPoints[right](t);
            \tkzDefPointBy[homothety=center R ratio 2](P)	\tkzGetPoint{u};
            \tkzLabelPoints[left](u);
            \tkzDefPointBy[homothety=center A ratio 2](P)	\tkzGetPoint{y};
            \tkzLabelPoints[below](y);
            \tkzMarkAngle[size=0.35,mark=||](y,P,t);
            \tkzDefPointBy[homothety=center A ratio 2](B)	\tkzGetPoint{v};
            \tkzLabelPoints[above](v);
            \tkzDefPointBy[homothety=center A ratio 2](C)	\tkzGetPoint{w};
            \tkzLabelPoints[above](w);
            \tkzDefPointBy[homothety=center A ratio 1.8](R)	\tkzGetPoint{x};
            \tkzLabelPoints[below](x);
        \end{tikzpicture}

    \end{enumerate}
\end{exercice*}
\begin{corrige}
    %\setcounter{partie}{0} % Pour s'assurer que le compteur de \partie est à zéro dans les corrigés
    % \phantom{rrr}
    \begin{enumerate}
        \item Les droites $(BC)$ et $(PR)$ sont toutes les deux perpendiculaires à la droite $(AC)$.
        
        De plus les droites $(CR)$ et $(BP)$ sont sécantes en $A$.

        \medskip
        Le théorème de Thalès permet donc d'écrire $\dfrac{AR}{AC}=\dfrac{AP}{AB}=\dfrac{RP}{BC}$.

        \medskip
        \item Les droites $(BC)$ et $(PR)$ forment avec la sécante $(BP)$
        les angles  $\widehat{sBA}$ et $\widehat{RPy}$ correspondants égaux, elles sont donc parallèles.
        
        De plus les droites $(CR)$ et $(BP)$ sont sécantes en $A$.

        \medskip
        Le théorème de Thalès permet donc d'écrire $\dfrac{AB}{AP}=\dfrac{AC}{AR}=\dfrac{BC}{RP}$.

    \end{enumerate}

\end{corrige}

