\begin{exercice*}
    \phantom{rrr}

    \vspace*{-10mm}
    \begin{minipage}{0.65\linewidth}
        Sur la figure ci-contre :
        \begin{itemize}
            \item Les points $C$, $S$ et $V$ sont alignés.
            \item Les points $A$, $S$ et $G$ sont alignés.            
            \item $SV=\Lg{0.6}$.
            \item $SG=\Lg{0.9}$.
            \item $SA=\Lg{2.1}$.
            \item $SC=\Lg{1}$.
        \end{itemize}
    \end{minipage}
    \begin{minipage}{0.35\linewidth}
        \hspace*{-15mm}        
        \Thales[Reciproque,FigurecroiseeSeule,Angle=30]{SACGV}{}{}{}{}{}{}  
    \end{minipage}

    Montrer que $(GV)$ et $(CA)$ ne sont pas parallèles.

    \hrefMathalea[\emoji{star-struck} \emoji{link} Sur \mathaleaLogo - Cas simples]{https://coopmaths.fr/mathalea.html?ex=3G21,s=1,s2=2,s3=3,i=1&v=l}
    
    \hrefMathalea[\emoji{star-struck} \emoji{link} Sur \mathaleaLogo - Complications]{https://coopmaths.fr/mathalea.html?ex=3G21,s=2,s2=2,s3=3,i=1&v=l}

\end{exercice*}
\begin{corrige}
    %\setcounter{partie}{0} % Pour s'assurer que le compteur de \partie est à zéro dans les corrigés
    \phantom{rrr}

    \Thales[Reciproque,Figurecroisee,Droites,Produit]{SACGV}{0.9}{2.1}{0.6}{1}{}{}  
    
\end{corrige}

