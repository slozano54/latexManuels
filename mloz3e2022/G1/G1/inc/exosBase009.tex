\begin{exercice*}
    La longueur de la ligne d'un téléphérique est de \Lg[m]{1437}. Après avoir parcouru \Lg[m]{450} en montant,
    il marque un temps d'arrêt.

    On suppose le câble du téléphérique rectiligne.

    \vspace*{10mm}
    Déterminer l'altitude à laquelle se trouve le téléphérique. Arrondir à l'unité.

    \begin{minipage}{1\linewidth}
    \begin{center}
        La figure n'est pas à l'échelle.

        \includegraphics[scale=0.3]{\currentpath/images/telepherique.png}    
    \end{center}
    \creditLibre{Cahier sesamath 2021 de $3^e$}
    \end{minipage}
    

\end{exercice*}
\begin{corrige}
    %\setcounter{partie}{0} % Pour s'assurer que le compteur de \partie est à zéro dans les corrigés
    % \phantom{rrr}
    \begin{itemize}
        \item Les droites $(AG)$ et $(DH)$ sont perpendiculaires à $(AB)$, donc elles sont parallèles entre elles.
        \item Les points $A$, $H$ et $B$ sont alignés car chacun sur le niveau de la mer.
        \item Les points $G$, $D$ et $B$ sont alignés car le câble du téléphérique est supposé rectiligne.
        \item \Thales[Droites]{BGADH}{450}{BH}{DH}{1437}{BA}{584}    
    \end{itemize}

    Donc à \Lg[m]{1} près, le téléphérique se trouve à une altitude d'environ \Lg[m]{183}.
\end{corrige}

