\begin{exercice*}
    Pour chaque figure ci-dessous, qui sont données à titre indicatif, dire s’il est possible d’appliquer la réciproque du théorème de Thalès.

    Justifier.
    \begin{enumerate}
        \item \phantom{rrr}
        
        \begin{Geometrie}[CoinHD={(6u,4u)},CoinBG={(-2u,0)}]
            % trace feuillet;
            pair A,B,C,D,E;
            A=u*(1,2);
            D-A=u*(3,0);
            E-A=u*(4,0);
            C-A=u*(4,1.5);
            B-A=u*(-2,-0.75);
            trace droite(B,C);
            trace droite(A,E);
            marque_p:="croix";
            u:=u/2;
            pointe(B,C,D,E);
            u:=u*2;
            labeloffset:=1.2*labeloffset;
            label.top(btex $A$ etex,A);
            label.bot(btex $B$ etex,B);
            label.top(btex $C$ etex,C);
            label.top(btex $D$ etex,D);
            label.top(btex $E$ etex,E);
            % trace appelation(A,D,-3mm,btex 3 etex);
            trace cotationmil(A,D,-3mm,10,btex 3 etex);
            trace appelation(A,C,3mm,btex 4 etex);
            trace appelation(B,A,-3mm,btex 3 etex);
            trace cotationmil(A,E,-9mm,10,btex 4 etex);
        \end{Geometrie}
        \item \phantom{rrr}
        
        \begin{Geometrie}[CoinHD={(6u,5u)},CoinBG={(0,0.5u)}]
            % trace feuillet;
            pair A,B,C,D,E;
            A=u*(1,2);
            D-A=u*(3,0);
            E-A=u*(4,0);
            path cAE,cAD;
            cAE=cercles(A,E);
            cAD=cercles(A,D);
            B=pointarc(cAD,20);
            C=pointarc(cAE,20);
            trace droite(B,C);
            trace droite(A,E);
            marque_p:="croix";
            u:=u/2;
            pointe(D,E,B,C);
            u:=u*2;
            labeloffset:=1.2*labeloffset;
            label.llft(btex $A$ etex,A);
            label.bot(btex $B$ etex,B);
            label.bot(btex $C$ etex,C);
            label.top(btex $D$ etex,D);
            label.top(btex $E$ etex,E);
            trace cotationmil(A,D,-3mm,10,btex 3 etex);
            trace cotationmil(A,E,-9mm,10,btex 4 etex);
            trace cotationmil(A,B,3mm,10,btex 3 etex);
            trace cotationmil(A,C,9mm,10,btex 4 etex);   
        \end{Geometrie}
        \pagebreak
        \item \phantom{rrr}
        
        \begin{Geometrie}[CoinHD={(6u,5u)},CoinBG={(0,0.5u)}]
            % trace feuillet;
            pair A,B,C,D,E;
            A=u*(1,2);
            D-A=u*(3,0);
            E-A=u*(4,0);
            path cAE,cAD;
            cAE=cercles(A,E);
            cAD=cercles(A,D);
            B=pointarc(cAD,20);
            C=pointarc(cAE,20);
            trace droite(B,C);
            trace droite(A,E);
            marque_p:="croix";
            u:=u/2;
            pointe(D,E,B,C);
            u:=u*2;
            labeloffset:=1.2*labeloffset;
            label.llft(btex $A$ etex,A);
            label.bot(btex $B$ etex,B);
            label.bot(btex $C$ etex,C);
            label.top(btex $D$ etex,D);
            label.top(btex $E$ etex,E);
            trace cotationmil(A,D,-3mm,10,btex \num{1.2} etex);
            trace cotationmil(A,E,-9mm,10,btex \num{3.6} etex);
            trace cotationmil(A,B,3mm,10,btex \num{1.3} etex);
            trace cotationmil(A,C,9mm,10,btex \num{5.2} etex);            
        \end{Geometrie}
        \item \phantom{rrr}
        
        \begin{Geometrie}[CoinHD={(6u,4u)},CoinBG={(-2u,0)}]
            % trace feuillet;
            pair A,B,C,D,E;
            A=u*(1,2);
            D-A=u*(3,0);
            E-A=u*(4,0);
            C-A=u*(4,1.5);
            B-A=u*(-2,-0.75);
            trace droite(B,C);
            trace droite(A,E);
            marque_p:="croix";
            u:=u/2;
            pointe(B,C,D,E);
            u:=u*2;
            labeloffset:=1.2*labeloffset;
            label.top(btex $A$ etex,A);
            label.bot(btex $B$ etex,B);
            label.top(btex $C$ etex,C);
            label.top(btex $D$ etex,D);
            label.top(btex $E$ etex,E);
            % trace appelation(A,D,-3mm,btex 3 etex);
            trace cotationmil(A,D,-3mm,10,btex \num{1.2} etex);
            trace appelation(A,C,3mm,btex \num{5.2} etex);
            trace appelation(B,A,-3mm,btex \num{1.3} etex);
            trace cotationmil(A,E,-9mm,10,btex \num{3.6} etex);
        \end{Geometrie}
    \end{enumerate}
\end{exercice*}
\begin{corrige}
    %\setcounter{partie}{0} % Pour s'assurer que le compteur de \partie est à zéro dans les corrigés
    % \phantom{rrr}
    \dots
\end{corrige}

