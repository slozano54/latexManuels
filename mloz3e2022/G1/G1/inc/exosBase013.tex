\begin{exercice*}
    La figure ci-dessous représente les plans pour la réalisation d'une desserte en bois.

    Déterminer si les plateaux $(AB)$ et $(CD)$ sont parallèles.

    \begin{minipage}{1\linewidth}
    \begin{center}
        La figure n'est pas à l'échelle.

        \includegraphics[scale=0.4]{\currentpath/images/desserteBois.png}    
    \end{center}
    \creditLibre{Cahier iparcours 2022 de $3^e$}
    \end{minipage}
    
    \hrefMathalea[\emoji{star-struck} \emoji{link} S'entraîner sur \mathaleaLogo - Cas simples]{https://coopmaths.fr/mathalea.html?ex=3G21,s=1,s2=2,s3=3,i=1&v=l}
    
    \hrefMathalea[\emoji{star-struck} \emoji{link} S'entraîner sur \mathaleaLogo - Complications]{https://coopmaths.fr/mathalea.html?ex=3G21,s=2,s2=2,s3=3,i=1&v=l}

\end{exercice*}
\begin{corrige}
    %\setcounter{partie}{0} % Pour s'assurer que le compteur de \partie est à zéro dans les corrigés
    % \phantom{rrr}

    D'une part, $\dfrac{OB}{OC}=\dfrac{45}{60}=0,75$. \hspace*{10mm} D'autre part, $\dfrac{BA}{DC}=\dfrac{76}{100}=0,76$.

    \begin{spacing}2
    On constate que $\dfrac{OB}{OC}\neq\dfrac{BA}{DC}$, la contraposée du théorème de Thalès permet donc d'affirmer que les droites $(AB)$ et $(CD)$ ne sont pas parallèles.
    \end{spacing}
    
    \psshadowbox{Les plateaux de la desserte ne sont donc pas parallèles.}
\end{corrige}

