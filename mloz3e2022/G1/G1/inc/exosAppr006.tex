\begin{exercice*}
    Jean Racine veut mesurer un jeune chêne avec une croix de bucheron comme le montre le schéma.
    Il place la croix de sorte que $O$, $D$ et $A$ d'une part et $O$, $E$ et $B$ d'autre part, soient alignés.

    \begin{itemize}
        \item $DE = \Lg{20}$.
        \item $OF = \Lg{35}$.        
        \item Jean place $[DE]$ verticalement et $[OF]$ horizontalement.
        \item Il mesure $BC = \Lg[m]{7.7}$.
    \end{itemize}

    \begin{minipage}{1\linewidth}
    \begin{center}
        La figure n'est pas à l'échelle.

        \includegraphics[scale=0.3]{\currentpath/images/croixDuBucheron.png}    
    \end{center}
    \creditLibre{Cahier iparcours 2022 de $3^e$}
    \end{minipage}    

    \begin{enumerate}
        \item $ABO$ est un agrandissement de $ODE$.
        
        Justifier que le coefficient d'agrandissement vaut $22$.
        \item Calculer la hauteur du chêne en mètres.
        \item Certaines croix du bucheron sont telles que $DE=OF$.
        
        Expliquer en quoi c'est un avantage.
    \end{enumerate}
\end{exercice*}
\begin{corrige}
    %\setcounter{partie}{0} % Pour s'assurer que le compteur de \partie est à zéro dans les corrigés
    % \phantom{rrr}
    \begin{enumerate}
        \item Coefficient d'agrandissement : $\dfrac{OH}{OF}=\dfrac{\Lg[m]{7.7}}{\Lg[m]{0.35}}=22$
        \item $AB$ est $22$ fois plus grand que $DE$ donc $AB=22\times \Lg{20}=\Lg{440}=\Lg[m]{4.4}$
        \item La hauteur de l'arbre est égale à : $\dfrac{OH\times DE}{OF}$ si $DE=OF$ alors la hauteur de l'arbre
        est égale à $OH$, c'est à dire à la distance de l'oeil à l'arbre, il suffit donc de mesurer la distance jusqu'à l'arbre.
    \end{enumerate}
\end{corrige}



