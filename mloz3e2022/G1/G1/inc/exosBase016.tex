\begin{exercice*}
    Sur la figure ci-contre :

    \begin{minipage}{0.65\linewidth}
        \begin{itemize}
            \item $BR=\Lg{2.5}$.
            \item $BL=\Lg{15}$.
            \item $BE=\Lg{1.5}$.
            \item $BI=\Lg{9}$.
            \item Les points $I$, $B$ et $E$ sont alignés.
            \item Les points $L$, $B$ et $R$ sont alignés.            
        \end{itemize}
    \end{minipage}
    \begin{minipage}{0.35\linewidth}        
        \Thales[Reciproque,FigurecroiseeSeule,Angle=40]{BRELI}{}{}{}{}{}{}  
    \end{minipage}

    Démontrer que les droites $(IL)$ et $(RE)$ sont parallèles.

    \hrefAleaTeX{https://urls.mathslozano.fr/3g12025ex16}
\end{exercice*}
\begin{corrige}
    %\setcounter{partie}{0} % Pour s'assurer que le compteur de \partie est à zéro dans les corrigés
    \phantom{rrr}

    \Thales[Reciproque,Figurecroisee,Droites,Produit]{BRELI}{15}{2.5}{9}{1.5}{}{}  
\end{corrige}

