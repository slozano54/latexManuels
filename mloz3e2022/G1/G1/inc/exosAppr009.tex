\begin{exercice*}
    \begin{itemize}
        \item Les points $D$, $F$, $A$ et $B$ sont alignés.
        \item Les points $E$, $G$, $A$ et $C$ sont alignés.
        \item Les droites $(DE)$ et $(FG)$ sont parallèles.
    \end{itemize}

    \begin{minipage}{1\linewidth}
    \begin{center}
        \includegraphics[scale=0.4]{\currentpath/images/pythagorePuisThales.png}    
    \end{center}
    \creditLibre{Cahier iparcours 2022 de $3^e$}
    \end{minipage}    

    \begin{enumerate}
        \item Montrer que le triangle $AFG$ est rectangle.
        \item Calculer la longueur du segment $[AD]$.
        \item En déduire la longueur du segment $[FD]$.
        \item Déterminer si les droites $(FG)$ et $(BC)$ sont parallèles.        
    \end{enumerate}
\end{exercice*}
\begin{corrige}
    %\setcounter{partie}{0} % Pour s'assurer que le compteur de \partie est à zéro dans les corrigés
    % \phantom{rrr}

    \begin{enumerate}
        \item \Pythagore[Reciproque]{FGA}{5}{3}{4}
        \item \Thales[Droites]{ADEFG}{5}{AG}{3}{AD}{AE}{8.1}
        
        $FD=AD-AF=\num{13.5}-5=\Lg{8.5}$.
        \item \Thales[Reciproque,Droites,Produit]{ABCFG}{5}{6.25}{4}{5}{}{}
    \end{enumerate}
\end{corrige}



