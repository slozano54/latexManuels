\begin{exercice*}
    Pour trouver la hauteur d'une éolienne, on a les renseignements suivants :

    \begin{itemize}
        \item Les points $O$, $A$ et $C$ sont alignés.
        \item Les points $O$, $B$ et $D$ sont alignés.
        \item Les angles $\widehat{OAB}$ et $\widehat{ACD}$ sont droits.
        \item $OA = \Lg[m]{11}$.
        \item $AC = \Lg[m]{594}$.
        \item $AB = \Lg[m]{1.5}$.
    \end{itemize}

    \begin{minipage}{1\linewidth}
    \begin{center}
        La figure n'est pas à l'échelle.

        \includegraphics[scale=0.4]{\currentpath/images/eolienne.png}    
    \end{center}
    \creditLibre{Cahier iparcours 2022 de $3^e$}
    \end{minipage}    

    \begin{enumerate}
        \item Expliquer pourquoi les droites $(AB)$ et $(CD)$ sont parallèles.
        \item Calculer la hauteur $CD$ de l'éolienne. Justifier.
    \end{enumerate}
\end{exercice*}
\begin{corrige}
    %\setcounter{partie}{0} % Pour s'assurer que le compteur de \partie est à zéro dans les corrigés
    % \phantom{rrr}
    \begin{enumerate}
        \item Les droites $(AB)$ et $(CD)$ sont perpendiculaires à la même droite $(AC)$, elles sont donc parallèles entre elles.
        \item $OC = OA+AC = \Lg[m]{11}+\Lg[m]{594} = \Lg[m]{605}$
        \begin{itemize}
            \item Les droites $(BD)$ et $(AC)$ sont sécantes en $O$.
            \item $(AB)$ et $(CD)$ sont parallèles.
        \end{itemize}

        Le théorème de Thalès permet d'écrire $\dfrac{OA}{OC}=\dfrac{OB}{OD}=\dfrac{BA}{DC}$ d'où $\dfrac{11}{605}=\dfrac{OB}{OD}=\dfrac{\num{1.5}}{DC}$.

        \medskip
        D'où $CD = \dfrac{605\times \num{1.5}}{11}=\num{82.5}$.

        \medskip
        L'éolienne mesure donc \Lg{82.5}.
    \end{enumerate}
\end{corrige}



