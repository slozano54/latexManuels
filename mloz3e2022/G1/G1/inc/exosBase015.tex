\begin{exercice*}
    Sur cette figure :
    
    \begin{minipage}{0.35\linewidth}        
        \hspace*{-15mm}\Thales[Reciproque,FigureSeule,Angle=40]{RPSTM}{8}{4}{8}{4}{}{}  
    \end{minipage}
    \begin{minipage}{0.65\linewidth}        
        \begin{itemize}
            \item $RM=\Lg{4.5}$.
            \item $RS=\Lg{6}$.
            \item $RT=\Lg{6}$.
            \item $RP=\Lg{8}$.
            \item Les points $R$, $T$ et $P$ sont alignés.
            \item Les points $R$, $M$ et $S$ sont alignés.            
        \end{itemize}
    \end{minipage}
    \begin{enumerate}
        \item Calculer et comparer les proportions $\dfrac{RT}{RP}$ et $\dfrac{RM}{RS}$.
        
        \medskip
        \item Justifier la position relative des droites $(MT)$ et $(SP)$.
    \end{enumerate}
    \hrefAleaTeX{https://urls.mathslozano.fr/3g12025ex15}
\end{exercice*}
\begin{corrige}
    %\setcounter{partie}{0} % Pour s'assurer que le compteur de \partie est à zéro dans les corrigés
    \phantom{rrr}
    \begin{spacing}2
    \begin{enumerate}
        \item $\dfrac{RT}{RP}=\dfrac{6}{8}=\dfrac{3}{4}$ et $\dfrac{RM}{RS}=\dfrac{4,5}{6}=\dfrac{3}{4}$. Donc $\dfrac{RT}{RP}=\dfrac{RM}{RS}$.
        \item On constate que $\dfrac{RT}{RP}\neq\dfrac{RM}{RS}$.
        
        De plus, les points les points $R$, $T$ et $P$ et les points $R$, $M$ et $S$ sont alignés dans cet ordre.            

        D'après la réciproque du théorème de Thalès, les droites $(TM)$ et $(FS)$ sont parallèles.
    \end{enumerate}  
    \end{spacing}
\end{corrige}

