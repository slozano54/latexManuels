\begin{exercice*}
    Soit un triangle $A'B'C'$ rectangle en $A'$ et d'aire \Aire{27}.
    Ce triangle est un agrandissement d'un triangle $ABC$, rectangle en $A$ tel que $AB=\Lg{3}$ et $AC=\Lg{2}$.

    \medskip
    Calculer les longueur $A'B'$ et $A'C'$.

    \hrefMathalea{https://coopmaths.fr/mathalea.html?ex=3G22-1,n=3,i=1&v=l}
\end{exercice*}
\begin{corrige}
    %\setcounter{partie}{0} % Pour s'assurer que le compteur de \partie est à zéro dans les corrigés
    % \phantom{rrr}
    Lorsque les longueurs sont multipliées par un rapport $k$, les aires, qui sont homogènes à 
    des produits de deux longueurs sont donc multipliées par $k^2$.

    $\mathcal{A}_{ABC} = \dfrac{3\times 2}{3} = \Aire{3}$.

    En notant $k$, le rapport d'agrandissement, $\mathcal{A}_{ABC} \times k^2 = \mathcal{A}_{A'B'C'}$.

    Donc $3\times k^2 = 27$, soit $k^2 = 9$, $k$ étant un nombre positif, $k = \sqrt{9} = 3$.

    On en déduit donc que :
    \begin{itemize}
        \item $A'B' = 3\times AB = 3 \times \Lg{3} = \Lg{9}$.
        \item $A'C' = 3\times AC = 3 \times \Lg{2} = \Lg{6}$.
    \end{itemize}
\end{corrige}

