\begin{exercice*}[Quotients égaux]
    Justifier que les couples de quotients suivants sont égaux en variant la méthode.
    \begin{multicols}3
        \begin{enumerate}
            \item $\dfrac{18}{5}$ et $\dfrac{72}{20}$.
            \item $\dfrac{2}{3}$ et $\dfrac{7}{10,5}$.
            \item $\dfrac{80}{29}$ et $\dfrac{480}{174}$.
        \end{enumerate}
    \end{multicols}

    \hrefMathalea{https://coopmaths.fr/mathalea.html?ex=4C20-2,s=2,n=4,i=0&v=l} % On peut personnaliser le texte entre crochets si on veut sinon supprimer les crochets
\end{exercice*}
\begin{corrige}
    %\setcounter{partie}{0} % Pour s'assurer que le compteur de \partie est à zéro dans les corrigés
    % \phantom{rrr}    
    \begin{spacing}{2}
    \begin{enumerate}        
        \item $\dfrac{18}{5} = \dfrac{18\times 2}{5\times 2} = \dfrac{36}{10} = 3,6$ et $\dfrac{72}{20} = \dfrac{72\div 2}{5\div 2} = \dfrac{36}{10} = 3,6$.
        \item $\dfrac{2}{3} = \dfrac{2\times 7}{3\times 7} = \dfrac{14}{21}$ et $\dfrac{7}{10,5}= \dfrac{7\times 2}{10,5\times 2} = \dfrac{14}{21}$.
        \item D'une manière générale, on peut tester si l'égalité des produits en croix est vraie.        
        
        Pour les quotients $\dfrac{\textcolor{red}{80}}{\textcolor{red}{29}}$ et $\dfrac{\textcolor{blue}{480}}{\textcolor{blue}{174}}$, 
        $\textcolor{red}{80}\times \textcolor{blue}{174} = \num{13920}$ et $\textcolor{blue}{480}\times \textcolor{red}{29} = \num{13920}$.

        Les produits en croix étant égaux, on peut en déduire que les quotients le sont aussi.        
    \end{enumerate}
    \end{spacing}
\end{corrige}

