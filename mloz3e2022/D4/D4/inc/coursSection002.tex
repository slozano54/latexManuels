\section{Effectifs et fréquences}
\begin{definition}[Effectifs et fréquences]
    Dans un tableau statistique, l'{\bf effectif} est le nombre de réponses associées \`{a} chaque valeur.\par
    L'ensemble des valeurs et des effectifs forme une {\bf série statistique}.\par
    En divisant l'effectif d'une valeur par l'effectif total, on obtient la {\bf fréquence}.
    $$f=\frac{\text{valeur de l'effectif}}{\text{valeur de l'effectif total}}$$
\end{definition}

\begin{remarque}
    Une fréquence peut s'exprimer :
    \begin{itemize}
        \item sous forme fractionnaire.
        \item sous forme de pourcentage, éventuellement en arrondissant.
        \item sous forme décimale, éventuellement en arrondissant.
    \end{itemize}
\end{remarque}

\begin{exemple*1}
    \titreExemple{Standardiste d'une radio FM}

    La standardiste d'une radio FM a noté le nombre d'appels téléphoniques reçus par tranches d'heures au cours d'une matinée. Elle obtient les résultats suivants :
    \smallskip
    \begin{spacing}{1}
        \begin{longtable}{|>{\columncolor{gray!20}\centering}m{0.45\textwidth}|*{5}{>{\centering\arraybackslash}m{0.11\textwidth}|}}%
            \hline
            \rowcolor{gray!20}{\bfseries Tranches horaires}&{\bfseries 9h-10h}&{\bfseries 10h-11h}&{\bfseries 11h-12h}&{\bfseries 12h-13h}&{\bfseries Total}\\
            \hline
            {\bfseries Effectifs\par(nombres d'appels)}&19&37&46&28&{\bf 130}\\
            \hline            
            {\bfseries Fréquences}&\parbox[c][2\baselineskip][c]{0.11\textwidth}{\hspace*{5mm}$\dfrac{19}{130}$}&&&&{\bf 1}\\
            \hline
            {\bfseries Fréquences en \%}&14,6&&&&{\bf 100}\\
            \hline
        \end{longtable}
    \end{spacing}
    \vspace*{-10mm}
\end{exemple*1}