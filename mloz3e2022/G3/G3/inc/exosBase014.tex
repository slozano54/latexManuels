\begin{exercice*}[D'après DNB]
    Grâce à une interface de codage par blocs, on a représenté ce motif.
    \scalebox{0.8}{     
        \begin{Geometrie}[CoinHD={(9u,4u)}]        
            % trace grille(0.5) withcolor gris;
            %labeloffset:=labeloffset*1.5;        
            pair A,B,C,D;
            A=u*(4,1);
            B=u*(8,1);
            C=rotation(A,B,-30);
            D=rotation(B,A,150);
            trace polygone(A,B,C,D);
            marque_s:=2;
            trace Codelongueur(A,B,B,C,C,D,D,A,2);
            trace Codeangle(D,C,B,0,btex \ang{150} etex);
            trace Codeangle(A,D,C,0,btex \ang{30} etex);
            trace cotationmil(A,B,-3mm,3mm,btex $50$ etex);
        \end{Geometrie}
    }

    \pagebreak
    Pour mémoire sur l'orientation.

    \smallskip
    \begin{Scratch}[Echelle=0.7]
        Place Boussole("90");
    \end{Scratch}
    
    \begin{enumerate}
        \item Compléter le programme ci-dessous en remplaçant les pointillés par les bonnes valeurs pour que le losange soit représenté tel qu'il est défini.\\
        \begin{Scratch}[Echelle=0.7]
            Place Drapeau;
            Place Effacer;
            Place Orienter("90");
            Place Bloc("Losange");
        \end{Scratch}
        \hspace*{10mm}
        \begin{Scratch}[Echelle=0.7]
            Place NouveauBloc("Losange");
            Place PoserStylo;
            Place Repeter("2");
            Place Avancer("\dots");
            Place Tournerd("\dots");
            Place Avancer("\dots");
            Place Tournerd("\dots");
            Place FinBlocRepeter;        
        \end{Scratch}    
        \item En utilisant le losange, on souhaite obtenir la rosace suivante.\\
        \begin{minipage}{0.6\linewidth}
            Déterminer la transformation géométrique partant du losange et répétée \num{12} fois qui a été utilisée pour obtenir la rosace.    
        \end{minipage}
        \begin{minipage}{0.4\linewidth}
            \begin{Geometrie}[CoinHD={(4u,4u)}]        
                \enonceTroisiemeGTroisExoQuatorze
            \end{Geometrie}   
        \end{minipage}        
        \item Compléter les pointillés dans le programme ci-dessous et colorier, sur la rosace ci-dessus, le premier losange réalisé par ce programme.\\
        \begin{Scratch}[Echelle=0.7]
            Place Drapeau;
            Place Effacer;
            Place Orienter("90");
            Place Repeter("12");
            Place Bloc("Losange");
            Place Tournerd("\dots");
            Place FinBlocRepeter;
        \end{Scratch}
    \end{enumerate}   
\end{exercice*}
\begin{corrige}
    %\setcounter{partie}{0} % Pour s'assurer que le compteur de \partie est à zéro dans les corrigés
    % \phantom{rrr}    
    Grâce à une interface de codage par blocs, on a représenté ce motif.
    \scalebox{0.8}{     
        \begin{Geometrie}[CoinHD={(9u,4u)}]        
            % trace grille(0.5) withcolor gris;
            %labeloffset:=labeloffset*1.5;        
            pair A,B,C,D;
            A=u*(4,1);
            B=u*(8,1);
            C=rotation(A,B,-30);
            D=rotation(B,A,150);
            trace polygone(A,B,C,D);
            marque_s:=2;
            trace Codelongueur(A,B,B,C,C,D,D,A,2);
            trace Codeangle(D,C,B,0,btex \ang{150} etex);
            trace Codeangle(A,D,C,0,btex \ang{30} etex);
            trace cotationmil(A,B,-3mm,3mm,btex $50$ etex);
        \end{Geometrie}
    }

    Pour mémoire sur l'orientation.

    \smallskip
    \begin{Scratch}[Echelle=0.7]
        Place Boussole("90");
    \end{Scratch}
    
    \begin{enumerate}
        \item Compléter le programme ci-dessous en remplaçant les pointillés par les bonnes valeurs pour que le losange soit représenté tel qu'il est défini.\\
        \begin{Scratch}[Echelle=0.7]
            Place Drapeau;
            Place Effacer;
            Place Orienter("90");
            Place Bloc("Losange");
        \end{Scratch}
        \hspace*{10mm}
        \begin{Scratch}[Echelle=0.7]
            Place NouveauBloc("Losange");
            Place PoserStylo;
            Place Repeter("2");
            Place Avancer("{\red 50}");
            Place Tournerd("{\red 30}");
            Place Avancer("{\red 50}");
            Place Tournerd("{\red 150}");
            Place FinBlocRepeter;        
        \end{Scratch}    
        \item En utilisant le losange, on souhaite obtenir la rosace suivante.\\
        \begin{minipage}{0.6\linewidth}
            Déterminer la transformation géométrique partant du losange et répétée \num{12} fois qui a été utilisée pour obtenir la rosace.\\
            {\red C'est une rotation de centre le point de départ, ou d'arrivée, du losange, de \ang{30} dans le sens horaire.}
        \end{minipage}
        \begin{minipage}{0.35\linewidth}
            \begin{Geometrie}[CoinHD={(4u,4u)}]        
                \enonceTroisiemeGTroisExoQuatorze
                %%% Correction
                fill A--B--C--D--cycle withcolor red;
            \end{Geometrie}   
        \end{minipage} 
    \end{enumerate}
    \Coupe
    \begin{enumerate}
        \setcounter{enumi}{2}
        \item Compléter les pointillés dans le programme ci-dessous et colorier, sur la rosace ci-dessus, le premier losange réalisé par ce programme.\\
        \begin{Scratch}[Echelle=0.7]
            Place Drapeau;
            Place Effacer;
            Place Orienter("90");
            Place Repeter("12");
            Place Bloc("Losange");
            Place Tournerd("{\red 30}");
            Place FinBlocRepeter;
        \end{Scratch}
    \end{enumerate}  
\end{corrige}

