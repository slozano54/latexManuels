\begin{exercice*}
    $A_1B_1C_1$ est l'image du triangle $ABC$ par une rotation. Déterminer son centre puis son angle.
    \hspace*{-10mm}
    \begin{Geometrie}[CoinHD={(9u,7u)}]        
        \enonceTroisiemeGTroisExoDouze
    \end{Geometrie}
\end{exercice*}
\begin{corrige}
    %\setcounter{partie}{0} % Pour s'assurer que le compteur de \partie est à zéro dans les corrigés
    % \phantom{rrr}    
    $A_1B_1C_1$ est l'image du triangle $ABC$ par une rotation. Déterminer son centre puis son angle.
    {\red Rotation de centre à l'intersection des médiatrices de $AA_1$, $BB_1$ et $CC_1$, d'angle \ang{120}, dans le sensinverse des aiguilles d'une montre.}

    \hspace*{-10mm}
    \begin{Geometrie}[CoinHD={(9u,7u)}]
        \enonceTroisiemeGTroisExoDouze
        %%% Correction
        pointe(O);
        label.urt(btex $O$ etex,O);
        marque_s:=2;
        drawoptions(withcolor red);
        trace mediatrice(C,CC);
        trace segment(C,CC);
        trace Codelongueur(C,milieu(C,CC),2);
        trace Codelongueur(CC,milieu(C,CC),2);
        trace codeperp(O,milieu(C,CC),CC,5);
        drawoptions(withcolor blue);
        trace mediatrice(B,BB);
        trace segment(B,BB);
        trace Codelongueur(B,milieu(B,BB),1);
        trace Codelongueur(BB,milieu(B,BB),1);
        trace codeperp(O,milieu(B,BB),BB,5);
        drawoptions(withcolor DarkGreen);
        trace mediatrice(A,AA);
        trace segment(A,AA);
        trace Codelongueur(A,milieu(A,AA),3);
        trace Codelongueur(AA,milieu(A,AA),3);
        trace codeperp(O,milieu(A,AA),A,5);
        drawoptions(dashed evenly withcolor (0.5,0,1));
        trace segment(C,O);
        trace segment(CC,O);
        drawoptions(withcolor (0.5,0,1));
        pair CCC;
        CCC=milieu(O,milieu(O,C));
        drawarrow CCC..rotation(CCC,O,60)..rotation(CCC,O,120);
        label.bot(btex $\ang{120}$ etex, rotation(CCC,O,60));
        drawoptions();    
    \end{Geometrie}
    \vspace*{-12mm}
\end{corrige}

