\begin{exercice*}
    Soit la rotation de centre $O$, d'angle \ang{60}, dans le sens inverse des aiguilles d'une montre.
    \begin{center}
        \begin{Geometrie}[CoinHD={(9u,7u)}]        
            \enonceTroisiemeGTroisExoTreize
        \end{Geometrie}
    \end{center}
    \begin{enumerate}
        \item Construire $A_1$, $B_1$, $C_1$, $D_1$ et $E_1$, les images respectives des points $A$, $B$, $C$, $D$ et $E$ par la rotation décrite ci-dessus.
        \item $A$ et $B$ sont sur le cercle de centre $O$ et passant par $A$. Faire une déduction sur leurs images.
        \item $C$ et $E$ appartiennent à la droite $(OA)$. Faire une déduction sur leurs images.
        \item $C$, $D$ et $E$ sont alignés. Faire une déduction sur leurs images.
    \end{enumerate}
\end{exercice*}
\begin{corrige}
    %\setcounter{partie}{0} % Pour s'assurer que le compteur de \partie est à zéro dans les corrigés
    % \phantom{rrr}
    Soit la rotation de centre $O$, d'angle \ang{60}, dans le sens inverse des aiguilles d'une montre.

    \hspace*{-15mm}
    \begin{Geometrie}[CoinHD={(9u,7u)}]        
        \enonceTroisiemeGTroisExoTreize
        %%% Correction
        trace cc withcolor blue;
        pair AA,BB,CC,DD,EE;
        AA=rotation(A,O,60);
        BB=rotation(B,O,60);
        CC=rotation(C,O,60);
        DD=rotation(D,O,60);
        EE=rotation(E,O,60);
        pointe(AA,BB,CC,DD,EE);        
        label.ulft(btex $A_1$ etex,AA);
        label.llft(btex $B_1$ etex,BB);
        label.rt(btex $C_1$ etex,CC);
        label.urt(btex $D_1$ etex,DD);
        label.llft(btex $E_1$ etex,EE);
        trace droite(C,E) dashed evenly withcolor red;
        trace droite(CC,EE) dashed evenly withcolor red;
        path ccc;
        ccc = cercles(O,.5*u);
        drawoptions(withcolor DarkGreen);
        drawarrow pointarc(ccc,60)..pointarc(ccc,90)..pointarc(ccc,120);
        label.top(btex $\ang{60}$ etex, pointarc(ccc,90));
        drawoptions(dashed evenly withcolor DarkGreen);
        trace segment(O,A);
        trace segment(O,AA);
        %%%
        drawoptions(withcolor (0.9,0.6,0.1));
        drawarrow pointarc(ccc,180)..pointarc(ccc,210)..pointarc(ccc,240);
        label.llft(btex $\ang{60}$ etex, pointarc(ccc,210));
        drawoptions(dashed evenly withcolor (0.9,0.6,0.1));
        trace segment(O,B);
        trace segment(O,BB);
        %%%
        drawoptions(withcolor (0,1,1));
        drawarrow pointarc(ccc,270)..pointarc(ccc,300)..pointarc(ccc,330);
        label.lrt(btex $\ang{60}$ etex, pointarc(ccc,300));
        drawoptions(dashed evenly withcolor (0,1,1));
        trace segment(O,C);
        trace segment(O,CC);
        %%%
        drawoptions(withcolor (1,0,1));
        drawarrow D..rotation(D,O,30)..DD;
        label.urt(btex $\ang{60}$ etex, rotation(D,O,30));
        drawoptions(dashed evenly withcolor (1,0,1));
        trace segment(O,D);
        trace segment(O,DD);
        %%%
        drawoptions(withcolor black);
        drawarrow E..rotation(E,O,30)..EE;
        label.urt(btex $\ang{60}$ etex, rotation(E,O,30));
        drawoptions(dashed evenly withcolor black);
        trace segment(O,E);
        trace segment(O,EE);
    \end{Geometrie}

    \begin{enumerate}
        \item {\red Construction}%Construire $A_1$, $B_1$, $C_1$, $D_1$ et $E_1$, les images respectives des points $A$, $B$, $C$, $D$ et $E$ par la rotation décrite ci-dessus.
        \item $A$ et $B$ sont sur le cercle de centre $O$ et passant par $A$. Faire une déduction sur leurs images.
        {\red Leurs images sont sur ce même cercle.}
        \item $C$ et $E$ appartiennent à la droite $(OA)$. Faire une déduction sur leurs images.
        {\red Leurs images sont sur la droite $(OA_1)$, image de $(OA)$.}
        \item $C$, $D$ et $E$ sont alignés. Faire une déduction sur leurs images.
        {\red Leurs images sont également alignés.}
    \end{enumerate}
\end{corrige}

