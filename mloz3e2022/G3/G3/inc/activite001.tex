\begin{activite}
    Sur la figure ci-dessous, le déplacement qui transforme $A$ en $A'$ s'appelle : rotation de centre $O$ qui transforme $A$ en $A'$, il a plusieurs caractéristiques : son centre, son angle et son sens.
    \begin{center}
        \includegraphics[scale=0.5]{\currentpath/images/activiteRotationTux.png}
    \end{center}
    \begin{enumerate}
        \item On dit que $A'$ est l'image de $A$ par cette rotation. Déterminer l'image de $B$.
        \item Indiquer la mesure de l'angle $\widehat{AOA'}$.
        \item Mesurer l'angle $\widehat{BOB'}$. Faire une remarque.
        \item Construire, sans le rapporteur, deux autres points $C$ et $C'$ tels que $\widehat{COC'}=\ang{130}$.
        
        Faire une remarque.
        \item Tracer et mesurer les segments $[AB]$ et $[A'B']$. Faire une remarque.
        \item Pour passer de Tux \textcircled{1} à Tux \textcircled{2}, on peu utiliser la rotation de centre $O$, d'angle \ang{130}, dans le sens inverse des aiguilles d'une montre.
        Déterminer les caractéristiques d'une autre rotation permettant le même déplacement.
    \end{enumerate}

    \begin{myBox}{\infoComplementsNumeriques{pluriel}}
        \begin{minipage}{\linewidth}
            \hrefConstruction{https://www.geogebra.org/classic/ga6b2key}{Animation Geogebra Tux}
            \creditGeogebra{David Evéquoz}
        \end{minipage}
    
        \begin{minipage}{\linewidth}
            \hrefConstruction{https://www.geogebra.org/classic/g7cj6udr}{Animation Geogebra triangle}
            \creditGeogebra{Élise Marchand}
        \end{minipage}
    \end{myBox}

\end{activite}