\begin{exercice*}
    \begin{enumerate}
        \item Construire, en rouge, l'image de la figure par la rotation de centre $S$ et d'angle \ang{120} dans le sens indiqué sur la figure.
        \item Construire, en bleu, l'image de la figure par la rotation de centre $S$ et d'angle \ang{240} dans le sens indiqué sur la figure.
        \item Déterminer une autre rotation permettant d'obtenir la figure bleue.
    \end{enumerate}
    \begin{Geometrie}[CoinHD={(7u,6u)}]
        % Pour changer la taille de la Grid
        coeff:=0.6;
        uni:=true;
        x.u:=coeff*cm;
        y.u:=coeff*(sqrt(3)/2)*cm;
        %%%                
        trace papiertriangle withcolor gris;
        marque_p:="croix";
        pair S;
        S=pptri(3,5);    
        pointe(S);
        label.top(btex $S$ etex,S);
        pair A,B,C,D,E,F,G,H,I,J,K,L,M,N;
        A=pptri(3,7);
        B=pptri(3,9);
        C=pptri(1,9);
        D=pptri(1,10);
        E=pptri(0,11);
        F=pptri(-1,11);
        G=pptri(-1,10);
        H=pptri(0,10);
        I=pptri(0,9);
        J=pptri(2,7);
        K=pptri(-0.5,10.5);
        L=pptri(5,9);
        path cc;
        cc=cercles(L,u);
        M=pointarc(cc,0);
        N=pointarc(cc,90);
        picture duck;
        duck = image ( 
            trace polygone(A,B,C,D,E,F,G,H,I,J);         
            marque_p:="plein";
            pointe(K);
        );
        trace duck;
        drawarrow arccercle(M,N,L);     
    \end{Geometrie}
\end{exercice*}
\begin{corrige}
    %\setcounter{partie}{0} % Pour s'assurer que le compteur de \partie est à zéro dans les corrigés
    % \phantom{rrr}    
    \begin{enumerate}
        \item Construire, en rouge, l'image de la figure par la rotation de centre $S$ et d'angle \ang{120} dans le sens indiqué sur la figure.
        \item Construire, en bleu, l'image de la figure par la rotation de centre $S$ et d'angle \ang{240} dans le sens indiqué sur la figure.
    \end{enumerate}
    \Coupe
    \begin{enumerate}
        \setcounter{enumi}{2}
        \item Déterminer une autre rotation permettant d'obtenir la figure bleue.
        {\red La figure bleue peut également être obtenue par la rotation de centre $S$ et d'angle \ang{120} dans le sens inverse de celui indiqué sur la figure.}
    \end{enumerate}
    \begin{Geometrie}[CoinHD={(7u,6u)}]
        % Pour changer la taille de la Grid
        coeff:=0.6;
        uni:=true;
        x.u:=coeff*cm;
        y.u:=coeff*(sqrt(3)/2)*cm;
        %%%                
        trace papiertriangle withcolor gris;
        marque_p:="croix";
        pair S;
        S=pptri(3,5);    
        pointe(S);
        label.top(btex $S$ etex,S);
        pair A,B,C,D,E,F,G,H,I,J,K,L,M,N;
        A=pptri(3,7);
        B=pptri(3,9);
        C=pptri(1,9);
        D=pptri(1,10);
        E=pptri(0,11);
        F=pptri(-1,11);
        G=pptri(-1,10);
        H=pptri(0,10);
        I=pptri(0,9);
        J=pptri(2,7);
        K=pptri(-0.5,10.5);
        L=pptri(5,9);
        path cc;
        cc=cercles(L,u);
        M=pointarc(cc,0);
        N=pointarc(cc,90);
        picture duck;
        duck = image ( 
            trace polygone(A,B,C,D,E,F,G,H,I,J);         
            marque_p:="plein";
            pointe(K);
        );
        trace duck;
        drawarrow arccercle(M,N,L);
        %%% Correction 1
        drawoptions(withcolor red);
        trace rotation(duck,S,120);
        %%% Correction 2
        drawoptions(withcolor blue);
        trace rotation(duck,S,240);
        drawoptions();      
    \end{Geometrie}
\end{corrige}

