    %\setcounter{partie}{0} % Pour s'assurer que le compteur de \partie est à zéro dans les corrigés
    % \phantom{rrr}
    Justifier que $HIJ$ et $FTS$ sont deux triangles semblables.

        \begin{tikzpicture}[scale=0.5]
            \quadrilageMailleCarree[white]{18}{6}
            % Points
            \coordinate (H) at (13,5);
            \coordinate (I) at (8,1);
            \coordinate (J) at (15,1);
            \coordinate (O) at (-30,1);
            \tkzDefPointBy[homothety=center O ratio 0.8](H); \tkzGetPoint{F};
            \tkzDefPointBy[homothety=center O ratio 0.8](I); \tkzGetPoint{T};
            \tkzDefPointBy[homothety=center O ratio 0.8](J); \tkzGetPoint{S};
            % Tracés
            \tkzLabelPoints[above](H,F);
            \tkzLabelPoints[below left](I,T);
            \tkzLabelPoints[below right](J,S);
            \tkzPicAngle["$\ang{25}$",draw=black,-,angle eccentricity=1.6,angle radius=5mm](S,T,F);
            \tkzPicAngle["$\ang{87}$",draw=black,-,angle eccentricity=1.4,angle radius=5mm](T,F,S);
            \tkzPicAngle["$\ang{25}$",draw=black,-,angle eccentricity=1.6,angle radius=5mm](J,I,H);
            \tkzPicAngle["$\ang{68}$",draw=black,-,angle eccentricity=1.4,angle radius=5mm](H,J,I);
            \draw (F) -- (T) -- (S) -- cycle;
            \draw (H) -- (I) -- (J) -- cycle;
        \end{tikzpicture}

    {\color{red} Dans le triangle $FTS$ : $\widehat{FST} = \ang{180} - (\ang{87} + \ang{25}) = \ang{180}-\ang{112}=\ang{68}$

    Dans le triangle $HIJ$ : $\widehat{IHJ} = \ang{180} - (\ang{25} + \ang{68}) = \ang{180}-\ang{93}=\ang{87}$

    Les angles des triangles $HIJ$ et $FTS$ sont deux à deux de même mesure, ils sont donc semblables.
    }
