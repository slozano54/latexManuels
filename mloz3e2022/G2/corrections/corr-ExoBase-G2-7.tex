    %\setcounter{partie}{0} % Pour s'assurer que le compteur de \partie est à zéro dans les corrigés
    \phantom{rrr}
    \newcommand{\pavageBaseGdeuxExSept}{%
        \foreach \a in {0,-2,-4}{%
            \draw[shift={(\a,-\a)}] (14,0) -- (14,2) -- (16,2) -- (16,4) -- (12,4) -- (12,0) -- cycle;
            \draw[shift={(\a,-\a)}] (15,2) -- (15,3) -- (13,3) -- (13,1) -- (14,1);
            \draw[shift={(\a,-\a)}] (12,2) -- (13,2);
            \draw[shift={(\a,-\a)}] (14,3) -- (14,4);
            \draw[shift={(\a,-\a)}] (14,0) -- (15,0) -- (15,1) -- (16,1) -- (16,2);
        }
        \foreach \a in {0}{%
            \draw[shift={(\a,-\a)},rotate around={90:(12,0)}] (14,0) -- (14,2) -- (16,2) -- (16,4) -- (12,4) -- (12,0) -- cycle;
            \draw[shift={(\a,-\a)},rotate around={90:(12,0)}] (15,2) -- (15,3) -- (13,3) -- (13,1) -- (14,1);
            \draw[shift={(\a,-\a)},rotate around={90:(12,0)}] (12,2) -- (13,2);
            \draw[shift={(\a,-\a)},rotate around={90:(12,0)}] (14,3) -- (14,4);
            \draw[shift={(\a,-\a)},rotate around={90:(12,0)}] (14,0) -- (15,0) -- (15,1) -- (16,1) -- (16,2);
        }
        \foreach \a in {0}{%
            \draw[shift={(\a,-\a)},rotate around={-90:(16,4)}] (14,0) -- (14,2) -- (16,2) -- (16,4) -- (12,4) -- (12,0) -- cycle;
            \draw[shift={(\a,-\a)},rotate around={-90:(16,4)}] (15,2) -- (15,3) -- (13,3) -- (13,1) -- (14,1);
            \draw[shift={(\a,-\a)},rotate around={-90:(16,4)}] (12,2) -- (13,2);
            \draw[shift={(\a,-\a)},rotate around={-90:(16,4)}] (14,3) -- (14,4);
            \draw[shift={(\a,-\a)},rotate around={-90:(16,4)}] (14,0) -- (15,0) -- (15,1) -- (16,1) -- (16,2);
        }
    }
    \newcommand{\imageGdeuxExSept}[2]{
        \tkzDefPointBy[homothety=center O ratio #1](A); \tkzGetPoint{Ab};
        \tkzDefPointBy[homothety=center O ratio #1](B); \tkzGetPoint{Bb};
        \tkzDefPointBy[homothety=center O ratio #1](C); \tkzGetPoint{Cb};
        \tkzDefPointBy[homothety=center O ratio #1](D); \tkzGetPoint{Db};
        \tkzDefPointBy[homothety=center O ratio #1](E); \tkzGetPoint{Eb};
        \tkzDefPointBy[homothety=center O ratio #1](F); \tkzGetPoint{Fb};
        \draw[color=#2,fill=#2,fill opacity=0.5] (Ab) -- (Bb) -- (Cb) -- (Db) -- (Eb) -- (Fb) -- cycle;
    }

    \begin{tikzpicture}[scale=0.5]
        \draw[help lines, color=black!30] (0,0) grid (16,16);
        \coordinate (O) at (16,0);
        \tkzDrawPoints[shape=cross out,size=3pt](O);
        \tkzLabelPoints[above](O);
        \pavageBaseGdeuxExSept
        \begin{scope}[shift={(-8,8)}]
            \pavageBaseGdeuxExSept
        \end{scope}
        \begin{scope}[rotate around={90:(8,0)}]
            \pavageBaseGdeuxExSept
        \end{scope}
        \begin{scope}[rotate around={-90:(16,8)}]
            \pavageBaseGdeuxExSept
        \end{scope}
        % Les points
        \coordinate (A) at (14,0);
        \coordinate (B) at (15,0);
        \coordinate (C) at (15,1);
        \coordinate (D) at (16,1);
        \coordinate (E) at (16,2);
        \coordinate (F) at (14,2);
        \draw[color=gray,fill=gray,fill opacity=0.5] (A) -- (B) -- (C) -- (D) -- (E) -- (F) -- cycle;
        \imageGdeuxExSept{2}{blue};
        \imageGdeuxExSept{4}{red};
        \imageGdeuxExSept{8}{mygreen};
    \end{tikzpicture}

    \begin{enumerate}
        \item Colorier \textbf{en bleu} l'image de la figure grise par l'homothétie de centre $O$ et de rapport $2$.
        \item Colorier \textbf{en rouge} l'image de la figure grise par l'homothétie de centre $O$ et de rapport $4$.
        \item Colorier \textbf{en vert} l'image de la figure grise par l'homothétie de centre $O$ et de rapport $8$.
    \end{enumerate}
