    %\setcounter{partie}{0} % Pour s'assurer que le compteur de \partie est à zéro dans les corrigés
    % \phantom{rrr}
    Terminer la construction de l'image du triangle $FMI$ par une homothétie de rapport $\num{0.5}$.

    \begin{tikzpicture}[scale=0.4]
        \quadrilageMailleCarree[white]{18}{8}
        % Points
        \coordinate (F) at (3,7);
        \coordinate (M) at (5,1);
        \coordinate (I) at (1,1);
        \coordinate (O) at (17,3);
        \tkzDefPointBy[homothety=center O ratio 0.5](F); \tkzGetPoint{F'};
        \tkzDefPointBy[homothety=center O ratio 0.5](M); \tkzGetPoint{M'};
        \tkzDefPointBy[homothety=center O ratio 0.5](I); \tkzGetPoint{I'};
        % Tracés
        \tkzLabelPoints[above](F,F');
        \tkzLabelPoints(I,M);
        \tkzDrawPoints[shape=cross out, size=5pt](F');
        \draw (F) -- (M) -- (I) -- cycle;
        % Correction
        \draw[color=red] (F') -- (M') -- (I') -- cycle;
        \tkzLabelPoints[color=red](I',M');
    \end{tikzpicture}

    {\color{red} Une homothétie de rapport $\num{0.5}$ multiplie les longueurs par $\num{0.5}$,
    c'est à dire qu'elle les divise par $2$ et les côtés homothétiques sont parallèles.

    Ces considérations suffisent à terminer la figure, inutile de construire le centre de l'homothétie.}

