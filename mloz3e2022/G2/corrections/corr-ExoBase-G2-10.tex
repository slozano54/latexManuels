    %\setcounter{partie}{0} % Pour s'assurer que le compteur de \partie est à zéro dans les corrigés
    % \phantom{rrr}
    Pour chacune des figure suivantes, construire le point $M_1$, image du point $M$ par  l'homothétie de centre $O$
    et de rapport $k$.

    \newcommand{\scaleGdeuxExDix}{0.42}
    \begin{enumerate}
        \item $k=\dfrac{5}{7}$

        \begin{tikzpicture}[scale=\scaleGdeuxExDix]
            \axeHomothety{2}{9}
            \imageHomothetyPoint[red]{5/7}
        \end{tikzpicture}
        \item $k=\dfrac{10}{7}$

        \begin{tikzpicture}[scale=\scaleGdeuxExDix]
            \axeHomothety{2}{9}
            \imageHomothetyPoint[red]{10/7}
        \end{tikzpicture}
        \item $k=2$

        \begin{tikzpicture}[scale=\scaleGdeuxExDix]
            \axeHomothety{2}{9}
            \imageHomothetyPoint[red]{2}
        \end{tikzpicture}
        \item $k=-1$

        \begin{tikzpicture}[scale=\scaleGdeuxExDix]
            \axeHomothety{9}{14}
            \imageHomothetyPoint[red]{-1}
        \end{tikzpicture}
        \item $k=-\dfrac{3}{5}$

        \begin{tikzpicture}[scale=\scaleGdeuxExDix]
            \axeHomothety{9}{14}
            \imageHomothetyPoint[red]{-3/5}
        \end{tikzpicture}
        \item $k=-\dfrac{7}{5}$

        \begin{tikzpicture}[scale=\scaleGdeuxExDix]
            \axeHomothety{9}{14}
            \imageHomothetyPoint[red]{-7/5}
        \end{tikzpicture}
    \end{enumerate}
