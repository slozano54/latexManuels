\begin{exercice*}
    Les triangles $ABC$ et $ADE$ sont tels que :
    \begin{itemize}
        \item $AB=\Lg{3.2}$ ; $AC=\Lg{4.4}$ ; $BC=\Lg{6.8}$.
        \item $AD=\Lg{4}$ ; $AE=\Lg{5.5}$ ; $DE=\Lg{8.5}$.
        \item $A$, $B$ et $D$ sont alignés dans cet ordre.
        \item $A$, $C$ et $E$ sont alignés dans cet ordre.
    \end{itemize}

    Déterminer si ces triangles sont semblables. Justifier.

        \begin{tikzpicture}[scale=0.35]            
            % \quadrilageMailleCarree[white]{18}{8}
            % Points
            \coordinate (A) at (4,7);
            \coordinate (B) at (2,2);
            \coordinate (C) at (13,3);                        
            \tkzDefPointBy[homothety=center A ratio 1.2](B); \tkzGetPoint{D};
            \tkzDefPointBy[homothety=center A ratio 1.2](C); \tkzGetPoint{E};
            % Tracés
            \tkzLabelPoints[above](A);
            \tkzLabelPoints[above left](B);            
            \tkzLabelPoints[above right](C); 
            \tkzLabelPoints[below left](D);            
            \tkzLabelPoints[below right](E);
            \draw (A) -- (D) -- (E) -- cycle;
            \draw (B) -- (C);
            % Texte
            \coordinate (texte) at (11,6);
            \tkzLabelPoint[right](texte){\parbox{0.4\linewidth}{Cette figure n'est pas en vraie grandeur.}}
        \end{tikzpicture} 
\end{exercice*}
\begin{corrige}
    %\setcounter{partie}{0} % Pour s'assurer que le compteur de \partie est à zéro dans les corrigés
    % \phantom{rrr}
    Les triangles $ABC$ et $ADE$ sont tels que :
    \begin{itemize}
        \item $AB=\Lg{3.2}$ ; $AC=\Lg{4.4}$ ; $BC=\Lg{6.8}$.
        \item $AD=\Lg{4}$ ; $AE=\Lg{5.5}$ ; $DE=\Lg{8.5}$.
    \end{itemize}

    Déterminer si ces triangles sont semblables. Justifier.

        \begin{tikzpicture}[scale=0.5]            
            % \quadrilageMailleCarree[white]{18}{8}
            % Points
            \coordinate (A) at (4,7);
            \coordinate (B) at (2,2);
            \coordinate (C) at (13,3);                        
            \tkzDefPointBy[homothety=center A ratio 1.2](B); \tkzGetPoint{D};
            \tkzDefPointBy[homothety=center A ratio 1.2](C); \tkzGetPoint{E};
            % Tracés
            \tkzLabelPoints[above](A);
            \tkzLabelPoints[above left](B);            
            \tkzLabelPoints[above right](C); 
            \tkzLabelPoints[below left](D);            
            \tkzLabelPoints[below right](E);
            \draw (A) -- (D) -- (E) -- cycle;
            \draw (B) -- (C);
            % Texte
            \coordinate (texte) at (10,5);
            \tkzLabelPoint[right](texte){\parbox{0.5\linewidth}{Cette figure n'est pas en vraie grandeur.}}
        \end{tikzpicture} 

        {\color{red} $\dfrac{AC}{AE}=\dfrac{\num{4.4}}{\num{5.5}}=\num{0.8}$ ; 

        $\dfrac{AB}{AD}=\dfrac{\num{3.2}}{\num{4}}=\num{0.8}$ ; 

        $\dfrac{BC}{DE}=\dfrac{\num{6.8}}{\num{8.5}}=\num{0.8}$ ; 
        
        Les quotients sont égaux donc les triangles sont semblables.
        }
\end{corrige}

