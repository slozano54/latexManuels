\subsection{Translation}
\newcommand{\coursVecteurFigUn}[1]{
    %line width=1pt,line cap=round,line join=round,>=triangle 45,x=1cm,y=1cm,
    \begin{tikzpicture}[scale=#1,>=triangle 45]
        \coordinate (A) at (-6,-6);
        \coordinate (C) at (-3,-4);
        \coordinate (E) at (-2,-1);
        \coordinate (G) at (-5,-3);
        \coordinate (B) at (1,-5);
        \tkzDefPointBy[translation= from A to B](G);
        \tkzGetPoint{H};
        \tkzDefPointBy[translation= from A to B](E);
        \tkzGetPoint{F};
        \tkzDefPointBy[translation= from A to B](C);
        \tkzGetPoint{D};
        \draw (A)--(C)--(E)--(G)--(A);
        \draw (B)--(D)--(F)--(H)--(B);
        \draw [->,color=orange] (A) -- (B);
        \draw [->,color=orange] (G) -- (H);
        \draw [->,color=orange] (E) -- (F);
        \draw [->,color=orange] (C) -- (D);
        \tkzDefMidPoint(A,B);
        \tkzGetPoint{M};
        \tkzLabelPoint[below](M){$\overrightarrow{AB}$}        
        \tkzDrawPoints[shape=cross out,size=2pt](A,B,C,D,E,F,G,H);
        \tkzLabelPoints[below](A,B,C,D);
        \tkzLabelPoints[above](E,F,G,H);
    \end{tikzpicture}
}
\begin{minipage}{0.6\linewidth}
    \begin{definition}
        $H$ est l’image de $G$ par la translation \textbf{qui transforme $\mathbf{A}$ en $\mathbf{B}$} signifie que $ABHG$ est un parallélogramme.
    \end{definition}
\end{minipage}
\begin{minipage}{0.4\linewidth}
    \hspace*{-10mm}
    \coursVecteurFigUn{0.7}
\end{minipage}

\begin{remarque}
    Une translation fait glisser une figure dans une direction, un sens et une longueur donnés.
    \begin{myBox}{\infoComplementsNumeriques{singulier}}
        \hrefConstruction{https://www.geogebra.org/classic/efdft7qt}{Animation Geogebra}
        \creditGeogebra{Gauthier}
    \end{myBox}
\end{remarque}

\begin{methode*1}[Construction de l'image d'un quadrilatère par translation]    
    \exercice
    Construire l'image de $ACEG$ par la translation qui transforme $A$ en $B$.
    \correction
    \begin{minipage}{0.45\linewidth}
        \begin{center}
            \coursVecteurFigUn{0.5}
        \end{center}
    \end{minipage}
    \begin{minipage}{0.55\linewidth}
        \textbf{Programme de construction}

        Pour construire l'image de $ACEG$ par \mbox{la translation de vecteur $\overrightarrow{AB}$}, il faut reproduire ce vecteur à partir de chaque sommet de la figure.
    \end{minipage}
    \begin{myBox}{\infoComplementsNumeriques{singulier}}
        \hrefConstruction{http://lozano.maths.free.fr/iep_local/figures_html/scr_iep_122.html}{Image d'un point à l'aide du compas}
        \creditInstrumentPoche
    \end{myBox}
\end{methode*1}
