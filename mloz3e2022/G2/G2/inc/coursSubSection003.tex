\subsection{Translation}
% \definNum{
% $M’$ est l’image de $M$ par la translation \textbf{qui transforme $\mathbf{A}$ en $\mathbf{B}$} signifie que : $ABM’M$ est un parallélogramme.
% }
% \newcommand{\coursVecteurFigUn}[1]{
% \definecolor{ududff}{rgb}{0.30196078431372547,0.30196078431372547,1}
% \begin{tikzpicture}[line width=1pt,line cap=round,line join=round,>=triangle 45,x=1cm,y=1cm,scale=#1]
% \clip(-7,-7) rectangle (6,1);
% \draw (-6,-6)-- (-3,-4);
% \draw (-3,-4)-- (-2,-1);
% \draw (-2,-1)-- (-5,-3);
% \draw (-5,-3)-- (-6,-6);
% \draw (2,-2)-- (1,-5);
% \draw (1,-5)-- (4,-3);
% \draw (4,-3)-- (5,0);
% \draw (5,0)-- (2,-2);
% \draw [->,color=orange] (-2,-1) -- (5,0);
% \draw [->,color=orange] (-5,-3) -- (2,-2);
% \draw [->,color=orange] (-3,-4) -- (4,-3);
% \draw [->,color=orange] (-6,-6) -- (1,-5);
% \draw (-2.426427963055143,-5.405001161440179) node[anchor=north west] {$\overrightarrow{AB}$};
% \begin{scriptsize}
% \draw (-6,-6)-- ++(-2.5pt,-2.5pt) -- ++(5pt,5pt) ++(-5pt,0) -- ++(5pt,-5pt);
% \draw (-6.475761517615175,-5.6483505337094115) node {$A$};
% \draw (1,-5)-- ++(-2.5pt,-2.5pt) -- ++(5pt,5pt) ++(-5pt,0) -- ++(5pt,-5pt);
% \draw (0.9609952989325763,-5.317395387423256) node {$B$};
% \draw (-3,-4)-- ++(-2.5pt,-2.5pt) -- ++(5pt,5pt) ++(-5pt,0) -- ++(5pt,-5pt);
% \draw (-3.3608895525689966,-3.5458119573032416) node {$C$};
% \draw (4,-3)-- ++(-2.5pt,-2.5pt) -- ++(5pt,5pt) ++(-5pt,0) -- ++(5pt,-5pt);
% \draw (4.484694209391066,-2.552946518444773) node {$D$};
% \draw (-2,-1)-- ++(-2.5pt,-2.5pt) -- ++(5pt,5pt) ++(-5pt,0) -- ++(5pt,-5pt);
% \draw (-2.212280515458218,-0.6450874398539891) node {$E$};
% \draw (5,0)-- ++(-2.5pt,-2.5pt) -- ++(5pt,5pt) ++(-5pt,0) -- ++(5pt,-5pt);
% \draw (5.146604501963379,0.4256497981306344) node {$F$};
% \draw (-5,-3)-- ++(-2.5pt,-2.5pt) -- ++(5pt,5pt) ++(-5pt,0) -- ++(5pt,-5pt);
% \draw (-5.307684530722859,-2.6113503677893886) node {$G$};
% \draw (2,-2)-- ++(-2.5pt,-2.5pt) -- ++(5pt,5pt) ++(-5pt,0) -- ++(5pt,-5pt);
% \draw (1.8370530391018138,-1.3848695315524564) node {$H$};
% \end{scriptsize}
% \end{tikzpicture}
% }
% \begin{center}
% %\includegraphics{coursvecteurs.1}
% \coursVecteurFigUn{1}
% \end{center}

% \CadreLampe{Remarque}{
% Une translation fait glisser une figure dans une direction, un sens et une longueur donnés.
% \infoComplementNumerique
% \lienCadre{https://www.geogebra.org/classic/efdft7qt}{Animation Geogebra}
% \creditGeogebra{Gauthier}
% }

% \Methode{}{
% \begin{center}
% \begin{minipage}{6cm}
% %\includegraphics[scale=0.5]{coursvecteurs.1}
% \coursVecteurFigUn{0.5}
% \end{minipage}
% \begin{minipage}{10cm}
% \begin{center}
% \underline{Programme de construction} :\par
% Pour construire l'image de ACEG par \mbox{la translation de vecteur $\vecteur{\strut AB}$}, il faut reproduire ce vecteur à partir de chaque sommet de la figure.
% \end{center}
% \end{minipage}
% \infoComplementNumerique
% \lienCadre{http://lozano.maths.free.fr/iep_local/figures_html/scr_iep_122.html}{Construction de l'image d'un point à l'aide du compas}
% \creditInstrumentPoche
% \end{center}
% }
