\subsection{Définition et construction}
\begin{definition}
    Étant donnés un point $O$ et un nombre $k$ non nul, on appelle \textbf{homothétie de centre $\mathbf{O}$ et de rapport $\mathbf{k}$}, la transformation qui à tout point $M$ associe le point $M'$ tel que :
    \begin{itemize}
        \item les points $O$, $M$ et $M'$ sont alignés.
        \item Situation de $M'$ selon le signe de $k$ :
        
        \hspace*{1cm}$\gtrdot$ si $k$ est \textbf{positif}, $M'$ est sur la demi-droite $[OM)$

        \hspace*{1cm}$\gtrdot$ si $k$ est \textbf{négatif}, $M'$ est sur la demi-droite d'origine $O$ qui ne contient pas $M$.
        \item la distance de $O$ à $M'$ est égale à :
        
        \hspace*{1cm}$\gtrdot$ $k$ fois la distance de $O$ à $M$ si $k$ est \textbf{positif}

        \hspace*{1cm}$\gtrdot$ $-k$ fois la distance de $O$ à $M$ si $k$ est \textbf{négatif}
    \end{itemize}
\end{definition}

\begin{exemple*1}
    {\color{red}\bfseries Homothétie de rapport positif}

    $M’$ est l’image de $M$ par l’homothétie de centre $O$ et de rapport $2$ signifie que :
    \begin{itemize}
        \item les points $O$, $M$ et $M'$ sont alignés.
        \item $M'$ est sur la demi-droite $[OM)$
        \item $OM'=2\times OM$ que l'on peut aussi écrire comme avec le théorème de Thalès : $\dfrac{OM'}{OM}=2$
    \end{itemize}
    \begin{center}
        \includegraphics[scale=0.6]{\currentpath/images/coursHomothetieAvion1.png} 
    \end{center}
\end{exemple*1}

\begin{exemple*1}
    {\color{red}\bfseries Homothétie de rapport négatif}

    $M’$ est l’image de $M$ par l’homothétie de centre $O$ et de rapport $-0.5$ signifie que :
    \begin{itemize}
        \item les points $O$, $M$ et $M'$ sont alignés.
        \item $M'$ est sur la demi-droite d'origine $O$ qui ne contient pas $M$.
        \item $OM'=-(-0,5)\times OM = 0,5\times OM$
    \end{itemize}
    \begin{center}
        \includegraphics[scale=0.6]{\currentpath/images/coursHomothetieAvion2.png} 
    \end{center}
\end{exemple*1}

% \Methode{}{
% \begin{center}
% \begin{minipage}{8cm}
% \includegraphics[scale=0.5]{../figures/courshomothetie3.ps}
% \end{minipage}
% \hfil
% \begin{minipage}{6cm}
% On construit respectivement les images $A’$, $B’$ et $C’$ de $A$, $B$ et $C$
% par l’homothétie de centre $O$ et de rapport $-2$.
% \end{minipage}
% \par
% \textbf{Par exemple pour construire $A’$ :}
% \begin{mylist}
% \item Tracer la droite $(OA)$.
% \item L’image $A’$ de $A$ se trouve sur la demi-droite d'origine $O$ qui ne contient pas $A$ car le rapport est négatif.
% \item $OA’= -(-2)\times OA = 2\times OA$.
% \end{mylist}
% On fait de même pour construire $B’$ et $C’$.
% \infoComplementNumerique
% \lienCadre{}{à venir}
% \end{center}
% }
