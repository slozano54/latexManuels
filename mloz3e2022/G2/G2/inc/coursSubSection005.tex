\subsection{Définition et construction}
\begin{definition}
    Étant donnés un point $O$ et un nombre $k$ non nul, on appelle \textbf{homothétie de centre $\mathbf{O}$ et de rapport $\mathbf{k}$}, la transformation qui à tout point $M$ associe le point $M'$ tel que :
    \begin{itemize}
        \item les points $O$, $M$ et $M'$ sont alignés.
        \item Situation de $M'$ selon le signe de $k$ :
        
        \hspace*{1cm}$\gtrdot$ si $k$ est \textbf{positif}, $M'$ est sur la demi-droite $[OM)$

        \hspace*{1cm}$\gtrdot$ si $k$ est \textbf{négatif}, $M'$ est sur la demi-droite d'origine $O$ qui ne contient pas $M$.
        \item la distance de $O$ à $M'$ est égale à :
        
        \hspace*{1cm}$\gtrdot$ $k$ fois la distance de $O$ à $M$ si $k$ est \textbf{positif}

        \hspace*{1cm}$\gtrdot$ $-k$ fois la distance de $O$ à $M$ si $k$ est \textbf{négatif}
    \end{itemize}
\end{definition}

\begin{exemple*1}
    {\color{red}\bfseries Homothétie de rapport positif}

    $M’$ est l’image de $M$ par l’homothétie de centre $O$ et de rapport $2$ signifie que :
    \begin{itemize}
        \item les points $O$, $M$ et $M'$ sont alignés.
        \item $M'$ est sur la demi-droite $[OM)$
        \item $OM'=2\times OM$ que l'on peut aussi écrire comme avec le théorème de Thalès : $\dfrac{OM'}{OM}=2$
    \end{itemize}
    \begin{center}
        \includegraphics[scale=0.6]{\currentpath/images/coursHomothetieAvion1.png} 
    \end{center}
\end{exemple*1}

\begin{exemple*1}
    {\color{red}\bfseries Homothétie de rapport négatif}

    $M’$ est l’image de $M$ par l’homothétie de centre $O$ et de rapport $-0.5$ signifie que :
    \begin{itemize}
        \item les points $O$, $M$ et $M'$ sont alignés.
        \item $M'$ est sur la demi-droite d'origine $O$ qui ne contient pas $M$.
        \item $OM'=-(-0,5)\times OM = 0,5\times OM$
    \end{itemize}
    \begin{center}
        \includegraphics[scale=0.6]{\currentpath/images/coursHomothetieAvion2.png} 
    \end{center}
\end{exemple*1}

\begin{methode*1}[Construction de l'image d'un triangle par homothétie]
    \exercice
        Construire l'image de $ABC$ par l'homothétie de centre $O$ et de rapport $-2$.
    \correction    
        \begin{tikzpicture}[scale=0.6]
            % Points
            \coordinate (O) at (0,0);
            \coordinate (A) at (2,0);
            \coordinate (B) at (4,2);
            \coordinate (C) at (5,-2);
            \tkzDefPointBy[homothety=center O ratio -2](A);\tkzGetPoint{A'};
            \tkzDefPointBy[homothety=center O ratio -2](B);\tkzGetPoint{B'};
            \tkzDefPointBy[homothety=center O ratio -2](C);\tkzGetPoint{C'};
            % Tracés
            \tkzLabelPoints(O,A,C);
            \tkzLabelPoints[above](B,C');
            \tkzLabelPoints(A',B');
            \draw[fill=gray!10] (A)--(B)--(C)--(A);
            \draw[fill=gray!10] (A')--(B')--(C')--(A');
            \tkzDrawSegment[dashed,add=0.1 and 0.1](A,A');
            \tkzDrawSegment[dashed,add=0.05 and 0.05](B,B');
            \tkzDrawSegment[dashed,add=0.05 and 0.05](C,C');
        \end{tikzpicture}

        On construit respectivement les images $A’$, $B’$ et $C’$ de $A$, $B$ et $C$
        par l’homothétie de centre $O$ et de rapport $-2$.

        \textbf{Programme de construction pour A'}
        \begin{enumerate}
            \item Tracer la droite $(OA)$.
            \item L’image $A’$ de $A$ se trouve sur la demi-droite d'origine $O$ qui ne contient pas $A$ car le rapport est négatif.
            \item $OA’= -(-2)\times OA = 2\times OA$.
        \end{enumerate}
        On fait de même pour construire $B’$ et $C’$.

    \begin{myBox}{\infoComplementsNumeriques{pluriel}}
                \hrefConstruction{http://lozano.maths.free.fr/iep_local/figures_html/homothetie001.html}{Image d'un point par une homothétie de rapport positif}

        \hrefConstruction{http://lozano.maths.free.fr/iep_local/figures_html/homothetie002.html}{Image d'un point par une homothétie de rapport négatif}

        \hrefConstruction{http://lozano.maths.free.fr/iep_local/figures_html/homothetie002.html}{Image d'un carré par une homothétie}
        \creditInstrumentPoche
    \end{myBox}
\end{methode*1}