\begin{propriete}[\admise]
Si deux triangles sont semblables alors leurs angles sont deux à deux égaux.
\end{propriete}

\begin{exemple*1}
    \titreExemple{Déduction}

    \begin{center}
        Puisque les triangles $BAL$ et $EDF$ sont semblables alors \textbf{ils ont les mêmes angles}.

        \begin{tikzpicture}[scale=0.8]
            \coordinate (A) at (0,4);            
            \coordinate (B) at (-2,2);
            \coordinate (L) at (-2,-1);
            \coordinate (F) at (-3,-3);
            \coordinate (E) at (6,-3);
            \coordinate (D) at (12,3);
            \tkzDrawSegment[color=red](B,L);
            \tkzDrawSegment[color=red](F,E);
            \tkzDrawSegment[color=blue](A,B);
            \tkzDrawSegment[color=blue](E,D);
            \tkzDrawSegment(L,A);
            \tkzDrawSegment(F,D);
            \tkzLabelPoints(L,F,E);
            \tkzLabelPoints[above](A,D);
            \tkzLabelPoints[left](B);
            \tkzMarkAngle[size=0.6,color=mygreen](L,B,A);
            \tkzMarkAngle[size=0.6,color=mygreen](D,E,F);            
            \tkzMarkAngle[color=red](A,L,B);
            \tkzMarkAngle[size=0.6,color=red](E,F,D);
            \tkzMarkAngle[color=blue](B,A,L);
            \tkzMarkAngle[color=blue](F,D,E);            
            \tkzFillAngle[size=0.6,fill=mygreen,fill opacity=0.2](L,B,A);
            \tkzFillAngle[size=0.6,fill=mygreen,fill opacity=0.2](D,E,F);            
            \tkzFillAngle[color=red,fill=red,fill opacity=0.2](A,L,B);
            \tkzFillAngle[size=0.6,fill=red,fill opacity=0.2](E,F,D);
            \tkzFillAngle[color=blue,fill=blue,fill opacity=0.2](B,A,L);
            \tkzFillAngle[color=blue,fill=blue,fill opacity=0.2](F,D,E);            
        \end{tikzpicture}
    \end{center}
\end{exemple*1}

\begin{propriete}[\admise]
Si deux triangles ont leurs angles deux à deux égaux alors ils sont semblables.
\end{propriete}

\begin{exemple*1}
    \titreExemple{Réciproquement}

    \begin{center}
        Dans Les triangles $BAL$ et $EDF$ on calcule le troisième angle.

        $\widehat{BLA} = \ang{180} - \ang{135} - \ang{23} = \ang{22}$ et $\widehat{DFE} = \ang{180} - \ang{135} - \ang{23} = \ang{22}$

        Puisque les triangles $BAL$ et $EDF$ ont les mêmes angles, alors \textbf{ils sont semblables}.

        \begin{tikzpicture}[scale=0.8]
            \coordinate (A) at (0,4);            
            \coordinate (B) at (-2,2);
            \coordinate (L) at (-2,-1);
            \coordinate (F) at (-3,-3);
            \coordinate (E) at (6,-3);
            \coordinate (D) at (12,3);
            \tkzDrawSegment[color=red](B,L);
            \tkzDrawSegment[color=red](F,E);
            \tkzDrawSegment[color=blue](A,B);
            \tkzDrawSegment[color=blue](E,D);
            \tkzDrawSegment(L,A);
            \tkzDrawSegment(F,D);
            \tkzLabelPoints(L,F,E);
            \tkzLabelPoints[above](A,D);
            \tkzLabelPoints[left](B);
            \tkzMarkAngle[color=mygreen,size=0.6](L,B,A);            
            \tkzMarkAngle[color=mygreen,size=0.6](D,E,F);            
            \tkzMarkAngle[color=red](A,L,B);
            \tkzMarkAngle[color=red,size=0.6](E,F,D);
            \tkzMarkAngle[color=blue](B,A,L);
            \tkzMarkAngle[color=blue](F,D,E);
            \tkzFillAngle[color=mygreen,size=0.6,fill=mygreen,fill opacity=0.2](L,B,A);            
            \tkzFillAngle[color=mygreen,size=0.6,fill=mygreen,fill opacity=0.2](D,E,F);            
            \tkzFillAngle[color=red,fill=red,fill opacity=0.2](A,L,B);
            \tkzFillAngle[color=red,size=0.6,fill=red,fill opacity=0.2](E,F,D);
            \tkzFillAngle[color=blue,fill=blue,fill opacity=0.2](B,A,L);
            \tkzFillAngle[color=blue,fill=blue,fill opacity=0.2](F,D,E);
            \tkzLabelAngle[pos=0.8,shift={(0.5,0)}](B,A,L){\scriptsize $\ang{23}$};
            \tkzLabelAngle[pos=0.8,shift={(-0.1,-0.3)}](L,B,A){\scriptsize $\ang{135}$};          
            \tkzLabelAngle[pos=1.3](F,D,E){\scriptsize $\ang{23}$};
            \tkzLabelAngle[pos=0.8](D,E,F){\scriptsize $\ang{135}$};
        \end{tikzpicture}
    \end{center}
\end{exemple*1}