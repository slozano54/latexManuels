\subsection{Symétrie axiale}
\begin{definition}
    $A$ et $A’$ sont symétriques \textbf{par rapport à la droite $\mathbf{(d)}$} signifie que :
    \begin{itemize}
        \item $[AA’]$ est perpendiculaire à $(d)$,
        \item $A$ et $A’$ sont à égale distance de $(d)$.
    \end{itemize}
\end{definition}

\begin{center}
    \psset{xunit=0.6cm,yunit=0.6cm,algebraic=true,dimen=middle,dotstyle=x,dotsize=3pt 0,linewidth=0.8pt,arrowsize=3pt 2,arrowinset=0.25}
    \begin{pspicture*}(-3.66,-3.88)(9.8,3.86)
        \pspolygon[linecolor=black,fillcolor=red!20,fillstyle=solid,opacity=0.4](1,2)(-1,2)(-1,0)(-3,-2)(-1,-2)(0,-1)(1,-2)(1,0)(0,1)
        \pspolygon[linecolor=black,fillcolor=mygreen!20,fillstyle=solid,opacity=0.4](5,2)(7,2)(7,0)(9,-2)(7,-2)(6,-1)(5,-2)(5,0)(6,1)
        \pspolygon[linecolor=black](2.58,2)(2.58,1.58)(3,1.58)(3,2)
        \psline[linecolor=black](1,2)(-1,2)
        \psline[linecolor=black](-1,2)(-1,0)
        \psline[linecolor=black](-1,0)(-3,-2)
        \psline[linecolor=black](-3,-2)(-1,-2)
        \psline[linecolor=black](-1,-2)(0,-1)
        \psline[linecolor=black](0,-1)(1,-2)
        \psline[linecolor=black](1,-2)(1,0)
        \psline[linecolor=black](1,0)(0,1)
        \psline[linecolor=black](0,1)(1,2)
        \psline[linecolor=red](3,6)(3,-3)
        \psline[linecolor=black](5,2)(7,2)
        \psline[linecolor=black](7,2)(7,0)
        \psline[linecolor=black](7,0)(9,-2)
        \psline[linecolor=black](9,-2)(7,-2)
        \psline[linecolor=black](7,-2)(6,-1)
        \psline[linecolor=black](6,-1)(5,-2)
        \psline[linecolor=black](5,-2)(5,0)
        \psline[linecolor=black](5,0)(6,1)
        \psline[linecolor=black](6,1)(5,2)
        \rput[tl](-0.8,3.4){figure rouge}
        \rput[tl](5.44,3.44){figure verte}
        \rput[tl](2.56,-3.1){axe (d)}
        \psline(1,2)(5,2)
        \psline(1,2)(3,2)
        \psline(1.93,2.09)(1.93,1.91)
        \psline(2,2.09)(2,1.91)
        \psline(2.07,2.09)(2.07,1.91)
        \psline(3,2)(5,2)
        \psline(3.93,2.09)(3.93,1.91)
        \psline(4,2.09)(4,1.91)
        \psline(4.07,2.09)(4.07,1.91)
        \begin{scriptsize}
            \psdots[linecolor=blue](1,2)
            \rput[bl](1.08,2.12){\blue{$A$}}
            \psdots[linecolor=blue](-1,2)
            \rput[bl](-0.92,2.12){\blue{$B$}}
            \psdots[linecolor=blue](-1,0)
            \rput[bl](-1.38,0.12){\blue{$C$}}
            \psdots[linecolor=blue](-3,-2)
            \rput[bl](-3.06,-2.36){\blue{$D$}}
            \psdots[linecolor=blue](-1,-2)
            \rput[bl](-0.9,-2.36){\blue{$E$}}
            \psdots[linecolor=blue](0,-1)
            \rput[bl](0.08,-0.88){\blue{$F$}}
            \psdots[linecolor=blue](1,-2)
            \rput[bl](0.98,-2.34){\blue{$G$}}
            \psdots[linecolor=blue](1,0)
            \rput[bl](1.08,0.12){\blue{$H$}}
            \psdots[linecolor=blue](0,1)
            \rput[bl](0.4,0.94){\blue{$I$}}
            \psdots[linecolor=blue](5,2)
            \rput[bl](5.08,2.12){\blue{$A'$}}
            \psdots[linecolor=blue](7,2)
            \rput[bl](7.08,2.12){\blue{$B'$}}
            \psdots[linecolor=blue](7,0)
            \rput[bl](7.08,0.12){\blue{$C'$}}
            \psdots[linecolor=blue](9,-2)
            \rput[bl](9.08,-1.88){\blue{$D'$}}
            \psdots[linecolor=blue](7,-2)
            \rput[bl](7.08,-1.88){\blue{$E'$}}
            \psdots[linecolor=blue](6,-1)
            \rput[bl](6.08,-0.88){\blue{$F'$}}
            \psdots[linecolor=blue](5,-2)
            \rput[bl](4.72,-2.4){\blue{$G'$}}
            \psdots[linecolor=blue](5,0)
            \rput[bl](4.78,0.2){\blue{$H'$}}
            \psdots[linecolor=blue](6,1)
            \rput[bl](6.08,1.12){\blue{$I'$}}
            \psdots[linecolor=black](3,2)
            \rput[bl](3.08,2.12){\black{$M$}}
        \end{scriptsize}
    \end{pspicture*}
\end{center}

\begin{remarque}
    Deux figures symétriques par rapport à un axe se superposent par un pliage le long de cet axe.
    \begin{myBox}{\infoComplementsNumeriques{singulier}}
        \hrefConstruction{https://www.geogebra.org/classic/jpxdv4ag}{Animation Geogebra}
        \creditGeogebra{Deguelle, Nolane}
    \end{myBox}
\end{remarque}

\begin{methode*1}[Construction de l'image de $M$ par rapport à l'axe $(d)$]    
    \exercice
    Construire l'image d'un point $M$ par rapport à une droite $(d)$, $M$ n'appartenant pas à la droite.
    \correction
    \begin{minipage}{0.35\linewidth}
        \begin{center}
            \psset{xunit=0.6cm,yunit=0.6cm,algebraic=true,dimen=middle,dotstyle=x,dotsize=3pt 0,linewidth=0.8pt,arrowsize=3pt 2,arrowinset=0.25}
            \begin{pspicture*}(0.7,-3.22)(5.58,1.36)
                \pspolygon[linecolor=mygreen,fillcolor=mygreen,fillstyle=solid,opacity=0.1](3.07,-0.85)(2.65,-0.93)(2.74,-1.34)(3.15,-1.26)
                \psline[linecolor=red](1.92,4.9)(3.92,-5.1)
                \psline(3.15,-1.26)(5.04,-0.88)
                \psline(4.3,-1.03)(4.23,-1.18)
                \psline(4.3,-1.03)(4.17,-0.92)
                \psline(4.1,-1.07)(4.02,-1.23)
                \psline(4.1,-1.07)(3.97,-0.96)
                \psline(1.26,-1.64)(3.15,-1.26)
                \psline(2.41,-1.41)(2.34,-1.56)
                \psline(2.41,-1.41)(2.28,-1.3)
                \psline(2.21,-1.45)(2.13,-1.6)
                \psline(2.21,-1.45)(2.08,-1.34)
                \rput[tl](1.88,0.68){axe (d)}
                \begin{scriptsize}
                    \psdots[linecolor=blue](1.92,4.9)
                    \rput[bl](2,5.02){\blue{$A$}}
                    \psdots[linecolor=blue](3.92,-5.1)
                    \rput[bl](4,-4.98){\blue{$B$}}
                    \psdots[linecolor=blue](1.26,-1.64)
                    \rput[bl](1.34,-1.52){\blue{$M$}}
                    \psdots[linecolor=blue](5.04,-0.88)
                    \rput[bl](5.12,-0.76){\blue{$M'$}}
                    \psdots[linecolor=black](3.15,-1.26)
                    \rput[bl](3.24,-1.14){\black{$I$}}
                \end{scriptsize}
            \end{pspicture*}
        \end{center}
    \end{minipage}
    \begin{minipage}{0.65\linewidth}
        \begin{enumerate}
            \item Tracer la perpendiculaire à l'axe (d) passant par M.
            \item Elle coupe (d) en $I$, placer le point $I$.
            \item Reporter la distance $PI$ de l'autre côté de l'axe (d).
            \item Placer le point $M'$, symétrique de $M$
        \end{enumerate}
    \end{minipage}
    \begin{myBox}{\infoComplementsNumeriques{pluriel}}
        \hrefConstruction{http://lozano.maths.free.fr/iep_local/figures_html/scr_iep_113.html}{Construction à l'aide de la règle et de l'équerre}
        
        \hrefConstruction{http://lozano.maths.free.fr/iep_local/figures_html/scr_iep_111.html}{Construction à l'aide du compas}
        \creditInstrumentPoche
    \end{myBox}
\end{methode*1}
