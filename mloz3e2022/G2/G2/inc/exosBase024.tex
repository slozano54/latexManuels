\begin{exercice*}
    Démontrer que les triangles $FGH$ et $FKL$ sont semblables.

    \begin{tikzpicture}[scale=0.3]            
        % \quadrilageMailleCarree[white]{18}{8}
        % Points
        \coordinate (F) at (4,4);
        \coordinate (G) at (0,7);
        \coordinate (H) at (6,7);                        
        \tkzDefPointBy[homothety=center F ratio -2.3](G); \tkzGetPoint{L};
        \tkzDefPointBy[homothety=center F ratio -2.3](H); \tkzGetPoint{K};
        % Tracés
        \tkzLabelPoints[left, xshift=-2mm](F);
        \tkzLabelPoints[above](G);            
        \tkzLabelPoints[above](H); 
        \tkzLabelPoints[below](K);            
        \tkzLabelPoints[below](L);
        \draw (F) -- (G) -- (H) -- cycle;
        \draw (F) -- (K) -- (L) -- cycle;        
        % Texte
        \coordinate (texte1) at (8,7);
        \coordinate (texte2) at (8,3);
        \tkzLabelPoint[right](texte1){\parbox{0.5\linewidth}{Les droites $(GL)$ et $(HK)$ sont sécantes en $F$.}}
        \tkzLabelPoint[right](texte2){\parbox{0.5\linewidth}{Les droites $(GH)$ et $(KL)$ sont parallèles.}}
    \end{tikzpicture} 
\end{exercice*}
\begin{corrige}
    %\setcounter{partie}{0} % Pour s'assurer que le compteur de \partie est à zéro dans les corrigés
    % \phantom{rrr}
    Démontrer que les triangles $FGH$ et $FKL$ sont semblables.

    \begin{tikzpicture}[scale=0.4]            
        % \quadrilageMailleCarree[white]{18}{8}
        % Points
        \coordinate (F) at (4,4);
        \coordinate (G) at (0,7);
        \coordinate (H) at (6,7);                        
        \tkzDefPointBy[homothety=center F ratio -2.3](G); \tkzGetPoint{L};
        \tkzDefPointBy[homothety=center F ratio -2.3](H); \tkzGetPoint{K};
        % Tracés
        \tkzLabelPoints[left, xshift=-2mm](F);
        \tkzLabelPoints[above](G);            
        \tkzLabelPoints[above](H); 
        \tkzLabelPoints[below](K);            
        \tkzLabelPoints[below](L);
        \draw (F) -- (G) -- (H) -- cycle;
        \draw (F) -- (K) -- (L) -- cycle;        
        % Texte
        \coordinate (texte1) at (7,6);
        \coordinate (texte2) at (7,3);
        \tkzLabelPoint[right](texte1){\parbox{0.5\linewidth}{Les droites $(GL)$ et $(HK)$ sont sécantes en $F$.}}
        \tkzLabelPoint[right](texte2){\parbox{0.5\linewidth}{Les droites $(GH)$ et $(KL)$ sont parallèles.}}
    \end{tikzpicture} 

    {\color{red} On peut appliquer le théorème de Thalès donc
    
    $\dfrac{FK}{FH}=\dfrac{FL}{FG}=\dfrac{KL}{HG}$.
    
    Les triangles sont donc semblables.

    \bigskip
    {\bfseries Une autre possibilité} est d'utiliser les angles alternes/internes et opposés par le sommet.}
\end{corrige}

