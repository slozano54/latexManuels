\begin{exercice*}
    Les deux quadrilatères ci-dessous sont homothétiques.
    \begin{enumerate}
        \item Coder les angles de même mesure sur la figure.
        \item On suppose que $AB=AC$, coder deux autres longueurs égales.
        \item Repasser en rouge deux segments parallèles.
    \end{enumerate}
    \begin{tikzpicture}[scale=0.5]            
        % \quadrilageMailleCarree{10}{6}
        % Points
        \coordinate (A) at (2,5);
        \coordinate (B) at (1,1);
        \coordinate (D) at (4,0.5);            
        \coordinate (C) at (6,3);
        \coordinate (O) at (35,1);
        \tkzDefPointBy[homothety=center O ratio 0.8](A); \tkzGetPoint{A'};
        \tkzDefPointBy[homothety=center O ratio 0.8](B); \tkzGetPoint{B'};
        \tkzDefPointBy[homothety=center O ratio 0.8](C); \tkzGetPoint{C'};
        \tkzDefPointBy[homothety=center O ratio 0.8](D); \tkzGetPoint{D'};
        % Tracés
        \tkzLabelPoints[above](A);
        \tkzLabelPoints[right](C);
        \tkzLabelPoints[left](B);
        \draw (A) -- (B) -- (D) -- (C) -- cycle;
        \draw (A') -- (B') -- (D') -- (C') -- cycle;
    \end{tikzpicture}
\end{exercice*}
\begin{corrige}
    %\setcounter{partie}{0} % Pour s'assurer que le compteur de \partie est à zéro dans les corrigés
    % \phantom{rrr}
    Les deux quadrilatères ci-dessous sont homothétiques.
    \begin{enumerate}
        \item Coder les angles de même mesure sur la figure.
        \item On suppose que $AB=AC$, coder deux autres longueurs égales.
        \item Repasser en rouge deux segments parallèles.
    \end{enumerate}
    \begin{tikzpicture}[scale=0.5]            
        % \quadrilageMailleCarree{10}{6}
        % Points
        \coordinate (A) at (2,5);
        \coordinate (B) at (1,1);
        \coordinate (D) at (4,0.5);            
        \coordinate (C) at (6,3);
        \coordinate (O) at (35,1);
        \tkzDefPointBy[homothety=center O ratio 0.8](A); \tkzGetPoint{A'};
        \tkzDefPointBy[homothety=center O ratio 0.8](B); \tkzGetPoint{B'};
        \tkzDefPointBy[homothety=center O ratio 0.8](C); \tkzGetPoint{C'};
        \tkzDefPointBy[homothety=center O ratio 0.8](D); \tkzGetPoint{D'};
        % Tracés
        \tkzLabelPoints[above](A);
        \tkzLabelPoints[right](C);
        \tkzLabelPoints[left](B);
        \draw (A) -- (B) -- (D) -- (C) -- cycle;
        \draw (A') -- (B') -- (D') -- (C') -- cycle;
        % Correction
        \tkzMarkAngle[size=1,color=red,mark=|,mkcolor=red,mksize=2pt](B,A,C);
        \tkzMarkAngle[size=1,color=red,mark=||,mkcolor=red,mksize=2pt](A,C,D);
        \tkzMarkAngle[size=1,color=red,mark=x,mkcolor=red,mksize=2pt](C,D,B);
        \tkzMarkAngle[size=1,color=red,mark=o,mkcolor=red,mksize=2pt](D,B,A);
        \tkzMarkSegments[mark=oo,color=red](B,A A,C);
        \tkzDrawSegment[thick,color=red](A,C);

        \tkzMarkAngle[size=1,color=red,mark=|,mkcolor=red,mksize=2pt](B',A',C');
        \tkzMarkAngle[size=1,color=red,mark=||,mkcolor=red,mksize=2pt](A',C',D');
        \tkzMarkAngle[size=1,color=red,mark=x,mkcolor=red,mksize=2pt](C',D',B');
        \tkzMarkAngle[size=1,color=red,mark=o,mkcolor=red,mksize=2pt](D',B',A');
        \tkzMarkSegments[mark=|||,color=red](B',A' A',C');
        \tkzDrawSegment[thick,color=red](A',C');
    \end{tikzpicture}
\end{corrige}

