\begin{exercice*}
    Une homothétie de centre $I$ et de rapport $-2$ transforme un segment $[AB]$ en un segment $[A'B']$.    
    \begin{enumerate}
        \item Faire une figure.        
        \item Justifier la position relative de $(AB)$ et $(A'B')$.
    \end{enumerate}
\end{exercice*}
\begin{corrige}
    %\setcounter{partie}{0} % Pour s'assurer que le compteur de \partie est à zéro dans les corrigés
    % \phantom{rrr}
    Une homothétie de centre $I$ et de rapport $-2$ transforme un segment $[AB]$ en un segment $[A'B']$.    

    \begin{enumerate}
        \item Faire une figure.
        
        % Correction
        \begin{tikzpicture}[scale=0.5]            
            % \quadrilageMailleCarree{10}{6}
            % Points
            \coordinate (I) at (4,4);
            \coordinate (A) at (1,5);
            \coordinate (B) at (5,6);            
            \tkzDefPointBy[homothety=center I ratio -2](A); \tkzGetPoint{A'};
            \tkzDefPointBy[homothety=center I ratio -2](B); \tkzGetPoint{B'};
            % Tracés
            \tkzLabelPoints[color=red](A,B,I,A',B');
            \draw[color=red] (A) -- (B);
            \draw[color=red] (A') -- (B');
            \draw[dashed,color=red] (A) -- (A');
            \draw[dashed,color=red] (B) -- (B');            
        \end{tikzpicture}
        \item Justifier la position relative de $(AB)$ et $(A'B')$.
        
        % Correction
        {\color{red} L'image d'une droite par une homothétie est une droite parallèle, or $(A'B')$ est l'image de $(AB)$ par l'homothétie de ccentre $I$ et de rapport $-2$
        donc $(A'B')$ et $(AB)$ sont parallèles.}
    \end{enumerate}
\end{corrige}

