\subsection{Propriétés}

% \proprNumBis{Conservations (admise)}{
% L'homothétie conserve les \textbf{proportions}, les \textbf{mesures d'angle} et l'\textbf{alignement des points}. 
% }

% \proprNumBis{Conséquence sur les longueurs et les aires (admise)}{
% Dans une homothétie de rapport \textbf{k positif} : 
% \begin{mylist}
% \item les \textbf{longueurs} sont multipliées par $k$
% \item les \textbf{aires} sont multipliées par $k^2$
% \CadreLampe{En effet}{
% Une aire est obtenue en multipliant deux longueurs, comme chaque longueur est multipliée par $k$ l'aire sera multipliée par $k\times k = k^2$.
% }
% \end{mylist}
% }

% \proprNumBis{(admise)}{
% La figure $F_2$ est l'image de la figure $F_1$ par cette homothétie :
% \begin{mylist}
% \item si $k$ est positif :\\
% \hspace*{1cm}$\gtrdot$ si $k>1$ alors $F_2$ est un \textbf{agrandissement} de $F_1$\\
% \hspace*{1cm}$\gtrdot$ si $0<k<1$ alors $F_2$ est une \textbf{réduction} de $F_1$
% \item si $k$ est négatif :\\
% \hspace*{1cm}$\gtrdot$ si $k<-1$ alors $F_2$ est un \textbf{agrandissement} de $F_1$\\
% \hspace*{1cm}$\gtrdot$ si $-1<k<0$ alors $F_2$ est une \textbf{réduction} de $F_1$
% \end{mylist}
% }

% \begin{center}
% %\includegraphics[scale=0.6]{../figures/courshomothetie4.ps}
% \begin{tikzpicture}[line width=0.8pt,line cap=round,line join=round,>=triangle 45,x=1cm,y=1cm,scale=0.45]
% \clip(-10,-15) rectangle (27,-3);
% \fill[color=red,fill=red,fill opacity=0.1] (0,-8) -- (0,-5) -- (1,-5) -- (2,-6) -- (3,-5) -- (4,-5) -- (4,-8) -- (3,-8) -- (3,-6) -- (2,-7) -- (1,-6) -- (1,-8) -- cycle;
% \fill[color=red,fill=red,fill opacity=0.1] (12,-14) -- (12,-5) -- (15,-5) -- (18,-8) -- (21,-5) -- (24,-5) -- (24,-14) -- (21,-14) -- (21,-8) -- (18,-11) -- (15,-8) -- (15,-14) -- cycle;
% \draw [shift={(1,-5)},color=mygreen,fill=mygreen,fill opacity=0.1] (0,0) -- (180:0.5) arc (180:315:0.5) -- cycle;
% \draw [shift={(15,-5)},color=mygreen,fill=mygreen,fill opacity=0.1] (0,0) -- (180:0.9442552027124941) arc (180:315:0.9442552027124941) -- cycle;
% \draw [color=red] (0,-8)-- (0,-5);
% \draw [color=red] (0,-5)-- (1,-5);
% \draw [color=red] (1,-5)-- (2,-6);
% \draw [color=red] (2,-6)-- (3,-5);
% \draw [color=red] (3,-5)-- (4,-5);
% \draw [color=red] (4,-5)-- (4,-8);
% \draw [color=red] (4,-8)-- (3,-8);
% \draw [color=red] (3,-8)-- (3,-6);
% \draw [color=red] (3,-6)-- (2,-7);
% \draw [color=red] (2,-7)-- (1,-6);
% \draw [color=red] (1,-6)-- (1,-8);
% \draw [color=red] (1,-8)-- (0,-8);
% %\draw[line width=2.4pt] (3.248441502037381,-1.327775143530903) -- (9.543476186787341,-1.327775143530903);
% \draw [color=red] (12,-14)-- (12,-5);
% \draw [color=red] (12,-5)-- (15,-5);
% \draw [color=red] (15,-5)-- (18,-8);
% \draw [color=red] (18,-8)-- (21,-5);
% \draw [color=red] (21,-5)-- (24,-5);
% \draw [color=red] (24,-5)-- (24,-14);
% \draw [color=red] (24,-14)-- (21,-14);
% \draw [color=red] (21,-14)-- (21,-8);
% \draw [color=red] (21,-8)-- (18,-11);
% \draw [color=red] (18,-11)-- (15,-8);
% \draw [color=red] (15,-8)-- (15,-14);
% \draw [color=red] (15,-14)-- (12,-14);
% \draw [dotted,domain=-12.740946597227518:27.547275385172227] plot(\x,{(-30-0*\x)/6});
% \draw [dotted,domain=-12.740946597227518:27.547275385172227] plot(\x,{(-48-3*\x)/6});
% \draw [dotted,domain=-12.740946597227518:27.547275385172227] plot(\x,{(-59-1.5*\x)/10});
% \draw  (4,-5)-- (4,-6.5);
% \draw  (4.188851040542487,-5.592624132881251) -- (3.81114895945749,-5.592624132881251);
% \draw  (4.188851040542487,-5.75) -- (3.81114895945749,-5.75);
% \draw  (4.188851040542487,-5.907375867118749) -- (3.81114895945749,-5.907375867118749);
% \draw  (4,-6.5)-- (4,-8);
% \draw  (4.188851040542487,-7.09262413288125) -- (3.81114895945749,-7.09262413288125);
% \draw  (4.188851040542487,-7.25) -- (3.81114895945749,-7.25);
% \draw  (4.188851040542487,-7.407375867118748) -- (3.81114895945749,-7.407375867118748);
% \draw  (24,-5)-- (24,-9.5);
% \draw  (24,-7.722127601356249) -- (23.71672343918622,-7.486063800678124);
% \draw  (24,-7.722127601356249) -- (24.283276560813718,-7.486063800678124);
% \draw  (24,-7.25) -- (23.71672343918622,-7.013936199321877);
% \draw  (24,-7.25) -- (24.283276560813718,-7.013936199321877);
% \draw  (24,-9.5)-- (24,-14);
% \draw  (24,-12.222127601356245) -- (23.71672343918622,-11.986063800678123);
% \draw  (24,-12.222127601356245) -- (24.283276560813718,-11.986063800678123);
% \draw  (24,-11.75) -- (23.71672343918622,-11.513936199321874);
% \draw  (24,-11.75) -- (24.283276560813718,-11.513936199321874);
% \begin{scriptsize}
% \draw (-6,-5) circle (1pt);
% \draw (-5.910833964273811,-5.5611859690252565) node {$O$};
% \draw (0,-8) circle (1pt);
% \draw (-0.4971041353888451,-7.355270854178998) node {$A$};
% \draw (0,-5) circle (1pt);
% \draw (-0.4656289619650954,-4.144803164956514) node {$B$};
% \draw (4,-6.5) circle (1pt);
% \draw (4.55647051597374,-6) node {$M$};
% \draw (0,-5.75) circle (1pt);
% \draw (-0.6230048290838444,-5.520087145165085) node {$N$};
% \draw  (12,-14) circle (1pt);
% \draw (11.46346176563608,-13.17817793757272) node {$A'$};
% \draw  (12,-5) circle (1pt);
% \draw (12.376241794924823,-4.333654205499013) node {$B'$};
% \draw (12,-7.25) circle (1pt);
% \draw (11.43198659221233,-6.631341865432751) node {$N'$};
% \draw (24,-9.5) circle (1pt);
% \draw (24.682828693735,-8.897554351942741) node {$M'$};
% \draw (1,-5) circle (1pt);
% \draw (1.2655055763411436,-4.3966045523465125) node {$P$};
% \draw (2,-6) circle (1pt);
% \draw (1.9579593916636393,-5.52610802082151) node {$Q$};
% \draw (15,-5) circle (1pt);
% \draw (15.366383270181053,-4.3966045523465125) node {$P'$};
% \draw (18,-8) circle (1pt);
% \draw (18,-7.386746027602748) node {$Q'$};
% \draw (4,-5) circle (1pt);
% \draw (4.255647051597374,-4.3966045523465125) node {$R$};
% \draw (4,-8) circle (1pt);
% \draw (4.55647051597374,-7.586746027602748) node {$S$};
% \draw (24,-5) circle (1pt);
% \draw (24.3682828693735,-4.3966045523465125) node {$R'$};
% \draw (24,-14) circle (1pt);
% \draw (24.682828693735,-13.698504151538969) node {$S'$};
% \end{scriptsize}
% \end{tikzpicture}
% \end{center}

% \Exemples[Exemple]{}{
% \begin{mylist}
% \item $M$ est le milieu de $[RS]$ donc son image $M'$ est le milieu de l'image de $[RS]$, à savoir $[R'S']$.
% \item $\widehat{B'P'Q'}$ est l'image de $\widehat{BPQ}$ donc ils ont la même mesure.
% \item $A$, $N$ et $B$ sont alignés donc leurs images $A'$, $N'$ et $B'$ le sont aussi.
% \item Les longueurs sont multipliées par $3$ donc les aires sont multipiées par $3^2=9$
% \end{mylist}
% }

% \proprNum{(admise)}{le triangle $A’B’C’$ est l’image du triangle $ABC$ par une homothétie de rapport $k$}{les longueurs des côtés du triangle $A’B’C’$ sont proportionnelles aux longueurs des côtés du triangle $ABC$ :
% \begin{mylist}
% \item si $k$ est positif alors le rapport de proportionnalité vaut $k$.
% \item si $k$ est négatif alors le rapport de proportionnalité vaut $-k$.
% \end{mylist}
% }

% \CadreLampe{Remarque}{
% Deux triangles qui sont liés par une homothétie sont dits \textbf{homothétiques}.\\
% \begin{tikzpicture}[line cap=round,line join=round,>=triangle 45,x=1cm,y=1cm,scale=0.7]
% \clip(0,-5) rectangle (17,5);
% \fill[line width=2pt,color=mygreen,fill=mygreen,fill opacity=0.1] (8.7,1.72) -- (7.145,-1.05) -- (10.47875,-2.91375) -- cycle;
% \fill[line width=2pt,color=red,fill=red,fill opacity=0.2] (12.79,4.04) -- (10.4575,-0.115) -- (15.458125,-2.9106249999999996) -- cycle;
% \draw [line width=2pt,color=mygreen] (8.7,1.72)-- (7.145,-1.05);
% \draw [line width=2pt,color=mygreen] (7.145,-1.05)-- (10.47875,-2.91375);
% \draw [line width=2pt,color=mygreen] (10.47875,-2.91375)-- (8.7,1.72);
% %\draw[line width=2pt] (4.33625,4.515) -- (12.21125,4.515);
% \draw [line width=2pt,color=red] (12.79,4.04)-- (10.4575,-0.115);
% \draw [line width=2pt,color=red] (10.4575,-0.115)-- (15.458125,-2.9106249999999996);
% \draw [line width=2pt,color=red] (15.458125,-2.9106249999999996)-- (12.79,4.04);
% \draw [line width=0.8pt,dash pattern=on 1pt off 1pt,domain=0.52:29.299999999999994] plot(\x,{(-20.3174--1.87*\x)/6.625});
% \draw [line width=0.8pt,dash pattern=on 1pt off 1pt,domain=0.52:29.299999999999994] plot(\x,{(-26.2984--4.64*\x)/8.18});
% \draw [line width=0.8pt,dash pattern=on 1pt off 1pt,domain=0.52:29.299999999999994] plot(\x,{(-29.0828--0.00625*\x)/9.95875});
% \begin{scriptsize}
% \draw [fill=black] (8.7,1.72) circle (1.5pt);
% \draw (8.64125,2.2181249999999997) node {$A$};
% \draw [fill=black] (7.145,-1.05) circle (1.5pt);
% \draw (6.69875,-0.7743749999999999) node {$B$};
% \draw [fill=black] (10.47875,-2.91375) circle (1.5pt);
% \draw (10.63625,-3.031875) node {$C$};
% \draw [fill=black] (0.52,-2.92) circle (1.5pt);
% \draw (0.71375,-2.533125) node {$O$};
% \draw [fill=black] (12.79,4.04) circle (1.5pt);
% \draw (12.78875,4.606875) node {$A'$};
% \draw [fill=black] (10.4575,-0.115) circle (1.5pt);
% \draw (10.1375,0.301875) node {$B'$};
% \draw [fill=black] (15.458125,-2.9106249999999996) circle (1.5pt);
% \draw (15.72875,-2.533125) node {$C'$};
% \end{scriptsize}
% \end{tikzpicture}

% }