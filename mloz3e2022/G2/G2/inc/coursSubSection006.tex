\begin{propriete}[Conservations \admise]
    L'homothétie conserve les \textbf{proportions}, les \textbf{mesures d'angle} et l'\textbf{alignement des points}. 
\end{propriete}
\begin{propriete}[Conséquence sur les longueurs et les aires \admise]
    Dans une homothétie de rapport \textbf{k positif} : 
    \begin{itemize}
        \item les \textbf{longueurs} sont multipliées par $k$
        \item les \textbf{aires} sont multipliées par $k^2$
    \end{itemize}
\end{propriete}
\begin{preuve}
    En effet, une aire est obtenue en multipliant deux longueurs, comme chaque longueur est multipliée par $k$ l'aire sera multipliée par $k\times k = k^2$.
\end{preuve}    
\begin{propriete}[\admise]
    La figure $F_2$ est l'image de la figure $F_1$ par cette homothétie :
    \begin{itemize}
        \item si $k$ est positif :
        
        \hspace*{1cm}$\gtrdot$ si $k>1$ alors $F_2$ est un \textbf{agrandissement} de $F_1$

        \hspace*{1cm}$\gtrdot$ si $0<k<1$ alors $F_2$ est une \textbf{réduction} de $F_1$
        \item si $k$ est négatif :
        
        \hspace*{1cm}$\gtrdot$ si $k<-1$ alors $F_2$ est un \textbf{agrandissement} de $F_1$

        \hspace*{1cm}$\gtrdot$ si $-1<k<0$ alors $F_2$ est une \textbf{réduction} de $F_1$
    \end{itemize}
\end{propriete}
\begin{center}
    \begin{tikzpicture}[line width=0.8pt,scale=0.37]
        % Points
        \coordinate (O) at (-6,-5);
        \coordinate (A) at (0,-8);
        \coordinate (B) at (0,-5);
        \coordinate (P) at (1,-5);
        \coordinate (Q) at (2,-6);
        \coordinate (T) at (3,-5);
        \coordinate (R) at (4,-5);
        \coordinate (S) at (4,-8);
        \coordinate (U) at (3,-8);
        \coordinate (V) at (3,-6);
        \coordinate (W) at (2,-7);
        \coordinate (X) at (1,-6);
        \coordinate (Y) at (1,-8);
        \coordinate (M) at (4,-6.5);
        \coordinate (N) at (0,-5.75);
        % Images
        \tkzDefPointBy[homothety=center O ratio 3](A); \tkzGetPoint{A'};
        \tkzDefPointBy[homothety=center O ratio 3](N); \tkzGetPoint{N'};
        \tkzDefPointBy[homothety=center O ratio 3](B); \tkzGetPoint{B'};
        \tkzDefPointBy[homothety=center O ratio 3](P); \tkzGetPoint{P'};
        \tkzDefPointBy[homothety=center O ratio 3](Q); \tkzGetPoint{Q'};
        \tkzDefPointBy[homothety=center O ratio 3](T); \tkzGetPoint{T'};
        \tkzDefPointBy[homothety=center O ratio 3](R); \tkzGetPoint{R'};
        \tkzDefPointBy[homothety=center O ratio 3](M); \tkzGetPoint{M'};
        \tkzDefPointBy[homothety=center O ratio 3](S); \tkzGetPoint{S'};
        \tkzDefPointBy[homothety=center O ratio 3](U); \tkzGetPoint{U'};
        \tkzDefPointBy[homothety=center O ratio 3](V); \tkzGetPoint{V'};
        \tkzDefPointBy[homothety=center O ratio 3](W); \tkzGetPoint{W'};
        \tkzDefPointBy[homothety=center O ratio 3](X); \tkzGetPoint{X'};
        \tkzDefPointBy[homothety=center O ratio 3](Y); \tkzGetPoint{Y'};        
        % Tracés
        \tkzLabelPoints(O);
        \tkzLabelPoints[above left](B);
        \tkzLabelPoints[below left](A,N);
        \tkzLabelPoints[above](P,Q);
        \tkzLabelPoints[above right](R,M,S);
        \tkzLabelPoints[above left](B');
        \tkzLabelPoints[below left](A');
        \tkzLabelPoints[above left](N');
        \tkzLabelPoints[above](P',Q');
        \tkzLabelPoints[above right](R',M',S');
        \draw[color=red,fill=red,fill opacity=0.1] (A) -- (B) -- (P) -- (Q) -- (T) -- (R) -- (S) -- (U) -- (V) -- (W) -- (X) -- (Y) -- cycle;        
        \draw[color=red,fill=red,fill opacity=0.1] (A') -- (B') -- (P') -- (Q') -- (T') -- (R') -- (S') -- (U') -- (V') -- (W') -- (X') -- (Y') -- cycle;
        \tkzMarkAngle[size=0.6,color=mygreen,fill=mygreen,fill opacity=0.1](B,P,Q);        
        \tkzMarkAngle[size=0.6,color=mygreen,fill=mygreen,fill opacity=0.1](B',P',Q'); 
        \tkzDrawSegment[thick](R,S);
        \tkzDrawSegment[thick](R',S');       
        \tkzMarkSegments[mark=oo,size=3pt](R,M M,S);
        \tkzMarkSegments[mark=s,size=3pt](R',M' M',S');
        \tkzDrawPoints[shape=cross out](O,A,N,B,P,Q,R,M,S);
        \tkzDrawPoints[shape=cross out](A',N',B',P',Q',R',M',S');
        \tkzDrawLine[dashed,add=0.1 and 0.1](O,R');
        \tkzDrawLine[dashed,add=0.1 and 0.1](O,M');
        \tkzDrawLine[dashed,add=0.1 and 0.1](O,A');    
    \end{tikzpicture}
\end{center}
\begin{exemple*1}
    \titreExemple{Déductions par conservation}

    \begin{itemize}
        \item $M$ est le milieu de $[RS]$ donc son image $M'$ est le milieu de l'image de $[RS]$, à savoir $[R'S']$.
        \item $\widehat{B'P'Q'}$ est l'image de $\widehat{BPQ}$ donc ils ont la même mesure.
        \item $A$, $N$ et $B$ sont alignés donc leurs images $A'$, $N'$ et $B'$ le sont aussi.
        \item Les longueurs sont multipliées par $3$ donc les aires sont multipiées par $3^2=9$
    \end{itemize}
\end{exemple*1}
\begin{propriete}[\admise]
    Si le triangle $A’B’C’$ est l’image du triangle $ABC$ par une homothétie de rapport $k$ alors les longueurs des côtés du triangle $A’B’C’$ sont proportionnelles aux longueurs des côtés du triangle $ABC$ :
    \begin{itemize}
        \item si $k$ est positif alors le rapport de proportionnalité vaut $k$.
        \item si $k$ est négatif alors le rapport de proportionnalité vaut $-k$.
    \end{itemize}
\end{propriete}
\begin{remarque}
    Deux triangles qui sont liés par une homothétie sont dits \textbf{homothétiques}.
    \begin{center}
        \begin{tikzpicture}[scale=0.3]
            % Points
            \coordinate (O) at (0,-3);
            \coordinate (A) at (8,2);
            \coordinate (B) at (6,-1);
            \coordinate (C) at (10,-3);
            \tkzDefPointBy[homothety=center O ratio 1.8](A); \tkzGetPoint{A'};
            \tkzDefPointBy[homothety=center O ratio 1.8](B); \tkzGetPoint{B'};
            \tkzDefPointBy[homothety=center O ratio 1.8](C); \tkzGetPoint{C'};
            % Tracés
            \tkzLabelPoints[above](O);
            \tkzLabelPoints[above left](A,B);
            \tkzLabelPoints(C);
            \tkzLabelPoints[above left](A',B');
            \tkzLabelPoints(C');
            \tkzDrawPoints[shape=cross out](O,A,B,C,A',B',C');
            \draw[line width=2pt,color=mygreen,fill=mygreen,fill opacity=0.1] (A) -- (B) -- (C) -- cycle;
            \draw[line width=2pt,color=red,fill=red,fill opacity=0.1] (A') -- (B') -- (C') -- cycle;        
            \tkzDrawSegment[dashed,add=0 and 0.1](O,A');
            \tkzDrawSegment[dashed,add=0 and 0.1](O,B');
            \tkzDrawSegment[dashed,add=0 and 0.1](O,C');
        \end{tikzpicture}
    \end{center}
\end{remarque}