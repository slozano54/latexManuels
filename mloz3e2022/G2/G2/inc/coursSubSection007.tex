\subsection{Définition}
% \definNum{
% Deux triangles sont \textbf{semblables} si les longueurs de leurs côtés sont proportionnelles.
% }
% \Exemples[Exemple]{}{
% \begin{minipage}{9cm}
% \begin{tikzpicture}[line width=1pt,line cap=round,line join=round,>=triangle 45,x=1cm,y=1cm,scale=0.6]
% \draw [color=lightgray,, xstep=1cm,ystep=1cm] (-5,-6) grid (10,0);
% \clip(-5,-6) rectangle (10,0);
% \draw [color=blue] (-3,-5)-- (-4,-3);
% \draw [color=blue] (5,-5)-- (9,-3);
% \draw [color=red] (-4,-3)-- (-2,-1);
% \draw [color=red] (5,-5)-- (1,-1);
% \draw (1,-1)-- (9,-3);
% \draw (-2,-1)-- (-3,-5);
% \begin{scriptsize}
% \draw [fill=black] (-3,-5) circle (1pt);
% \draw (-2.96,-5.26) node {$T$};
% \draw [fill=black] (-4,-3) circle (1pt);
% \draw (-4.4,-2.74) node {$O$};
% \draw [fill=black] (5,-5) circle (1pt);
% \draw (4.96,-5.6) node {$W$};
% \draw [fill=black] (9,-3) circle (1pt);
% \draw (9.16,-2.7) node {$E$};
% \draw [fill=black] (-2,-1) circle (1pt);
% \draw (-1.84,-0.7) node {$P$};
% \draw [fill=black] (1,-1) circle (1pt);
% \draw (0.96,-0.5) node {$B$};
% \end{scriptsize}
% \end{tikzpicture}
% \end{minipage}
% \hfill
% \begin{minipage}{7cm}
% Sur la figure ci-contre, à l'aide du quadrillage, on peut constater que :
% \begin{mylist}
% \item $TO =2\times WE$
% \item $OP =2\times EB$
% \item $PT =2\times BW$
% \end{mylist}
% D'après la définition, puisque les longueurs de leurs côtés sont proportionnelles, \textbf{les triangles $\mathbf{TOP}$ et $\mathbf{WEB}$ sont donc semblables}.
% \end{minipage}
% }

% \Exemples[Exemple]{}{
% \begin{minipage}{7cm}
% Les triangles $MER$ et $GIF$ sont semblables.\\
% Le tableau suivant est donc un tableau de proportionnalité :
% \par\vspace{0.5cm}
% \begin{tabular}{|c|c|c|c|}
% \hline 
% Triangle $MER$ & $ME$ & $ER$ & $RM$ \\ 
% \hline 
% Triangle $GIF$ & $GI$ & $IF$ & $FG$ \\ 
% \hline 
% \end{tabular}
% Les trois rapports suivants sont donc égaux : $\dfrac{ME}{GI}=\dfrac{ER}{IF}=\dfrac{RM}{FG}$
% \end{minipage}
% \hfill
% \begin{minipage}{9cm}
% \begin{tikzpicture}[line width=1pt,line cap=round,line join=round,>=triangle 45,x=1cm,y=1cm,scale=0.5]
% \clip(-4,-4) rectangle (14,6);
% \draw [color=red] (0,4)-- (-2,2);
% \draw [color=red] (6,-3)-- (12,3);
% \draw [color=blue] (-2,2)-- (-2,-1);
% \draw [color=blue] (6,-3)-- (-3,-3);
% \draw (-3,-3)-- (12,3);
% \draw (-2,-1)-- (0,4);
% \begin{scriptsize}
% \draw [fill=black] (0,4) circle (1pt);
% \draw (0.64,4.16) node {$M$};
% \draw [fill=black] (-2,2) circle (1pt);
% \draw (-2.4,2.26) node {$E$};
% \draw [fill=black] (6,-3) circle (1pt);
% \draw (5.96,-3.46) node {$I$};
% \draw [fill=black] (12,3) circle (1pt);
% \draw (12.16,3.5) node {$G$};
% \draw [fill=black] (-2,-1) circle (1pt);
% \draw (-2,-1.46) node {$R$};
% \draw [fill=black] (-3,-3) circle (1pt);
% \draw (-3.1,-3.44) node {$F$};
% \end{scriptsize}
% \end{tikzpicture}
% \end{minipage}
% }

% \CadreLampe{Remarques}{
% \begin{mylist}
% \item Si deux triangles sont \textbf{homothétiques} alors ils sont \textbf{semblables}.
% \item La réciproque, Si deux triangles sont \textbf{semblables} alors ils sont  \textbf{homothétiques} est FAUSSE.
% \end{mylist}
% }
