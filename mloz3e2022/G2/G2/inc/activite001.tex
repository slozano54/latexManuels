\begin{activite}[Les homothéties]
    \begin{myBox}{Découverte avec Geogebra}

        \vspace*{3mm}
        \href{https://www.geogebra.org/classic/wydhjpuc}{\emoji{play-button} Suivre ce lien pour lancer l'animation Geogebra}

        \vspace*{3mm}
        \href{https://www.geogebra.org/m/vaxxugnh}{\emoji{red-question-mark} Compléter ce QCM de découverte Geogebra}

        \vspace*{3mm}
        \creditGeogebra{David Evéquoz}
    \end{myBox}

    {\large
    \begin{tikzpicture}
        \node (image) at (5,0) {\includegraphics[scale=0.4]{\currentpath/images/avionGeogebra.png}};
        \coordinate[label=left:{\bfseries figure 1}] (fig1) at (0,-2);
        \coordinate[label=right:{\bfseries figure 2}] (fig2) at (3,-2);
        \coordinate[label=above:$H(0;k)$] (transf) at (1.5,-2);            
        \draw[->,thick] (fig1)--(fig2);        
    \end{tikzpicture}
    }

    \og La \textbf{figure 2} est l'image de la \textbf{figure 1} par une homothétie de centre $O$ et de rapport $k$ \fg

    {\bfseries Dans cette \href{https://www.geogebra.org/classic/wydhjpuc}{appliquette Geogebra}, commencer par cocher la case \og Image \fg}
    \begin{remarque}
        Parfois, plusieurs réponses sont possibles.
    \end{remarque}
    \begin{enumerate}
        \item Sans rien modifier, qu'est-ce qu'une homothétie ?
        \begin{itemize}
            \item[$\square$] Un agrandissement ou une réduction de l'objet.
            \item[$\square$] Une transformation qui change les dimensions de l'objet.
            \item[$\square$] Un simple déplacement de l'objet.            
        \end{itemize}
        \item Modifier le rapport d'homothétie $\mathbf{k}$ et observer !
        \begin{itemize}
            \begin{multicols}{2}
                \item si $\mathbf{k > 1}$, alors \ldots
                \begin{itemize}
                    \item[$\square$] Rien ne change
                    \item[$\square$] L'image est agrandie
                    \item[$\square$] L'image est réduite
                    \item[$\square$] L'image est inversée
                \end{itemize}
                \item si $\mathbf{k = 1}$, alors \ldots
                \begin{itemize}
                    \item[$\square$] Rien ne change
                    \item[$\square$] L'image est agrandie
                    \item[$\square$] L'image est réduite
                    \item[$\square$] L'image est inversée
                \end{itemize}
                \item si $\mathbf{0< k < 1}$, alors \ldots
                \begin{itemize}
                    \item[$\square$] Rien ne change
                    \item[$\square$] L'image est agrandie
                    \item[$\square$] L'image est réduite
                    \item[$\square$] L'image est inversée
                \end{itemize}
                \item si $\mathbf{k < 0}$, alors \ldots
                \begin{itemize}
                    \item[$\square$] Rien ne change
                    \item[$\square$] L'image est agrandie
                    \item[$\square$] L'image est réduite
                    \item[$\square$] L'image est inversée
                \end{itemize}
            \end{multicols}
            \item L'image est inversée mais ses dimensions ne changent  pas lorsque \ldots
            \begin{itemize}
                \item[$\square$] $k = 0$
                \item[$\square$] $k = -1$
                \item[$\square$] $k = 1$
                \item[$\square$] $k = 2$
            \end{itemize}
        \end{itemize}    
    \end{enumerate}

    {\bfseries Décocher la case \og Image \fg et cocher la case \og Figure géométrique \fg}    
    \begin{enumerate}
        \setcounter{enumi}{2}
        \item Comment calculer $k$ à l'aide des longueurs des segments ?
        \begin{itemize}            
            \item[$\square$] En divisant la longueur d'un segment image par la longueur d'un segment origine ?
            \begin{multicols}{2}
                \item[$\square$] $\dfrac{OB'}{OB}$
                \item[]
                \item[$\square$] $\dfrac{OC}{OC'}$            
                \item[$\square$] $\dfrac{A'B'}{AB}$
                \item[]
                \item[$\square$] $C'D'-CD$
            \end{multicols}
        \end{itemize}
        \item Quel est le lien entre le rapport d'homothétie $\mathbf{k}$ et les longueurs ?
        
        Modifier $\mathbf{k}$ pour vérifier cette conjecture.
        \begin{itemize}
            \item[$\square$] $A'B' = k\times AB$
            \item[$\square$] $AB = k\times A'B'$
            \item[$\square$] $k\times OB = O'B'$
        \end{itemize}
        \item Est-ce qu'une homothétie modifie la mesure des angles ?
        
        Modifier $\mathbf{k}$ pour vérifier cette conjecture.
        \begin{itemize}
            \item[$\square$] Oui
            \item[$\square$] Non            
        \end{itemize}
        \item Est-ce qu'une homothétie est une isométrie ? Pourquoi ?
        \begin{itemize}
            \item[$\square$] Oui, car les longueurs sont conservées
            \item[$\square$] Non,  car les longueurs ne sont pas conservées
        \end{itemize}
    \end{enumerate}
\end{activite}