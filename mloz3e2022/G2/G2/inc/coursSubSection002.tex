\subsection{Symétrie centrale}
\begin{definition}
    $M$ et $M’$ sont symétriques \textbf{par rapport au point $\mathbf{O}$} signifie que :
    \begin{itemize}
        \item $M$, $O$ et $M’$ sont alignés,
        \item $MO = OM’$.
    \end{itemize}
\end{definition}

\begin{center}    
    \begin{tikzpicture}[x=0.8cm,y=0.8cm]
        \draw [color=gray!50,, xstep=0.6cm,ystep=0.6cm] (-8,-1) grid (6,5);        
        \draw[line width=1pt, color=red,fill=red,fill opacity=0.4] (-5,4) -- (-3,4) -- (-4,3) -- (-3,2) -- (-3,0) -- (-4,1) -- (-5,0) -- (-7,0) -- (-5,2) -- cycle;
        \draw[line width=1pt,color=mygreen,fill=mygreen,fill opacity=0.4] (3,0) -- (1,0) -- (2,1) -- (1,2) -- (1,4) -- (2,3) -- (3,4) -- (5,4) -- (3,2) -- cycle;
        \begin{scriptsize}
            \draw [color=blue] (-5,4)-- ++(-2.5pt,-2.5pt) -- ++(5pt,5pt) ++(-5pt,0) -- ++(5pt,-5pt);
            \draw[color=blue] (-4.84,4.42) node {$A$};
            \draw [color=blue] (-3,4)-- ++(-2.5pt,-2.5pt) -- ++(5pt,5pt) ++(-5pt,0) -- ++(5pt,-5pt);
            \draw[color=blue] (-2.84,4.42) node {$B$};
            \draw [color=blue] (-4,3)-- ++(-2.5pt,-2.5pt) -- ++(5pt,5pt) ++(-5pt,0) -- ++(5pt,-5pt);
            \draw[color=blue] (-3.58,3.16) node {$C$};
            \draw [color=blue] (-3,2)-- ++(-2.5pt,-2.5pt) -- ++(5pt,5pt) ++(-5pt,0) -- ++(5pt,-5pt);
            \draw[color=blue] (-2.84,2.42) node {$D$};
            \draw [color=blue] (-3,0)-- ++(-2.5pt,-2.5pt) -- ++(5pt,5pt) ++(-5pt,0) -- ++(5pt,-5pt);
            \draw[color=blue] (-2.84,0.42) node {$E$};
            \draw [color=blue] (-4,1)-- ++(-2.5pt,-2.5pt) -- ++(5pt,5pt) ++(-5pt,0) -- ++(5pt,-5pt);
            \draw[color=blue] (-3.84,0.42) node {$F$};
            \draw [color=blue] (-5,0)-- ++(-2.5pt,-2.5pt) -- ++(5pt,5pt) ++(-5pt,0) -- ++(5pt,-5pt);
            \draw[color=blue] (-4.9,-0.4) node {$G$};
            \draw [color=blue] (-7,0)-- ++(-2.5pt,-2.5pt) -- ++(5pt,5pt) ++(-5pt,0) -- ++(5pt,-5pt);
            \draw[color=blue] (-7.03,-0.36) node {$H$};
            \draw [color=blue] (-5,2)-- ++(-2.5pt,-2.5pt) -- ++(5pt,5pt) ++(-5pt,0) -- ++(5pt,-5pt);
            \draw[color=blue] (-5.3,2.36) node {$I$};
            \draw [color=blue] (-1,2)-- ++(-2.5pt,-2.5pt) -- ++(5pt,5pt) ++(-5pt,0) -- ++(5pt,-5pt);
            \draw[color=blue] (-0.84,2.42) node {$O$};
            \draw [color=blue] (3,0)-- ++(-2.5pt,-2.5pt) -- ++(5pt,5pt) ++(-5pt,0) -- ++(5pt,-5pt);
            \draw[color=blue] (3.24,0.42) node {$A'$};
            \draw [color=blue] (1,0)-- ++(-2.5pt,-2.5pt) -- ++(5pt,5pt) ++(-5pt,0) -- ++(5pt,-5pt);
            \draw[color=blue] (0.62,0.36) node {$B'$};
            \draw [color=blue] (2,1)-- ++(-2.5pt,-2.5pt) -- ++(5pt,5pt) ++(-5pt,0) -- ++(5pt,-5pt);
            \draw[color=blue] (1.42,1.18) node {$C'$};
            \draw [color=blue] (1,2)-- ++(-2.5pt,-2.5pt) -- ++(5pt,5pt) ++(-5pt,0) -- ++(5pt,-5pt);
            \draw[color=blue] (0.58,2.18) node {$D'$};
            \draw [color=blue] (1,4)-- ++(-2.5pt,-2.5pt) -- ++(5pt,5pt) ++(-5pt,0) -- ++(5pt,-5pt);
            \draw[color=blue] (1.24,4.42) node {$E'$};
            \draw [color=blue] (2,3)-- ++(-2.5pt,-2.5pt) -- ++(5pt,5pt) ++(-5pt,0) -- ++(5pt,-5pt);
            \draw[color=blue] (2.14,3.66) node {$F'$};
            \draw [color=blue] (3,4)-- ++(-2.5pt,-2.5pt) -- ++(5pt,5pt) ++(-5pt,0) -- ++(5pt,-5pt);
            \draw[color=blue] (3.24,4.42) node {$G'$};
            \draw [color=blue] (5,4)-- ++(-2.5pt,-2.5pt) -- ++(5pt,5pt) ++(-5pt,0) -- ++(5pt,-5pt);
            \draw[color=blue] (5.24,4.42) node {$H'$};
            \draw [color=blue] (3,2)-- ++(-2.5pt,-2.5pt) -- ++(5pt,5pt) ++(-5pt,0) -- ++(5pt,-5pt);
            \draw[color=blue] (3.52,2.02) node {$I'$};
        \end{scriptsize}
    \end{tikzpicture}
\end{center}

\begin{remarque}
    Deux figures symétriques par symétrie centrale se superposent par un demi-tour autour du centre de symétrie.
    \begin{myBox}{\infoComplementsNumeriques{singulier}}
        \hrefConstruction{https://www.geogebra.org/classic/fbfhreb8}{Animation Geogebra}
        \creditGeogebra{Roumazeille Boris}
    \end{myBox}
\end{remarque}

\begin{methode*1}[Construction de l'image d'un point par symétrie centrale]    
    \exercice
    Construire l'image d'un point $M$ par rapport à un point $O$,
    
    $M$ étant distinct de $O$.
    \correction
    \begin{minipage}{0.35\linewidth}
        \begin{center}
            \begin{tikzpicture}[scale=0.8]
                \coordinate [label=left:$M$] (M) at (-1.5,0.5);
                \coordinate [label=above:$O$] (O) at (0,0);
                \coordinate [label=right:$M'$] (M') at (1.5,-0.5);
                \tkzDrawPoints[shape=cross out](M,O,M');
                \tkzMarkSegments[mark=||,size=3pt](M,O O,M');
                \tkzDrawSegment(M,M');
            \end{tikzpicture}
        \end{center}
    \end{minipage}
    \begin{minipage}{0.65\linewidth}
        \textbf{Programme de construction}
        \begin{enumerate}
            \item Tracer la demi-droite $[MO)$.
            \item Avec le compas, placer M' sur la demi-droite $[MO)$
            
            tel que $M'O=OM$.
        \end{enumerate}
    \end{minipage}
    \begin{myBox}{\infoComplementsNumeriques{singulier}}
        \hrefConstruction{http://lozano.maths.free.fr/iep_local/figures_html/scr_iep_116.html}{Construction à l'aide de la règle et du compas}
        \creditInstrumentPoche
    \end{myBox}
\end{methode*1}

