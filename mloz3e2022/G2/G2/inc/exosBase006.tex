\begin{exercice*}[Message codé]
    On considère {\color{red} l'homothétie de centre $A'$ et de rapport $2$}
    et {\color{mygreen} celle de centre $A'$ et de rapport $3$}.

    \begin{tikzpicture}[scale=0.36]            
        % \draw[help lines, color=black!30] (0,0) grid (22,16);            
        % Centre de l'homothétie
        \coordinate (A') at (0,0);
        % Points figure origine        
        \coordinate (P) at (3,1);
        \coordinate (Q) at (5,3);
        \coordinate (C) at (5,5);
        \coordinate (D) at (7,5);
        \coordinate (F) at (6,4);
        \coordinate (G) at (7,3);
        \coordinate (H) at (7,1);
        % \coordinate (J) at (6,2);
        \coordinate (K) at (5,1);
        % Points figure rouge
        \tkzDefPointBy[homothety=center A' ratio 2](P); \tkzGetPoint{J};
        \tkzDefPointBy[homothety=center A' ratio 2](Q); \tkzGetPoint{R};
        \tkzDefPointBy[homothety=center A' ratio 2](C); \tkzGetPoint{A};
        \tkzDefPointBy[homothety=center A' ratio 2](D); \tkzGetPoint{B};
        \tkzDefPointBy[homothety=center A' ratio 2](F); \tkzGetPoint{U};
        \tkzDefPointBy[homothety=center A' ratio 2](G); \tkzGetPoint{I};
        \tkzDefPointBy[homothety=center A' ratio 2](H); \tkzGetPoint{L};
        \tkzDefPointBy[homothety=center A' ratio 2](J); \tkzGetPoint{M};
        \tkzDefPointBy[homothety=center A' ratio 2](K); \tkzGetPoint{N};        
        % Points figure verte
        \tkzDefPointBy[homothety=center A' ratio 3](P); \tkzGetPoint{Z};
        \tkzDefPointBy[homothety=center A' ratio 3](Q); \tkzGetPoint{O};
        \tkzDefPointBy[homothety=center A' ratio 3](C); \tkzGetPoint{S};
        \tkzDefPointBy[homothety=center A' ratio 3](D); \tkzGetPoint{T};
        \tkzDefPointBy[homothety=center A' ratio 3](F); \tkzGetPoint{E};
        \tkzDefPointBy[homothety=center A' ratio 3](G); \tkzGetPoint{V};
        \tkzDefPointBy[homothety=center A' ratio 3](H); \tkzGetPoint{W};
        \tkzDefPointBy[homothety=center A' ratio 3](J); \tkzGetPoint{X};
        \tkzDefPointBy[homothety=center A' ratio 3](K); \tkzGetPoint{Y};        
        % Marques figure origine
        \tkzDrawPoints[shape=cross out, size=3pt](A',P,Q,C,D,F,G,H,J,K);
        \tkzLabelPoints[above](A',P,C,D,F,G,J);
        \tkzLabelPoints[above left](Q,K);
        \tkzLabelPoints[above right](H);
        % Tracés figure origine
        \draw[color=gray,fill=gray, fill opacity=0.2] (P)--(Q)--(C)--(D)--(F)--(G)--(H)--(J)--(K)--cycle;
        % Marques figure rouge
        \tkzDrawPoints[shape=cross out, size=3pt](J,R,A,B,U,I,L,M,N);
        \tkzLabelPoints[above](A,B,U,I,M);
        \tkzLabelPoints[above left](R);
        %\tkzLabelPoints[above right](L);
        \tkzLabelPoints[below](N,L);
        % Tracés figure rouge
        \draw[color=red,fill=red, fill opacity=0.2] (J)--(R)--(A)--(B)--(U)--(I)--(L)--(M)--(N)--cycle;
        % Marques figure verte
        \tkzDrawPoints[shape=cross out, size=3pt](Z,O,S,T,E,V,W,X,Y);
        \tkzLabelPoints[above](Z,S,T,E,V,X);
        \tkzLabelPoints[above left](O,Y);
        \tkzLabelPoints[below](W);
        % Tracés figure verte
        \draw[color=mygreen,fill=mygreen, fill opacity=0.2] (Z)--(O)--(S)--(T)--(E)--(V)--(W)--(X)--(Y)--cycle;        
    \end{tikzpicture}

    Pour décoder le message ci-dessous, remplacer chaque point par son image obtenue
    par l'homothétie conrrespondant à la couleur de la lettre.

    {\color{red}PCJCG}{\color{mygreen}C} \hfill {\color{red}H}{\color{mygreen}F} \hfill {\color{mygreen}CQ}{\color{red}H}{\color{mygreen}F}{\color{red}GH} \hfill 
    {\color{red}K}{\color{mygreen}F} \hfill {\color{mygreen}GQ}{\color{red}G}{\color{mygreen}D} \hfill {\color{red}H}'{\color{mygreen}Q}{\color{red}JDQ}{\color{mygreen}F}.      
\end{exercice*}
\begin{corrige}
    %\setcounter{partie}{0} % Pour s'assurer que le compteur de \partie est à zéro dans les corrigés
    % \phantom{rrr}
    On considère {\color{red} l'homothétie de centre $A'$ et de rapport $2$}
    et {\color{mygreen} celle de centre $A'$ et de rapport $3$}.

    \hspace*{-10mm}
    \begin{tikzpicture}[scale=0.36]            
        % \draw[help lines, color=black!30] (0,0) grid (22,16);            
        % Centre de l'homothétie
        \coordinate (A') at (0,0);
        % Points figure origine        
        \coordinate (P) at (3,1);
        \coordinate (Q) at (5,3);
        \coordinate (C) at (5,5);
        \coordinate (D) at (7,5);
        \coordinate (F) at (6,4);
        \coordinate (G) at (7,3);
        \coordinate (H) at (7,1);
        % \coordinate (J) at (6,2);
        \coordinate (K) at (5,1);
        % Points figure rouge
        \tkzDefPointBy[homothety=center A' ratio 2](P); \tkzGetPoint{J};
        \tkzDefPointBy[homothety=center A' ratio 2](Q); \tkzGetPoint{R};
        \tkzDefPointBy[homothety=center A' ratio 2](C); \tkzGetPoint{A};
        \tkzDefPointBy[homothety=center A' ratio 2](D); \tkzGetPoint{B};
        \tkzDefPointBy[homothety=center A' ratio 2](F); \tkzGetPoint{U};
        \tkzDefPointBy[homothety=center A' ratio 2](G); \tkzGetPoint{I};
        \tkzDefPointBy[homothety=center A' ratio 2](H); \tkzGetPoint{L};
        \tkzDefPointBy[homothety=center A' ratio 2](J); \tkzGetPoint{M};
        \tkzDefPointBy[homothety=center A' ratio 2](K); \tkzGetPoint{N};        
        % Points figure verte
        \tkzDefPointBy[homothety=center A' ratio 3](P); \tkzGetPoint{Z};
        \tkzDefPointBy[homothety=center A' ratio 3](Q); \tkzGetPoint{O};
        \tkzDefPointBy[homothety=center A' ratio 3](C); \tkzGetPoint{S};
        \tkzDefPointBy[homothety=center A' ratio 3](D); \tkzGetPoint{T};
        \tkzDefPointBy[homothety=center A' ratio 3](F); \tkzGetPoint{E};
        \tkzDefPointBy[homothety=center A' ratio 3](G); \tkzGetPoint{V};
        \tkzDefPointBy[homothety=center A' ratio 3](H); \tkzGetPoint{W};
        \tkzDefPointBy[homothety=center A' ratio 3](J); \tkzGetPoint{X};
        \tkzDefPointBy[homothety=center A' ratio 3](K); \tkzGetPoint{Y};        
        % Marques figure origine
        \tkzDrawPoints[shape=cross out, size=3pt](A',P,Q,C,D,F,G,H,J,K);
        \tkzLabelPoints[above](A',P,C,D,F,G,J);
        \tkzLabelPoints[above left](Q,K);
        \tkzLabelPoints[above right](H);
        % Tracés figure origine
        \draw[color=gray,fill=gray, fill opacity=0.2] (P)--(Q)--(C)--(D)--(F)--(G)--(H)--(J)--(K)--cycle;
        % Marques figure rouge
        \tkzDrawPoints[shape=cross out, size=3pt](J,R,A,B,U,I,L,M,N);
        \tkzLabelPoints[above](A,B,U,I,M);
        \tkzLabelPoints[above left](R);
        %\tkzLabelPoints[above right](L);
        \tkzLabelPoints[below](N,L);
        % Tracés figure rouge
        \draw[color=red,fill=red, fill opacity=0.2] (J)--(R)--(A)--(B)--(U)--(I)--(L)--(M)--(N)--cycle;
        % Marques figure verte
        \tkzDrawPoints[shape=cross out, size=3pt](Z,O,S,T,E,V,W,X,Y);
        \tkzLabelPoints[above](Z,S,T,E,V,X);
        \tkzLabelPoints[above left](O,Y);
        \tkzLabelPoints[below](W);
        % Tracés figure verte
        \draw[color=mygreen,fill=mygreen, fill opacity=0.2] (Z)--(O)--(S)--(T)--(E)--(V)--(W)--(X)--(Y)--cycle;        
    \end{tikzpicture}

    Pour décoder le message ci-dessous, remplacer chaque point par son image obtenue
    par l'homothétie conrrespondant à la couleur de la lettre.

    {\color{red}PCJCG}{\color{mygreen}C} \hfill {\color{red}H}{\color{mygreen}F} \hfill {\color{mygreen}CQ}{\color{red}H}{\color{mygreen}F}{\color{red}GH} \hfill 
    {\color{red}K}{\color{mygreen}F} \hfill {\color{mygreen}GQ}{\color{red}G}{\color{mygreen}D} \hfill {\color{red}H}'{\color{mygreen}Q}{\color{red}JDQ}{\color{mygreen}F}.

    {\color{red}JAMAIS \hfill LE \hfill SOLEIL \hfill NE \hfill VOIT \hfill L'OMBRE.}

\end{corrige}

