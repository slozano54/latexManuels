\subsection{Rotation}
\begin{definition}
    $M’$ est l’image de $M$ par la rotation de \textbf{centre $\mathbf{O}$} et d’\textbf{angle $\mathbf{\alpha}$} dans le sens inverse des aiguilles d’une montre signifie que : 
    \begin{itemize}
        \item $\widehat{MOM'}=\mathbf{\alpha}$ de M vers M’ dans le sens inverse des aiguilles d’une montre,
        \item $MO = OM’$
    \end{itemize}
\end{definition}

\begin{center}
    \begin{minipage}{0.45\linewidth}
        \includegraphics[scale=0.4]{\currentpath/images/coursRotationTux.png}
    \end{minipage}
    \begin{minipage}{0.45\linewidth}        
        \begin{tikzpicture}[scale=1,>=triangle 45]
            % Points
            \coordinate (A) at (2,3);
            \coordinate (B) at (4.5,3);
            \tkzDefMidPoint(A,B);
            \tkzGetPoint{I};
            \coordinate (C) at (4,4);
            \tkzDefMidPoint(A,C);
            \tkzGetPoint{K};
            \coordinate (O) at (2,0);
            \tkzDrawPoints[shape=cross out](O);
            \tkzLabelPoints(O);
            % Tracés
            \tkzDrawSegment[style=red, dim={$~r~$,-5pt,midway,font=\scriptsize,sloped}](I,B);
            \draw (A)--(B)--(C)--(A);
            \draw (K)--(I);
            \tkzDrawPoints[shape=cross out](I);        
            \tkzLabelPoints[below right](B);
            \tkzLabelPoints[below left](A,I);
            \tkzLabelPoints[above](C,K);
            \tkzMarkRightAngles[size=0.1](A,K,I A,C,B);
            \tkzDrawCircle(I,A);
            % Images
            \tkzDefPointBy[rotation=center O angle 100](A);\tkzGetPoint{A'};
            \tkzDefPointBy[rotation=center O angle 100](B);\tkzGetPoint{B'};
            \tkzDefPointBy[rotation=center O angle 100](C);\tkzGetPoint{C'};
            \tkzDefPointBy[rotation=center O angle 100](K);\tkzGetPoint{K'};
            \tkzDefPointBy[rotation=center O angle 100](I);\tkzGetPoint{I'};
            % Tracés
            \tkzDrawSegment[style=red, dim={$~r~$,-5pt,midway,font=\scriptsize,sloped}](I',B');
            \draw (A')--(B')--(C')--(A');
            \draw (K')--(I');
            \tkzDrawPoints[shape=cross out](I');        
            \tkzLabelPoints[above left](B');
            \tkzLabelPoints[below right](A',I');
            \tkzLabelPoints[above left](C');
            \tkzLabelPoints(K');
            \tkzMarkRightAngles[size=0.1](A',K',I' A',C',B');
            \tkzDrawCircle(I',A');
            % Traits de construction
            \tkzDrawArc[dashed,color=orange,->](O,A)(A');            
            \tkzDrawArc[dashed,color=orange,->](O,C)(C');
            \tkzDrawArc[dashed,color=orange,->](O,K)(K');
            \tkzDrawArc[dashed,color=orange,->](O,I)(I');
            \draw[dashed, color=orange](O)--(A);
            \draw[dashed, color=orange](O)--(I);
            \draw[dashed, color=orange](O)--(C);
            \draw[dashed, color=orange](O)--(A');
            \draw[dashed, color=orange](O)--(I');
            \draw[dashed, color=orange](O)--(C');
            \tkzMarkAngle[size=0.5](A,O,A');
            \tkzLabelAngle[pos=0.8](A,O,A'){$\ang{100}$};
            % Sens de rotation
            \coordinate (sens) at (2,5);
            \tkzDefPointBy[rotation= center O angle 15](sens);\tkzGetPoint{sens'};
            \tkzDrawArc[->](O,sens)(sens');
            \tkzLabelArc[above](O,sens,sens'){Sens positif};
        \end{tikzpicture}
    \end{minipage}
\end{center}

\begin{remarque}
    Une rotation fait tourner une figure autour d’un point selon un angle.
    \begin{myBox}{\infoComplementsNumeriques{pluriel}}
        \hrefConstruction{https://www.geogebra.org/classic/ga6b2key}{Animation Geogebra Tux}
        \creditGeogebra{David Evéquoz}

        \hrefConstruction{https://www.geogebra.org/classic/g7cj6udr}{Animation Geogebra triangle}
        \creditGeogebra{Élise Marchand}
    \end{myBox}
\end{remarque}

% \newpage


% \Methode{}{
% \begin{center}
% $$\begin{tabular}{m{8cm}|m{8cm}}
% \multicolumn{1}{c|}{Rotation de sens direct {\resizebox{!}{0.5cm}{$\circlearrowleft$}} de centre O}
% &
% \multicolumn{1}{c}{Rotation de sens indirect {\resizebox{!}{0.5cm}{$\circlearrowright$}} de centre O}\\
% \multicolumn{1}{c|}{\par et d'angle 140\degres{}}
% &
% \multicolumn{1}{c}{\par et d'angle 60\degres{}}\\
% \multicolumn{1}{c|}{\includegraphics[scale=1]{coursrota.1}}
% &
% \multicolumn{1}{c}{\includegraphics[scale=1]{coursrota.2}}\\
% \end{tabular}$$

% \begin{mylist}
% \item Tracer la demi-droite $\left[Ox\right)$ avec le rapporteur. Attention au sens\\
% \item Avec le compas pointé en $O$, reporter la longueur $OM$ sur $\left[Ox\right)$.\\
% \item Le point d'intersection est l'image $M'$.
% \end{mylist}

% \infoComplementNumerique
% \lienCadre{http://lozano.maths.free.fr/iep_local/figures_html/scr_iep_106.html}{Construction de l'image d'un point sens direct}
% \lienCadre{http://lozano.maths.free.fr/iep_local/figures_html/scr_iep_144.html}{Construction de l'image d'une polygone par report au compas}
% \creditInstrumentPoche
% \end{center}
% }
