\subsection{Rotation}
\begin{definition}
    $M’$ est l’image de $M$ par la rotation de \textbf{centre $\mathbf{O}$} et d’\textbf{angle $\mathbf{\alpha}$} dans le sens inverse des aiguilles d’une montre signifie que : 
    \begin{itemize}
        \item $\widehat{MOM'}=\mathbf{\alpha}$ de M vers M’ dans le sens inverse des aiguilles d’une montre,
        \item $MO = OM’$
    \end{itemize}
\end{definition}

\begin{center}
    \includegraphics[scale=0.5]{\currentpath/images/coursRotationTux.png}
\end{center}

\begin{remarque}
    Une rotation fait tourner une figure autour d’un point selon un angle.
    \begin{myBox}{\infoComplementsNumeriques{pluriel}}
        \hrefConstruction{https://www.geogebra.org/classic/ga6b2key}{Animation Geogebra Tux}
        \creditGeogebra{David Evéquoz}

        \hrefConstruction{https://www.geogebra.org/classic/g7cj6udr}{Animation Geogebra triangle}
        \creditGeogebra{Élise Marchand}
    \end{myBox}
\end{remarque}

% \newpage

\begin{center}
    %\includegraphics[scale=1]{coursrota.4}
    \begin{tikzpicture}[scale=1,>=triangle 45]
        \coordinate (A) at (2,3);
        \coordinate (B) at (7,3);
        \tkzDefMidPoint(A,B);
        \tkzGetPoint{I};
        \coordinate (C) at (6,5);
        \draw (A)--(B)--(C)--(A);
    \end{tikzpicture}
\end{center}

% \Methode{}{
% \begin{center}
% $$\begin{tabular}{m{8cm}|m{8cm}}
% \multicolumn{1}{c|}{Rotation de sens direct {\resizebox{!}{0.5cm}{$\circlearrowleft$}} de centre O}
% &
% \multicolumn{1}{c}{Rotation de sens indirect {\resizebox{!}{0.5cm}{$\circlearrowright$}} de centre O}\\
% \multicolumn{1}{c|}{\par et d'angle 140\degres{}}
% &
% \multicolumn{1}{c}{\par et d'angle 60\degres{}}\\
% \multicolumn{1}{c|}{\includegraphics[scale=1]{coursrota.1}}
% &
% \multicolumn{1}{c}{\includegraphics[scale=1]{coursrota.2}}\\
% \end{tabular}$$

% \begin{mylist}
% \item Tracer la demi-droite $\left[Ox\right)$ avec le rapporteur. Attention au sens\\
% \item Avec le compas pointé en $O$, reporter la longueur $OM$ sur $\left[Ox\right)$.\\
% \item Le point d'intersection est l'image $M'$.
% \end{mylist}

% \infoComplementNumerique
% \lienCadre{http://lozano.maths.free.fr/iep_local/figures_html/scr_iep_106.html}{Construction de l'image d'un point sens direct}
% \lienCadre{http://lozano.maths.free.fr/iep_local/figures_html/scr_iep_144.html}{Construction de l'image d'une polygone par report au compas}
% \creditInstrumentPoche
% \end{center}
% }
