\begin{exercice*}
    Pedro joue au jeu de pile ou face. Il obtient 13 fois pile et 7 fois face.
    \begin{enumerate}
        \item Calcule la fréquence d'apparition de l'événement pile.
        \item Peux-tu en déduire que la pièce de Pedro est truquée ?
    \end{enumerate}
    Pedro appelle ses amis à l'aide. Ils effectuent chacun 20 lancers.
    \begin{center}
        \begin{tabular}{|>{\columncolor[gray]{0.8}}l|c|c|c|c|}
            \hline
            Amis & Lucien & Léonard & Louis & Sergio \\
            \hline
            Pile & 11     & 13      & 8     & 7      \\
            \hline
            Face & 9      & 7       & 12    & 13     \\
            \hline
        \end{tabular}
    \end{center}
    \begin{enumerate}
        \setcounter{enumi}{2}
        \item En cumulant les résultats de Pedro et de ses amis, remplis le tableau ci-dessous (fréquences arrondies à $10^{-2}$).
        \begin{center}
            \begin{tabular}{|>{\columncolor[gray]{0.8}}l|c|c|c|c|c|}
                \hline
                Au bout de ... lancers & 20 & 40 & 60 & 80 & 100 \\
                \hline
                Nombre de pile         & 13 & 24 &    &    &     \\
                \hline
                Fréquence d'apparition &    &    &    &    &     \\
                \hline
            \end{tabular}
        \end{center}

        \smallskip
        \item Utilise le tableau pour construire le graphique suivant.
        \begin{center}
            \Stat[%
            Qualitatif,
            Graphique,
            Donnee={Nombre de lancers},
            Effectif={Fréquences en \%},
            Unitex=1,Pasx=2,PasGrillex=0.5,
            Unitey=5,Pasy=0.5,PasGrilley=0.1,
            Grille,LectureFine,
            EpaisseurBatons=3,
            BatonsVides={1,2,3,4,5},                        
            ]{%
            20/0.5,40/0.5,60/0.5,80/0.5,100/0.5%
            }
        \end{center}
        \item Que peux-tu en déduire pour la pièce de Pedro ?
    \end{enumerate}
    \hrefLien{https://manuel.sesamath.net/numerique/diapo.php?atome=47194\&ordre=1}{Sesamath 3e 2021 ex11}
\end{exercice*}
\begin{corrige}
    ...
\end{corrige}
