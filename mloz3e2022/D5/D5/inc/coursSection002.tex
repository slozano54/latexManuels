\section{Calculer des probabilités}
\begin{myBox}{\emoji{light-bulb}\emoji{light-bulb} Intuitivement \emoji{light-bulb}\emoji{light-bulb}}
    \textit{La probabilité d'un événement c'est la chance qu'on a de voir se réaliser cet événement.}
    
    \smallskip
    Par exemple, on a 1 chance sur 2 d'obtenir pile lorsqu'on lance une pièce bien équilibrée, cela signifie que la probabilité de l'événement "obtenir pile" vaut $\frac12$.
    
    \smallskip
    Lorsqu'on ne peut pas, a priori, déterminer le nombre de cas possibles, on \textbf{répète un grand nombre de fois} l'expérience. On approche ainsi la probabilité des événements en considérant les fréquences qui tendent à se stabiliser avec le nombre croissant d'expériences.
\end{myBox}
\begin{definition}
    Dans une expérience aléatoire, la probabilité d'un événement est égale au quotient :
    $$\dfrac{\text{Nombre de cas favorables}}{\text{Nombre de cas possibles}}$$
\end{definition}

\begin{vocabulaire}
    On dit que l'on a une situation d'\textbf{équiprobabilité} lorsque toutes les issues ont la même probabilité de se réaliser.
    Dans ce cas, la probabilité d'un évènement vaut : 
    $$\dfrac{\text{nombre d'issues favorables}}{\text{nombre de cas possibles}}$$
\end{vocabulaire}

\begin{propriete}[\admise]
    Lorsqu'une expérience aléatoire est une situation d'équiprobabilité et qu'il y a $n$ issues alors la probabilité d'un événement élémentaire vaut $\dfrac{1}{n}$
\end{propriete}

\begin{remarque}
   dans un exercice, pour signifier que l'on est dans une situation d'équiprobabilité, on a des expressions du type : \og on lance un dé \textbf{non pipé} \fg ; \og dans une urne, les boules sont \textbf{indiscernables} au toucher \fg ; \og on rencontre \textbf{au hasard} une personne parmi\dots \fg.
\end{remarque}

\begin{exemple*1}   
    On tire une carte dans un jeu non truqué de 52 cartes.\\Quelle est la probabilité d'obtenir une tête ?
    \correction     
    Le jeu est non truqué, il y a donc équiprobabilité. Les issues possibles sont le valet, la dame et le roi de pique, de carreau, de trèfle et de coeur ce qui fait $4\times3$ cartes donc, $P =\dfrac{12}{52}$.
 \end{exemple*1}
 
\begin{exemples*1}   
   \begin{itemize}
        \item Tirer à pile ou face, la probabilité de chaque événement élémentaire vaut $\dfrac12$.
        \item Lancer un dé à six faces, la probabilité de chaque événement élémentaire vaut $\dfrac16$.
        \item Une urne contient 3 boules rouges et 5 boules jaunes.\\
        Les boules sont indiscernables au toucher.\\
        \textbf{La représentation sous forme d'arbre n'est plus au programme de cycle 4.}
        %Arbre des possibilités :
        \begin{tabular}{cccccccc}
            &&&\pnode(0,0.2em){D}{}&&&&\\
            &&&&&&&\\
            \pnode(0,0.2em){R1}{}&\pnode(0,0.2em){R2}{}&\pnode(0,0.2em){R3}{}&\pnode(0,0.2em){J1}{}&\pnode(0,0.2em){J2}{}&\pnode(0,0.2em){J3}{}&\pnode(0,0.2em){J4}{}&\pnode(0,0.2em){J5}{}\\
            R&R&R&J&J&J&J&J\\
        \end{tabular}
        \ncarc{->}{D}{R1}\ncarc{->}{D}{R2}\ncarc{->}{D}{R3}
        \ncarc{->}{D}{J1}\ncarc{->}{D}{J2}\ncarc{->}{D}{J3}\ncarc{->}{D}{J4}\ncarc{->}{D}{J5}
        \hfill Chaque boule a une probabilité de $\dfrac18$ d'être tirée\\
        
        \smallskip
        Et l'arbre des possibilités simplifié:\hfill
        \begin{minipage}{0.5\linewidth}        
            \hspace{0.3\linewidth}\pnode(0,0.2em){R}{R}\par
            \pnode(0,0.2em){D}{}\par
            \hspace{0.3\linewidth}\pnode(0,0.2em){J}{J}\par
            \ncarc{->}{D}{R}
            \naput{$\dfrac38$}
            \ncarc{->}{D}{J}
            \nbput{$\dfrac58$}
        \end{minipage}

        \bigskip
        \textbf{On lui préférera la présentation sous forme de tableau à double entrée :}
        
        \smallskip
        \begin{tabular}{|c|c|c|c|}
            \hline
            &Boules rouges&Boules jaunes&Total\\\hline
            Effectif&3&5&8\\\hline
        \end{tabular}
        \begin{list}{$\gtrdot$}{}
            \item La probabilité de l'événement élémentaire "sortir une boule rouge" vaut donc : \\$\dfrac38=0,375=37,5\%$
            \item La probabilité de l'événement élémentaire "sortir une boule jaune" vaut donc : \\$\dfrac58=0,625=62,5\%$
        \end{list}

        \smallskip
        \psshadowbox{Cette situation n'est donc pas une situation d'équiprobabilité.}
    \end{itemize}
\end{exemples*1}

\begin{propriete}
    \begin{itemize}
        \item La probabilité d'un événement est toujours comprises entre 0 et 1.
        \item La somme des probabilités de tous les événements élémentaires possibles vaut 1.
        \item La probabilité d'un événement certain vaut 1 et réciproquement.
        \item La probabilité d'un événement impossible vaut 0 et réciproquement.
    \end{itemize}
\end{propriete}

\begin{exemple*1}
    On lance un dé classique équilibré à six faces.
    
    Quelle est la probabilité d'obtenir un 9 ?
    
    Quelle est la probabilité d'obtenir un nombre entier ?
    \correction
    Nous sommes dans une situation d'équiprobabilité.
    \begin{itemize} 
        \item On ne peut pas obtenir un 9 avec un dé à 6 faces donc, $P =\dfrac06 =0$.
        \item Tous les nombres obtenus sont entiers donc, $P =\dfrac66 =1$.
    \end{itemize}
\end{exemple*1}
