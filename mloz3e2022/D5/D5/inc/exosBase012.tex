\begin{exercice*}
    Un sac contient 20 jetons qui sont soit jaunes, soit verts, soit rouges, soit bleus. On considère l'expérience suivante : «tirer au hasard un jeton, noter sa couleur et remettre le jeton dans le sac». Chaque jeton a la même probabilité d'être tiré.
    
    Le professeur, qui connaît la composition du sac, a simulé un grand nombre de fois l'expérience avec un tableur. Il a représenté ci-dessous la fréquence d'apparition des différentes couleurs après 1 000 tirages.
    \begin{center}
        \begin{tabular}{c}
            \includegraphics[width=0.45\textwidth]{\currentpath/images/d5ex12.png}
        \end{tabular}
    \end{center}
    \begin{enumerate}
        \item Quelle couleur est la plus présente dans le sac ?
        \item Le professeur a construit la feuille de calcul suivante.
        \begin{center}
            \begin{Tableur}[Bandeau=false,LargeurUn=40pt,Largeur=60pt,Colonnes=3]
                Nombre de tirages & Nombre de fois où un jeton rouge est apparu & Fréquence d'apparition de la couleur rouge \\
                1                 & 0                                           & 0                                          \\
                2                 & 0                                           & 0                                          \\
                3                 & 0                                           & 0                                          \\
                4                 & 0                                           & 0                                          \\
                5                 & 0                                           & 0                                          \\
                6                 & 1                                           & \num{0,166 666 667}                        \\
                7                 & 1                                           & \num{0,142 857 143}                        \\
                8                 & 1                                           & \num{0,125        }                        \\
                9                 & 1                                           & \num{0,111 111 111}                        \\
                10                & 1                                           & \num{0,1          }                        \\
            \end{Tableur}
        \end{center}

        \smallskip
        \item Quelle formule a-t-il saisie dans la cellule {\ttfamily C2} avant de la recopier vers le bas ?
        \item Quelle pourrait être la composition du sac ?
    \end{enumerate}
    \hrefLien{https://manuel.sesamath.net/numerique/diapo.php?atome=47195\&ordre=1}{Sesamath 3e 2021 ex12}
\end{exercice*}
\begin{corrige}
    ...
\end{corrige}
