\begin{exercice*}
    Dans la vitrine d'un magasin A sont présentés au total 45 modèles de chaussures. Certaines sont conçues pour la ville, d'autres pour le sport et sont de trois couleurs différentes : noires, blanches ou marron.

    \begin{center}
        \begin{tabular}{|>{\columncolor[gray]{0.8}}l|c|c|c|}
            \hline
            \rowcolor[gray]{0.8}Modèle & Pour la ville & Pour le sport & Total \\
            \hline
            Noir   &               & 5             & 20    \\
            \hline
            Blanc  & 7             &               &       \\
            \hline
            Marron &               & 3             &       \\
            \hline
            Total  & 27            &               & 45    \\
            \hline
        \end{tabular}
    \end{center}
    \begin{enumerate}
        \item Complète le tableau suivant.
        \item On choisit un modèle de chaussures au hasard dans cette vitrine. Quelle est la probabilité de choisir un modèle de couleur noire ?
        \item Quelle est la probabilité de choisir un modèle pour le sport ?
        \item Quelle est la probabilité de choisir un modèle pour la ville de couleur marron ?
        \item Dans la vitrine d'un magasin B, on trouve 54 modèles de chaussures, dont 30 de couleur noire. On choisit au hasard un modèle de chaussures dans la vitrine du magasin A puis dans celle du magasin B. Dans laquelle des deux vitrines a-t-on le plus de chance d'obtenir un modèle de couleur noire ? Justifie.
    \end{enumerate}
    \hrefLien{https://manuel.sesamath.net/numerique/diapo.php?atome=47187\&ordre=1}{Sesamath 3e 2021 ex4}
\end{exercice*}
\begin{corrige}
    ...
\end{corrige}
