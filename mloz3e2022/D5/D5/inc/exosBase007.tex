\begin{exercice*}[D'après DNB]
    Mathilde fait tourner deux roues de loterie A et B comportant chacune quatre secteurs numérotés comme sur le schéma ci-dessous.
    \begin{center}
        \begin{tabular}{cc}
            \begin{Geometrie}
                pair O,A,B,C,D;
                O=u*(2,2);
                path cc;
                cc=cercles(O,u);
                A=pointarc(cc,0);
                B=pointarc(cc,90);
                C=pointarc(cc,180);
                D=pointarc(cc,270);
                trace cc;
                trace segment(A,C);
                trace segment(B,D);
                label(TEX("\ding{173}"),iso(A,B));%2
                label(TEX("\ding{172}"),iso(B,C));%1
                label(TEX("\ding{175}"),iso(C,D));%4
                label(TEX("\ding{174}"),iso(D,A));%3
                drawarrow (pointarc(cc,45)+u*(0.3,0.3))--pointarc(cc,45) withpen pencircle scaled 1.5;
            \end{Geometrie}
            &
                \begin{Geometrie}
                    pair O,A,B,C,D;
                    O=u*(2,2);
                    path cc;
                    cc=cercles(O,u);
                    A=pointarc(cc,0);
                    B=pointarc(cc,90);
                    C=pointarc(cc,180);
                    D=pointarc(cc,270);
                    trace cc;
                    trace segment(A,C);
                    trace segment(B,D);
                    label(TEX("\ding{178}"),iso(A,B));%2
                    label(TEX("\ding{177}"),iso(B,C));%1
                    label(TEX("\ding{180}"),iso(C,D));%4
                    label(TEX("\ding{179}"),iso(D,A));%3
                    drawarrow (pointarc(cc,45)+u*(0.3,0.3))--pointarc(cc,45) withpen pencircle scaled 1.5;
                \end{Geometrie}
            \\
            Roue A & Roue B \\
        \end{tabular}
    \end{center}
    La probabilité d'obtenir chacun des secteurs d'une roue est la même. Les flèches indiquent les deux secteurs obtenus.
    
    L'expérience de Mathilde est la suivante : «elle fait tourner les deux roues pour obtenir un nombre à deux chiffres. Le chiffre obtenu avec la roue A est le chiffre des dizaines et celui avec la roue B est le chiffre des unités».

    Dans l'exemple ci-dessus, elle obtient le nombre 27 (roue A : 2 et roue B : 7).
    \begin{enumerate}
        \item Écris tous les nombres possibles issus de cette expérience.
        \item Prouve que la probabilité d'obtenir un nombre supérieur à 40 est 0,25.
        \item Quelle est la probabilité que Mathilde obtienne un nombre divisible par 3 ?
    \end{enumerate}
    \hrefLien{https://manuel.sesamath.net/numerique/diapo.php?atome=47190\&ordre=1}{Sesamath 3e 2021 ex7}
\end{exercice*}
\begin{corrige}
    ...
\end{corrige}
