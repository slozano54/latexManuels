\begin{exercice*}[D'après DNB]
    Deux urnes contiennent des boules numérotées indiscernables au toucher. Le schéma ci-dessous représente le contenu de chacune des urnes.
    \begin{center}
        \begin{tabular}{cc}
            \SchemaProba[%
                ListeNombres={1,2,3},%
                Echelle=0.65,%
                Pas=3,%
                Repartition={1,1,1}%
            ]
            &
            \SchemaProba[%
                ListeNombres={2,3,5,6},%
                Echelle=0.65,%
                Pas=3,%
                Repartition={1,1,1,1}%
            ]
            \\
            Urne D & Urne U \\
        \end{tabular}
    \end{center}
    On forme un nombre entier à deux chiffres en tirant au hasard une boule dans chaque urne :
    \begin{itemize}
        \item le chiffre des dizaines est le numéro de la boule issue de l'urne D ;
        \item le chiffre des unités est le numéro de la boule issue de l'urne U.
    \end{itemize}
    \begin{enumerate}
        \item A-t-on plus de chance de former un nombre pair que de former un nombre impair ?
        \item Indique les nombres premiers qu'on peut former lors de cette expérience.
        \item Montre que la probabilité de former un nombre premier est égale à $\frac{1}{6}$.
        \item Définis un événement dont la probabilité de réalisation est égale à $\frac{1}{3}$.
    \end{enumerate}
    \hrefLien{https://manuel.sesamath.net/numerique/diapo.php?atome=47191\&ordre=1}{Sesamath 3e 2021 ex8 - diapo1}

    \hrefLien{https://manuel.sesamath.net/numerique/diapo.php?atome=47191\&ordre=2}{Sesamath 3e 2021 ex8 - diapo2}
\end{exercice*}
\begin{corrige}
    ...
\end{corrige}
