\begin{exercice*}[D'après DNB]
    Une classe de 3ème est constituée de 25 élèves. Certains sont externes, les autres sont demi-pensionnaires (DP). Le tableau ci-dessous donne la composition de la classe.
    \begin{center}
        \begin{tabular}{|>{\columncolor[gray]{0.8}}l|c|c|c|}
            \hline
            \rowcolor[gray]{0.8}& Garçons & Filles & Total \\
            \hline
            Externes &         & 3      &       \\
            \hline        
            DP       & 9       & 11     &       \\
            \hline
            Total    &         &        & 25    \\
            \hline
        \end{tabular}
    \end{center}
    \begin{enumerate}
        \item Complète le tableau.
        \item On choisit au hasard un élève de cette classe. Quelle est la probabilité pour que cet élève soit une fille ?
        \item Quelle est la probabilité pour que cet élève soit externe ?
        \item Si cet élève est demi-pensionnaire, quelle est la probabilité que ce soit un garçon ?
    \end{enumerate}
    \hrefLien{https://manuel.sesamath.net/numerique/diapo.php?atome=47186\&ordre=1}{Sesamath 3e 2021 ex3}
\end{exercice*}
\begin{corrige}
    ...
\end{corrige}
