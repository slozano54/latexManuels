\begin{exercice*}[D'après DNB]
    Pour fêter son anniversaire, Yoshi a acheté à la boutique deux boîtes de macarons.
    \begin{itemize}
        \item La boîte numéro 1 est composée de : «4 macarons au chocolat, 3 macarons au café, 2 macarons à la vanille et 3 macarons au caramel».
        \item La boîte numéro 2 est composée de : «2 macarons au chocolat, 1 macaron à la fraise, 1 macaron à la framboise et 2 macarons à la vanille».
    \end{itemize}    
    On suppose dans la suite que les macarons sont indiscernables au toucher.
    \begin{enumerate}
        \item Si on choisit au hasard un macaron dans la boîte numéro 1, quelle est la probabilité que ce soit un macaron au café ?
        \item Au bout d'une heure il reste 3 macarons au chocolat et 2 macarons au café dans la boîte numéro 1 et 2 macarons au chocolat et 1 macaron à la fraise dans la boîte numéro 2. Téhora n'aime pas le chocolat mais apprécie tous les autres parfums. Elle choisit un macaron au hasard dans la boîte numéro 1, puis un second dans la boîte numéro 2. Quelle est la probabilité qu'elle obtienne deux macarons qui lui plaisent ?
    \end{enumerate}
    \hrefLien{https://manuel.sesamath.net/numerique/diapo.php?atome=47192\&ordre=1}{Sesamath 3e 2021 ex9}
\end{exercice*}
\begin{corrige}
    ...
\end{corrige}
