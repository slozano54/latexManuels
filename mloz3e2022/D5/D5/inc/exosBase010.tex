\begin{exercice*}[D'après DNB]
    On dispose de deux urnes :
    \begin{itemize}
        \item une urne bleue contenant trois boules bleues numérotées 2, 3 et 4 ;
        \item une urne rouge contenant quatre boules rouges numérotées 2, 3, 4 et 5.
    \end{itemize}
    Dans chaque urne, les boules sont indiscernables au toucher et ont la même probabilité d'être tirées.
    On s'intéresse à l'expérience aléatoire suivante : « On tire au hasard une boule bleue et on note son numéro, puis on tire au hasard une boule rouge et on note son numéro. »

    Par exemple, si on tire la boule bleue numérotée 3 puis la boule rouge numérotée 4, le tirage obtenu sera noté (3 ; 4). On précise que le tirage (3 ; 4) est différent du tirage (4 ; 3).

    Dans les deux questions suivantes, on définit les deux événements suivants : « On obtient deux nombres premiers. » et « La somme des nombres est égale à 12. »
    \begin{enumerate}
        \item Pour chacun des deux événements précédents, dis s'il est possible ou impossible lorsqu'on effectue l'expérience aléatoire.
        \item Quel est le nombre de tirages possibles ?
        \item Détermine la probabilité de l'événement : « On obtient deux nombres premiers. »
        \item Détermine la probabilité de l'événement : « La somme des nombres est égale à 12. »
        \item On obtient un « double » lorsque les deux boules tirées portent le même numéro. Justifie que la probabilité d'obtenir un « double » lors de cette expérience est $\frac{1}{4}$.
    \end{enumerate}
    On souhaite simuler cette expérience 1 000 fois. Pour cela, on a commencé à écrire un programme,à ce stade, encore incomplet.

    Voici des copies d’écran.

    Boule bleue, Boule rouge et Nombre de doubles sont des variables. Le bloc « Tirer deux boules » est à insérer dans le script principal.

    \clearpage
    \begin{center}
        \textbf{Bloc « Tirer deux boules »}

        \begin{Scratch}[Echelle=0.75]
            Placer NouveauBloc("Tirer deux boules");
            Placer MettreVar("Boule bleue",OpAlea("2","\textbf{B}"));
            Placer MettreVar("Boule rouge",OpAlea("2","\textbf{C}"));
        \end{Scratch}

        \textbf{Script principal}

        \begin{Scratch}[Echelle=0.75]
            Placer Drapeau;
            Placer Repeter("\textbf{A}");
            Placer Si(TestOpEgal(OvalVar("Boule bleue"),OvalVar("Boule rouge")));
            Placer AjouterVar("1","Nombres de doubles");
            Placer FinBlocSi;
            Placer FinBlocRepeter;
        \end{Scratch}
    \end{center}
    \begin{enumerate}
        \setcounter{enumi}{5}
        \item Par quels nombres faut-il remplacer les lettres A,B et C ?
        \item Dans le script principal, indique où placer le bloc ci-dessous.
              \begin{center}
                  \begin{Scratch}[Echelle=0.75]
                      Placer Bloc("Tirer deux boules");
                  \end{Scratch}
              \end{center}
        \item Dans le script principal, indique où placer le bloc ci-dessous.        
              \begin{center}
                  \begin{Scratch}[Echelle=0.75]
                      Placer MettreVar("Nombre de doubles","0");
                  \end{Scratch}
              \end{center}
        \item On souhaite obtenir la fréquence d’apparition du nombre de « doubles » obtenus. Parmi les instructions ci-dessous, laquelle faut-il placer à la fin du script principal après la boucle « répéter » ? Entoure la bonne réponse.
        \begin{description}
            \item[Proposition 1 : ] {\LARGE \ScratchEnLigne*{Dire(OvalVar("Nombre de doubles"))}}
            \bigskip
            \item[Proposition 2 : ] {\LARGE \ScratchEnLigne*{Dire(OpDiv(OvalVar("Nombre de doubles"),"1000"))}}
            \bigskip
            \item[Proposition 3 : ] {\LARGE \ScratchEnLigne*{Dire(OpMul(OvalVar("Nombre de doubles"),"1000"))}}
        \end{description}
    \end{enumerate}
    \hrefLien{https://manuel.sesamath.net/numerique/diapo.php?atome=47193\&ordre=1}{Sesamath 3e 2021 ex10}
\end{exercice*}
\begin{corrige}
    ...
\end{corrige}
