\begin{exercice}
    La \textbf{machine $\boldsymbol{d}$}, qui n'accepte que des nombres entiers positifs, renvoie le nombre de diviseurs du nombre de départ.
    \begin{center}
        \scalebox{0.7}{
            \machineMaths{$\boldsymbol{d}$}{Nombre total\\de diviseurs}{Nombre\\entrant : $51$}{Nombre de\\diviseurs}
        }
    \end{center}
    \vspace*{-5mm} 
    \begin{enumerate}
        \item Que renvoie la machine si le nombre entrant vaut $51$ ? Formuler la réponse avec le mot \textbf{image}.
        \item Quelle est une valeur possible du nombre entrant si la machine renvoie  $2$ ? En existe-t-il plusieurs ?
        \item Quelle est l'image de $76$ par la \textbf{fonction $\boldsymbol{d}$} ? Écrire la réponse sous la forme $\boldsymbol{d(76)=\ldots}$.
        \item Peut-on trouver plusieurs antécédents de 3 par la fonction $d$ ? Qu'ont-ils de commun ?
    \end{enumerate}
\end{exercice}
\begin{corrige}
    La \textbf{machine $\boldsymbol{d}$}, qui n'accepte que des nombres entiers positifs, renvoie le nombre de diviseurs du nombre de départ.

    \hspace*{-13mm}
    \scalebox{0.5}{
        \machineMaths{$\boldsymbol{d}$}{Nombre total\\de diviseurs}{Nombre\\entrant : $51$}{Nombre de\\diviseurs}
    }
    \begin{enumerate}
        \item Que renvoie la machine si le nombre entrant vaut $51$ ? Formuler la réponse avec le mot \textbf{image}.
        
        {\red $1;~3;~17;~51$ sont les diviseurs de $51$ donc l'image de $51$ par $d$ vaut $4$.}
        \item Quelle est une valeur possible du nombre entrant si la machine renvoie  $2$ ? En existe-t-il plusieurs ?
        
        {\red Un antécédent de $2$ par $d$ est un nombre qui admet deux diviseurs exactement. Il y a $2;~3;~5;~7;~\dots$ ce sont tous des nombres premiers.}
        \item Quelle est l'image de $76$ par la \textbf{fonction $\boldsymbol{d}$} ? Écrire la réponse sous la forme $\boldsymbol{d(76)=\ldots}$.
        
        {\red $1;~2;~4;~19;~38;~76$ sont les diviseurs de $76$ donc l'image de $76$ par $d$ vaut $6$.}
    % \end{enumerate}
    % \Coupe  
    % \begin{enumerate}
    % \setcounter{enumi}{3}
        \item Peut-on trouver plusieurs antécédents de 3 par la fonction $d$ ? Qu'ont-ils de commun ?
        
        {\red 
        % Les diviseurs de $4$ sont $1;~2;~4$ donc $d(4) = 3$.\\
        Les diviseurs de $9$ sont $1;~3;~9$ donc $d(9) = 3$.\\
        Les diviseurs de $25$ sont $1;~5;~25$ donc $d(25) = 5$.\\
        Ce sont tous des carrés de nombres premiers.       
        }
    \end{enumerate}
\end{corrige}
