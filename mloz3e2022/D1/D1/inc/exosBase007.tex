\begin{exercice}
    Soit la fonction $h$ qui a tout nombre associe\\ sa moitié.
    \begin{enumerate}
        \item Déterminer l'image de $16$.
        \item Déterminer l'image de $9$.
        \item Calculer $h(12)$.
        \item Recopier et compléter : $h(\dots{})=16$.
        \item Exprimer $h(x)$ en fonction de $x$.
    \end{enumerate}
\end{exercice}
\begin{corrige}
    Soit la fonction $h$ qui a tout nombre associe sa moitié.

    \begin{enumerate}
        \item Déterminer l'image de $16$.
        
        {\red La moitié de $16$ vaut $8$ donc l'image de $16$ par $h$ vaut $8$. $h(16)=8$.}
        \item Déterminer l'image de $9$.
        
        {\red De même, $h(9)=\num{4.5}$.}
        \item Calculer $h(12)$.
        
        {\red $h(12)=12\div 2 = 6$.}
        \item Recopier et compléter : $h(\dots{})=16$.
        
        {\red $32\div 2 = 16$ donc $h(32)=16$.}
        \item Exprimer $h(x)$ en fonction de $x$.
        
        {\red $h(x) = x\div 2$ ou $h(x)=\dfrac{x}{2}$.}
    \end{enumerate}
\end{corrige}
