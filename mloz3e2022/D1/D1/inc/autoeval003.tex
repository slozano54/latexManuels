\begin{exercice}
    On considère la situation de proportionnalité suivante : avec \Masse[kg]{100} de blé, on fait \Masse[kg]{80} de farine.
    \begin{enumerate}
        \item Déterminer la quantité de farine que l'on obtient avec \Masse[kg]{330} de blé.
        \item Déterminer la quantité de blé necessaire pour obtenir \Masse[kg]{185} de farine.
    \end{enumerate}
\end{exercice}
\begin{corrige}
    Avec \Masse[kg]{100} de blé, on fait \Masse[kg]{80} de farine.
    \begin{enumerate}
        \item Déterminer la quantité de farine que l'on obtient avec \Masse[kg]{330} de blé.
        
        {\red $\dfrac{\text{\Masse[kg]{80}}}{\text{\Masse[kg]{100}}}\times \text{\Masse[kg]{330}}=\text{\Masse[kg]{264}}$

        Avec \Masse[kg]{330} de blé, on obtient \Masse[kg]{264} de farine.
        }
        \item Déterminer la quantité de blé necessaire pour obtenir \Masse[kg]{185} de farine.

        {\red $\dfrac{\text{\Masse[kg]{100}}}{\text{\Masse[kg]{80}}}\times \text{\Masse[kg]{185}}=\text{\Masse[kg]{231.25}}$

        Il faut \Masse[kg]{231.25} de blé pour obtenir \Masse[kg]{185} de farine.
        }        
    \end{enumerate}    
\end{corrige}