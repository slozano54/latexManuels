\begin{exercice}
    On considère le tableau ci-dessous.

    \begin{tabular}{|l|c|c|c|}
        \hline
        \og\ Poids\fg\ d'une lettre (en \Masse[g]{})&15&30&90\\
        \hline
        Affranchissement (en €)&0,50&0,75&1\\
        \hline
    \end{tabular}

    \smallskip
    Expliquer s'il y a proportionnalité ou non entre l'affranchissement d'une lettre et son poids.
\end{exercice}
\begin{corrige}
    On considère le tableau ci-dessous.

    \begin{tabular}{|l|c|c|c|}
        \hline
        \og\ Poids\fg\ d'une lettre (en \Masse[g]{})&15&30&90\\
        \hline
        Affranchissement (en €)&0,50&0,75&1\\
        \hline
    \end{tabular}

    \smallskip
    Expliquer s'il y a proportionnalité ou non entre l'affranchissement d'une lettre et son poids.

    {\red $\dfrac{15}{\num{0.50}}=\dfrac{15\times 2}{\num{0.50}\times 2}=\dfrac{30}{1}$.

    On constate que $\dfrac{15}{\num{0.50}}\neq\dfrac{30}{\num{0.75}}$, donc l'affranchissement d'une lettre et son poids ne sont pas proportionnels.
    }
\end{corrige}