\begin{exercice}
    Une entreprise fabrique chaque jour un produit. $x$ est la masse journalière produite en kg. $x$ peut varier
    entre $0$ et $45$. Le coût de production de ces $x$ kg de produit exprimé en euros est donné par la formule : $C(x) = x^2-20x+200$.
    Ce produit est vendu \Prix[0]{34} le kg. On suppose que tous les objets fabriqués sont vendus.
    \begin{enumerate}
        \item Déterminer le coût de production de \Masse[kg]{10} de produit.
        \item Déterminer la recette liée à la vente de ces \Masse[kg]{10}.
        \item Déterminer le bénéfice réalisé.
        \item Déterminer la recette $R(x)$ réalisée lorsque l'entreprise fabrique et vend $x$ kg de produit.
        \item Déterminer le bénéfice $B(x)$ correspondant.
        \item Tracer dans un repère de ce type, la représentation graphique de la fonction $B$.\par\smallskip     
        \scalebox{0.68}{
        \Fonction[%
        Calcul=-x*x+54*x-200,
        Trace,CouleurTrace=rouge,
        Xmin=-0.95,Xmax=9,Xstep=5,
        Ymin=-0.5,Ymax=6.5,Ystep=100,
        Origine={(0.95,0.5)},
        Grille,PasGrilleX=1,PasGrilleY=1,
        Graduations,PasGradX=5,PasGradY=100,
        Bornea=45.5,Borneb=45.6,
        % LabelC=0.95,NomCourbe=$B(x)$
        ]{}
        }
        \item Déterminer la valeur de $x$ pour laquelle le bénéfice est maximal. Donner le bénéfice correspondant.
    \end{enumerate}
\end{exercice}
\begin{corrige}
    Une entreprise fabrique chaque jour un produit. On appelle $x$ la masse journalière produite en kg. $x$ peut varier
    entre $0$ et $45$. Le coût de production de ces $x$ kg de produit exprimé en euros est donné par la formule : $C(x) = x^2-20x+200$.
    Le prix de vente de ce produit est de \Prix[0]{34} le kg. On suppose que tous les objets fabriqués sont vendus.

    \begin{enumerate}
        \item Déterminer le coût de production de \Masse[kg]{10} de produit.
        
        {\red $C(10)=10^2-20\times 10 + 200 = 100$, donc le coût de production de \Masse[kg]{10} de produit est de \Prix[0]{100}.}
        \item Déterminer la recette liée à la vente de ces \Masse[kg]{10}.
        
        {\red $34\times 10 = 340$, donc la recette est de \Prix[0]{340}.}
        \item Déterminer le bénéfice réalisé.
        
        {\red $340-100 = 240$, donc le bénéfice est de \Prix[0]{240}.}
        \item Déterminer la recette $R(x)$ réalisée lorsque l'entreprise fabrique et vend $x$ kg de produit.
        
        {\red $R(x)=34\times x = 34x$ \Prix[0]{}.}
        \item Déterminer le bénéfice $B(x)$ correspondant.
        
        {\red $B(x)=R(x)-C(x)$\\$B(x)=34x-(x^2-20x+200)$\\$B(x)=3x-x^2+20x-200$\\$B(x)=-x^2+54x-200$\\}
        \item Tracer dans un repère de ce type, la représentation graphique de la fonction $B$.
        
        \scalebox{0.6}{
        \Fonction[%
        Calcul=-x*x+54*x-200,
        Trace,CouleurTrace=rouge,
        Xmin=-0.95,Xmax=9,Xstep=5,
        Ymin=-0.5,Ymax=6.5,Ystep=100,
        Origine={(0.95,0.5)},
        Grille,PasGrilleX=1,PasGrilleY=1,
        Graduations,PasGradX=5,PasGradY=100,
        Bornea=4,Borneb=45,
        LabelC=0.95,NomCourbe=$B(x)$
        ]{}
        }
    \end{enumerate}
    \Coupe
    \begin{enumerate}
    \setcounter{enumi}{6}
        \item Déterminer la valeur de $x$ pour laquelle le bénéfice est maximal. Donner le bénéfice correspondant.
        
        {\red Le bénéfice semble maximal lorsque $x$ est environ égal à $27$ et il est d'environ \Prix[0]{530}.}
    \end{enumerate} 
\end{corrige}
