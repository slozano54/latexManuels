\begin{exercice}
    Ce graphique représente une fonction $g$ pour des valeurs de $x$ comprises entre $-5$ et $12$.
    
    \Fonction[Points,Catmull,UniteX=0.4,UniteY=0.4]{%
    0/-5/5/0§%
    0/-4/1/0§%
    0/-3/0/0§%
    0/-2/1/0§%
    0/-1/2/0§%
    0/1/3/0§%
    0/3/2/0§%
    0/4/1/0§%
    0/5/-1/0§%
    0/6/-2/0§%
    0/7/-3/0§%
    0/8/-2/0§%
    0/9/-1/0§%
    0/10/1/0§%
    0/11/3/0%
    }
    \begin{enumerate}
        \item Placer le point $E$ de la courbre d'abscisse $1$.
        \item Lire et écrire l'ordonnée de $E$.
        \item Placer le point $F$ de la courbe d'abscisse $8$.
        \item Lire et écrire l'ordonnée de $F$.
        \item Placer les points $G_1$, $G_2$, $G_3$ et $G_4$ de la courbe qui ont pour ordonnée $1$.
        \item Écrire les coordonnées dess points $G_1$, $G_2$, $G_3$ et $G_4$.
        \item Déterminer le nombre de points de la courbe dont l'ordonnée $-2$.
        \item Écrire les coordonnées des points précédents.
    \end{enumerate}
\end{exercice}
\begin{corrige}
    Ce graphique représente une fonction $g$ pour des valeurs de $x$ comprises entre $-5$ et $12$.
    
    \hspace*{-7mm}
    \Fonction[Points,Catmull,UniteX=0.45,UniteY=0.45]{%
    0/-5/5/0§%
    0/-4/1/0§%
    0/-3/0/0§%
    0/-2/1/0§%
    0/-1/2/0§%
    0/1/3/0§%
    0/3/2/0§%
    0/4/1/0§%
    0/5/-1/0§%
    0/6/-2/0§%
    0/7/-3/0§%
    0/8/-2/0§%
    0/9/-1/0§%
    0/10/1/0§%
    0/11/3/0%
    }
    \begin{enumerate}
        \item Placer le point $E$ de la courbre d'abscisse $1$.
        
        {\red En attente clef ProfCollege}
        \item Lire et écrire l'ordonnée de $E$.
        
        {\red $E(1;3)$}
        \item Placer le point $F$ de la courbe d'abscisse $8$.
        
        {\red En attente clef ProfCollege}
        \item Lire et écrire l'ordonnée de $F$.
        
        {\red $F(8;-2)$}
        \item Placer les points $G_1$, $G_2$, $G_3$ et $G_4$ de la courbe qui ont pour ordonnée $1$.
        
        {\red En attente clef ProfCollege}
        \item Écrire les coordonnées dess points $G_1$, $G_2$, $G_3$ et $G_4$.
        
        {\red $G_1(-4;1)$, $G_2(-2;1)$, $G_3(4;1)$ et $G_4(10;1)$}
        \item Déterminer le nombre de points dont l'ordonnée $-2$.
        
        {\red Il y a deux points d'ordonnée $-2$.}
        \item Écrire les coordonnées des points précédents.
        
        {\red $(6;-2)$ et $(8;-2)$.}
    \end{enumerate}
\end{corrige}
