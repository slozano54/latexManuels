\begin{exercice}
    La \textbf{machine $\boldsymbol{h}$} renvoie la somme du triple du nombre de départ et de 1.
    \begin{center}
        \scalebox{0.7}{
            \machineMaths{$\boldsymbol{h}$}{Multiplier par $3$\\Ajouter $1$}{Nombre\\entrant : $10$}{Nombre associé\\au nombre entrant ???}
        }
    \end{center}
    \vspace*{-5mm} 
    \begin{enumerate}
        \item Que renvoie la machine si le nombre entrant vaut $10$ ? Formuler la réponse avec le mot \textbf{image}.
        \item Combien vaut le nombre entrant si la machine renvoie  $265$ ? Formuler la réponse avec le mot \textbf{antécédent}.
        \item Quelle est l'image de $-42$ par la \textbf{fonction $\boldsymbol{h}$} ? Écrire la réponse sous la forme $\boldsymbol{h(-42)=\ldots}$.
        \item Que renvoie la machine si le nombre entrant vaut $x$ ? Écrire la réponse sous la forme $\boldsymbol{h(x)=\ldots}$.
        \item Comme dans l'exemple ci-dessous, écrire le diagramme de la fonction $\boldsymbol{h}$.
            \begin{exemple*1}
                Voici le diagramme d'une machine qui double puis qui ajoute $5$.
                
                \medskip
                \begin{tikzpicture}[line cap=round,line join=round,>=triangle 45,x=1cm,y=1cm]
                    \draw [line width=2pt,color=J1] (-10,0.5) -- (-9,0.5) -- (-9,-0.5) -- (-10,-0.5) -- cycle;
                    \node[text width=3cm,text centered, scale=1] at(-9.5,0){$x$};                
                    \draw [line width=2pt,color=J1] (-9,0) -- (-8.5,0);
                    \draw [line width=2pt,color=J1] (-8,0) circle(0.5);
                    \node [text width=3cm,text centered, scale=1] at(-8,0){$\times 2$};
                    \draw [->,line width=2pt,color=J1] (-7.5,0) -- (-6.5,0);
                    \draw [line width=2pt,color=J1] (-6.5,0.5) -- (-5.5,0.5) -- (-5.5,-0.5) -- (-6.5,-0.5) -- cycle;
                    \node [text width=3cm,text centered, scale=1] at(-6,0){$2x$};                
                    \draw [line width=2pt,color=J1] (-5.5,0) -- (-5,0);
                    \draw [line width=2pt,color=J1] (-4.5,0) circle(0.5);
                    \node [text width=3cm,text centered, scale=1] at(-4.5,0){$+5$};
                    \draw [->,line width=2pt,color=J1] (-4,0) -- (-3,0);
                    \draw [line width=2pt,color=J1] (-3,0.5) -- (0,0.5) -- (0,-0.5) -- (-3,-0.5) -- cycle;
                    \node [text width=3cm,text centered, scale=1] at(-1.5,0){$dc(x)=2x+5$};                      
                \end{tikzpicture}
            \end{exemple*1}

            \medskip
        \item Écrire maintenant la fonction $h$ en utilisant la forme $\boldsymbol{h:x} \longmapsto \ldots$
    \end{enumerate}
\end{exercice}
\begin{corrige}
    La \textbf{machine $\boldsymbol{h}$} renvoie la somme du triple du nombre de départ et de 1.

    \hspace*{-15mm}
    \scalebox{0.5}{
        \machineMaths{$\boldsymbol{h}$}{Multiplier par $3$\\Ajouter $1$}{Nombre\\entrant : $10$}{Nombre associé\\au nombre entrant ???}
    }
      
    \begin{enumerate}
        \item Que renvoie la machine si le nombre entrant vaut $10$ ? Formuler la réponse avec le mot \textbf{image}.
        
        {\red $10\times 3 + 1 = 31$ donc $31$ est l'image de $10$ par la  foinction $h$.}
        \item Combien vaut le nombre entrant si la machine renvoie  $265$ ? Formuler la réponse avec le mot \textbf{antécédent}.
        
        {\red $(265-1)\div 3 = 88$ donc $88$ est un antécédent de $265$ par la  foinction $h$.}
        \item Quelle est l'image de $-42$ par la \textbf{fonction $\boldsymbol{h}$} ? Écrire la réponse sous la forme $\boldsymbol{h(-42)=\ldots}$.
        
        {\red $-42\times 3 + 1 = -125$ donc $h(-42)=-125$.}
        \item Que renvoie la machine si le nombre entrant vaut $x$ ? Écrire la réponse sous la forme $\boldsymbol{h(x)=\ldots}$.
        
        {\red $x\times 3 + 1 = 3x+1$ donc $h(x)=3x+1$.}
        \item Comme dans l'exemple ci-dessous, écrire le diagramme de la fonction $\boldsymbol{h}$.
        \medskip
        \hspace*{-13mm}
        \scalebox{0.8}{\red
        \begin{tikzpicture}[line cap=round,line join=round,>=triangle 45,x=1cm,y=1cm]
            \draw [line width=2pt,color=J1] (-10,0.5) -- (-9,0.5) -- (-9,-0.5) -- (-10,-0.5) -- cycle;
            \node[text width=3cm,text centered, scale=1] at(-9.5,0){$x$};                
            \draw [line width=2pt,color=J1] (-9,0) -- (-8.5,0);
            \draw [line width=2pt,color=J1] (-8,0) circle(0.5);
            \node [text width=3cm,text centered, scale=1] at(-8,0){$\times 3$};
            \draw [->,line width=2pt,color=J1] (-7.5,0) -- (-6.5,0);
            \draw [line width=2pt,color=J1] (-6.5,0.5) -- (-5.5,0.5) -- (-5.5,-0.5) -- (-6.5,-0.5) -- cycle;
            \node [text width=3cm,text centered, scale=1] at(-6,0){$3x$};                
            \draw [line width=2pt,color=J1] (-5.5,0) -- (-5,0);
            \draw [line width=2pt,color=J1] (-4.5,0) circle(0.5);
            \node [text width=3cm,text centered, scale=1] at(-4.5,0){$+1$};
            \draw [->,line width=2pt,color=J1] (-4,0) -- (-3,0);
            \draw [line width=2pt,color=J1] (-3,0.5) -- (0,0.5) -- (0,-0.5) -- (-3,-0.5) -- cycle;
            \node [text width=3cm,text centered, scale=1] at(-1.5,0){$h(x)=3x+1$};                      
        \end{tikzpicture}
        }        
        \item Écrire maintenant la fonction $h$ en utilisant la forme $\boldsymbol{h:x} \longmapsto \ldots$
        
        {\red $h : x \longmapsto 3x+1$}
    \end{enumerate}
\end{corrige}
