\begin{exercice}
    Soit la fonction $f$ définie par $f:x\longmapsto x^2-2x-1$ pour $x$ compris entre $-1$ et $4$.
    
    \begin{enumerate}
        \item Recopier et compléter le tableau de valeurs pour $f$.
        
        \[\begin{array}{|>{\columncolor{gray!15}}c|*{6}{>{\centering\arraybackslash}p{5mm}|}}%
            \hline
            x&-1&0&1&2&3&4\\\hline
            f(x)&&&&&&\\\hline
        \end{array}
        \]
        \item Donner les coordonnées des points $A$, $B$, $C$, $D$, $E$ et $F$ appartenant à la courbe de $f$ d'abscisses respectives $-1$; $0$; $1$; $2$; $3$ et $4$.
        \item Placer ces points dans un repère comme ci-dessous et tracer une ébauche de la courbe au crayon de papier.
        \scalebox{1.5}{
        \Reperage[%
        Plan,%
        Pasx=2,Unitey=0.5,Unitex=1,Pasy=4%
        %AffichageNom,LectureCoord,,AffichageAbs=1%
        ]{-2/-8/A,6/26/B}
        }
        \item Pour plus de précision dans le tracé, on détermine d'autres points appartenant à cette courbe. Recopier et compléter le tableau
        de valeurs pour $f$.
        \[\begin{array}{|>{\columncolor{gray!15}}c|*{5}{>{\centering\arraybackslash}p{7mm}|}}%
            \hline
            x&\num{-0.5}&\num{0.5}&\num{1.5}&\num{2.5}&\num{3.5}\\\hline
            f(x)&&&&&\\\hline
        \end{array}
        \]
        \item Donner les coordonnées des points $G$, $H$, $I$, $J$ et $K$ appartenant à la courbe de $f$ d'abscisses respectives $\num{-0.5}$; $\num{0.5}$; $\num{1.5}$; $\num{2.5}$ et $\num{3.5}$.
        \item Placer les points et améliorer le tracé.
    \end{enumerate}    
\end{exercice}
\begin{corrige}
    Soit la fonction $f$ définie par $f:x\longmapsto x^2-2x-1$ pour $x$ compris entre $-1$ et $4$.
    
    \begin{enumerate}
        \item Recopier et compléter le tableau de valeurs pour $f$.        

        {\red
        \Fonction[Calcul=x**2-2*x-1,Tableau]{-1,0,1,2,3,4}
        }
    \end{enumerate}
    % \Coupe
    \begin{enumerate}
        \setcounter{enumi}{1}
        \item Donner les coordonnées des points $A$, $B$, $C$, $D$, $E$ et $F$ appartenant à la courbe de $f$ d'abscisses respectives $-1$; $0$; $1$; $2$; $3$ et $4$.
        
        {\red $A(-1;2)$\hfill$B(0;-1)$\hfill$C(1;-2)$\hfill$D(2;-1)$\hfill$E(3;2)$\hfill$F(4;7)$}
        \item Placer ces points dans un repère comme ci-dessous et tracer une ébauche de la courbe au crayon de papier.
        \Fonction[Points,Catmull,PasX=0.5,UniteY=0.5,PasY=0.25]{%
        0/-1/2/0§%
        0/0/-1/0§%
        0/1/-2/0§%
        0/2/-1/0§%
        0/3/2/0§%
        0/4/7/0%        
        }
        \item Pour plus de précision dans le tracé, on détermine d'autres points appartenant à cette courbe. Recopier et compléter le tableau
        de valeurs pour $f$.
        \[\begin{array}{|>{\columncolor{gray!15}}c|*{5}{>{\centering\arraybackslash}p{7mm}|}}%
            \hline
            x&\num{-0.5}&\num{0.5}&\num{1.5}&\num{2.5}&\num{3.5}\\\hline
            f(x)&&&&&\\\hline
        \end{array}
        \]
        {\red
        \Fonction[Calcul=x**2-2*x-1,Tableau,Largeur=7mm]{-0.5,0.5,1.5,2.5,3.5}
        }
        \item Donner les coordonnées des points $G$, $H$, $I$, $J$ et $K$ appartenant à la courbe de $f$ d'abscisses respectives $\num{-0.5}$; $\num{0.5}$; $\num{1.5}$; $\num{2.5}$ et $\num{3.5}$.
        
        {\red $G(\num{-0.5};\num{0.25})$\hfill$H(\num{0.5};\num{-1.75})$\hfill$I(\num{1.5};\num{-1.75})$\hfill
        
        $J(\num{2.5};\num{0.25})$\hfill$K(\num{3.5};\num{4.25})$\hfill
        }
        \item Placer les points et améliorer le tracé.

    \end{enumerate} 
            \Coupe
        
        \Fonction[Points,Catmull,PasX=0.5,UniteY=0.5,PasY=0.25]{%        
        0/-1/2/0§%
        0/-0.5/0.25/0§%
        0/0/-1/0§%
        0/0.5/-1.75/0§%
        0/1/-2/0§%
        0/1.5/-1.75/0§%
        0/2/-1/0§%
        0/2.5/0.25/0§%
        0/3/2/0§%
        0/3.5/4.25/0§%
        0/4/7/0%
        }
\end{corrige}
