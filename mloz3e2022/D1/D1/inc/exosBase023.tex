\begin{exercice}
    Soit la fonction $f$ définie par $f:x\longmapsto x^2-2x-1$ pour $x$ compris entre $-1$ et $4$.
    
    \begin{enumerate}
        \item Recopier et compléter le tableau de valeurs pour $f$.
        
        \[\begin{array}{|>{\columncolor{gray!15}}c|*{6}{>{\centering\arraybackslash}p{5mm}|}}%
            \hline
            x&-1&0&1&2&3&4\\\hline
            f(x)&&&&&&\\\hline
        \end{array}
        \]
        \item Donner les coordonnées des points $A$, $B$, $C$, $D$, $E$ et $F$ appartenant à la courbe de $f$ d'abscisses respectives $-1$; $0$; $1$; $2$; $3$ et $4$.
        \item Placer ces points dans un repère comme ci-dessous et tracer une ébauche de la courbe au crayon de papier.
        \Reperage[%
        Plan,%
        Pasx=2,Unitey=0.5,Pasy=4,%
        %AffichageNom,LectureCoord,,AffichageAbs=1%
        ]{-2/-8/A,6/26/B}
        \item Pour plus de précision dans le tracé, on détermine d'autres points appartenant à cette courbe. Recopier et compléter le tableau
        de valeurs pour $f$.
        \[\begin{array}{|>{\columncolor{gray!15}}c|*{5}{>{\centering\arraybackslash}p{7mm}|}}%
            \hline
            x&\num{-0.5}&\num{0.5}&\num{1.5}&\num{2.5}&\num{3.5}\\\hline
            f(x)&&&&&\\\hline
        \end{array}
        \]
        \item Donner les coordonnées des points $G$, $H$, $I$, $J$ et $K$ appartenant à la courbe de $f$ d'abscisses respectives $\num{-0.5}$; $\num{0.5}$; $\num{1.5}$; $\num{2.5}$ et $\num{3.5}$.
        \item Placer les points et améliorer le tracé.
    \end{enumerate}    
\end{exercice}
\begin{corrige}
 
\end{corrige}
