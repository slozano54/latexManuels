\begin{exercice}
    La vitesse d'un train en \Vitesse{}, $t$ minutes après le départ, vaut $3t^2$ pour $0 \leq t \leq 10$.

    On appelle $v$ la fonction qui, au temps écoulé depuis le départ exprimé en minutes, associe la vitesse du train en \Vitesse{}.
    \begin{enumerate}
        \item Calculer $v(5)$.
        \item Donner une interprétation du résultat précédent.
        \item Déterminer l'antécédent de \num{168.75} par $v$.
        \item Donner une interprétation du résultat précédent.
        \item Le graphique ci-dessous représente l'évolution de la vitesse, en \Vitesse{}, du train en fonction du temps
        écoulé, en minutes, depuis son départ.

        \medskip
        \scalebox{0.7}{
        \Fonction[%
            Calcul=3*(x**2),
            Trace,CouleurTrace=bleu,
            Xmin=0,Xmax=10,Xstep=2.5,
            Ymin=0,Ymax=5.8,Ystep=60,
            Origine={(1,.5)},
            Grille,PasGrilleX=1,PasGrilleY=0.5,
            Graduations,PasGradY=30,PasGradX=5,
            Bornea=0,Borneb=10,
            LabelX={\bf Temps (en min)},
            LabelY={\bf Vitesse (en km/h)},
            Traces={
                draw placepoint(10,300)--placepoint(20,300) withcolor bleu;
            }            
            ]{}
        }

        \item Déterminer le temps, environ, que met le train pour atteindre \Vitesse{120}.
        \item Déterminer la vitesse maximale du train.
        \item Déterminer au bout de combien de temps cette vitesse maximale est atteinte.
        \item Préciser une expression de la fonction $v$\\pour $0 \leq x \leq 20$.
    \end{enumerate}
\end{exercice}
\begin{corrige}
    La vitesse d'un train en \Vitesse{}, $t$ minutes après le départ, vaut $3t^2$ pour $0 \leq t \leq 10$.

    On appelle $v$ la fonction qui, au temps écoulé depuis le départ exprimé en minutes, associe la vitesse du train en \Vitesse{}.

    \begin{enumerate}
        \item Calculer $v(5)$.
        
        {\red $v(5)=3\times 5^2=3\times 25 = 75$.}
        \item Donner une interprétation du résultat précédent.
        
        {\red Après 5minutes, le train roule à \Vitesse{75}.}
        \item Déterminer l'antécédent de \num{168.75} par $v$.
        
        {\red $3t^2=\num{168.75}$ soit $t^2=\num{56.25}$. Puisque $t$ est une durée, c'est un nombre positif, donc $t=\num{7.5}$.}
        \item Donner une interprétation du résultat précédent.
        
        {\red Le train roule à \Vitesse{168.75} \num{7.5} min après le départ, donc \Temps{;;;;7;30}.}
        \item Le graphique ci-dessous représente l'évolution de la vitesse, en \Vitesse{}, du train en fonction du temps
        écoulé, en minutes, depuis son départ.

        \medskip
        \hspace*{-10mm}
        \scalebox{0.7}{
        \Fonction[%
            Calcul=3*(x**2),
            Trace,CouleurTrace=bleu,
            Xmin=0,Xmax=10,Xstep=2.5,
            Ymin=0,Ymax=5.8,Ystep=60,
            Origine={(1,.5)},
            Grille,PasGrilleX=1,PasGrilleY=0.5,
            Graduations,PasGradY=30,PasGradX=5,
            Bornea=0,Borneb=10,
            LabelX={\bf\footnotesize Temps (min)},
            LabelY={\bf\footnotesize Vitesse (en km/h)},
            Traces={
                draw placepoint(10,300)--placepoint(20,300) withcolor bleu;
            }            
            ]{}
        }

        \item Déterminer le temps, environ, que met le train pour atteindre \Vitesse{120}.
        
        {\red Le train met environ \Temps{;;;;7;} pour atteindre \Vitesse{120}.}
    \end{enumerate}
    \Coupe
    \begin{enumerate}
        \setcounter{enumi}{6}
        \item Déterminer la vitesse maximale du train.
        
        {\red La vitesse maximale est de \Vitesse{300}.}
        \item Déterminer au bout de combien de temps cette vitesse maximale est atteinte.
        
        {\red La vitesse maximale est atteinte après \Temps{;;;;10;}.}
        \item Préciser une expression de la fonction $v$\\pour $0 \leq x \leq 20$.
        
        {\red $v(x)=3x^2$ pour $0 \leq x \leq 10$
        
        $v(x)=30$ pour $10 \leq x \leq 20$}
    \end{enumerate}
    \vspace*{-9mm}
\end{corrige}
