\section{Tableau de valeurs}
\begin{definition}
    Un \textbf{tableau de valeurs} d'une fonction $f$ donne, sur la première ligne (ou colonne), différentes valeurs de la variable $x$ et, en vis-à-vis sur la deuxième ligne (ou colonne), les images $f(x)$ qui leur sont associées.
\end{definition}

\begin{remarques}
    \begin{itemize}
        \item Un tableau de valeurs n'est pas unique.
        \item Il dépend du choix des valeurs de $x$ sur la première ligne (ou colonne).
    \end{itemize}
\end{remarques}

\begin{exemple*1}
    Voici un \textbf{tableau de valeurs} de la fonction $g(x)=x^2+3$ entre -7 et 7.\\
    Avec un pas de 1, cela signifie que l'on ajoute 1 entre chaque valeur de x.\\
    Il serait très facile de faire un \textbf{tableau de valeurs} à l'aide d'un tableur.
        
    \Fonction[Calcul=x**2+3,Nom=g,Tableau]{-7,-6,-5,-4,-3,-2,-1,0,1,2,3,4,5,6,7}   
\end{exemple*1}