\begin{exercice}
    On considère le tableau ci-dessous.

    \begin{tabular}{|l|c|c|c|}
        \hline
        Distance (en km)&40&80&240\\
        \hline
        Tarif SNCF (en €)&6,50&13&39\\
        \hline
    \end{tabular}

    \smallskip
    Expliquer s'il y a proportionnalité ou non entre le tarif SNCF et la distance parcourue.
\end{exercice}
\begin{corrige}
    On considère le tableau ci-dessous.

    \begin{tabular}{|l|c|c|c|}
        \hline
        Distance (en km)&40&80&240\\
        \hline
        Tarif SNCF (en €)&6,50&13&39\\
        \hline
    \end{tabular}
    
    \smallskip
    Expliquer s'il y a proportionnalité ou non entre le tarif SNCF et la distance parcourue.
    
    {\red $\dfrac{40}{\num{6.50}}=\dfrac{40\times 2}{\num{6.50}\times 2}=\dfrac{80}{13}$ et $\dfrac{240}{39}=\dfrac{240\div 3}{39\div 3}=\dfrac{80}{13}$.

    On constate que $\dfrac{40}{\num{6.50}}=\dfrac{80}{13}=\dfrac{240}{39}$, donc le tarif SNCF est proportionnel à la distance parcourue.
    }
\end{corrige}