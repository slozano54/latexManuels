\section{Représentation graphique}
\begin{definition}
    La \textbf{représentation graphique} d'une fonction $f$ est la courbe constituée de l'ensemble des points, c'est à dire tous les points, de coordonnées $(x;f(x))$.    
\end{definition}

\begin{minipage}{0.6\linewidth}
    \begin{exemple*1}
        Tracer la courbe représentant la fonction $f$ définie par $f(x)=x^2$.
        \correction
        On établit un tableau de valeurs de la fonction $f$, on reporte les coordonnées des points dans un repère puis on les relie à la main.
    
        \smallskip
        \Fonction[Calcul=x**2,Tableau]{-3,-2,-1,0,1,2,3}    
    \end{exemple*1}        
\end{minipage}
\begin{minipage}{0.4\linewidth}
    \begin{center}
        \scalebox{0.77}{
            \Fonction[Trace,Calcul=x**2,Xmin=-3.25,Ymin=-0.5,Origine={(3.25,0.5)},Xmax=3.25,Ymax=9.5,%
                Bornea=-3.1,Borneb=3.1,PasGrilleX=0.5,PasGrilleY=0.5,LabelC=0.95,Grille,Graduations,NomCourbe=$f(x){=}x^2$,%
                CouleurTrace=Crimson]{}
        }
    \end{center}
\end{minipage}

\begin{minipage}{0.6\linewidth}
    \begin{exemple*1}
        Tracer la courbe représentant la fonction $h$ définie par $h(x)=x^3$.
        \correction
        On établit un tableau de valeurs de la fonction $h$, on reporte les coordonnées des points dans un repère puis on les relie à la main.
    
        \smallskip
        \Fonction[Calcul=x**3,Tableau]{-3,-2,-1,0,1,2,3}    
    \end{exemple*1}        
\end{minipage}
\begin{minipage}{0.4\linewidth}
    \begin{center}
        \scalebox{0.7}{
            \Fonction[Trace,Calcul=x**3,Xmin=-2.25,Ymin=-8.25,Origine={(2.25,8.25)},Xmax=2.25,Ymax=8.25,%
                Bornea=-2.1,Borneb=2.1,PasGrilleX=0.5,PasGrilleY=0.5,LabelC=0.95,Grille,Graduations,NomCourbe=$h(x){=}x^3$,%
                CouleurTrace=Crimson]{}
        }
    \end{center}
\end{minipage}

\begin{methode*1}[Lire une image à partir d'une courbe]
    \begin{itemize}
        \item On place $x$ sur l'axe des abscisses.
        \item On se déplace verticalement pour rencontrer $\mathcal{C}_f$.
        \item On lit $f(x)$ sur l'axe des ordonnées.
    \end{itemize}
    \exercice
    Lire l'image de $1$ par la fonction $f$.
    \correction \phantom{rrr}

    \begin{minipage}{0.6\linewidth}
        \vspace*{-20mm}
        \begin{itemize}
            \item On a placé $x=1$. 
            \item On se déplace verticalement pour rencontrer $\mathcal{C}_f$.
            \item On lit $f(1)=-2$.
        \end{itemize}
        \psshadowbox{L'image de $1$ par $f$ est $-2$.}   
    \end{minipage}
    \begin{minipage}{0.4\linewidth}
        \begin{center}
            \scalebox{0.8}{
                \Fonction[%
                    Calcul=(x-2)*(x-2)-3,
                    Trace,
                    Grille,PasGrilleX=1,PasGrilleY=1,
                    Origine={(1,3.25)},Graduations,
                    CouleurTrace=red,                
                    Bornea=-1,Borneb=4.5,
                    Xmin=-1,Xmax=4.8,
                    Ymin=-3.25,Ymax=2.8,                
                    Traces={%
                        drawarrow placepoint(1,0)--placepoint(1,-2) withcolor bleu;
                        drawarrow placepoint(1,-2)--placepoint(0,-2) withcolor bleu;
                    }
                ]{}
            }
         \end{center}
    \end{minipage}
\end{methode*1}


\begin{methode*1}[Lire des antécédents à partir d'une courbe]
    \begin{itemize}
        \item On trace une horizontale passant par cette valeur.
        \item À partir des points d'intersection, on se déplace verticalement vers l'axe des abscisses pour lire les antécédents.
    \end{itemize}
    \exercice
    Trouver les antécédents de $1$ par la fonction $f$.
    \correction \phantom{rrr}

    \begin{minipage}{0.6\linewidth}
        \vspace*{-20mm}
        \begin{itemize}
            \item On a tracé la dropite horizontale d'ordonnée $y=1$. 
            \item À partir des points d'intersection, on se déplace verticalement vers l'axe des abscisses pour lire les antécédents.
            \item On lit $x=0$ et $x=4$.
        \end{itemize}
        \psshadowbox{Les antécédents de $1$ par $f$ sont $0$ et $4$.}
    \end{minipage}
    \hfill
    \begin{minipage}{0.35\linewidth}
        \begin{center} 
            \scalebox{1}{
                \Fonction[%
                    Calcul=(x-2)*(x-2)-3,
                    Trace,
                    Grille,PasGrilleX=1,PasGrilleY=1,
                    Origine={(1,3.25)},Graduations,
                    CouleurTrace=red,                
                    Bornea=-1,Borneb=4.5,
                    Xmin=-1,Xmax=4.8,
                    Ymin=-3.25,Ymax=2.8,                
                    Traces={%
                        draw placepoint(-1,1)--placepoint(5,1) withcolor bleu;
                        drawarrow placepoint(0,1)--placepoint(0,0) withcolor bleu;
                        drawarrow placepoint(4,1)--placepoint(4,0) withcolor bleu;
                    }
                ]{}
            }
         \end{center}
    \end{minipage}
\end{methode*1}
