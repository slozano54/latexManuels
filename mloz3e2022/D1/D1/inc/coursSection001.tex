\section{Généralités}
\begin{definition}
    Une \textbf{fonction} est un procédé de calcul qui à un nombre associe un autre nombre.
\end{definition}

\begin{vocabulaireNotations}        
        \begin{itemize}
            \item $f$ est le nom de la fonction.
            \item $f(x)$, se lit "$f$ de $x$", est l'image de $x$ par la fonction $f$.
            \item $x$ est un antécédent de $f(x)$ par la fonction $f$.
            \item  $f : x\longmapsto f(x)$ se lit " $f$ qui à $x$ associe $f(x)$"
        \end{itemize}
\end{vocabulaireNotations}

\begin{remarques}
    \begin{enumerate}
        \item L'\textbf{image} d'un nombre est \textbf{unique}.
        \item Un nombre peut avoir \textbf{plusieurs antécédents}
    \end{enumerate}
\end{remarques}

\begin{exemple*1}
    L'égalité $f(-7)=5$, que l'on peut aussi écrire $f : -7\longmapsto 5$, se traduit en langage courant par l'une ou l'autre de ces deux phrases :
    \begin{itemize}
        \item Le nombre -7 est \textbf{un antécédent} du nombre 5 par la fonction $f$.
        \item Le nombre 5 est \textbf{l'image} du nombre -7 par la fonction $f$.
    \end{itemize}
\end{exemple*1}

\begin{exemple*1}
    Soit la fonction $g : x \longmapsto x^2 + 3$.
    
    Calculer les images de -7 et de 7 par cette fonction.
    \correction
    \begin{multicols}2
        \begin{list}{}
        \item $g(x)=x^2 + 3$
        \item $g(-7)=(-7)^2 + 3$
        \item $g(-7)=49 + 3$
        \item $g(-7)=52$ 
        \end{list}
        \begin{list}{}
        \item $g(x)=x^2 + 3$
        \item $g(7)=7^2 + 3$
        \item $g(7)=49 + 3$
        \item $g(7)=52$ 
        \end{list}
    \end{multicols}
    \begin{itemize}
        \item On remarque que les nombres $-7$ et $7$ ont \textbf{la même image} par la fonction $g$.\\
        Cette image est égale à 52.
        \item On peut dire que le nombre 52 admet, au moins, \textbf{deux antécédents} par la fonction $g$.\\
        Ces antécédents sont $-7$ et $7$.
    \end{itemize}
\end{exemple*1}