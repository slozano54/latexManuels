\begin{exercice}
    La \textbf{machine $\boldsymbol{f}$} renvoie le \textbf{périmètre} \footnote{\textbf{Rappel :} Le périmètre d'un polygone est égal à la somme des longueurs de ses côtés} d'un carré de côté $x$.
    \begin{center}
        \scalebox{0.7}{
            \machineMaths{$\boldsymbol{f}$}{Périmètre\\d'un carré}{carré de\\côté \Lg[cm]{54}}{Périmètre \\ ??? \Lg[cm]{}}
        }
    \end{center}
    \vspace*{-5mm} 
  \begin{enumerate}
    \item Que renvoie la machine si le côté vaut  \Lg[cm]{54} ? Formuler la réponse avec le mot \textbf{image}.
    \item Combien vaut le côté si la machine renvoie  \Lg[cm]{144} ? Formuler la réponse avec le mot \textbf{antécédent}.
    \item Quelle est l'image de $35$ par la \textbf{fonction $\boldsymbol{f}$ } ? Écrire la réponse sous la forme $\boldsymbol{f(35)=\ldots}$.
    \item Que renvoie la machine si le côté vaut $x$ cm ? Écrire la réponse sous la forme $\boldsymbol{f(x)=\ldots}$.
    \item Comme dans l'exemple ci-dessous, écrire le diagramme de la fonction $\boldsymbol{f}$.
        \begin{exemple*1}
            Voici le diagramme d'une machine qui triple.
            
            \medskip
            \begin{tikzpicture}[line cap=round,line join=round,>=triangle 45,x=1cm,y=1cm]
                \draw [line width=2pt,color=J1] (-10,0.5) -- (-9,0.5) -- (-9,-0.5) -- (-10,-0.5) -- cycle;
                \node[text width=3cm,text centered, scale=1] at(-9.5,0){$x$};            
                \draw [line width=2pt,color=J1] (-9,0) -- (-8.5,0);
                \draw [line width=2pt,color=J1] (-8,0) circle(0.5);
                \node [text width=3cm,text centered, scale=1] at(-8,0){$\times 3$};
                \draw [->,line width=2pt,color=J1] (-7.5,0) -- (-6.5,0);
                \draw [line width=2pt,color=J1] (-6.5,0.5) -- (-4,0.5) -- (-4,-0.5) -- (-6.5,-0.5) -- cycle;
                \node [text width=3cm,text centered, scale=1] at(-5.25,0){$t(x)=3x$};                    
            \end{tikzpicture}
        \end{exemple*1}

        \medskip
    \item Écrire maintenant la fonction $f$ en utilisant la forme $\boldsymbol{f:x} \longmapsto \ldots$
\end{enumerate}

\end{exercice}
\begin{corrige}
    La \textbf{machine $\boldsymbol{f}$} renvoie le \textbf{périmètre} d'un carré de côté $x$.
    
    \hspace*{-13mm}
    \scalebox{0.5}{
        \machineMaths{$\boldsymbol{f}$}{Périmètre\\d'un carré}{carré de\\côté \Lg[cm]{54}}{Périmètre \\ ??? \Lg[cm]{}}
    }

    
  \begin{enumerate}
    \item Que renvoie la machine si le côté vaut  \Lg[cm]{54} ? Formuler la réponse avec le mot \textbf{image}.
    
    {\red $4\times 54 = 216$ donc l'image de $54$ par la fonction $f$ vaut $216$.}
    \item Combien vaut le côté si la machine renvoie  \Lg[cm]{144} ? Formuler la réponse avec le mot \textbf{antécédent}.
    
    {\red $144\div 4 = 36$ donc un antécédent de $144$ par la fonction $f$ vaut $36$.}
    \item Quelle est l'image de $35$ par la \textbf{fonction $\boldsymbol{f}$ } ? Écrire la réponse sous la forme $\boldsymbol{f(35)=\ldots}$.
    
    {\red $4\times 35 = 140$ donc $f(35)=140$}
    \item Que renvoie la machine si le côté vaut $x$ cm ? Écrire la réponse sous la forme $\boldsymbol{f(x)=\ldots}$.
    
    {\red $4\times x = 4x$ donc $f(x)=4x$}
    \end{enumerate}
    \Coupe  
    \begin{enumerate}
    \setcounter{enumi}{4}
    \item Comme dans l'exemple ci-dessous, écrire le diagramme de la fonction $\boldsymbol{f}$.

    \medskip
    \scalebox{0.8}{
        \begin{tikzpicture}[line cap=round,line join=round,>=triangle 45,x=1cm,y=1cm]
            \draw [line width=2pt,color=J1] (-10,0.5) -- (-9,0.5) -- (-9,-0.5) -- (-10,-0.5) -- cycle;
            \node[text width=3cm,text centered, scale=1] at(-9.5,0){{\red $x$}};            
            \draw [line width=2pt,color=J1] (-9,0) -- (-8.5,0);
            \draw [line width=2pt,color=J1] (-8,0) circle(0.5);
            \node [text width=3cm,text centered, scale=1] at(-8,0){{\red $\times 4$}};
            \draw [->,line width=2pt,color=J1] (-7.5,0) -- (-6.5,0);
            \draw [line width=2pt,color=J1] (-6.5,0.5) -- (-4,0.5) -- (-4,-0.5) -- (-6.5,-0.5) -- cycle;
            \node [text width=3cm,text centered, scale=1] at(-5.25,0){{\red $f(x)=4x$}};                    
        \end{tikzpicture}
    }
    \item Écrire maintenant la fonction $f$ en utilisant la forme $\boldsymbol{f:x} \longmapsto \ldots$
    
    {\red $f:x \longmapsto 4x$}
\end{enumerate}
\vspace*{-3mm}
\end{corrige}
