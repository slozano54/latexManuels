\begin{exercice}
    Soit ce programme de calcul.

    \begin{minipage}{0.7\linewidth}
    \begin{myBox}{Programme de calcul}
        \begin{itemize}
            \item Choisir un nombre.
            \item Lui ajouter $5$.
            \item Multiplier cette somme par $3$.
            \item Soustraire $6$ à ce produit.
        \end{itemize}
    \end{myBox}    
    \end{minipage}

    \par\smallskip
    
    \begin{enumerate}
        \item Tester ce programme avec $2$.
        \item En notant $x$ le nombre choisi au départ, déterminer la fonction $g$ correspondant à ce programme.
        \item Déterminer $g(0)$.
        \item Déterminer le nombre de départ pour obtenir $18$.
    \end{enumerate}
\end{exercice}
\begin{corrige}
    Soit ce programme de calcul.
    \begin{minipage}{0.7\linewidth}
    \begin{myBox}{Programme de calcul}
        \noindent
        \begin{itemize}
            \item Choisir un nombre.
            \item Lui ajouter $5$.
            \item Multiplier cette somme par $3$.
            \item Soustraire $6$ à ce produit.
        \end{itemize}
    \end{myBox}
    \end{minipage}
    \begin{enumerate}
        \item Tester ce programme avec $2$.
        
        {\red \ProgCalcul{2,+5 *3 -6}}
        \item En notant $x$ le nombre choisi au départ, déterminer la fonction $g$ correspondant à ce programme.
        
        {\red \ProgCalcul[SansCalcul]{x,+5 *3 -6, x+5 (x+5)\times3 (x+5)\times3-6}
        
        $(x+5)\times3-6=3x+15-6$\\$\phantom{(x+5)\times3-6}=3x+9$\\ D'où $g(x)=3x+9$.
        }
        \item Déterminer $g(0)$.
        
        {\red $g(0)=3\times0+ 9$\\$g(0)=9$}
        \item Déterminer le nombre de départ pour obtenir $18$.
        
        {\red $g(x)=18$\\$3x+9=18$\\$3x=9$\\$x=3$}
    \end{enumerate}
\end{corrige}
