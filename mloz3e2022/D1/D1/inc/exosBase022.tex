\begin{exercice}
    Ce graphique représente une fonction $g$.
    
    Recopier et compléter les phrases.
    
    \Fonction[%
    Calcul=(x+1)*(x-3),
    Trace,CouleurTrace=rouge,
    Xmin=-3.25,Xmax=3.75,
    Ymin=-4.75,Ymax=1.5,Xstep=1,Ystep=1,
    Origine={(3.25,4.75)},
    Grille,PasGrilleX=0.25,PasGrilleY=0.25,
    Graduations,PasGradX=1,PasGradY=1,
    Bornea=-3,Borneb=4,
    LabelC=0.95,NomCourbe=$(C_g)$
    ]{}
    
    \begin{enumerate}
        \item L'image de $1$ par la fonction $g$ est \dots{}.
        \item Les antécédents de $0$ par la fonction $g$ sont \dots{}.
        \item $g(2)=\dots{}$
        \item Les nombres ayant pour image $-3$ par la fonction $g$ sont \dots{}.
    \end{enumerate}    
\end{exercice}
\begin{corrige}
 
\end{corrige}
