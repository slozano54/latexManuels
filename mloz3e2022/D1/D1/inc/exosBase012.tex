\begin{exercice}
    Soit la fonction $h:\longmapsto -\dfrac{2}{3}x$. Calculer.
    \begin{enumerate}        
        \item L'image de $7$.\smallskip
        \item $h\left(-\dfrac{5}{2}\right)$\smallskip
        \item L'antécédent de $1$.
        \item Le nombre qui a pour image $\dfrac{3}{4}$.
    \end{enumerate}
\end{exercice}
\begin{corrige}
    Soit la fonction $h:\longmapsto -\dfrac{2}{3}x$. Calculer.

    \begin{enumerate}        
        \item L'image de $7$.
        
        {\red $-\dfrac{2}{3}\times 7 = -\dfrac{14}{3}$}
        \item $h\left(-\dfrac{5}{2}\right)$ {\red = $-\dfrac{2}{3}\times \left(-\dfrac{5}{2}\right) = \dfrac{5}{3}$}        
        \item L'antécédent de $1$.
        
        {\red $-\dfrac{2}{3}\times x=1$ donc $x=-\dfrac{3}{2}$}
        \item Le nombre qui a pour image $\dfrac{3}{4}$.
        
        {\red $-\dfrac{2}{3}\times x = \dfrac{3}{4}$ donc $x=\dfrac{3}{4}\times \left(-\dfrac{3}{2}\right)$ soit $x=-\dfrac{9}{8}$}
    \end{enumerate}
\end{corrige}
