\begin{exercice}
    Soit la fonction $f$ définie par $f:x\longmapsto\dfrac{x+2}{x-1}$
    \begin{enumerate}
        \item Déterminer les valeurs de $x$ pour lesquelles $f$ n'est pas définie. Justifier.
        \item Calculer
        \begin{multicols}{2}
            \begin{itemize}
                \item $f(-2)$
                \item $f(-1)$
                \item $f(\num{-0.5})$
                \item $f(0)$
                \item $f(2)$
                \item $f(4)$
            \end{itemize}               
        \end{multicols}
        \item Déduire un antécédent par $f$ pour chaque nombre suivant.
        \begin{multicols}{2}
            \begin{itemize}
                \item $-2$
                \item $-1$
                \item $\num{-0.5}$
                \item $0$
                \item $2$
                \item $4$
            \end{itemize}               
        \end{multicols}
        \item Au regard des deux questions précédentes, faire une conjecture.
        \item Les plus inconscients pourront entamer une justification de cette conjecture.
    \end{enumerate}    
\end{exercice}
\begin{corrige}
    Soit la fonction $f$ définie par $f:x\longmapsto\dfrac{x+2}{x-1}$

    \begin{enumerate}
        \item Déterminer les valeurs de $x$ pour lesquelles $f$ n'est pas définie. Justifier.
        
        {\red La division par $0$ n'est pas définie donc $f$ n'est pas définie pour $x=1$.}
        \item Calculer
        \begin{multicols}{2}
            \begin{itemize}
                \item $f(-2)$ {\red $=0$ }
                \item $f(-1)$ {\red $=\num{-0.5}$ }
                \item $f(\num{-0.5})$ {\red $=-1$ }
                \item $f(0)$ {\red $=-2$}
                \item $f(2)$ {\red $=4$}
                \item $f(4)$ {\red $=2$}
            \end{itemize}               
        \end{multicols}
        \item Déduire un antécédent par $f$ pour chaque nombre suivant.
        \begin{multicols}{2}
            \begin{itemize}
                \item $-2$ {\red : $0$}
                \item $-1$ {\red : $\num{-0.5}$}
                \item $\num{-0.5}$ {\red : $-1$}
                \item $0$ {\red : $-2$}
                \item $2$ {\red : $4$}
                \item $4$ {\red : $2$}
            \end{itemize}               
        \end{multicols}
        \item Au regard des deux questions précédentes, faire une conjecture.
        
        {\red Il semble que pour tout nombre, l'image de l'image soit le nombre de départ.}
    \end{enumerate}
    \Coupe
    \begin{enumerate}
    \setcounter{enumi}{4}
        \item Les plus inconscients pourront entamer une justification de cette conjecture.
        
        {\red Par du calcul littéral/fractionnaire, à condition que $x\neq 1$ :
        
        $f(f(x)) = \dfrac{f(x)+2}{f(x)-1}$\\\bigskip
        $f(f(x)) = \dfrac{\dfrac{x+2}{x-1}+2}{\dfrac{x+2}{x-1}-1}$\\\bigskip
        $f(f(x)) = \dfrac{\dfrac{x+2+2x-2}{x-1}}{\dfrac{x+2-x+1}{x-1}}$\\\bigskip
        $f(f(x)) = \dfrac{\dfrac{3x}{x-1}}{\dfrac{3}{x-1}}$\\\bigskip
        $f(f(x)) = \dfrac{3x}{x-1}\times \dfrac{x-1}{3}$\\\bigskip
        On peut simplifier par $x-1$ puisque $x\neq1$\\\bigskip
        $f(f(x)) = \dfrac{3x}{3}$\\\bigskip
        On peut simplifier par $3$\\\bigskip
        $f(f(x))=x$
        }
    \end{enumerate}   
\end{corrige}
