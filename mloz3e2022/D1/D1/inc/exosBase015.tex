\begin{exercice}
    Soit une fonction définie par 
    \begin{myBox}{Programme de calcul}
        \begin{itemize}
            \item Choisir un nombre.
            \item Multiplier ce nombre par $6$.
            \item Ajouter $6$ au résultat obtenu.
        \end{itemize}
    \end{myBox}
    \begin{enumerate}
        \item 
        \begin{enumerate}
            \item Appliquer ce programme de calcul au nombre $8$.
            \item Traduire ce calcul par une phrase contenant \textit{image}.
        \end{enumerate}        
        \item Soit $f$ la fonction définie par $f(x)= 2x+9$.
        \begin{enumerate}
            \item Calculer l'image de $2$.
            \item Traduire ce calcul par une phrase contenant \textit{image}.
        \end{enumerate}
        \item Soit $g$ la fonction définie par $g:x\longmapsto 8x+5$.
        \begin{enumerate}
        \item Calculer l'image de $8$.
        \item Traduire ce calcul par une phrase contenant \textit{image}.
    \end{enumerate}
    
        \item Soit la fonction $h$ définie par le diagramme :
        
        \hspace*{-13mm}
        \scalebox{0.8}{
        \begin{tikzpicture}[line cap=round,line join=round,>=triangle 45,x=1cm,y=1cm]
            \draw [line width=2pt,color=J1] (-10,0.5) -- (-9,0.5) -- (-9,-0.5) -- (-10,-0.5) -- cycle;
            \node[text width=3cm,text centered, scale=1] at(-9.5,0){$x$};                
            \draw [line width=2pt,color=J1] (-9,0) -- (-8.5,0);
            \draw [line width=2pt,color=J1] (-8,0) circle(0.5);
            \node [text width=3cm,text centered, scale=1] at(-8,0){$\times 5$};
            \draw [->,line width=2pt,color=J1] (-7.5,0) -- (-6.5,0);
            \draw [line width=2pt,color=J1] (-6.5,0.5) -- (-5.5,0.5) -- (-5.5,-0.5) -- (-6.5,-0.5) -- cycle;
            \node [text width=3cm,text centered, scale=1] at(-6,0){$5x$};                
            \draw [line width=2pt,color=J1] (-5.5,0) -- (-5,0);
            \draw [line width=2pt,color=J1] (-4.5,0) circle(0.5);
            \node [text width=3cm,text centered, scale=1] at(-4.5,0){$+2$};
            \draw [->,line width=2pt,color=J1] (-4,0) -- (-3,0);
            \draw [line width=2pt,color=J1] (-3,0.5) -- (0,0.5) -- (0,-0.5) -- (-3,-0.5) -- cycle;
            \node [text width=3cm,text centered, scale=1] at(-1.5,0){$h(x)=5x+2$};                      
        \end{tikzpicture}
        }  
        \begin{enumerate}
            \item Calculer l'image de $9$.
            \item Traduire ce calcul par une phrase contenant \textit{image}.
        \end{enumerate}    
    \end{enumerate}    
\end{exercice}
\begin{corrige}
    \begin{enumerate}
        \item Avec ce programme de calcul :
        
        \begin{enumerate}
        \item 
        \setlength{\fboxrule}{0.5mm}
        \par\vspace{0.25cm}
        \noindent\fcolorbox{orange}{white}{\parbox{\linewidth-2\fboxrule-2\fboxsep}{
        \begin{itemize}
             \item On choisit le nombre 8
             \item On multiplie ce nombre par 6 : $6 \times 8 = 48$. 
             \item On ajoute 6 au résultat obtenu : $48+6=54$.
         \end{itemize}}}
        \par\vspace{0.25cm} 

        \item L'image de 8 par cette fonction vaut 54.\\
    On peut aussi dire que 54 est l'image de 8 par cette fonction.
    \end{enumerate}
    
        \item \begin{enumerate}
        \item Calculons l'image par $f$ de $x= 2$ :
    \\
    $f({\boldsymbol{x}})= 2 {\boldsymbol{x}}+9$
    \\
    $f({\boldsymbol{2}})= 2\times {\boldsymbol{2}}+9$
    \\
    $f({\boldsymbol{2}})= 4+9$
    \\
    $f({\boldsymbol{2}})= 13$
        \item L'image de 2 par la fonction $f$ vaut 13.
    \\
     On peut aussi dire que 13 est l'image de 2 par la fonction $f$.
    \end{enumerate}
    
        \item \begin{enumerate}
        \item Calculons l'image par $g$ de $x= 8$ :
    \\
    $g:{\boldsymbol{x}}\longmapsto 8 {\boldsymbol{x}}+5$
    \\
    $g:{\boldsymbol{8}}\longmapsto 8\times {\boldsymbol{8}}+5$
    \\
    $g:{\boldsymbol{8}}\longmapsto 64+5$
    \\
    $g:{\boldsymbol{8}}\longmapsto 69$
        \item L'image de 8 par la fonction $g$ vaut 69.
    \\
     On peut aussi dire que 69 est l'image de 8 par la fonction $g$.
    \end{enumerate}
    
        \item \begin{enumerate}
        \item Calculons l'image par $g$ de $x=$ 9 :\\
        \hspace*{-15mm}
        \scalebox{0.8}{
        \begin{tikzpicture}[line cap=round,line join=round,>=triangle 45,x=1cm,y=1cm]
            \draw [line width=2pt,color=J1] (-10,0.5) -- (-9,0.5) -- (-9,-0.5) -- (-10,-0.5) -- cycle;
            \node[text width=3cm,text centered, scale=1] at(-9.5,0){$9$};                
            \draw [line width=2pt,color=J1] (-9,0) -- (-8.5,0);
            \draw [line width=2pt,color=J1] (-8,0) circle(0.5);
            \node [text width=3cm,text centered, scale=1] at(-8,0){$\times 5$};
            \draw [->,line width=2pt,color=J1] (-7.5,0) -- (-6.5,0);
            \draw [line width=2pt,color=J1] (-6.5,0.5) -- (-5.5,0.5) -- (-5.5,-0.5) -- (-6.5,-0.5) -- cycle;
            \node [text width=3cm,text centered, scale=1] at(-6,0){$45$};                
            \draw [line width=2pt,color=J1] (-5.5,0) -- (-5,0);
            \draw [line width=2pt,color=J1] (-4.5,0) circle(0.5);
            \node [text width=3cm,text centered, scale=1] at(-4.5,0){$+2$};
            \draw [->,line width=2pt,color=J1] (-4,0) -- (-3,0);
            \draw [line width=2pt,color=J1] (-3,0.5) -- (0,0.5) -- (0,-0.5) -- (-3,-0.5) -- cycle;
            \node [text width=3cm,text centered, scale=1] at(-1.5,0){$h(9)=47$};                      
        \end{tikzpicture}
        } 
      
        \item L'image de 9 par la fonction $g$ vaut 47.\\
        On peut aussi dire que 47 est l'image de 9 par la fonction $g$.
    \end{enumerate}
    
    \end{enumerate}
    
    
\end{corrige}
