\begin{exercice}
    On considère la fonction $k$ qui à tout nombre \\associe son inverse.
    \begin{enumerate}
        \item Déterminer l'image de $3$.
        \item Déterminer le nombre qui a pour image $-5$.
        \item Déterminer le nombre qui a pour antécédent \num{-8.25}.
        \item Recopier et compléter : $k(\dots{})=16$ et $k(\dfrac{3}{2})=\dots{}$.
        \item Exprimer $k(x)$, en fonction de $x$.
    \end{enumerate}
\end{exercice}
\begin{corrige}
    On considère la fonction $k$ qui à tout nombre \\associe son inverse.

    \begin{enumerate}
        \item Déterminer l'image de $3$.
        
        \smallskip
        {\red $\dfrac{1}{3}$}
        \smallskip
        \item Déterminer le nombre qui a pour image $-5$.
        
        \smallskip
        {\red $\dfrac{1}{-5}=\dfrac{-1}{5}$}
        \smallskip
        \item Déterminer le nombre qui a pour \hbox{antécédent \num{-8.25}.}
        
        {\red $\dfrac{1}{\num{-8.25}}=\dfrac{-1}{\num{8.25}}=\dfrac{-4}{33}$}
        \item Recopier et compléter : $k({\red \dfrac{1}{16}})=16$ et $k(\dfrac{3}{2})={\red \dfrac{2}{3}}$.
        \item Exprimer $k(x)$, en fonction de $x$.
        
        {\red $k(x)=\dfrac{1}{x}$}
    \end{enumerate}
\end{corrige}
