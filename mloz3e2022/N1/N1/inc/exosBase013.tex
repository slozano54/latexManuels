\begin{exercice*}[Courrier]
    Jean Veu souhaite proposer sa candidature pour un emploi dans une entreprise.
    Il doit envoyer dans une seule enveloppe : $2$ copies de sa lettre de motivation
    et $2$ copies de son Curriculum Vit\ae (CV). Chaque copie est rédigée sur une feuille
    au format A4. Il souhaite faire partir son courrier en lettre prioritaire. Pour déterminer
    le prix du timbre, il obtient sur Internet la grille de tarif d'affranchissement suivante.

    \begin{tabularx}{0.48\textwidth}{|C|C|}
        \hline
        \rowcolor{gray!20}\multicolumn{2}{|>{\centering\hsize=\dimexpr2\hsize+2\tabcolsep+\arrayrulewidth\relax}X|}{LETTRE PRIORITAIRE}\tabularnewline        
        \hline
        \rowcolor{gray!20}MASSE JUSQU'À & TARIFS NETS \tabularnewline
        \hline
        \Masse{20}&\Prix{0.8} \tabularnewline
        \hline
        \Masse{100}&\Prix{1.6} \tabularnewline
        \hline
        \Masse{250}&\Prix{3.2} \tabularnewline
        \hline
        \Masse{500}&\Prix{4.8} \tabularnewline
        \hline
        \Masse[kg]{3}&\Prix{6.4} \tabularnewline
        \hline
    \end{tabularx}

    Afin de choisir le bon tarif d'affranchissement, il réunit les informations suivantes :
    \begin{itemize}
        \item Masse de son paquet de $50$ enveloppes : \Masse{175}.
        \item Dimensions d'une feuille A4 : \Lg{21} de largeur et \Lg{29.7} de longueur.
        \item Grammage d'une feuille A4 : $80$ g/m\up{$2$}.
    
        \begin{myBox}{Note}
            Le grammage est la masse par m\up{$2$} de feuille.
        \end{myBox} 
    \end{itemize}

    Quel tarif d'affranchissement doit-il choisir ?
\end{exercice*}
\begin{corrige}
    %\setcounter{partie}{0} % Pour s'assurer que le compteur de \partie est à zéro dans les corrigés
    \phantom{rrr}    
    \begin{multicols}2
        \begin{enumerate}
            \item .
            \item .
        \end{enumerate}
    \end{multicols}
\end{corrige}

