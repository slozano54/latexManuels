\begin{exercice*}[Gazon]
    Le jardinier d'un club de rugby décide de semer à nouveau du gazon sur l'aire de jeu.
    Pour que celui-ci pousse correctement, il installe un système d'arrosage automatique
    qui se déclenche le matin et le soir, à chaque fois, pendant $15$ minutes.
    \begin{itemize}
        \item Le système d'arrosage est constitué de $12$ circuits indépendants.
        \item Chaque circuit est composé de $4$ arroseurs.
        \item Chaque arroseur a un débit de \Vol[m]{0.4} d'eau par heure.
    \end{itemize}

    Combien de litres d'eau auront été consommés si on arrose le gazon pendant tout le mois de juillet ?

    \begin{myBox}{Rappels}
        \begin{itemize}
            \item \Vol[m]{1} = \Capa[l]{1000}
            \item Le mois de juillet compte $31$ jours.
        \end{itemize}
    \end{myBox}    

\end{exercice*}
\begin{corrige}
    %\setcounter{partie}{0} % Pour s'assurer que le compteur de \partie est à zéro dans les corrigés
    \phantom{rrr}    
    \begin{multicols}2
        \begin{enumerate}
            \item .
            \item .
        \end{enumerate}
    \end{multicols}
\end{corrige}

