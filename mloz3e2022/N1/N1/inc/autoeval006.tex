\begin{exercice}
    $(+12)-(-4)=$~\ldots
    \begin{ChoixQCM}{4}
        \item $(+8)$
        \item $(+16)$
        \item $(-16)$
        \item $(-8)$
    \end{ChoixQCM}
\end{exercice}
\begin{corrige}
    Réponse \reponseQCM{b}. En effet, soustraire un nombre revient à ajouter son opposé, donc soustraire $(-4)$ revient à ajouter $(+4)$.
    Donc $(+12)-(-4)=(+12)+(+4)=(+16)$
\end{corrige}