\begin{exercice*}[Salle d'étude]
    Dans une salle d'étude, la moitié des élèves font des mathématiques, le quart étudie de
    l'histoire, le septième de l’allemand et trois élèves font du dessin. Il n'y a pas d’autre élève que ceux
    dont l'activité est mentionnée.
    \begin{enumerate}
        \item Démontre qu'il y a $28$ élèves qui se trouvent dans la salle.
        \item Calcule le nombre d'élèves qui font des mathématiques, de l'histoire et de l'allemand.
    \end{enumerate}
\end{exercice*}
\begin{corrige}
    %\setcounter{partie}{0} % Pour s'assurer que le compteur de \partie est à zéro dans les corrigés
    \phantom{rrr}    
    \begin{multicols}2
        \begin{enumerate}
            \begin{spacing}2
            \item $\dfrac12 + \dfrac14 + \dfrac17 = \dfrac{14}{28}+\dfrac{7}{28}+\dfrac{4}{28}=\dfrac{25}{28}$
            
            $1 - \dfrac{25}{28} = \dfrac{3}{28}$
            
            Il reste donc $\dfrac{3}{28}$ des élèves qui font du dessin.

            Comme ils sont $3$, l'effectif de la classe est de $28$.
            \item Mathématiques : $\dfrac{1}{2}\times 28 = 14$
            
            Histoire : $\dfrac{1}{4}\times 28 = 7$

            Allemand : $\dfrac{1}{7}\times 28 = 4$

            Vérification : $14 + 7 + 4 + 3 = 28$
            \end{spacing}
        \end{enumerate}
    \end{multicols}
\end{corrige}

