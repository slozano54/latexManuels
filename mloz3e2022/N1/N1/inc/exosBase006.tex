\begin{exercice*}[Salle d'étude]
    Dans une salle d'étude, la moitié des élèves font des mathématiques, le quart étudie de
    l'histoire, le septième de l’allemand et trois élèves font du dessin. Il n'y a pas d’autre élève que ceux
    dont l'activité est mentionnée.
    \begin{enumerate}
        \item Démontre qu'il y a $28$ élèves qui se trouvent dans la salle.
        \item Calcule le nombre d'élèves qui font des mathématiques, de l'histoire et de l'allemand.
    \end{enumerate}
\end{exercice*}
\begin{corrige}
    %\setcounter{partie}{0} % Pour s'assurer que le compteur de \partie est à zéro dans les corrigés
    \phantom{rrr}    
    \begin{multicols}2
        \begin{enumerate}
            \item .
            \item .
        \end{enumerate}
    \end{multicols}
\end{corrige}

