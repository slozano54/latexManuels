% Les enigmes ne sont pas numérotées par défaut donc il faut ajouter manuellement la numérotation
% si on veut mettre plusieurs enigmes
\refstepcounter{exercice}
\numeroteEnigme
\begin{enigme}
    Comment jouer ?
    \begin{itemize}
        \item Deux joueurs.
        \item Chacun sa couleur.
        \item Chacun son tour, un joueur prend une position en inscrivant le resultat d'un calcul à l'intersection choisie.
        \item Pour gagner c'est comme au puissance 4 !
    \end{itemize}
    \begin{myBox}{\emoji{warning} \emoji{warning} \emoji{warning}}
        Les produits doivent être simplifiés !
    \end{myBox}

    \PQuatre[Autre,Couleur=.2Gray+.8White]{
        $\dfrac{1}{2}$/$\dfrac{2}{3}$/$\dfrac{4}{5}$/$\dfrac{6}{7}$/$\dfrac{10}{11}$/$\dfrac{12}{13}$/$\dfrac{16}{17}$,
        $\dfrac2{3}$/$\dfrac1{16}$/$\dfrac1{32}$/$\dfrac2{3}$/$\dfrac7{10}$/$\dfrac{13}{4}$/$\dfrac{17}{32}$
    }
\end{enigme}

% Pour le corrigé, il faut décrémenter le compteur, sinon il est incrémenté deux fois
\addtocounter{exercice}{-1}
\begin{corrige}
Pas de correction, c'est un jeu !
\end{corrige}