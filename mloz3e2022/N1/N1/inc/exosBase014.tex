\begin{exercice*}[Rebonds]
    On lache une balle rebondissante à une hauteur de \Lg[m]{1}. À chaque rebond,
    elle rebondit aux trois quarts de la hauteur d'où elle est tombée.

    Quelle est la hauteur de la balle au troisième rebond ?
\end{exercice*}
\begin{corrige}
    %\setcounter{partie}{0} % Pour s'assurer que le compteur de \partie est à zéro dans les corrigés
    % \phantom{rrr}    
    $\Lg[m]{1}\times\dfrac{3}{4}\times\dfrac{3}{4}\times\dfrac{3}{4} \simeq \Lg[m]{0.42}$

    La hauteur de la balle au troisième rebond sera de \Lg{42}.

\end{corrige}

