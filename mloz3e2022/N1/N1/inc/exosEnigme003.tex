% Les enigmes ne sont pas numérotées par défaut donc il faut ajouter manuellement la numérotation
% si on veut mettre plusieurs enigmes
\refstepcounter{exercice}
\numeroteEnigme
\begin{enigme}
    Comment jouer ?
    \begin{itemize}
        \item Deux joueurs
        \item Chacun sa couleur
        \item Chacun son tour, un joueur prend une position en inscrivant le resultat d'un calcul à l'intersection choisie.
        \item Pour gagner c'est comme au puissance 4 !
    \end{itemize}
    
    \PQuatre[Addition,Autre,Couleur=.2Gray+.8White]{
        $\dfrac1{2}$/$\dfrac1{4}$/$\dfrac1{8}$/$\dfrac1{16}$/$\dfrac1{32}$/$\dfrac1{64}$/$\dfrac1{128}$,
        $\dfrac1{128}$/$\dfrac1{64}$/$\dfrac1{32}$/$\dfrac1{16}$/$\dfrac1{8}$/$\dfrac1{4}$/$\dfrac1{2}$
    }
\end{enigme}

% Pour le corrigé, il faut décrémenter le compteur, sinon il est incrémenté deux fois
\addtocounter{exercice}{-1}
\begin{corrige}
Pas de correction, c'est un jeu !
\end{corrige}