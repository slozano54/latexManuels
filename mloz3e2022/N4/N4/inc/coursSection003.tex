\section{Calcul littéral : Techniques de factorisation}
\subsection{Avec un facteur commun}
\begin{propriete}[\admise]
    Si $a,b,k$  sont trois expressions quelconques alors 
    \begin{align*}
        k\times a+k\times b&=k\times(a+b)\\
        k\times a-k\times b&=k\times(a-b)
    \end{align*}
\end{propriete}

\begin{exemples*1}
    Factoriser les expressions $A(x)=(x+2)(2x-1)+(x+2)x$ et $B(x)=(4x+3)(x-1)-(4x-2)(x-1)$. 
    \begin{align*}
        A(x)&=\underline{(x+2)}(2x-1)+\underline{(x+2)}x\qquad&B(x)&=(4x+3)\underline{(x-1)}-(4x-2)\underline{(x-1)}\\
        A(x)&=(x+2)\times(2x-1+x)                             &B(x)&=(x-1)(4x+3-(4x-2))\\
        A(x)&=(x+2)\times(3x-1)                               &B(x)&=(x-1)(4x+3-4x+2)\\
         &                                                    &B(x)&=(x-1)\times5\\
         &                                                    &B(x)&=5(x-1)         
    \end{align*}
\end{exemples*1}

\subsection{Avec une égalité remarquable}
\begin{propriete}[\admise]
    Si $a,b$ sont deux expresions quelconques alors    
    \begin{itemize}        
        \item $a^2+2\times a\times b+b^2 =(a+b)^2$
        
        \smallskip
        \item $a^2-2\times a\times b+b^2 =(a-b)^2$
        
        \smallskip
        \item $a^2-b^2                   =(a+b)\times(a-b)$
    \end{itemize}
\end{propriete}

\begin{exemples*1}
    Factoriser les expressions $A(x)=x^2+16x+64$, $B(x)=4x^2-4x+1$ et $C(x)=x^2-25$. 
    \begin{align*}
        A(x)&=x^2+16x+64&B(x)&=x^2-4x+1 &C(x)&=x^2-25\\
        A(x)&=x^2+2\times x\times8+8^2&B(x)&=(2x)^2-2\times2x\times1+1^2&C(x)&=x^2-5^2\\
        A(x)&=(x+8)^2&B(x)&=(2x-1)^2&C(x)&=(x+5)(x-5)
    \end{align*}
\end{exemples*1}
% \Exemples{Factoriser}{
% $$\Eqalign{
% A&=x^2+16x+64\qquad&B&=4x^2-4x+1\qquad&C&=x^2-25\cr
% A&=x^2+2\times x\times8+8^2&B&=(2x)^2-2\times2x\times1+1^2&C&=x^2-5^2\cr
% A&=(x+8)^2&B&=(2x-1)^2&C&=(x+5)(x-5)\cr
% }$$
% \par\vspace{0.25cm}

% }