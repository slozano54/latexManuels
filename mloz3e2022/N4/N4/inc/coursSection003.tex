\section{Calcul litt\'eral : Techniques de factorisation}
\subsection{Avec un facteur commun}
% \proprNum{(admise)}{$a,b,k$  sont trois expressions quelconques}{
% $$\Eqalign{
% k\times a+k\times b&=k\times(a+b)\cr
% k\times a-k\times b&=k\times(a-b)\cr
% }$$
% }

% \Exemples{Factoriser}{
% $$\Eqalign{
% A&=\underline{(x+2)}(2x-1)+\underline{(x+2)}x\qquad&B&=(4x+3)\underline{(x-1)}-(4x-2)\underline{(x-1)}\cr
% A&=(x+2)\times(2x-1+x)&B&=(x-1)(4x+3-(4x-2))\cr
% A&=(x+2)\times(3x-1)&B&=(x-1)(4x+3-4x+2)\cr
% &&B&=(x-1)\times5=5(x-1)\cr
% }$$
% \par\vspace{0.25cm}
% $A(x)=5x+3x$\par
% $B(x,y)=8x-8y$\par
% $C(t)=3t^2+5t$\par
% $D(u,v)=25u-35v$\par
% $E(x)=4x^2+x$\par
% $F(z)=9z^2-6z$
% }

\subsection{Avec une égalité remarquable}
% \proprNum{(admise)}{$a,b$ sont deux expresions quelconques}{
% $$\Eqalign{
% a^2+2\times a\times b+b^2&=(a+b)^2\cr
% \cr
% a^2-2\times a\times b+b^2&=(a-b)^2\cr
% \cr
% a^2-b^2&=(a+b)\times(a-b)\cr
% }$$
% }

% \Exemples{Factoriser}{
% $$\Eqalign{
% A&=x^2+16x+64\qquad&B&=4x^2-4x+1\qquad&C&=x^2-25\cr
% A&=x^2+2\times x\times8+8^2&B&=(2x)^2-2\times2x\times1+1^2&C&=x^2-5^2\cr
% A&=(x+8)^2&B&=(2x-1)^2&C&=(x+5)(x-5)\cr
% }$$
% \par\vspace{0.25cm}
% $A(z)=z^2-81$\par
% $B(t)=36-4t^2$\par
% $C(x)=(x+1)^2-9$\par
% $D(y)=4y^2-12y+9$\par
% $E(z)=28z+4+49z^2$\par
% $F(u)=4u^2+3u+\dfrac{9}{16}$\par
% $G(v)=v^2-4v+4$\par
% $H(r)=(5r+3)(r-2)+(r^2-4r+4)$
% }