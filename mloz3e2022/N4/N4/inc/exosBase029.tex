\begin{exercice*}
    Melisa pense qu'en multipliant deux nombres impairs consécutifs\footnote{c'est à dire qui se suivent} et en ajoutant $1$, le résultat obtenu est toujours un multiple de $4$.
    \begin{enumerate}
        \item {\bfseries Étude d'un exemple : }
        
        $5$ et $7$ sont deux nombres impairs consécutifs.
        \begin{enumerate}
            \item Calculer $5\times 7 + 1$.
            \item Déterminer si Melisa a raison sur cet exemple.
        \end{enumerate}
        \item \tableurLogo Le tableau ci-dessous montre le travail de Melisa dans une feuille de calcul.
        
        \smallskip
        \scalebox{0.6}{
        \begin{Tableur}[Bandeau=false,Colonnes=5,Largeur=80pt]
            &Nombre impair&Nombre impair suivant&Produit&Résultat obtenu\\
            $x$&$2x+1$&$2x+3$&$(2x+1)(2x+3)$&$(2x+1)(2x+3)+1$\\
            $0$&$1$&$3$&$3$&$4$\\
            $1$&$3$&$5$&$15$&$16$\\
            $2$&$5$&$7$&$35$&$36$\\
            $3$&$7$&$9$&$63$&$64$\\
            $4$&$9 $&$11$&$99 $&$100$\\
            $5$&$11$&$13$&$143$&$144$\\
            $6$&$13$&$15$&$195$&$196$\\
            $7$&$15$&$17$&$255$&$256$\\
            $8$&$17$&$19$&$323$&$324$\\
            $9$&$19$&$21$&$399$&$400$\\
        \end{Tableur}       
        }
        \smallskip
        \begin{enumerate}
            \item Déterminer le résultat obtenu en prenant comme premier nombre impair $17$.
            \item Montrer que le résultat de la question précédente est un multiple de $4$.
            \item Recopier les deux formules que l'on peut saisir dans la cellule D3.\\
            \textit{Auccune justification n'est attendue.}
            \begin{description}
                \item[Formule 1 : ] \fbox{$=(2*A3+1)*(2*A3+3)$}
                \item[Formule 2 : ] \fbox{$=(2*B3+1)*(2*C3+3)$}
                \item[Formule 3 : ] \fbox{$=B3*C3$}
                \item[Formule 4 : ] \fbox{$=(2*D3+1)*(2*D3+3)$}
            \end{description}
        \end{enumerate}
        \item {\bfseries Étude algébrique : }
        \begin{enumerate}
            \item Développer et réduire $(2x+1)(2x+3)+1$.
            \item Démontrer que Melisa avait raison.
        \end{enumerate}
    \end{enumerate}
\end{exercice*}
\begin{corrige}
    %\setcounter{partie}{0} % Pour s'assurer que le compteur de \partie est à zéro dans les corrigés
    %\phantom{rrr}    
    Melisa pense qu'en multipliant deux nombres impairs consécutifs et en ajoutant $1$, le résultat obtenu est toujours un multiple de $4$.

    \begin{enumerate}
        \item {\bfseries Étude d'un exemple : }
        
        $5$ et $7$ sont deux nombres impairs consécutifs.
        \begin{enumerate}
            \item Calculer $5\times 7 + 1$.\\
            {\red $5\times 7 + 1 = 35+1$\\$5\times 7 + 1 = 36$}\\
            \item Déterminer si Melisa a raison sur cet exemple.\\
            {\red $36 = 9\times 4$, donc Melisa a raison sur cet exemple.}\\
        \end{enumerate}
        \setcounter{enumi}{1}
        \item \tableurLogo Le tableau ci-dessous montre le travail de Melisa dans une feuille de calcul.
        
        \smallskip
        \hspace*{-10mm}
        \scalebox{0.55}{
        \begin{Tableur}[Bandeau=false,Colonnes=5,Largeur=80pt]
            &Nombre impair&Nombre impair suivant&Produit&Résultat obtenu\\
            $x$&$2x+1$&$2x+3$&$(2x+1)(2x+3)$&$(2x+1)(2x+3)+1$\\
            $0$&$1$&$3$&$3$&$4$\\
            $1$&$3$&$5$&$15$&$16$\\
            $2$&$5$&$7$&$35$&$36$\\
            $3$&$7$&$9$&$63$&$64$\\
            $4$&$9 $&$11$&$99 $&$100$\\
            $5$&$11$&$13$&$143$&$144$\\
            $6$&$13$&$15$&$195$&$196$\\
            $7$&$15$&$17$&$255$&$256$\\
            $8$&$17$&$19$&$323$&$324$\\
            $9$&$19$&$21$&$399$&$400$\\
        \end{Tableur}       
        }
        \smallskip
        \begin{enumerate}
            \item Déterminer le résultat obtenu en prenant comme premier nombre impair $17$.\\
            {\red Avec $17$, on obtient : \\
            $(2\times 17 + 1)\times (2\times 17 + 3)+1=35\times 37+1$\\
            $(2\times 17 + 1)\times (2\times 17 + 3)+1=\num{1295}+1$\\
            $(2\times 17 + 1)\times (2\times 17 + 3)+1=\num{1296}$\\
            }
            \item Montrer que le résultat de la question précédente est un multiple de $4$.\\
            {\red $\num{1296}=324\times 4$, c'est donc bien un multiple de $4$.}\\
        \end{enumerate}
    \end{enumerate}
    \Coupe
    \begin{enumerate}        
        \begin{enumerate}
            \setcounter{enumii}{2}
            \item Recopier les deux formules que l'on peut saisir dans la cellule D3.\\
            \textit{Auccune justification n'est attendue.}\\            
            {\bfseries Formule 1 : } \fbox{$=(2*A3+1)*(2*A3+3)$}\\
            {\bfseries Formule 2 : } \fbox{$=(2*B3+1)*(2*C3+3)$}\\
            {\bfseries Formule 3 : } \fbox{$=B3*C3$}\\
            {\bfseries Formule 4 : } \fbox{$=(2*D3+1)*(2*D3+3)$}\\
            {\red Les formules 1 et 3 peuvent être saisies en D3.\\
            {\bfseries Formule 1 : } \fbox{$=(2*A3+1)*(2*A3+3)$}\\
            {\bfseries Formule 3 : } \fbox{$=B3*C3$}\\            
            }            
        \end{enumerate}
        \setcounter{enumi}{2}
        \item {\bfseries Étude algébrique : }\\
        \begin{enumerate}
            \item Développer et réduire $(2x+1)(2x+3)+1$.\\
            {\red $(2x+1)(2x+3)+1 = 4x^2+6x+2x+3+1$.\\
            $(2x+1)(2x+3)+1 = 4x^2+8x+4$.\\
            }
            \item Démontrer que Melisa avait raison.\\
            {\red Si on choisit $x$ au départ, les résultat final est $4x^2+8x+4$, or $4x^2+8x+4=(x^2+2x+1)\times 4$ qui est bien un multiple de $4$.}
            
        \end{enumerate}
    \end{enumerate}
\end{corrige}

