\begin{exercice*}
    Melisa pense qu'en multipliant deux nombres impairs consécutifs\footnote{c'est à dire qui se suivent} et en ajoutant $1$, le résultat obtenu est toujours un multiple de $4$.
    \begin{enumerate}
        \item {\bfseries Étude d'un exemple : }
        
        $5$ et $7$ sont deux nombres impairs consécutifs.
        \begin{enumerate}
            \item Calculer $5\times 7 + 1$.
            \item Déterminer si Melisa a raison sur cet exemple.
        \end{enumerate}
        \item \tableurLogo Le tableau ci-dessous montre le travail de Melisa dans une feuille de calcul.
        
        \smallskip
        \scalebox{0.6}{
        \begin{Tableur}[Bandeau=false,Colonnes=5,Largeur=80pt]
            &Nombre impair&Nombre impair suivant&Produit&Résultat obtenu\\
            $x$&$2x+1$&$2x+3$&$(2x+1)(2x+3)$&$(2x+1)(2x+3)+1$\\
            $0$&$1$&$3$&$3$&$4$\\
            $1$&$3$&$5$&$15$&$16$\\
            $2$&$5$&$7$&$35$&$36$\\
            $3$&$7$&$9$&$63$&$64$\\
            $4$&$9 $&$11$&$99 $&$100$\\
            $5$&$11$&$13$&$143$&$144$\\
            $6$&$13$&$15$&$195$&$196$\\
            $7$&$15$&$17$&$255$&$256$\\
            $8$&$17$&$19$&$323$&$324$\\
            $9$&$19$&$21$&$399$&$400$\\
        \end{Tableur}       
        }
        \smallskip
        \begin{enumerate}
            \item Déterminer le résultat obtenu en prenant comme premier nombre impair $17$.
            \item Montrer que le résultat précédetn est un multiple de $4$.
            \item Recopier les deux formules que l'on peut saisir dans la cellule D3.\\
            \textit{Auccune justification n'est attendue.}
            \begin{description}
                \item[Formule 1 : ] \fbox{$=(2*A3+1)*(2*A3+3)$}
                \item[Formule 2 : ] \fbox{$=(2*B3+1)*(2*C3+3)$}
                \item[Formule 3 : ] \fbox{$=B3*C3$}
                \item[Formule 4 : ] \fbox{$=(2*D3+1)*(2*D3+3)$}
            \end{description}
        \end{enumerate}
        \item {\bfseries Étude algébrique : }
        \begin{enumerate}
            \item Développer et réduire $(2x+1)(2x+3)+1$.
            \item Démontrer que Melisa avait raison.
        \end{enumerate}
    \end{enumerate}
\end{exercice*}
\begin{corrige}
    %\setcounter{partie}{0} % Pour s'assurer que le compteur de \partie est à zéro dans les corrigés
    %\phantom{rrr}    
    \dots
\end{corrige}

