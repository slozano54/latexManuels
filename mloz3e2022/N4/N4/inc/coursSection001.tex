\section{Vocabulaire}

% \definNum{On appelle SOMME une suite d'opérations dans laquelle \underline{la} dernière opération est une addition ou une soustraction.
% }

% \CadreLampe{Remarque}{
% Une soustraction peut être ramenée à une addition. Soustraire un nombre revenant à ajouter l'opposé de ce nombre.
% }

% \Exemples{}{
% \begin{itemize}
% \item[]$A= 2+x$ est la somme des deux termes $2$ et $x$;
% \item[]$B=\dfrac{2}{7}+\dfrac{t}{4}-\dfrac{8}{11}$ est la somme des trois termes $\dfrac{2}{7}$, $\dfrac{t}{4}$ et $-\dfrac{8}{11}$;
% \item[]$C=z\times 7,1 + 4\div 8$ est la somme des deux termes $z\times 7,1$ et $4\div 8$;
% \item[]$D=2\times (y + 2\times 3) + 8\div (5 - 7\times t) - 8\times \dfrac{7}{2}$ est la somme des trois termes $2\times (y + 2\times 3)$, $8\div (5 - 7\times t)$ et $8\times \dfrac{7}{2}$.
% \end{itemize}
% }

% \definNum{On appelle PRODUIT une suite d'opérations dans laquelle la dernière opération est une multiplication.
% }

% \Exemples{}{
% \begin{itemize}
% \item[] $A=7\times 5,1$ est le produit des deux facteurs $7$ et $5,1$;
% \item[] $B=8\times \dfrac{7}{3} \times (-5,7)$ est le produit des trois facteurs $8$, $\dfrac{7}{3}$ et$(-5,7)$;
% \item[] $C=2(x+7)$ est le produit des deux facteurs $2$ et $(x+7)$;
% \item[] $D=7y\times (x-8)\times (2x+3)$ est le produit des trois facteurs $7y\times$, $(x-8)$ et $(2x+3)$;
% \item[] $E=(y-9)^{2}$ est le produit des deux facteurs $(y-9)$ et $(y-9)$ !
% \end{itemize}
% }

% \definNum{\textbf{\underline{Développer}} ( une expression ) c'est transformer les produits en sommes.}

% \definNum{\textbf{\underline{Factoriser}} ( une expression ) c'est transformer une somme ou une différence en produit.}

% \definNum{\textbf{\underline{Réduire}} ( une expression ) c'est rassembler les termes de même "statut" dans les sommes.}

% \Exemples[Exemple]{}{
% $-3x^{2}+3x-7+8x^{2}+9-4x=\textcolor{red}{-3x^{2}}\textcolor{mygreen}{+3x}-7\textcolor{red}{+8x^{2}}+9\textcolor{mygreen}{-4x}$\par
% $-3x^{2}+3x-7+8x^{2}+9-4x=\textcolor{red}{-3x^{2}} \textcolor{red}{+8x^{2}} \textcolor{mygreen}{+3x} \textcolor{mygreen}{-4x} -7+9$\par
% $-3x^{2}+3x-7+8x^{2}+9-4x=\underbrace{\textcolor{red}{5x^{2}}}_{\textcolor{red}{terme\;en\;x^2}}\quad \underbrace{\textcolor{mygreen}{-x}}_{\textcolor{mygreen}{terme\;en\;x}} \quad \underbrace{+ 2}_{terme\;constant}$
% }