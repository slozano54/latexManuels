\begin{exercice*}
    Voici un programme de calcul :
    \myProgCalcul{$\leadsto$}{Programme de calcul}{%
        \ProgCalcul[Enonce,ThemePerso]{%
            Choisir un nombre.,
            Calculer son double.,
            Soustraire $1$.,
            Calculer le carré du résultat.,
            Soustraire $64$.,
        }
    }
    \begin{enumerate}
        \item Montrer que si on choisit $4$, on obtient $-15$.
        \item En appelant $x$ le nombre de départ, écrire un expression traduisant ce programme.
        \item Factoriser $G = (2x-1)^2-64$.
        \item Pour chacune des valeurs de $x$ suivantes, calculer la valeur de $G$ en précisant la forme de l'expression $G$ de la question précédente utilisée.
        \begin{multicols}{2}
            \begin{itemize}
                \item $x=0$
                \item $x=\dfrac12$
                \item $x=-\dfrac72$
                \item $x=\dfrac92$
            \end{itemize}
        \end{multicols}
    \end{enumerate}
\end{exercice*}
\begin{corrige}
    %\setcounter{partie}{0} % Pour s'assurer que le compteur de \partie est à zéro dans les corrigés
    %\phantom{rrr}    
    Voici un programme de calcul :
    \myProgCalcul{$\leadsto$}{Programme de calcul}{%
        \ProgCalcul[Enonce,ThemePerso]{%
            Choisir un nombre.,
            Calculer son double.,
            Soustraire $1$.,
            Calculer le carré du résultat.,
            Soustraire $64$.,
        }
    }
    \begin{enumerate}
        \item Montrer que si on choisit $4$, on obtient $-15$.\\        
        \hspace*{-5mm}{\red $4 \leadsto 2\times 4=8 \leadsto 8-1=7 \leadsto 7^2=49 \leadsto 49-64=-15$}\\
        \item En appelant $x$ le nombre de départ, écrire un expression traduisant ce programme.\\
        {\red $x \leadsto 2\times x \leadsto 2x-1 \leadsto (2x-1)^2 \leadsto (2x-1)^2-64$}\\
        \item Factoriser $G = (2x-1)^2-64$.\\
        {\red $G=(2x-1)^2 - 8^2$\\$G=(2x-1+8)(2x-1-8)$\\$G=(2x+7)(2x-9)$}\\
        \item Pour chacune des valeurs de $x$ suivantes, calculer la valeur de $G$ en précisant la forme de l'expression $G$ de la question précédente utilisée.
        \begin{itemize}
            \item $x=0$\\
            {\red Avec $G=(2x+7)(2x-9)$, $G(0)=7\times (-9)=-63$}\\
        \end{itemize}
    \end{enumerate}
    \Coupe
    \begin{enumerate}        
        \begin{itemize}
            \item $x=\dfrac12$\\
            {\red Avec $G=(2x-1)^2-64$, $G(\dfrac12)=0-64=-64$}\\
            \item $x=-\dfrac72$\\
            {\red Avec $G=(2x+7)(2x-9)$, $G(-\dfrac72)=0\times (2\times(-\dfrac72) -9))=0$}\\
            \item $x=\dfrac92$\\
            {\red Avec $G=(2x+7)(2x-9)$, $G(\dfrac92)=(2\times \dfrac92 + 7)\times 0=0$}\\            
        \end{itemize}
    \end{enumerate}
    \vspace*{-20mm}
\end{corrige}

