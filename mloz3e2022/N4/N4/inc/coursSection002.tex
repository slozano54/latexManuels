\section{Calcul littéral : Techniques de développement}
\subsection{Distributivité simple}

Penser à mettre opposé d'une somme en cas particulier.
\begin{propriete}[\admise]
    Si $a$, $b$ et $k$ sont trois nombres (connus ou inconnus) alors$$k\times(a+b)=k\times a+k\times b$$
\end{propriete}

\begin{remarque}
    \phantom{rrr}\\
    $k\times(a-b)=k\times(a+(-b))$\\
    $k\times(a-b)=k\times a+k\times (-b)$\\
    $k\times(a-b)=ka-kb$
\end{remarque}

\begin{propriete}[Cas particulier de distributivité]
    {\bfseries L'opposé d'une somme algébrique} est égal à la somme des {\bfseries opposés de chacun de ses termes.}
\end{propriete}

\begin{preuve}
    L'opposé d'une somme algébrique est égale au produit de cette somme par $-1$.\\
    La distribution de $-1$ sur chacun des termes de cette somme le transformera en son opposé.
\end{preuve}

\subsection{Double distributivité}
\begin{propriete}[\admise]
    Si $a$, $b$, $c$ et $d$ sont 4 nombres (connus ou inconnus) alors $$ \Distri[Cours,Fleches]{2}{3}{4}{5}$$ 
\end{propriete}

\begin{preuve}
    à proposer à M. LOZANO sur la base du volontariat : Note bonus à la clef \emoji{shushing-face}
\end{preuve}

\subsection{Egalités remarquables}
\begin{propriete}[Carré d'une somme de deux termes]
    \begin{itemize}
    \item \underline{En français} : 
    \og{}Le carré de la somme de deux termes est égal à la somme des carrés des deux termes {\bfseries augmentée} de leur double produit.\fg{}

    \bigskip
    \item \underline{En \og{}langue\fg{} mathématique} :
    Si $a$ et $b$ sont des nombres (connus ou inconnus) alors 
    
    $$\underbrace{\parbox{0.1\linewidth}{\centering $(a+b)^2$}}_{\parbox{0.3\linewidth}{\centering\vspace*{1mm}Carré d'une\\somme}}
    =\underbrace{\parbox{0.1\linewidth}{\centering $a^2$}}_{\parbox{0.1\linewidth}{\centering\vspace*{1mm}carré du\\1\ier terme}}
    +\underbrace{\parbox{0.1\linewidth}{\centering $b^2$}}_{\parbox{0.1\linewidth}{\centering\vspace*{1mm}carré du\\2\ieme terme}}
    +\underbrace{\centering 2\times a\times b}_{\parbox{0.3\linewidth}{\centering\vspace*{1mm}double produit\\( nos deux abbés !)}}$$    
    \end{itemize}
\end{propriete}

\begin{preuve}
    \phantom{rrr}\\
    $(a+b)^2=(a+b)(a+b)$\\
    $(a+b)^2=a\times a+a\times b+b\times a+b\times b$\\
    $(a+b)^2=a^2+ab+ba+b^2$\\
    $(a+b)^2=a^2+b^2+2ab$ $\square$
\end{preuve}

% \Exemples{}{
% \begin{enumerate}
% \item Développer et réduire : $A(x)=(2x+7)^2$, $B(x)=\left(x+\dfrac{1}{2}\right)^2$ et $C(x)=\left(\dfrac{4}{5}+\dfrac{3x}{2}\right)^2$.
% \item Calculer mentalement : $101^2$ et $73^2$
% \end{enumerate}
% }

% \newpage
% \proprNumBis{\bf Carré d'une différence de deux termes}{
% \par\vspace{0.5cm}
% \underline{En français} : " Le carré de la différence de deux termes est égal à la somme des carrés des deux termes DIMINU\'{E}E de leur double produit"
% \par\vspace{0.5cm}
% \underline{En "langue" mathématique} :\par 
% Si \colorbox{green!30}{\parbox{\textwidth-25\fboxsep}{$a$ et $b$ sont des nombres (connus ou inconnus)}} 
% \\alors \colorbox{red!30}{\parbox{\textwidth-25\fboxsep}
% {$$(a-b)^2=\underbrace{a^2}_{\begin{tabular}{c}
% carré du\\
% 1\ier terme
% \end{tabular}}
% +\underbrace{b^2}_{\begin{tabular}{c}
% carré du\\
% 2\ieme terme
% \end{tabular}}
% -\underbrace{2\times a\times b}_{\mbox{double produit ( nos deux abbés !)}}$$
% }}}
% \CadreLampe{Démonstration}{
% Avec la double distributivité ou à l'aide de la propriété précédente.
% $(a-b)^2=(a+(-b))^2$\par
% $(a-b)^2=a^2+(-b)^2+2a\times(-b)$\par
% $(a-b)^2=a^2+b^2-2ab$ $\square$
% }

% \Exemples{}{
% \begin{enumerate}
% \item Développer et réduire : $A(x)=(x-8)^2$, $B(x)=(5x-7)^2$ et $C(y)=\left(\dfrac{2}{3}+\dfrac{4}{7}y\right)^2$.
% \item Calculer mentalement : $99^2$ et $18^2$
% \end{enumerate}
% }

% \proprNumBis{\bf Produit de la somme de deux termes par leur différence}{
% \par\vspace{0.5cm}
% \underline{En français} : " Le produit de la somme de deux nombres par leur différence est égal à la différence de leurs carrés "
% \par\vspace{0.5cm}
% \underline{En "langue" mathématique} :\par 
% Si \colorbox{green!30}{\parbox{\textwidth-25\fboxsep}{$a$ et $b$ sont des nombres (connus ou inconnus)}}
% \\
% alors \colorbox{red!30}{\parbox{\textwidth-25\fboxsep}
% {$$(a+b)(a-b)=\underbrace{a^2}_{\begin{tabular}{c}
% carré du\\
% 1\ier terme
% \end{tabular}}-\underbrace{b^2}_{\begin{tabular}{c}
% carré du\\
% 2\ieme terme
% \end{tabular}}$$
% }}}

% \CadreLampe{Démonstration}{
% Avec la double distributivité.
% $(a+b)(a-b)=a\times a-a\times b+b\times a -b\times b$\par
% $(a+b)(a-b)=a^2-ab+ba-b^2$\par
% $(a+b)(a-b)=a^2-b^2$ $\square$
% }

% \Exemples{}{
% \begin{enumerate}
% \item Développer et réduire : $A(x)=(x+8)(x-8)$ et $B(x)=(5x-7)(5x+7)$.
% \item Calculer mentalement : $99\times101$ et $27\times 33$
% \end{enumerate}
% }


% \CadreLampe{Remarque}{
% Ces égalités sont vraies quelles que soient les valeurs de $a$ et de $b$.
% }
