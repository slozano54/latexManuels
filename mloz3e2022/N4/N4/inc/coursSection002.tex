\section{Calcul littéral : Techniques de développement}
\subsection{Distributivité simple}

Penser à mettre opposé d'une somme en cas particulier.
\begin{propriete}[\admise]
    Si $a$, $b$ et $k$ sont trois nombres (connus ou inconnus) alors$$k\times(a+b)=k\times a+k\times b$$
\end{propriete}

\begin{remarque}
    \phantom{rrr}\\
    $k\times(a-b)=k\times(a+(-b))$\\
    $k\times(a-b)=k\times a+k\times (-b)$\\
    $k\times(a-b)=ka-kb$
\end{remarque}

\begin{propriete}[Cas particulier de distributivité]
    {\bfseries L'opposé d'une somme algébrique} est égal à la somme des {\bfseries opposés de chacun de ses termes.}
\end{propriete}

\begin{preuve}
    L'opposé d'une somme algébrique est égale au produit de cette somme par $-1$.\\
    La distribution de $-1$ sur chacun des termes de cette somme le transformera en son opposé.
\end{preuve}

\subsection{Double distributivité}
\begin{propriete}[\admise]
    Si $a$, $b$, $c$ et $d$ sont 4 nombres (connus ou inconnus) alors $$ \Distri[Cours,Fleches]{2}{3}{4}{5}$$ 
\end{propriete}

\begin{preuve}
    à proposer à M. LOZANO sur la base du volontariat : Note bonus à la clef \emoji{shushing-face}
\end{preuve}

\subsection{Egalités remarquables}
\begin{propriete}[Carré d'une somme de deux termes]
    \begin{itemize}
    \item \underline{En français} : 
    \og{}Le carré de la somme de deux termes est égal à la somme des carrés des deux termes {\bfseries augmentée} de leur double produit.\fg{}

    \bigskip
    \item \underline{En \og{}langue\fg{} mathématique} :
    Si $a$ et $b$ sont des nombres (connus ou inconnus) alors 
    
    $$\underbrace{\parbox{0.1\linewidth}{\centering $(a+b)^2$}}_{\parbox{0.3\linewidth}{\centering\vspace*{1mm}Carré d'une\\somme}}
    =\underbrace{\parbox{0.1\linewidth}{\centering $a^2$}}_{\parbox{0.1\linewidth}{\centering\vspace*{1mm}carré du\\1\ier terme}}
    +\underbrace{\parbox{0.1\linewidth}{\centering $b^2$}}_{\parbox{0.1\linewidth}{\centering\vspace*{1mm}carré du\\2\ieme terme}}
    +\underbrace{\centering 2\times a\times b}_{\parbox{0.3\linewidth}{\centering\vspace*{1mm}double produit\\( nos deux abbés !)}}$$    
    \end{itemize}
\end{propriete}

\begin{preuve}
    \phantom{rrr}\\
    $(a+b)^2=(a+b)(a+b)$\\
    $(a+b)^2=a\times a+a\times b+b\times a+b\times b$\\
    $(a+b)^2=a^2+ab+ba+b^2$\\
    $(a+b)^2=a^2+b^2+2ab$ $\square$
\end{preuve}

\begin{exemples*1}
    Développer et réduire $A(x)=(2x+7)^2$, $B(x)=\left(x+\dfrac{1}{2}\right)^2$ et $C(x)=\left(\dfrac{4}{5}+\dfrac{3x}{2}\right)^2$
    \correction
    \begin{align*}
        A(x)&=\Distri[Remarquable]{2}{7}{}{}        &B(x)&=\left(x+\dfrac{1}{2}\right)^2&C(x)&=\left(\dfrac{4}{5}+\dfrac{3x}{2}\right)^2\\
        A(x)&=\Distri[Remarquable,Etape=2]{2}{7}{}{}&B(x)&=x^2+2\times x\times \dfrac{1}{2} + \left(\dfrac{1}{2}\right)^2&C(x)&=\left(\dfrac{4}{5}\right)^2+2\times\dfrac{4}{5}\times\dfrac{3x}{2} + \left(\dfrac{3x}{2}\right)^2\\
        A(x)&=\Distri[Remarquable,Etape=3]{2}{7}{}{}&B(x)&=x^2+ x + \dfrac{1}{4}&C(x)&=\dfrac{16}{25}+\dfrac{6x}{5}+ \dfrac{9x^2}{4}
    \end{align*}
\end{exemples*1}

\begin{exemples*1}
    Expliquer comment calculer $A=101^2$ et $B=73^2$ à l'aide d'une égalité remarquable.
    \correction
    \begin{align*}
        A&=101^2                            &B&=73^2\\
        A&=(100+1)^2                        &B&=(70+3)^2\\
        A&=100^2+2\times 100\times 1 + 1^2  &B&=70^2 + 2\times 70\times 3 + 3^2\\
        A&=\num{10000}+200 + 1              &B&=\num{4900} + 420 + 9\\
        A&=\num{10201}                      &B&=\num{5329}
    \end{align*}
\end{exemples*1}

\begin{propriete}[Carré d'une différence de deux termes]
    \begin{itemize}
    \item \underline{En français} : 
    \og{}Le carré de la différence de deux termes est égal à la somme des carrés des deux termes {\bfseries diminuée} de leur double produit\fg{}

    \bigskip
    \item \underline{En \og{}langue\fg{} mathématique} :
    Si $a$ et $b$ sont des nombres (connus ou inconnus) alors 
    
    $$\underbrace{\parbox{0.1\linewidth}{\centering $(a-b)^2$}}_{\parbox{0.3\linewidth}{\centering\vspace*{1mm}Carré d'une\\différence}}
    =\underbrace{\parbox{0.1\linewidth}{\centering $a^2$}}_{\parbox{0.1\linewidth}{\centering\vspace*{1mm}carré du\\1\ier terme}}
    +\underbrace{\parbox{0.1\linewidth}{\centering $b^2$}}_{\parbox{0.1\linewidth}{\centering\vspace*{1mm}carré du\\2\ieme terme}}
    -\underbrace{\centering 2\times a\times b}_{\parbox{0.3\linewidth}{\centering\vspace*{1mm}double produit\\( nos deux abbés !)}}$$    
    \end{itemize}
\end{propriete}

\begin{preuve}
    On peut le faire avec la double distributivité comme précédemment ou justement à l’aide de la propriété précédente.\\
    $(a-b)^2=(a+(-b))^2$\\
    $(a-b)^2=a^2+(-b)^2+2a\times(-b)$\par
    $(a-b)^2=a^2+b^2-2ab$ $\square$
\end{preuve}

\begin{exemples*1}
    Développer et réduire $A(x)=(x-8)^2$, $B(x)=(5x-7)^2$ et $C(y)=\left(\dfrac{2}{3}+\dfrac{4}{7}y\right)^2$
    \correction
    \begin{align*}
        A(x)&=\Distri[Remarquable]{1}{-8}{}{}        &B(x)&=\Distri[Remarquable]{5}{-7}{}{}         &C(y)&=\left(\dfrac{2}{3}+\dfrac{4}{7}y\right)^2\\
        A(x)&=\Distri[Remarquable,Etape=2]{1}{-8}{}{}&B(x)&=\Distri[Remarquable,Etape=2]{5}{-7}{}{} &C(y)&=\left(\dfrac{2}{3}\right)^2+2\times\dfrac{2}{3}\times\dfrac{4}{7}y + \left(\dfrac{4}{7}y\right)^2\\
        A(x)&=\Distri[Remarquable,Etape=3]{1}{-8}{}{}&B(x)&=\Distri[Remarquable,Etape=3]{5}{-7}{}{} &C(y)&=\dfrac{4}{9}+\dfrac{16}{21}y+ \dfrac{16}{49}y^2
    \end{align*}
\end{exemples*1}

\begin{exemples*1}
    Expliquer comment calculer $A=99^2$ et $B=18^2$ à l'aide d'une égalité remarquable.
    \correction
    \begin{align*}
        A&=99^2                            &B&=18^2\\
        A&=(100-1)^2                       &B&=(20-2)^2\\
        A&=100^2-2\times 100\times 1 + 1^2 &B&=20^2 - 2\times 20\times 2 + 2^2\\
        A&=\num{10000}-200 + 1             &B&=400 - 80 + 4\\
        A&=\num{9801}                      &B&=324
    \end{align*}
\end{exemples*1}

\begin{propriete}[Produit de la somme de deux termes par leur différence]
    \begin{itemize}
    \item \underline{En français} : 
    \og{}Le produit de la somme de deux nombres par leur différence est égal à la différence de leurs carrés\fg{}

    \bigskip
    \item \underline{En \og{}langue\fg{} mathématique} :
    Si $a$ et $b$ sont des nombres (connus ou inconnus) alors 
    
    $$\underbrace{\parbox{0.2\linewidth}{\centering $(a+b)(a-b)$}}_{\parbox{0.3\linewidth}{\centering\vspace*{1mm}Produit d'une somme \\ et d'une différence \\ entre les mêmes termes}}
    =\underbrace{\parbox{0.1\linewidth}{\centering $a^2$}}_{\parbox{0.1\linewidth}{\centering\vspace*{1mm}carré du\\1\ier terme}}
    -\underbrace{\parbox{0.1\linewidth}{\centering $b^2$}}_{\parbox{0.1\linewidth}{\centering\vspace*{1mm}carré du\\2\ieme terme}}
    $$    
    \end{itemize}
\end{propriete}

\begin{preuve}
    Avec la double distributivité.\\
    $(a+b)(a-b)=a\times a-a\times b+b\times a -b\times b$\\
    $(a+b)(a-b)=a^2-ab+ba-b^2$\\
    $(a+b)(a-b)=a^2-b^2$ $\square$
\end{preuve}

\begin{exemples*1}
    Développer et réduire  $A(t)=(t+8)(t-8)$ et $B(y)=(5y-7)(5y+7)$
    \correction
    \begin{align*}
        A(x)&=\Distri[Lettre=t,Remarquable]{1}{8}{1}{-8}        &B(x)&=\Distri[Lettre=y,Remarquable]{5}{-7}{5}{7}\\
        A(x)&=\Distri[Lettre=t,Remarquable,Etape=2]{1}{8}{1}{-8}&B(x)&=\Distri[Lettre=y,Remarquable]{5}{7}{5}{-7}\\
        A(x)&=\Distri[Lettre=t,Remarquable,Etape=3]{1}{8}{1}{-8}&B(x)&=\Distri[Lettre=y,Remarquable,Etape=2]{5}{7}{5}{-7}\\
            &                                                   &B(x)&=\Distri[Lettre=y,Remarquable,Etape=3]{5}{7}{5}{-7} 
    \end{align*}
\end{exemples*1}

\begin{exemples*1}
    Expliquer comment calculer $A=99\times 101$ et $B=27\times 33$ à l'aide d'une égalité remarquable.
    \correction
    \begin{align*}
        A&=99\times 101 &B&=33\times 27\\
        A&=(100-1)\times (100+1) &B&=(30+3)\times (30-3)\\
        A&= 100^2-1^2&B&=30^2-3^2\\
        A&= \num{10000}-1&B&=900-9\\
        A&= \num{9999}&B&=891
    \end{align*}
\end{exemples*1}

\begin{remarque}
    Ces égalités sont vraies quelles que soient les valeurs de $a$ et de $b$.
\end{remarque}