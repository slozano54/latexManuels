\begin{exercice*}
    Dans le programme ci-dessous, \og $x$ \fg{}, \og Etape1 \fg{}, \og Etape2 \fg{} et , \og Resultat \fg{} sont quatre variables.

    \hspace*{-15mm}
    \begin{minipage}{0.6\linewidth}
        \begin{Scratch}[Echelle=0.7]
            Place Drapeau;
            Place Demander("Choisir un nombre");
            Place MettreVar("$x$",OvalCap("réponse"));
            Place DireT("Je multiplie le nombre par $6$","2");
            Place MettreVar("Etape1",OpMul("$6$",OvalVar("$x$")));
            Place DireT("J'ajoute 10 au résultat","2");
            Place MettreVar("Etape2",OpAdd(OvalVar("Etape1"),"$10$"));
            Place DireT("Bonjour !","2");
            Place MettreVar("Resultat",OpDiv(OvalVar("Etape2"),"$2$"));
            Place DireT(OpRegrouper("J'obtiens finalement ",OvalVar("Resultat")),"2");
        \end{Scratch}
    \end{minipage}    
    \begin{minipage}{0.35\linewidth}
        \vspace*{-36mm}
        \hspace*{15mm}
        \scalebox{0.9}{
        \fbox{
        \parbox{\linewidth}{
        \fbox{Créer une variable}        
        
        \smallskip
        \raisebox{1ex}{\emoji{check-box-with-check}}
        \begin{Scratch}[Echelle=0.7]
            Place OvalVar("Etape1");
        \end{Scratch}

        \smallskip
        \raisebox{1ex}{\emoji{check-box-with-check}}
        \begin{Scratch}[Echelle=0.7]
            Place OvalVar("Etape2");
        \end{Scratch}

        \smallskip
        \raisebox{1ex}{\emoji{check-box-with-check}}
        \begin{Scratch}[Echelle=0.7]
            Place OvalVar("Resultat");
        \end{Scratch}

        \smallskip
        \raisebox{1ex}{\emoji{check-box-with-check}}
        \begin{Scratch}[Echelle=0.7]
            Place OvalVar("$x$");
        \end{Scratch}
        }
        }
        }
    \end{minipage}
    \begin{enumerate}
        \item Cassandra a fait fonctionner ce programme en choisissant le nombre $5$. Justifier que ce qui est dit à la fin est \og J'obtiens finalement $20$\fg{}.
        \item Déterminer ce que dit le programme, si Alycia le fait fonctionner en choisissant $7$ au départ.
        \item Clément fait fonctionner le programme. À la fin, ce qui est dit est \og J'obtiens finalement $8$\fg{}. Déterminer le nombre que Clément a choisi au départ.
        \item Si on note $x$ le nombre choisi au départ, écrire en fonction de $x$ l'expression obtenue à la fin du programme, puis réduire cette expression.
        \item Lola utilise le programme suivant :
        
        \begin{minipage}{1\linewidth}
        \myProgCalcul{$\leadsto$}{Programme de calcul}{%
        \ProgCalcul[Enonce,ThemePerso]{%
            Choisir un nombre.,
            Lui ajouter $2$.,
            Multiplier le résultat par $5$.,
            }
        }
        \end{minipage}

        \smallskip
        Déterminer si on peut choisir un nombre pour lequel le résultat obtenu par Lola est le même que celui obtenu par le programme initial.
    \end{enumerate}
\end{exercice*}
\begin{corrige}
    %\setcounter{partie}{0} % Pour s'assurer que le compteur de \partie est à zéro dans les corrigés
    %\phantom{rrr}    
    Dans le programme ci-dessous, \og $x$ \fg{}, \og Etape1 \fg{}, \og Etape2 \fg{} et , \og Resultat \fg{} sont quatre variables.

    \scalebox{0.8}{
    \begin{minipage}{0.6\linewidth}
        \begin{Scratch}[Echelle=0.7]
            Place Drapeau;
            Place Demander("Choisir un nombre");
            Place MettreVar("$x$",OvalCap("réponse"));
            Place DireT("Je multiplie le nombre par $6$","2");
            Place MettreVar("Etape1",OpMul("$6$",OvalVar("$x$")));
            Place DireT("J'ajoute 10 au résultat","2");
            Place MettreVar("Etape2",OpAdd(OvalVar("Etape1"),"$10$"));
            Place DireT("Bonjour !","2");
            Place MettreVar("Resultat",OpDiv(OvalVar("Etape2"),"$2$"));
            Place DireT(OpRegrouper("J'obtiens finalement ",OvalVar("Resultat")),"2");
        \end{Scratch}
    \end{minipage}    
    \begin{minipage}{0.35\linewidth}
        \vspace*{-36mm}
        \hspace*{17mm}
        \scalebox{0.9}{
        \fbox{
        \parbox{1.1\linewidth}{
        \fbox{Créer une variable}        
        
        \smallskip
        \raisebox{1ex}{\emoji{check-box-with-check}}
        \begin{Scratch}[Echelle=0.7]
            Place OvalVar("Etape1");
        \end{Scratch}

        \smallskip
        \raisebox{1ex}{\emoji{check-box-with-check}}
        \begin{Scratch}[Echelle=0.7]
            Place OvalVar("Etape2");
        \end{Scratch}

        \smallskip
        \raisebox{1ex}{\emoji{check-box-with-check}}
        \begin{Scratch}[Echelle=0.7]
            Place OvalVar("Resultat");
        \end{Scratch}

        \smallskip
        \raisebox{1ex}{\emoji{check-box-with-check}}
        \begin{Scratch}[Echelle=0.7]
            Place OvalVar("$x$");
        \end{Scratch}
        }
        }
        }
    \end{minipage}
    }

    \begin{enumerate}
        \item Cassandra a fait fonctionner ce programme en choisissant le nombre $5$. Justifier que ce qui est dit à la fin est \og J'obtiens finalement $20$\fg{}.\\
        {\red $6\times 5 =30$ ; $30+10 = 40$ ; $40\div 2 = 20$\\
        Donc à la fin le programme dit : \og J'obtiens finalement $20$\fg{}.
        }\\
        \item Déterminer ce que dit le programme, si Alycia le fait fonctionner en choisissant $7$ au départ.\\
        {\red $6\times 7 = 42$ ; $42+10 = 52$ ; $52\div 2 = 26$\\
        Donc à la fin le programme dit : \og J'obtiens finalement $26$\fg{}.
        }\\
        \item Clément fait fonctionner le programme. À la fin, ce qui est dit est \og J'obtiens finalement $8$\fg{}. Déterminer le nombre que Clément a choisi au départ.\\
        {\red $8\times 2 = 16$ ; $16-10=6$ et $6\div 6=1$\\ Clément avait choisi $1$.}\\
        \item Si on note $x$ le nombre choisi au départ, écrire en fonction de $x$ l'expression obtenue à la fin du programme, puis réduire cette expression.\\
        {\red $6\times x =6x$ puis $6x+10$ et enfin $(6x+10)\div 2 = 3x+5$}\\
        \item Lola utilise le programme suivant :
    \end{enumerate}
    \begin{minipage}{1\linewidth}
        \myProgCalcul{$\leadsto$}{Programme de calcul}{%
        \ProgCalcul[Enonce,ThemePerso]{%
            Choisir un nombre.,
            Lui ajouter $2$.,
            Multiplier le résultat par $5$.,
            }
        }
    \end{minipage}
\Coupe
    Déterminer si on peut choisir un nombre pour lequel le résultat obtenu par Lola est le même que celui obtenu par le programme initial.\\
    {\red si on note $x$ le nombre de départ dans le programme de Lola, $x+2$ puis $(x+2)\times5=5x+10$\\
    On veut donc un nombre tel que $3x+5=5x+10$ soit $-2x=5$ soit $x=\num{-2.5}$.\\
    Le programme de calcul de Lola et le programme initial donnent donc le même résultat pour $x=\num{-2.5}$.
    }
\end{corrige}

