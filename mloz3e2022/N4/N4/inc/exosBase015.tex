\begin{exercice*}
    {\bfseries Objectif :} Déterminer une règle permettant de multiplier un nombre à deux chiffres par \num{11}.
    \begin{enumerate}
        \item Calculer $11\times 27$ et $11\times 41$.
        \item Déterminer une relation entre les chiffres de chaque résultat.
        \item Soit $N$ un nombre à deux chiffres. On note $U$ son chiffre des unités et $D$ son chiffre des dizaines. Écrire $N$ en fonction de $D$ et de $U$.
        \item Démontrer que $11N = 100D + 10(U+D) + U$.
        \item Essayer d'énoncer une règle en français.
        \item En appliquant cette règle, calculer mentalement $11\times 16$ et $11\times 35$. 
    \end{enumerate}
\end{exercice*}
\begin{corrige}
    %\setcounter{partie}{0} % Pour s'assurer que le compteur de \partie est à zéro dans les corrigés
    %\phantom{rrr}    
    {\bfseries Objectif :} Déterminer une règle permettant de multiplier un nombre à deux chiffres par \num{11}.
    \begin{enumerate}
        \item Calculer $11\times 27$ et $11\times 41$.\\
        {\red $11\times 27 = 297$ et $11\times 41 = 451$}
        \item Déterminer une relation entre les chiffres de chaque résultat.\\
        {\red On remarque que $2+7=9$ et $4+1=5$}
        \item Soit $N$ un nombre à deux chiffres. On note $U$ son chiffre des unités et $D$ son chiffre des dizaines. Écrire $N$ en fonction de $D$ et de $U$.\\
        {\red $N=10d+U$}
        \item Démontrer que $11N = 100D + 10(U+D) + U$.\\
        {\red $11N=11\times (10D + U)$\\$11N=110D+11U$\\$N=100D + 10(U+D) + U$}        
        \item Essayer d'énoncer une règle en français.\\
        {\red Pour multiplier un nombre de deux chiffres par $11$, on additionne les deux chiffres et on place la somme au milieu de ces deux chiffres.}
        \item En appliquant cette règle, calculer mentalement $11\times 16$ et $11\times 35$.\\
        {\red $11\times 16 = 176$ et $11\times 35 = 385$}
    \end{enumerate}
\end{corrige}

