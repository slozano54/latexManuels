\begin{exercice*}[Démonstrations]
    \begin{enumerate}
        \item Recopier et compléter.
        \begin{align*}
            -(a+b)&= \dots\times(a+b)\\
            -(a+b)&= \dots\times\dots + \dots\times\dots\\
            -(a+b)&= \dots+\dots
        \end{align*}
        Donc \og{} L'opposé d'une \dots{} est égal à la somme des \dots \fg{}.
        \item Recopier et compléter.
        \begin{align*}
            (a+b)&= \dots\times(a+b)\\
            (a+b)&= \dots\times\dots + \dots\times\dots\\
            (a+b)&= \dots+\dots
        \end{align*}
        Donc dans ce cas les parenthèses étaient inutiles.
        \item Comme au 1), démontrer que \og{} L'oppposé d'une différence est égal à la différence des opposés \fg{}.
        \item Dire si l'assertion suivante  \og{} L'opposé d'un produit est égal au produit de opposés.\fg{} est vraie ou fausse. Justifier.
    \end{enumerate}
\end{exercice*}
\begin{corrige}
    %\setcounter{partie}{0} % Pour s'assurer que le compteur de \partie est à zéro dans les corrigés
    %\phantom{rrr}    
    \begin{enumerate}
        \item Recopier et compléter.
        \begin{align*}
            -(a+b)&= {\red (-1)}\times(a+b)\\
            -(a+b)&= {\red (-1)}\times{\red a} + {\red (-1)}\times{\red b}\\
            -(a+b)&= {\red (-a)}+{\red (-b)}
        \end{align*}
        Donc \og{} L'opposé d'une {\red somme} est égal à la somme des {\red opposés} \fg{}.
    \end{enumerate}
    \Coupe
    \begin{enumerate}
        \setcounter{enumi}{1}
        \item Recopier et compléter.
        \begin{align*}
            (a+b)&= {\red 1}\times(a+b)\\
            (a+b)&= {\red 1}\times{\red a} + {\red 1}\times{\red b}\\
            (a+b)&= {\red a}+{\red b}
        \end{align*}
        Donc dans ce cas les parenthèses étaient iutiles.
        \item Comme au 1), démontrer que \og{} L'oppposé d'une différence est égal à la différence des opposés \fg{}.\\
        \item {\red 
        \begin{align*}
            -(a-b)&= (-1)\times(a-b)\\
            -(a-b)&= (-1)\times a - (-1)\times b\\
            -(a-b)&= (-a)-(-b)
        \end{align*}
        Donc \og{} L'oppposé d'une différence est égal à la différence des opposés \fg{}.
        }
        \item Dire si l'assertion suivante  \og{} L'opposé d'un produit est égal au produit de opposés.\fg{} est vraie ou fausse. Justifier.\\
        {\red 
        \begin{align*}
            -(a\times b)&= (-1)\times a\times b\\
            -(a\times b)&= (-a)\times b \text{ ou } a\times (-b)
        \end{align*}
        Donc \og{} L'oppposé d'un produit n'est pas égal au produit des opposés \fg{}.
        }
    \end{enumerate}
\end{corrige}

