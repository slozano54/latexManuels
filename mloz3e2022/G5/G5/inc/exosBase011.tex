\begin{exercice*}
    Abel a acheté, pour ses enfants, un ballon gonflable en forme de sphère. Le diamètre de ce ballon est de \Lg[cm]{30}.
    \begin{enumerate}
        \item Calculer le volume du ballon, arrondi au \Vol[cm]{}.
        \item À présent, Abel doit le gonfler. À chaque expiration, il souffle \Vol[cm]{500} d'air dans le ballon.
        
        Déterminer le nombre de fois qu'il devra souffler pour le gonfler au maximum.
        \item Calculer la surface de ce ballon.
    \end{enumerate}
\end{exercice*}
\begin{corrige}
    Abel a acheté, pour ses enfants, un ballon gonflable en forme de sphère. Le diamètre de ce ballon est de \Lg[cm]{30}.

    \begin{enumerate}
        \item Calculer le volume du ballon, arrondi au \Vol[cm]{}.
        
        {\color{red} Le diamètre du baollon vaut \Lg[cm]{30} donc son volume vaut $\mathcal{V}_{\text{ballon}}=\dfrac43\pi r^3=\dfrac43\pi 15^3=\num{4500}\pi~\Vol[cm]{}$.
        
        Soit $\mathcal{V}_{\text{ballon}}\approx\Vol[cm]{14137}$ au \Vol[cm]{} près.}
        \item À présent, Abel doit le gonfler. À chaque expiration, il souffle \Vol[cm]{500} d'air dans le ballon.
        
        Déterminer le nombre de fois qu'il devra souffler pour le gonfler au maximum.

        \textcolor{red}{$\dfrac{\num{4500}\pi}{500}\approx\num{28.274}$ donc Abel soufflera 29 fois.}
        \item Calculer la surface de ce ballon.
        
        {\color{red} $\mathcal{A}_{\text{ballon}}=4\pi r^2=43\pi 15^2=\num{900}\pi~\Aire[cm]{}$.
        
        La surface de ce ballon est donc de $900\pi~\Aire[cm]{}$, soit environ \Aire[cm]{2827}.
        }
    \end{enumerate}
\end{corrige}
