\begin{exercice*}
    \phantom{rrr}

    \begin{minipage}{0.45\linewidth}
        On considère un cylindre de révolution de rayon \Lg[cm]{2,5} et de hauteur \Lg[cm]{3,5}.
    \end{minipage}
    \hfill        
    \begin{minipage}{0.5\linewidth}
        \begin{center}
            % \begin{Geometrie}[Cadre="aucun"]
            %     pair A[],B[],O[],M;
            %     path BaseA,BaseB,SectionA,SectionB;
            %     numeric SLHauteur,SLCoeffSection;
            %     SLHauteur:=3;
            %     SLCoeffSection:=0.4;
            %     O0=u*(0,0);
            %     O1-O0=u*(0,SLHauteur);            
            %     M-O0=u*(0,SLCoeffSection*SLHauteur);            
            %     BaseA=fullcircle scaled 4u  yscaled 0.25;
            %     BaseB=BaseA shifted (0,u*SLHauteur);
            %     SectionA=BaseA shifted (0,u*SLCoeffSection*SLHauteur);
            %     fill SectionA withcolor PaleGreen;
            %     A0=point 0.4*(length BaseA) of BaseA;
            %     A1=point 0.9*(length BaseA) of BaseA;
            %     B0-A0=u*(0,SLHauteur);
            %     B1-A1=u*(0,SLHauteur);
            %     SectionB=A0--A1--B1--B0--cycle;
            %     fill SectionB withcolor PaleGreen; 
            %     draw (subpath (0.5*(length BaseA),length BaseA) of BaseA); 
            %     draw (subpath (0,0.5*(length BaseA)) of BaseA) dashed evenly;
            %     draw BaseB;
            %     draw point 0.5*(length BaseA) of BaseA--point 0.5*(length BaseB) of BaseB;
            %     draw point length BaseA of BaseA--point length BaseB of BaseB;
            %     draw (subpath (0.5*(length SectionA),length SectionA) of SectionA); 
            %     draw (subpath (0,0.5*(length SectionA)) of SectionA) dashed evenly;
            %     draw B0--A0--A1 dashed evenly;
            %     draw A1--B1--B0;
            %     draw segment(1.5[O0,O1],1.5[O1,O0]) dashed dashpattern(on6 off3 on3 off 3);
            %     dotlabel.llft(btex $O'$ etex,O0);
            %     dotlabel.llft(btex $O$ etex,O1);
            %     dotlabel.llft(btex $M$ etex,M);
            % \end{Geometrie}
            \scalebox{0.8}{
                \Solide[%
                    Nom=cylindre,
                    Distance=50,            
                    Axes,
                    Section="face",
                    CoefSection=0.7,
                    CouleurSection=PaleGreen,
                    ListeSommets={O,O'},
                    PointsSection={I,J,K,L,M},
                    Traces={
                        color A,B,C,D;
                        numeric SLangle;
                        SLangle:=80;
                        A=(\useKV[ClesSolides]{RayonCylindre}*cosd(SLangle),\useKV[ClesSolides]{RayonCylindre}*sind(SLangle),0);
                        B=(-\useKV[ClesSolides]{RayonCylindre}*cosd(SLangle),-\useKV[ClesSolides]{RayonCylindre}*sind(SLangle),0);
                        C-B=(0,0,\useKV[ClesSolides]{HauteurCylindre});
                        D-A=(0,0,\useKV[ClesSolides]{HauteurCylindre});                
                        path SLSection;
                        SLSection=Projette(A)--Projette(B)--Projette(C)--Projette(D)--cycle;
                        fillcolor:=PaleGreen;
                        transparence SLSection;% withcolor PaleGreen;                
                        % draw Projette(A)--Projette(B)--Projette(C)--Projette(D);
                        draw chemin(C,D,A);
                        trace chemin(A,B,C) dashed evenly;
                        Label.urt(btex $O$ etex,O);
                        Label.llft(btex $O'$ etex,O');
                        DotLabel.urt(btex M etex,M);                
                    }
                ]%
            }
        \end{center}
    \end{minipage}
    \begin{enumerate}
        \item Dessiner ci-dessous, en vraie grandeur, la section du cylindre par un plan perpendiculaire à son axe $(OO')$.
        \\\vspace*{60mm}
        \item Dessiner ci-dessous, en vraie grandeur, la section de ce cylindre par un plan parallèle à son axe contenant $O$ et $O'$.
        \\\vspace*{60mm}
    \end{enumerate}
\end{exercice*}
\begin{corrige}
    \phantom{rrr}

    \begin{minipage}{0.4\linewidth}
        On considère un cylindre de révolution de rayon \Lg[cm]{2,5} et de hauteur \Lg[cm]{3,5}.
    \end{minipage}
    \hfill        
    \begin{minipage}{0.55\linewidth}
        \begin{center}
            % \begin{Geometrie}[Cadre="aucun"]
            %     pair A[],B[],O[],M;
            %     path BaseA,BaseB,SectionA,SectionB;
            %     numeric SLHauteur,SLCoeffSection;
            %     SLHauteur:=3;
            %     SLCoeffSection:=0.4;
            %     O0=u*(0,0);
            %     O1-O0=u*(0,SLHauteur);            
            %     M-O0=u*(0,SLCoeffSection*SLHauteur);            
            %     BaseA=fullcircle scaled 4u  yscaled 0.25;
            %     BaseB=BaseA shifted (0,u*SLHauteur);
            %     SectionA=BaseA shifted (0,u*SLCoeffSection*SLHauteur);
            %     fill SectionA withcolor PaleGreen;
            %     A0=point 0.4*(length BaseA) of BaseA;
            %     A1=point 0.9*(length BaseA) of BaseA;
            %     B0-A0=u*(0,SLHauteur);
            %     B1-A1=u*(0,SLHauteur);
            %     SectionB=A0--A1--B1--B0--cycle;
            %     fill SectionB withcolor PaleGreen; 
            %     draw (subpath (0.5*(length BaseA),length BaseA) of BaseA); 
            %     draw (subpath (0,0.5*(length BaseA)) of BaseA) dashed evenly;
            %     draw BaseB;
            %     draw point 0.5*(length BaseA) of BaseA--point 0.5*(length BaseB) of BaseB;
            %     draw point length BaseA of BaseA--point length BaseB of BaseB;
            %     draw (subpath (0.5*(length SectionA),length SectionA) of SectionA); 
            %     draw (subpath (0,0.5*(length SectionA)) of SectionA) dashed evenly;
            %     draw B0--A0--A1 dashed evenly;
            %     draw A1--B1--B0;
            %     draw segment(1.5[O0,O1],1.5[O1,O0]) dashed dashpattern(on6 off3 on3 off 3);
            %     dotlabel.llft(btex $O'$ etex,O0);
            %     dotlabel.llft(btex $O$ etex,O1);
            %     dotlabel.llft(btex $M$ etex,M);
            % \end{Geometrie}
            \scalebox{0.8}{
                \Solide[%
                    Nom=cylindre,
                    Distance=50,            
                    Axes,
                    Section="face",
                    CoefSection=0.7,
                    CouleurSection=PaleGreen,
                    ListeSommets={O,O'},
                    PointsSection={I,J,K,L,M},
                    Traces={
                        color A,B,C,D;
                        numeric SLangle;
                        SLangle:=80;
                        A=(\useKV[ClesSolides]{RayonCylindre}*cosd(SLangle),\useKV[ClesSolides]{RayonCylindre}*sind(SLangle),0);
                        B=(-\useKV[ClesSolides]{RayonCylindre}*cosd(SLangle),-\useKV[ClesSolides]{RayonCylindre}*sind(SLangle),0);
                        C-B=(0,0,\useKV[ClesSolides]{HauteurCylindre});
                        D-A=(0,0,\useKV[ClesSolides]{HauteurCylindre});                
                        path SLSection;
                        SLSection=Projette(A)--Projette(B)--Projette(C)--Projette(D)--cycle;
                        fillcolor:=PaleGreen;
                        transparence SLSection;% withcolor PaleGreen;                
                        % draw Projette(A)--Projette(B)--Projette(C)--Projette(D);
                        draw chemin(C,D,A);
                        trace chemin(A,B,C) dashed evenly;
                        Label.urt(btex $O$ etex,O);
                        Label.llft(btex $O'$ etex,O');
                        DotLabel.urt(btex M etex,M);                
                    }
                ]%
            }
        \end{center}
    \end{minipage}
    \begin{enumerate}
        \item Dessiner ci-dessous, en vraie grandeur, la section du cylindre par un plan perpendiculaire à son axe $(OO')$.
        
        \begin{Geometrie}
            pair O;
            O=u*(5,5);
            trace cercles(O,2.5u) withcolor red;
            dotlabel.urt(btex $M$ etex,O) withcolor red;
        \end{Geometrie}
        \item Dessiner ci-dessous, en vraie grandeur, la section de ce cylindre par un plan parallèle à son axe contenant $O$ et $O'$.
        
        \medskip
        \begin{Geometrie}
            pair A;
            A=u*(0.5,0.5);            
            draw A--A shifted (0,5u)--A shifted (3.5u,5u)--A shifted (3.5u,0)--cycle withcolor red;
        \end{Geometrie}
    \end{enumerate}
\end{corrige}
