\begin{exercice*}
    On considère les trois solides suivants :
    \begin{itemize}
        \item la boule de centre $O$ et de rayon $SO$ tel que $SO = \Lg[cm]{3}$ ;
        \item la pyramide $SEFGH$ de hauteur \Lg[cm]{3} dont la base est le carré $EFGH$ de côté \Lg[cm]{6} ;
        \item le cube $ABCDEFGH$ d’arête \Lg[cm]{6}.
    \end{itemize}
    
    Ces trois solides sont placés dans un récipient.

    Ce récipient est représenté par le pavé droit $ABCDIJKL$, de hauteur \Lg[cm]{15}, dont la base est le carré $ABCD$ \mbox{de côté \Lg[cm]{6}}.
    \vspace*{-7mm}
    \begin{center}
        \scalebox{0.9}{
            \Solide[%
                Nom=pave,
                Largeur=2,
                Hauteur=4,
                Profondeur=1.5,
                ListeSommets={A,B,C,D,L,I,J,K},
                Aretes=false,
                Traces={
                    color S,E,F,G,H,O;
                    S=(0.75,1,2.5);
                    E-A=(0,0,1.5);
                    F-B=(0,0,1.5);
                    G-C=(0,0,1.5);
                    H-D=(0,0,1.5);
                    O=(0.75,1,3.25);
                    DotLabel.top(btex S etex,S);
                    Label.lft(btex E etex,E);
                    Label.lrt(btex F etex,F);
                    Label.lrt(btex G etex,G);
                    Label.lrt(btex H etex,H);
                    DotLabel.urt(btex O etex,O);
                    path myCircl;
                    myCircl = cercles(O,S,A,C,I);
                    pas:=100;
                    for k=0 upto pas:
                        fill homothetie(myCircl,Projette(O),(pas-k)/pas) withcolor ((pas-k)/pas)[white,LightGray];
                    endfor;
                    trace myCircl;
                    trace chemin(A,B,C,K,L,I,J,K);
                    trace chemin(I,A);
                    trace chemin(E,F,G);
                    trace chemin(F,B);
                    trace chemin(A,D,C) dashed evenly;
                    trace chemin(E,S,H) dashed evenly;
                    trace chemin(F,S,G) dashed evenly;
                    trace chemin(E,H,G) dashed evenly;
                    trace chemin(H,D) dashed evenly;
                }
            ]
        }

        \textit{La figure n’est pas en vraie grandeur.}
    \end{center}
    \begin{enumerate}
        \item Calculer le volume du cube $ABCDEFGH$, en \Vol[cm]{}.
        \item Calculer le volume de la pyramide $SEFGH$, en \Vol[cm]{}.
        \item Calculer le volume de la boule, en \Vol[cm]{} et arrondir à l’unité près.
        \item En déduire le volume occupé par les trois  solides à l’intérieur du pavé $ABCDIJKL$, en \Vol[cm]{}.
        \item Dire s'il est possible ou non de verser dans ce récipient \Capa[cL]{20} d’eau sans qu’elle ne déborde. Justifier.
    \end{enumerate}
\end{exercice*}
\begin{corrige}
    On considère les trois solides suivants :
    \begin{itemize}
        \item la boule de centre $O$ et de rayon $SO$ tel que $SO = \Lg[cm]{3}$ ;
        \item la pyramide $SEFGH$ de hauteur \Lg[cm]{3} dont la base est le carré $EFGH$ de côté \Lg[cm]{6} ;
        \item le cube $ABCDEFGH$ d’arête \Lg[cm]{6}.
    \end{itemize}
    
    Ces trois solides sont placés dans un récipient.

    Ce récipient est représenté par le pavé droit $ABCDIJKL$, de hauteur \Lg[cm]{15}, dont la base est le carré $ABCD$ \mbox{de côté \Lg[cm]{6}}.
    \begin{center}
        \scalebox{0.9}{
            \Solide[%
                Nom=pave,
                Largeur=2,
                Hauteur=4,
                Profondeur=1.5,
                ListeSommets={A,B,C,D,L,I,J,K},
                Aretes=false,
                Traces={
                    color S,E,F,G,H,O;
                    S=(0.75,1,2.5);
                    E-A=(0,0,1.5);
                    F-B=(0,0,1.5);
                    G-C=(0,0,1.5);
                    H-D=(0,0,1.5);
                    O=(0.75,1,3.25);
                    DotLabel.top(btex S etex,S);
                    Label.lft(btex E etex,E);
                    Label.lrt(btex F etex,F);
                    Label.lrt(btex G etex,G);
                    Label.lrt(btex H etex,H);
                    DotLabel.urt(btex O etex,O);
                    path myCircl;
                    myCircl = cercles(O,S,A,C,I);
                    pas:=100;
                    for k=0 upto pas:
                        fill homothetie(myCircl,Projette(O),(pas-k)/pas) withcolor ((pas-k)/pas)[white,LightGray];
                    endfor;
                    trace myCircl;
                    trace chemin(A,B,C,K,L,I,J,K);
                    trace chemin(I,A);
                    trace chemin(E,F,G);
                    trace chemin(F,B);
                    trace chemin(A,D,C) dashed evenly;
                    trace chemin(E,S,H) dashed evenly;
                    trace chemin(F,S,G) dashed evenly;
                    trace chemin(E,H,G) dashed evenly;
                    trace chemin(H,D) dashed evenly;
                }
            ]
        }

        \textit{La figure n’est pas en vraie grandeur.}
    \end{center}
    \begin{enumerate}
        \item Calculer le volume du cube $ABCDEFGH$, en \Vol[cm]{}.
        
        {\color{red}$\mathcal{V}_{\text{cube}}=6^3=\Vol[cm]{216}$, le volume du cube est de \Vol[cm]{216}.}
        \item Calculer le volume de la pyramide $SEFGH$, en \Vol[cm]{}.
        
        {\color{red}$\mathcal{V}_{\text{pyramide}}=\dfrac13 B\times h= 36\times 3 \div 3 = \Vol[cm]{36}$}
    \end{enumerate}
    \Coupe
    \begin{enumerate}
        \setcounter{enumi}{2}
        \item Calculer le volume de la boule, en \Vol[cm]{} et arrondir à l’unité près.
        
        {\color{red}$\mathcal{V}_{\text{boule}}=\dfrac43 \pi R^3=\dfrac43 \pi 3^3 = 36\pi\approx\Vol[cm]{113}$}
        \item En déduire le volume occupé par les trois  solides à l’intérieur du pavé $ABCDIJKL$, en \Vol[cm]{}.
        
        {\color{red}$\mathcal{V}_{\text{total}}=\mathcal{V}_{\text{cube}}+\mathcal{V}_{\text{pyramide}}+\mathcal{V}_{\text{boule}}\approx\Vol[cm]{216}+\Vol[cm]{36}+\Vol[cm]{113}$ soit environ \Vol[cm]{365}}
        \item Dire s'il est possible ou non de verser dans ce récipient \Capa[cL]{20} d’eau sans qu’elle ne déborde. Justifier.

        {\color{red}$\mathcal{V}_{\text{pavé}}=L\times \ell \times h = 6\times 6\times 15 = \Vol[cm]{540}$
        
        $\mathcal{V}_{\text{pavé}}-\mathcal{V}_{\text{total}}\approx 540 -365$ Il reste donc environ \Vol[cm]{175}=\Capa[cL]{17.5} dans le récipient. On en pourra donc pas ajouter \Capa[cL]{20} sans faire déborder.
        }
    \end{enumerate}
\end{corrige}
