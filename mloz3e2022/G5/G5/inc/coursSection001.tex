\section{Rappels : Cercles et disques}
\subsection{Cercles}
\begin{definition}
    Le \textbf{cercle} de centre $O$ et de rayon $R$ est l'ensemble des points du plan situés à une distance $R$ du point $O$.
\end{definition}
\begin{propriete}[\admise]
    Si $(\mathscr C)$ est un cercle de rayon $R$ alors son périmètre vaut $2\pi R$ : \og{} deux pierres \fg{}.
\end{propriete}
\begin{exemple*1}
    \titreExemple{Utilisation du vocabulaire}

    \begin{minipage}{0.5\linewidth}
        Sur la figure ci-contre :
        \begin{itemize}
            \item $[EB]$ est un diamètre.
            \item $[OC]$ est un rayon.
            \item $OC=R$.
        \end{itemize}
    \end{minipage}
    \begin{minipage}{0.5\linewidth}
        \begin{center}
            \begin{Geometrie}
                u:=u*0.6; 
                pair O,B,C,D,E;
                path co;
                O=u*(3,3);
                co=cercles(O,2u);
                C=pointarc(co,15);
                B=pointarc(co,45);
                D=pointarc(co,105);
                E=pointarc(co,225);
                trace co;
                trace segment(E,B) dashed evenly;
                trace segment(O,C) dashed evenly;
                label.ulft(btex $O$ etex,O);
                label.rt(btex $C$ etex,C);
                label.urt(btex $B$ etex,B);
                %nomme.top(D);
                label.top(btex $(\mathcal{C})$ etex,D);
                label.llft(btex $E$ etex,E);
            \end{Geometrie}
        \end{center}
    \end{minipage}
    \vspace*{-5mm}
\end{exemple*1}
\begin{methode}[Longueur d'un cercle]
    \exercice
        \begin{enumerate}
            \item Calculer la longueur d'un cercle de rayon \Lg[cm]{3,7}.
            \item \textit{Soit un cercle de longueur \Lg[cm]{18}}
            \begin{enumerate}
                \item Calculer la valeur exacte de son rayon.
                \item Donner une valeur approchée au dixième de ce rayon.
            \end{enumerate}
        \end{enumerate}
    \correction
    \begin{enumerate}
        \item $2\times\pi\times \Lg[cm]{3,7}\approx\Lg[cm]{23.2}$.
        \item \textit{Soit un cercle de longueur \Lg[cm]{18}}
        \begin{enumerate}
            \item $R=\Lg[cm]{18}\div (2\times\pi)$ soit $R=\Lg[cm]{9}\div \pi$.
            \item $R\approx \Lg[cm]{2.9}$
        \end{enumerate}
    \end{enumerate}
\end{methode}    
\subsection{Disques}
\begin{definition}
    Le \textbf{disque} de centre $O$ et de rayon $R$ est l'ensemble des points du plan situés à une distance inférieure ou égale à $R$ du point $O$.    
\end{definition}
\begin{propriete}[\admise]
    Si $(\mathscr D)$ est un disque de rayon $R$ alors son aire vaut $\pi R^2$ : \og{} pierre carrée \fg{}.
\end{propriete}
\begin{exemple*1}
    \titreExemple{Utilisation du vocabulaire}

    \begin{minipage}{0.5\linewidth}
        Sur la figure ci-contre :
        \begin{itemize}
            \item $[EB]$ est un diamètre.
            \item $[OC]$ est un rayon.
            \item $OC=R$.
        \end{itemize}
    \end{minipage}
    \begin{minipage}{0.5\linewidth}
        \begin{center}
            % \includegraphics[scale=1]{coursespace.12} 
            \begin{Geometrie}
                u:=u*0.6; 
                pair O,B,C,D,E;
                path co;
                O=u*(3,3);
                co=cercles(O,2u);
                trace co;
                C=pointarc(co,15);
                B=pointarc(co,45);
                D=pointarc(co,105);
                E=pointarc(co,225);
                trace segment(E,B) dashed evenly;
                trace segment(O,C) dashed evenly;
                label.ulft(btex $O$ etex,O);
                label.rt(btex $C$ etex,C);
                label.urt(btex $B$ etex,B);
                %nomme.top(D);
                label.top(btex $(\mathcal{D})$ etex,D);
                label.llft(btex $E$ etex,E);
            \end{Geometrie}
        \end{center}
    \end{minipage}
    \vspace*{-5mm}
\end{exemple*1}
\begin{methode}[Aire d'un disque]
    \exercice
        \begin{enumerate}
            \item Calculer l'aire d'un disque de rayon \Lg[cm]{3,7}.
            \item \textit{Soit un disque d'aire \Aire[cm]{27}}
            \begin{enumerate}
                \item Calculer la valeur exacte de son rayon.
                \item Donner une valeur approchée au dixième de ce rayon.
            \end{enumerate}
        \end{enumerate}
    \correction
    \begin{enumerate}
        \item $\pi\times \Lg[cm]{3,7}\times \Lg[cm]{3,7}\approx\Aire[cm]{43}$.
        \item \textit{Soit un disque d'aire \Aire[cm]{27}}
        \begin{enumerate}
            \item $R^2=\Aire[cm]{27}\div \pi$ soit $R=\sqrt{27\div \pi}~\Lg[cm]{}$.
            \item $R\approx\Lg[cm]{9.2}$.
        \end{enumerate}
    \end{enumerate}
\end{methode}    
