\begin{exercice*}
    Une gélule a la forme d'un cylindre droit, de longueur \Lg[cm]{1}, avec une demi-sphère collée à chacune de ses bases, de rayon \Lg[mm]{3}.
    \begin{center}
        \AssemblageSolides[Type=CylindreBoules,Anglex=90,HauteurCylindre=1.5,RayonCylindre=0.5]
    \end{center}
    \begin{enumerate}
        \item Reporter sur la figure les longueurs de l'énoncé, exprimées en millimètre.
        \item Calculer le volume total exact de la gélule, puis son volume arrondi à l'unité.
    \end{enumerate}
\end{exercice*}
\begin{corrige}
    Une gélule a la forme d'un cylindre droit, de longueur \Lg[cm]{1}, avec une demi-sphère collée à chacune de ses bases, de rayon \Lg[mm]{3}.

    \begin{center}
        \AssemblageSolides[%
            Type=CylindreBoules,%
            Anglex=90,%
            HauteurCylindre=1.5,%
            RayonCylindre=0.5,%            
            Traces={
                color O,O';
                O=(0,0,0);
                O'-O=(0,-1.5,0);                
                drawoptions(withcolor red);
                trace cotation(Projette(O'),Projette(O),-15mm,-2mm,TEX("\Lg[mm]{10}"));
                color A,B,C,D;
                C=(0.5*cosd(110),0,0.5*sind(110));
                D=(-0.5*cosd(110),0,-0.5*sind(110));
                A=(-0.5*cosd(110),-1.5,-0.5*sind(110));
                B=(0.5*cosd(110),-1.5,0.5*sind(110));
                trace chemin(C,D) dashed withdots;
                trace chemin(A,B) dashed withdots;
                trace cotation(Projette(C),Projette(D),15mm,-2mm,TEX("\Lg[mm]{6}"));
                trace cotation(Projette(A),Projette(B),15mm,-2mm,TEX("\Lg[mm]{6}"));
            }
        ]
    \end{center}

    \begin{enumerate}
        \item Reporter sur la figure les longueurs de l'énoncé, exprimées en millimètre.
        
        \textcolor{red}{cf figure}
    \end{enumerate}
    \Coupe
    \begin{enumerate}
        \setcounter{enumi}{1}
        \item Calculer le volume total exact de la gélule, puis son volume arrondi à l'unité.
        
        {\color{red}%
        $\mathcal{V}_{\text{cylindre}}=\pi r^2 h=\pi\times3^2\times10=90\pi~\Vol[mm]{}$.
        
        $\mathcal{V}_{\text{demi-boule}}=\dfrac12\times\dfrac43\pi r^3=\dfrac12\times\dfrac43\pi\times3^3=18\pi~\Vol[mm]{}$.

        $\mathcal{V}_{\text{gélule}}=\mathcal{V}_{\text{cylindre}}+\mathcal{V}_{\text{demi-boule}}$

        $\mathcal{V}_{\text{gélule}}=90\pi+2\times 18\pi=126\pi~\Vol[mm]{}$

        $\mathcal{V}_{\text{gélule}}\approx \Vol[mm]{396}$ au \Vol[mm]{} près.
        }
    \end{enumerate}
\end{corrige}
