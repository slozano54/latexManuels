\begin{exercice*}
    Sur ce globe, déterminer les villes qui se trouvent entre :
    \begin{enumerate}
        \item l'équateur et la latitude \ang{20}N,
        \item le méridien de Greenwich et la longitude \ang{30}O.
    \end{enumerate}
    \begin{center}
        % \includegraphics[scale=0.4]{\currentpath/images/exosBase028.png}    
        % nombre de points
        %(longitude,latitude),"texte",position du texte
        \Cartographie[%
            Impression,Echelle=2.5,Maillage,Allegee=10,
            VillesI={%
                14,
                (-21,64),"\noexpand\tiny Reykjavik",6,
                (0,51),"\noexpand\tiny Londres",6,
                (37,55),"\noexpand\tiny Moscou",2,
                (-35,-5),"\noexpand\tiny Natal",1,
                (18,-33),"\noexpand\tiny Le Cap",8,
                (-17,14),"\noexpand\tiny Dakar",5,
                (31,30),"\noexpand\tiny Le Caire",6,
                (55,-20),"\noexpand\tiny Saint Denis la Réunion",2,
                (69,34),"\noexpand\tiny Kaboul",2,
                (-9,38),"\noexpand\tiny Lisbonne",6,
                (72,19),"\noexpand\tiny Bombay",2,
                (9,0),"\noexpand\tiny Libreville",3,
                (0,-45),"\noexpand\tiny\textcolor{LightSteelBlue}{Méridien de Greenwich}",7,
                (70,0),"\noexpand\tiny\textcolor{red}{Équateur}",1
            }
        ]{5}{20}
    \end{center}    
\end{exercice*}
\begin{corrige}
    Sur ce globe, déterminer les villes qui se trouvent entre :

    \begin{enumerate}
        \item l'équateur et la latitude \ang{20}N,
        
        \textcolor{red}{Les villes situées entre l'équateur et la latitude \ang{20}N sont Libreville, Bombay et Dakar.}
        \item le méridien de Greenwich et la longitude \ang{30}O.
        
        \textcolor{red}{Les villes situées entre le méridien de Greenwich et la longitude \ang{30}O sont Dakar, Lisbonne, Londres et Reykjavik.}
    \end{enumerate}
    \begin{center}
        % \includegraphics[scale=0.3]{\currentpath/images/exosBase028.png}
        % nombre de points
        %(longitude,latitude),"texte",position du texte
        \Cartographie[%
            Impression,Echelle=1.5,Maillage,Allegee=10,
            VillesI={%
                14,
                (-21,64),"\noexpand\tiny Reykjavik",6,
                (0,51),"\noexpand\tiny Londres",6,
                (37,55),"\noexpand\tiny Moscou",2,
                (-35,-5),"\noexpand\tiny Natal",1,
                (18,-33),"\noexpand\tiny Le Cap",8,
                (-17,14),"\noexpand\tiny Dakar",5,
                (31,30),"\noexpand\tiny Le Caire",6,
                (55,-20),"\noexpand\tiny Saint Denis la Réunion",2,
                (69,34),"\noexpand\tiny Kaboul",2,
                (-9,38),"\noexpand\tiny Lisbonne",6,
                (72,19),"\noexpand\tiny Bombay",2,
                (9,0),"\noexpand\tiny Libreville",3,
                (0,-45),"\noexpand\tiny\textcolor{LightSteelBlue}{Méridien de Greenwich}",7,
                (70,0),"\noexpand\tiny\textcolor{red}{Équateur}",1
            }
        ]{5}{20}
    \end{center}
\end{corrige}
