\begin{exercice*}
    % assemblage cubes et prisme
    \begin{mplibcode}
        input PfCSolid;
        nb:=1;
        subh:=1;
        Initialisation(1500,60,20,50);
        Objetcube1("a=0.5");
        TR:=(0,0.5,0);
        ObjetDeplacement2(1);
        TR:=(0,1,0);
        ObjetDeplacement3(1);
        TR:=(0,1.5,0);
        ObjetDeplacement4(1);
        TR:=(0.5,1.5,0);
        ObjetDeplacement5(1);
        TR:=(0,0,0.5);
        ObjetDeplacement6(1);
        color A,B,C;
        A=(0,0,0);
        B=(0,0.5,0);
        C=(-0.5,0,0);
        TR:=(0.25,1.25,0.25);
        ObjetPrisme7("axe=(0,0,1)","h=0.25")(A,B,C);
        TR:=(0,0,0);
        nbobj:=7;
        DessineFusion;
    \end{mplibcode}

    % gelule
    \AssemblageSolides[Type=CylindreBoules,Anglex=90,HauteurCylindre=1.5,RayonCylindre=0.5]

    % coneBoule
    \AssemblageSolides[Type=ConeBoule,HauteurCone=1.5,RayonCone=0.5]

    \Solide[%
        Nom=pave,
        Largeur=2,
        Hauteur=4,
        Profondeur=1.5,
        ListeSommets={A,B,C,D,L,I,J,K},
        Aretes=false,
        Traces={
            color S,E,F,G,H,O;
            S=(0.75,1,2.5);
            E-A=(0,0,1.5);
            F-B=(0,0,1.5);
            G-C=(0,0,1.5);
            H-D=(0,0,1.5);
            O=(0.75,1,3.25);
            DotLabel.top(btex S etex,S);
            Label.lft(btex E etex,E);
            Label.lrt(btex F etex,F);
            Label.lrt(btex G etex,G);
            Label.lrt(btex H etex,H);
            DotLabel.urt(btex O etex,O);
            % color S[];
            % trace myCircl;            
            % %https://metapost.gutenberg-asso.fr/?page=exemples&id=267
            % s:=0.01;
            % for i=0 step s until 1:
            %     S[i]=(0.75,1,2.5+i);
            %     trace cercles(O,S[i],A,C,I);
            %     fill cercles(O,S[i],A,C,I) withcolor (1-i)*white;
            %     %fill myCircl shifted (i*(S-O)) withcolor i*white;
            path myCircl;
            myCircl = cercles(O,S,A,C,I);
            pas:=100;
            for k=0 upto pas:
                fill homothetie(myCircl,Projette(O),(pas-k)/pas) withcolor ((pas-k)/pas)[white,LightGray];
            endfor;
            trace myCircl;
            % fill cercles(O,S,A,C,I) withcolor LightGray;
            trace chemin(A,B,C,K,L,I,J,K);
            trace chemin(I,A);
            trace chemin(E,F,G);
            trace chemin(F,B);
            trace chemin(A,D,C) dashed evenly;
            trace chemin(E,S,H) dashed evenly;
            trace chemin(F,S,G) dashed evenly;
            trace chemin(E,H,G) dashed evenly;
            trace chemin(H,D) dashed evenly;
        }
    ]

    \begin{Geometrie}[Cadre="aucun"]
        pair O;
        O=(0,0);
      
        vardef Gradient(expr chem,PtC,coulA,coulB,pas)=
        save $;
        picture $;
      
        $=image(
        for k=0 upto pas:
        fill homothetie(chem,PtC,(pas-k)/pas) withcolor ((pas-k)/pas)[coulA,coulB];
        endfor;
        );
        $
        enddef;
      
        path cc;
        cc=cercles(O,3u);
        trace Gradient(cc,O,white,Crimson,100);
        trace cc;
      
        path dd;
      
        dd=u*(4,0)--u*(8,0)--u*(8,4)--u*(4,4)--cycle;
        trace Gradient(dd,u*(5,1),LightSteelBlue,Purple,100);
        trace dd;
      \end{Geometrie}

      \begin{Geometrie}[Cadre="aucun"]
        vardef Gradient(expr chem,PtC,coulA,coulB,pas)=
            save $;
            picture $;      
            $=image(
                for k=0 upto pas:
                    fill homothetie(chem,PtC,(pas-k)/pas) withcolor ((pas-k)/pas)[coulA,coulB];
                endfor;
            );
            $
        enddef;
        pair A[],B[],O;
        A0=u*(1,1);
        A1-A0=u*(5,0);
        A2-A1=u*(1.5,1.2);
        A3-A2=u*(-5,0);
        draw A0--A1--A2;
        draw A2--A3--A0 dashed evenly;
        B0-A0=u*(0,1.8);
        B1-A1=u*(0,1.8);
        B2-A2=u*(0,1.8);
        B3-A3=u*(0,1.8);
        draw B0--B1--B2--B3--cycle;
        for k=0 upto 2:
            draw A[k]--B[k];
        endfor;
        draw A[3]--B[3] dashed evenly;
        O=iso(B0,B1,B2,B3);
        path creux;
        creux = fullcircle scaled 4u; 
        draw (subpath (0.5*(length creux),length creux) of creux) shifted O dashed evenly;
        path section;
        section=(creux yscaled 0.25) shifted O;
        trace Gradient(section,O,white,LightGray,100);                
        draw section;
        trace cotationmil(A0,A1,-3mm,15,btex \Lg[cm]{10} etex);
        trace cotation(A0,B0,3mm,3mm,btex \Lg[cm]{4} etex);        
        trace cotationmil(point 0.5*(length section) of section,point 1*(length section) of section,-0mm,15,btex \Lg[cm]{8} etex);
      \end{Geometrie}

      \begin{mplibcode}
        input PfCSolid;
        nb:=1;
        subh:=1;
        incolor:=0.5[LightGray,blanc] ;
        outcolor:=0.5[LightGray,blanc] ;
        Initialisation(1500,60,20,50);        
        Objetcube1("a=1");        
        %
        Objetcube2("a=0.5");
        %
        TR:=(0.75,0.25,0.25);
        ObjetDeplacement3(2);
        TR:=(0.25,-0.75,0.75);
        ObjetDeplacement4(2);
        TR:=(-0.25,0.75,-0.25);
        ObjetDeplacement5(2);
        color Dessus,Face,Droite;
        Dessus=(1,2,2.5);
        Face=(2.5,1,0.5);
        Droite=(2,6,1);
        drawarrow Projette(Dessus+(0,0,-0.2))--Projette(Dessus+(0,0,-0.6)) withpen pencircle scaled 1bp;
        Label(btex Vue de dessus etex,Dessus);
        drawarrow Projette(Face+(-0.4,0,0))--Projette(Face+(-0.8,0,0)) withpen pencircle scaled 1bp;
        Label(btex Vue etex,Face+(0,0,0.2));
        Label(btex de face etex,Face);
        drawarrow Projette(Droite+(0,-0.4,0))--Projette(Droite+(0,-1,0)) withpen pencircle scaled 1bp;
        Label(btex Vue etex,Droite+(0,0,0.2));
        Label(btex de droite etex,Droite);
        nbobj:=5;        
        DessineFusion;
    \end{mplibcode}
\end{exercice*}
\begin{corrige}
    ...
\end{corrige}
