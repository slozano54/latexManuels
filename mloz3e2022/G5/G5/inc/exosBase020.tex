\begin{exercice*}
    $EABC$ est un tétraèdre tel que $AB=\Lg[cm]{12}$ ; $BC=\Lg[cm]{8}$ et $BE=\Lg[cm]{16}$.

    \begin{minipage}{0.65\linewidth}
        $MNP$ est la section de la pyramide par un plan, parallèle à la base, passant par le point $N$ de $[EB]$ tel que $EN=\Lg[cm]{6,4}$.
    \end{minipage}
    \hfill        
    \begin{minipage}{0.3\linewidth}
        \begin{center}
            \scalebox{0.8}{
                \begin{Geometrie}
                    pair A,B,C,E,M,N,P;
                    numeric SLCoeff;
                    SLCoeff:=0.5;
                    path SLSection;
                    A=u*(1,1);
                    B-A=u*(1.5,0.75);
                    C-A=u*(3,0.5);
                    E-B=u*(0,3);                    
                    M=SLCoeff[E,A];
                    N=SLCoeff[E,B];
                    P=SLCoeff[E,C];
                    SLSection=M--N--P--cycle;
                    fill SLSection withcolor PaleTurquoise;
                    draw A--B--C dashed evenly;
                    draw A--E--C--cycle;
                    draw E--B dashed evenly;
                    draw M--N--P dashed evenly;
                    draw M--P;
                    trace codeperp(A,B,C,5);
                    trace codeperp(E,B,A,5);
                    trace codeperp(C,B,E,5);
                    label.top(btex $E$ etex,E);
                    label.llft(btex $A$ etex,A);
                    label.ulft(btex $B$ etex,B);
                    label.lrt(btex $C$ etex,C);
                    label.lft(btex $M$ etex,M);
                    label.ulft(btex $N$ etex,N);
                    label.rt(btex $P$ etex,P);
                \end{Geometrie}
            }
        \end{center}
    \end{minipage}
    \begin{enumerate}
        \item Déterminer la nature du triangle $MNP$.
        \item Calculer la valeur exacte de $MN$.
        \item Calculer la valeur exacte de $NP$.
        \item Tracer le triangle $MNP$ en vraie grandeur.
        \item Calculer la valeur exacte de $MP$.
    \end{enumerate}
\end{exercice*}
\begin{corrige}
    $EABC$ est un tétraèdre tel que $AB=\Lg[cm]{12}$ ; $BC=\Lg[cm]{8}$ et $BE=\Lg[cm]{16}$.

    \begin{minipage}{0.65\linewidth}
        $MNP$ est la section de la pyramide par un plan, parallèle à la base, passant par le point $N$ de $[EB]$ tel que $EN=\Lg[cm]{6,4}$.
    \end{minipage}
    \hspace*{-5mm}
    \begin{minipage}{0.3\linewidth}
        \begin{center}
            \scalebox{0.8}{
                \begin{Geometrie}
                    pair A,B,C,E,M,N,P;
                    numeric SLCoeff;
                    SLCoeff:=0.5;
                    path SLSection;
                    A=u*(1,1);
                    B-A=u*(1.5,0.75);
                    C-A=u*(3,0.5);
                    E-B=u*(0,3);                    
                    M=SLCoeff[E,A];
                    N=SLCoeff[E,B];
                    P=SLCoeff[E,C];
                    SLSection=M--N--P--cycle;
                    fill SLSection withcolor PaleTurquoise;
                    draw A--B--C dashed evenly;
                    draw A--E--C--cycle;
                    draw E--B dashed evenly;
                    draw M--N--P dashed evenly;
                    draw M--P;
                    trace codeperp(A,B,C,5);
                    trace codeperp(E,B,A,5);
                    trace codeperp(C,B,E,5);
                    label.top(btex $E$ etex,E);
                    label.llft(btex $A$ etex,A);
                    label.ulft(btex $B$ etex,B);
                    label.lrt(btex $C$ etex,C);
                    label.lft(btex $M$ etex,M);
                    label.ulft(btex $N$ etex,N);
                    label.rt(btex $P$ etex,P);
                \end{Geometrie}
            }
        \end{center}
    \end{minipage}

    \vspace*{5mm}
    \begin{enumerate}
        \item Déterminer la nature du triangle $MNP$.
        
        {\color{red} Le triangle $MNP$ est une section du tétraèdre parallèle à la face $ABC$, il est donc de même nature que le triangle $ABC$, c'est donc un triangle rectangle en $N$.}
    \end{enumerate}
    \Coupe
    \begin{enumerate}
        \setcounter{enumi}{1}
        \item Calculer la valeur exacte de $MN$.
        
        {\color{red} Les droites $(MA)$ et $(NB)$ sont sécantes en $E$ et les droites $(MN)$ et $(AB)$ sont parallèles, le théorème de Thalès garantit donc que l'on peut écrire
        l'égalité $\dfrac{EA}{EM}=\dfrac{EB}{EN}=\dfrac{AB}{MN}$ d'où $\dfrac{16}{\num{6.4}}=\dfrac{12}{MN}$ et donc $MN=\dfrac{12\times\num{6.4}}{16}=\Lg[cm]{4.8}$}
        \item Calculer la valeur exacte de $NP$.
        
        {\color{red} Les droites $(PC)$ et $(NB)$ sont sécantes en $E$ et les droites $(PN)$ et $(BC)$ sont parallèles, le théorème de Thalès garantit donc que l'on peut écrire
        l'égalité $\dfrac{EB}{EN}=\dfrac{EC}{EP}=\dfrac{BC}{PN}$ d'où $\dfrac{PN}{8}=\dfrac{\num{6.4}}{16}$ et donc $PN=\dfrac{8\times\num{6.4}}{16}=\Lg[cm]{3.2}$}
        \item Tracer le triangle $MNP$ en vraie grandeur.
        
        \begin{Geometrie}
            pair M,N,P;
            N=u*(1,1);
            M-N=u*(0,4.8);
            P-N=u*(3.2,0);
            drawoptions(withcolor red);
            draw M--N--P--cycle;
            trace codeperp(P,N,M,5);
            trace cotationmil(N,P,-5mm,15,btex \Lg[cm]{3.2} etex);
            trace cotationmil(N,M,5mm,15,btex \Lg[cm]{4.8} etex);
            label.top(btex $M$ etex,M);
            label.llft(btex $N$ etex,N);
            label.lrt(btex $P$ etex,P);
        \end{Geometrie}
        \item Calculer la valeur exacte de $MP$.
        
        {\color{red} $MNP$ est un triangle rectangle en $N$, le théorème de Pythagore garantit que $MN^2+NP^2=MP^2$ soit $MP^2=\num{4.8}^2+\num{3.2}^2=\num{33.28}$, $MP$ étant une longueur, $MP$ est une nombre positif
        donc $MP=\sqrt{\num{33.28}}~\Lg[cm]{}$}
    \end{enumerate}
\end{corrige}

