\begin{exercice*}
    Soit un cube d'arête \Lg[cm]{5}.
    \begin{enumerate}
        \item Déterminer l'aire totale de sa surface en \Aire[cm]{}, c'est-à-dire la surface composée par ses 6 faces.
        \item Calculer le volume de ce cube, en \Vol[cm]{}.
        \item Un autre cube a une surface totale 16 fois plus grande. Déterminer son volume, en \Vol[cm]{}.
    \end{enumerate}    
\end{exercice*}
\begin{corrige}
    Soit un cube d'arête \Lg[cm]{5}.
    \begin{enumerate}
        \item Déterminer l'aire totale de sa surface en \Aire[cm]{}, c'est-à-dire la surface composée par ses 6 faces.
        
        {\color{red}L'aire d'un face valant $\Lg[cm]{5}\times\Lg[cm]{5}$ soit $\Aire[cm]{25}$, l'aire totale vaut $6\times\Aire[cm]{25}=\Aire[cm]{150}$}
        \item Calculer le volume de ce cube, en \Vol[cm]{}.
        
        {\color{red}Le Volume vaut $\Lg[cm]{5}\times\Lg[cm]{5}\times\Lg[cm]{5}$ soit \Vol[cm]{125}.}
        \item Un autre cube a une surface totale 16 fois plus grande. Déterminer son volume, en \Vol[cm]{}.
        
        {\color{red}Un agrandissement de coefficient $k$ multipie les longueurs par $k$ donc les aires le sont par $k^2$.
        Ici, l'aire a été multipliée par 16, $k$ étant un nombre positif, $k=4$. Le volumes sont eux multipliés par $k^3$ donc $\mathcal{V}=4^3\times\mathcal{V}_{\text{premier cube}}=64\times 125$ soit \Vol[cm]{8000}.}
    \end{enumerate}
\end{corrige}
