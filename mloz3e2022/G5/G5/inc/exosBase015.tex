\begin{exercice*}
    Ce moule à gâteau a la forme d'un pavé droit à base carrée dans lequel on a évidé une demi-boule.

    \begin{center}
        \scalebox{0.9}{
            \begin{Geometrie}[Cadre="aucun"]
                vardef Gradient(expr chem,PtC,coulA,coulB,pas)=
                    save $;
                    picture $;      
                    $=image(
                        for k=0 upto pas:
                            fill homothetie(chem,PtC,(pas-k)/pas) withcolor ((pas-k)/pas)[coulA,coulB];
                        endfor;
                    );
                    $
                enddef;
                pair A[],B[],O;
                A0=u*(1,1);
                A1-A0=u*(5,0);
                A2-A1=u*(1.5,1.2);
                A3-A2=u*(-5,0);
                draw A0--A1--A2;
                draw A2--A3--A0 dashed evenly;
                B0-A0=u*(0,1.8);
                B1-A1=u*(0,1.8);
                B2-A2=u*(0,1.8);
                B3-A3=u*(0,1.8);
                draw B0--B1--B2--B3--cycle;
                for k=0 upto 2:
                    draw A[k]--B[k];
                endfor;
                draw A[3]--B[3] dashed evenly;
                O=iso(B0,B1,B2,B3);
                path creux;
                creux = fullcircle scaled 4u; 
                draw (subpath (0.5*(length creux),length creux) of creux) shifted O dashed evenly;
                path section;
                section=(creux yscaled 0.25) shifted O;
                trace Gradient(section,O,white,LightGray,100);                
                draw section;
                trace cotationmil(A0,A1,-3mm,15,btex \Lg[cm]{10} etex);
                trace cotation(A0,B0,3mm,3mm,btex \Lg[cm]{4} etex);        
                trace cotationmil(point 0.5*(length section) of section,point 1*(length section) of section,-0mm,15,btex \Lg[cm]{8} etex);
            \end{Geometrie}
        }
    \end{center}
    \vspace*{-7mm}
    \begin{enumerate}
        \item Calculer le volume de plastique nécessaire pour fabriquer ce moule, arrondi au centième de \Vol[cm]{}.
        \item Ce moule a servi à Catherine pour faire un gâteau qu'elle veut à présent napper de chocolat. Déterminer la surface de gâteau à recouvrir, arrondie au centième de \Aire[cm]{}.
    \end{enumerate}
\end{exercice*}
\begin{corrige}
    Ce moule à gâteau a la forme d'un pavé droit à base carrée dans lequel on a évidé une demi-boule.

    \begin{center}
        \scalebox{0.9}{
        \begin{Geometrie}[Cadre="aucun"]
            vardef Gradient(expr chem,PtC,coulA,coulB,pas)=
                save $;
                picture $;      
                $=image(
                    for k=0 upto pas:
                        fill homothetie(chem,PtC,(pas-k)/pas) withcolor ((pas-k)/pas)[coulA,coulB];
                    endfor;
                );
                $
            enddef;
            pair A[],B[],O;
            A0=u*(1,1);
            A1-A0=u*(5,0);
            A2-A1=u*(1.5,1.2);
            A3-A2=u*(-5,0);
            draw A0--A1--A2;
            draw A2--A3--A0 dashed evenly;
            B0-A0=u*(0,1.8);
            B1-A1=u*(0,1.8);
            B2-A2=u*(0,1.8);
            B3-A3=u*(0,1.8);
            draw B0--B1--B2--B3--cycle;
            for k=0 upto 2:
                draw A[k]--B[k];
            endfor;
            draw A[3]--B[3] dashed evenly;
            O=iso(B0,B1,B2,B3);
            path creux;
            creux = fullcircle scaled 4u; 
            draw (subpath (0.5*(length creux),length creux) of creux) shifted O dashed evenly;
            path section;
            section=(creux yscaled 0.25) shifted O;
            trace Gradient(section,O,white,LightGray,100);                
            draw section;
            trace cotationmil(A0,A1,-3mm,15,btex \Lg[cm]{10} etex);
            trace cotation(A0,B0,3mm,3mm,btex \Lg[cm]{4} etex);        
            trace cotationmil(point 0.5*(length section) of section,point 1*(length section) of section,-0mm,15,btex \Lg[cm]{8} etex);
          \end{Geometrie}
        }
    \end{center}
    \begin{enumerate}
        \item Calculer le volume de plastique nécessaire pour fabriquer ce moule, arrondi au centième de \Vol[cm]{}.
        
        {\color{red} $\mathcal{V}_{\text{moule}}=\mathcal{V}_{\text{pavé}} - \mathcal{V}_{\text{demi-boule}}$

        $\mathcal{V}_{\text{moule}}=10\times 10\times 4 - \dfrac12\times\dfrac43\pi\times 4^3$

        $\mathcal{V}_{\text{moule}}=400 - \dfrac{\num{128}\pi}{3}~\Vol[cm]{}\approx\Vol[cm]{266}$
        }
    \end{enumerate}
    \Coupe
    \begin{enumerate}
        \setcounter{enumi}{1}
        \item Ce moule a servi à Catherine pour faire un gâteau qu'elle veut à présent napper de chocolat. Déterminer la surface de gâteau à recouvrir, arrondie au centième de \Aire[cm]{}.
        
        {\color{red} $\dfrac12\times 4\pi \times 4^2=32\pi~\Aire[cm]{}\approx\Aire[cm]{100.53}$

        La surface recouvrir est donc d'environ $\Aire[cm]{100.53}$.
        }
    \end{enumerate}
\end{corrige}
