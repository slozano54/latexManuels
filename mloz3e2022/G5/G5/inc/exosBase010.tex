\begin{exercice*}
    $ABCDEFGH$ est un cube d’arête $AB = \Lg[cm]{12}$.
    
    \smallskip
    \begin{minipage}{0.6\linewidth}
        \begin{itemize}
            \item $I$ est le milieu du segment $[AB]$;
            \item $J$ est le milieu du segment $[AE]$;
            \item $K$ est le milieu du segment $[AD]$.
        \end{itemize}    
    \end{minipage}
    \hfill
    \begin{minipage}{0.35\linewidth}
        \Solide[%
            Phi=20,
            ListeSommets={E,H,G,F,B,A,D,C},
            Traces={
                color I,J,K;
                I=iso(A,B);
                J=iso(A,E);
                K=iso(A,D);
                Label.ulft(btex $I$ etex,I);                
                Label.lft(btex $J$ etex,J);
                Label.urt(btex $K$ etex,K);
                trace chemin(I,K,J);
                trace chemin(J,I) dashed evenly;
            }
        ]
    \end{minipage}
    \begin{enumerate}
        \item Calculer l’aire du triangle $AIK$.
        \item Calculer le volume de la pyramide $AIKJ$ de base $AKI$.
        \item Déterminer la fraction du volume du cube que représente le volume de la pyramide $AIKJ$. Écrire le résultat
        sous forme d’une fraction de numérateur 1.
    \end{enumerate}
\end{exercice*}
\begin{corrige}
    $ABCDEFGH$ est un cube d’arête $AB = \Lg[cm]{12}$.
    
    \smallskip
    \begin{minipage}{0.5\linewidth}
        \begin{itemize}
            \item $I$ est le milieu de $[AB]$;
            \item $J$ est le milieu de $[AE]$;
            \item $K$ est le milieu de $[AD]$.
        \end{itemize}    
    \end{minipage}
    \hfill
    \begin{minipage}{0.35\linewidth}
        \Solide[%
            Phi=20,
            ListeSommets={E,H,G,F,B,A,D,C},
            Traces={
                color I,J,K;
                I=iso(A,B);
                J=iso(A,E);
                K=iso(A,D);
                Label.ulft(btex $I$ etex,I);                
                Label.lft(btex $J$ etex,J);
                Label.urt(btex $K$ etex,K);
                trace chemin(I,K,J);
                trace chemin(J,I) dashed evenly;
            }
        ]
    \end{minipage}

    \begin{enumerate}
        \item Calculer l’aire du triangle $AIK$.
        \\\textcolor{red}{$AIK$ est un triangle rectangle en $A$ tel que $AI=AK=\Lg[cm]{6}$ donc son aire vaut $\mathcal{A}_{AIK}=6\times 6 \div 2 =\Aire[cm]{18}$}
        \item Calculer le volume de la pyramide $AIKJ$ de base $AKI$.
        \\{\color{red}
        La hauteur de la pyramide $AIKJ$ est $AJ=\Lg[cm]{6}$ donc son volume vaut $\mathcal{V}_{AIKJ}=\text{Aire de la base}\times\text{hauteur}\div3$
        
        $\mathcal{V}_{AIKJ}=18\times 6 \div 3 = \Vol[cm]{36}$}
    \end{enumerate}
    \Coupe
    \begin{enumerate}
        \setcounter{enumi}{2}
        \item Déterminer la fraction du volume du cube que représente le volume de la pyramide $AIKJ$.
        
        Écrire le résultat sous forme d’une fraction de numérateur 1.
        \\{\color{red}{
            Le volume du cube vaut $\mathcal{V}_{\text{cube}}=12^3=\Vol[cm]{1728}$,
         
            \smallskip
            d'où la fraction $\dfrac{36}{\num{1728}}=\dfrac{1}{48}$}
        }
    \end{enumerate}
\end{corrige}
