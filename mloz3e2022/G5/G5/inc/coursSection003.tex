\section{Sections de solides}
\subsection{La sphère}
\begin{remarque}
    lorsqu'elle existe, la section d'une sphère par un plan est un cercle.
\end{remarque}

\begin{changemargin}{-5mm}{-5mm}
    \begin{propriete}[Disjonction des cas]
        Soit $(\mathcal{P})$ un plan perpendiculaire en $H$ à l'un des diamètres $[NS]$ d'une sphère de rayon $r$. 

        $OH$ est {\bf la distance du point $O$ au plan $(\mathcal{P})$}.
        \columnseprule0.4pt
        \begin{multicols}{3}
        \centerline{\underline{$0<OH<r$}}
        \par Le cercle de section a pour centre $H$.

        \vspace*{10mm}
        \scalebox{0.7}{
        $$\begin{Geometrie}
            u:=30;
            z0=(3,3)*u;
            dotlabel.urt(btex $O$ etex,z0);
            z1=(3.5,1)*u;
            label.lrt(btex $S$ etex,z1);
            z2=(2.5,5)*u;
            label.urt(btex $N$ etex,z2);
            % remplissages ici avant les path
            fill fullcircle scaled 4.1231u scaled 0.88 yscaled 0.15 rotated 14 shifted (z0+ unitvector(z2-z1)*u) withcolor Cornsilk;
            % fin remplissages    
            path contour, ellipse,p,q,r,s,coupe;
            contour=(fullcircle scaled 4.1231u);
            ellipse=(contour yscaled 0.25) rotated 14;    
            coupe=(contour scaled 0.88 yscaled 0.15) rotated 14;
            p=(0.5,4)*u--(0.5,4)*u + 0.45*((6,5.5)*u-(0.5,4)*u);
            s=(6,5.5)*u--(0.5,4)*u + 0.45*((6,5.5)*u-(0.5,4)*u);
            r=contour shifted z0;
            draw z0--point 0 of ellipse shifted z0 dashed evenly;
            label.bot(btex $r$ etex,0.5[z0,point 0 of ellipse shifted z0]);
            q=(0.5,4)*u--(0,2.5)*u--(5.5,4)*u--(6,5.5)*u;
            draw contour shifted z0;
            draw (subpath (0.5*(length ellipse),length ellipse) of ellipse) shifted z0;
            draw (subpath (0,0.5*(length ellipse)) of ellipse) shifted z0 dashed evenly;
            draw q;
            draw z1--z2 dashed evenly;
            draw z1--z1 + 0.08(z1-z2);
            draw z2--z2+0.08(z2-z1);
            dotlabel.rt(btex $H$ etex,z0+unitvector(z2-z1)*u);
            dotlabel.ulft(btex $M$ etex,point 4 of (coupe shifted(z0+unitvector(z2-z1)*u)));
            draw point 4 of (coupe shifted(z0+ unitvector(z2-z1)*u))--z0 dashed evenly;
            draw point 4 of (coupe shifted(z0+ unitvector(z2-z1)*u))--z0+ unitvector(z2-z1)*u dashed evenly;
            draw (subpath (0.5*(length coupe),length coupe) of coupe) shifted (z0+ unitvector(z2-z1)*u);
            label.urt(btex $(\mathcal{P})$ etex,(0,2.5)*u);
            z6= point 4 of (coupe shifted(z0+ unitvector(z2-z1)*u));
            z7= z0+ unitvector(z2-z1)*u;
            draw (subpath (0,0.5*(length coupe)) of coupe) shifted (z0+ unitvector(z2-z1)*u) dashed evenly;
            draw codeperp(z6,z7,z0,5);
            draw (0.5,4)*u--(p intersectionpoint r);draw (p intersectionpoint r)--(s intersectionpoint r) dashed evenly;
            draw (s intersectionpoint r)--(6,5.5)*u;
        \end{Geometrie}$$
        }
        \vskip2.5cm
        \par\centerline{\underline{$OH=0$}}
        \par
        Le cercle de section a le même centre $O$ et le même rayon $r$ que la sphère : on dit que c'est un {\bf grand cercle} de la sphère.

        \scalebox{0.7}{
            $$\begin{Geometrie}
                u:=30;
                z0=(3,3)*u;
                dotlabel.urt(btex $O$ etex,z0);
                z1=(3.5,1)*u;
                label.lrt(btex $S$ etex,z1);
                z2=(2.5,5)*u;
                label.urt(btex $N$ etex,z2);
                % remplissages ici avant les path
                fill fullcircle scaled 4.1231u yscaled 0.25 rotated 14 shifted (z0) withcolor Cornsilk;
                % fin remplissages    
                path contour, ellipse,p,q,r,s,coupe;
                contour=(fullcircle scaled 4.1231u);
                ellipse=(contour yscaled 0.25) rotated 14;    
                coupe=(contour scaled 0.88 yscaled 0.15) rotated 14;
                p=(0.5,4)*u--(0.5,4)*u + 0.45*((6,5.5)*u-(0.5,4)*u);
                s=(6,5.5)*u--(0.5,4)*u + 0.45*((6,5.5)*u-(0.5,4)*u);
                r=contour shifted z0;
                draw z0--point 0 of ellipse shifted z0 dashed evenly;
                label.bot(btex $r$ etex,0.5[z0,point 0 of ellipse shifted z0]);
                q=(0.5,4)*u--(0,2.5)*u--(5.5,4)*u--(6,5.5)*u;
                draw contour shifted z0;
                draw (subpath (0.5*(length ellipse),length ellipse) of ellipse) shifted z0;
                draw (subpath (0,0.5*(length ellipse)) of ellipse) shifted z0 dashed evenly;
                draw q shifted (-unitvector(z2-z1)*u);
                draw z1--z2 dashed evenly;
                draw z1--z1 + 0.08(z1-z2);
                draw z2--z2+0.08(z2-z1);
                dotlabel.llft(btex $H$ etex,z0);
                dotlabel.urt(btex $O$ etex,z0);
                label.urt(btex $(\mathcal{P})$ etex,(0,2.5)*u shifted (-unitvector(z2-z1)*u));
                draw codeperp(z1,z0,0.5[z0,point 0 of ellipse shifted z0],5);
                draw (0.5,4)*u shifted (-unitvector(z2-z1)*u)--(p shifted (-unitvector(z2-z1)*u) intersectionpoint r);
                draw (p shifted (-unitvector(z2-z1)*u) intersectionpoint r)--(s shifted (-unitvector(z2-z1)*u) intersectionpoint r) dashed evenly;
                draw (s shifted (-unitvector(z2-z1)*u) intersectionpoint r)--(6,5.5)*u shifted (-unitvector(z2-z1)*u);
            \end{Geometrie}$$
        }
        \vskip1.5cm
        \par\centerline{\underline{$OH=r$}}
        \par
        Le cercle de section a pour centre $S$ et pour rayon $0$. On dit que {\bf le plan $(\mathcal{P})$ est tangent à la sphère en $S$}.\par

        \scalebox{0.7}{
            $$\begin{Geometrie}[CoinBG={(0,-u)}]
                u:=30;
                z0=(3,3)*u;
                dotlabel.urt(btex $O$ etex,z0);
                z1=(3.5,1)*u;
                label.lrt(btex $S$ etex,z1);
                z2=(2.5,5)*u;
                label.urt(btex $N$ etex,z2);
                % remplissages ici avant les path
                % fill fullcircle scaled 4.1231u scaled 0.88 yscaled 0.15 rotated 14 shifted (z0+ unitvector(z2-z1)*u) withcolor Cornsilk;
                % fin remplissages    
                path contour, ellipse,p,q,r,s,coupe;
                contour=(fullcircle scaled 4.1231u);
                ellipse=(contour yscaled 0.25) rotated 14;    
                coupe=(contour scaled 0.88 yscaled 0.15) rotated 14;
                p=(0.5,4)*u--(0.5,4)*u + 0.45*((6,5.5)*u-(0.5,4)*u);
                s=(6,5.5)*u--(0.5,4)*u + 0.45*((6,5.5)*u-(0.5,4)*u);
                r=contour shifted z0;
                draw z0--point 0 of ellipse shifted z0 dashed evenly;
                label.bot(btex $r$ etex,0.5[z0,point 0 of ellipse shifted z0]);
                q=(0.5,4)*u--(0,2.5)*u--(5.5,4)*u--(6,5.5)*u;
                draw contour shifted z0;
                draw (subpath (0.5*(length ellipse),length ellipse) of ellipse) shifted z0;
                draw (subpath (0,0.5*(length ellipse)) of ellipse) shifted z0 dashed evenly;
                draw q shifted (-unitvector(z2-z1)*3u);
                draw z1--z2 dashed evenly;
                draw z1--z1 + 0.08(z1-z2);
                draw z2--z2+0.08(z2-z1);
                dotlabel.llft(btex $H$ etex,z1);
                label.urt(btex $(\mathcal{P})$ etex,(0,2.5)*u shifted (-unitvector(z2-z1)*3u));
                z6= point 4 of (coupe shifted(z0+ unitvector(z2-z1)*u));
                z7= z0+ unitvector(z2-z1)*u;
                draw codeperp(s shifted (-unitvector(z2-z1)*3u) intersectionpoint r,z1,z0,5);
                draw z1--s shifted (-unitvector(z2-z1)*3u) intersectionpoint r dashed evenly;
                draw (0.5,4)*u shifted (-unitvector(z2-z1)*3u)--(p shifted (-unitvector(z2-z1)*3u) intersectionpoint r);
                draw (p shifted (-unitvector(z2-z1)*3u) intersectionpoint r)--(s shifted (-unitvector(z2-z1)*3u) intersectionpoint r) dashed evenly;
                draw (s shifted (-unitvector(z2-z1)*3u) intersectionpoint r)--(6,5.5)*u shifted (-unitvector(z2-z1)*3u);
            \end{Geometrie}$$
        }
        \end{multicols}
        \columnseprule0pt
    \end{propriete}
\end{changemargin}
\begin{exemple*1}
    \titreExemple{Utilisation du vocabulaire}
    
    \begin{minipage}{0.4\linewidth}
        \begin{center}
            \scalebox{0.7}{
                $$\begin{Geometrie}
                    u:=30;
                    z0=(3,3)*u;
                    dotlabel.urt(btex $O$ etex,z0);
                    z1=(3.5,1)*u;
                    label.lrt(btex $S$ etex,z1);
                    z2=(2.5,5)*u;
                    label.urt(btex $N$ etex,z2);
                    % remplissages ici avant les path
                    fill fullcircle scaled 4.1231u scaled 0.88 yscaled 0.15 rotated 14 shifted (z0+ unitvector(z2-z1)*u) withcolor Cornsilk;
                    % fin remplissages    
                    path contour, ellipse,p,q,r,s,coupe;
                    contour=(fullcircle scaled 4.1231u);
                    ellipse=(contour yscaled 0.25) rotated 14;    
                    coupe=(contour scaled 0.88 yscaled 0.15) rotated 14;
                    p=(0.5,4)*u--(0.5,4)*u + 0.45*((6,5.5)*u-(0.5,4)*u);
                    s=(6,5.5)*u--(0.5,4)*u + 0.45*((6,5.5)*u-(0.5,4)*u);
                    r=contour shifted z0;
                    draw z0--point 0 of ellipse shifted z0 dashed evenly;
                    label.bot(btex $r$ etex,0.5[z0,point 0 of ellipse shifted z0]);
                    q=(0.5,4)*u--(0,2.5)*u--(5.5,4)*u--(6,5.5)*u;
                    draw contour shifted z0;
                    draw (subpath (0.5*(length ellipse),length ellipse) of ellipse) shifted z0;
                    draw (subpath (0,0.5*(length ellipse)) of ellipse) shifted z0 dashed evenly;
                    draw q;
                    draw z1--z2 dashed evenly;
                    draw z1--z1 + 0.08(z1-z2);
                    draw z2--z2+0.08(z2-z1);
                    dotlabel.rt(btex $H$ etex,z0+unitvector(z2-z1)*u);
                    dotlabel.ulft(btex $M$ etex,point 4 of (coupe shifted(z0+unitvector(z2-z1)*u)));
                    draw point 4 of (coupe shifted(z0+ unitvector(z2-z1)*u))--z0 dashed evenly;
                    draw point 4 of (coupe shifted(z0+ unitvector(z2-z1)*u))--z0+ unitvector(z2-z1)*u dashed evenly;
                    draw (subpath (0.5*(length coupe),length coupe) of coupe) shifted (z0+ unitvector(z2-z1)*u);
                    label.urt(btex $(\mathcal{P})$ etex,(0,2.5)*u);
                    z6= point 4 of (coupe shifted(z0+ unitvector(z2-z1)*u));
                    z7= z0+ unitvector(z2-z1)*u;
                    draw (subpath (0,0.5*(length coupe)) of coupe) shifted (z0+ unitvector(z2-z1)*u) dashed evenly;
                    draw codeperp(z6,z7,z0,5);
                    draw (0.5,4)*u--(p intersectionpoint r);draw (p intersectionpoint r)--(s intersectionpoint r) dashed evenly;
                    draw (s intersectionpoint r)--(6,5.5)*u;
                \end{Geometrie}$$
            }
        \end{center}
    \end{minipage}
    \begin{minipage}{0.6\linewidth}
        Le plan $(\mathcal{P})$ coupe la sphère de centre $O$ et de rayon $r$ : la section obtenue est un cercle de centre $H$ et de rayon $MH$ que l'on calcule à l'aide du théorème de Pythagore.
    \end{minipage}
    \vspace*{-5mm}
\end{exemple*1}
\begin{propriete}
    \textit{
        \begin{itemize}
            \item Soit une sphère $(\mathscr S)$ de centre $O$ et de rayon $r$.
            \item Soit un plan $(\mathscr P)$ situé à la distance $d$ de $O$, $d<r$, $d=OH$.
            \item H est le pied de la perpendiculaire à $(\mathscr P)$ passant par $O$.
        \end{itemize}
    }
    \par
    \begin{center}
    La section de la sphère $(\mathscr S)$ et du plan $(\mathscr P)$ :\\
    c'est le cercle de centre H et de rayon $MH=\sqrt{r^2-d^2}$
\end{center}
\end{propriete}
\begin{preuve}
    Dans le triangle $MHO$, rectangle en H, la relation de Pythagore s'écrit :
    $$MH^2+HO^2=OM^2$$
    On note $OH=d$ et $OM=r$, on a donc :
    $$MH^2+d^2=r^2$$ 
    d'où $MH^2=r^2-d^2$, or $d<r$ donc $r^2-d^2>0$ ce qui garantit l'existence de sa racine carrée.\\
    d'où $\mathbf{MH=\sqrt{r^2-d^2}}$ $\square$
\end{preuve}

\begin{methode*1}[Applications]
    \exercice
    \begin{enumerate}
        \item \textit{On coupe une sphère de centre $O$ et de rayon $21$ cm par un plan situé à $6$ cm de $O$.\\Soit $I$ le centre de la section.}
        \begin{enumerate}
            \item Calculer la valeur exacte du rayon de la section.
            \item Arrondir ce rayon à l'unité.
        \end{enumerate}
        \item \textit{La section d'une sphère de $8$ cm de rayon et d'un plan est un cercle de $4$ cm de rayon.}\\
        Calculer la distance du centre de la sphère au centre de la section.
    \end{enumerate}
    \correction
    \begin{enumerate}
        \item \textit{On coupe une sphère de centre $O$ et de rayon $21$ cm par un plan situé à $6$ cm de $O$.\\Soit $I$ le centre de la section.}
        \begin{enumerate}
            \item $\text{Rayon}_{\text{section}}=\sqrt{(\Lg[cm]{21})^2-(\Lg[cm]{6})^2}$ soit $\text{Rayon}_{\text{section}}=\sqrt{405}~\Lg[cm]{}$.
            \item $\text{Rayon}_{\text{section}}\approx\Lg[cm]{21}$.
        \end{enumerate}
        \item \textit{La section d'une sphère de $8$ cm de rayon et d'un plan est un cercle de $4$ cm de rayon.}\\
        $(\Lg[cm]{4})^2=\sqrt{(\Lg[cm]{8})^2-r^2}$ soit $16=\sqrt{64-r^2}$ d'où $16=64-r^2$ donc $r^2=48$ et $r=\sqrt{48}\Lg[cm]{}\approx\Lg[cm]{6.9}$.
    \end{enumerate}
\end{methode*1}

\subsection{Le parallélépipède rectangle}
\begin{propriete}[Section par un plan parallèle à une face \admise]
    \begin{minipage}{0.5\linewidth}
        \begin{center}        
            \scalebox{0.7}{
            \begin{Geometrie}[CoinBG={(-2u,-2u)},CoinHD={(7u,6u)}]
                z0=(0,0)*u;
                label.bot(btex $E$ etex,z0);
                z1=(-1,1)*u;label.llft(btex $H$ etex,z1);
                z2=(-1,3)*u;label.ulft(btex $D$ etex,z2);
                z3=(0,2)*u; label.top(btex $A$ etex,z3);
                z4=z0 shifted (2.5u,0);label.lrt(btex $K$ etex,z4);
                z5=z1 shifted (2.5u,0);label.bot(btex $L$ etex,z5);
                z6=z2 shifted (2.5u,0);label.ulft(btex $I$ etex,z6);
                z7=z3 shifted (2.5u,0);label.lrt(btex $J$ etex,z7);
                draw codeperp(z5,z4,z7,5);draw codeperp(z6,z7,z4,5);draw codeperp(z5,z6,z7,5);draw codeperp(z4,z5,z6,5);
                z8 = z0 shifted (6u,0);z9 = z1 shifted (6u,0);z10 = z2 shifted (6u,0);z11 = z3 shifted (6u,0);
                draw z0--z1--z2--z3--cycle; draw z0--z4;draw z1--z5 dashed evenly; draw z2--z6; draw z3--z7;
                draw z4--z7--z6; draw z5--z6 dashed evenly; draw z4--z5 dashed evenly; draw z8--z9--z10 dashed evenly;draw z10--z11--z8;
                z12=(1,3)*u;z13=(1,4.5)*u;z14=(3,2.5)*u;z15=(3,-1)*u;z16=(2,0)*u;
                draw z12--z13--z14--z15--z16;draw z10 --((z13--z14) intersectionpoint(z10--z12));
                draw z9--z5 dashed evenly; draw z8--((z8--z0) intersectionpoint(z14--z15));
                draw z11--((z11--z3) intersectionpoint(z14--z15));labeloffset:=0.6cm;
                label.lrt(btex $(\mathcal P)$ etex,z13 shifted (-0.35u,0));
                labeloffset:=3bp;
            \end{Geometrie}
            }
    
            $(\mathcal{P})$ est parallèle à la face $ADHE$.
        \end{center}        
    \end{minipage}
    \begin{minipage}{0.5\linewidth}
        La section d'un parallélépipède rectangle par un plan parallèle à une face est un rectangle ayant les mêmes dimensions que cette face.    
    \end{minipage}
\end{propriete}
\begin{propriete}[Section par un plan parallèle à une arête \admise]
    \begin{minipage}{0.5\linewidth}
        \begin{center}        
            \scalebox{0.7}{
            \begin{Geometrie}[CoinBG={(-2u,-2u)},CoinHD={(7u,6u)}]
                z0=(0,0)*u;label.bot(btex $E$ etex,z0);
                z1=(-1,1)*u;label.ulft(btex $H$ etex,z1);
                z2=(-1,3)*u;label.ulft(btex $D$ etex,z2);
                z3=(0,2)*u; label.top(btex $A$ etex,z3);
                z4=z0 shifted (3.5u,0);label.lrt(btex $K$ etex,z4);
                z5=z1 shifted (1.5u,0);label.bot(btex $L$ etex,z5);
                z6= z2 shifted(1.5u,0);label.ulft(btex $I$ etex,z6);
                z7=z3 shifted (3.5u,0);label.lrt(btex $J$ etex,z7);
                z8 = z0 shifted (6u,0);z9 = z1 shifted (6u,0);z10 = z2 shifted (6u,0);z11 = z3 shifted (6u,0);
                draw z0--z1--z2--z3--cycle;draw z0--z4;draw z1--z5 dashed evenly; draw z2--z6; draw z3--z7;
                draw z4--z7--z6; draw z5--z6 dashed evenly; draw z4--z5 dashed evenly;draw z10--z11--z8;
                draw codeperp(z5,z4,z7,5);draw codeperp(z6,z7,z4,5);draw codeperp(z5,z6,z7,5);draw codeperp(z4,z5,z6,5);
                z12=(-0.75,0.75)*u;z13=(-1.5,1)*u;z14=(-1.5,4.5)*u;z15=(5.25,2.25)*u;z16=(5.25,-1.25)*u;z17=(1.5,0)*u;
                z20=z15 - 0.1(z15-z14);z21=z16 - 0.1(z15-z14);
                draw z12--z13--z14--z20--z21--z17;draw z10--((z14--z15) intersectionpoint(z10--z2));
                 draw z10 --((z14--z15) intersectionpoint(z10--z9)) dashed evenly;
                draw z11--((z20--z21) intersectionpoint(z11--z3));
                draw z8 --((z20--z21) intersectionpoint(z0--z8));
                draw z8--z9--z10 dashed evenly; draw z9--((z20--z21) intersectionpoint(z9--z1)) dashed evenly;
                labeloffset:=0.3cm;
                label.lrt(btex $(\mathcal P)$ etex,z14);
            \end{Geometrie}
            }
    
            $(\mathcal{P})$ est parallèle à l'arête $[AE]$.
        \end{center}        
    \end{minipage}
    \begin{minipage}{0.5\linewidth}
        La section d'un parallélépipède rectangle par un plan parallèle à une arête est un rectangle.
    \end{minipage}
\end{propriete}

\begin{remarque}
    Lorsque le parallélépipède est un \textbf{cube} les sections parallèles aux faces sont des carrés.
\end{remarque}

\subsection{Le cylindre de révolution}
\begin{propriete}[Section par un plan perpendiculaire à l'axe \admise]
    \begin{minipage}{0.5\linewidth}
        \begin{center}        
            \scalebox{0.5}{
            \begin{Geometrie}[CoinBG={(-4u,-3u)},CoinHD={(6u,6u)}]
                z0=(0,0)*u;
                path c;
                c=(fullcircle scaled 4cm)yscaled 0.375;
                z1=(-4,-1)*u;z2=(-2.75,1)*u; z3=(3.75,1)*u;z4=(2.5,-1)*u;
                draw z2--z1--z4--z3; draw  (subpath (0,0.5*(length c)) of c)dashed evenly;
                draw (subpath(0.5*(length c),length c) of c)  ;z6=(0,4u);
                draw point 0 of c -- point 0 of (c shifted z6);label.urt(btex $(\mathcal P)$etex,z1 shifted (0.2u,0));
                draw point 0.5*(length c) of c -- point 0.5*(length c) of (c shifted z6);
                path d,e; d=point 0.5*(length c) of c -- point 0.5*(length c) of (c shifted z6);
                e=point 0 of c -- point 0 of (c shifted z6);
                z5=(0,-2u);draw z2-- ((z2--z3)intersectionpoint d);
                draw z3--((z3--z2) intersectionpoint e);z20=(-2.5,5.4)*u;z21=(-0.5,3.5)*u;z12=(0,-7.5u);z24=z21 shifted z12;
                draw z20; draw z24;
                draw z6-- ((((subpath(0.5*(length c),length c) of (c shifted z5)))intersectionpoint(z5+0.2(z5-z6)--z6))--z6) dashed evenly; draw z6--z6+0.2(z6-z5);
                draw(((subpath(0.5*(length c),length c) of (c shifted z5)))intersectionpoint(z5+0.2(z5-z6)--z6)) --z5+0.2(z5-z6);
                draw (c shifted z6);draw (subpath(0.5*(length c),length c) of c)  shifted z5;
                draw (subpath(0,0.5*(length c)) of c) shifted z5 dashed evenly;
                draw ((z1--z4) intersectionpoint (point 0.5*(length c) of c -- point 0.5*(length c) of (c shifted z5)))-- point 0.5*(length c) of (c shifted z5);
                draw ((z1--z4) intersectionpoint (point 0 of c -- point 0 of (c shifted z5)))-- point 0 of (c shifted z5);
                draw z0-- point 0.9*(length c)of c dashed evenly; label.top(btex$R$etex,0.5[z0,point 0.9*(length c)of c]);
                draw z6--point 0.9*(length c)of (c shifted z6) dashed evenly; label.top(btex$R$etex,0.5[z6,point 0.9*(length c)of (c shifted z6)]);
            \end{Geometrie}
            }
    
            $(\mathcal{P})$ est perpendiculaire à l'axe de révolution.
        \end{center}        
    \end{minipage}
    \begin{minipage}{0.5\linewidth}
        La section d'un cylindre par un plan perpendiculaire à son axe de révolution est un disque de même rayon que sa base.
    \end{minipage}
\end{propriete}
\begin{propriete}[Section par un plan parallèle à l'axe \admise]
    \begin{minipage}{0.5\linewidth}
        \begin{center}        
            \scalebox{0.5}{
            \begin{Geometrie}[CoinBG={(-3u,-4u)},CoinHD={(6u,6u)}]
                z0=(0,0)*u;
                path c,ch,cb;
                c=(fullcircle scaled 4cm)yscaled 0.375;
                ch=subpath (0,0.5*(length c)) of c;cb=subpath(0.5*(length c),length c) of c;
                z1=(-2.5,5.4)*u;z2=(-0.5,3.5)*u;z12=(0,-7.5u);z5=(0,-2u);z6=(0,4u);
                z3=z1 shifted z12; z4=z2 shifted z12;label.ulft(btex $(\mathcal P)$ etex,z4 shifted (0,0.5u));
                draw z1--z3--z4--z2--cycle;draw (ch shifted z6)  cutafter (z1--z2);
                draw (cb shifted z6) cutbefore (z2--z4);
                draw subpath (0,0.35*(length cb)) of (cb shifted z6);draw subpath (0.9*(length ch),length ch) of (ch shifted z6);
                draw (cb shifted z5) cutbefore (z2--z4);
                draw subpath (0,0.35*(length cb)) of (cb shifted z5);draw subpath (0.9*(length ch),length ch) of (ch shifted z5);
                draw subpath (0,0.9*(length ch)) of (ch shifted z5)dashed evenly;
                draw point 0 of c shifted z5--point 0 of ch shifted z6;
                draw point 0.5*(length c) of (c shifted z5)--point 0.5*(length c) of (c shifted z6);
                draw point 0.9*(length ch) of (ch shifted z6) -- point 0.35*(length cb)  of cb shifted z6;
                draw point 0.35*(length cb)  of cb shifted z6--point 0.35*(length cb)  of cb shifted z5;
                draw point 0.9*(length ch) of (ch shifted z5) -- point 0.35*(length cb)  of (cb shifted z5) dashed evenly;
                draw point 0.9*(length ch) of (ch shifted z5) -- point 0.9*(length ch) of (ch shifted z6) dashed evenly;
                z13=point  0.9*(length ch) of (ch shifted z6);z14= point 0.35*(length cb)  of cb shifted z6;
                z15=point 0.9*(length ch) of (ch shifted z5);z16 = point 0.35*(length cb)  of (cb shifted z5);
                draw codeperp(z15,z13,z14,5); draw codeperp(z13,z14,z16,5); draw codeperp(z14,z16,z15,5); draw codeperp(z16,z15,z13,5);
                draw z6-- ((((subpath(0.5*(length c),length c) of (c shifted z5)))intersectionpoint(z5+0.2(z5-z6)--z6))--z6) dashed evenly; draw z6--z6+0.2(z6-z5);
                draw(((subpath(0.5*(length c),length c) of (c shifted z5)))intersectionpoint(z5+0.2(z5-z6)--z6)) --z5+0.2(z5-z6);
                draw z5-- point 0.9*(length c)of (c shifted z5) dashed evenly; label.top(btex$R$etex,0.5[z5,point 0.9*(length c)of (c shifted z5)]);
                draw z6--point 0.9*(length c)of (c shifted z6) dashed evenly; label.top(btex$R$etex,0.5[z6,point 0.9*(length c)of (c shifted z6)]);
            \end{Geometrie}
            }
    
            $(\mathcal{P})$ est perpendiculaire à l'axe de révolution.
        \end{center}        
    \end{minipage}
    \begin{minipage}{0.5\linewidth}
        La section d'un cylindre par un plan parallèle à son axe de révolution est un rectangle.
    \end{minipage}
\end{propriete}

\subsection*{La pyramide et le cône de révolution}
\begin{propriete}
    \begin{center}
        \hfill
        \scalebox{0.4}{
            \begin{Geometrie}[CoinBG={(-5u,-3u)},CoinHD={(6u,8u)}]
                labeloffset:=1.5bp;
                z5=(0,7)*u;
                z0=(0,0)*u;z1=(4.5,2)*u;z2=(1.5,-2)*u;z3=(-4.5,-2)*u;z4=(-1.5,2)*u;
                draw z1--z2--z3 ;draw z3--z4--z1 dashed evenly; z6=0.6[z3,z5];
                z7=whatever [z6,z6+2(z2-z3)]=whatever[z2,z5];z8=whatever[z7,z7+2(z1-z2)]=whatever[z1,z5];
                z9=whatever[z8,z8+2(z3-z2)]=whatever[z4,z5];
                draw z6-- z7--z8; draw z8--z9--z6 dashed evenly;
                draw z6--z5--z7;draw z5--z9 dashed evenly;draw z5--z8 ;
                z10=z6-0.4(z8-z6); z11=z7+0.4(z7-z9);z12=z8+0.4(z8-z6);draw z10--z11--z12;z13=z9+0.4(z9-z7);
                draw z12--(z12--z12+0.5(z13-z12)) intersectionpoint (z1--z5);
                draw z13--(z13--z13+0.5(z12-z13)) intersectionpoint (z3--z5);
                draw z13--z10;draw z3--(z3--z6)intersectionpoint (z10--z11);
                draw z2--(z2--z7)intersectionpoint (z10--z11);
                draw z1--(z1--z8)intersectionpoint (z11--z12);
                draw z4--z9 dashed evenly;draw z6--z8 dashed evenly; draw z7--z9 dashed evenly;
                draw z2--z4 dashed evenly; draw z1--z3 dashed evenly;
                z15=whatever[z7,z9]=whatever[z6,z8];
                draw codeperp(z3,z0,z15,7); draw codeperp(z6,z15,z5,7);
                label.urt(btex \large $(\mathcal P)$ etex,z10 shifted(0.2u,0.1u));
                draw z0--z5 dashed evenly;
            \end{Geometrie}
        }
        \hfill
        \scalebox{0.6}{
            \begin{Geometrie}[CoinBG={(-6u,-2u)},CoinHD={(6u,6u)}]
                z0=(0,0)*u;
                path c,cb,ch; c=(fullcircle scaled 4cm)yscaled 0.3;
                ch=subpath (0,0.5*(length c)) of c;cb=subpath(0.5*(length c),length c) of c;
                z1=(0,5u);
                draw ch dashed evenly; draw cb;draw z0--z1 dashed evenly;
                draw (ch scaled 0.6) shifted (0,2u) dashed evenly;draw (cb scaled 0.6) shifted (0,2u);
                draw z1--point 0 of ((c scaled 0.6)shifted (0,2u));draw z1--point 0.5*(length c) of ((c scaled 0.6)shifted (0,2u));
                z2=point 0.625*(length c) of ((c scaled 0.6)shifted (0,2u));
                z3=point 0.125*(length c) of ((c scaled 0.6)shifted (0,2u));
                z4=point 0.875*(length c) of ((c scaled 0.6)shifted (0,2u));
                z5=point 0.375*(length c) of ((c scaled 0.6)shifted (0,2u));
                draw z5+0.5(z5-z4)--z2+0.7(z2-z3)--z4+0.5(z4-z5)--z3+0.7(z3-z2);
                draw (z3+0.7(z3-z2)) -- ((z3+0.7(z3-z2)--z5+0.5(z5-z4))intersectionpoint (z1--point 0 of ((c scaled 0.6)shifted (0,2u))));
                draw (z5+0.5(z5-z4))-- ((z3+0.7(z3-z2)--z5+0.5(z5-z4))intersectionpoint (z1--point 0.5*(length c) of ((c scaled 0.6)shifted (0,2u))));
                z6=whatever[z2,z3]=whatever[z4,z5];
                draw z6--point 0.6*(length c) of (c scaled 0.6)shifted (0,2u);
                draw z6--point 0 of (c scaled 0.6) shifted (0,2u) dashed evenly;
                draw point 0 of c--z0 dashed evenly;
                draw z0--z1 dashed evenly;
                draw z0--point 0.6*(length c) of c;
                label.urt(btex $(\mathcal P)$ etex,z2+0.6(z2-z3));
                draw point 0 of c--(point 0 of (c scaled 0.6)shifted (0,2u)--point 0 of c)intersectionpoint (z4+0.5(z4-z5)--z2+0.7(z2-z3));
                draw point 0.5*(length c) of c--(point 0.5*(length c) of (c scaled 0.6)shifted (0,2u)--point 0.5*(length c) of c) intersectionpoint (z4+0.5(z4-z5)--z2+0.7(z2-z3));
                draw codeperp(z1,z6,point 0.6*(length c) of (c scaled 0.6)shifted (0,2u),5);
                draw codeperp(z1,z0,point 0.6*(length c) of c,5);
                draw codeperp(z1,z6,point 0 of (c scaled 0.6)shifted (0,2u),7) dashed evenly;
                draw codeperp(z1,z0,point 0 of c,7) dashed evenly;
            \end{Geometrie}
        }
        \hfill\phantom{rrr}

        $(\mathcal{P})$ est parallèle au plan de base.
    \end{center}
    \smallskip
    La section d'une pyramide ou d'un cône est de même nature que la base, c'est une réduction de celle-ci
\end{propriete}

% \section{Agrandissement - réduction}
% \definNum{
% \begin{itemize}
% \item L'{\bf agrandissement de rapport $k$} d'un objet est la transformation qui consiste à \textbf{multiplier} toutes ses longueurs \textbf{par un nombre $k$ supérieur à 1}.
% \item La {\bf réduction de rapport $k$} d'un objet est la transformation qui consiste à \textbf{multiplier} toutes ses longueurs \textbf{par un nombre $k$ inférieur à 1}.
% \end{itemize}
% }
% \Exemples{}{
% \begin{itemize}
% \item une maquette réalisée à l'échelle $1/100$ est la réduction de rapport $\dfrac1{100}$ de l'objet réel.
% \item une feuille de format A3 (rectangle de dimensions $29,7\,cm$ et $42\,cm$) est un agrandissement de rapport $\sqrt2$ d'une feuille de format A4 (rectangle de dimensions $21\,cm$ et $29,7\,cm$).
% \end{itemize}
% }

% \proprNumBis{(Admise)}{Dans un agrandissement ou une réduction de rapport $k$ :
% \begin{itemize}
% \item les aires sont multipliées par $k^2$.
% \item les volumes sont multipliés par $k^3$.
% \end{itemize}
% }

% \Exemples{}{
% \begin{itemize}
% \item Deux feuilles de format A4 sont nécessaires pour recouvrir exactement une feuille de format A3 ($k=\sqrt2$ et $k^2=2$ soit le double de la surface).
% \item huit petits cubes d'arête $a$ sont nécessaires pour remplir un cube d'arête $2a$ ($k=2$ et $k^3=8$).
% \end{itemize}
% }

% \proprNumBis{(Admise)}{
% Dans un agrandissement ou une réduction, il y a conservation des \textbf{mesures d'angles}, de la \textbf{perpendicularité} et du \textbf{parallélisme}.
% }

% \Exemples[Exemple dans l'espace : section de la pyramide ou du c\^{o}ne]{}{
% C'est l'application du \textbf{théor\`{e}me de Thal\`{e}s} qui permet le calcul du coefficient d'agrandissement ou de réduction.\\
% Les plans de sections étant parall\`{e}les aux bases, son application est licite! 
% \begin{center}
% \begin{tabular}{c|c|c}
% \includegraphics[scale=0.85]{../figures/coursespace.9}&&\includegraphics[scale=0.85]{../figures/coursespace.10}\\
% &&\\
% $k=\dfrac{SO'}{SO}=\dfrac{SA'}{SA}=\dfrac{O'A'}{OA}=\ldots$&&$k=\dfrac{SO'}{SO}=\dfrac{SA'}{SA}=\dfrac{A'B'}{AB}=\dfrac{A'O'}{AO}=\ldots$\\
% &&\\
% \textbf{c\^{o}ne de révolution}&&\textbf{pyramide}\\
% \end{tabular}
% \end{center}
% }

% \section{Le globe terrestre}
% \definNumTitre{Vocabulaire}{
% \begin{minipage}{5cm}
% \begin{center}
% \textbf{Figure 1:}
% \end{center}
% \scalebox{0.7}{
% \psset{xunit=0.6cm,yunit=0.6cm,algebraic=true,dotstyle=*,dotsize=3pt 0,linewidth=0.8pt,arrowsize=3pt 2,arrowinset=0.25}
% \begin{pspicture*}(-6,-5.5)(6,5.8)
% \pscircle(0,0){3}
% \rput{0}(0,2.77){\psellipse[linestyle=dotted](0,0)(4.16,0.83)}
% \rput{90}(0,0){\psellipse[linestyle=dotted](0,0)(5,3.6)}
% \rput{0}(0,0){\psellipse[linestyle=dashed,dash=8pt 8pt](0,0)(5,1)}
% %\psarc(0,0){5}{180}{0}
% \rput{90}(0,0){\psellipse[linestyle=dashed,dash=8pt 8pt](0,0)(5,1)}
% %\psarc(0,0){5}{180}{0}
% \psline[linestyle=dashed,dash=4pt 4pt](0,0)(0.98,-0.98)
% \psline[linestyle=dashed,dash=4pt 4pt](0,0)(-3.55,-0.7)
% %\psarc[linewidth=2pt](0,0){5}{224.75}{281.31}
% \psarcellipse[linewidth=2pt](0,0)(5,1){180}{0}
% \rput{90}(0,0){\psarcellipse[linewidth=2pt](0,0)(5,1){180}{0}}
% \psline[linestyle=dashed,dash=4pt 4pt](0,0)(-3.22,2.24)
% %\psarc[linewidth=2pt](0,0){5}{63.36}{98.13}
% \psdots(0,0)
% \rput[bl](0.13,0.2){$O$}
% \psdots(0,-5)
% \rput[bl](0.13,-5.5){$S$}
% \psdots(0,5)
% \rput[bl](0.13,5.2){$N$}
% \psdots(-3.22,2.24)
% \rput[bl](-3.1,2.43){$C$}
% \psdots(0.98,-0.98)
% \rput[bl](1.1,-0.77){$A$}
% \psdots(-3.55,-0.7)
% \rput[bl](-3.43,-0.5){$B$}
% \psdots(-5,0)
% \rput[bl](-5.8,0){$W$}
% \psdots(5,0)
% \rput[bl](5.2,0){$E$}
% \end{pspicture*}
% }
% \end{minipage}
% \begin{minipage}{12cm}
% La Terre est assimilable à une boule d'environ 6400 km de rayon.\\
% Appelons $O$ le centre de la Terre. Le point $N$ représente le pôle Nord, le point $S$ le pôle Sud.\\
% Sur la sphère représentant la surface terrestre, un grand cercle de centre $O$ passant par $N$ et $S$ est appelé \textbf{méridien}.\\
% Le grand cercle de centre $O$ et tracé dans un plan perpendiculaire au diamètre $[NS]$ est, lui, appelé \textbf{l'équateur}.\\
% Ici est tracé le méridien qui sert de référence, appelé \textbf{méridien de Greenwich} (\textit{car il passe par Greenwich, petite ville située non loin de Londres})\\
% Chaque point à la surface de la Terre peut être repéré grâce à deux nombres: la \textbf{longitude} et la \textbf{latitude}.\\
% La longitude est calculée par rapport au méridien de Greenwich, la latitude par rapport à l'équateur; par exemple, le point $C$ sur cette figure, qui représente la position de la ville de Chicago, a pour longitude $\widehat{AOB}=87^{\circ}$, et pour latitude $\widehat{BOC}=41^{\circ}$
% \end{minipage}
% }

% \Methode[Application]{}{
% \begin{minipage}{5cm}
% \begin{center}
% \textbf{Figure 2:}
% \end{center}
% \scalebox{0.7}{
% \psset{xunit=0.6cm,yunit=0.6cm,algebraic=true,dotstyle=*,dotsize=3pt 0,linewidth=0.8pt,arrowsize=3pt 2,arrowinset=0.25}
% \begin{pspicture*}(-6,-5.5)(6,5.8)
% \pscircle(0,0){3}
% \rput{0}(0,2.77){\psellipse[linestyle=dotted](0,0)(4.16,0.83)}
% \rput{0}(0,0){\psellipse[linestyle=dashed,dash=8pt 8pt](0,0)(5,1)}
% %\platitude longitude istansarc(0,0){5}{180}{0}
% \rput{90}(0,0){\psellipse[linestyle=dashed,dash=8pt 8pt](0,0)(5,1)}
% \rput{0}(0,2.77){\psarcellipse[linewidth=2pt](0,0)(4.16,0.83){220}{300}}
% %\psarc(0,0){5}{180}{0}
% \psline[linestyle=dashed,dash=8pt 8pt](0,0)(0,2.77)
% %\psarc[linewidth=2pt](0,2.77){4.16}{219.38}{301.24}
% \psdots(0,2.77)
% \rput[bl](-0.13,2.97){$O'$}
% \psdots(0,0)
% \rput[bl](0.13,0.2){$O$}
% \psdots(0,-5)
% \rput[bl](0.13,-5.5){$S$}
% \psdots(0,5)
% \rput[bl](0.13,5.2){$N$}
% \psdots(-3.22,2.24)
% \rput[bl](-3.1,2.43){$C$}
% \psdots(0.98,-0.98)
% \rput[bl](1.1,-0.77){$A$}
% \psdots(2.16,2.06)
% \rput[bl](2.3,2.27){$I$}
% \psdots(-5,0)
% \rput[bl](-5.8,0){$W$}
% \psdots(5,0)
% \rput[bl](5.2,0){$E$}
% \psarcellipse[linewidth=2pt](0,0)(5,1){180}{0}
% \rput{90}(0,0){\psarcellipse[linewidth=2pt](0,0)(5,1){180}{0}}
% \end{pspicture*}
% }
% \end{minipage}
% \begin{minipage}{12cm}
% Le cercle de centre $O'$ et passant par $C$, parallèle au plan de l'équateur, est appelé  \textbf{parallèle}, justement.\\
% Ce n'est pas ce que l'on appelle un grand cercle (car il n'a pas $O$ pour centre).\\
% La situation d'Istanbul, ville située sur le même parallèle que Chicago (et qui a donc la même latitude, mais pas la même longitude), est représentée par le point $I$.\\ Les question à traiter sont les suivantes:\\
% \textbf{1.} Connaissant les coordonnées (longitude et latitude) des deux villes, \textbf{quel est le chemin le plus court pour les joindre en avion}? en suivant le parallèle passant par $I$ et $C$? (\textit{voir figure 2}), ou en suivant le grand cercle passant par $I$ et $C$? (\textit{voir figure 3})\\
% \textbf{2.} Quelle est l'aire totale, en km$^2$, de la surface terrestre ? Quel est le volume total, en km$^3$, de la Terre ? (\textit{donner les réponses sous forme scientifique})
% \end{minipage}
% }


% \Methode[Application]{}{
% \begin{minipage}{5cm}
% \begin{center}
% \textbf{Figure 3:}
% \end{center}
% \scalebox{0.7}{
% \psset{xunit=0.6cm,yunit=0.6cm,algebraic=true,dotstyle=*,dotsize=3pt 0,linewidth=0.8pt,arrowsize=3pt 2,arrowinset=0.25}
% \begin{pspicture*}(-6,-5.5)(6,5.8)
% \pscircle(0,0){3}
% \rput{0}(0,0){\psellipse[linestyle=dashed,dash=8pt 8pt](0,0)(5,1)}
% %\psarc(0,0){5}{180}{0}
% \rput{90}(0,0){\psellipse[linestyle=dashed,dash=8pt 8pt](0,0)(5,1)}
% %\psarc(0,0){5}{180}{0}
% \rput{-7.59}(0,0){\psellipse[linestyle=dotted](0,0)(5,2.51)}
% \rput{-7.59}(0,0){\psarcellipse[linewidth=2pt](0,0)(5,2.51){68}{135}}
% %\psarc[linewidth=2pt](0,0){5}{68.05}{134.21}
% \psdots(0,0)
% \rput[bl](0.13,0.2){$O$}
% \psdots(0,-5)
% \rput[bl](0.13,-5.5){$S$}
% \psdots(0,5)
% \rput[bl](0.13,5.2){$N$}
% \psdots(-3.22,2.24)
% \rput[bl](-3.1,2.43){$C$}
% \psdots(0.98,-0.98)
% \rput[bl](1.1,-0.77){$A$}
% \psdots(2.16,2.06)
% \rput[bl](2.3,2.27){$I$}
% \psdots(-5,0)
% \rput[bl](-5.8,0){$W$}
% \psdots(5,0)
% \rput[bl](5.2,0){$E$}
% \psarcellipse[linewidth=2pt](0,0)(5,1){180}{0}
% \rput{90}(0,0){\psarcellipse[linewidth=2pt](0,0)(5,1){180}{0}}
% \end{pspicture*}
% }
% \end{minipage}
% \begin{minipage}{12cm}
% Voici ce dont vous avez besoin pour répondre à ces questions:
% \begin{itemize}
% \item  Coordonnées géographiques de Chicago:\\
% Latitude $41^{\circ}$ Nord, longitude $87^{\circ}$ Ouest.
% \item  Coordonnées géographiques d'Istanbul:\\
% Latitude $41^{\circ}$ Nord, longitude $28^{\circ}$ Est.
% \item  $OO'=4200$ km, $\widehat{COI}=79^{\circ}$
% \item  Formule pour calculer la longueur d'un arc de cercle défini par un angle de mesure $\alpha$:\\ $L=2\times\pi\times R\times\frac{\alpha}{360}$
% \item  Aire d'une sphère de rayon $R$:\\
% $\mathcal{A}=4\times\pi\times R^2$.
% \item  Volume d'une boule de rayon $R$:\\
% $\mathcal{V}=\frac{4}{3}\times\pi\times R^3$
% \end{itemize}
% \end{minipage}
% }

% \subsection{Compléments numériques - Constructions}

% \Animations{
% \lienCadre{http://lozano.maths.free.fr/iep_local/figures_html/scr_iep_078.html}{Section d'une pyramide par un plan}
% \lienCadre{http://lozano.maths.free.fr/iep_local/figures_html/scr_iep_077.html}{Section d'un pavé par un plan}
% \creditInstrumentPoche
% }