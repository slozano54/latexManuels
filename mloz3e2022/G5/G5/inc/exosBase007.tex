\begin{exercice*}
    Cette figure représente une boule de centre O et de diamètre \Lg[cm]{5}.

    \scalebox{1.1}{
        \begin{Geometrie}
            u:=30;
            z0=(3,3)*u;%centre
            numeric angleInclinaison;
            angleInclinaison:=-14;
            dotlabel.urt(btex $O$ etex,z0);
            z1=(2.5,1)*u;%pole sud
            dotlabel.lrt(btex $F$ etex,z1);
            z2=(3.5,5)*u;%pole nord
            dotlabel.urt(btex $E$ etex,z2);
            % remplissages ici avant les path
            % fill fullcircle scaled 4.1231u scaled 0.88 yscaled 0.15 rotated 14 shifted (z0+ unitvector(z2-z1)*u) withcolor Cornsilk;
            % fin remplissages    
            path contour, ellipse,greenwich,meridien;
            path r;% contour translaté
            contour=(fullcircle scaled 4.1231u);
            ellipse=(contour yscaled 0.25) rotated angleInclinaison;
            greenwich=(contour scaled 1 xscaled 0.32) rotated angleInclinaison;
            r=contour shifted z0;
            z3=point 0.6*(length ellipse) of ellipse shifted z0;
            dotlabel.bot(btex $J$ etex,z3);
            z4=point 0.8*(length ellipse) of ellipse shifted z0;
            dotlabel.llft(btex $G$ etex,z4);            
            z5=point 0.5*(length ellipse) of ellipse shifted z0;
            dotlabel.lft(btex $B$ etex,z5);
            z6=point 0 of ellipse shifted z0;
            dotlabel.urt(btex $A$ etex,z6);
            z7=z4+ 0.5(z4-z0);
            dotlabel.llft(btex $K$ etex,z7);
            z8=z0 + 0.5(z0-z3);
            dotlabel.urt(btex $I$ etex,z8);
            path diam;            
            diam=z3--z0 + 2(z0-z3);
            z9=diam intersectionpoint r;
            draw z3--z9 dashed withdots withpen pencircle scaled 1.5bp;
            draw z9--z9 + 0.7(z0-z3);
            draw z3--z3 + 0.7(z3-z0);
            draw z0--z4 dashed withdots withpen pencircle scaled 1.5bp;
            draw z4--z7 + 1.5(z4-z0);
            draw z5--z6 dashed withdots withpen pencircle scaled 1.5bp;            
            %
            draw contour shifted z0;
            %equateur
            draw (subpath (0.5*(length ellipse),length ellipse) of ellipse) shifted z0 withpen pencircle scaled 1.3bp;;
            draw (subpath (0,0.5*(length ellipse)) of ellipse) shifted z0 dashed evenly;
            %greenwich
            draw (subpath (-0.25*(length greenwich),0.25*(length greenwich)) of greenwich) shifted z0 withpen pencircle scaled 1.3bp;;
            draw (subpath (0.25*(length greenwich),0.75*(length greenwich)) of greenwich) shifted z0 dashed evenly;                    
            % Axe de rotation
            draw z1--z2 dashed evenly;
            draw z1--z1 + 0.08(z1-z2);
            draw z2--z2+0.08(z2-z1);
        \end{Geometrie}
    }

        \begin{enumerate}
            \item Compléter le tableau ci-dessous.
             
            Points appartenant à \dots

            \begin{tabular}{|*{2}{>{\centering\arraybackslash}m{0.4\linewidth}|}}
                \hline
                \cellcolor{LightGray}La sphère de centre $O$ et de rayon $OA$&\\\hline
                \cellcolor{LightGray}La boule de centre $O$ et de rayon $OA$&\\\hline
                \cellcolor{LightGray}Aucun des deux&\\\hline
            \end{tabular}
            \item Placer, sur la figure, le point $H$, diamétralement opposé à $G$. Puis placer, sur la demi-droite $[OG)$, un point $L$ qui appartient à la boule de rayon $OA$.  
            \item Compléter.
            \begin{itemize}
                \item $[AB]$ est un \pointilles[3cm] de la sphère.
                \item $[OG]$ est un \pointilles[3cm] de la sphère.
                \item $[OJ]$ est un \pointilles[3cm] de la sphère.
                \item $[GH]$ est un \pointilles[3cm] de la sphère.
                \item Le cercle de centre $O$ et de diamètre $[EF]$ est                
                appelé \pointilles[3cm] de la sphère.
            \end{itemize}
            \item Calculer le périmètre du cercle de centre $O$ et de diamètre $[EF]$.
        \end{enumerate}
\end{exercice*}
\begin{corrige}
    Cette figure représente une boule de centre O et de diamètre \Lg[cm]{5}.

    \hspace*{-10mm}\scalebox{1.3}{
        \begin{Geometrie}
            u:=30;
            z0=(3,3)*u;%centre
            numeric angleInclinaison;
            angleInclinaison:=-14;
            dotlabel.urt(btex $O$ etex,z0);
            z1=(2.5,1)*u;%pole sud
            dotlabel.lrt(btex $F$ etex,z1);
            z2=(3.5,5)*u;%pole nord
            dotlabel.urt(btex $E$ etex,z2);
            % remplissages ici avant les path
            % fill fullcircle scaled 4.1231u scaled 0.88 yscaled 0.15 rotated 14 shifted (z0+ unitvector(z2-z1)*u) withcolor Cornsilk;
            % fin remplissages    
            path contour, ellipse,greenwich,meridien;
            path r;% contour translaté
            contour=(fullcircle scaled 4.1231u);
            ellipse=(contour yscaled 0.25) rotated angleInclinaison;
            greenwich=(contour scaled 1 xscaled 0.32) rotated angleInclinaison;
            r=contour shifted z0;
            z3=point 0.6*(length ellipse) of ellipse shifted z0;
            dotlabel.bot(btex $J$ etex,z3);
            z4=point 0.8*(length ellipse) of ellipse shifted z0;
            dotlabel.llft(btex $G$ etex,z4);            
            z5=point 0.5*(length ellipse) of ellipse shifted z0;
            dotlabel.lft(btex $B$ etex,z5);
            z6=point 0 of ellipse shifted z0;
            dotlabel.urt(btex $A$ etex,z6);
            z7=z4+ 0.5(z4-z0);
            dotlabel.llft(btex $K$ etex,z7);
            z8=z0 + 0.5(z0-z3);
            dotlabel.urt(btex $I$ etex,z8);
            path diam;            
            diam=z3--z0 + 2(z0-z3);
            z9=diam intersectionpoint r;
            draw z3--z9 dashed withdots withpen pencircle scaled 1.5bp;
            draw z9--z9 + 0.7(z0-z3);
            draw z3--z3 + 0.7(z3-z0);
            draw z0--z4 dashed withdots withpen pencircle scaled 1.5bp;
            draw z4--z7 + 1.5(z4-z0);
            draw z5--z6 dashed withdots withpen pencircle scaled 1.5bp;
            %
            draw contour shifted z0;
            %equateur
            draw (subpath (0.5*(length ellipse),length ellipse) of ellipse) shifted z0 withpen pencircle scaled 1.3bp;;
            draw (subpath (0,0.5*(length ellipse)) of ellipse) shifted z0 dashed evenly;
            %greenwich
            draw (subpath (-0.25*(length greenwich),0.25*(length greenwich)) of greenwich) shifted z0 withpen pencircle scaled 1.3bp;;
            draw (subpath (0.25*(length greenwich),0.75*(length greenwich)) of greenwich) shifted z0 dashed evenly;                    
            % Axe de rotation
            draw z1--z2 dashed evenly;
            draw z1--z1 + 0.08(z1-z2);
            draw z2--z2+0.08(z2-z1);
            %correction
            z10=point 0.3*(length ellipse) of ellipse shifted z0;
            dotlabel.llft(btex \textcolor{red}{$H$} etex,z10) withcolor red;
            draw z10--z0 dashed withdots withpen pencircle scaled 1.5bp withcolor red;
            z11=z0 + 0.5(z4-z0);
            dotlabel.lft(btex \textcolor{red}{$L$} etex,z11) withcolor red;
        \end{Geometrie}
    }
    \Coupe
    \begin{enumerate}
        \item Compléter le tableau ci-dessous.
            
        Points appartenant à \dots

        \begin{tabular}{|*{2}{>{\centering\arraybackslash}m{0.4\linewidth}|}}
            \hline
            \cellcolor{LightGray}La sphère de centre $O$ et de rayon $OA$&\textcolor{red}{$A$;$E$;$B$;$F$;$J$;$G$}\\\hline
            \cellcolor{LightGray}La boule de centre $O$ et de rayon $OA$ &\textcolor{red}{$A$;$E$;$B$;$F$;$J$;$G$;$O$;$I$}\\\hline
            \cellcolor{LightGray}Aucun des deux&\textcolor{red}{$K$}\\\hline
        \end{tabular}
        \item Placer, sur la figure, le point $H$, diamétralement opposé à $G$. Puis placer, sur la demi-droite $[OG)$, un point $L$ qui appartient à la boule de rayon $OA$.  
        \item Compléter.
        \begin{itemize}
            \item $[AB]$ est un \textcolor{red}{diamètre} de la sphère.
            \item $[OG]$ est un \textcolor{red}{rayon} de la sphère.
            \item $[OJ]$ est un \textcolor{red}{rayon} de la sphère.
            \item $[GH]$ est un \textcolor{red}{diamètre} de la sphère.
            \item Le cercle de centre $O$ et de diamètre $[EF]$ est                
            appelé \textcolor{red}{grand cercle} de la sphère.
        \end{itemize}
        \item Calculer le périmètre du cercle de centre $O$ et de diamètre $[EF]$.
        
        \textcolor{red}{$2\times\pi\times R = 5\pi \approx \Lg[cm]{15.7}$}
    \end{enumerate}
\end{corrige}
