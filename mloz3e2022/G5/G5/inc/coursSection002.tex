\section{Sphères et boules}
\subsection{Sphères}
\begin{remarque}
    Le solide obtenu en faisant tourner un cercle autour d'un de ses diamètres est une sphère.
\end{remarque}
\begin{definition}
    Une \textbf{sphère} de centre $O$ et de rayon $r$ est l'ensemble des points de l'espace situés à une distance $r$ du point $O$.
\end{definition}
\begin{exemples*1}
    Balles, ballons,etc.. ce sont des objets creux.
\end{exemples*1}
\begin{propriete}[\admise]
    Si $(\mathscr S)$ est une sphère de rayon $r$ alors son aire vaut $4\pi r^2$.
\end{propriete}
\begin{exemple*1}
    \titreExemple{Utilisation du vocabulaire}

    \begin{minipage}{0.5\linewidth}
        Sur la figure ci-contre :
        \begin{itemize}
            \item $[NS]$ est un diamètre de la sphère.
            \item $[OM]$ est un rayon de la sphère.
            \item $OM=r$.
        \end{itemize}
    \end{minipage}
    \begin{minipage}{0.5\linewidth}
        \begin{center}
            \scalebox{0.7}{
            \begin{Geometrie}
                u:=1cm;
                z0=(3,3)*u;
                dotlabel.urt(btex $O$ etex,z0);
                z1=(3.5,1)*u;
                label.lrt(btex $S$ etex,z1);
                z2=(2.5,5)*u;
                label.urt(btex $N$ etex,z2);
                path contour, ellipse,p,q,r,s,coupe;
                contour=(fullcircle scaled 4.1231cm);
                ellipse=(contour yscaled 0.25) rotated 14;
                coupe=(contour scaled 0.88 yscaled 0.15) rotated 14;
                draw z0--point 0 of ellipse shifted z0 dashed evenly;
                label.bot(btex $r$ etex,0.5[z0,point 0 of ellipse shifted z0]);
                draw contour shifted z0;
                draw (subpath (0.5*(length ellipse),length ellipse) of ellipse) shifted z0;
                draw (subpath (0,0.5*(length ellipse)) of ellipse) shifted z0 dashed evenly;
                draw z1--z2 dashed evenly;
                draw z1--z1 + 0.08(z1-z2);
                draw z2--z2+0.08(z2-z1);
                dotlabel.ulft(btex $M$ etex,point 4 of (coupe shifted(z0+unitvector(z2-z1)*u)));
                draw point 4 of (coupe shifted(z0+ unitvector(z2-z1)*u))--z0 dashed evenly;
            \end{Geometrie}
            }
        \end{center}
    \end{minipage}
    \vspace*{-10mm}
\end{exemple*1}
\subsection{Boules}
\begin{definition}
    Une \textbf{boule} de centre $O$ et de rayon $r$ est l'ensemble des points de l'espace situés à une distance inférieures ou égale à $r$ du point $O$.
\end{definition}
\begin{exemples*1}
    Boules de pétanques, balles de golf, etc... ce sont des objets pleins.
\end{exemples*1}
\begin{propriete}[\admise]
    Si $(\mathscr B)$ est une boule de rayon $r$ alors son volume vaut $\dfrac43 \pi r^3$.
\end{propriete}
\begin{exemple*1}
    \titreExemple{Utilisation du vocabulaire}

    \begin{minipage}{0.5\linewidth}
        Sur la figure ci-contre :
        \begin{itemize}
            \item $[NS]$ est un diamètre de la boule.
            \item $[OM]$ est un rayon de la boule.
            \item $OM=r$.
        \end{itemize}
    \end{minipage}
    \begin{minipage}{0.5\linewidth}
        \begin{center}
            \scalebox{0.7}{
            \begin{Geometrie}
                u:=1cm;
                z0=(3,3)*u;
                dotlabel.urt(btex $O$ etex,z0);
                z1=(3.5,1)*u;
                label.lrt(btex $S$ etex,z1);
                z2=(2.5,5)*u;
                label.urt(btex $N$ etex,z2);
                path contour, ellipse,p,q,r,s,coupe;
                contour=(fullcircle scaled 4.1231cm);
                ellipse=(contour yscaled 0.25) rotated 14;
                coupe=(contour scaled 0.88 yscaled 0.15) rotated 14;
                draw z0--point 0 of ellipse shifted z0 dashed evenly;
                label.bot(btex $r$ etex,0.5[z0,point 0 of ellipse shifted z0]);
                draw contour shifted z0;
                draw (subpath (0.5*(length ellipse),length ellipse) of ellipse) shifted z0;
                draw (subpath (0,0.5*(length ellipse)) of ellipse) shifted z0 dashed evenly;
                draw z1--z2 dashed evenly;
                draw z1--z1 + 0.08(z1-z2);
                draw z2--z2+0.08(z2-z1);
                dotlabel.ulft(btex $M$ etex,point 4 of (coupe shifted(z0+unitvector(z2-z1)*u)));
                draw point 4 of (coupe shifted(z0+ unitvector(z2-z1)*u))--z0 dashed evenly;
            \end{Geometrie}
            }
        \end{center}
    \end{minipage}
    \vspace*{-10mm}
\end{exemple*1}
\begin{methode}[Aire d'une sphère et volume d'une boule.]
    \exercice
    \begin{enumerate}
        \item Dessiner une sphère de centre $O$ en perspective.
        \item Calculer l'aire exacte d'une sphère de rayon \Lg[cm]{12}.
        \item Calculer le volume exact d'une boule de rayon \Lg[cm]{12}.
        \item Arrondir ces résultats au centième.
        \item \textit{Soit une sphère d'aire \Aire[cm]{7850}}
        \begin{enumerate}
            \item Calculer la valeur exacte de son rayon.
            \item Calculer la valeur exacte du volume de la boule correspondante.
            \item Arrondir ce volume au \Vol[cm]{} près.
            \item Convertir ce volume en \Vol[dm]{}.
        \end{enumerate}
    \end{enumerate}
    \correction
    \begin{enumerate}
        \item Schéma
        
        \scalebox{0.6}{
            \begin{Geometrie}
                u:=1cm;
                z0=(3,3)*u;
                dotlabel.urt(btex $O$ etex,z0);
                z1=(3.5,1)*u;
                label.lrt(btex $S$ etex,z1);
                z2=(2.5,5)*u;
                label.urt(btex $N$ etex,z2);
                path contour, ellipse,p,q,r,s,coupe;
                contour=(fullcircle scaled 4.1231cm);
                ellipse=(contour yscaled 0.25) rotated 14;
                coupe=(contour scaled 0.88 yscaled 0.15) rotated 14;
                draw z0--point 0 of ellipse shifted z0 dashed evenly;
                label.bot(btex $r$ etex,0.5[z0,point 0 of ellipse shifted z0]);
                draw contour shifted z0;
                draw (subpath (0.5*(length ellipse),length ellipse) of ellipse) shifted z0;
                draw (subpath (0,0.5*(length ellipse)) of ellipse) shifted z0 dashed evenly;
                draw z1--z2 dashed evenly;
                draw z1--z1 + 0.08(z1-z2);
                draw z2--z2+0.08(z2-z1);
                dotlabel.ulft(btex $M$ etex,point 4 of (coupe shifted(z0+unitvector(z2-z1)*u)));
                draw point 4 of (coupe shifted(z0+ unitvector(z2-z1)*u))--z0 dashed evenly;
            \end{Geometrie}
        }
        \item $4\times\pi\times(\Lg[cm]{12})^2\approx\Aire[cm]{1810.56}$.
        \item $\dfrac43 \times\pi(\Lg[cm]{12})^3\approx\Vol[cm]{7238.23}$.
        \item Arrondir ces résultats au centième.
        \item \textit{Soit une sphère d'aire \Aire[cm]{7850}}
        \begin{enumerate}
            \item $R=\sqrt{\dfrac{\num{7850}}{4\times\pi}}$.
            \item $V=\dfrac43 \times\pi\times\left(\sqrt{\dfrac{\num{7850}}{4\times\pi}}\right)^3$ Argh !!!
            \item $V\approx\Vol[cm]{93960}$.
            \item $V\approx\Vol[dm]{94}$.
        \end{enumerate}
    \end{enumerate}
\end{methode} 