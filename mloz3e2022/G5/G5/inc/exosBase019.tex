\begin{exercice*}
    On réalise la section $ABB'A'$ par un plan parallèle à l'axe d'un cylindre.

    \begin{minipage}{0.65\linewidth}
        La hauteur $[OO']$ mesure \Lg[cm]{5} et le rayon $[OA]$ 
        mesure \Lg[cm]{3}. Le triangle $AOB$ soit rectangle en $O$.
    \end{minipage}
    \hfill        
    \begin{minipage}{0.3\linewidth}
        \begin{center}
            \scalebox{0.7}{
                \Solide[%
                    Nom=cylindre,%
                    % Axes,%
                    ListeSommets={I,J},%
                    Traces={
                        color A,B,C,D;
                        numeric SLangle;
                        SLangle:=270;
                        D=(\useKV[ClesSolides]{RayonCylindre}*cosd(SLangle),\useKV[ClesSolides]{RayonCylindre}*sind(SLangle),0);
                        C=(\useKV[ClesSolides]{RayonCylindre}*cosd(SLangle+90),\useKV[ClesSolides]{RayonCylindre}*sind(SLangle+90),0);
                        A-D=(0,0,\useKV[ClesSolides]{HauteurCylindre});
                        B-C=(0,0,\useKV[ClesSolides]{HauteurCylindre});
                        trace chemin(A,I,B,A);
                        trace chemin(D,J,C,D) dashed evenly;
                        trace chemin(A,D) dashed evenly;
                        trace chemin(B,C) dashed evenly;
                        trace chemin(I,J) dashed evenly;
                        trace codeperp(A,I,B,5);
                        trace codeperp(D,J,C,5);
                        Label.urt(btex $O$ etex,I);
                        Label.lrt(btex $O'$ etex,J);
                        Label.ulft(btex $A$ etex,A);
                        Label.lrt(btex $B$ etex,B);
                        Label.bot(btex $B'$ etex,C);
                        Label.urt(btex $A'$ etex,D);
                    }
                ]%
            }
        \end{center}
    \end{minipage}
    \begin{enumerate}
        \item Préciser la nature du triangle $AOB$.
        \item Justifier la nature de la section $ABB'A'$.
        \item Calculer l'aire de $ABB'A'$, arrondie au dixième.
    \end{enumerate}
\end{exercice*}
\begin{corrige}
    On réalise la section $ABB'A'$ par un plan parallèle à l'axe d'un cylindre.
    La hauteur $[OO']$ mesure \Lg[cm]{5} et le rayon $[OA]$ 
    mesure \Lg[cm]{3}. Le triangle $AOB$ soit rectangle en $O$.
    \begin{minipage}{\linewidth}
        \begin{center}
            \scalebox{0.8}{
                \Solide[%
                    Nom=cylindre,%
                    % Axes,%
                    ListeSommets={I,J},%
                    Traces={
                        color A,B,C,D;
                        numeric SLangle;
                        SLangle:=270;
                        D=(\useKV[ClesSolides]{RayonCylindre}*cosd(SLangle),\useKV[ClesSolides]{RayonCylindre}*sind(SLangle),0);
                        C=(\useKV[ClesSolides]{RayonCylindre}*cosd(SLangle+90),\useKV[ClesSolides]{RayonCylindre}*sind(SLangle+90),0);
                        A-D=(0,0,\useKV[ClesSolides]{HauteurCylindre});
                        B-C=(0,0,\useKV[ClesSolides]{HauteurCylindre});
                        trace chemin(A,I,B,A);
                        trace chemin(D,J,C,D) dashed evenly;
                        trace chemin(A,D) dashed evenly;
                        trace chemin(B,C) dashed evenly;
                        trace chemin(I,J) dashed evenly;
                        trace codeperp(A,I,B,5);
                        trace codeperp(D,J,C,5);
                        Label.urt(btex $O$ etex,I);
                        Label.lrt(btex $O'$ etex,J);
                        Label.ulft(btex $A$ etex,A);
                        Label.lrt(btex $B$ etex,B);
                        Label.bot(btex $B'$ etex,C);
                        Label.urt(btex $A'$ etex,D);
                    }
                ]%
            }
        \end{center}
    \end{minipage}
    \begin{enumerate}
        \item Préciser la nature du triangle $AOB$.
        
        {\color{red} $[OA]$ et $[OB]$ sont des rayons d'un même cercle donc le triangle $AOB$ est un triangle rectangle et isocèle en $O$.}
        \item Justifier la nature de la section $ABB'A'$.
        
        {\color{red}La section du cylindre par un plan parallèle à son axe est rectangulaire donc la section $ABB'A'$ est un rectangle.}
        \item Calculer l'aire de $ABB'A'$, arrondie au dixième.
        
        {\color{red}Le triangle $AOB$ est rectangle en $O$ donc, le théorème de Pythagore garantit que $OA^2+OB^2=AB^2$, d'où $AB^2=3^2+3^2=18$ et $AB=\sqrt{18}~\Lg[cm]{}$.        
        $AB\times AA' = \sqrt{18}\times 5 \approx \Aire[cm]{21.2}$.
        
        L'aire de $ABB'A'$ est donc d'environ $\Aire[cm]{21.2}$.
        }
    \end{enumerate}
\end{corrige}
