\begin{exercice*}
    On fait subir un agrandissement de coefficient 5 à une pyramide.
    La pyramide obtenue a un volume de \Vol[cm]{2 000}.
    
    Déterminer le volume de la pyramide de départ.
\end{exercice*}
\begin{corrige}
    On fait subir un agrandissement de coefficient 5 à une pyramide.
    La pyramide obtenue a un volume de \Vol[cm]{2 000}.
    
    Déterminer le volume de la pyramide de départ.

    {\color{red}Un agrandissement de coefficient $k$ multipie les longueurs par $k$ donc les volumes le sont par $k^3$.
    $\mathcal{V}_{\text{pyramide obtenue}}=5^3\times\mathcal{V}_{\text{pyramide initiale}}$
    
    soit $\mathcal{V}_{\text{pyramide initiale}}=\dfrac{\num{2000}}{5^3}=\dfrac{\num{2000}}{125}$ soit \Vol[cm]{16}.
    }
\end{corrige}
