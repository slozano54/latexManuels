\begin{exercice*}[Agrandissement de tétraèdre][\tice]
    Aller sur \href{https://www.geogebra.org/classic#3d}{\underline{\textcolor{blue}{Geogebra3D}}} \hspace*{10mm}\qrcode[hyperlink,height=0.4in]{https://www.geogebra.org/classic#3d}

    \begin{center}
        \begin{Geometrie}[CoinBG={u*(-10,-10)},TypeTrace="Espace"]
            Initialisation(1500,30,20,50);
            color O,A,B,C;
            O=(0,0,0);
            A=(1,0,0);
            B=(0,1,0);
            C=(0,0,1);
            path faceTetra;
            faceTetra = Projette(A)--Projette(B)--Projette(C)--cycle;
            fill faceTetra withcolor PaleGreen;
            trace faceTetra;
            % [Ox)
            drawarrow Projette(O)--Projette(A+1.5[O,A]) dashed dashpattern(on6 off3 on3 off 3);
            remplis (fullcircle scaled 1mm) shifted(Projette(O));
            remplis (fullcircle scaled 1mm) shifted(Projette(A));
            % [Oy)
            drawarrow Projette(O)--Projette(B+1.5[O,B]) dashed dashpattern(on6 off3 on3 off 3);
            remplis (fullcircle scaled 1mm) shifted(Projette(B));
            % [Oz)
            drawarrow Projette(O)--Projette(C+1.5[O,C]) dashed dashpattern(on6 off3 on3 off 3);
            remplis (fullcircle scaled 1mm) shifted(Projette(C));
        \end{Geometrie}
    \end{center}
    \begin{enumerate}
        \item Effectuer cette construction :
        \begin{itemize}
            \item Afficher la fenêtre Graphique.
            \item Créer un curseur n entier de 1 à 10, avec un incrément de 1 dans la fenêtre graphique 2D.
            \item Dans la zone de saisie, entrer les coordonnées de ces points pour les placer dans le repère :
            
            $O=(0,0,0)$ ; $A=( n ,0,0)$ ; $B=(0, n ,0)$ ; $C=(0,0, n )$.
            \item Construire le tétraèdre OABC à l'aide du bouton \textit{Pyramide}.
        \end{itemize}
        On s'intéresse à son volume.
        \item Compléter ce tableau.
        
        \medskip
        \begin{tabular}{|>{\columncolor{LightGray}\centering\arraybackslash}m{0.2\linewidth}|*{5}{>{\centering\arraybackslash}m{0.1\linewidth}|}}
            \hline
            n&1&2&3&4&5\\\hline
            Volume&&&&&\\\hline
        \end{tabular}

        \medskip
        \begin{tabular}{|>{\columncolor{LightGray}\centering\arraybackslash}m{0.2\linewidth}|*{5}{>{\centering\arraybackslash}m{0.1\linewidth}|}}
            \hline
            n&1&2&3&4&5\\\hline
            Volume&&&&&\\\hline
        \end{tabular}

        \medskip
        \item Déterminer si c'est un tableau de proportionnalité.        
    \end{enumerate}
\end{exercice*}
\begin{corrige}
    Pas de correction.
\end{corrige}
