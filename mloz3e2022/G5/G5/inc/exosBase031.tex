\begin{exercice*}
    Le gros globe de cristal est un trophée attribué au vainqueur de la coupe du monde de ski.
    Ce trophée pèse \Masse[kg]{9} et mesure \Lg[cm]{46} de hauteur.

    \begin{enumerate}
        \item Le biathlète français Martin Fourcade a remporté le sixième gros globe de cristal de
        sa carrière en 2017 à Pyeongchang en Corée du Sud.

        Donner approximativement la latitude et la longitude de ce lieu repéré sur la carte ci-dessous.

        \hspace*{-10mm}
            % \includegraphics[scale=0.5]{\currentpath/images/mappemonde}            
        \Cartographie[Projection,TypeProjection="winkel",VillesI={2,(130,35),"\noexpand\tiny Peyongchang",8},Echelle=1.5,Allegee=10]{}{}

        \item On considère que ce globe est composé d'un cylindre en cristal de diamètre \Lg[cm]{6}, surmonté d'une boule de cristal.
        Voir schéma ci-contre. Montrer qu'une valeur approchée du volume de la boule de ce trophée est de \Vol[cm]{6371}.

        \begin{center}
            \begin{minipage}{0.4\linewidth}
                \vspace*{-10mm}
                \includegraphics[scale=0.5]{\currentpath/images/globe}
            \end{minipage}            
            \begin{minipage}{0.4\linewidth}
                \begin{Geometrie}
                    pair A,B,C,D,E[];
                    numeric l;
                    l:=2.3;
                    A=u*(0.7,0.7);
                    B-A=u*(0,l);
                    C-B=u*(0,l);
                    D=A shifted (0.7*u*l,1.5*u*l);
                    E0=A shifted (0.7*u*l-0.3u,0);
                    E1=A shifted (0.7*u*l+0.3u,0);
                    E2=A shifted (0.7*u*l+0.3u,2.3u);
                    E3=A shifted (0.7*u*l-0.3u,2.3u);
                    trace cercles(D,1.15u);
                    trace E0--E1--E2--E3--cycle;
                    trace segment(A,A shifted (3u,0)) dashed evenly;
                    trace segment(B,B shifted (3u,0)) dashed evenly;
                    trace segment(C,C shifted (3u,0)) dashed evenly;
                    trace cotationmil(A,B,3mm,15,btex \Lg[cm]{23} etex);
                    trace cotationmil(B,C,3mm,15,btex \Lg[cm]{23} etex);
                    trace cotation(E0,E1,-3mm,-3mm,btex \Lg[cm]{6} etex);
                \end{Geometrie}
            \end{minipage}
        \end{center}
        
        \item Marie affirme que le volume de la boule de cristal représente environ 90\,\% du volume total du trophée.

        Déterminer si elle a raison.
    \end{enumerate}
\end{exercice*}
\begin{corrige}
        Le gros globe de cristal est un trophée attribué au vainqueur de la coupe du monde de ski.
    Ce trophée pèse \Masse[kg]{9} et mesure \Lg[cm]{46} de hauteur.

    \begin{enumerate}
        \item Le biathlète français Martin Fourcade a remporté le sixième gros globe de cristal de
        sa carrière en 2017 à Pyeongchang en Corée du Sud.

        Donner approximativement la latitude et la longitude de ce lieu repéré sur la carte ci-dessous.

        \textcolor{red}{Coordonnées de Peyongchang : \ang{130}E ; \ang{35}N}

        % \includegraphics[scale=0.5]{\currentpath/images/mappemonde}
        \hspace*{-8mm}            
        \Cartographie[Projection,TypeProjection="winkel",VillesI={2,(130,35),"\noexpand\tiny Peyongchang",8},Echelle=1.3,Allegee=10]{}{}

        \item On considère que ce globe est composé d'un cylindre en cristal de diamètre \Lg[cm]{6}, surmonté d'une boule de cristal.
        Voir schéma ci-contre. Montrer qu'une valeur approchée du volume de la boule de ce trophée est de \Vol[cm]{6371}.

        \begin{minipage}{0.4\linewidth}
            \vspace*{-5mm}
            \includegraphics[scale=0.3]{\currentpath/images/globe}
        \end{minipage}            
        \begin{minipage}{0.4\linewidth}
            \scalebox{0.6}{
            \begin{Geometrie}
                pair A,B,C,D,E[];
                numeric l;
                l:=2.3;
                A=u*(0.7,0.7);
                B-A=u*(0,l);
                C-B=u*(0,l);
                D=A shifted (0.7*u*l,1.5*u*l);
                E0=A shifted (0.7*u*l-0.3u,0);
                E1=A shifted (0.7*u*l+0.3u,0);
                E2=A shifted (0.7*u*l+0.3u,2.3u);
                E3=A shifted (0.7*u*l-0.3u,2.3u);
                trace cercles(D,1.15u);
                trace E0--E1--E2--E3--cycle;
                trace segment(A,A shifted (3u,0)) dashed evenly;
                trace segment(B,B shifted (3u,0)) dashed evenly;
                trace segment(C,C shifted (3u,0)) dashed evenly;
                trace cotationmil(A,B,3mm,15,btex \Lg[cm]{23} etex);
                trace cotationmil(B,C,3mm,15,btex \Lg[cm]{23} etex);
                trace cotation(E0,E1,-3mm,-3mm,btex \Lg[cm]{6} etex);
            \end{Geometrie}
            }
        \end{minipage}

        {\color{red}On sait que: $R = \Lg[cm]{11,5}$        
            $\mathcal{V}_{\text{boule de cristal}} = \dfrac{4}{3} \times  \pi \times  R^3 =  \dfrac{4}{3} \times  \pi \times   11,5^3 $
            
            $\mathcal{V}_{\text{boule de cristal}} \approx \Vol[cm]{6371}.$
        }
    % \end{enumerate}
    % \Coupe
    % \begin{enumerate}
    %     \setcounter{enumi}{2}
        \item Marie affirme que le volume de la boule de cristal représente environ 90\,\% du volume total du trophée.

        Déterminer si elle a raison.

        {\color{red}Calculons le volume du socle.
            
            $\mathcal{V}_{\text{socle}} = \pi r^2 \times H = \pi \times 3^2 \times 23 \approx \Vol[cm]{650}$
            
            $\mathcal{V}_{\text{trophée}}= \mathcal{V}_{\text{boule de cristal}} + \mathcal{V}_{\text{socle}}$.

            $\mathcal{V}_{\text{trophée}}=\approx \num{6371} + 650$.

            $\mathcal{V}_{\text{trophée}}\approx \Vol[cm]{7021}$.
            
            Or $\dfrac{\num{6371}}{\num{7021}}\approx  \num{0,907}$ soit environ 91\,\%.
            
            Marie a donc raison.
        }
    \end{enumerate}
\end{corrige}
