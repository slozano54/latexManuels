\begin{exercice*}
    Un triangle $A'B'C'$, rectangle en $A'$ et d'aire \Aire[cm]{27}, est un agrandissement d'un triangle $ABC$, rectangle en $A$, tel que $AB = \Lg[cm]{3}$ et $AC = \Lg[cm]{2}$.

    Calculer les longueurs $A'B'$ et $A'C'$.
\end{exercice*}
\begin{corrige}
    Un triangle $A'B'C'$, rectangle en $A'$ et d'aire \Aire[cm]{27}, est un agrandissement d'un triangle $ABC$, rectangle en $A$, tel que $AB = \Lg[cm]{3}$ et $AC = \Lg[cm]{2}$.

    Calculer les longueurs $A'B'$ et $A'C'$.

    {\color{red}$\mathcal{A}_\text{ABC}=\dfrac{3\times 2}{2}=\Aire[cm]{3}$.
    
    Un agrandissement de coefficient $k$ multipie les longueurs par $k$ donc les aires le sont par $k^2$. Ici, l'aire a été multipliée par 9 donc $k=3$ car $k$ est un nombre positif.
    
    Le coefficient d'aggrandissement vaut donc 3 donc $A'B'=3\times AB=\Lg[cm]{9}$ et $A'C'=3\times AC=\Lg[cm]{6}$.}
\end{corrige}
