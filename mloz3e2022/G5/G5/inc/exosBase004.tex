\begin{exercice*}
    Associer chaque patron au nom du solide qui lui correspond parmi prisme droit, pyramide, cône de révolution, cube, pavé droit et cylindre de révolution.

    \begin{minipage}[t][35mm][c]{0.45\linewidth}%1
        \scalebox{0.7}{%
        \begin{Geometrie}[CoinHD={u*(4.5,4.5)}]
            % trace feuillet;
            numeric pi;
            pi = 3.141592653589793;
            pair A[],B[];
            path cA[];
            u:=0.5*u;
            B0=u*(1,3.5);
            B1-B0=u*(2*pi,0);
            B2-B0=u*(2*pi,2);
            B3-B0=u*(0,2);
            A0-B0=u*(2,-1);
            A1-B0=u*(4,3);
            trace polygone(B0,B1,B2,B3);
            trace cercles(A0,u);
            trace cercles(A1,u);
        \end{Geometrie}
        }
        \vfill
        \textbf{a/}\pointilles
    \end{minipage}
    \hfill
    \begin{minipage}[t][35mm][c]{0.45\linewidth}
        \scalebox{0.7}{
            \begin{Geometrie}[CoinHD={u*(7,4.5)}]
                % trace feuillet;
                pair A[];
                A0=u*(1,1);
                A1-A0=u*(1.5,0);
                A2-A1=u*(1,0);
                A9-A1=u*(0,-1);
                A3-A2=u*(sqrt(3.25),0);
                A7-A0=u*(0,2);
                A6-A1=u*(0,2);
                A5-A2=u*(0,2);
                A4-A3=u*(0,2);
                A8-A6=u*(0,1);
                draw A0--A3--A4--A7--cycle;
                draw A8--A9;
                draw A8--A7;
                draw A0--A9;
                draw A2--A5;
            \end{Geometrie}
        }
        \vfill
        \textbf{b/}\pointilles
    \end{minipage}

    \begin{minipage}[t][35mm][c]{0.45\linewidth}%1
        \scalebox{1}{%
            \Patron{%
                oxoo,
                xxxx,
                ooxo
            }%
        }
        \vfill
        \textbf{c/}\pointilles
    \end{minipage}
    \hfill
    \begin{minipage}[t][35mm][c]{0.45\linewidth}
        \scalebox{1}{
            \begin{Geometrie}[CoinHD={u*(4.5,4.5)}]
                % trace feuillet;
                pair O[],A[];
                u:=0.5*u;
                O0=u*(6.5,4.5);
                O1-O0=u*(-4,0);
                trace cercles(O1,u);
                path cc;
                cc=cercles(O0,3u);
                A0=pointarc(cc,-120);
                A1=pointarc(cc,-240);
                trace arccercle(A1,A0,O0);
                trace chemin(A0,O0,A1);
            \end{Geometrie}
        }
        \vfill
        \textbf{d/}\pointilles
    \end{minipage}

    \begin{minipage}[t][35mm][c]{0.45\linewidth}%1
        \scalebox{0.7}{%
        \Patron[%
            Pave,
            Largeur=0.8cm,%
            Profondeur=2.5cm,%
            Hauteur=1.2cm]{%
            hopl,
            lhpo,
            hopo,
            lopo
        }
        }
        \vfill
        \textbf{e/}\pointilles
    \end{minipage}
    \hfill
    \begin{minipage}[t][35mm][c]{0.45\linewidth}
        \scalebox{1}{
            \begin{Geometrie}[CoinHD={u*(4.5,4.5)}]
                % trace feuillet;
                pair A[],S[];
                u:=0.4*u;
                A0=u*(6,7);                
                A1-A0=u*(2,-2);
                A2=rotation(A0,A1,90);
                A3=rotation(A1,A0,-90);
                trace polygone(A0,A1,A2,A3);                                
                S0=cercles(A0,4u) intersectionpoint cercles(A1,4u);
                S1=subpath (6,8) of cercles(A1,4u) intersectionpoint subpath (6,8) of cercles(A2,4u);
                S2=cercles(A3,4u) intersectionpoint subpath (4,6) of cercles(A2,4u);
                S3=cercles(A3,4u) intersectionpoint cercles(A0,4u);
                trace chemin(A0,S0,A1,S1,A2,S2,A3,S3,A0);
            \end{Geometrie}
        }
        \vfill
        \textbf{f/}\pointilles
    \end{minipage}
\end{exercice*}
\begin{corrige}
    Associer chaque patron au nom du solide qui lui correspond parmi prisme droit, pyramide, cône de révolution, cube, pavé droit et cylindre de révolution.

    \begin{minipage}[t][35mm][c]{0.5\linewidth}%1
        \scalebox{0.7}{%
        \begin{Geometrie}[CoinHD={u*(4.5,4.5)}]
            % trace feuillet;
            numeric pi;
            pi = 3.141592653589793;
            pair A[],B[];
            path cA[];
            u:=0.5*u;
            B0=u*(1,3.5);
            B1-B0=u*(2*pi,0);
            B2-B0=u*(2*pi,2);
            B3-B0=u*(0,2);
            A0-B0=u*(2,-1);
            A1-B0=u*(4,3);
            trace polygone(B0,B1,B2,B3);
            trace cercles(A0,u);
            trace cercles(A1,u);
        \end{Geometrie}
        }
        \vfill
        \textbf{a/}\textcolor{red}{Cylindre de révolution}
    \end{minipage}
    \hfill
    \begin{minipage}[t][35mm][c]{0.45\linewidth}
        \scalebox{0.7}{
            \begin{Geometrie}[CoinHD={u*(7,4.5)}]
                % trace feuillet;
                pair A[];
                A0=u*(1,1);
                A1-A0=u*(1.5,0);
                A2-A1=u*(1,0);
                A9-A1=u*(0,-1);
                A3-A2=u*(sqrt(3.25),0);
                A7-A0=u*(0,2);
                A6-A1=u*(0,2);
                A5-A2=u*(0,2);
                A4-A3=u*(0,2);
                A8-A6=u*(0,1);
                draw A0--A3--A4--A7--cycle;
                draw A8--A9;
                draw A8--A7;
                draw A0--A9;
                draw A2--A5;
            \end{Geometrie}
        }
        \vfill
        \textbf{b/}\textcolor{red}{Prisme droit}
    \end{minipage}

    \begin{minipage}[t][35mm][c]{0.45\linewidth}%1
        \scalebox{1}{%
            \Patron{%
                oxoo,
                xxxx,
                ooxo
            }%
        }
        \vfill
        \textbf{c/}\textcolor{red}{Cube}
    \end{minipage}
    \hfill
    \begin{minipage}[t][35mm][c]{0.45\linewidth}
        \scalebox{1}{
            \begin{Geometrie}[CoinHD={u*(4.5,4.5)}]
                % trace feuillet;
                pair O[],A[];
                u:=0.5*u;
                O0=u*(6.5,4.5);
                O1-O0=u*(-4,0);
                trace cercles(O1,u);
                path cc;
                cc=cercles(O0,3u);
                A0=pointarc(cc,-120);
                A1=pointarc(cc,-240);
                trace arccercle(A1,A0,O0);
                trace chemin(A0,O0,A1);
            \end{Geometrie}
        }
        \vfill
        \textbf{d/}\textcolor{red}{Cône de révolution}
    \end{minipage}

    \begin{minipage}[t][35mm][c]{0.45\linewidth}%1
        \scalebox{0.7}{%
        \Patron[%
            Pave,
            Largeur=0.8cm,%
            Profondeur=2.5cm,%
            Hauteur=1.2cm]{%
            hopl,
            lhpo,
            hopo,
            lopo
        }
        }
        \vfill
        \textbf{e/}\textcolor{red}{Pavé droit}
    \end{minipage}
    \hfill
    \begin{minipage}[t][35mm][c]{0.45\linewidth}
        \scalebox{1}{
            \begin{Geometrie}[CoinHD={u*(4.5,4.5)}]
                % trace feuillet;
                pair A[],S[];
                u:=0.4*u;
                A0=u*(6,7);                
                A1-A0=u*(2,-2);
                A2=rotation(A0,A1,90);
                A3=rotation(A1,A0,-90);
                trace polygone(A0,A1,A2,A3);                                
                S0=cercles(A0,4u) intersectionpoint cercles(A1,4u);
                S1=subpath (6,8) of cercles(A1,4u) intersectionpoint subpath (6,8) of cercles(A2,4u);
                S2=cercles(A3,4u) intersectionpoint subpath (4,6) of cercles(A2,4u);
                S3=cercles(A3,4u) intersectionpoint cercles(A0,4u);
                trace chemin(A0,S0,A1,S1,A2,S2,A3,S3,A0);
            \end{Geometrie}
        }
        \vfill
        \textbf{f/}\textcolor{red}{Pyramide}
    \end{minipage}
\end{corrige}
