\begin{exercice*}
    On empile des cubes pour construire ce solide.

    \begin{minipage}{0.2\linewidth}
        \VueCubes[%
        Seul,
        Creation,%
        Profondeur=3,Largeur=3]{%
            3,4,5,%
            2,3,4,%
            1,2,3%
        }    
    \end{minipage}
    \hfill
    \begin{minipage}{0.75\linewidth}
        \begin{enumerate}
            \item Déterminer le nombre minimum de cubes unités nécessaires pour obtenir un pavé droit.
            \item Déterminer alors le nombre de cubes que compte ce pavé.
        \end{enumerate}        
    \end{minipage}
\end{exercice*}
\begin{corrige}
    On empile des cubes pour construire ce solide.

    \begin{minipage}{0.2\linewidth}
        \VueCubes[%
        Seul,
        Creation,%
        Profondeur=3,Largeur=3]{%
            3,4,5,%
            2,3,4,%
            1,2,3%
        }    
    \end{minipage}
    \hfill
    \begin{minipage}{0.7\linewidth}
        \begin{enumerate}
            \item Déterminer le nombre minimum de cubes unités nécessaire pour obtenir un pavé droit.
            
            {\color{red} Il manque 3, 6 et 9 cubes respectivement de droite à gauche.
            
            Cela fait donc 18 en tout.}
            \item Déterminer alors le nombre de cubes que compte ce pavé.
            
            {\color{red} Le pavé droit comptera alors 
            
            $3\times 3\times 5 = 45$ cubes.
            }
        \end{enumerate}        
    \end{minipage}
\end{corrige}
