\begin{exercice*}
    Le culbuto ci-dessous est un jouet pour enfant qui oscille sur une base sphérique.
    \begin{center}
        \AssemblageSolides[%
            Type=ConeBoule,%
            HauteurConeA=2,%
            RayonCone=1,%
            Traces={
                color O,A,B,S;
                O=(0,0,0);
                A-O=(\useKV[Assemblage]{RayonCone}*cosd(120),\useKV[Assemblage]{RayonCone}*sind(120),0);
                B-O=(-\useKV[Assemblage]{RayonCone}*cosd(120),-\useKV[Assemblage]{RayonCone}*sind(120),0);
                S-O=(0,0,\useKV[Assemblage]{HauteurConeA});
                trace cotationmil(Projette(O),Projette(S),0mm,15,TEX("\Lg[cm]{20}"));
                trace cotationmil(Projette(B),Projette(A),-20mm,15,TEX("\Lg[cm]{20}"));
            }%
        ]
    \end{center}
    \begin{enumerate}
        \item Calculer son volume exact, puis arrondir au \Vol[cm]{}.
        \item La base sphérique est remplie de sable. Déterminer la proportion du jouet qui est occupée par le sable.
    \end{enumerate}
\end{exercice*}
\begin{corrige}
    Le culbuto ci-dessous est un jouet pour enfant qui oscille sur une base sphérique.
    
    \begin{center}
        \scalebox{0.9}{
            \AssemblageSolides[%
                Type=ConeBoule,%
                HauteurConeA=2,%
                RayonCone=1,%
                Traces={
                    color O,A,B,S;
                    O=(0,0,0);
                    A-O=(\useKV[Assemblage]{RayonCone}*cosd(120),\useKV[Assemblage]{RayonCone}*sind(120),0);
                    B-O=(-\useKV[Assemblage]{RayonCone}*cosd(120),-\useKV[Assemblage]{RayonCone}*sind(120),0);
                    S-O=(0,0,\useKV[Assemblage]{HauteurConeA});
                    trace cotationmil(Projette(O),Projette(S),0mm,15,TEX("\Lg[cm]{20}"));
                    trace cotationmil(Projette(B),Projette(A),-20mm,15,TEX("\Lg[cm]{20}"));
                }%
            ]
        }
    \end{center}

    \begin{enumerate}
        \item Calculer son volume exact, puis arrondir au \Vol[cm]{}.
        
        {\color{red} $\mathcal{V}_{\text{culbuto}}=\mathcal{V}_{\text{cône}} + \mathcal{V}_{\text{demi-boule}}$

        $\mathcal{V}_{\text{culbuto}}=\dfrac13\pi\times 10^2\times 20 + \dfrac12\times\dfrac43\pi\times 10^3$

        $\mathcal{V}_{\text{culbuto}}=\dfrac{\num{2000}\pi}{3}+\dfrac{\num{2000}\pi}{3}=\dfrac{\num{4000}\pi}{3}\approx\Vol[cm]{4189}$
        }
        \item La base sphérique est remplie de sable. Déterminer la proportion du jouet qui est occupée par le sable.
        
        {\color{red} $\dfrac{\mathcal{V}_{\text{demi-boule}}}{\mathcal{V}_{\text{culbuto}}}=\dfrac{\dfrac{\num{2000}\pi}{3}}{\dfrac{\num{4000}\pi}{3}}=\dfrac12$

        donc le sable occupe la moitié du volume du jouet.
        }
    \end{enumerate}
\end{corrige}
