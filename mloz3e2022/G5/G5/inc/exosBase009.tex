\begin{exercice*}
    Pour obtenir ce solide, on a empilé et collé \num{6} cubes de \Lg[cm]{4} d’arête et un prisme droit.
    La hauteur du prisme est égale à la moitié de l’arête des cubes.

    Calculer son volume en \Vol[cm]{}.
    
    \scalebox{0.9}{
    \begin{Geometrie}
        pair F,A[],B[],C[],D[],E[];
        numeric dec;
        dec:=1.5;
        % point de fuite
        F=u*(8,6);
        A0=u*(1,1);
        A1-A0=u*(0,dec);
        A2-A1=u*(0,dec);
        A5-A0=u*(dec,-0.3);
        A4-A5=u*(0,dec);
        A3-A4=u*(0,dec);
        % lignes de fuite
        path ligneFuite[];
        ligneFuite0=A0--F;
        ligneFuite1=A1--F;
        ligneFuite2=A2--F;
        ligneFuite5=A5--F;
        ligneFuite4=A4--F;
        ligneFuite3=A3--F;
        % placement des points du reseau
        B0=point 0.17*(length ligneFuite0) of ligneFuite0;
        B1=point 0.17*(length ligneFuite1) of ligneFuite1;
        B2=point 0.17*(length ligneFuite2) of ligneFuite2;
        B5=point 0.17*(length ligneFuite5) of ligneFuite5;
        B4=point 0.17*(length ligneFuite4) of ligneFuite4;
        B3=point 0.17*(length ligneFuite3) of ligneFuite3;
        C0=point 0.31*(length ligneFuite0) of ligneFuite0;
        C1=point 0.31*(length ligneFuite1) of ligneFuite1;
        C2=point 0.31*(length ligneFuite2) of ligneFuite2;
        C5=point 0.31*(length ligneFuite5) of ligneFuite5;
        C4=point 0.31*(length ligneFuite4) of ligneFuite4;
        C3=point 0.31*(length ligneFuite3) of ligneFuite3;
        D0=point 0.41*(length ligneFuite0) of ligneFuite0;
        D1=point 0.41*(length ligneFuite1) of ligneFuite1;
        D2=point 0.41*(length ligneFuite2) of ligneFuite2;
        D5=point 0.41*(length ligneFuite5) of ligneFuite5;
        D4=point 0.41*(length ligneFuite4) of ligneFuite4;
        D3=point 0.41*(length ligneFuite3) of ligneFuite3;
        E0=point 0.48*(length ligneFuite0) of ligneFuite0;
        E1=point 0.48*(length ligneFuite1) of ligneFuite1;
        E2=point 0.48*(length ligneFuite2) of ligneFuite2;
        E5=point 0.48*(length ligneFuite5) of ligneFuite5;
        E4=point 0.48*(length ligneFuite4) of ligneFuite4;
        E3=point 0.48*(length ligneFuite3) of ligneFuite3;        
        draw A0--A2--A3--A5--cycle;
        draw A1--A4--E4;
        draw A2--B2--B3--B5;
        draw A5--D5;
        draw A3--B3;
        draw C5--C4;                
        path pathPourIntersectionsUtiles[];
        pathPourIntersectionsUtiles0=B3--B5;
        pathPourIntersectionsUtiles1=C1--C4;
        draw pathPourIntersectionsUtiles0;
        draw C4--(pathPourIntersectionsUtiles0 intersectionpoint pathPourIntersectionsUtiles1);
        draw (pathPourIntersectionsUtiles0 intersectionpoint ligneFuite1)--D1;
        draw D1--D4--D5;
        D99-D5=D5-D0;
        D98-D4=D4-D1;
        E99-E5=E5-E0;
        E98-E4=E4-E1;
        draw D5--D99--D98--D4;
        draw D99--E99--E98--E4;
        draw D98--E98;
        draw D1--iso(D1,D2)--iso(D4,D3)--iso(E3,E4)--E4;
        draw iso(D4,D3)--D4;
        draw iso(D1,D2)--iso(E3,E4);
        % fleches pour faces
        pair legendeFleche[];
        legendeFleche0=iso(A2,B2)-0.5(A3-A2);
        drawarrow legendeFleche0--(iso(A2,B2)-0.1(A3-A2));
        label.top(btex Face arrière etex, legendeFleche0);
        legendeFleche1=iso(A0,A5)-0.5(B5-A5);
        drawarrow legendeFleche1--(iso(A0,A5)-0.1(B5-A5));
        label.bot(btex Face de gauche etex, legendeFleche1);
        legendeFleche2=iso(B5,C5)-0.5(A4-B5);
        drawarrow legendeFleche2--(iso(B5,C5)-0.1(A4-B5));
        label.lrt(btex Face avant etex, legendeFleche2);
        legendeFleche3=E4-0.8(D4-iso(E4,E98));
        drawarrow legendeFleche3--(E4-0.1(D4-iso(E4,E98)));
        label.urt(btex Face de droite etex, legendeFleche3);
    \end{Geometrie}
    }
\end{exercice*}    
\begin{corrige}
    Pour obtenir ce solide, on a empilé et collé \num{6} cubes de \Lg[cm]{4} d’arête et un prisme droit.
    La hauteur du prisme est égale à la moitié de l’arête des cubes.

    Calculer son volume en \Vol[cm]{}.

    \medskip
    \begin{Geometrie}
        pair F,A[],B[],C[],D[],E[];
        numeric dec;
        dec:=1.5;
        % point de fuite
        F=u*(8,6);
        A0=u*(1,1);
        A1-A0=u*(0,dec);
        A2-A1=u*(0,dec);
        A5-A0=u*(dec,-0.3);
        A4-A5=u*(0,dec);
        A3-A4=u*(0,dec);
        % lignes de fuite
        path ligneFuite[];
        ligneFuite0=A0--F;
        ligneFuite1=A1--F;
        ligneFuite2=A2--F;
        ligneFuite5=A5--F;
        ligneFuite4=A4--F;
        ligneFuite3=A3--F;
        % placement des points du reseau
        B0=point 0.17*(length ligneFuite0) of ligneFuite0;
        B1=point 0.17*(length ligneFuite1) of ligneFuite1;
        B2=point 0.17*(length ligneFuite2) of ligneFuite2;
        B5=point 0.17*(length ligneFuite5) of ligneFuite5;
        B4=point 0.17*(length ligneFuite4) of ligneFuite4;
        B3=point 0.17*(length ligneFuite3) of ligneFuite3;
        C0=point 0.31*(length ligneFuite0) of ligneFuite0;
        C1=point 0.31*(length ligneFuite1) of ligneFuite1;
        C2=point 0.31*(length ligneFuite2) of ligneFuite2;
        C5=point 0.31*(length ligneFuite5) of ligneFuite5;
        C4=point 0.31*(length ligneFuite4) of ligneFuite4;
        C3=point 0.31*(length ligneFuite3) of ligneFuite3;
        D0=point 0.41*(length ligneFuite0) of ligneFuite0;
        D1=point 0.41*(length ligneFuite1) of ligneFuite1;
        D2=point 0.41*(length ligneFuite2) of ligneFuite2;
        D5=point 0.41*(length ligneFuite5) of ligneFuite5;
        D4=point 0.41*(length ligneFuite4) of ligneFuite4;
        D3=point 0.41*(length ligneFuite3) of ligneFuite3;
        E0=point 0.48*(length ligneFuite0) of ligneFuite0;
        E1=point 0.48*(length ligneFuite1) of ligneFuite1;
        E2=point 0.48*(length ligneFuite2) of ligneFuite2;
        E5=point 0.48*(length ligneFuite5) of ligneFuite5;
        E4=point 0.48*(length ligneFuite4) of ligneFuite4;
        E3=point 0.48*(length ligneFuite3) of ligneFuite3;        
        draw A0--A2--A3--A5--cycle;
        draw A1--A4--E4;
        draw A2--B2--B3--B5;
        draw A5--D5;
        draw A3--B3;
        draw C5--C4;                
        path pathPourIntersectionsUtiles[];
        pathPourIntersectionsUtiles0=B3--B5;
        pathPourIntersectionsUtiles1=C1--C4;
        draw pathPourIntersectionsUtiles0;
        draw C4--(pathPourIntersectionsUtiles0 intersectionpoint pathPourIntersectionsUtiles1);
        draw (pathPourIntersectionsUtiles0 intersectionpoint ligneFuite1)--D1;
        draw D1--D4--D5;
        D99-D5=D5-D0;
        D98-D4=D4-D1;
        E99-E5=E5-E0;
        E98-E4=E4-E1;
        draw D5--D99--D98--D4;
        draw D99--E99--E98--E4;
        draw D98--E98;
        draw D1--iso(D1,D2)--iso(D4,D3)--iso(E3,E4)--E4;
        draw iso(D4,D3)--D4;
        draw iso(D1,D2)--iso(E3,E4);
        % fleches pour faces
        pair legendeFleche[];
        legendeFleche0=iso(A2,B2)-0.5(A3-A2);
        drawarrow legendeFleche0--(iso(A2,B2)-0.1(A3-A2));
        label.top(btex Face arrière etex, legendeFleche0);
        legendeFleche1=iso(A0,A5)-0.5(B5-A5);
        drawarrow legendeFleche1--(iso(A0,A5)-0.1(B5-A5));
        label.bot(btex Face de gauche etex, legendeFleche1);
        legendeFleche2=iso(B5,C5)-0.5(A4-B5);
        drawarrow legendeFleche2--(iso(B5,C5)-0.1(A4-B5));
        label.lrt(btex Face avant etex, legendeFleche2);
        legendeFleche3=E4-0.8(D4-iso(E4,E98));
        drawarrow legendeFleche3--(E4-0.1(D4-iso(E4,E98)));
        label.urt(btex Face de droite etex, legendeFleche3);
    \end{Geometrie}

    {\color{red}%
    $\mathcal{V}_{cube}=4^3=\Vol[cm]{64}$

    $\mathcal{V}_{prisme}=(4\times 4 \div 2)\times 2=\Vol[cm]{16}$

    $\mathcal{V}_{solide}=6\mathcal{V}_{cube}+\mathcal{V}_{prisme}$

    $\mathcal{V}_{solide}=6\times\Vol[cm]{64}+\Vol[cm]{16}$

    $\mathcal{V}_{solide}=\Vol[cm]{400}$

    Ce solide a un volume de \Vol[cm]{400}.
    }
\end{corrige}
