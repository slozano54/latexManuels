\begin{exercice*}
    Une figure a une aire de \Aire[cm]{124}. Après réduction, on obtient une nouvelle figure dont l'aire est \Aire[cm]{89,59}.
    
    Déterminer le rapport de réduction.
\end{exercice*}
\begin{corrige}
    Une figure a une aire de \Aire[cm]{124}. Après réduction, on obtient une nouvelle figure dont l'aire est \Aire[cm]{89,59}.
    
    Déterminer le rapport de réduction.

    {\color{red}Une réduction de coefficient $k$ multipie les longueurs par $k$ donc les aires le sont par $k^2$.
    Ici, l'aire a été multipliée par $\dfrac{\num{89.59}}{124}=\num{0.7225}$, $k$ étant un nombre positif, $k=\sqrt{\num{0.7225}}=\num{0.85}$.
    }
\end{corrige}
