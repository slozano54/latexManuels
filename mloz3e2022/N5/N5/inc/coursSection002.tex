\section{Équation-Produit}
\begin{propriete}[\admise]
    Si dans un produit l'un \textbf{au moins} des facteurs est nul alors le produit est nul.
\end{propriete}

Réciproquement,

\begin{propriete}[\admise]
    Si un produit est nul alors l'un \textbf{au moins} de ses facteurs est nul.
\end{propriete}    
\begin{myBox}{\emoji{light-bulb} En "langue" mathématique}
    \centerline{Si $a=0$ ou $b=0$ alors $a\times b=0$}
    \par Réciproquement 
    \par\centerline{Si $a\times b=0$ alors $a=0$ ou $b=0$}
\end{myBox}

\begin{exemple*1}
    Résoudre l'équation $(3x+7)(-8x-4)=0$
    
    \centerline{C'est une équation-produit alors}\par
    $$\Eqalign{
    3x-7&=0\qquad\mbox{ou}\qquad&-8x-4&=0\cr
    3x&=7\qquad\mbox{ou}\qquad&-8x&=4\cr
    x&=\frac73\qquad\mbox{ou}\qquad&x&=\frac4{-8}=-\frac12\cr
    }$$
    \par\centerline{Les solutions de l'équation $(3x-7)(-8x-4)=0$ sont $\dfrac73$ et $-\dfrac12$}
\end{exemple*1}
