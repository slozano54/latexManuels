\begin{exercice*}
    \begin{myBox}{\emoji{light-bulb}}
        Le professeur choisit trois nombres entiers relatifs consécutifs, rangés dans l’ordre croissant.
        \begin{itemize}
            \item Leslie calcule le produit du troisième nombre par le double du premier.
            \item Jonathan calcule le carré du deuxième nombre, puis il ajoute 2 au résultat obtenu.
        \end{itemize}
    \end{myBox}
    \begin{enumerate}
        \item Leslie a écrit le calcul suivant : $11\times (2 \times 9)$.
        
        Jonathan a écrit le calcul suivant : $10^2+2$.
        \begin{enumerate}
            \item Effectuer les calculs de Leslie et Jonathan.
            \item Déterminer les trois entiers choisis par le professeur.
        \end{enumerate}
    \begin{myBox}{\emoji{light-bulb}}
        Le professeur choisit maintenant trois nouveaux entiers. Leslie et Jonathan obtiennent alors tous les deux le même résultat.
    \end{myBox}
    \item Déterminer si le professeur a choisi $6$ comme deuxième nombre.
    \item Déterminer si le professeur a choisi $-7$ comme deuxième nombre.
    \begin{myBox}{\emoji{light-bulb}}
        Arthur prétend qu'en prenant pour inconnue le deuxième nombre entier, qu'il décide d'appeler n,
        l'équation $n^2 = 4$ permet de retrouver le ou les nombres choisis par le professeur.
    \end{myBox}
    \item Déterminer s'il a raison. Justifier la réponse en expliquant comment il a trouvé cette équation, puis donner les valeurs possibles des entiers choisis.
    \end{enumerate}
\end{exercice*}
% \begin{corrige}
%     ...
% \end{corrige}
