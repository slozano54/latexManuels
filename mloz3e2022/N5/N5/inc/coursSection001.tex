\section{Equations de référence}
\begin{definition}
    \begin{itemize}
        \item Une \textbf{équation} est une égalité dans laquelle intervient un nombre inconnu, le plus souvent représenté par une lettre.
        \item \textbf{Résoudre une équation}, c'est trouver toutes les valeurs possibles du nombre inconnu qui vérifient l'égalité : chacune de ces valeurs est appelée 
        \textbf{une solution de l'équation}.
    \end{itemize}
\end{definition}

\begin{propriete}[\admise]
    On ne change pas une égalité si on ajoute (ou on soustrait) le même nombre aux deux membres de l'égalité.
    \begin{center}
        Si $a=b$ alors $a+c=b+c$.
        \\
        Si $a=b$ alors $a-d=b-d$.
    \end{center}
\end{propriete}

\begin{propriete}[\admise]
    On ne change pas une égalité si on multiplie (ou on divise) par le même nombre \textbf{non nul} les deux membres de l'égalité.
    \begin{center}
    Si $a=b$ et $c\not=0$ alors $a\times c=b\times c$.
    \\
    Si $a=b$ et $d\not=0$ alors $a\div d=b\div d$.
    \end{center}
\end{propriete}

\begin{exemple*1}
    \phantom{rrr}

    \begin{minipage}{0.5\linewidth}    
        \begin{center}
            \begin{align*}
                x+1&=-4\\\intertext{\color{red}{On soustrait 1 aux deux membres}}
                x+1\color{red}{-1}&=-4\color{red}{-1}\\
                x&=-5\\            
            \end{align*}
        \end{center}
    \end{minipage}
    \begin{minipage}{0.5\linewidth}    
        \begin{center}
            \begin{align*}
                3x-6&=21\\\intertext{\color{red}{On ajoute 6 aux deux membres}}
                3x-6\color{red}{+6}&=21\color{red}{+6}\\            
                3x&=27\\\intertext{\color{red}{On divise les deux membres par 3}}
                \dfrac{3x}{\color{red}{3}}&=\frac{27}{\color{red}{3}}\\
                x&=9\\
            \end{align*}
        \end{center}
    \end{minipage}
\end{exemple*1}

\begin{exemple*1}
    \phantom{rrr}

    \begin{minipage}{0.5\linewidth}    
        \begin{center}
            \begin{align*}
                25&=4x+3\\\intertext{\color{red}{On soustrait 3 aux deux membres}}
                25\color{red}{-3}&=4x+3\color{red}{-3}\\
                22&=4x\\\intertext{\color{red}{On divise les deux membres par 4}}
                \dfrac{22}{\color{red}{4}}&=\frac{4x}{\color{red}{4}}\\
                x&=\frac{22}{4}\\
            \end{align*}
        \end{center}
    \end{minipage}
    \begin{minipage}{0.5\linewidth}    
        \begin{center}
            \begin{align*}
                4x-5&=x+19\\\intertext{\color{red}{On soustrait $x$ aux deux membres}}
                4x\color{red}{-x}-5&=x\color{red}{-x}+19\\
                3x-5&=19\\\intertext{\color{red}{On ajoute 5 aux deux membres}}
                3x-5\color{red}{+5}&=19\color{red}{+5}\\
                3x&=24\\\intertext{\color{red}{On divise les deux membres par 3}}
                \frac{x}{\color{red}{3}}&=\frac{24}{\color{red}{3}}\\
                x&=8\\
            \end{align*}
        \end{center}
    \end{minipage}

    \begin{minipage}{0.5\linewidth}    
        \begin{center}
            \begin{align*}
                \dfrac{x}{3}&=-4\\\intertext{\color{red}{On multiplie les deux membres par 3}}
                \dfrac{x}{3}\color{red}{\times 3}&=-4\color{red}{\times 3}\\
                x=-12\\
            \end{align*}
        \end{center}
    \end{minipage}
    \begin{minipage}{0.5\linewidth}    
        \begin{center}
            \begin{align*}
                \dfrac{7}{x}&=21\\\intertext{\color{red}{Les produits en croix sont égaux}}
                x\times 21&=7\\\intertext{\color{red}{On divise les deux membres par 21}}
                \frac{x \times 21}{\color{red}{21}}&=\frac{7}{\color{red}{21}}\\
                x&=\frac13\\
            \end{align*}
        \end{center}
    \end{minipage}

    \vspace*{-15mm}
    \begin{minipage}{\linewidth}    
        \begin{center}
            \begin{align*}
                5x+3&=7x-5\\\intertext{\color{red}{On soustrait $5x$ aux deux membres}}
                5x+3\color{red}{-5x}&=7x-5\color{red}{-5x}\\
                3&=2x-5\\\intertext{\color{red}{On ajoute 5 aux deux membres}}
                3\color{red}{+5}&=2x-5\color{red}{+5}\\
                8&=2x\\\intertext{\color{red}{On divise les deux membres par 2}}
                8\color{red}{\div 2}&=2x\color{red}{\div 2}\\
                x&=4\\
            \end{align*}
        \end{center}
    \end{minipage}
\end{exemple*1}