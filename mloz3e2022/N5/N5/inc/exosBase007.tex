\begin{exercice*}
    Soit $A=(2x-8)(x+3) + 5(x+3)$
    \begin{enumerate}
        \item Factoriser $A$.
        \item Calculer $A$ pour $x=4$.
        \item Résoudre $A=0$.
    \end{enumerate}
\end{exercice*}
\begin{corrige}
    Soit $A=(2x-8)(x+3) + 5(x+3)$

    \begin{enumerate}
        \item Factoriser $A$.
        
        $A=(x+3)\times[(2x-8)+5]$\\
        $A=(x+3)\times(2x-8+5)$\\
        $A=(x+3)\times(2x-3)$\\
        $A=(x+3)(2x-3)$
        \item Calculer $A$ pour $x=4$.
        
        $A=(x+3)(2x-3)$\\
        $A=(4+3)(2\times 4 -3)$\\
        $A=7\times 5$\\
        $A=35$
    \end{enumerate}
    \Coupe
    \begin{enumerate}
        \setcounter{enumi}{2}
        \item Résoudre $A=0$.
        
        \ResolEquation[Produit,Facteurs,Entier,Simplification]{1}{3}{2}{-3}
    \end{enumerate}
\end{corrige}
