\begin{exercice*}
    \begin{itemize}
        \item Jean Rage pense à un nombre.
        \item Il lui soustrait \num{10}.
        \item Il élève le tout au carré.
        \item Il soustrait au résultat le carré du nombre auquel il a pensé.
        \item Il obtient \num{-340}.
    \end{itemize}
\end{exercice*}
\begin{corrige}
    \begin{itemize}
        \item Jean Rage pense à un nombre. \textcolor{red}{$x$}
        \item Il lui soustrait \num{10}. \textcolor{red}{$x-10$}
        \item Il élève le tout au carré. \textcolor{red}{$(x-10)^2$}
        \item Il soustrait au résultat le carré du nombre auquel il a pensé. \textcolor{red}{$(x-10)^2-x^2$}
        \item Il obtient \num{-340}. \textcolor{red}{$(x-10)^2-x^2=-340$}
    \end{itemize}    

    Par exemple, on développe et on réduit pour obtenir l'équation : $-20x+100=-340$

    \ResolEquation[Decomposition,Fleches,Ecart=0.85,Simplification,Decimal]{-20}{100}{0}{-340}
\end{corrige}
