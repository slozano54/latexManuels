\vspace*{-5mm}
\begin{changemargin}{-7mm}{-7mm}
    %pre-001
    \begin{prerequis}[Connaisances \emoji{red-heart} et compétences \emoji{diamond-suit} du cycle 3]    
   \begin{itemize}        
       \item[\emoji{red-heart}] Vocabulaire associé à ces objets et à leurs propriétés : côté, sommet, angle, hauteur.
       \columnbreak
       \item[\emoji{diamond-suit}] Reconnaître, nommer, décrire des triangles, dont les triangles particuliers (triangle rectangle, triangle isocèle, triangle équilatéral).       
   \end{itemize}
\end{prerequis}
    % \begin{debat}[multiplication par 9]
    %     \begin{minipage}{0.8\linewidth} 
    %         On peut facilement retrouver le résultat de la \og table des 9 \fg{} avec\dots{}\\
    %         une paire de main ! \\\medskip
    %         Par exemple, calculons $8\times9$ :
    %         \begin{itemize}
    %             \item placer les faces des mains devant soi, abaisser le 8\up{e} doigt en partant de la gauche ;
    %             \item les doigts à gauche du doigt abaissé correspondent au nombre de dizaines ;
    %             \item les doigts à droite du doigt abaissé correspondent au nombre d'unités.
    %         \end{itemize}
    %         Ici, $8\times9 =72$. \\\medskip
    %         Cette technique fonctionne pour la multiplication par 9 d'un nombre entier compris entre 1 et 10 inclus.
    %     \end{minipage}
    %     \hfill
    %     \begin{minipage}{0.15\linewidth}             
    %         \begin{center}
    %             \includegraphics[width=4cm]{\currentpath/images/mains}
    %         \end{center}
    %     \end{minipage}
    %     \begin{cadre}[B2][F4]
    %     \begin{center}
    %             \hrefVideo{https://www.yout-ube.com/watch?v=BPL7gmfH7V8}{\bf Table de multiplication avec les doigts}
                
    %             \vspace*{2mm}
    %             Chaîne YouTube {\it Les Parents Créatifs}.
    %     \end{center}
    %     \end{cadre}
    % \end{debat}
\end{changemargin}