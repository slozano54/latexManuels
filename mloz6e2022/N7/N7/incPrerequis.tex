\vspace*{-5mm}
\begin{changemargin}{-7mm}{-7mm}
    %pre-001
    \begin{prerequis}[Connaisances \emoji{red-heart} et compétences \emoji{diamond-suit} du cycle 3]    
   \begin{itemize}        
       \item[\emoji{red-heart}] Vocabulaire associé à ces objets et à leurs propriétés : côté, sommet, angle, hauteur.
       \columnbreak
       \item[\emoji{diamond-suit}] Reconnaître, nommer, décrire des triangles, dont les triangles particuliers (triangle rectangle, triangle isocèle, triangle équilatéral).       
   \end{itemize}
\end{prerequis}
    \begin{debat}[Le symbole du pourcentage] 
        Dans les textes du Moyen Âge, on peut voir des notations comme \og per cento \fg{} ou \og per c. \fg{} ou \og p. cento \fg. La première trace d'un symbole voisin de celui utilisé actuellement se trouverait dans un manuscrit italien anonyme, écrit vers 1425, sous la forme : $P.c$\degre. \\
        Le \og $P$ \fg{} s'est ensuite perdu, et on trouve, vers 1650, la notation : $\dfrac{o}{o}$, puis la barre est devenue oblique. Les deux \og o \fg{} sont ensuite assimilés aux deux zéros de 100. \\
        \begin{center}
           \textcolor{B1}{\fontsize{70}{80}\selectfont \%}
        \end{center}
        \bigskip
        \begin{cadre}[B2][F4]
           \begin{center}
             \hrefVideo{https://www.yout-ube.com/watch?v=gLbsxj8mv-U}{\bf Facture d'électricité}, issues de journaux télévisés.
           \end{center}
        \end{cadre}
     \end{debat}
\end{changemargin}