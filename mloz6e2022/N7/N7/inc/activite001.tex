\begin{changemargin}{-10mm}{-15mm}
\begin{activite}[Carte, échelle]
    \vspace*{-7mm}
    \hspace*{-10mm}{\bf Objectifs :} calculer des distances ; résoudre un problème d'échelle.    
    \begin{center}
        \includegraphics[width=16cm]{\currentpath/images/europe}
    \end{center}
    \partie[Quelques distances]
        Sachant que la distance à vol d'oiseau de Lisbonne à Amsterdam est de \Lg[km]{1800}, répondre aux questions :
        \begin{enumerate}
            \item Quelle est la distance de Lisbonne à Paris ? \dotfill 
            \item Quelle est la distance de Paris à Oslo ? \dotfill 
            \item Quelle est la distance de Londres à Dublin ? \dotfill 
        \end{enumerate}     
    \partie[Construction de l'échelle]
    \vspace*{-5mm}
        \begin{enumerate}
            \item À combien de kilomètres dans la réalité correspond 1 centimètre sur la carte ? \dotfill 
            \item Par combien de centimètres sur la carte sont représentés 500 km dans la réalité ? \dotfill
            \item Ajouter une échelle sur la carte.
        \end{enumerate} 
    \vspace*{-5mm}
    \partie[Tableau récapitulatif]
        \begin{center}
            {\small
            \renewcommand{\arraystretch}{1.2}
            \begin{Ctableau}{\linewidth}{7}{c}
            \hline
            Trajet & Lisb.-Ams. & Lisb.-Paris & Paris-Oslo & Lond.-Dub. & & \\
                \hline
            Distance réelle en km & & & & & 200 & \\
            \hline
            Distance sur la carte en cm & & & & & & 1 \\
            \hline
        \end{Ctableau}}
        \end{center}
 \end{activite}
 \vspace*{-25mm}
\end{changemargin}