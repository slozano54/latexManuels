%pre-001
\begin{prerequis}[Connaisances \emoji{red-heart} et compétences \emoji{diamond-suit} du cycle 3]    
   \begin{itemize}        
       \item[\emoji{red-heart}] Vocabulaire associé à ces objets et à leurs propriétés : côté, sommet, angle, hauteur.
       \columnbreak
       \item[\emoji{diamond-suit}] Reconnaître, nommer, décrire des triangles, dont les triangles particuliers (triangle rectangle, triangle isocèle, triangle équilatéral).       
   \end{itemize}
\end{prerequis}
\vfill
\begin{debat}[des aires en images]
    Dans un rectangle, on peut mettre deux triangles de même surface, son {\bf aire} correspond donc au double de l'aire de l'un de ces triangles.
    \begin{center}
       \begin{pspicture}(0,-0.5)(8.5,2.5)
          \psframe[fillstyle=solid,fillcolor=A3,linecolor=A3](0.5,0)(3.5,2)
          \psframe[fillstyle=solid,fillcolor=A3,linecolor=A3](5,0)(8.5,2)
          \pspolygon[linecolor=B1](5,0)(8.5,0)(6,2)
          \psline[linecolor=B1,linestyle=dashed](6,0)(6,2)
          \psarc{<-}(5.5,1){0.3}{-45}{180}
          \psarc{->}(7.25,1){0.3}{25}{250}
       \end{pspicture}
    \end{center}
    \begin{cadre}[B2][F4]
       \begin{center}
          \hrefVideo{https://www.youtube.com/watch?v=5_PiZrfLghQ}{\bf Aire de figures simples}, chaîne YouTube de {\it Science silencieuse}.
       \end{center}
    \end{cadre}
\end{debat}
\vfill