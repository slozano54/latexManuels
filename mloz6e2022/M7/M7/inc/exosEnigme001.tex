% Les enigmes ne sont pas numérotées par défaut donc il faut ajouter manuellement la numérotation
% si on veut mettre plusieurs enigmes
%\refstepcounter{exercice}
%\numeroteEnigme
\vspace*{-10mm}
\begin{enigme}[La formule de Pick]
    On travaille dans un réseau pointé à maille carrée. On note $u.\ell.$ l'unité de longueur et $u.a.$ l'unité d'aire. \\
    On appelle polygone de Pick, un polygone non aplati construit sur un tel réseau et dont chacun des sommets est un point du réseau. On considère la figure $FORMULES$ suivante :
    \begin{center}
       \small
       {\psset{unit=0.5}
       \begin{pspicture}(6,0)(22,11.5)
          \pstGeonode[fillstyle=solid,fillcolor=lightgray!30,CurveType=polygon,PosAngle={90,135,-135,-45,-45,45,45}](9,10){R}(7,6){O}(7,1){F}(12,1){S}(12,3){E}(16,3){L}(16,6){U}(12,6){M}
          \psline[fillstyle=solid,fillcolor=lightgray!30](9,10)(7,6)(7,1)(12,1)(12,3)(16,3)(16,6)(12,6)(9,10)
          \psframe[fillstyle=solid,fillcolor=lightgray!30](20,5)(21,6)
          \psgrid[griddots=1,gridlabels=0,subgriddiv=1,gridwidth=0.5mm](6,0)(22,11)
          \rput(20.5,4.5){1 $u.a.$}
          \psline{<->}(20,9)(21,9)
          \rput(20.5,8.5){1 $u.\ell.$}      
       \end{pspicture}}
    \end{center}
 
    \partie[avec les formules classiques]
        \vspace*{-5mm}
        \begin{enumerate}
            \item Décomposer la figure $FORMULES$ en figures connues dont on connait les formules de calcul d'aire ? \\ [2mm]
                \makebox[\linewidth]{\dotfill} \\
            \item Calculer alors l'aire du polygone $FORMULES$, en unités d'aire, en détaillant les étapes du raisonnement. \\ [3mm]
                \makebox[\linewidth]{\dotfill} \\ [3mm]
                \makebox[\linewidth]{\dotfill} \\ [3mm]
                \makebox[\linewidth]{\dotfill}
        \end{enumerate}
 
    \partie[avec la formule de Pick]
        \vspace*{-5mm}
        La formule de Pick permet de calculer l'aire $\mathcal{A}$ d'un polygone de Pick, à partir du nombre $i$ de points du réseau strictement intérieurs à ce polygone et du nombre $b$ de points du réseau sur le bord du polygone : \fbox{$\mathcal{A} =i+\dfrac{b}{2}-1$}. 
        \begin{enumerate}
        \setcounter{enumi}{2}
            \item Appliquer cette formule au polygone $FORMULES$. \\ [3mm]
                $i =\makebox[0.23\linewidth]{\dotfill} \quad b =\makebox[0.23\linewidth]{\dotfill}$ \quad donc, $\mathcal{A} = \makebox[0.23\linewidth]{\dotfill}$ \\
            \item Appliquer la formule de Pick aux trois polygones de Pick $MOFS$, $MOR$ et $MULE$. \\
            Vérifier que la somme des résultats obtenus est égale à l'aire totale de la figure. \\ [3mm]
            $MOFS : \;\; i =\makebox[0.23\linewidth]{\dotfill} \quad b =\makebox[0.23\linewidth]{\dotfill}$ \quad donc, $\mathcal{A}_1 = \makebox[0.23\linewidth]{\dotfill}$ \\ [3mm]
            $MOR : \quad i =\makebox[0.23\linewidth]{\dotfill} \quad b =\makebox[0.23\linewidth]{\dotfill}$ \quad donc, $\mathcal{A}_2 = \makebox[0.23\linewidth]{\dotfill}$ \\ [3mm]
            $MULE : \; i =\makebox[0.23\linewidth]{\dotfill} \quad b =\makebox[0.23\linewidth]{\dotfill}$ \quad donc, $\mathcal{A}_3 = \makebox[0.23\linewidth]{\dotfill}$ \\ [3mm]
            Aire totale : \dotfill
        \end{enumerate}
\end{enigme}  
% % Pour le corrigé, il faut décrémenter le compteur, sinon il est incrémenté deux fois
% \addtocounter{exercice}{-1}
% \begin{corrige}
%     Correction enigme de la fin de la partie cours.  
%     
% \end{corrige}