\vspace*{-10mm}
\section{Aires usuelles}
\subsection{Triangles}
{\renewcommand*{\arraystretch}{1.5}
      \begin{longtable}{|m{0.2\textwidth}|>{\centering\arraybackslash}m{0.2\textwidth}|m{0.15\textwidth}|>{\centering\arraybackslash}m{0.425\textwidth}|}%
        \hline
        \rowcolor{gray!20}\multicolumn{1}{|c|}{\bf Nom de la figure}&{\bf Représentation}&\multicolumn{1}{|c|}{\bf Périmètre}&{\bf Exemple}\\
        \hline
        \textbf{ Triangle} de côté $b$ et de hauteur relative à ce côté $h$&
        \hspace*{-10mm}
        \begin{Geometrie}[CoinHD={u*(5,4)}]
            trace feuillet withcolor white;
            pair A,B,C,H;
            A=u*(1,1);
            B-A=u*(3,0);            
            C-A=u*(1,2);
            H=projection(C,A,B);
            trace polygone(A,B,C);
            trace cotationmil(C,H,3mm,5,btex $h$ etex) withcolor red;
            trace segment(C,H) withcolor red;
            trace segment(A,B) withcolor blue;
            trace cotationmil(A,B,-3mm,5,btex $b$ etex) withcolor blue;
            trace codeperp(B,H,C,5) withcolor red;
        \end{Geometrie}        
        &$\Eqalign{
        {\mathcal A}&=\dfrac{b\times h}{2}\cr
        }$&
        \begin{minipage}{0.18\textwidth}
          \vspace*{-10mm}\hspace*{-15mm}
        \begin{Geometrie}[CoinHD={u*(5,4)}]
          trace feuillet withcolor white;
          pair A,B,C,H;
          A=u*(1,1);
          B-A=u*(3,0);            
          C-A=u*(1,2);
          H=projection(C,A,B);
          trace polygone(A,B,C);
          trace cotationmil(C,H,3mm,15,btex \Lg[mm]{26} etex) withcolor red;
          trace segment(C,H) withcolor red;
          trace segment(A,B) withcolor blue;
          trace cotationmil(A,B,-3mm,15,btex \Lg[mm]{27} etex) withcolor blue;
          trace codeperp(B,H,C,5) withcolor red;
        \end{Geometrie}
      \end{minipage}      
      \begin{minipage}{0.124\textwidth}
      $\Eqalign{
        {\mathcal A}&=\dfrac{\Lg[mm]{27}\times \Lg[mm]{26}}{2}\cr
        {\mathcal A}&=\Aire[mm]{351}\cr
        }$
      \end{minipage}        
        \\
        \hline
      \end{longtable}
}
% \begin{center}
% \begin{tabular}{|m{8cm}|m{4.25cm}|c|}
% \hline
% \multicolumn{1}{|c|}{\bf Nom de la figure}&\multicolumn{1}{c|}{\bf Représentation}&\multicolumn{1}{c|}{\bf P\'erim\`{e}tre et aire}\\
% \hline
% \textbf{ Triangle} de côté $c$ et de hauteur relative à ce côté $h$&\includegraphics{aires_vol.6}&$\Eqalign{
% {\cal P}&=\mbox{somme des côtés}\cr
% {\cal A}&=\dfrac{c\times h}{2}\cr
% }$\\
% \hline
% \end{tabular}
% \end{center}

\subsection{Quadilat\`{e}res}
Mettre trapèze et parallélogrgramme à la fin avec une note de justification à partir du triangle.

Deux tableaux Rectangle, carré, losange puis trapèze et parallelo.

{\renewcommand*{\arraystretch}{1.5}
      \begin{longtable}{|m{0.25\textwidth}|>{\centering\arraybackslash}m{0.25\textwidth}|m{0.15\textwidth}|>{\centering\arraybackslash}m{0.325\textwidth}|}%
        \hline
        \rowcolor{gray!20}\multicolumn{1}{|c|}{\bf Nom de la figure}&{\bf Représentation}&\multicolumn{1}{|c|}{\bf Périmètre}&{\bf Exemple}\\
        \hline
        a&b&c&d\\\hline
        a&b&c&d\\\hline
        a&b&c&d\\\hline
      \end{longtable}
}
% \begin{center}
% \begin{tabular}{|m{8cm}|m{4.25cm}|c|}
% \hline
% \multicolumn{1}{|c|}{\bf Nom de la figure}&\multicolumn{1}{c|}{\bf Représentation}&\multicolumn{1}{c|}{\bf P\'erim\`{e}tre et aire}\\
% \hline
% \textbf{ Trapèze} de petite base $b$, de grande base $B$ et de hauteur $h$&\includegraphics{aires_vol.3}&$\Eqalign{
% {\cal P}&=\mbox{somme des côtés}\cr
% {\cal A}&=\frac{(B+b)\times h}{2}\cr
% }$\\
% \hline
% \textbf{ Parallélogramme} de côté $c$ et de hauteur relative à ce côté $h$&\includegraphics{aires_vol.4}&$\Eqalign{
% {\cal P}&=\mbox{somme des côtés}\cr
% {\cal A}&=c\times h\cr
% }$\\
% \hline
% \textbf{ Losange} de côté $c$, de grande diagonale $D$ et de petite diagonale $d$&\includegraphics{aires_vol.5}&$\Eqalign{
% {\cal P}&=4c\cr
% {\cal A}&=\dfrac{d\times D}{2}\cr
% }$\\
% \hline
% \textbf{ Rectangle} de longueur $L$ et de largeur $l$&\includegraphics{aires_vol.2}&$\Eqalign{
% {\cal P}&=2(l+L)\cr
% {\cal A}&=L\times l\cr
% }$\\
% \hline
% \textbf{ Carré} de côté $c$&\includegraphics{aires_vol.1}&$\Eqalign{
% {\cal P}&=4c\cr
% {\cal A}&=c^2\cr
% }$\\
% \hline
% \end{tabular}
% \end{center}

\subsection{Disques}
{\renewcommand*{\arraystretch}{1.5}
      \begin{longtable}{|m{0.25\textwidth}|>{\centering\arraybackslash}m{0.25\textwidth}|m{0.15\textwidth}|>{\centering\arraybackslash}m{0.325\textwidth}|}%
        \hline
        \rowcolor{gray!20}\multicolumn{1}{|c|}{\bf Nom de la figure}&{\bf Représentation}&\multicolumn{1}{|c|}{\bf Périmètre}&{\bf Exemple}\\
        \hline
        a&b&c&d\\\hline
        a&b&c&d\\\hline
        a&b&c&d\\\hline
      \end{longtable}
}
% \begin{center}
% \begin{tabular}{|m{8cm}|m{4.25cm}|c|}
% \hline
% \multicolumn{1}{|c|}{\bf Nom de la figure}&\multicolumn{1}{c|}{\bf Représentation}&\multicolumn{1}{c|}{\bf P\'erim\`{e}tre et aire}\\
% \hline
% \textbf{ Cercle et disque} de rayon $r$&\includegraphics{aires_vol.7}&$\Eqalign{
% {\cal P}&=2\pi r\cr
% {\cal A}&=\pi r^2\cr
% }$\\
% \hline
% \end{tabular}
% \end{center}