\vspace*{-10mm}
\section{Aires usuelles}
\subsection{Triangles}
{\renewcommand*{\arraystretch}{1.5}
      \begin{longtable}{|m{0.2\textwidth}|>{\centering\arraybackslash}m{0.2\textwidth}|m{0.15\textwidth}|>{\centering\arraybackslash}m{0.425\textwidth}|}%
        \hline
        \rowcolor{gray!20}\multicolumn{1}{|c|}{\bf Nom de la figure}&{\bf Représentation}&\multicolumn{1}{|c|}{\bf Aire}&{\bf Exemple}\\
        \hline
        {\textbf{ Triangle} de côté $b$ et de hauteur relative\par à ce côté $h$}&
        \hspace*{-10mm}
        \begin{Geometrie}[CoinHD={u*(5,4)}]
            trace feuillet withcolor white;
            pair A,B,C,H;
            A=u*(1,1);
            B-A=u*(3,0);            
            C-A=u*(1,2);
            H=projection(C,A,B);
            trace polygone(A,B,C);
            trace cotationmil(C,H,3mm,5,btex $h$ etex) withcolor red;
            trace segment(C,H) withcolor red;
            trace segment(A,B) withcolor blue;
            trace cotationmil(A,B,-3mm,5,btex $b$ etex) withcolor blue;
            trace codeperp(B,H,C,5) withcolor red;
        \end{Geometrie}        
        &$\Eqalign{
        {\mathcal A}&=\dfrac{b\times h}{2}\cr
        }$&
        \begin{minipage}{0.18\textwidth}
          \vspace*{-10mm}\hspace*{-15mm}
        \begin{Geometrie}[CoinHD={u*(5,4)}]
          trace feuillet withcolor white;
          pair A,B,C,H;
          A=u*(1,1);
          B-A=u*(3,0);            
          C-A=u*(1,2);
          H=projection(C,A,B);
          trace polygone(A,B,C);
          trace cotationmil(C,H,3mm,15,btex \Lg[mm]{26} etex) withcolor red;
          trace segment(C,H) withcolor red;
          trace segment(A,B) withcolor blue;
          trace cotationmil(A,B,-3mm,15,btex \Lg[mm]{27} etex) withcolor blue;
          trace codeperp(B,H,C,5) withcolor red;
        \end{Geometrie}
      \end{minipage}      
      \begin{minipage}{0.124\textwidth}
      $\Eqalign{
        {\mathcal A}&=\dfrac{\Lg[mm]{27}\times \Lg[mm]{26}}{2}\cr
        {\mathcal A}&=\Aire[mm]{351}\cr
        }$
      \end{minipage}        
        \\
        \hline
      \end{longtable}
}

\subsection{Quadilatères}
{\renewcommand*{\arraystretch}{1.5}
      \begin{longtable}{|m{0.2\textwidth}|>{\centering\arraybackslash}m{0.3\textwidth}|m{0.15\textwidth}|>{\centering\arraybackslash}m{0.325\textwidth}|}%
        \hline
        \rowcolor{gray!20}\multicolumn{1}{|c|}{\bf Nom de la figure}&{\bf Représentation}&\multicolumn{1}{|c|}{\bf Aire}&{\bf Exemple}\\
        \hline
        {\textbf{ Rectangle}\par de longueur $L$\par et de largeur $l$}&
        \begin{Geometrie}[CoinHD={u*(4,3)}]
          trace feuillet withcolor white;
          pair A,B,C,D;
          A=u*(0.5,0.25);
          B=u*(3.75,0.25);
          C=3/5[B,rotation(A,B,-90)];
          D-A=C-B;
          trace A--B--C--D--cycle;
          trace cotationmil(D,C,3mm,5,btex $L$ etex) withcolor red;
          trace cotationmil(A,D,3mm,5,btex $l$ etex) withcolor red;
          trace codeperp(A,B,C,5);
          trace codeperp(B,C,D,5);
          trace codeperp(C,D,A,5);
        \end{Geometrie}  
        &$\Eqalign{
        {\mathcal A}&=L\times l\cr
        }$
        &
        \begin{Geometrie}[CoinHD={u*(4,3)}]
          trace feuillet withcolor white;
          pair A,B,C,D;
          A=u*(0.5,0.25);
          B=u*(3.75,0.25);
          C=3/5[B,rotation(A,B,-90)];
          D-A=C-B;
          trace A--B--C--D--cycle;
          trace cotationmil(D,C,3mm,20,btex \Lg[dm]{0.3} etex) withcolor red;
          trace cotationmil(A,D,3mm,20,btex \Lg[dm]{0.2} etex) withcolor red;
          trace codeperp(A,B,C,5);
          trace codeperp(B,C,D,5);
          trace codeperp(C,D,A,5);
        \end{Geometrie}  
        $\Eqalign{
        {\mathcal A}&=\Lg[dm]{0.3}\times\Lg[dm]{0.2}\cr
        {\mathcal A}&=\Aire[dm]{0.06}\cr
        }$\\
        \hline
        \textbf{ Carré} de côté $c$&
        \begin{Geometrie}[CoinHD={u*(3,3)}]
          trace feuillet withcolor white;
          pair A,B,C,D;
          A=u*(0.25,0.25);
          B=u*(2.5,0.25);
          C=rotation(A,B,-90);
          D-A=C-B;
          trace A--B--C--D--cycle;
          marque_s:=2;
          trace cotationmil(D,C,3mm,5,btex $c$ etex) withcolor red;
          trace codeperp(A,B,C,5);
          trace codeperp(B,C,D,5);
          trace codeperp(C,D,A,5);
          trace codesegments(A,B,B,C,2);
          trace codesegments(C,D,D,A,2);
        \end{Geometrie} 
        &$\Eqalign{
        {\mathcal A}&=c\times c\cr
        {\mathcal A}&=c^2\cr
        }$&
        \begin{Geometrie}[CoinHD={u*(3,3)}]
          trace feuillet withcolor white;
          pair A,B,C,D;
          A=u*(0.25,0.25);
          B=u*(2.5,0.25);
          C=rotation(A,B,-90);
          D-A=C-B;
          trace A--B--C--D--cycle;
          marque_s:=2;
          trace cotationmil(D,C,3mm,15,btex \Lg{5} etex) withcolor red;
          trace codeperp(A,B,C,5);
          trace codeperp(B,C,D,5);
          trace codeperp(C,D,A,5);
          trace codesegments(A,B,B,C,2);
          trace codesegments(C,D,D,A,2);
        \end{Geometrie}
        $\Eqalign{
          {\mathcal A}&=\Lg{5}\times \Lg{5}\cr
          {\mathcal A}&=(\Lg{5})^2\cr
          {\mathcal A}&=\Aire{25}\cr
        }$\\
        \hline
        \multicolumn{4}{|c|}{\emoji{light-bulb} Les aires des quadriltères suivants peuvent être obtenues en les complétant ou en les transformant en rectangles.\emoji{light-bulb}}
        \\\hline
        {\textbf{ Losange}\par de grande diagonale $D$\par et de petite diagonale $d$}&
        \hspace*{-3mm}        
        \begin{Geometrie}[CoinBG={u*(0,-0.75)},CoinHD={u*(5,2.5)}]
            u:=0.5*u;
            trace feuillet withcolor white;
            pair A,B,C,D,H;
            A=u*(0.75,0.75);
            B=u*(5.25,0.75);
            C=rotation(A,B,-140);
            D=symetrie(B,A,C);
            H=iso(A,C);
            trace A--B--C--D--cycle;
            trace cotationmil(A,C,-10mm,5,btex $D$ etex) withcolor red;
            trace cotationmil(D,B,-5mm,5,btex $d$ etex) withcolor red;
            trace codeperp(C,H,B,5) dashed evenly withcolor Grey;
            trace A--C dashed evenly withcolor LightGrey;
            trace B--D dashed evenly withcolor LightGrey;            
        \end{Geometrie}        
        &$\Eqalign{
        {\mathcal A}&=\dfrac{d\times D}{2}\cr
        }$&
        \hspace*{-3mm}        
        \begin{Geometrie}[CoinBG={u*(0,-0.75)},CoinHD={u*(5,2.5)}]
            u:=0.5*u;
            trace feuillet withcolor white;
            pair A,B,C,D,H;
            A=u*(0.75,0.75);
            B=u*(5.25,0.75);
            C=rotation(A,B,-140);
            D=symetrie(B,A,C);
            H=iso(A,C);
            trace A--B--C--D--cycle;
            trace cotationmil(A,C,-10mm,15,btex $\Lg[m]{0.04}$ etex) withcolor red;
            trace cotationmil(D,B,-5mm,15,btex $\Lg{3}$ etex) withcolor red;
            trace codeperp(C,H,B,5) dashed evenly withcolor Grey;
            trace A--C dashed evenly withcolor LightGrey;
            trace B--D dashed evenly withcolor LightGrey;            
        \end{Geometrie}        
        $\Eqalign{
          {\mathcal A}&=\dfrac{\Lg{3}\times \Lg[m]{0.04}}{2}\cr
          {\mathcal A}&=\dfrac{\Lg{3}\times \Lg{4}}{2}\cr
          {\mathcal A}&=\Aire{6}\cr
        }$
        \\
        \hline
        {\textbf{ Trapèze}\par de petite base $b$,\par de grande base $B$\par et de hauteur $h$}&
        \begin{Geometrie}[CoinHD={u*(4,2.75)}]
            u:=0.5*u;
            trace feuillet withcolor white;
            pair A,B,C,D,H;
            A=u*(0.75,0.75);
            B=u*(7.25,0.75);
            C=3/5[B,rotation(A,B,-60)];
            D-C=u*(-3,0);
            H=projection(D,A,B);
            trace A--B--C--D--cycle;
            trace cotationmil(D,C,3mm,5,btex $b$ etex) withcolor red;
            trace cotationmil(A,B,-2mm,5,btex $B$ etex) withcolor red;
            trace cotationmil(D,H,4mm,5,btex $h$ etex) withcolor red;
            trace D--H dashed evenly;
            trace codeperp(C,D,H,5);
            trace codeperp(D,H,B,5);
        \end{Geometrie}        
        &$\Eqalign{
        {\mathcal A}&=\dfrac{(b+B)\times h}{2}\cr
        }$&
        \begin{Geometrie}[CoinHD={u*(4,2.75)}]
          u:=0.5*u;
          trace feuillet withcolor white;
          pair A,B,C,D,H;
          A=u*(0.75,0.75);
          B=u*(7.25,0.75);
          C=3/5[B,rotation(A,B,-60)];
          D-C=u*(-3,0);
          H=projection(D,A,B);
          trace A--B--C--D--cycle;
          trace cotationmil(D,C,3mm,15,btex $\Lg[dm]{0.3}$ etex) withcolor red;
          trace cotationmil(A,B,-2mm,15,btex $\Lg[m]{0.04}$ etex) withcolor red;
          trace cotationmil(D,H,4mm,15,btex $\Lg{2}$ etex) withcolor red;
          trace D--H dashed evenly;
          trace codeperp(C,D,H,5);
          trace codeperp(D,H,B,5);
        \end{Geometrie}
        \vspace*{5mm}
        $\Eqalign{
          {\mathcal A}&=\dfrac{(\Lg[dm]{0.3}+\Lg[m]{0.04})\times \Lg{2}}{2}\cr
          {\mathcal A}&=\dfrac{(\Lg{3}+\Lg{4})\times \Lg{2}}{2}\cr
          {\mathcal A}&=\Aire{12}\cr
        }$
        \\
        \hline
        {\textbf{ Parallélogramme}\par de côté $c$ et de hauteur\par relative à ce côté $h$}&
        \begin{Geometrie}[CoinBG={u*(0,-0.25)},CoinHD={u*(4,2.5)}]
          u:=0.5*u;
          trace feuillet withcolor white;
          pair A,B,C,D,H;
          A=u*(0.75,0.75);
          B=u*(7.25,0.75);
          C=0.5[B,rotation(A,B,-100)];
          D-C=A-B;
          H=projection(D,A,B);
          trace A--B--C--D--cycle;
          trace cotationmil(A,B,-2mm,5,btex $c$ etex) withcolor red;
          trace cotationmil(D,H,4mm,5,btex $h$ etex)withcolor red;
          trace D--H dashed evenly;
          trace codeperp(C,D,H,5);
          trace codeperp(D,H,B,5);
        \end{Geometrie}        
        &$\Eqalign{
        {\mathcal A}&=c\times h\cr
        }$&
        \begin{Geometrie}[CoinBG={u*(0,-0.25)},CoinHD={u*(4,2.5)}]
          u:=0.5*u;
          trace feuillet withcolor white;
          pair A,B,C,D,H;
          A=u*(0.75,0.75);
          B=u*(7.25,0.75);
          C=0.5[B,rotation(A,B,-100)];
          D-C=A-B;
          H=projection(D,A,B);
          trace A--B--C--D--cycle;
          trace cotationmil(A,B,-2mm,15,btex $\Lg[m]{30}$ etex) withcolor red;
          trace cotationmil(D,H,4mm,15,btex $\Lg{40}$ etex)withcolor red;
          trace D--H dashed evenly;
          trace codeperp(C,D,H,5);
          trace codeperp(D,H,B,5);
        \end{Geometrie}         
        $\Eqalign{
          {\mathcal A}&=\Lg[m]{30}\times \Lg{40}\cr
          {\mathcal A}&=\Lg[m]{30}\times \Lg[m]{0.4}\cr
          {\mathcal A}&=\Aire[m]{12}\cr
        }$
        \\
        \hline
      \end{longtable}
}

\subsection{Disques}
{\renewcommand*{\arraystretch}{1.5}
      \begin{longtable}{|m{0.25\textwidth}|>{\centering\arraybackslash}m{0.25\textwidth}|m{0.15\textwidth}|>{\centering\arraybackslash}m{0.325\textwidth}|}%
        \hline
        \rowcolor{gray!20}\multicolumn{1}{|c|}{\bf Nom de la figure}&{\bf Représentation}&\multicolumn{1}{|c|}{\bf Aire}&{\bf Exemple}\\
        \hline
        \textbf{ Disque} de rayon $R$&        
        \begin{Geometrie}[CoinBG={u*(0,-0.5)},CoinHD={u*(4,3.5)}]
          u:=0.5*u;
          trace feuillet withcolor white;
          pair A,B;
          A=u*(4,3);
          path cc;
          cc=cercles(A,u*3.5);
          B=point(0.15*length cc) of cc;
          trace cc;
          marque_p:="croix";
          MarquePoint(A);
          trace cotationmil(A,B,0,5,btex $R$ etex) withcolor red;
        \end{Geometrie}        
        &$\Eqalign{
        {\mathcal A}&=\pi\times R\times R\cr
        {\mathcal A}&=\pi\times R^2\cr
        }$&
        \begin{Geometrie}[CoinBG={u*(0,-0.5)},CoinHD={u*(4,3.5)}]
          u:=0.5*u;
          trace feuillet withcolor white;
          pair A,B;
          A=u*(4,3);
          path cc;
          cc=cercles(A,u*3.5);
          B=point(0.15*length cc) of cc;
          trace cc;
          marque_p:="croix";
          MarquePoint(A);
          trace cotationmil(A,B,0,15,btex $\Lg{1.2}$ etex) withcolor red;
        \end{Geometrie}        
        $\Eqalign{
          {\mathcal A}&=\pi\times \Lg{1.2}\times \Lg{1.2}\cr
          {\mathcal A}&=\pi\times (\Lg{1.2})^2\cr
          {\mathcal A}&\simeq\Aire{4.52} \cr
        }$
        \\
        \hline
      \end{longtable}
}

\begin{changemargin}{0mm}{-5mm}
  \begin{myBox}{\emoji{light-bulb} Moyen mnémotechnique \emoji{light-bulb}}  
    Si on note $R$ le rayon du cercle, que l'on simplifie l'expression littérale en réordonnant ses facteurs :
    $$\pi\times R \times R = \pi \times R^2 = \pi R^2$$
    L'expression $\pi R^2$ se lit alors \og{} Pierre carrée \fg{}.  
  \end{myBox}
\end{changemargin}
