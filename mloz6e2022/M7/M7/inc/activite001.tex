\begin{changemargin}{-10mm}{-10mm}
   \vspace*{-20mm}
   \begin{activite}[Aire d'un triangle]    
      \vspace*{-7mm}
      {\bf Objectifs :} 
      \begin{itemize}
         \item calculer l'aire d'un rectangle, d'un triangle ;
         \item déterminer la formule de l'aire d'un rectangle, d'un triangle.
      \end{itemize}
         \partie[aire du rectangle]
         \vspace*{-7mm}
            \begin{enumerate}
               \item Découper le rectangle 1 ci-dessous, puis le paver de carrés de côté \Lg[cm]{1}.
                  \begin{center}
                     {\psset{unit=0.5}
                     \begin{pspicture}(0,-0.3)(5,3.5)
                        \psframe(0,0)(5,3)
                        \psgrid[subgriddiv=0,gridlabels=0pt](0,0)(1,3)
                        \psline(3,0)(3,1)(1,1)
                        \psline(2,0)(2,2)(1,2)
                     \end{pspicture}}
                  \end{center}
               \item Combien de carrés de \ucm{1} de côté y a-t-il dans ce rectangle ? \dotfill \medskip
               \item Quelle est l'aire du rectangle ? \dotfill \medskip
               \item Rappeler la formule de l'aire d'un rectangle de longueur $L$ et de largeur $\ell$ : \dotfill
            \end{enumerate}
            
         \partie[aire du triangle rectangle]
         \vspace*{-7mm}
            \begin{enumerate}
               \setcounter{enumi}{4}
               \item Découper le rectangle 2 puis tracer l'une de ses diagonales. \medskip
               \item Découper le rectangle suivant la diagonale, que peut-on dire des deux triangles obtenus ?
               
               \medskip\dotfill
               \item Comment peut-on obtenir l'aire d'un triangle en fonction de l'aire du rectangle ?
               
               \medskip\dotfill
               \item En déduire la formule de l'aire d'un triangle rectangle de base $b$ et de hauteur $h$ : \dotfill
            \end{enumerate}
            
         \partie[aire du triangle quelconque]
         \vspace*{-7mm}
            \begin{enumerate}
               \setcounter{enumi}{8}
               \item Découper le rectangle 3 puis placer une point A sur l'un de ses côtés. \medskip
               \item Tracer le triangle de sommet A et de base le côté opposé à A, le découper. \medskip
               \item Est-il possible de coller les deux morceaux restants sur le triangle à la manière d'un puzzle ? \dotfill
               
               \medskip
               \item En déduire la formule de l'aire d'un triangle de base $b$ et de hauteur $h$. \dotfill
            \end{enumerate}
      % \begin{center}
         \vspace*{-5mm}
         \hspace*{-5mm}
         \begin{pspicture}(0,-0.5)(5.5,3)
            \psframe(0,0)(5,3)
            \rput(2.5,-0.5){rectangle 1}
         \end{pspicture}
         \begin{pspicture}(0,-0.5)(5.5,4)
            \psframe(0,0)(5,3)
            \rput(2.5,-0.5){rectangle 2}
         \end{pspicture}
         \begin{pspicture}(0,-0.5)(5,4)
            \psframe(0,0)(5,3)
            \rput(2.5,-0.5){rectangle 3}
         \end{pspicture}
         % \end{center} 
   \end{activite}
\end{changemargin} 