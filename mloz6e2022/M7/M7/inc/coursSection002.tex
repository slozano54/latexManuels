\section{Unités d'aire}
\begin{changemargin}{0mm}{-20mm}
    \begin{definition}
        L'unité d'aire usuelle est le \textbf{mètre carré}, \Aire[m]{}, qui représente l'aire d'un carré de côté \Lg[m]{1}.\\
        On utilise aussi ses \textbf{multiples} et ses \textbf{sous-multiples}.\\
            \begin{center}
                \textbf{Tableau de référence}\par\vspace{5mm}
                \begin{tabular}{|>{\centering\arraybackslash}m{0.09\linewidth}||>{\centering\arraybackslash}m{0.09\linewidth}|>{\centering\arraybackslash}m{0.09\linewidth}|>{\centering\arraybackslash}m{0.09\linewidth}||>{\centering\arraybackslash}m{0.09\linewidth}||>{\centering\arraybackslash}m{0.09\linewidth}|>{\centering\arraybackslash}m{0.09\linewidth}|>{\centering\arraybackslash}m{0.09\linewidth}|}
                    \hline
                                        &\multicolumn{3}{c||}{\textbf{Multiples de l'unité}} & \multicolumn{1}{c||}{\textbf{Unité}} & \multicolumn{3}{c|}{\textbf{Sous-multiples de l'unité}} \\\hline
                    Unités d'aire       &\Aire[km]{}   & \Aire[hm]{}  & \Aire[dam]{} & \Aire[m]{}   & \Aire[dm]{}   & \Aire[cm]{}  & \Aire[mm]{}  \\\hline
                    Unités agraires     &$\bigotimes$  & \Aire[ha]{1} & \Aire[a]{1}  & \Aire[ca]{1} & $\bigotimes$  & $\bigotimes$ & $\bigotimes$ \\\hline
                    Valeur en \Aire[m]{}&$\bigotimes$&\Aire[m]{10000}&\Aire[m]{100}& \Aire[m]{1}&\Aire[m]{0.01}&$\bigotimes$&$\bigotimes$ \\ \hline
                    \rule[-2ex]{0pt}{6ex}  &$\bigotimes$& $\dfrac{1}{100}$ \Aire[km]{}    & $\dfrac{1}{100}$ \Aire[hm]{}   & $\dfrac{1}{100}$ \Aire[dam]{} & $\dfrac{1}{100}$ \Aire[m]{} & $\dfrac{1}{100}$ \Aire[dm]{}    & $\dfrac{1}{100}$ \Aire[cm]{} \\\hline
                \end{tabular}
            \end{center}
    \end{definition}
\end{changemargin}

\vspace*{-20mm}
\begin{changemargin}{0mm}{-20mm}
\begin{methode}[Convertir des aires]
    Pour convertir des longueurs, on effectue des multiplications ou des divisions par 100,\num{10 000} \dots \\
    Pour la surface d'un terrain, de terres agricoles ou forestières, on utilise des \textbf{unités de mesure agraires}.
    \exercice
        Convertir 
        \begin{itemize}
            \item \Aire[dam]{1} en \Aire[m]{}.
            \item \Aire[dm]{1} en \Aire[m]{}.
            \item \Aire[ha]{1} en \Aire[m]{}.
            \item \Aire[a]{1} en \Aire[m]{}.
            \item \Aire[ha]{1} en \Aire[a]{}.
        \end{itemize}
    \correction
        \begin{itemize}
            \item $\Aire[dam]{1} = \Aire[m]{100}$   donc $\Aire[dam]{53}  = 53 \times \Aire[m]{100}           = \Aire[m]{5300}$.
            \item $\Aire[dm]{1}  = \Aire[m]{0.01}$  donc $\Aire[dm]{5}    = 5 \div \Aire[m]{100}              = \Aire[m]{0.05}$.
            \item $\Aire[ha]{1}  = \Aire[m]{10000}$ donc $\Aire[ha]{7.81} = \num{7.81} \times \Aire[m]{10000} = \Aire[m]{78100}$.
            \item $\Aire[a]{1}   = \Aire[m]{100}$   donc $\Aire[a]{8.5}   = \num{8.5} \times \Aire[m]{100}    = \Aire[m]{850}$.
            \item $\Aire[ha]{1}  = \Aire[a]{100}$   donc $\Aire[ha]{17}   = 17 \times \Aire[a]{100}           = \Aire[a]{17000}$.
        \end{itemize}
\end{methode}
\end{changemargin}
