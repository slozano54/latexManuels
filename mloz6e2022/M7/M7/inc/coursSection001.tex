\section{Définitions}
\dots
% \subsection{Aires}
% \definNum{L'\textbf{aire} d'une figure c'est la mesure de sa surface.}

% \Remarques[Remarque]{Une aire s'exprime avec une \textbf{unité d'aire}.}

% \subsection{Quelques remarques}
% \proprNumBis{}{
% \begin{mylist}
% \item Deux figures non superposables peuvent avoir le \textbf{même périmètre}.
% \item Deux figures non superposables peuvent avoir la \textbf{même aire}.
% \item Deux figures peuvent avoir la même aire mais des \textbf{périmètres différents}.
% \item Deux figures peuvent avoir le même périmètre mais des \textbf{aires différentes}.
% \end{mylist}
% }

% \Exemples{}{
% \begin{minipage}{7cm}
% \definecolor{qqqqff}{rgb}{0,0,1}
% \definecolor{qqwuqq}{rgb}{0,0.39215686274509803,0}
% \definecolor{zzttqq}{rgb}{0.6,0.2,0}
% \definecolor{ccqqqq}{rgb}{0.8,0,0}
% \definecolor{ffqqqq}{rgb}{1,0,0}
% \definecolor{cqcqcq}{rgb}{0.7529411764705882,0.7529411764705882,0.7529411764705882}
% \scalebox{0.6}{
% \begin{tikzpicture}[line cap=round,line join=round,>=triangle 45,x=1cm,y=1cm]
% \draw [color=cqcqcq,, xstep=1cm,ystep=1cm] (-8,-10) grid (4,1);
% \clip(-8,-10) rectangle (4,1);
% \fill[line width=2pt,color=zzttqq,fill=zzttqq,fill opacity=0.10000000149011612] (-2,-8) -- (-1,-8) -- (-1,-7) -- (-2,-7) -- cycle;
% \fill[line width=2pt,color=zzttqq,fill=zzttqq,fill opacity=0.10000000149011612] (-7,-6) -- (-7,0) -- (-4,0) -- (-4,-1) -- (-6,-1) -- (-6,-6) -- cycle;
% \fill[line width=2pt,color=qqwuqq,fill=qqwuqq,fill opacity=0.1] (-4,-6) -- (-2,-6) -- (-2,-2) -- (-4,-2) -- cycle;
% \fill[line width=2pt,color=qqqqff,fill=qqqqff,fill opacity=0.1] (-1,-6) -- (-1,-1) -- (0,-1) -- (0,-3) -- (1,-3) -- (1,-4) -- (2,-4) -- (2,-5) -- (3,-5) -- (3,-6) -- cycle;
% \draw [line width=2pt,color=ffqqqq] (-7,-8)-- (-6,-8);
% \draw [color=ccqqqq](-7.84,-7.8) node[anchor=north west] {\parbox{3.72 cm}{\begin{center}unité de longueur \par u.l. \end{center}}};
% \draw [line width=2pt,color=zzttqq] (-2,-8)-- (-1,-8);
% \draw [line width=2pt,color=zzttqq] (-1,-8)-- (-1,-7);
% \draw [line width=2pt,color=zzttqq] (-1,-7)-- (-2,-7);
% \draw [line width=2pt,color=zzttqq] (-2,-7)-- (-2,-8);
% \draw [color=zzttqq](-2.36,-7.8) node[anchor=north west] {\parbox{2.92 cm}{\begin{center}unité d'aire \par u.a. \end{center}}};
% \draw [line width=2pt,color=zzttqq] (-7,-6)-- (-7,0);
% \draw [line width=2pt,color=zzttqq] (-7,0)-- (-4,0);
% \draw [line width=2pt,color=zzttqq] (-4,0)-- (-4,-1);
% \draw [line width=2pt,color=zzttqq] (-4,-1)-- (-6,-1);
% \draw [line width=2pt,color=zzttqq] (-6,-1)-- (-6,-6);
% \draw [line width=2pt,color=zzttqq] (-6,-6)-- (-7,-6);
% \draw [line width=2pt,color=qqwuqq] (-4,-6)-- (-2,-6);
% \draw [line width=2pt,color=qqwuqq] (-2,-6)-- (-2,-2);
% \draw [line width=2pt,color=qqwuqq] (-2,-2)-- (-4,-2);
% \draw [line width=2pt,color=qqwuqq] (-4,-2)-- (-4,-6);
% \draw [line width=2pt,color=qqqqff] (-1,-6)-- (-1,-1);
% \draw [line width=2pt,color=qqqqff] (-1,-1)-- (0,-1);
% \draw [line width=2pt,color=qqqqff] (0,-1)-- (0,-3);
% \draw [line width=2pt,color=qqqqff] (0,-3)-- (1,-3);
% \draw [line width=2pt,color=qqqqff] (1,-3)-- (1,-4);
% \draw [line width=2pt,color=qqqqff] (1,-4)-- (2,-4);
% \draw [line width=2pt,color=qqqqff] (2,-4)-- (2,-5);
% \draw [line width=2pt,color=qqqqff] (2,-5)-- (3,-5);
% \draw [line width=2pt,color=qqqqff] (3,-5)-- (3,-6);
% \draw [line width=2pt,color=qqqqff] (3,-6)-- (-1,-6);
% \begin{Large}
% \draw[color=zzttqq] (-5.58,-0.52) node {$Fig1$};
% \draw[color=qqwuqq] (-3.08,-3.42) node {$Fig2$};
% \draw[color=qqqqff] (0.22,-4.4) node {$Fig3$};
% \end{Large}
% \end{tikzpicture}
% }
% \end{minipage}
% \begin{minipage}{9.5cm}
% \begin{mylist}
% \item Fig1 et Fig2 ont la \textbf{même aire} mais pas le même périmètre.
% \item Fig1 et Fig3 ont le \textbf{même périmètre} mais pas la même aire.
% \end{mylist}
% \end{minipage}
% \par\vspace{5mm}
% \begin{tabular}{|>{\centering}p{3cm}|>{\centering}p{3cm}|>{\centering}p{3cm}|>{\centering}p{3cm}|}
% \cline{2-4}
% \multicolumn{1}{c|}{}&\cellcolor{brown!30}Fig1&\cellcolor{green!30}Fig2&\cellcolor{blue!30}Fig3 \cr \hline
% Périmètre& 18 u.l. & 12 u.l. & 18 u.l. \cr \hline
% Aire& 8 u.a.&8 u.a.&11 u.a. \cr \hline 
% \end{tabular}
% }