%pre-001
\begin{prerequis}[Connaisances \emoji{red-heart} et compétences \emoji{diamond-suit} du cycle 3]    
   \begin{itemize}        
       \item[\emoji{red-heart}] Vocabulaire associé à ces objets et à leurs propriétés : côté, sommet, angle, hauteur.
       \columnbreak
       \item[\emoji{diamond-suit}] Reconnaître, nommer, décrire des triangles, dont les triangles particuliers (triangle rectangle, triangle isocèle, triangle équilatéral).       
   \end{itemize}
\end{prerequis}

\vfill

\begin{debat}[Débat : le GPS] 
   GPS est l'acronyme de l'anglais \og Global Positioning System \fg{}, qui signifie \og système de positionnement global\fg. Le système comprend au moins vingt-quatre satellites émetteurs, et un nombre quasi-illimité de récepteurs. Le récepteur GPS calcule les informations émises par quatre satellites au minimum, pour obtenir les coordonnées en longitude, latitude, et altitude du point où il se trouve. Il permet de situer de manière assez précise un objet ou un personnage sur la Terre. Actuellement, la plupart des téléphones portables possèdent un GPS.
   \flushright{\footnotesize\it Source : Wikidia, l'encyclopédie des 8-13 ans}
   \begin{center}
      \begin{pspicture}(0,-0.3)(5,4.2)
         \psset{linewidth=1mm}
         \pscircle[linecolor=A1](2,2){1}
         \pscircle[linecolor=B1](3.7,1.5){1}
         \pscircle[linecolor=J1](3,3){1}
         \psline[linewidth=0.4mm,arrowsize=0.5]{->}(0,3)(2.95,2.05)
         \rput(-1,3){Je suis là !}
      \end{pspicture}
   \end{center}
   \begin{cadre}[B2][F4]
      \begin{center}
         Vidéo : \href{https://www.yout-ube.com/watch?v=WoqpQbWdacQ}{\bf Comment fonctionne un GPS}, chaîne YouTube {\it Unisciel}, série {\it Kezako}.
      \end{center}
   \end{cadre}
\end{debat}