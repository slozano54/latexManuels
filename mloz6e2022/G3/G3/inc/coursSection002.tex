\section{Se repérer sur un plan ou une carte}

\begin{definition}
   \begin{minipage}{8cm}
      Une {\bf rose des vents} est une figure qui indique les 4 points cardinaux : est, nord, ouest et sud et éventuellement les orientations intermédiaires.
   \end{minipage}
   \hfill
   \begin{minipage}{4cm}
      \psset{unit=0.5}
      \footnotesize
      \begin{pspicture}(-3,-4)(4,4)
         \pspolygon[fillstyle=solid,fillcolor=white](2.5;45)(0,1)(2.5;135)(-1,0)(2.5;-135)(0,-1)(2.5;-45)(1,0)
         \pspolygon[fillstyle=solid,fillcolor=white](0,-3)(1;-45)(3,0)(1;45)(0,3)(1;135)(-3,0)(1;-135)
         \psline(-3,0)(3,0)
         \psline(0,-3)(0,3)
         \psline(2.5;45)(2.5;-135)
         \psline(2.5;135)(2.5;-45)
         \rput(3.5;0){E}
         \rput(3.4;45){NE}
         \rput(3.5;90){N}
         \rput(3.4;135){NO}
         \rput(3.5;180){O}
         \rput(3.4;225){SO}
         \rput(3.5;270){S}
         \rput(3.4;315){SE}
      \end{pspicture}
   \end{minipage}
\end{definition}

\begin{center}
   \begin{minipage}{0.45\linewidth}
      \includegraphics[scale=0.5]{\currentpath/images/plan_college.eps}
   \end{minipage}
   \qquad
   \begin{minipage}{0.45\linewidth}
      Sur une carte, on peut se repérer grâce à la rose des vents, mais sur les cartes actuelles et les GPS, bien souvent, seule la direction du nord est indiquée. Elle permet à elle seule de se repérer dans le plan.
   \end{minipage}
\end{center}