\begin{changemargin}{-15mm}{-15mm}
\section{Construction de droites perpendiculaires et parallèles}

Pour construire des droites perpendiculaires et parallèles à une droite passant par un point, on peut utiliser le quadrillage de sa feuille ou une règle et une équerre.

\begin{methode}[Construction d'une perpendiculaire avec une équerre]
   Pour construire la perpendiculaire $(d)$ à une droite $(D)$ passant par un point $A$, on place la règle le long de la droite $(D)$, puis on place un bord de l'équerre (contenant l'angle droit) contre la règle avec l'angle droit au niveau du point $A$ et on trace la droite $(d)$.
   \exercice
      \begin{pspicture}(0,-0.5)(3.5,2.6)
         \psline(0,0)(4,1)
         \psdot[dotstyle=+](1,0.25)
         \rput(0.7,0.45){$A$}
         \rput(3.6,1.2){$(D)$}
      \end{pspicture} 
   \correction
      \qquad
      \begin{pspicture}(0,-0.5)(3.5,2.6)
         \psline(0,0)(4,1)
         \psdot[dotstyle=+](1,0.25)
         \rput(0.7,0.45){$A$}
         \equerre{1}{0.25}{14}{1}
         \regle{0}{-0.29}{14}{1}
         \rput(3.6,1.2){$(D)$}
      \end{pspicture}      
      \qquad
      \begin{pspicture}(0,-0.5)(3.5,2.6)
         \psline(0,0)(4,1)
         \psdot[dotstyle=+](1,0.25)
         \rput(0.7,0.45){$A$}
         \psline[linecolor=A1,linewidth=0.05](1.135,-0.3)(0.34,2.9)
         \equerre{1.02}{0.28}{14}{1}
         \rput(3.6,1.2){$(D)$}
         \rput(0.8,2.7){\textcolor{A1}{$(d)$}}
      \end{pspicture}
\end{methode}

\begin{methode}[Construction d'une parallèle à la règle et à l'équerre]
   Pour tracer la parallèle $(\Delta)$ à $(D)$ passant par $B$, on trace la perpendiculaire $(d)$ à $(D)$ passant par $B$, puis on trace la perpendiculaire $(\Delta)$ à $(d)$ passant par $B$.
   \exercice
      \begin{pspicture}(-0.5,-0.4)(3.5,3)
         \psline(0,0)(4,1)
         \psdot[dotstyle=+](0.65,1.69)
         \rput(0.55,1.45){$B$}
         \rput(3.6,1.2){$(D)$}
      \end{pspicture}  
   \correction   
      \begin{pspicture}(-0.3,-0.4)(3.5,3)
         \psline(0,0)(4,1)
         \psdot[dotstyle=+](0.65,1.69)
         \rput(0.55,1.45){$B$}
         \psline[linecolor=A1,linewidth=0.05](1.135,-0.3)(0.34,2.9)
         \equerre{1.02}{0.28}{14}{1}
         \regle{0}{-0.25}{14}{1}
         \rput(3.6,1.2){$(D)$}
         \rput(0.6,2.5){\textcolor{A1}{$(d)$}}
      \end{pspicture} 
      \begin{pspicture}(-1,-0.4)(3.5,3)
         \psline(0,0)(4,1)
         \psdot[dotstyle=+](0.65,1.69)
         \rput(0.55,1.45){$B$}
         \psline[linecolor=B2,linewidth=0.05](-0.2,1.5)(3.8,2.5)
          \equerre{0.68}{1.68}{-76}{1}
         \psline[linecolor=A1,linewidth=0.05](1.135,-0.3)(0.34,2.9)
         \rput(3.6,1.2){$(D)$}
         \rput(0.8,2.5){\textcolor{A1}{$(d)$}}
         \rput(3.5,2.85){\textcolor{B1}{$(\Delta)$}}
         \regle{0.25}{2}{-76}{1}
      \end{pspicture} 
\end{methode}
   
\begin{remarque}
   deux droites perpendiculaires à une même droite sont parallèles.
\end{remarque}
\end{changemargin}
