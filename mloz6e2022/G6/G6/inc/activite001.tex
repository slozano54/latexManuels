\begin{changemargin}{-15mm}{-10mm}
\begin{activite}[La carte au trésor]
   {\bf Objectifs :} tracer une droite parallèle ; une droite perpendiculaire ; se repérer sur un plan ; suivre un programme.
      Un lutin trouve un jour un parchemin en sortant de sa maison.
Ce parchemin est en fait la carte d’un trésor caché. Voici ce qui est écrit dessus :
      \begin{center}
         \fbox{\parbox{0.85\linewidth}{
            \og {\it À partir de cet endroit, fait 550 m perpendiculairement à la Route de la baie, vers la mer. \\
            Ensuite, fait 900 m parallèlement à la route de la ville, vers le nord-ouest. \\
            Poursuis ta route parallèlement à la route de la baie en faisant 750 m vers le sud-est. \\
            Enfin, perpendiculairement à la route de la ville vers le nord-est, fait 270 m. \\
            Tu trouveras ainsi le trésor.} \fg
         }}
      \end{center}
      Où se trouve le trésor ? Faire les tracés nécessaires sur la feuille. \\ [3mm]
         \includegraphics[width=15.8cm]{\currentpath/images/tresor}
\end{activite}
\end{changemargin}