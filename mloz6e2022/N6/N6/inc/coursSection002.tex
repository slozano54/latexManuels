\section{Propriétés de la multiplication}

\begin{propriete}
   Dans un calcul, on effectue en {\bf priorité} les calculs entre parenthèses les plus intérieures, puis les multiplications et divisions, et enfin les additions et soustractions de gauche à droite.
\end{propriete}

\begin{exemple*1}
   $8\times(5\times(\underline{8-2})) =8\times(\underline{5\times6}) =8\times30 =240$.
\end{exemple*1}

\medskip

\begin{propriete}
   \begin{itemize}
      \item La multiplication est {\bf commutative} : lors du calcul d’un produit de plusieurs facteurs, on peut changer l’ordre des facteurs.
      \item La multiplication est {\bf distributive} : pour effectuer un produit en ligne, on peut décomposer un ou les deux facteurs afin de simplifier les calculs. 
   \end{itemize}
   \ \\ [-14mm]
\end{propriete}

\begin{exemple*1}
   \begin{itemize}
      \item Commutativité : $36\times2 =2\times36 =72$.
      \item Distributivité simple : $40\times23 =40\times(20+3)=40\times20+40\times3 =800+120 =920$.
   \end{itemize}
\end{exemple*1}

\begin{propriete}
   Pour obtenir un \textbf{ordre de grandeur} d'un produit, on multiplie des ordres de grandeur de chaque facteur.
\end{propriete}

\begin{exemple*1}
   Ordre de grandeur de $785, 98\times 103,89$ : un ordre de grandeur de chacun des deux facteurs est 800 et 100. Donc, l'ordre de grandeur du résultat vaut $800\times100 =80\,000$.
\end{exemple*1}