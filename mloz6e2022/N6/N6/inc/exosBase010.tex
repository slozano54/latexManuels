\begin{exercice*}
   \begin{multicols}{2}
      \begin{enumerate}
         \item $3\times(31-28)\times5$
         \item $7\times(13-2)$
         \item $8\times11\div(5+6)$
         \item $64\div(3+5)$
         \item $3\times(90-11\times8)$
         \item $(39-21)\div3$
      \end{enumerate}
   \end{multicols}
   \hrefMathalea{https://coopmaths.fr/alea/?uuid=4c10a&id=6C33&n=6&d=10&s=2&s2=false&cd=1&cols=2&v=eleve&title=Exercices&es=0211}
\end{exercice*}
\begin{corrige}
   %\setcounter{partie}{0} % Pour s'assurer que le compteur de \partie est à zéro dans les corrigés
   \begin{enumerate}
      \item A = $3\times({\color[HTML]{f15929}\boldsymbol{31-28}})\times5$ \\A = $3\times3\times5$ \\A = $45$ \\
      \item B = $7\times({\color[HTML]{f15929}\boldsymbol{13-2}})$ \\B = $7\times11$ \\B = $77$ \\
      \item C = $8\times11\div({\color[HTML]{f15929}\boldsymbol{5+6}})$ \\C = ${\color[HTML]{f15929}\boldsymbol{8\times11}}\div11$ \\C = $88\div11$ \\C = $8$ \\
      \item D = $64\div({\color[HTML]{f15929}\boldsymbol{3+5}})$ \\D = $64\div8$ \\D = $8$ \\
      \item E = $3\times(90-{\color[HTML]{f15929}\boldsymbol{11\times8}})$ \\E = $3\times({\color[HTML]{f15929}\boldsymbol{90-88}})$ \\E = $3\times2$ \\E = $6$ \\
      \item F = $({\color[HTML]{f15929}\boldsymbol{39-21}})\div3$ \\F = $18\div3$ \\F = $6$ \\
      \end{enumerate}
\end{corrige}
 