\section{Section 1}
\subsection{Sous-section 1.1}
\begin{definition}[Titre optionnel]
    Dans le cours, on utilise assez souvent des cadres du type
    définition (comme ici par exemple).    
\end{definition}
\begin{remarque}
    Ceci est une remarque utilisant une commande du paquet profcollege.

    \begin{center}
      \Stat[Tableau,CouleurTab=blue!50,Frequence,Angle,ECC]{%
        2/1,5/3,6.5/5,8/4,9/7,12.25/2,15/5
      }
    \end{center}


\end{remarque}
\begin{propriete}[Titre optionnel]
  Dans le cours, on utilise assez souvent des cadres du type
  définition, comme ici par exemple pour une propriete.
\end{propriete}
\begin{remarques}
  \begin{itemize}
    \item remarque.
    \item remarque.
  \end{itemize}
\end{remarques}

\subsection{Sous-section 1.2}
\begin{theoreme}[Titre optionnel]
  Dans le cours, on utilise assez souvent des cadres du type
  définition, comme ici par exemple pour un théorème.
\end{theoreme}
\begin{notation}
  notation
\end{notation}
\begin{notations}
  \begin{itemize}
    \item notation.
    \item notation.
  \end{itemize}
\end{notations}
\begin{preuve}
  Ceci est une preuve\par Deuxième ligne de la preuve
\end{preuve}
\begin{exemple}
  Texte de l’exemple
  \correction
  
\end{exemple}

\begin{exemple*1}
  \phantom{rrr}
  
  \Stat[Qualitatif,Graphique,Angle,Hachures]{Lundi/25,Mardi/18,Mercredi/17,Jeudi/10,Vendredi/5,Samedi/20}
  \correction
  \phantom{rrr}
  
  \Stat[Qualitatif,Graphique,SemiAngle,Rayon=4cm,AffichageDonnees,ListeCouleurs={Turquoise,Cornsilk,LightSteelBlue,MistyRose,PaleGreen,Teal,GreenYellow}]{Lundi/25,Mardi/18,Mercredi/17,Jeudi/10,Vendredi/5,Samedi/20}
\end{exemple*1}

\begin{exemple}[0.6]
  Texte de l’exemple très long sur une ligne, très très très long.
  On peut modifier la répartition horizontale  à l'aide d'un argument optionnel valant par défaut 0,4, valant ici 0,6.
  \correction
  Texte de la correction en vis à vis
\end{exemple}