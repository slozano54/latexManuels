\vspace*{-8mm}
%pre-001
\begin{prerequis}[Connaisances \emoji{red-heart} et compétences \emoji{diamond-suit} du cycle 3]    
   \begin{itemize}        
       \item[\emoji{red-heart}] Vocabulaire associé à ces objets et à leurs propriétés : côté, sommet, angle, hauteur.
       \columnbreak
       \item[\emoji{diamond-suit}] Reconnaître, nommer, décrire des triangles, dont les triangles particuliers (triangle rectangle, triangle isocèle, triangle équilatéral).       
   \end{itemize}
\end{prerequis}
\vspace*{-5mm}
\begin{debat}[degré et rapporteur]
    Le {\bf degré} (comme mesure d'angle, pas de température !) vient des babyloniens, qui comptaient en base sexagésimale (60). Il correspond à 1/360 d'un tour complet. L'origine de 360 n'est pas établie avec certitude, mais il y a très probablement un lien avec le fait qu'une année compte 365 jours, que 360 est divisible par une multitude de nombres, ou encore qu'un triangle équilatéral, figure la plus simple hormis le cercle, possède des angles de \udeg{60}. \\
    On mesure un angle avec un {\bf rapporteur}, nouvel instrument de géométrie des élèves après la règle, l'équerre et le compas !
    \begin{center} 
       \begin{Geometrie}
        pair A,B;
        A=u*(4,1);
        B=u*(8,1);
        trace rapporteurdouble(A,B,1);
       \end{Geometrie}
    \end{center}
    \bigskip
    \begin{cadre}[B2][F4]
       \begin{center}
          \hrefVideo{https://www.yout-ube.com/watch?v=hahNyuD_WfY}{\bf Les angles et leur mesure}, chaîne YouTube {\it Unisciel}, épisode de la série {\it Math.ing}.
       \end{center}
    \end{cadre}
\end{debat}