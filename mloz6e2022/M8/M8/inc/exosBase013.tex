\begin{exercice*}
    Cette figure représente un {\bfseries spidron} : elle est constitué de triangles équilatéraux et de triangles ayant deux angles égaux à \ang{30}.
    
    Reproduire cette figure {\bfseries en vraie grandeur}.

    \begin{Geometrie}[TypeTrace="MainLevee",CoinBG={(-u,-15u)},CoinHD={(15u,15u)}]
        u:=u*0.8;
        input \persopath/commandes/spidron.mp
        trace spidron(12,10);
        trace appelation(M2,M1,2mm,btex \Lg[cm]{12} etex);
        trace Codeangle(M1,M2,M3,0,btex \ang{30} etex);
        trace Codeangle(M3,M1,M2,0,btex \ang{30} etex);
        marque_s:=marque_s/3;
        trace Codelongueur(M2,M3,M3,M4,M4,M2,2);
        trace Codeangle(M3,M4,M5,0,btex \ang{30} etex);
        trace Codeangle(M5,M3,M4,0,btex \ang{30} etex);
        trace Codelongueur(M4,M5,M5,M6,M6,M4,4);
        trace Codeangle(M7,M5,M6,0,btex \ang{30} etex);
        trace Codeangle(M5,M6,M7,0,btex \ang{30} etex);
        trace Codelongueur(M7,M8,M8,M6,M6,M7,3);
    \end{Geometrie}
\end{exercice*}
