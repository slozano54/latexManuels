% Les enigmes ne sont pas numérotées par défaut donc il faut ajouter manuellement la numérotation
% si on veut mettre plusieurs enigmes
%\refstepcounter{exercice}
%\numeroteEnigme
\begin{enigme}[Coupe des quatre maisons de Poudlard]
    On a retrouvé dans un grimoire les points obtenus lors de la coupe des quatre maisons de Poudlard, par trimestre, et on souhaiterait obtenir un diagramme circulaire de chacun de ces résultats afin de mieux pouvoir les comparer.
    \partie[1\up{er} trimestre]
        \begin{minipage}{0.7\linewidth}
            Voilà les points obtenus au premier trimestre par les différentes maisons :
            \begin{center}
                {\renewcommand{\arraystretch}{1.2}
                \small
                \begin{ltableau}{\linewidth}{5}
                \hline
                Griffondor & Poufsouffle & Serdaigle & Serpentard & Total\\
                \hline
                100 & 90 & 120 & 50 & \\
                \hline
                & & & & \ang{360} \\
                \hline
                \end{ltableau}}
            \end{center}
            \begin{enumerate}
                \item Compléter le tableau avec le total des points de l'ensemble des maisons.
                \item En déduire l'angle en degrés de chaque secteur pour chaque maison.
                \item Représenter les résultats dans le diagramme circulaire ci-contre.
            \end{enumerate}
        \end{minipage}
        \hfill
        \begin{minipage}{0.2\linewidth}
            \vspace*{-10mm}
            \begin{Geometrie}[CoinHD={(4.6u,4.6u)}]                
                pair O;
                O=u*(2.3,2.3);
                draw cercles(O,2.3u);
                marque_p:="croix";
                pointe(O);            
            \end{Geometrie}
        \end{minipage}       
    \partie[2\up{e} trimestre]
        \begin{minipage}{0.7\linewidth}
            Voilà les points obtenus au deuxième trimestre par les différentes maisons :
            \begin{center}
                {\renewcommand{\arraystretch}{1.2}
                \small
                \begin{ltableau}{\linewidth}{5}
                \hline
                Griffondor & Poufsouffle & Serdaigle & Serpentard & Total\\
                \hline
                60 & 50 & 20 & 50 & \\
                \hline
                & & & & \ang{360} \\
                \hline
                \end{ltableau}}
            \end{center}
            \begin{enumerate}
                \item Compléter le tableau avec le total des points de l'ensemble des maisons.
                \item En déduire l'angle en degrés de chaque secteur pour chaque maison.
                \item Représenter les résultats dans le diagramme circulaire ci-contre.
            \end{enumerate}
        \end{minipage}
        \hfill
        \begin{minipage}{0.2\linewidth}
            \vspace*{-10mm}
            \begin{Geometrie}[CoinHD={(4.6u,4.6u)}]                
                pair O;
                O=u*(2.3,2.3);
                draw cercles(O,2.3u);
                marque_p:="croix";
                pointe(O);            
            \end{Geometrie}
        \end{minipage}       
    \partie[3\up{e} trimestre]
        \begin{minipage}{0.7\linewidth}
            Voilà les points obtenus au troisième trimestre par les différentes maisons :
            \begin{center}
                {\renewcommand{\arraystretch}{1.2}
                \small
                \begin{ltableau}{\linewidth}{5}
                \hline
                Griffondor & Poufsouffle & Serdaigle & Serpentard & Total\\
                \hline
                40 & 70 & 50 & 100 & \\
                \hline
                & & & & \ang{360} \\
                \hline
                \end{ltableau}}
            \end{center}
            \begin{enumerate}
                \item Compléter le tableau avec le total des points de l'ensemble des maisons.
                \item En déduire l'angle en degrés de chaque secteur pour chaque maison.
                \item Représenter les résultats dans le diagramme circulaire ci-contre.
            \end{enumerate}
        \end{minipage}
        \hfill
        \begin{minipage}{0.2\linewidth}
            \vspace*{-10mm}
            \begin{Geometrie}[CoinHD={(4.6u,4.6u)}]                
                pair O;
                O=u*(2.3,2.3);
                draw cercles(O,2.3u);
                marque_p:="croix";
                pointe(O);            
            \end{Geometrie}
        \end{minipage}       
    \partie[total de l'année]
        \begin{minipage}{0.7\linewidth}
            \begin{center}
                {\renewcommand{\arraystretch}{1.2}
                \small
                \begin{ltableau}{\linewidth}{5}
                \hline
                Griffondor & Poufsouffle & Serdaigle & Serpentard & Total\\
                \hline
                & & & & \\
                \hline
                & & & & \ang{360} \\
                \hline
                \end{ltableau}}
            \end{center}
            \begin{enumerate}
                \item Compléter le tableau avec le total des points sur l'année entière.
                \item En déduire l'angle en degrés de chaque secteur pour chaque maison.
                \item Représenter les résultats dans le diagramme circulaire ci-contre.
            \end{enumerate}
        \end{minipage}
        \hfill
        \begin{minipage}{0.2\linewidth}
            \vspace*{-10mm}
            \begin{Geometrie}[CoinHD={(4.6u,4.6u)}]                
                pair O;
                O=u*(2.3,2.3);
                draw cercles(O,2.3u);
                marque_p:="croix";
                pointe(O);            
            \end{Geometrie}
        \end{minipage}       
\end{enigme}  
% % Pour le corrigé, il faut décrémenter le compteur, sinon il est incrémenté deux fois
% \addtocounter{exercice}{-1}
% \begin{corrige}
%     Correction enigme de la fin de la partie cours.  
%     
% \end{corrige}