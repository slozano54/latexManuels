\begin{exercice*}
    \begin{changemargin}{-10mm}{-10mm}
    \begin{myBox}{\emoji{face-with-monocle} Culture G}
        \begin{flushleft}
            Cassiopée est une des 88 constellations du ciel, visible dans
            l'hémisphère nord. À l'opposé de la Grande Ourse par rapport à la 
            Petite Ourse, Cassiopée est très facilement reconnaissable grâce à 
            sa forme de \og W\fg.\par
            La constellation représente la reine Cassiopée dans la mythologie
            grecque, femme de Céphée et mère d'Andromède, à côté desquels elle se
            trouve.\\
            Cette constellation fait partie du groupe de constellations rattachées au mythe d'Andromède.
        \end{flushleft}
        {\tiny \creditLibre{\url{http://fr.wikipedia.org/wiki/Cassiopée_(constellation)}}}
    \end{myBox}
    \end{changemargin}

    \begin{Geometrie}
        % figure(-4u,0,10u,10u);
        pair C,S,G,R,E;
        marque_p:="creux";
        C=u*(3,1);
        S=rotation(pointarc(cercles(C,1.8u),0),C,115);
        G=1.5/1.8[S,rotation(C,S,80)];
        R=1.2/1.5[G,rotation(S,G,-116)];
        E=1.6/1.2[R,rotation(G,R,121)];
        trace chemin(C,S,G,R,E);
        pointe(C,S,G,R,E);
        marque_r:=7.5;
        marque_p:="plein";
        pointe(symetrie(cercles(C,0.9u) intersectionpoint cercles(S,2.2u),S,C));
        pointe(symetrie(cercles(C,2.4u) intersectionpoint cercles(S,1u),S,C));
        pointe(symetrie(cercles(C,4.5u) intersectionpoint cercles(S,3.1u),S,C));
        pointe(cercles(S,1.6u) intersectionpoint cercles(G,2.2u));
        pointe(symetrie(cercles(S,2.2u) intersectionpoint cercles(G,1.2u),S,C));
        pointe(symetrie(cercles(R,2.7u) intersectionpoint cercles(G,2.7u),R,G));
        pointe(symetrie(cercles(R,2.7u) intersectionpoint cercles(G,3u),R,G));
        pointe(symetrie(cercles(R,3.4u) intersectionpoint cercles(G,3.4u),R,G));
        pointe(symetrie(cercles(R,4.4u) intersectionpoint cercles(G,4.4u),R,G));
        pointe(symetrie(cercles(R,4.4u) intersectionpoint cercles(G,4.6u),R,G));
        pointe(symetrie(cercles(G,4.8u) intersectionpoint cercles(E,4.2u),E,G));
        pointe(cercles(G,3.6u) intersectionpoint cercles(E,1.2u));
        label.rt(btex Caph etex,C);
        label.ulft(btex Schedar etex,S);
        label.rt(btex Gam etex,G);
        label.lft(btex Ruchbah etex,R);
        label.rt(btex Eps etex,E);
    \end{Geometrie}
    Reproduire la figure ci-dessus, avec ces données :
    \begin{multicols}{2}
        \begin{itemize}
            \item $ER=\Lg[cm]{1.6}$;
            \item $RG=\Lg[cm]{1.2}$;
            \item $GS=\Lg[cm]{1.5}$;
            \item $SC=\Lg[cm]{1.8}$;
            \item $\widehat{ERG}=\ang{121}$
            \item $\widehat{RGS}=\ang{116}$
            \item $\widehat{GSC}=\ang{80}$
        \end{itemize}
    \end{multicols}
\end{exercice*}