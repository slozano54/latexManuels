% \newsavebox{\dangerbox}
% \newlength{\marge}\setlength{\marge}{5.5mm}
% \newlength{\margehaut}\setlength{\margehaut}{2mm}
% \newlength{\margegauche}
% \newlength{\extraline}\setlength{\extraline}{3mm}
 
% \newenvironment{Infor}[1][\linewidth-3\marge-\widthof{\Huge\Info}]{%
%   \small
%   \setlength{\margegauche}{#1}%
%   %\par
%   \begin{lrbox}{\dangerbox}
%     \begin{minipage}{\linewidth-2\marge-2\pslinewidth-\extraline}
%       \par\vspace*{\margehaut}
% }
% {%
%     \end{minipage}%
%   \end{lrbox}
%   \rput[tl](0,0){%
%     \psframebox[framesep=\marge]{%
%       \usebox{\dangerbox}%
%     }%
%     \setlength{\dimen0}{-\dp\dangerbox-\marge-0.5\pslinewidth}%
%     \rput(-0.5\pslinewidth,\dimen0){\psline(\extraline,0)(0,0)(0,-\extraline)}
%   }%
%   \rput(\marge,0){%
%     \rput(\margegauche,0){%
%       \psline[linewidth=2\pslinewidth,linecolor=white](0,0)(2.4em,0)
%     }
%   }
%   \rput(1.65\marge,0){\rput(\margegauche,0){\Huge\Info}}
%   \par
%   \setlength{\marge}{\ht\dangerbox+\dp\dangerbox+2\marge+\extraline}
%   \vspace{\marge}
% }
 
% Trace le \og W \fg\ de Cassiopée avec la figure et les données
% ci-dessous.
% \par
% \compo{1}{6anglesexo36}{1}{%
%   \fbox{
%     \begin{minipage}{200pt}
%       \textbf{Données} :\par
%       $ER=1,6$~cm;\par
%       $RG=1,2$~cm;\par
%       $GS=1,5$~cm;\par
%       $SC=1,8$~cm;\par
%       $\widehat{ERG}=121$\degres\par
%       $\widehat{RGS}=116$\degres\par
%       $\widehat{GSC}=80$\degres
%     \end{minipage}
% }}
% \par
% \vspace*{5mm}
% \begin{Infor}
%   Cassiopée est une des 88 constellations du ciel, visible dans
%   l'hémisphère nord. \`A l'opposé de la Grande Ourse par rapport à la
%   Petite Ourse, Cassiopée est très facilement reconnaissable grâce à
%   sa forme de \og W\fg.
% \par La constellation représente la reine Cassiopée dans la mythologie
% grecque, femme de Céphée et mère d'Andromède, à côté desquels elle se
% trouve.\\Cette constellation fait partie du groupe de constellations
% rattachées au mythe d'Andromède.
% \par\hfill{\small Source :
%   \url{http://fr.wikipedia.org/wiki/Cassiopée_(constellation)}}
% \end{Infor}

%%%% METAPOST
% verbatimtex
% %&latex
% \documentclass[12pt]{article}
% \begin{document}
% etex

% input geometriesyr16
% figure(-4u,0,10u,10u);
% pair C,S,G,R,E;
% marque_p:="creux";
% C=u*(3,1);
% S=rotation(pointarc(cercles(C,1.8u),0),C,115);
% G=1.5/1.8[S,rotation(C,S,80)];
% R=1.2/1.5[G,rotation(S,G,-116)];
% E=1.6/1.2[R,rotation(G,R,121)];
% trace chemin(C,S,G,R,E);
% pointe(C,S,G,R,E);
% marque_r:=7.5;
% marque_p:="plein";
% pointe(symetrie(cercles(C,0.9u) intersectionpoint cercles(S,2.2u),S,C));
% pointe(symetrie(cercles(C,2.4u) intersectionpoint cercles(S,1u),S,C));
% pointe(symetrie(cercles(C,4.5u) intersectionpoint cercles(S,3.1u),S,C));
% pointe(cercles(S,1.6u) intersectionpoint cercles(G,2.2u));
% pointe(symetrie(cercles(S,2.2u) intersectionpoint cercles(G,1.2u),S,C));
% pointe(symetrie(cercles(R,2.7u) intersectionpoint cercles(G,2.7u),R,G));
% pointe(symetrie(cercles(R,2.7u) intersectionpoint cercles(G,3u),R,G));
% pointe(symetrie(cercles(R,3.4u) intersectionpoint cercles(G,3.4u),R,G));
% pointe(symetrie(cercles(R,4.4u) intersectionpoint cercles(G,4.4u),R,G));
% pointe(symetrie(cercles(R,4.4u) intersectionpoint cercles(G,4.6u),R,G));
% pointe(symetrie(cercles(G,4.8u) intersectionpoint cercles(E,4.2u),E,G));
% pointe(cercles(G,3.6u) intersectionpoint cercles(E,1.2u));
% label.rt(btex Caph etex,C);
% label.ulft(btex Schedar etex,S);
% label.rt(btex Gam etex,G);
% label.lft(btex Ruchbah etex,R);
% label.rt(btex Eps etex,E);
% fin;
% end