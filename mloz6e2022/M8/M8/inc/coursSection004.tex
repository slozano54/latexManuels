\section{Bissectrice d'un angle}

% \definNum{
% \begin{minipage}{9cm}
% La \textbf{bissectrice} d'un angle est la demi-droite qui a pour origine le sommet de l'angle, et qui partage l'angle en deux angles de même mesure.
% \end{minipage}
% \hfill
% \begin{minipage}{6cm}
% \includegraphics[scale=0.6]{figures/bissectrice.1}
% \end{minipage}
% }

% \newcommand{\scaleBissectrice}{0.65}
% \Methode{Tracer la bissectrice avec un rapporteur}{
% \begin{mylist}
% \item On mesure l'angle à l'aide du rapporteur;
% \item puis on divise cette mesure par 2,
% \item et on trace l'angle moitié.
% \end{mylist}

% \begin{tabular}{c c}
% \begin{minipage}{8.5cm}
% \includegraphics[scale=\scaleBissectrice]{figures/rapporteur.9}
% \end{minipage}
% &
% \hspace{-2cm}
% \begin{minipage}{8.5cm}
% \includegraphics[scale=\scaleBissectrice]{figures/rapporteur.10}
% \end{minipage}
% \end{tabular}
% \par\vspace{1cm}
% }


% \Methode{Tracer la bissectrice avec un compas}{
% \begin{mylist}
% \item On trace deux arcs de cercle de centre $A$, de même rayon,
% \item On nomme $I$ et $J$ les points d'intersection avec les deux côtés de l'angle aux ;
% \item puis, en prenant pour centres ces deux points, on trace à nouveau deux arcs de même rayon que les arcs précédents,
% \item Ils se coupent en un point $D$.
% \item La bissectrice de l'angle $\widehat{BAC}$ est la demi-droite $[AD)$.
% \end{mylist} 
    
% \begin{tabular}{c c }
% \begin{minipage}{8.5cm}
% \includegraphics[scale=\scaleBissectrice]{figures/rapporteur.11}
% \end{minipage}
% &
% \begin{minipage}{8.5cm}
% \includegraphics[scale=\scaleBissectrice]{figures/rapporteur.12}
% \end{minipage}
% \\
% \multicolumn{2}{c}{
% \begin{minipage}{8.5cm}
% \includegraphics[scale=\scaleBissectrice]{figures/rapporteur.13}
% \end{minipage}}
% \\
% \end{tabular}
% }

% \newpage
% \subsection{Compléments numériques}

% \begin{bclogo}[couleur=gray!20, arrondi = 0.1,logo=\bcplume]{Animations en ligne. Source :  \href{https://instrumenpoche.sesamath.net/}{https://instrumenpoche.sesamath.net/}}

% \href{http://lozano.maths.free.fr/iep_local/figures_html/scr_iep_154.html}{\psshadowbox{Construire un gabarit d'angle}}.\par\vspace{0.25cm}
% \href{http://lozano.maths.free.fr/iep_local/figures_html/scr_iep_005.html}{\psshadowbox{Mesure d'un angle aigu au rapporteur (1)}}.\par\vspace{0.25cm}
% \href{http://lozano.maths.free.fr/iep_local/figures_html/scr_iep_008.html}{\psshadowbox{Mesure d'un angle obtu au rapporteur(1)}}.\par\vspace{0.25cm}
% \href{http://lozano.maths.free.fr/iep_local/figures_html/scr_iep_003.html}{\psshadowbox{Mesure d'un angle aigu au rapporteur (2)}}.\par\vspace{0.25cm}
% \href{http://lozano.maths.free.fr/iep_local/figures_html/scr_iep_004.html}{\psshadowbox{Mesure d'un angle aigu au rapporteur (2)}}.\par\vspace{0.25cm}
% \href{http://lozano.maths.free.fr/iep_local/figures_html/scr_iep_173.html}{\psshadowbox{Mesure d'un angle au rapporteur}}.\par\vspace{0.25cm}
% \href{http://lozano.maths.free.fr/iep_local/figures_html/scr_iep_006.html}{\psshadowbox{Mesure d'un angle aigu au rapporteur départ 0$^{\circ}$}}.\par\vspace{0.25cm}
% \href{http://lozano.maths.free.fr/iep_local/figures_html/scr_iep_007.html}{\psshadowbox{Mesure d'un angle aigu au rapporteur départ 180$^{\circ}$}}.\par\vspace{0.25cm}
% \href{http://lozano.maths.free.fr/iep_local/figures_html/scr_iep_009.html}{\psshadowbox{Mesure d'un angle obtu au rapporteur départ 0$^{\circ}$}}.\par\vspace{0.25cm}
% \href{http://lozano.maths.free.fr/iep_local/figures_html/scr_iep_010.html}{\psshadowbox{Mesure d'un angle obtu au rapporteur départ 180$^{\circ}$}}.\par\vspace{0.25cm}
% \href{http://lozano.maths.free.fr/iep_local/figures_html/scr_iep_011.html}{\psshadowbox{Bissectrice au rapporteur}}.\par\vspace{0.25cm}
% \href{http://lozano.maths.free.fr/iep_local/figures_html/scr_iep_012.html}{\psshadowbox{Bissectrice au compas (losange)}}.\par\vspace{0.25cm}
% \href{http://lozano.maths.free.fr/iep_local/figures_html/scr_iep_013.html}{\psshadowbox{Reproduire un angle au compas}}.\par\vspace{0.25cm}
% \href{http://lozano.maths.free.fr/iep_local/figures_html/scr_iep_127.html}{\psshadowbox{Vérifier un angle droit à l'équerre}}.\par\vspace{0.25cm}
% \href{http://lozano.maths.free.fr/iep_local/figures_html/scr_iep_032.html}{\psshadowbox{Bissectrice d'un triangle au compas (losange)}}.
% \end{bclogo}