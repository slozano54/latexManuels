\section{Bissectrice d'un angle}
\begin{changemargin}{0mm}{-15mm}
    \begin{definition}
        \begin{minipage}{0.6\linewidth}
        La \textbf{bissectrice} d'un angle est la demi-droite qui a pour origine le sommet de l'angle, et qui partage l'angle en deux angles de même mesure.
        \end{minipage}
        \hfill
        \begin{minipage}{0.4\linewidth}
            \begin{Geometrie}[CoinBG={(-4u,-2u)},CoinHD={(4u,4u)}]
                u:=0.3*u;
                pair A,B,C',D,C;
                B=u*(3,-1);
                A=u*(-3,0);
                C'=rotation(B,A,70);
                D=1.9[A,rotation(B,A,35)];
                C=0.7[A,C'];
                trace demidroite(A,B);
                trace demidroite(A,C);
                marque_p:="croix";
                label.bot(btex $A$ etex,A);
                pointe(A);
                label.bot(btex $B$ etex,B);
                pointe(B);
                label.ulft(btex $C$ etex,C);
                pointe(C);
                trace bissectrice(B,A,C) withcolor Crimson dashed dashpattern(on12bp off6bp on3bp off6bp);
                trace marqueangle(B,A,D,2);
                marque_a:=0.6*marque_a;
                trace marqueangle(D,A,C,2);
                trace appelation(A,D,3mm,btex bissectrice de $\widehat{BAC}$ etex) withcolor Crimson;
            \end{Geometrie}
        \end{minipage}
    \end{definition}

    \begin{methode}[Tracer la bissectrice avec un rapporteur]
        Pour tracer la bissectrice d'un angle à l'aide du rapporteur :
        \begin{itemize}
            \item {\bfseries Mesurer} l'angle à l'aide du rapporteur.
            \item {\bfseries Diviser} cette mesure par 2,
            \item {\bfseries Tracer} l'angle moitié.
        \end{itemize}
        \exercice
        Tracer la bissectrice de $\widehat{BAC}$ à l'aide du rapporteur. \par\vspace*{5mm}
        \begin{Geometrie}[CoinHD={(6.5u,5u)}]
            labeloffset:=labeloffset*1.5;
            trace feuillet withcolor white;
            pair A,B[],C[];
            A=u*(3.5,1);
            B0=u*(7,1);
            marque_p:="croix";
            picture figUn;
            figUn = image (
                pointe(A);
                label.bot(btex $A$ etex, A) rotatedaround(A, -10);
                B1 = B0 shifted (0.5*u,0);
                pointe(B1);
                label.bot(btex $B$ etex, B1)  rotatedaround(B1, -10);
                trace demidroite(A,B0) withpen pencircle scaled 1.5bp;
                C0 = B0 rotatedaround(A, 60);
                C1 = B1 rotatedaround(A, 60);
                pointe(C1);
                label.rt(btex $C$ etex, C1)  rotatedaround(C1, -10);            
                trace demidroite(A,C0) withpen pencircle scaled 1.5bp;
                trace marqueangle(B0,A,C0,0);
            );
            trace figUn rotatedaround(A, 10) shifted (-3u,0);
        \end{Geometrie}
        \correction
        $\widehat{BAC}$ mesure \ang{60} donc il faudra marquer un angle de \ang{30}. \par\vspace*{5mm}
        \begin{Geometrie}[CoinHD={(8.5u,5u)}]
            labeloffset:=labeloffset*1.5;
            marque_s:=marque_s/2;
            trace feuillet withcolor white;
            pair A,B[],C[];
            A=u*(3.5,1);
            B0=u*(7,1);
            marque_p:="croix";
            picture figUn;
            figUn = image (
                trace rapporteurdouble(A,B0,1) withcolor DimGrey;
                pointe(A);
                label.bot(btex $A$ etex, A) rotatedaround(A, -10);
                B1 = B0 shifted (0.5*u,0);
                pointe(B1);
                label.bot(btex $B$ etex, B1)  rotatedaround(B1, -10);
                trace demidroite(A,B0) withpen pencircle scaled 1.5bp;
                C0 = B0 rotatedaround(A, 60);
                C1 = B1 rotatedaround(A, 60);
                pointe(C1);
                label.rt(btex $C$ etex, C1)  rotatedaround(C1, -10);            
                trace demidroite(A,C0) withpen pencircle scaled 1.5bp;
                B2 = B0 shifted (0.5u,u);
                B3 = B0 shifted(-u,0);
                C3 = B3 rotatedaround(A, 60);
                C4 = B3 rotatedaround(A, 30);
                trace segment(B3 shifted (0.2u,0) rotatedaround(A, 30),B3 shifted (0.4u,0) rotatedaround(A, 30)) withpen pencircle scaled 1.5bp withcolor blue;
                C2 = C3 shifted (2.3u,0.3u);
                trace bissectrice(B0,A,C0) withcolor Crimson dashed dashpattern(on12bp off6bp on3bp off6bp);
                trace Codeangle(B0,A,C4,2,btex \ang{30} etex);
                marque_a:=1.3*marque_a;
                trace Codeangle(C4,A,C0,2,btex \ang{30} etex);
            );
            trace figUn rotatedaround(A, 10);
        \end{Geometrie}
    \end{methode}

    \begin{methode}[Tracer la bissectrice avec un compas]
        Pour tracer la bissectrice d'un angle à l'aide du compas, on peut par exemple construire un losange.
        \exercice
        Tracer la bissectrice de $\widehat{BAC}$ à l'aide du compas. \par\vspace*{5mm}
        \begin{Geometrie}[CoinHD={(6.5u,5u)}]
            labeloffset:=labeloffset*1.5;
            trace feuillet withcolor white;
            pair A,B[],C[];
            A=u*(3.5,1);
            B0=u*(7,1);
            marque_p:="croix";
            picture figUn;
            figUn = image (
                pointe(A);
                label.bot(btex $A$ etex, A) rotatedaround(A, -10);
                B1 = B0 shifted (0.5*u,0);
                pointe(B1);
                label.bot(btex $B$ etex, B1)  rotatedaround(B1, -10);
                trace demidroite(A,B0) withpen pencircle scaled 1.5bp;
                C0 = B0 rotatedaround(A, 60);
                C1 = B1 rotatedaround(A, 60);
                pointe(C1);
                label.rt(btex $C$ etex, C1)  rotatedaround(C1, -10);            
                trace demidroite(A,C0) withpen pencircle scaled 1.5bp;
                trace marqueangle(B0,A,C0,0);
            );
            trace figUn rotatedaround(A, 10) shifted (-3u,0);
        \end{Geometrie}
        \correction
        $A$ sera l'un des sommets du losange, deux autres sommets seront sur les côtés de $\widehat{BAC}$. Le losange est matérialisé en bleu.\par\vspace*{5mm}
        \begin{Geometrie}[CoinHD={(8.5u,5u)}]
            labeloffset:=labeloffset*1.5;
            marque_s:=marque_s/2;
            trace feuillet withcolor white;
            pair A[],B[],C[];
            A0=u*(3.5,1);
            B0=u*(7,1);
            marque_p:="croix";
            picture figUn;
            figUn = image (
                pointe(A0);
                label.bot(btex $A$ etex, A0) rotatedaround(A0, -10);
                B1 = B0 shifted (0.5*u,0);
                pointe(B1);
                label.bot(btex $B$ etex, B1)  rotatedaround(B1, -10);
                trace demidroite(A0,B0);
                C0 = B0 rotatedaround(A0, 60);
                C1 = B1 rotatedaround(A0, 60);
                pointe(C1);
                label.rt(btex $C$ etex, C1)  rotatedaround(C1, -10);            
                trace demidroite(A0,C0);
                trace bissectrice(B0,A0,C0) withcolor Crimson dashed dashpattern(on12bp off6bp on3bp off6bp);
                B2 = B0 shifted (-1.5*u,0);
                C2 = B2 rotatedaround(A0, 60);
                A1=rotation(B2,C2,60);
                trace (subpath(-1,2) of cercles(A0,B2)) dashed evenly withcolor 0.7white;
                trace coupdecompas(A0,B2,10) dashed evenly;
                trace coupdecompas(A0,C2,10) dashed evenly;
                trace coupdecompas(C2,A1,10) dashed evenly;
                trace coupdecompas(B2,A1,10) dashed evenly;
                trace polygone(A0,B2,A1,C2) withcolor blue dashed evenly withpen pencircle scaled 1.5bp;
                trace Codelongueur(A0,B2,3) withcolor blue;
                trace Codelongueur(A0,C2,3) withcolor blue;
                trace Codelongueur(A1,B2,3) withcolor blue;
                trace Codelongueur(A1,C2,3) withcolor blue;                
                trace compas(A0,C2 rotatedaround(A0,5),1);
            );
            trace figUn rotatedaround(A0, 10);
        \end{Geometrie}
    \end{methode}
\end{changemargin}
