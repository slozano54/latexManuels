\section{Qu'est-ce qu'un angle ?}
% \definNum{
% Un \textbf{angle} est une portion de plan délimitée par deux demi-droites ayant la même origine.\\
% Les deux demi-droites sont appelées \textbf{côtés} de l'angle, alors que leur origine commune est appelée \textbf{sommet} de l'angle.
% }
% \Exemples[Illustration]{}{
% \begin{center}
% \psset{xunit=0.4cm,yunit=0.4cm,algebraic=true,dotstyle=*,dotsize=3pt 0,linewidth=0.8pt,arrowsize=3pt 2,arrowinset=0.25}
% \begin{pspicture*}(-1,-2)(15,11)
% \psplot{0}{13.87}{(-0--2*x)/6}
% \psplot{0}{13.87}{(-0--7*x)/5}
% \psarc[fillcolor=lightgray](0,0){2.46}{18.43}{54.46}
% \rput[lt](6.3,9.83){\parbox{1.23 cm}{$x$}}
% \rput[tl](10.77,3.43){$y$}
% \rput[tl](2.07,2){$\alpha$}
% \rput[tl](8.97,6.63){\Rnode{U}{\psframebox{Côtés de l'angle}}}
% \rput[tl](1.43,-0.4){\Rnode{W}{\psframebox{Sommet de l'angle}}}
% \rput(9,3.2){\rnode{V}{}}
% \ncline{->}{U}{V}
% \rput(7.5,10.3){\rnode{V'}{}}
% \ncline{->}{U}{V'}
% \rput(0,-0.3){\rnode{V''}{}}
% \ncline{->}{W}{V''}
% \psdots(0,0)
% \rput[bl](-0.8,-0.3){$O$}
% \psdots(6,2)
% \rput[bl](5.93,1.3){$B$}
% \psdots(5,7)
% \rput[bl](4.5,7.3){$A$}
% \end{pspicture*}
% \end{center} 

% Cet angle peut être désigné par différentes écritures:
% \begin{mylist}
% \item On peut l'appeler $\alpha$ (\textit{lettre de l'alphabet grec, qui se prononce "alpha", équivalent de notre "a"}).
% \item On peut l'appeler $\widehat{AOB}$ ou encore $\widehat{BOA}$ (\textit{il faut seulement que la lettre désignant le sommet de l'angle soit placée au milieu}).
% \item On peut également l'appeler $\widehat{xOy}$ ou $\widehat{yOx}$.
% \end{mylist}
% }
