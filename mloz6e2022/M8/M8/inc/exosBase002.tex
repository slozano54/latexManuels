\begin{exercice*}
   Jean-Michto veut mesure l'angle coloré.
   \begin{enumerate}
       \item  
       Expliquer pourquoi il a mal placé son rapporteur.\\.\\
       \begin{Geometrie}[CoinBG={(0,-3.5u)},CoinHD={(7.5u,4u)}]
         trace feuillet;
         input \persopath/commandes/rapporteurs.mp
         pair A[],B;
         A0=u*(3.75,0.5);
         B=u*(7.75,0.5);
         Total:=false;% pour graduer de 10en 10
         CentreCroix:=true;%Pour avoir une croix au centre du rapporteur
         draw rapporteuraleph(A0,B,1);        
         A1=A0 shifted (-u,0); 
         A2=A0 shifted (-u,u);
         A3=A0 shifted (3u,0);
         draw demidroite(A3,A1) withpen pencircle scaled 1.5bp;
         draw demidroite(A3,A2) withpen pencircle scaled 1.5bp;
         draw marqueangle(A2,A3,A1,0);
         fill coloreangle(A2,A3,A1) withcolor Grey;
      \end{Geometrie}
      \par\vspace*{4mm}\dotfill
      \par\vspace*{4mm}\dotfill
      \item Expliquer comment il peut quand même obtenir la mesure de l'angle avec le rapporteur placer ainsi.\\\smallskip
      \begin{Geometrie}[CoinBG={(0,-3u)},CoinHD={(7.5u,4.5u)}]
         trace feuillet;
         input \persopath/commandes/rapporteurs.mp
         pair A[],B;
         A0=u*(3.75,1);
         B=u*(7.75,1);
         Total:=false;% pour graduer de 10en 10
         CentreCroix:=true;%Pour avoir une croix au centre du rapporteur
         draw rapporteuraleph(A0,B,1);
         A1=rotation(A0 shifted (-u,0),A0,-30); 
         A2=rotation(A0 shifted (-u,0),A0,40);           
         draw demidroite(A0,A1) withpen pencircle scaled 1.5bp;
         draw demidroite(A0,A2) withpen pencircle scaled 1.5bp;
         draw marqueangle(A1,A0,A2,0);
         fill coloreangle(A1,A0,A2) withcolor Grey;
      \end{Geometrie}
      \par\vspace*{4mm}\dotfill
      \par\vspace*{4mm}\dotfill
      \par\vspace*{4mm}\dotfill
   \end{enumerate}
\end{exercice*}
\begin{corrige}
   \begin{enumerate}
      \item Jean-Michto a mal placé son rapporteur pour faire la mesure de l'angle coloré.\\\smallskip
      Expliquer pourquoi.\\
      \begin{Geometrie}[CoinBG={(0,-3.5u)},CoinHD={(7.5u,4u)}]
          trace feuillet;
          input \persopath/commandes/rapporteurs.mp
          pair A[],B;
          A0=u*(3.75,0.5);
          B=u*(7.75,0.5);
          Total:=false;% pour graduer de 10en 10
          CentreCroix:=true;%Pour avoir une croix au centre du rapporteur
          draw rapporteuraleph(A0,B,1);        
          A1=A0 shifted (-u,0); 
          A2=A0 shifted (-u,u);
          A3=A0 shifted (3u,0);
          draw demidroite(A3,A1) withpen pencircle scaled 1.5bp;
          draw demidroite(A3,A2) withpen pencircle scaled 1.5bp;
          draw marqueangle(A2,A3,A1,0);
          fill coloreangle(A2,A3,A1) withcolor Grey;
      \end{Geometrie}
      
      {\red Le sommet de l'angle n'est pas positionné sur le centre du rapporteur.}
      \item Expliquer comment il peut quand même mesurer l'angle coloré avec le rapporteur placer ainsi.\\\smallskip
      \begin{Geometrie}[CoinBG={(0,-3u)},CoinHD={(7.5u,4.5u)}]
          trace feuillet;
          input \persopath/commandes/rapporteurs.mp
          pair A[],B;
          A0=u*(3.75,1);
          B=u*(7.75,1);
          Total:=false;% pour graduer de 10en 10
          CentreCroix:=true;%Pour avoir une croix au centre du rapporteur
          draw rapporteuraleph(A0,B,1);
          A1=rotation(A0 shifted (-u,0),A0,-30); 
          A2=rotation(A0 shifted (-u,0),A0,40);           
          draw demidroite(A0,A1) withpen pencircle scaled 1.5bp;
          draw demidroite(A0,A2) withpen pencircle scaled 1.5bp;
          draw marqueangle(A1,A0,A2,0);
          fill coloreangle(A1,A0,A2) withcolor Grey;
      \end{Geometrie}

      {\red Le sommet de l'angle est bien positionné sur le centre du rapporteur, même si aucun des côtés de l'angle n'est aligné avec
      l'une des graduations \ang{0}, il suffit de faire la différence des mesures indiquées par les côtés de l'angle.\\
      $\ang{220}-\ang{150}=\ang{70}$
      }
  \end{enumerate}
\end{corrige}
   