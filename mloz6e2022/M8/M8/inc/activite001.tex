% \vspace*{-17mm}
\begin{changemargin}{-5mm}{-20mm}
    \begin{activite}[Degré et rapporteur\dots{} bis]
        {\bf Objectifs :} découvrir le degré ; mesurer un angle donné. 
        \partie[mesure en degré]
        L'angle droit est composé de 90 petits angles comme celui-ci :
        \begin{pspicture}(-1,0)(5,0)
            \psline(5;0)(0,0)(5;1)
        \end{pspicture} \\
        cet angle sert d'unité, sa mesure est appelée un degré. 
        On dit que l'angle droit mesure 90 degrés, et on note \ang{90}. \\
        On a représenté un angle de \ang{90}, par \og pas \fg{} de \ang{5}. Écrire les graduations sur la figure.
        \begin{center}
            {\psset{unit=0.9}
            \begin{pspicture}(-1.5,-0.5)(7,7.5)
                \multido{\i=5+5}{17}{\psline[linewidth=0.001](0,0)(7;\i)}
                \psarc(0,0){6.8}{0}{90}
                \psline(7,0)(0,0)(0,7)
                \rput(7.3,0){\ang{0}}
                \rput(0,7.3){\ang{90}}
            \end{pspicture}
            }
        \end{center}
        
        \partie[mesure des angles principaux]
            \begin{enumerate}
                \item Découper un angle aigu quelconque. Comment peut-on mesurer cet angle grâce à l'angle droit gradué ci-dessus ?
                \par\vspace*{5mm}               
                \dotfill \par\vspace*{5mm}
                \dotfill 
                \item Construire un triangle équilatéral de côté \Lg[cm]{5} puis le découper. \\ [2mm]
                    En utilisant l'angle droit gradué, déterminer la mesure de ses angles : \dotfill 
                \item Plier le triangle suivant une de ses hauteurs. \\ [2mm]
                    En utilisant l'angle droit gradué, déterminer la mesure de ses angles : \dotfill 
                \item Construire un carré de côté \Lg[cm]{5} puis le découper suivant sa diagonale. \\ [2mm]
                    En utilisant l'angle droit gradué, déterminer la mesure de ses angles : \dotfill
            \end{enumerate}
            {\bf Dorénavant, on utilisera le rapporteur, un instrument permettant de mesurer un angle en degrés}.
    \end{activite}
\end{changemargin}
 