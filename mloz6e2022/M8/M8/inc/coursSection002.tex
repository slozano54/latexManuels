\section{Mesure d'un angle}

\begin{changemargin}{0mm}{-20mm}
    \begin{myvocabulaire}
        On peut mesurer \og l'ouverture \fg d'un angle;\\
        L'unité de mesure que l'on utilise au collège est le \textbf{degré}.\\
        L'instrument qui nous servira à mesurer des angles s'appelle un \textbf{rapporteur}.\\ 
        \begin{tabular}{c c}
            \begin{minipage}{0.5\linewidth}
                Voici un rapporteur, gradué en degrés; ce rapporteur a une double graduation, qui va de 0 à 180 degrés.\\
                \emoji{warning} \textbf{Attention} \emoji{warning}\\ 
                Cette double graduation est source d'erreur.
            \end{minipage}
            &
            \begin{minipage}{0.5\linewidth}
                \begin{center}
                    \begin{Geometrie}
                        pair A,B;
                        A=u*(4,1);
                        B=u*(8,1);
                        trace rapporteurdouble(A,B,1);
                    \end{Geometrie}
                \end{center}
            \end{minipage}
        \end{tabular}
    \end{myvocabulaire}

    \begin{methode*2*2}[Mesurer un angle à l'aide du rapporteur]
        Pour déterminer la mesure en degrés de l'angle $\widehat{BAC}$:
        \begin{itemize}
            \item Placer le \textbf{centre du rapporteur sur le sommet de l'angle}, ici le point $A$.
            \item Faire \textbf{pivoter le rapporteur autour de son centre} de façon à ce que l'un des côtés de l'angle passe par une des deux graduations "0" (intérieure ou extérieure), et que l'autre côté de l'angle passe sous une autre graduation du rapporteur.
            \item \textbf{En faisant bien attention à ne pas se tromper de graduation}, compter le nombre de graduations à partir du zéro pour arriver jusqu'au deuxième côté de l'angle.
        \end{itemize}
        \exercice
        Mesurer $\widehat{BAC}$. \par\vspace*{5mm}
        \begin{Geometrie}[CoinHD={(8.5u,5u)}]
            trace feuillet withcolor white;
            pair A,B[],C[];
            A=u*(3.5,1);
            B0=u*(7,1);
            marque_p:="croix";
            picture figUn;
            figUn = image (
                pointe(A);
                label.bot(btex $A$ etex, A) rotatedaround(A, -10);
                B1 = B0 shifted (0.5*u,0);
                pointe(B1);
                label.bot(btex $B$ etex, B1)  rotatedaround(B1, -10);
                trace demidroite(A,B0) withpen pencircle scaled 1.5bp;
                C0 = B0 rotatedaround(A, 60);
                C1 = B1 rotatedaround(A, 60);
                pointe(C1);
                label.rt(btex $C$ etex, C1)  rotatedaround(C1, -10);            
                trace demidroite(A,C0) withpen pencircle scaled 1.5bp;
                trace marqueangle(B0,A,C0,0);
            );
            trace figUn rotatedaround(A, 10);
        \end{Geometrie}
        \correction
        $\widehat{BAC}$ mesure \ang{60}. \par\vspace*{5mm}
        \begin{Geometrie}[CoinHD={(8.5u,5u)}]
            trace feuillet withcolor white;
            pair A,B[],C[];
            A=u*(3.5,1);
            B0=u*(7,1);
            marque_p:="croix";
            picture figUn;
            figUn = image (
                trace rapporteurdouble(A,B0,1);
                pointe(A);
                label.bot(btex $A$ etex, A) rotatedaround(A, -10);
                B1 = B0 shifted (0.5*u,0);
                pointe(B1);
                label.bot(btex $B$ etex, B1)  rotatedaround(B1, -10);
                trace demidroite(A,B0) withpen pencircle scaled 1.5bp;
                C0 = B0 rotatedaround(A, 60);
                C1 = B1 rotatedaround(A, 60);
                pointe(C1);
                label.rt(btex $C$ etex, C1)  rotatedaround(C1, -10);            
                trace demidroite(A,C0) withpen pencircle scaled 1.5bp;
                trace marqueangle(B0,A,C0,0);
                B2 = B0 shifted (0.5u,u);
                B3 = B0 shifted(-u,0);
                drawarrow B2--B3 dashed evenly withcolor red;
                label.top(btex {\red $0$ intérieur} etex, B2) rotatedaround(B2, -10);
                C3 = B3 rotatedaround(A, 60);
                C2 = C3 shifted (2.3u,0.3u);
                drawarrow C2--C3 dashed evenly withcolor blue;
                label.top(btex {\blue lecture de l'angle : \ang{60}} etex, C2) rotatedaround(C2, -10);
            );
            trace figUn rotatedaround(A, 10);
        \end{Geometrie}
        \exercice
        Mesurer $\widehat{BAC}$. \par\vspace*{5mm}
        \begin{Geometrie}[CoinHD={(8.5u,5u)}]
            trace feuillet withcolor white;
            pair A,B[],C[];
            A=u*(4.5,1);
            B0=u*(1,1);
            marque_p:="croix";
            picture figUn;
            figUn = image (
                pointe(A);
                label.bot(btex $A$ etex, A) rotatedaround(A, 15);
                B1 = B0 shifted (-0.5*u,0);
                pointe(B1);
                label.bot(btex $B$ etex, B1)  rotatedaround(B1, 15);
                trace demidroite(A,B0) withpen pencircle scaled 1.5bp;
                C0 = B0 rotatedaround(A, -109);
                C1 = B1 rotatedaround(A, -109);
                pointe(C1);
                label.rt(btex $C$ etex, C1)  rotatedaround(C1, 15);
                trace demidroite(A,C0) withpen pencircle scaled 1.5bp;
                trace marqueangle(C0,A,B0,0);
            );
            trace figUn rotatedaround(A, -15);
        \end{Geometrie}
        \correction
        $\widehat{BAC}$ mesure \ang{109}. \par\vspace*{5mm}
        \begin{Geometrie}[CoinHD={(8.5u,5u)}]
            trace feuillet withcolor white;
            pair A,B[],C[];
            A=u*(4.5,1);
            B0=u*(1,1);
            marque_p:="croix";
            picture figUn;
            figUn = image (
                trace rapporteurdouble(A,B0,-1);
                pointe(A);
                label.bot(btex $A$ etex, A) rotatedaround(A, 15);
                B1 = B0 shifted (-0.5*u,0);
                pointe(B1);
                label.bot(btex $B$ etex, B1)  rotatedaround(B1, 15);
                trace demidroite(A,B0) withpen pencircle scaled 1.5bp;
                C0 = B0 rotatedaround(A, -109);
                C1 = B1 rotatedaround(A, -109);
                pointe(C1);
                label.rt(btex $C$ etex, C1)  rotatedaround(C1, 15);
                trace demidroite(A,C0) withpen pencircle scaled 1.5bp;
                trace marqueangle(C0,A,B0,0);
                B2 = B0 shifted (-0.5u,u);
                B3 = B0 shifted(0.8u,0);
                drawarrow B2--B3 dashed evenly withcolor red;
                label.top(btex {\red $0$ exterieur} etex, B2) rotatedaround(B2, 15);
                C3 = B3 rotatedaround(A, -109);
                C2 = C3 shifted (-1.7u,0.9u);
                drawarrow C2--C3 dashed evenly withcolor blue;
                label.top(btex {\blue lecture de l'angle : \ang{109}} etex, C2) rotatedaround(C2, 15);
            );
            trace figUn rotatedaround(A, -15);
        \end{Geometrie}
    \end{methode*2*2}

    \begin{methode}[Tracer un angle à l'aide du rapporteur]
        Pour tracer un angle $\widehat{BAC}$ de mesure donnée à partir d'un segment $[AB]$ est déjà tracé :
        \begin{itemize}
            \item Placer le \textbf{centre du rapporteur sur le point qui sera le sommet de l'angle}, ici le point $A$.
            \item \textbf{Prolonger} le segment $[AB]$ en la demi-droite $[AB]$, si necessaire.
            \item Faire \textbf{pivoter le rapporteur autour de son centre} de façon à ce que la demi-droite $[AB]$ passe par une des deux graduations "0", intérieure ou extérieure.
            \item \textbf{En faisant bien attention à ne pas se tromper de graduation}, compter le nombre de graduations à partir du zéro pour arriver jusqu'à la mesure demandée, et faire une marque au crayon.
            \item Ôter le rapporteur et tracer le deuxième côté de l'angle.
        \end{itemize}
        \exercice
        Tracer $\widehat{BAC}$ mesurant \ang{75}. \par\vspace*{5mm}
        \begin{Geometrie}[CoinHD={(8.5u,5u)}]
            trace feuillet withcolor white;
            pair A,B[],C[];
            A=u*(3.5,1);
            B0=u*(7,1);
            marque_p:="croix";
            picture figUn;
            figUn = image (
                pointe(A);
                label.bot(btex $A$ etex, A);
                B1 = B0 shifted (0.5*u,0);
                pointe(B1);
                label.bot(btex $B$ etex, B1);
                trace segment(A,B1) withpen pencircle scaled 1.5bp;
            );
            trace figUn shifted (-2u,0);
        \end{Geometrie}
        \correction
        $\widehat{BAC}$ mesure \ang{60}. \par\vspace*{5mm}
        \begin{Geometrie}[CoinHD={(8.5u,5u)}]
            trace feuillet withcolor white;
            pair A,B[],C[];
            A=u*(3.5,1);
            B0=u*(7,1);
            marque_p:="croix";
            picture figUn;
            figUn = image (
                trace rapporteurdouble(A,B0,1);
                pointe(A);
                label.bot(btex $A$ etex, A);
                B1 = B0 shifted (0.5*u,0);
                pointe(B1);
                label.bot(btex $B$ etex, B1);
                trace segment(A,B1) withpen pencircle scaled 1.5bp;
                trace demidroite(A,B1) withpen pencircle scaled 1.5bp dashed evenly;
                C0 = B0 rotatedaround(A, 75);
                C1 = B1 rotatedaround(A, 75);
                pointe(C1);
                label.rt(btex $C$ etex, C1);
                trace demidroite(A,C1) withpen pencircle scaled 1.5bp dashed evenly;
                trace Codeangle(B0,A,C0,0,btex \ang{75} etex);
                B2 = B0 shifted (0.5u,u);
                B3 = B0 shifted(-u,0);
                drawarrow B2--B3 dashed evenly withcolor red;
                label.top(btex {\red $0$ intérieur} etex, B2);
                C3 = B3 shifted (0.3u,0) rotatedaround(A, 75);
                trace segment(B3 shifted (0.2u,0) rotatedaround(A, 75),B3 shifted (0.4u,0) rotatedaround(A, 75)) withpen pencircle scaled 1.5bp withcolor blue;
                C2 = C3 shifted (2.3u,0.3u);
                drawarrow C2--C3 dashed evenly withcolor blue;
                label.top(btex {\blue marquage de l'angle : \ang{75}} etex, C2);
            );
            trace figUn;
        \end{Geometrie}
    \end{methode}
\end{changemargin}

% \Methode{Tracer un angle à l'aide d'un rapporteur ?}{
% \begin{tabular}{c c}
% \begin{minipage}{8cm}
% \includegraphics[width=8.5cm]{figures/rapporteur.7}
% \end{minipage}
% &
% \hspace{-3cm}
% \begin{minipage}{8cm}
% \includegraphics[width=8.5cm]{figures/rapporteur.8}
% \end{minipage}
% \\
% \end{tabular}
% \ \\
% Pour tracer un angle $\widehat{BAC}$ mesurant $75^{\circ}$ (\textit{en supposant que le segment $[AB]$ est déjà tracé}):
% \begin{mylist}
% \item Placer le \textbf{centre du rapporteur sur le point qui sera le sommet de l'angle}, ici le point $A$.
% \item Si besoin est, on \textbf{prolonge} le segment $[AB]$ en la demi-droite $[AB]$.
% \item On fait \textbf{pivoter le rapporteur autour de son centre} de façon à ce que la demi-droite $[AB]$ passe par une des deux graduations "0", intérieure ou extérieure.
% \item \textbf{En faisant bien attention à ne pas se tromper de graduation}, compter le nombre de graduations à partir du zéro pour arriver jusqu'à la mesure demandée, ici $75^{\circ}$, et faire une marque au crayon.
% \item Ôter le rapporteur et tracer le deuxième côté de l'angle.
% \end{mylist}
% }
