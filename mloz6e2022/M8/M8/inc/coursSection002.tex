\section{Mesure d'un angle}

% \definNum{
% On peut mesurer \og l'ouverture \fg d'un angle;\\
% L'unité de mesure que l'on utilise au collège est le \textbf{degré}.\\
% L'instrument qui nous servira à mesurer des angles s'appelle un \textbf{rapporteur}.\\ 
% \begin{tabular}{c c}
% \begin{minipage}{8.5cm}
% Voici un rapporteur, gradué en degrés; ce rapporteur a une double graduation, qui va de 0 à 180 degrés.
% %\danger \textbf{Attention !!} Cette double graduation est source de nombreuses erreurs...
% \end{minipage}
% &
% \begin{minipage}{8.5cm}
% \begin{center}
% \includegraphics[width=8cm]{figures/rapporteur.4}
% \end{center}
% \end{minipage}
% \\
% \end{tabular}
% }

% \Methode{Mesurer un angle à l'aide du rapporteur ?}{
% \begin{tabular}{c c}
% \begin{minipage}{8.5cm}
% \includegraphics[width=7.5cm]{figures/rapporteur.5}
% \end{minipage}
% &
% \begin{minipage}{8.5cm}
% \includegraphics[width=7.5cm]{figures/rapporteur.6}
% \end{minipage}
% \\
% $\widehat{BAC}=60^{\circ}$ & $\widehat{BAC}=109^{\circ}$ \\
% \end{tabular}
% \ \\

% Pour déterminer la mesure en degrés de l'angle $\widehat{BAC}$:
% \begin{mylist}
% \item Placer le \textbf{centre du rapporteur sur le sommet de l'angle}, ici le point $A$.
% \item On fait \textbf{pivoter le rapporteur autour de son centre} de façon à ce que l'un des côtés de l'angle passe par une des deux graduations "0" (intérieure ou extérieure), et que l'autre côté de l'angle passe sous une autre graduation du rapporteur.
% \item \textbf{En faisant bien attention à ne pas se tromper de graduation}, compter le nombre de graduations à partir du zéro pour arriver jusqu'au deuxième côté de l'angle.
% \end{mylist}
% }

% \Methode{Tracer un angle à l'aide d'un rapporteur ?}{
% \begin{tabular}{c c}
% \begin{minipage}{8cm}
% \includegraphics[width=8.5cm]{figures/rapporteur.7}
% \end{minipage}
% &
% \hspace{-3cm}
% \begin{minipage}{8cm}
% \includegraphics[width=8.5cm]{figures/rapporteur.8}
% \end{minipage}
% \\
% \end{tabular}
% \ \\
% Pour tracer un angle $\widehat{BAC}$ mesurant $75^{\circ}$ (\textit{en supposant que le segment $[AB]$ est déjà tracé}):
% \begin{mylist}
% \item Placer le \textbf{centre du rapporteur sur le point qui sera le sommet de l'angle}, ici le point $A$.
% \item Si besoin est, on \textbf{prolonge} le segment $[AB]$ en la demi-droite $[AB]$.
% \item On fait \textbf{pivoter le rapporteur autour de son centre} de façon à ce que la demi-droite $[AB]$ passe par une des deux graduations "0", intérieure ou extérieure.
% \item \textbf{En faisant bien attention à ne pas se tromper de graduation}, compter le nombre de graduations à partir du zéro pour arriver jusqu'à la mesure demandée, ici $75^{\circ}$, et faire une marque au crayon.
% \item Ôter le rapporteur et tracer le deuxième côté de l'angle.
% \end{mylist}
% }
