%pre-001
\begin{prerequis}[Connaisances \emoji{red-heart} et compétences \emoji{diamond-suit} du cycle 3]    
   \begin{itemize}        
       \item[\emoji{red-heart}] Vocabulaire associé à ces objets et à leurs propriétés : côté, sommet, angle, hauteur.
       \columnbreak
       \item[\emoji{diamond-suit}] Reconnaître, nommer, décrire des triangles, dont les triangles particuliers (triangle rectangle, triangle isocèle, triangle équilatéral).       
   \end{itemize}
\end{prerequis}

\vfill

\begin{debat}[crop circle]
    Un {\bf cercle de culture} (traduction de {\bf crop circle}), ou {\bf agrogramme} ou encore {\bf agroglyphe} est un très grand motif ou un ensemble de motifs géométriques réalisé dans un champ de céréales en couchant les épis au sol, ils sont apparus pour la première fois dans les années 1970. Ces formes impressionnantes sont visibles depuis le ciel. Pendant plusieurs années, on croyait que leur origine était extra-terrestre jusqu'à ce que l'on découvre leur vraie origine\dots{} terrestre.
    \begin{center} 
       \includegraphics[width=5cm]{\currentpath/images/crop}
    \end{center}
    \bigskip
    \begin{cadre}[B2][F4]
       \begin{center}        
        Vidéo n°1 : \href{https://www.youtube-nocookie.com/embed/AqjSJuhdZ_s?playlist=AqjSJuhdZ_s&autoplay=1&iv_load_policy=3&loop=1&modestbranding=1&start=}{\bf making of de la réalisation}, {\it youtube}.

        Vidéo n°2 : \href{https://www.francetvinfo.fr/replay-jt/france-2/20-heures/video-en-alsace-les-mysterieux-cercles-dans-le-champ-etaient-un-cours-de-maths_3505711.html}{\bf Les mystérieux cercles dans le champ étaient un cours de maths}, {\it franceinfo}, émission {\it L'Oeil du 20 h}.
       \end{center}
    \end{cadre}
 \end{debat}