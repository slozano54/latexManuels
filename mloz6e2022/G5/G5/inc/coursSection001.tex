\section{Le cerlce, le disque}

\begin{definition}
    Le \textbf{cercle} de centre O de rayon $r$ est l'ensemble des points situés à une distance $r$ du point O. \\ 
    Le \textbf{disque} est l'intérieur du cercle.
 \end{definition}
 
 {\psset{unit=0.6}
 \begin{pspicture}(-5,0.1)(10,4.9)
    \psdots(2,2)
    \rput(2,1.6){$O$}
    \psarc(2,2){2}{70}{-45}
    \psarc[linestyle=dashed](2,2){2}{-45}{70}
    \psline[linecolor=B1,arrowsize=0.25]{<->}(0,2)(2,2)
    \rput(1,2.3){\textcolor{B1}{$r$}}
    \rput(10,2){\begin{minipage}{3.2cm} Un cercle se dessine en général à l'aide d'un compas. \end{minipage}}
 \end{pspicture}}
 
% \begin{minipage}{14.5cm}
%     Les longueurs $OB$, $OT$, $OM$ et $OS$ mesurent toutes $4,5$ cm.
%     \par\vspace{0.25cm}
%     \definNum{Lorsque des points sont à la même distance d'un point donné, on dit qu'ils sont \'{E}QUIDISTANTS de ce point.}
%     \par\vspace{0.25cm}
%     $B,T,M$ et $S$ sont équidistants de $O$.
%     \par\vspace{0.25cm}
%     \definNum{L'ensemble des points qui sont équidistants de $O$ est appelé \colorbox{red!30}{CERCLE} de CENTRE $O$ et de RAYON la distance en question.\par
%     On le note $(\mathscr C)$}
%     \end{minipage}
%     \begin{minipage}{5cm}
%     \includegraphics[scale=1]{figures/courscercles.1} 
%     \end{minipage}
    
%     \begin{minipage}{14.5cm}
%     \ColCadre{Vocabulaire}{
%     \begin{mylist}
%     \item Le segment $[OA]$ est \underline{UN} RAYON du cercle $(\mathscr C)$.
%     \par
%     La distance $OA$ est \underline{LE} RAYON du cercle $(\mathscr C)$.
%     \item Le segment $[CB]$ est \colorbox{green!30}{\underline{UN} DIAM\`{E}TRE} du cercle $(\mathscr C)$.
%     \par
%     La distance $CB$ est \colorbox{green!30}{\underline{LE} DIAM\`{E}TRE} du cercle $(\mathscr C)$.
%     \par
%     Les points $C$ et $B$ sont diamètralement opposés.
%     \item Le segment $[DE]$ joint deux points du cercle, c'est une \colorbox{blue!30}{CORDE}.
%     \end{mylist}
%     }
%     \par\vspace{0.5cm}
%     \end{minipage}
%     \begin{minipage}{5cm}
%     \includegraphics[scale=1]{figures/courscercles.2} 
%     \end{minipage}
    
%     \proprNum{(admise)}{le point $A$ est sur le cerlce de centre $O$ et de rayon $R$}{$OA=R$}
%     \par
%     \proprNum{(admise)}{$OA=R$}{le point $A$ est sur le cerlce de centre $O$ et de rayon $R$}
    
%     \subsection{Arc de cercle}
%     \begin{minipage}{14.5cm}
%     \definNum{
%     \begin{mylist}
%     \item Le "petit morceau" de cercle compris entre $A$ et $B$ est un \colorbox{red!30}{ARC du cercle $(\mathscr C)$}.
%     \item son centre et son rayon sont les mêmes que ceux du cercle $(\mathscr C)$.
%     \item "L'arc de cercle AB" se note $\widehat{AB}$, son angle $\widehat{AOB}$ vaut ici $60$\degre.
%     \end{mylist}
%     }
%     \end{minipage}
%     \begin{minipage}{5cm}
%     \includegraphics[scale=1]{figures/courscercles.3} 
%     \end{minipage}
    
    