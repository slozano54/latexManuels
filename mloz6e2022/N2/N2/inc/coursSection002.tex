\section{Écrire des nombres décimaux}

\begin{propriete}
   On peut écrire un nombre décimal dans un tableau de numération comportant une partie entière identique aux nombres entiers à laquelle on ajoute une partie décimale :
   \begin{center}
      \begin{tabular}{*{6}{|c}@{{\LARGE \textbf{,}}}*{5}{|c}|}
         \hline
         \multicolumn{3}{|c|}{\textbf{Milliers}} & \multicolumn{3}{|c|}{\textbf{Unités}}&\multicolumn{5}{c|}{}\\ \hline
         %%
         \rotatebox{90}{\parbox{3.5cm}{\textbf{Centaines}\\ \num{100000}}} & 
         \rotatebox{90}{\parbox{3.5cm}{\textbf{Dizaines}\\ \num{10000}}} & 
         \rotatebox{90}{\parbox{3.5cm}{\textbf{Unités}\\ \num{1000}}} & 
         \rotatebox{90}{\parbox{3.5cm}{\textbf{Centaines}\\ \num{100}}} & 
         \rotatebox{90}{\parbox{3.5cm}{\textbf{Dizaines}\\ \num{10}}} &
         \cellcolor{gray!25}\rotatebox{90}{\parbox{3.5cm}{\textbf{Unités}\\ \num{1} }} &
         \rotatebox{90}{\parbox{3.5cm}{\textbf{Dixièmes}\\ $\frac{1}{10}$}} & 
         \rotatebox{90}{\parbox{3.5cm}{\textbf{Centièmes}\\ $\frac{1}{100}$}} & 
         \rotatebox{90}{\parbox{3.5cm}{\textbf{Millièmes}\\ $\frac{1}{\num{1000}}$}} &
         \rotatebox{90}{\parbox{3.5cm}{\textbf{Dix-millièmes}\\ $\frac{1}{\num{10000}}$}} & 
         \rotatebox{90}{\parbox{3.5cm}{\textbf{Cent-millièmes}\\ $\frac{1}{\num{100000}}$}} 
          \\ \hline 
         %%
          &  &  &  & 4& \cellcolor{gray!25}5& 2& 1& 7& & \\ \hline
          & 8& 3& 5& 0& \cellcolor{gray!25}2&  &  &  & & \\ \hline
          &  & 2& 1& 4& \cellcolor{gray!25}7& 5& 6& 8& & \\ \hline
          &  &  &  &  &  \cellcolor{gray!25}&  &  &  & & \\ \hline
         \end{tabular}   
   \end{center}  
\end{propriete}

\begin{exemple*1}
   Dans le nombre \num{1 030 288 016.807} :      
   \begin{itemize}
      \item le chiffre des dizaines de millions est 3 ;
      \item le chiffre des dixièmes est 8 ;
      \item le chiffre des centièmes est 0 ;
      \item le chiffre des millièmes est 7.
   \end{itemize}
\end{exemple*1}