\section{Reconnaître une situations de proportionnalité} %%%%%%

\begin{methode*2*2}[Proportionnel ou pas ?]
   Pour reconnaître des grandeurs proportionnelles, on peut vérifier qu'il existe un coefficient de proportionnalité entre ces grandeurs.
   \exercice
      \begin{itemize}
         \item Le périmètre d'un cercle est-il proportionnel à son rayon ?
         \item L'aire d'un disque est-elle proportionnelle à son rayon ?
      \end{itemize}
   \correction
      \begin{itemize}
         \item On a $p =2\times \pi\times r =\;$\fbox{$2\times\pi$}$\times r$. \\
            $2\times\pi$ est un coefficient constant, le périmètre est donc bien proportionnel à son rayon.
         \item On a $A =\pi\times r^2 =\pi\times r\times r =\;$\fbox{$\pi\times r$}$\times r$. \\
      $\pi\times r$ varie en fonction de $r$, l'aire n'est donc pas proportionnelle à son rayon.
      \end{itemize}
   \exercice
      Ces deux tableaux T$_1$ et T$_2$ sont-ils des tableaux de proportionnalité ? \\ [2mm]
   T$_1$
      {\renewcommand{\arraystretch}{1.2}
      \begin{tabular}{|c|c|c|}
         \hline
         10 & 22 & 30 \\
         \hline
         12 & 26,4 & 36 \\
         \hline
      \end{tabular}
      \quad
      T$_2$
      \begin{tabular}{|c|c|c|c|}
         \hline
         10 & 22 & 30 & 45 \\
         \hline
         12 & 26,4 & 36 & 56 \\
         \hline
      \end{tabular}}
   \correction
      On calcule tous les quotients : \\ [2mm]
      $\dfrac{12}{10} =1,2$ \, ; \, $\dfrac{26,4}{22} =1,2$ \, ; \, $\dfrac{36}{30} =1,2$ \, ; \, $\dfrac{56}{45} \approx1,24$. \medskip
      \begin{itemize}
         \item T$_1$ est un tableau de proportionnalité de coefficient de proportionnalité 1,2.
         \item T$_2$ n'est pas un tableau de proportionnalité car le dernier quotient n'est pas égal aux autres.
      \end{itemize}
\end{methode*2*2}
