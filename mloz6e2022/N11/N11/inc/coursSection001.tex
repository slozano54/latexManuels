\section{Proportionnalité}
\begin{definition}
    Lorsque, dans un tableau, le \textbf{rapport (quotient)} de deux valeurs correspondantes est \textbf{constant (vaut toujours la même chose)}, c'est un \textbf{tableau de proportionnalité}
\end{definition}

\begin{exemple*1}
    \renewcommand{\arraystretch}{1.8}
    \[\begin{tabular}{c|>{\centering}p{3cm}|>{\centering}p{1.2cm}|>{\centering}p{1.2cm}|>{\centering}p{1.2cm}|>{\centering}p{1.2cm}|c}
    \cline{2-6}
    \rnode{A}{} & $1^{\grave{e}re}$ ligne : $x$&$0,5$&$1$&$1,5$&$2,5$& \rnode{C}{}\tabularnewline
    \cline{2-6}
    \rnode{B}{}& $2^{\grave{e}me}$ ligne : $y$&$1,5$&$3$&$4,5$&$7,5$& \rnode{D}{}\tabularnewline
    \cline{2-6} 
    \end{tabular}\]
    \renewcommand{\arraystretch}{1}

    \ncbar[angle=180,arm=0.8]{->}{A}{B} \ncput*{$\times3$}
    \ncbar[angle=0,arm=0.8]{->}{D}{C} \ncput*{$\div 3$}

    $\dfrac{1,5}{0,5} =\dfrac{3}{1}=\dfrac{4,5}{1,5}=\dfrac{7,5}{2,5}$ Ce nombre est \textbf{constant} et \textbf{plus grand que 1}.

    C'est le \textcolor{mygreen}{\textbf{coefficient de proportionnalité}} du tableau.

    Il permet d'exprimer la grandeur $y$ en fonction de la grandeur $x$, ici $y=\textcolor{mygreen}{\mathbf{3}}\times x$.
\end{exemple*1}


\begin{remarque}
    $\dfrac{0,5}{1,5} =\dfrac{1}{3} =\dfrac{1,5}{4,5}=\dfrac{2,5}{7,5}=\dfrac13$ est aussi un rapport constant mais inf\'erieur \`{a} 1.

    C'est l'autre coefficient du tableau.
\end{remarque}

\begin{definition}
    Deux grandeurs sont dites \textbf{proportionnelles}, lorsque, rang\'ees dans un tableau, les valeurs de l'une avec les valeurs de l'autre forment un tableau de proportionnalit\'e.    
\end{definition}

\begin{remarque}
    Deux grandeurs ne sont pas toujours proportionnelles.

    Par exemple, la \textbf{taile} d'une personne et son \textbf{âge} ne sont pas des grandeurs proportionnelles.    
\end{remarque}
 
