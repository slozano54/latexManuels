\begin{exercice*}
   Une pâtissière a pesé ses beignets, que l’on
   suppose de masse identique, et a trouvé :
   \begin{center}
     \setlength{\tabcolsep}{3\tabcolsep}
       \begin{tabular}{cc}
         \Masse{300}&\Masse{450}\\
         \begin{mplibcode}
            input fiziko;
            path p[];
            p2 := fullcircle scaled 7mm;
            addStrandToKnot (Donut) (p2,1/5cm,"s","") ;
            picture Donuts;
            Donuts=image(draw knotFromStrands(Donut););
            for k=0 upto 1:
            draw Donuts shifted(k*1.25cm,0);
            endfor;
         \end{mplibcode}
         &
         \begin{mplibcode}
            input fiziko;
            path p[];
            p2 := fullcircle scaled 7mm;
            addStrandToKnot (Donut) (p2,1/5cm,"s","") ;
            picture Donuts;
            Donuts=image(draw knotFromStrands(Donut););
            for k=0 upto 2:
            draw Donuts shifted(k*1.25cm,0);
            endfor;
         \end{mplibcode}\\
       \end{tabular}
     \end{center}
     Combien pèsent cinq, six, dix et un beignet ?
     \begin{center}
       \setlength{\tabcolsep}{0.5\tabcolsep}
     \Propor[Largeur=18pt,Condense,GrandeurA=\begin{tabular}{cc}Nombre\\de beignets\end{tabular},GrandeurB=Masse (en \Masse{}),Math]{2/300,3/450,\pointilles/\pointilles,\pointilles/\pointilles,\pointilles/\pointilles,\pointilles/\pointilles}
     \end{center}
\end{exercice*}
