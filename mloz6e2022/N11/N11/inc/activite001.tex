\begin{activite}[Le puzzle de Brousseau]
    {\bf Objectifs :} mettre en \oe uvre un ou des moyens pour résoudre un problème d'agrandissement ; reproduire une figure géométrique en respectant des mesures ; rendre compte d'un travail en groupe.
    \partie[présentation du puzzle]
        Ci-dessous se trouve un puzzle composé de quatre pièces A, B, C et D dont les mesures dont indiquées sur la figure.
        \begin{center}
            \begin{pspicture}(-1.5,-1.2)(12.5,11.5)
            \psframe(0,0)(11,11)
            \psline(4,0)(4,11)
            \psline(4,6)(11,6)
            \psline(7,0)(7,6)
            {\small
                \rput(12,3){\ucm{6}}
                \rput(12,8.5){\ucm{5}}
                \rput(2,-0.5){\ucm{4}}
                \rput(5.5,-0.5){\ucm{2}}
                \rput(9,-0.5){\ucm{5}}}
                \rput(2,5.5){A}
                \rput(5.5,3){B}
                \rput(9,3){C}
                \rput(7.5,8.5){D}
            \end{pspicture}
            \end{center}
    
    \partie[travail demandé]
        Par groupes de trois ou quatre, vous allez devoir refaire le même puzzle mais en plus grand : il faudra s'accorder sur la procédure à adopter pour agrandir les éléments du puzzle, se répartir la construction des pièces en faisant les calculs individuellement puis assembler les morceaux pour reconstituer le puzzle agrandi. \\
        Le compte-rendu de vos recherches sera présenté sous la forme d’une affiche par groupe. \\
        \begin{center}
            {\bf C'est parti\dots{} le segment de \ucm{4} devra mesurer \ucm{6} sur votre puzzle agrandi.}
        \end{center}
 \end{activite}