% Les enigmes ne sont pas numérotées par défaut donc il faut ajouter manuellement la numérotation
% si on veut mettre plusieurs enigmes
% \refstepcounter{exercice}
% \numeroteEnigme
\begin{enigme}[Des samoussas !]
    \begin{changemargin}{-10mm}{-10mm}
        Lors d'une réception de 45 personnes, on demande au cuisinier de préparer des samoussas à raison de 4 samoussas par personne.
        \begin{center}
           \includegraphics[width=17cm]{\currentpath/images/samoussas}
        \end{center}
        Le cuisinier a déjà dans sa cuisine les épices, l'ail et la menthe et il a relevé les prix au supermarché du coin : \\
        \begin{center}
           \parbox{7.5cm}{\includegraphics[width=7.5cm]{\currentpath/images/brick}} \qquad \parbox{8cm}{\includegraphics[width=8cm]{\currentpath/images/viande} \\ [3mm]\includegraphics[width=8cm]{\currentpath/images/oignons}}
        \end{center}
        \medskip
        \begin{cadre}[B2][F4]
           À combien revient la confection des samoussas ? 
        \end{cadre}
        Par groupes de trois ou quatre, proposer une affiche expliquant les achats du cuisinier.
         
        \vfill\hfill{\it\footnotesize Source : inspiré de \href{https://irem.univ-reunion.fr/IMG/pdf/apprendre_avec_taches_complexes_au_cycle_3.pdf}{Apprendre avec ls tâches complexes}}
    \end{changemargin}
\end{enigme}

% Pour le corrigé, il faut décrémenter le compteur, sinon il est incrémenté deux fois
% \addtocounter{exercice}{-1}

% \begin{corrige}
%     Correction du binz.
% \end{corrige}

