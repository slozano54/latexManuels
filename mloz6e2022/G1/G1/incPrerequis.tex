%pre-001
\begin{prerequis}[Connaisances \emoji{red-heart} et compétences \emoji{diamond-suit} du cycle 3]    
   \begin{itemize}        
       \item[\emoji{red-heart}] Vocabulaire associé à ces objets et à leurs propriétés : côté, sommet, angle, hauteur.
       \columnbreak
       \item[\emoji{diamond-suit}] Reconnaître, nommer, décrire des triangles, dont les triangles particuliers (triangle rectangle, triangle isocèle, triangle équilatéral).       
   \end{itemize}
\end{prerequis}
\begin{debat}[Les définitions d'Euclide]
    La mathématicien grec {\it Euclide}, considéré comme le père de la géométrie, définit les objets géométriques au III\up{e} siècle av. J.-C. Son 1\up{er} livre comprend notamment les définitions suivantes :
    \begin{itemize}
       \item le {\bf point} est ce dont la partie est nulle ;
       \item une {\bf ligne} est une longueur sans largeur ;
       \item la {\bf ligne droite} est celle qui est également placée entre ses points.
    \end{itemize}
    \begin{pspicture}(-4,-0.25)(9,2.75)
       \psset{linecolor=B1}
       \psdot(0,1)
       \psbezier(2,0)(3,2)(4,0.5)(5,2)
       \psline(6,0)(9,2)
    \end{pspicture}
    \bigskip
    \begin{cadre}[B2][F4]
       \begin{center}
          Vidéo : \href{https://www.yout-ube.com/watch?v=enZpq8jvFEs}{\bf L'axiomatique. Les éléments d'Euclide}, chaîne YouTube de {\it Monsieur Phi}.
          
          De 2 min 30 s à 8 min.
       \end{center}
    \end{cadre}
 \end{debat}