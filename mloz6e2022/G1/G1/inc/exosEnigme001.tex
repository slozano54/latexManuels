% Les enigmes ne sont pas numérotées par défaut donc il faut ajouter manuellement la numérotation
% si on veut mettre plusieurs enigmes
% \refstepcounter{exercice}
% \numeroteEnigme
\begin{enigme}[La pipopipette]
   \partie[présentation du jeu]
      La pipopipette ou \og jeu des petits carrés \fg{} est un jeu se pratiquant à deux joueurs en tour par tour dont l'idée serait attribuée à des élèves de l'École polytechnique à la fin du {\small XIX}\up{e} siècle : pipo désignait à l'époque cette école en argot. \\
      {\bf But du jeu :} former des carrés. Le gagnant est celui qui a fermé le plus de carrés. \\
      {\bf Règles du jeu :}
         \begin{itemize}
            \item À chaque tour, chaque joueur trace un petit segment suivant le quadrillage de la feuille.
            \item Chaque fois qu'un joueur peut fermer un carré, il marque le carré de son signe rejoue.
            \item Quand la grille est remplie, on compte le nombre de carrés fermés pour chaque joueur.
         \medskip
         \end{itemize}
      {\bf Exemple avec un terrain de 2 $\times$ 2:}
         \begin{center}
         {\psset{unit=0.6}
            \begin{tabular}{*{6}{>{\centering\arraybackslash}p{2.4cm}}}
               \begin{pspicture}(0,0)(2,1.8)
                  \psgrid[subgriddiv=0,gridlabels=0,gridcolor=lightgray](0,0)(2,2)
                  \psset{linewidth=0.5mm}
                  \psline[linecolor=A1](0,2)(1,2)
               \end{pspicture}
               &
               \begin{pspicture}(0,0)(2,1.8)
                  \psgrid[subgriddiv=0,gridlabels=0,gridcolor=lightgray](0,0)(2,2)
                  \psset{linewidth=0.5mm,linecolor=A1}
                  \psline(0,2)(1,2)
                  \psset{linecolor=B1}
                  \psline(1,0)(2,0)
               \end{pspicture}
               &
               \begin{pspicture}(0,0)(2,1.8)
                  \psgrid[subgriddiv=0,gridlabels=0,gridcolor=lightgray](0,0)(2,2)
                  \psset{linewidth=0.5mm,linecolor=A1}
                  \psline(0,2)(2,2)
                  \psset{linecolor=B1}
                  \psline(1,0)(2,0)
               \end{pspicture}
               &
               \begin{pspicture}(0,0)(2,1.8)
                  \psgrid[subgriddiv=0,gridlabels=0,gridcolor=lightgray](0,0)(2,2)
                  \psset{linewidth=0.5mm,linecolor=A1}
                  \psline(0,2)(2,2)
                  \psset{linecolor=B1}
                  \psline(0,0)(2,0)
               \end{pspicture}
               &
               \begin{pspicture}(0,0)(2,1.8)
                  \psgrid[subgriddiv=0,gridlabels=0,gridcolor=lightgray](0,0)(2,2)
                  \psset{linewidth=0.5mm,linecolor=A1}
                  \psline(0,1)(0,2)(2,2)
                  \psset{linecolor=B1}
                  \psline(0,0)(2,0)
               \end{pspicture}
               &
               \begin{pspicture}(0,0)(2,1.8)
                  \psgrid[subgriddiv=0,gridlabels=0,gridcolor=lightgray](0,0)(2,2)
                  \psset{linewidth=0.5mm,linecolor=A1}
                  \psline(0,1)(0,2)(2,2)
                  \psset{linecolor=B1}
                  \psline(0,0)(2,0)(2,1)
               \end{pspicture} \\
               \textcolor{A1}{joueur 1} & \textcolor{B1}{joueur 2} & \textcolor{A1}{joueur 1} & \textcolor{B1}{joueur 2} & \textcolor{A1}{joueur 1} & \textcolor{B1}{joueur 2} \\ [5mm]
               \begin{pspicture}(0,0)(2,2)
                  \psgrid[subgriddiv=0,gridlabels=0,gridcolor=lightgray](0,0)(2,2)
                  \psset{linewidth=0.5mm,linecolor=A1}
                  \psline(1,1)(0,1)(0,2)(2,2)
                  \psset{linecolor=B1}
                  \psline(0,0)(2,0)(2,1)
               \end{pspicture}
               &
               \begin{pspicture}(0,0)(2,2)
                  \psgrid[subgriddiv=0,gridlabels=0,gridcolor=lightgray](0,0)(2,2)
                  \psset{linewidth=0.5mm,linecolor=A1}
                  \psline(1,1)(0,1)(0,2)(2,2)
                  \psset{linecolor=B1}
                  \psline(0,0)(2,0)(2,1)
                  \psline(1,1)(1,2)
                  \psdot[dotstyle=+](0.5,1.5)
               \end{pspicture}
               &
               \begin{pspicture}(0,0)(2,2)
                  \psgrid[subgriddiv=0,gridlabels=0,gridcolor=lightgray](0,0)(2,2)
                  \psset{linewidth=0.5mm,linecolor=A1}
                  \psline(1,1)(0,1)(0,2)(2,2)
                  \psset{linecolor=B1}
                  \psline(0,1)(0,0)(2,0)(2,1)
                  \psline(1,1)(1,2)
                  \psdot[dotstyle=+](0.5,1.5)
               \end{pspicture}
               &
               \begin{pspicture}(0,0)(2,2)
                  \psgrid[subgriddiv=0,gridlabels=0,gridcolor=lightgray](0,0)(2,2)
                  \psset{linewidth=0.5mm,linecolor=A1}
                  \psline(1,0)(1,1)(0,1)(0,2)(2,2)
                  \psdot[dotstyle=*](0.5,0.5)
                  \psset{linecolor=B1}
                  \psline(0,1)(0,0)(2,0)(2,1)
                  \psline(1,1)(1,2)
                  \psdot[dotstyle=+](0.5,1.5)
               \end{pspicture}
               &
               \begin{pspicture}(0,0)(2,2)
                  \psgrid[subgriddiv=0,gridlabels=0,gridcolor=lightgray](0,0)(2,2)
                  \psset{linewidth=0.5mm,linecolor=A1}
                  \psline(1,0)(1,1)(0,1)(0,2)(2,2)
                  \psline(1,1)(2,1)
                  \psdots[dotstyle=*](0.5,0.5)(1.5,0.5)
                  \psset{linecolor=B1}
                  \psline(0,1)(0,0)(2,0)(2,1)
                  \psline(1,1)(1,2)
                  \psdot[dotstyle=+](0.5,1.5)
               \end{pspicture}
               &
                \begin{pspicture}(0,0)(2,2)
                 \psgrid[subgriddiv=0,gridlabels=0,gridcolor=lightgray](0,0)(2,2)
                  \psset{linewidth=0.5mm,linecolor=A1}
                  \psline(1,0)(1,1)(0,1)(0,2)(2,2)
                  \psline(1,1)(2,1)(2,2)
                  \psdots[dotstyle=*](0.5,0.5)(1.5,0.5)(1.5,1.5)
                  \psset{linecolor=B1}
                  \psline(0,1)(0,0)(2,0)(2,1)
                  \psline(1,1)(1,2)
                  \psdot[dotstyle=+](0.5,1.5)
               \end{pspicture}
               \\
               \textcolor{A1}{joueur 1} & \textcolor{B1}{joueur 2} & \textcolor{B1}{joueur 2} & \textcolor{A1}{joueur 1} & \textcolor{A1}{joueur 1} & \textcolor{A1}{joueur 1} \\
            \end{tabular}}
         \end{center}
      Le \textcolor{A1}{joueur 1} a pris possession de trois carrés alors que le \textcolor{B1}{joueur 2} en a un seul, c'est donc le \textcolor{A1}{joueur 1} qui gagne. \\

   \partie[à vous de jouer !!!]
      En binôme, compléter  ces grilles.
      \begin{center}
        {\psset{unit=0.9}
         \begin{tabular}{>{\centering\arraybackslash}p{2cm}>{\centering\arraybackslash}p{3.1cm}>{\centering\arraybackslash}p{4.2cm}>{\centering\arraybackslash}p{5cm}}
            \begin{pspicture}(0,0)(2,2)
               \psgrid[subgriddiv=0,gridlabels=0,gridcolor=lightgray](0,0)(2,2)
            \end{pspicture}
            &
            \begin{pspicture}(0,0)(3,3)
               \psgrid[subgriddiv=0,gridlabels=0,gridcolor=lightgray](0,0)(3,3)
            \end{pspicture}
            &
            \begin{pspicture}(0,0)(4,4)
               \psgrid[subgriddiv=0,gridlabels=0,gridcolor=lightgray](0,0)(4,4)
            \end{pspicture}
           &
           \begin{pspicture}(0,0)(5,4.5)
               \psgrid[subgriddiv=0,gridlabels=0,gridcolor=lightgray](0,0)(5,5)
            \end{pspicture} \\ [5mm]
         \end{tabular} \\
         \begin{tabular}{>{\centering\arraybackslash}p{6.2cm}>{\centering\arraybackslash}p{9cm}}
            \begin{pspicture}(0,0)(6,3)
               \psgrid[subgriddiv=0,gridlabels=0,gridcolor=lightgray](0,0)(6,3)
            \end{pspicture}
            &
            \begin{pspicture}(0,0)(9,3.5)
               \psgrid[subgriddiv=0,gridlabels=0,gridcolor=lightgray](0,0)(9,4)
            \end{pspicture} \\
         \end{tabular}
        }
      \end{center}
\end{enigme}

% Pour le corrigé, il faut décrémenter le compteur, sinon il est incrémenté deux fois
% \addtocounter{exercice}{-1}
% \begin{corrige}
%     Correction enigme de la fin de la partie cours.  
% \end{corrige}