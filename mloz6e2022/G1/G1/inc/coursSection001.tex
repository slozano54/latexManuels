\section{Droites, demi-droites, segments}

\begin{definition}
   \begin{itemize}
      \item Une {\bf droite} est une ligne rectiligne infinie. On peut noter une droite de différentes façons : \\
      \begin{pspicture}(-0.5,-0.7)(13,1)
         \pstGeonode[PosAngle=-90,PointSymbol=+](0,0){A}(3,0.5){B}
         \pstLineAB[nodesep=-0.5,linecolor=A1]{A}{B}
         \pstLabelAB{A}{B}{\small droite $(AB)$}
         \psline[linecolor=A1](4,0.25)(8,0)
         \rput(7.25,-0.35){$(d)$}
         \rput{-4}(6,0.5){\small droite $(d)$}
         \psline[linecolor=A1](8.5,-0.25)(12.5,0.5)
         \rput(9,-0.15){+}
         \rput(9,-0.5){$T$}
         \rput(12,0.2){$u$}
         \rput{9}(10.5,0.5){\small droite $(Tu)$}
      \end{pspicture}
      \item Une {\bf demi-droite} est une portion de droite limitée d'un seul côté par un point appelé origine. La demi-droite d'origine $A$ passant par $B$ se note $[AB)$. \\
      \begin{pspicture}(-5,-0.2)(5,1)
         \pstGeonode[PosAngle=180,PointSymbol=+](0,0){A}
         \pstGeonode[PosAngle=-90,PointSymbol=+](4,0.5){B}
         \pstLineAB[nodesepB=-1,linecolor=A1]{A}{B}
         \pstLabelAB{A}{B}{\small demi-droite $[AB)$}
      \end{pspicture}
      \item Un \textbf{segment} est une portion de droite limitée par deux points appelés extrémités. Le segment d'extrémités $A$ et $B$ se note $[AB]$ ou $[BA]$. \\
      \begin{pspicture}(-5,0.4)(5,1.2)
         \pstGeonode[PosAngle={180,-90},PointSymbol=+](0,0.2){A}(4,0.7){B}
         \pstLineAB[linecolor=A1]{A}{B}
         \pstLabelAB{A}{B}{\small segment $[AB]$}
      \end{pspicture}
   \end{itemize}
\end{definition}

\begin{exemple*1}
   \begin{minipage}{5cm}
      \begin{pspicture}(-1,-0.25)(4,1.8)
         \pstGeonode[PosAngle=-90,PointSymbol=+](0,0){A}(3,0){B}(2,1.25){C}
         \pstLineAB[nodesep=-0.5]{A}{B}
         \pstLineAB[nodesepB=-0.75]{A}{C}
         \pstLineAB{C}{B}
      \end{pspicture}
   \end{minipage}
   \begin{minipage}{6.5cm}
   \begin{itemize}
      \item $(AB)$ est une droite ;
      \item $[AC)$ est la demi-droite d'origine $A$ passant par $C$ ;
      \item $[CB]$ est le segment d'extrémités $C$ et $B$.
   \end{itemize}
   \end{minipage}
\end{exemple*1}

\begin{remarque}
    \begin{itemize}
        \item Le crochet signifie que l'objet est limité de ce côté.
        \item La parenthèse signifie que l'objet est illimité.
    \end{itemize}
\end{remarque}