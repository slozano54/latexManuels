% Les enigmes ne sont pas numérotées par défaut donc il faut ajouter manuellement la numérotation
% si on veut mettre plusieurs enigmes
% \refstepcounter{exercice}
% \numeroteEnigme
\vspace*{-15mm}
\begin{enigme}[La relation d'Euler\footnote{Leonhard Euler, né le 15 avril 1707 à Bâle (Suisse) et mort à 76 ans le 7 septembre 1783 à Saint-Pétersbourg (Empire russe), est un mathématicien et physicien suisse.}]
    \begin{changemargin}{-10mm}{-10mm}
        On a reproduit page suivante une représentation en perspective cavalière de neuf solides. Pour chacun de ces solides, effectuer les actions suivantes à l'aide du tableau ci-dessous :
        \begin{itemize}
           \item retrouver son nom dans le tableau (indiquer le numéro n) ;
           \item trouver le nombre de faces F ;
           \item trouver le nombre de sommets S ;
           \item trouver le nombre d'arrêtes A ;
           \item calculer la valeur de F + S $-$ A, que remarque-t-on ?
           \item dire si le solide est régulier, c'est-à-dire si toutes ses faces sont identiques et régulières, et tous les angles du solide sont identiques ;
           \item dire si le solide est convexe, c'est-à-dire s'il n'a pas de \og creux \fg{} ou de trou ;
           \item enfin, dire s'il s'agit d'un solide de Platon (polyèdre régulier et convexe).
        \end{itemize}
        \begin{center}   
           {\renewcommand{\arraystretch}{2.2}
           \begin{tabular}{|>{\centering\arraybackslash}m{2.5cm}|>{\centering\arraybackslash}p{0.9cm}|>{\centering\arraybackslash}p{0.9cm}|>{\centering\arraybackslash}p{0.9cm}|>{\centering\arraybackslash}p{0.9cm}|>{\centering\arraybackslash}p{1.9cm}|>{\centering\arraybackslash}m{1.4cm}|>{\centering\arraybackslash}m{1.4cm}|>{\centering\arraybackslash}m{1.4cm}|}
              \hline
              \rowcolor{LightGray}Nom du solide & n & F & S & A & F + S $-$ A & est-il régulier ? & est-il convexe ? & solide de Platon ? \\
              \hline
              \cellcolor{LightGray}Tétraèdre & & & & & & & & \\
              \hline
              \cellcolor{LightGray}Polyèdre étoilé & & & & & & & & \\
              \hline
              \cellcolor{LightGray}Octaèdre & & & & & & & & \\
              \hline
              \cellcolor{LightGray}Pyramide & & & & & & & & \\
              \hline
              \cellcolor{LightGray}Icosaèdre & & & & & & & & \\
              \hline
              \cellcolor{LightGray}Prisme & & & & & & & & \\
              \hline
              \cellcolor{LightGray}Cube & & & & & & & & \\
              \hline
              \cellcolor{LightGray}Beignoïde  & & & & & & & & \\
              \hline
              \cellcolor{LightGray}Dodécaèdre & & & & & & & & \\
              \hline    
        \end{tabular}}
        \end{center}
        \hfill{\footnotesize\it Source : d'après une activité parue dans la revue {\it Envol}. n\degre129, octobre-novembre-décembre 2004.}
        {\psset{unit=0.8}
        \begin{pspicture}(-4,-6)(18,18)
           \rput(12,16){\psSolid[object=cube,a=1,RotZ=20,action=draw*, fillcolor=orange!20] \\
              {\Huge 2}}
           \rput(6,-4){\psSolid[object=tetrahedron,r=1.1,RotZ=20,action=draw*,fillcolor=A1!20] \\
               {\Huge 9}}
           \rput(12,0){\psSolid[object=dodecahedron,a=0.9,RotZ=0,action=draw*,fillcolor=B2!20] \\
               {\Huge 8}}
           \rput(0,8){\psSolid[object=octahedron,a=1,RotZ=20,action=draw*,fillcolor=yellow!20] \\
               {\Huge 4}}
           \rput(6,4){\psSolid[object=icosahedron,a=1,RotZ=60,action=draw*,fillcolor=lightgray!20] \\
               {\Huge6}}
           \rput(12,7.7){\psSolid[object=prisme,h=0.6,RotZ=10,action=draw*,fillcolor=green!20] \\
               {\Huge 5}}
           \rput(0,17){\psSolid[object=new, sommets=0 -0.7 0 -0.7 0 0 0 1 0 1 0 0 0 0 -2,faces={[3 2 1 0][4 0 3][4 3 2][4 2 1][4 1 0]},RotZ=40,action=draw*,fillcolor=magenta!20]{\Huge 1}}
           \rput(5.5,12){\psSolid[object=anneau,r=0.4,R=1,h=1.5,ngrid=4,RotZ=30,RotY=50,action=draw*,fillcolor=cyan!20] \\
              {\Huge 3}}
           \rput(0,0){\psSolid[object=new,sommets=-0.3 -0.3 -0.3 0.3 -0.3 -0.3 0.3 0.3 -0.3 -0.3 0.3 -0.3 -0.3 -0.3 0.3 0.3 -0.3 0.3 0.3 0.3 0.3 -0.3 0.3 0.3 1.5 0 0 0 1.5 0 0 0 1.5 0 0 -1.5 -1.5 0 0 0 -1.5 0,faces={[1 2 6 5][0 3 2 1][4 5 6 7][0 1 5 4][3 7 6 2][0 4 7 3][11 3 2][11 0 3][11 1 0][11 2 1][12 7 3][12 4 7][12 0 4][12 3 0][13 1 5][13 0 1][13 4 0][13 5 4][8 2 6][8 6 5][8 5 1][8 1 2][9 3 7][9 7 6][9 6 2][9 2 3][10 6 7][10 5 6][10 7 4][10 4 5]},RotZ=20,RotX=10,action=draw*,fillcolor=green!20] \\
              {\Huge 7}}
        \end{pspicture}}
    \end{changemargin}    
\end{enigme}

% Pour le corrigé, il faut décrémenter le compteur, sinon il est incrémenté deux fois
% \addtocounter{exercice}{-1}
% \begin{corrige}
%     dsfjsdlfkj
% \end{corrige}

\begin{corrige}
    \smallskip
       {\renewcommand{\arraystretch}{2}
       \small
          \begin{CLtableau}{\linewidth}{5}{c}
             \hline
             Nom du solide & n & F & S & A \\
             \hline
             Tétraèdre & \textcolor{red}{9} & \textcolor{red}{4} & \textcolor{red}{4} & \textcolor{red}{6} \\
             \hline
             Polyèdre étoilé & \textcolor{red}{7} & \textcolor{red}{24} & \textcolor{red}{14} & \textcolor{red}{36} \\
             \hline
             Octaèdre & \textcolor{red}{4} & \textcolor{red}{8} & \textcolor{red}{6} & \textcolor{red}{12} \\
             \hline
             Pyramide & \textcolor{red}{1} & \textcolor{red}{5} & \textcolor{red}{5} & \textcolor{red}{8} \\
             \hline
             Icosaèdre & \textcolor{red}{6} & \textcolor{red}{20} & \textcolor{red}{12} & \textcolor{red}{30} \\
             \hline
             Prisme & \textcolor{red}{5} & \textcolor{red}{5} & \textcolor{red}{6} & \textcolor{red}{9} \\
             \hline
              Cube & \textcolor{red}{2} & \textcolor{red}{6} & \textcolor{red}{8} & \textcolor{red}{12} \\
             \hline
             Beignoïde & \textcolor{red}{3} & \textcolor{red}{16} & \textcolor{red}{16} & \textcolor{red}{32} \\
             \hline
             Dodécaèdre & \textcolor{red}{8} & \textcolor{red}{12} & \textcolor{red}{20} & \textcolor{red}{30} \\
             \hline    
          \end{CLtableau}
       \bigskip
          \begin{CLtableau}{\linewidth}{5}{c}
             \hline
             Nom du solide & \!\!F+S-A & \!\!régulier & \!\!convexe & \!\!Platon \\
             \hline
             Tétraèdre & \textcolor{red}{2} & \textcolor{red}{oui} & \textcolor{red}{oui} & \textcolor{red}{oui} \\
             \hline
             Polyèdre étoilé & \textcolor{red}{2} & \textcolor{red}{non} & \textcolor{red}{non} & \textcolor{red}{non} \\
             \hline
             Octaèdre & \textcolor{red}{2} & \textcolor{red}{oui} & \textcolor{red}{oui} & \textcolor{red}{oui} \\
             \hline
             Pyramide & \textcolor{red}{2} & \textcolor{red}{non} & \textcolor{red}{oui} & \textcolor{red}{non} \\
             \hline
              Icosaèdre & \textcolor{red}{2} & \textcolor{red}{oui} & \textcolor{red}{oui} & \textcolor{red}{oui} \\
             \hline
             Prisme & \textcolor{red}{2} & \textcolor{red}{non} & \textcolor{red}{oui} & \textcolor{red}{non} \\
             \hline
             Cube & \textcolor{red}{2} & \textcolor{red}{oui} & \textcolor{red}{oui} & \textcolor{red}{oui} \\
             \hline
             Beignoïde & \textcolor{red}{0} & \textcolor{red}{non} & \textcolor{red}{non} & \textcolor{red}{non} \\
             \hline
             Dodécaèdre & \textcolor{red}{2} & \textcolor{red}{oui} & \textcolor{red}{oui} & \textcolor{red}{oui} \\
             \hline
       \end{CLtableau}}
\end{corrige}  