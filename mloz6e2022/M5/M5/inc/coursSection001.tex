\section{Volume par dénombrement}
\begin{definition}
    La \textbf{mesure} de l'espace occupé par un solide, dans une unité choisie, s'appelle {\bf le volume} de ce solide.
\end{definition}

\begin{methode*1}[Connaître le volume d'un solide]
    Pour connaître le volume d'un solide, on calcule le nombre d'{\bf unités de volume} qui sont nécessaires pour remplir exactement cet espace.
    \exercice
    En prenant comme unité de volume un cube : 
    \begin{Geometrie}
        u:=0.5*u;
        pair A,B,C,D,E,F,G,H;
        A=u*(1,1);
        B=u*(2,1);
        F=u*(2,2);
        E=u*(1,2);
        D=A shifted (0.5u,0.4u);
        C=B shifted (0.5u,0.4u);
        H=E shifted (0.5u,0.4u);
        G=F shifted (0.5u,0.4u);
        remplis polygone(B,C,G,F) withcolor Cornsilk;
        remplis polygone(E,H,G,F) withcolor 0.7Cornsilk+0.3black;
        remplis polygone(A,B,F,E) withcolor 0.9Cornsilk+0.1black;
        trace A--B--F--E--cycle;
        trace polygone(B,C,G,F);
        trace chemin(G,H,E);
    \end{Geometrie}

    Déterminer le volume de cet assemblage dont voici deux représentations :

    \VueCubes[%
    Creation,%
    Profondeur=3,%
    Largeur=4,%
    CouleurCube=Cornsilk]{%
        1,2,3,%
        0,2,1,%
        0,2,1,%
        1,1,1%
    }
    \correction
    En éclatant le solide, on peut plus facilement compter les cubes qui le composent :

    \VueCubes[%
    Creation,%
    Profondeur=7,%
    Largeur=4,%
    CouleurCube=Cornsilk]{%
        1,0,0,2,0,0,3,%
        0,0,0,2,0,0,1,%
        0,0,0,2,0,0,1,%,
        1,0,0,1,0,0,1%
    }

    On obtient alors $2+7+6=15$, cet assemblage a donc un volume de $15$
    \begin{Geometrie}
        u:=0.5*u;
        pair A,B,C,D,E,F,G,H;
        A=u*(1,1);
        B=u*(2,1);
        F=u*(2,2);
        E=u*(1,2);
        D=A shifted (0.5u,0.4u);
        C=B shifted (0.5u,0.4u);
        H=E shifted (0.5u,0.4u);
        G=F shifted (0.5u,0.4u);
        remplis polygone(B,C,G,F) withcolor Cornsilk;
        remplis polygone(E,H,G,F) withcolor 0.7Cornsilk+0.3black;
        remplis polygone(A,B,F,E) withcolor 0.9Cornsilk+0.1black;
        trace A--B--F--E--cycle;
        trace polygone(B,C,G,F);
        trace chemin(G,H,E);
    \end{Geometrie}  
\end{methode*1}

\begin{remarque}
    Deux solides peuvent avoir des formes différentes mais un volume identique.

    En prenant ce cube pour unité de volume :
    \begin{Geometrie}
        u:=0.5*u;
        pair A,B,C,D,E,F,G,H;
        A=u*(1,1);
        B=u*(2,1);
        F=u*(2,2);
        E=u*(1,2);
        D=A shifted (0.5u,0.4u);
        C=B shifted (0.5u,0.4u);
        H=E shifted (0.5u,0.4u);
        G=F shifted (0.5u,0.4u);
        remplis polygone(B,C,G,F) withcolor Cornsilk;
        remplis polygone(E,H,G,F) withcolor 0.7Cornsilk+0.3black;
        remplis polygone(A,B,F,E) withcolor 0.9Cornsilk+0.1black;
        trace A--B--F--E--cycle;
        trace polygone(B,C,G,F);
        trace chemin(G,H,E);
    \end{Geometrie}
    
    Les deux solides ci-dessous ont un volume de 12 unités :

    \smallskip
    \begin{center}        
        \begin{Geometrie}
            u:=0.5*u;
            pair A,B,C,D,E,F,G,H;
            A=u*(1,1);
            B=u*(4,1);
            F=u*(4,3);
            E=u*(1,3);
            D=A shifted (u,0.8u);
            C=B shifted (u,0.8u);
            H=E shifted (u,0.8u);
            G=F shifted (u,0.8u);
            remplis polygone(B,C,G,F) withcolor Cornsilk;
            remplis polygone(E,H,G,F) withcolor 0.7Cornsilk+0.3black;
            remplis polygone(A,B,F,E) withcolor 0.9Cornsilk+0.1black;
            trace A--B--F--E--cycle;
            trace polygone(B,C,G,F);
            trace chemin(G,H,E);
            trace chemin(1/3[A,B],1/3[E,F],1/3[H,G]);
            trace chemin(2/3[A,B],2/3[E,F],2/3[H,G]);
            trace chemin(iso(B,C),iso(G,F),iso(E,H));
            trace chemin(iso(A,E),iso(B,F),iso(C,G));
        \end{Geometrie}
        \hspace*{20mm}
        \begin{Geometrie}
            u:=0.5*u;
            pair A,B,C,D,E,F,G,H,I,J,K,L;
            A=u*(1,1);
            B=u*(5,1);
            C=B shifted (0.5u,0.4u);
            D=C shifted (0.5u,0.4u);
            E=B shifted (0,u);
            G=C shifted (0,2u);
            F=D shifted (0,2u);
            H=A shifted (0,u);
            I=H shifted (0.5u,0.4u);
            J=I shifted (0,u);
            K=J shifted (0.5u,0.4u);
            L=iso(C,G);
            remplis polygone(B,D,F,G,L,E) withcolor Cornsilk;
            remplis polygone(F,G,J,K) withcolor 0.7Cornsilk+0.3black;
            remplis polygone(H,E,L,I) withcolor 0.7Cornsilk+0.3black;
            remplis polygone(J,G,L,I) withcolor 0.9Cornsilk+0.1black;
            remplis polygone(H,E,B,A) withcolor 0.9Cornsilk+0.1black;            
            trace polygone(B,D,F,G,L,E);
            trace chemin(F,K,J,G);
            trace chemin(E,H,I,L);
            trace chemin(H,A,B);
            trace I--J;
            trace chemin(1/4[A,B],1/4[H,E],1/4[I,L],1/4[J,G],1/4[K,F]);
            trace chemin(2/4[A,B],2/4[H,E],2/4[I,L],2/4[J,G],2/4[K,F]);
            trace chemin(3/4[A,B],3/4[H,E],3/4[I,L],3/4[J,G],3/4[K,F]);
            trace chemin(C,L,iso(D,F));
        \end{Geometrie}
    \end{center}
\end{remarque}