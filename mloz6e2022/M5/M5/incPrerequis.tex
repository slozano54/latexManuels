\vspace*{-8mm}
%pre-001
\begin{prerequis}[Connaisances \emoji{red-heart} et compétences \emoji{diamond-suit} du cycle 3]    
   \begin{itemize}        
       \item[\emoji{red-heart}] Vocabulaire associé à ces objets et à leurs propriétés : côté, sommet, angle, hauteur.
       \columnbreak
       \item[\emoji{diamond-suit}] Reconnaître, nommer, décrire des triangles, dont les triangles particuliers (triangle rectangle, triangle isocèle, triangle équilatéral).       
   \end{itemize}
\end{prerequis}
\vspace*{-5mm}
\begin{debat}[Débat : {\Capa[L]{1}} = {\Vol[dm]{1}}]
    Cette correspondance est à connaître. Pourtant, elle ne paraît pas si naturelle que cela : elle signifie que l'eau contenue dans une bouteille d'un litre remplirait exactement un cube de \Lg[dm]{1} de côté.
    \begin{center}
       \begin{pspicture}(0,-0.25)(6,5.5)
          \psline(0.5,0.5)(0.5,3.75) %bouteille
          \psline(2,0.5)(2,3.75)
          \psarc(1.5,3.75){0.5}{0}{90}
          \psarc(1,3.75){0.5}{90}{180}
          \psline(1,4.25)(1,5)
          \psline(1.5,4.25)(1.5,5)
          \psellipticarc(1.25,0.5)(0.75,0.3){180}{0}
          \psellipticarc(1.25,3.5)(0.75,0.3){180}{0}
          \psellipticarc[linestyle=dashed](1.25,3.5)(0.75,0.3){0}{180}
          \psellipse(1.25,5)(0.25,0.1)
          \rput(1.25,2){\textcolor{B1}{\Capa[L]{1}}}
          \psframe(3,0.5)(5,2.5) %cube
          \psline(3,2.5)(3.75,3.25)(5.75,3.25)(5.75,1.25)(5,0.5)
          \psline(5,2.5)(5.75,3.25)
          \rput(4,0.2){\textcolor{B1}{$\Lg[dm]{1} =\Lg[cm]{10}$}}
          \rput(4,1.5){\textcolor{B1}{$\Vol[dm]{1}$}}
       \end{pspicture}
    \end{center}    
    \begin{cadre}[B2][F4]
       \begin{center}
          \hrefVideo{https://www.yout-ube.com/watch?v=DRKmlWtUN0k}{\bf Correspondance entre unités de volume et de capacité}
          
          chaîne YouTube de {\it Jean-Charles Toussaint}.
       \end{center}
    \end{cadre}
 \end{debat}