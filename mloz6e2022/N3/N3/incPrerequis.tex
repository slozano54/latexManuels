%pre-001
\begin{prerequis}[Connaisances \emoji{red-heart} et compétences \emoji{diamond-suit} du cycle 3]    
   \begin{itemize}        
       \item[\emoji{red-heart}] Vocabulaire associé à ces objets et à leurs propriétés : côté, sommet, angle, hauteur.
       \columnbreak
       \item[\emoji{diamond-suit}] Reconnaître, nommer, décrire des triangles, dont les triangles particuliers (triangle rectangle, triangle isocèle, triangle équilatéral).       
   \end{itemize}
\end{prerequis}

\vfill

\begin{debat}[Débat : vrai ou faux ?]
   \begin{enumerate}
      \item $4,26<4,249$ \hfill vrai \psframe(0,0)(0.25,0.25) \qquad faux \psframe(0,0)(0.25,0.25) \qquad \textcolor{white}{espace}
      \item $1,4 =1,40$ \hfill vrai \psframe(0,0)(0.25,0.25) \qquad faux \psframe(0,0)(0.25,0.25) \qquad \textcolor{white}{espace}
      \item 2,47 est le successeur de 2,46 \hfill vrai \psframe(0,0)(0.25,0.25) \qquad faux \psframe(0,0)(0.25,0.25) \qquad \textcolor{white}{espace}
      \item $2,3+7,12 =9,15$ \hfill vrai \psframe(0,0)(0.25,0.25) \qquad faux \psframe(0,0)(0.25,0.25) \qquad \textcolor{white}{espace}
      \item Entre 5 et 8, il y a une infinité de nombres décimaux \hfill vrai \psframe(0,0)(0.25,0.25) \qquad faux \psframe(0,0)(0.25,0.25) \qquad \textcolor{white}{espace}
   \end{enumerate}
   \bigskip
   \begin{center}
      \textcolor{B1}{\huge $1,11111>1,1111>1,111>1,11>1,1>1$}
   \end{center}
   \bigskip
   \begin{cadre}[B2][F4]
      \begin{center}
         Vidéo : \href{https://lesfondamentaux.reseau-canope.fr/discipline/mathematiques/nombres/comparer-les-decimaux/ranger-des-nombres.html}{\bf Ranger des nombres} du site {\it Canopé}, épisode de la série {\it Les fondamentaux}
      \end{center}
   \end{cadre}
\end{debat}