% Les enigmes ne sont pas numérotées par défaut donc il faut ajouter manuellement la numérotation
% si on veut mettre plusieurs enigmes
\refstepcounter{exercice}
\phantom{\numeroteEnigme}
\begin{enigme}[{\LARGE Nombres croisés}]    
   \partie[mode d'emploi]
    %   Une grille de nombres croisés se comporte comme une grille de mots croisés :
      \begin{itemize}
         \item horizontalement, les lignes sont repérées par des chiffres romains de I à V ;
         \item verticalement, les colonnes sont repérées par des lettres de A à E ;
         \item les indications des nombres à trouver sont indiquées sous la grille de nombres croisés, repérées par des chiffres romains ou des lettres selon s'ils sont horizontaux ou verticaux ;
         \item il est interdit d'écrire dans les cases noires ;
         \item si une ligne ou une colonne ne possède pas de carré noir, il y a un seul nombre à écrire ;
         \item si une ligne ou colonne possède un carré noir, deux nombres sont à écrire, ils sont définis par deux lignes différentes. \\
      \end{itemize}
   
   \partie[let's go !]
Compléter la grille à l'aide de nombres entiers correspondant aux définitions.
   \begin{center}
      {\psset{unit=1.2}\begin{pspicture}(-1,-0.5)(5,5.8)
         \psgrid[gridlabels=0,subgriddiv=0](0,0)(5,5)
         \psframe[fillstyle=solid,fillcolor=black](0,0)(1,1)
         \psframe[fillstyle=solid,fillcolor=black](4,1)(5,2)
         \psframe[fillstyle=solid,fillcolor=black](1,2)(2,3)
         \psframe[fillstyle=solid,fillcolor=black](3,4)(4,5)
         \rput(0.5,5.4){\bf A}
         \rput(1.5,5.4){\bf B}
         \rput(2.5,5.4){\bf C}
         \rput(3.5,5.4){\bf D}
         \rput(4.5,5.4){\bf E}
         \rput(-0.4,0.5){\bf V}
         \rput(-0.4,1.5){\bf IV}
         \rput(-0.4,2.5){\bf III}
         \rput(-0.4,3.5){\bf II}
         \rput(-0.4,4.5){\bf I}
      \end{pspicture}}
   \end{center}   
   \begin{multicols}{2}        
        \begin{itemize}
            \begin{spacing}{1.2}
            \item[] {\bf Horizontalement} 
            \begin{itemize}
                \item[] {\bf I} : Partie entière de 328,54.
                \item[] \phantom{{\bf I} : }Chiffre des centièmes de 634,152.
                \item[] {\bf II} : Son chiffre des dizaines est le triple
                \item[] \phantom{{\bf II} : }de celui des unités.
                \item[] {\bf III} : Chiffre des dixièmes de 34.
                \item[] \phantom{{\bf III} : }Entier précédant 178,356.
                \item[] {\bf IV} : Entier compris entre 8\,000 et 9\,000.
                \item[] {\bf V} : Quarante-deux centaines.
            \end{itemize}
            \columnbreak
            \item[] {\bf Verticalement}
                \begin{itemize}
                    \item[] {\bf A} : $(3\times1 000) + (5\times100) + (8\times1)$.
                    \item[] {\bf B} : Nombre de dixièmes dans 2,6.
                    \item[] \phantom{{\bf B} : }Partie entière de $\dfrac{2\,498}{100}$.
                    \item[] {\bf C} : Quatre-vingt-six milliers et cent-deux unités.
                    \item[] {\bf D} : En additionnant tous les chiffres
                    \item[] \phantom{{\bf D} : }du nombre, on trouve 20.
                    \item[] {\bf E} : Entier qui suit 537,56.
                    \item[] \phantom{{\bf E} : }Entier qui précède 1.
                \end{itemize}
            \end{spacing}
        \end{itemize}
     \end{multicols}
\end{enigme}
% Pour le corrigé, il faut décrémenter le compteur, sinon il est incrémenté deux fois
\addtocounter{exercice}{-1}
\begin{corrige}
    \phantom{rrr}

    \begin{center}
        {\psset{unit=1.2}\begin{pspicture}(-1,-0.5)(5,5.8)
           \psgrid[gridlabels=0,subgriddiv=0](0,0)(5,5)
           \psframe[fillstyle=solid,fillcolor=black](0,0)(1,1)
           \psframe[fillstyle=solid,fillcolor=black](4,1)(5,2)
           \psframe[fillstyle=solid,fillcolor=black](1,2)(2,3)
           \psframe[fillstyle=solid,fillcolor=black](3,4)(4,5)
           \rput(0.5,5.4){\bf A}
           \rput(1.5,5.4){\bf B}
           \rput(2.5,5.4){\bf C}
           \rput(3.5,5.4){\bf D}
           \rput(4.5,5.4){\bf E}
           \rput(-0.4,0.5){\bf V}
           \rput(-0.4,1.5){\bf IV}
           \rput(-0.4,2.5){\bf III}
           \rput(-0.4,3.5){\bf II}
           \rput(-0.4,4.5){\bf I}
           
           \rput(0.5,4.5){\red $\mathbf{3}$}
           \rput(1.5,4.5){\red $\mathbf{2}$}
           \rput(2.5,4.5){\red $\mathbf{8}$}
           \rput(4.5,4.5){\red $\mathbf{5}$}

           \rput(0.5,3.5){\red $\mathbf{5}$}
           \rput(1.5,3.5){\red $\mathbf{6}$}
           \rput(2.5,3.5){\red $\mathbf{6}$}
           \rput(3.5,3.5){\red $\mathbf{9}$}
           \rput(4.5,3.5){\red $\mathbf{3}$}

           \rput(0.5,2.5){\red $\mathbf{0}$}           
           \rput(2.5,2.5){\red $\mathbf{1}$}
           \rput(3.5,2.5){\red $\mathbf{7}$}
           \rput(4.5,2.5){\red $\mathbf{8}$}

           \rput(0.5,1.5){\red $\mathbf{8}$}
           \rput(1.5,1.5){\red $\mathbf{2}$}
           \rput(2.5,1.5){\red $\mathbf{0}$}
           \rput(3.5,1.5){\red $\mathbf{4}$}
           
           \rput(1.5,0.5){\red $\mathbf{4}$}
           \rput(2.5,0.5){\red $\mathbf{2}$}
           \rput(3.5,0.5){\red $\mathbf{0}$}
           \rput(4.5,0.5){\red $\mathbf{0}$}
        \end{pspicture}}
     \end{center}
\end{corrige}