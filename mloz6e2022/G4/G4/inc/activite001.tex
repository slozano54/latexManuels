\begin{activite}[Avec des allumettes]
    {\bf Objectifs} : construire puis nommer des triangles. 
       Devant vous, vous avez douze allumettes. Pour chacune des questions suivantes, faire la construction si elle est possible avec des allumettes puis faire un dessin pour schématiser la situation. \\
       \vspace*{-7mm}
       \begin{enumerate}
          \item Aligner cinq allumettes en les plaçant les unes à côté des autres.
          \smallskip
          \begin{center}
             %\includegraphics[width=2.5cm]{allumette}\includegraphics[width=2.5cm]{\currentpath/images/allumette}\includegraphics[width=2.5cm]{allumette}\includegraphics[width=2.5cm]{allumette}\includegraphics[width=2.5cm]{allumette}
             \multido{}{5}{\includegraphics[width=2.5cm]{\currentpath/images/allumette}}
          \end{center}
          \smallskip
          À partir de ce segment de longueur cinq allumettes et en utilisant les douze allumettes, combien peut-on construire de triangles différents ? Ces triangles sont-ils particuliers ? \\ [4cm]
          \item Mêmes questions si l'on place quatre allumettes côte à côte. \\ [3cm]
          \item Mêmes questions si l'on place six allumettes côte à côte. \\ [3cm]
          \item Mêmes questions si l'on place sept allumettes côte à côte ou plus. \\ [3cm]
       \end{enumerate}
 \end{activite}