\section{Construction d'un triangle connaissant trois longueurs}

\begin{methode*1}
   Pour construire un triangle $ABC$ dont on connaît les longueurs des trois côtés :
   \begin{itemize}
      \item on trace à la règle graduée l'un des côtés (en général le plus grand), par exemple $[AB]$ ;
      \item on trace un arc de cercle de centre $A$ et de rayon $AC$ ;
      \item on trace un arc de cercle de centre $B$ et de rayon $BC$ ;
      \item le point $C$ se situe à l'intersection des deux arcs de cercle.
   \end{itemize}
   \exercice
   Tracer le triangle $ABC$ tel que : $AB =\ucm{3,5}$ ; $BC =\ucm{2,2}$ et $CA =\ucm{3,2}$.
   %\correction
   \small
   \psset{unit=0.7}
   \begin{pspicture}(-1.5,-1.5)(4.8,3.5)
      \pstGeonode[PointSymbol=none,PosAngle={225,-45}](0,0){A}(3.5,0){B}
      \pstLineAB{A}{B}
      \rput(1.75,-0.25){3,5 cm}
   \end{pspicture}
   \begin{pspicture}(0,-1.5)(4.8,3.5)
      \pstGeonode[PointSymbol=none,PosAngle={225,-45}](0,0){A}(3.5,0){B}
      \pstLineAB{A}{B}
      \psset{linecolor=A1}
      \psarc(0,0){3.2}{30}{60}
      \rput{55}(0.8,1.3){\textcolor{A1}{3,2 cm}}
      \compas{1.5}{1.2}{55}{0.9}{34.5}
      \end{pspicture}
   \begin{pspicture}(0,-1.5)(4.8,3.5)
      \pstGeonode[PointSymbol=none,PosAngle={225,-45}](0,0){A}(3.5,0){B}
      \pstLineAB{A}{B}
      \psset{linecolor=A1}
      \psarc(0,0){3.2}{30}{60}
      \psset{linecolor=B1}
      \psarc[fillstyle=none](3.5,0){2.2}{90}{130}
      \rput{-45}(2.8,0.8){\textcolor{B1}{2,2 cm}}
      \compas{3.3}{1.1}{130}{0.9}{23}
   \end{pspicture}
   \begin{pspicture}(0,-1.5)(2,3.5)
      \pstGeonode[CurveType=polygon,PointSymbol=none,PosAngle={225,90,-45}](0,0){A}(2.5,2){C}(3.5,0){B}
   \end{pspicture}

   \hrefConstruction{http://lozano.maths.free.fr/iep_local/figures_html/scr_iep_046.html}{Construire un triangle connaissant trois côtés}
   \creditInstrumentPoche
\end{methode*1}

\begin{remarque}
   on a deux choix de construction pour $C$, d'un côté ou de l'autre du segment.
\end{remarque}