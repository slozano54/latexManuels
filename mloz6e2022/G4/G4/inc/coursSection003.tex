\section{Compléments numériques}

\begin{list}{}{}   
   \begin{multicols}{2}
      \item \hrefConstruction{http://lozano.maths.free.fr/iep_local/figures_html/scr_iep_048.html}{Triangle - trois segments}
      \item \hrefConstruction{http://lozano.maths.free.fr/iep_local/figures_html/scr_iep_046.html}{Triangle - trois côtés }
      \item \hrefConstruction{http://lozano.maths.free.fr/iep_local/figures_html/scr_iep_037.html}{Triangle isocèle}
      \item \hrefConstruction{http://lozano.maths.free.fr/iep_local/figures_html/scr_iep_050.html}{Triangle équilatéral (compas) }
   \end{multicols}
   \item \hrefConstruction{http://lozano.maths.free.fr/iep_local/figures_html/scr_iep_049.html}{Triangle équilatéral inscrit dans un cercle (angle de 120°)}
   \item \hrefConstruction{http://lozano.maths.free.fr/iep_local/figures_html/scr_iep_056.html}{Triangle rectangle à partir de l'hypoténuse et d'un autre côté}
   \item \hrefConstruction{http://lozano.maths.free.fr/iep_local/figures_html/scr_iep_055.html}{Triangle rectangle à partir des côtés perpendiculaires}
   \item \hrefConstruction{http://lozano.maths.free.fr/iep_local/figures_html/scr_iep_042.html}{Longueur égale au périmètre d'un triangle (compas)}
\end{list}

\creditInstrumentPoche
