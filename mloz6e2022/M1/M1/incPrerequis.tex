\vspace*{-5mm}
%pre-001
\begin{prerequis}[Connaisances \emoji{red-heart} et compétences \emoji{diamond-suit} du cycle 3]    
   \begin{itemize}        
       \item[\emoji{red-heart}] Vocabulaire associé à ces objets et à leurs propriétés : côté, sommet, angle, hauteur.
       \columnbreak
       \item[\emoji{diamond-suit}] Reconnaître, nommer, décrire des triangles, dont les triangles particuliers (triangle rectangle, triangle isocèle, triangle équilatéral).       
   \end{itemize}
\end{prerequis}

\vfill

\begin{debat}[Débat : le compas, un instrument de report de mesures]
    Étymologiquement le mot {\bf compas} vient du latin {\it compassare} signifiant - qui partage le même pas, la même mesure - et qui fait donc référence à un instrument qui mesure et non qui trace des cercles.
    \flushright{\it\footnotesize Source : https://compas-passion.jimdo.com}. \\
    \begin{center}
        \small
        \includegraphics[height=4cm]{\currentpath/images/compas_simple.eps}
        \qquad
        \includegraphics[height=4cm]{\currentpath/images/compas_secteur.eps}
        \qquad
        \includegraphics[height=4cm]{\currentpath/images/compas_epaisseur.eps}
        \qquad
        \includegraphics[height=4cm]{\currentpath/images/compas_3jambes.eps} \\
        Compas simple \quad Compas à secteur courbe \; Compas d'épaisseur \quad Compas à trois jambes
    \end{center}
    \vfill
    \begin{cadre}[B2][F4]
        \begin{center}
            Vidéo : \href{https://www.youtube.com/watch?v=gXOW4e708Hs&vl=fr}{\bf Dessin d'un cercle parfait sans outils}, chaîne YouTube de {\it Rajarts}.      \end{center}
    \end{cadre}
\end{debat}