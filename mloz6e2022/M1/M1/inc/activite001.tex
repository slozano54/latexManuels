\begin{activite}[L'art de mesurer]
    {\bf Objectifs} : mesurer une longueur et le périmètre d'un polygone avec un étalon et une règle ; comparer des méthodes, communiquer.
    \partie[mesurer grâce à un étalon]
        On considère la bandelette en bas de page. La découper puis donner en fonction de cette bande la mesure de la longueur d'un stylo,
        le périmètre d'une feuille de papier A4 et le périmètre de votre table. \\
        Écrire les résultats dans le tableau ci-dessous et comparer avec les autres élèves de la classe. \\
    \vspace*{-5mm}    
    \partie[créer une graduation]
        Avec la bandelette entière, il est difficile d'évaluer correctement les mesures demandées. Plier la bande en dix parties égales : une partie correspond donc à un dixième de bandelette. Mesurer de nouveau les trois objets cités en nombre de dixièmes de bandelette. \\
        Écrire les résultats dans le tableau ci-dessous et comparer avec les autres élèves de la classe. \\
    \vspace*{-5mm}    
    \partie[mesurer à l'aide d'une règle graduée]
        Enfin, refaire ces mesures à l'aide d'une règle graduée. \\
        Écrire les résultats dans le tableau ci-dessous et comparer avec les autres élèves de la classe. \\
        Débattre des avantages et inconvénients de chacun des outils. \\
    \vspace*{-5mm}
        \begin{center}
            {\renewcommand{\arraystretch}{1.5}
            \small
            \begin{tabular}{|c|>{\centering\arraybackslash}p{4cm}|>{\centering\arraybackslash}p{4cm}|>{\centering\arraybackslash}p{4cm}|}
                \cline{2-4}
                \multicolumn{1}{c|}{} & Longueur du stylo & Périmètre d'une feuille format A4 & Périmètre de la table \\
                \cline{2-4}
                \multicolumn{1}{c|}{} & \begin{pspicture}(0,0.3)(4,4)
                    \pspolygon(1.8,0.5)(2.2,0.5)(2.2,3)(2,3.4)(1.8,3)
                    \psarc(2,3.4){0.06}{-120}{-70}
                    \psline(1.8,3)(1.9,2.9)(2,3)(2.1,2.9)(2.2,3)
                    \psset{linecolor=lightgray}
                    \psline(1.9,0.5)(1.9,2.9)
                    \psline(2,0.5)(2,2.98)
                    \psline(2.1,0.5)(2.1,2.9)
                \end{pspicture}
                & \begin{pspicture}(0,0.3)(4,4)
                    \psframe(1,0.5)(3,3.5)
                    \rput(2,2){\begin{minipage}{1.5cm} \footnotesize Feuille\\de\\papier \end{minipage}}
                    \rput(2.5,3){\large A4}
                \end{pspicture}
                & \begin{pspicture}(0,0.3)(4,4)
                        \psline(0.5,0.7)(0.5,2.5)(1.5,3.3)(3.5,3.3)(2.5,2.5)(2.5,0.7)
                        \psline(3.5,1.7)(3.5,3.3)
                        \psline(0.5,2.5)(2.5,2.5)
                        \psline(0.5,2.3)(2.5,2.3)(3.5,3.1)
                        \psline(1.5,1.7)(1.5,2.3)
                    \end{pspicture} \\
                \hline
                Avec l'étalon & & & \\
                \hline
                Etalon gradué & & & \\
                \hline
                Règle graduée & & & \\
                \hline
            \end{tabular}}
        \end{center}
    \vfill
    Bandelette à découper :
    \begin{center}
       \begin{pspicture}(0,0)(15,1.5)
          \psframe(0,0)(15,1.5)
      \end{pspicture}
   \end{center}
 \end{activite}