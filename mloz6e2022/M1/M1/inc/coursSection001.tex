\section{Section 1}
\subsection{Sous-section 1.1}
\begin{definition}[Titre optionnel]
    Dans le cours, on utilise assez souvent des cadres du type
    définition (comme ici par exemple).    
\end{definition}
\begin{remarque}
    Ceci est une remarque utilisant une commande du paquet profcollege.
    Les formules peuvent être positionnées partout dans la page.
    Ici c'est placé juste pour tester !

    \Formule[Aire,Surface=triangle,Ancre={([xshift=-3cm,yshift=-3cm]current page.east)}]
  
    \Formule[Aire,Surface=losange,Ancre={([xshift=15cm,yshift=5cm]current page.south west)},Angle=-20,Largeur=6cm]
    
    \begin{center}
      \definecolor{myyellow}{RGB}{242,226,149}
      % Positionnement relatif de l'ancre.
      \Formule[Couleur=myyellow!15,Perimetre,Surface=cercle,Ancre={([xshift=-4cm,yshift=-3cm]current page.north east)},Angle=-30]
      
      % Positionnement absolu de l'ancre.
      \Formule[Perimetre,Surface=parallelogramme,Ancre={(0,0)}]   
      \par\vspace{3cm}   
    \end{center}


\end{remarque}
\begin{propriete}[Titre optionnel]
  Dans le cours, on utilise assez souvent des cadres du type
  définition, comme ici par exemple pour une propriete.
\end{propriete}
\begin{remarques}
  \begin{itemize}
    \item remarque.
    \item remarque.
  \end{itemize}
\end{remarques}

\subsection{Sous-section 1.2}
\begin{theoreme}[Titre optionnel]
  Dans le cours, on utilise assez souvent des cadres du type
  définition, comme ici par exemple pour un théorème.
\end{theoreme}
\begin{notation}
  notation
\end{notation}
\begin{notations}
  \begin{itemize}
    \item notation.
    \item notation.
  \end{itemize}
\end{notations}
\begin{preuve}
  Ceci est une preuve\par Deuxième ligne de la preuve
\end{preuve}
\begin{exemple}
  Texte de l’exemple
  \correction
  
\end{exemple}

\begin{exemple*1}
  \phantom{rrr}
  Texte

  \correction
  \phantom{rrr}
  Texte  
  
\end{exemple*1}

\begin{exemple}[0.6]
  Texte de l’exemple très long sur une ligne, très très très long.
  On peut modifier la répartition horizontale  à l'aide d'un argument optionnel valant par défaut 0,4, valant ici 0,6.
  \correction
  Texte de la correction en vis à vis
\end{exemple}