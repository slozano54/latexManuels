\section{Périmètre d'un polygone}
\begin{definition}
   Le \textbf{périmètre} d'une figure est la mesure du contour de cette figure.
\end{definition}

\begin{propriete}
   Pour calculer le périmètre d'un polygone, on additionne la mesure de chacun de ses segments.
\end{propriete}
\begin{exemple}[0.5]      
   \phantom{rrr}

   {\psset{unit=0.4}
   \begin{pspicture}(-0.5,0)(11,5.5)
      \psgrid[subgriddiv=0,gridlabels=0pt,gridcolor=gray](11,5)
      \put(1,1){\pspolygon[fillstyle=solid,fillcolor=B2,linewidth=0.1](0,0)(2,0)(2,3)(0,3)(0,2)(1,2)(1,1)(0,1)(0,0)}
      \put(4,1){\pspolygon[fillstyle=solid,fillcolor=A2,linewidth=0.1](0,0)(2,0)(2,2)(1,2)(1,3)(0,3)(0,0)}
      \put(7,1){\pspolygon[fillstyle=solid,fillcolor=J2,linewidth=0.1](1,0)(2,0)(2,1)(3,1)(3,2)(2,2)(2,3)(1,3)(1,2)(0,2)(0,1)(1,1)(1,0)}
      \rput(2.5,2.4){\textbf{A}}
      \rput(5,2.4){\textbf{B}}
      \rput(8.5,2.4){\textbf{C}}
      \psline[linewidth=0.4mm]{|-|}(6,4)(7,4)
      \rput(6.5,4.6){\small$u.\ell$.}
   \end{pspicture}}
\correction
   $u.\ell$ désigne l'unité de longueur, les périmètres sont :
   \begin{itemize}
      \item pour la figure A : 12 $u.\ell$;   
      \item pour la figure B : 10 $u.\ell$;
      \item pour la figure C : 12 $u.\ell$
   \end{itemize} 
\end{exemple}