%pre-001
\begin{prerequis}[Connaisances \emoji{red-heart} et compétences \emoji{diamond-suit} du cycle 3]    
   \begin{itemize}        
       \item[\emoji{red-heart}] Vocabulaire associé à ces objets et à leurs propriétés : côté, sommet, angle, hauteur.
       \columnbreak
       \item[\emoji{diamond-suit}] Reconnaître, nommer, décrire des triangles, dont les triangles particuliers (triangle rectangle, triangle isocèle, triangle équilatéral).       
   \end{itemize}
\end{prerequis}

\vfill

\begin{debat}[un peu de vocabulaire] 
    {\bf Perpendiculaire} : chez les Romains, {\it perpendiculum} désigne le fil à plomb ; en ancien français, un {\it perpendicle} est aussi un fil à plomb mais signifie également verticale et ce n'est  que dans la deuxième moitié du {\small XVI}\up{e} siècle que le mot {\it perpendiculaire} prend le sens que nous connaissons actuellement. \\
    {\bf Parallèle} : chez les Grecs, {\it parallelos} signifie \og placé en regard \fg{} et désigne aussi des cercles concentriques. Le mot est formé à partir de {\it para}, \og à côté \fg{} et de {\it allelon}, \og les uns et les autres \fg{}. Le mot parallèle est introduit au {\small XVI}\up{e} siècle dans le vocabulaire mathématique.
    {\centerline{\footnotesize\it Source : Les mots et les maths, Bertrand Hauchecorne, ellipses poche 2014.}}
    \begin{center}
       \begin{pspicture}(0,-0.5)(5,2.5)
          \psset{linecolor=B1,linewidth=1mm}
          \psline(0,0)(2,0)
          \psline(1,0)(1,2)
          \psline(3.5,0)(4.5,2)
          \psline(4,0)(5,2)
       \end{pspicture}
    \end{center}
    \begin{cadre}[B2][F4]
       \begin{center}
          \hrefVideo{https://lesfondamentaux.reseau-canope.fr/video/reconnaitre-des-droites-perpendiculaires.html}{\bf Reconnaître des droites perpendiculaires}

          \vspace*{2mm}
          \hrefVideo{https://lesfondamentaux.reseau-canope.fr/video/reconnaitre-des-droites-paralleles.html}{\bf Reconnaître des droites parallèles} 

          \vspace*{2mm}         
          Site {\it Canopé}, épisode de la série {\it Les fondamentaux}.
       \end{center}
    \end{cadre}
 \end{debat}