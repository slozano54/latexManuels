% Les enigmes ne sont pas numérotées par défaut donc il faut ajouter manuellement la numérotation
% si on veut mettre plusieurs enigmes
% \refstepcounter{exercice}
% \numeroteEnigme
\vspace*{-10mm}
\enigme[La grenouille qui saute]
   Prendre une feuille dont la longueur est deux fois plus longue que la largeur. \\
   Les plis en courts pointillés sont des plis \og vallée \fg, les plis en longs pointillés sont des plis \og montagne \fg. \\
   {\small
   \psset{linecolor=PartieStatistique}
   \begin{tabular}{p{3.8cm}p{3.8cm}p{3.8cm}p{3.8cm}}
      \begin{pspicture}(0.1,-0.25)(3.9,4.5)
         \psframe(1,0)(3,4)
         \psline[linestyle=dotted](1,2)(3,2)
         \rput(0.7,2){\textcolor{PartieStatistique}{A}}
         \rput(3.3,2){\textcolor{PartieStatistique}{B}}
      \end{pspicture}
      &
      \begin{pspicture}(0.1,-0.25)(3.9,4.5)
         \psframe(1,0)(3,4)
         \psline[linestyle=dotted](1,2)(3,2)
         \psline[linestyle=dotted](1,2)(3,4)
         \rput(0.7,2){\textcolor{PartieStatistique}{A}}
         \rput(3.3,2){\textcolor{PartieStatistique}{B}}
      \end{pspicture}
      &
      \begin{pspicture}(0.1,-0.25)(3.9,4.5)
         \psframe(1,0)(3,4)
         \psline[linestyle=dotted](1,2)(3,2)
         \psline[linestyle=dotted](1,2)(3,4)
         \psline[linestyle=dotted](1,4)(3,2)
         \psline[linestyle=dotted](1,0)(3,2)
         \psline[linestyle=dotted](1,2)(3,0)
          \rput(0.7,2){\textcolor{PartieStatistique}{A}}
         \rput(3.3,2){\textcolor{PartieStatistique}{B}}
      \end{pspicture}
      &
      \begin{pspicture}(0.1,-0.25)(3.9,4.5)
         \psframe(1,0)(3,4)
         \psline[linestyle=dotted](1,2)(3,2)
         \psline[linestyle=dotted](1,2)(3,4)
         \psline[linestyle=dotted](1,4)(3,2)
         \psline[linestyle=dotted](1,0)(3,2)
         \psline[linestyle=dotted](1,2)(3,0)
         \psline[linestyle=dashed](1,3)(3,3)
         \psline[linestyle=dashed](1,1)(3,1)
         \rput(0.7,3){\textcolor{PartieStatistique}{C}}
         \rput(3.3,3){\textcolor{PartieStatistique}{D}}
         \rput(0.7,2){\textcolor{PartieStatistique}{A}}
         \rput(3.3,2){\textcolor{PartieStatistique}{B}}
         \rput(0.7,1){\textcolor{PartieStatistique}{E}}
         \rput(3.3,1){\textcolor{PartieStatistique}{F}}
      \end{pspicture} \\
      1) Plier la feuille en deux selon la médiane la plus courte du rectangle : (AB)
      &
      2) Dans le carré du haut, plier suivant la diagonale passant par A puis déplier
      &
      3) Plier puis déplier de la même manière les trois autres diagonales
      &
      4) Plier suivant les médianes (CD) et (EF) des deux carrés \\
      \begin{pspicture}(0.1,-0.25)(3.9,3.5)
         \psframe(1,0)(3,2)
         \psline[linestyle=dotted](1,0)(3,2)
         \psline[linestyle=dotted](1,2)(3,0)
         \psline[linestyle=dashed](1,1)(3,1)
         \rput(0.7,2){\textcolor{PartieStatistique}{\footnotesize A}}
         \rput(3.3,2){\textcolor{PartieStatistique}{\footnotesize B}}
         \rput(0.7,1){\textcolor{PartieStatistique}{\footnotesize E}}
         \rput(3.3,1){\textcolor{PartieStatistique}{\footnotesize F}}      
         \psline(3,2)(2,3)(1,2)
         \psdot(2,2)
         \rput(2,1.7){\textcolor{PartieStatistique}{\footnotesize C}}
         \rput(2,2.3){\textcolor{PartieStatistique}{\footnotesize D}}    
      \end{pspicture}
      &
      \begin{pspicture}(0.1,-0.25)(3.9,3.5)
         \pspolygon(2,1)(3,2)(2,3)(1,2)
         \psline(1,2)(3,2)
         \psdot(2,2)
         \rput(1.85,1.8){\textcolor{PartieStatistique}{\footnotesize C}}
         \rput(1.85,2.2){\textcolor{PartieStatistique}{\footnotesize D}}
         \rput(2.15,1.8){\textcolor{PartieStatistique}{\footnotesize E}}
         \rput(2.15,2.2){\textcolor{PartieStatistique}{\footnotesize F}}
         \rput(0.8,2){\textcolor{PartieStatistique}{\footnotesize A}}
         \rput(3.2,2){\textcolor{PartieStatistique}{\footnotesize B}}    
      \end{pspicture}
      &
      \begin{pspicture}(0.1,-0.25)(3.9,3.5)
         \pspolygon(2,1)(3,2)(2,3)(1,2)
         \psline(1,2)(3,2)
         \psline[linestyle=dotted](2,2)(2.5,2.5)
         \rput(0.8,2){\textcolor{PartieStatistique}{\footnotesize A}}
         \rput(3.2,2){\textcolor{PartieStatistique}{\footnotesize B}}
         \rput(2,3.2){\textcolor{PartieStatistique}{\footnotesize G}}
         \psarc[linecolor=cyan]{->}(2.6,2.6){0.35}{-45}{135}
     \end{pspicture}
      &
      \begin{pspicture}(0.1,-0.25)(3.9,3.5)
         \pspolygon(2,1)(3,2)(2,3)(1,2)
         \psline(1,2)(3,2)
         \psline(2.5,1.5)(1.5,2.5)
         \psline(1.5,1.5)(2.5,2.5)
         \psline(2,1)(2,3)
      \end{pspicture} \\
      5) Joindre les points C et D l'un contre l'autre et les rabattre sur la droite (AB)
      &
      6) Faire la même chose avec les points E et F
      &
      7) Rabattre le coin B supérieur droit sur le sommet G
      &
      8) Faire de même avec les trois autres coins en A et B \\
      \begin{pspicture}(0.1,0.75)(3.9,3.5)
         \pspolygon(2,1)(3,2)(2,3)(1,2)
         \psline(1,2)(3,2)
         \psline(2.5,1.5)(1.5,2.5)
         \psline(1.5,1.5)(2.5,2.5)
         \psline(2,1)(2,3)
         \psline[linestyle=dotted](2,2)(2.29,2.71)
         \psarc[linecolor=cyan]{<-}(2.35,2.75){0.2}{-40}{130}
      \end{pspicture}
      &
      \begin{pspicture}(0.1,0.75)(3.9,3.5)
         \pspolygon(2,1)(3,2)(2,3)(1,2)
         \psline(1,2)(3,2)
         \pspolygon[fillstyle=solid,fillcolor=white](1.29,1.29)(1.71,1.29)(2.29,2.71)(2.71,2.71)
         \pspolygon[fillstyle=solid,fillcolor=white](1.29,2.71)(1.7,2.71)(2.29,1.29)(2.71,1.29)
     \end{pspicture}
      &
     \begin{pspicture}(0.1,0.75)(3.9,3.5)
         \pspolygon(1.29,1.29)(1.71,1.29)(2.29,2.71)(2.71,2.71)
         \pspolygon(1.29,2.71)(1.7,2.71)(2.29,1.29)(2.71,1.29)
         \pspolygon[fillstyle=solid,fillcolor=white](2,1)(3,2)(2,3)(1,2)
         \psline[linestyle=dotted](2,1)(2,3)
         \psline[linestyle=dashed](1,2)(3,2)
         \rput(2,3.2){\textcolor{PartieStatistique}{\footnotesize G}}
         \rput(2,0.8){\textcolor{PartieStatistique}{\footnotesize H}}
      \end{pspicture}
      &
      \begin{pspicture}(0.1,0.75)(3.9,3.5)
         \pspolygon(1.29,1.29)(1.71,1.29)(2.29,2.71)(2.71,2.71)
         \pspolygon(1.29,2.71)(1.7,2.71)(2.29,1.29)(2.71,1.29)
         \pspolygon[fillstyle=solid,fillcolor=white](2,1)(3,2)(2,3)(1,2)
         \psline[linestyle=dotted](2,1)(2,3)
         \psline[linestyle=dotted](1.42,1.58)(2,3)
         \psline[linestyle=dotted](2.58,1.58)(2,3)
         \rput(2,3.2){\textcolor{PartieStatistique}{\footnotesize G}}
         \rput(2,0.8){\textcolor{PartieStatistique}{\footnotesize H}}
         \rput(0.8,2){\textcolor{PartieStatistique}{\footnotesize B}}
         \rput(3.2,2){\textcolor{PartieStatistique}{\footnotesize A}}
         \psarc[linecolor=cyan]{<-}(2.5,2.1){0.35}{200}{-20}
         \psarc[linecolor=cyan]{->}(1.5,2.1){0.35}{200}{-20}
      \end{pspicture} \\
      9) Replier le triangle supérieur droit de manière à superposer son hypoténuse à son côté inférieur
      &
      10) Faire la même chose avec les trois autres triangles
      &
      11) Retourner la feuille, plier puis déplier suivant la diagonale (GH)
      &
      12) Amener les points A et B sur la diagonale (GH) \\ 
       \begin{pspicture}(0.1,0.75)(3.9,3.5)
         \pspolygon(1.29,1.29)(1.71,1.29)(2.29,2.71)(2.71,2.71)
         \pspolygon(1.29,2.71)(1.7,2.71)(2.29,1.29)(2.71,1.29)
         \pspolygon[fillstyle=solid,fillcolor=white](2,2)(2.5,2.5)(2,3)(1.5,2.5)
         \pspolygon[fillstyle=solid,fillcolor=white](2,1)(2.58,1.58)(2,3)(1.42,1.58)
         \psline[linestyle=dotted](2,1)(2,1.58)
         \psline(2,1.58)(2,3)
         \psline(1.42,1.58)(2.58,1.58)
         \psline(1.58,2)(2,1.58)(2.42,2)
         \psline[linestyle=dotted](1.5,1.48)(2.5,1.48)
         \psarc[linecolor=cyan]{<-}(1.9,1.58){0.5}{80}{-80}
      \end{pspicture}
      &
      \begin{pspicture}(0.1,0.75)(3.9,3.5)
         \pspolygon(1.29,1.29)(1.71,1.29)(2.29,2.71)(2.71,2.71)
         \pspolygon(1.29,2.71)(1.7,2.71)(2.29,1.29)(2.71,1.29)
         \pspolygon[fillstyle=solid,fillcolor=white](2,2)(2.5,2.5)(2,3)(1.5,2.5)
         \pspolygon[fillstyle=solid,fillcolor=white](1.5,1.5)(2.5,1.5)(2.58,1.58)(2,3)(1.42,1.58)
         \psline(2,1.58)(2,3)
         \psline(1.42,1.58)(2.58,1.58)
         \psline(1.58,2)(2,1.58)(2.42,2)
         \pspolygon[fillstyle=solid,fillcolor=white](1.5,1.5)(2.5,1.5)(2,2)
         \psline(1.71,1.29)(1.5,1.5)
         \psline(2.29,1.29)(2.5,1.5)
         \psline[linecolor=cyan]{->}(1.7,2.1)(1.95,1.7)
         \psline[linecolor=cyan]{->}(2.3,2.1)(2.05,1.7)
      \end{pspicture}
      &
      \begin{pspicture}(0.1,0.75)(3.9,3.5)
         \pspolygon(1.29,1.29)(1.71,1.29)(2.29,2.71)(2.71,2.71)
         \pspolygon(1.29,2.71)(1.7,2.71)(2.29,1.29)(2.71,1.29)
         \pspolygon[fillstyle=solid,fillcolor=white](2,2)(2.5,2.5)(2,3)(1.5,2.5)
         \pspolygon[fillstyle=solid,fillcolor=white](1.5,1.5)(2.5,1.5)(2.58,1.58)(2,3)(1.42,1.58)
         \psline(2,1.58)(2,3)
         \psline(1.42,1.58)(2.58,1.58)
         \psline(1.58,2)(2,1.58)(2.42,2)
         \pspolygon[fillstyle=solid,fillcolor=white](1.5,1.5)(2.5,1.5)(2,2)
         \psline(1.71,1.29)(1.5,1.5)
         \psline(2.29,1.29)(2.5,1.5)
        \psline[linestyle=dashed](1.58,2)(2.42,2)
         \psline[linestyle=dotted](1.48,1.75)(2.52,1.75)
         \psarc[linecolor=cyan]{<-}(2.9,1.7){0.15}{80}{-80}
         \psarc[linecolor=cyan]{->}(2.7,2.1){0.15}{-100}{100}
      \end{pspicture}
      &
      \begin{pspicture}(0.1,0.75)(3.9,3.5)
         \pspolygon(2,2.05)(2.29,2.71)(2.71,2.71)
         \pspolygon(1.29,2.71)(1.7,2.71)(2,2.05)
         \pspolygon(1.48,2.24)(2.52,2.24)(2.42,2)(1.58,2)
        \pspolygon(1.48,2.24)(1.41,2.08)(1.5,2)(2.5,2)(2.59,2.08)(2.52,2.24)
         \psline(1.41,2.08)(1.54,2.08)
         \psline(2.59,2.08)(2.46,2.08)
         \psline(1.5,2)(1.55,2.06)
         \psline(2.5,2)(2.45,2.06)      
         \pspolygon[fillstyle=solid,fillcolor=white](2,2)(2.5,2.5)(2,3)(1.5,2.5)
         \pspolygon[fillstyle=solid,fillcolor=white](1.58,2)(2,3)(2.42,2)
         \psline(2,2)(2,3)
         \psline(1.79,2)(1.72,1.79)(1.5,2)(1.32,1.79)(1.72,1.79)
         \psline(2.21,2)(2.28,1.79)(2.5,2)(2.68,1.79)(2.28,1.79)  
      \end{pspicture}
       \\
      13) Plier le triangle inférieur au niveau des pattes de derrière
      &
      14) Rentrer les coins dans les encoches du triangle plié
      &
      15) Faire un pli montagne au niveau du sommet du triangle inférieur puis un pli vallée entre ce pli et la base &
      16) On obtient une jolie grenouille\dots{} ou pas ! \\
   \end{tabular}}

   Il ne reste plus qu'à décorer la grenouille ;-)
