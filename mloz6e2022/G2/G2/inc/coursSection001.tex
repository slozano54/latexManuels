\section{Droites perpendiculaires}

\begin{definition}
   Lorsque deux droites forment un angle droit, on dit qu'elles sont \textbf{perpendiculaires}.
\end{definition}

\begin{exemple}
   \begin{pspicture}(-1.5,-0.5)(4,3)
      \psline(0,0)(4,2)
      \psline(2.5,0)(1,3)
      \pspolygon[linecolor=A1](2,1)(2.25,1.125)(2.125,1.375)(1.875,1.25)
      \equerre{2}{1}{116.5}{1}
      \rput(3.8,1.6){$(d)$}
      \rput(1.5,2.9){$(\Delta)$}
   \end{pspicture}
   \correction
      Les deux droites $(d)$ et $(\Delta)$ sont perpendiculaires. \\
      On peut le constater par exemple à l'aide d'une équerre. \\
      On code grâce à un petit carré au niveau de l'angle droit. \\ [5mm]
      On note \fbox{$(d)\perp(\Delta)$}
\end{exemple}
