\section{Droites parallèles}

\begin{definition}
   Lorsque deux droites ne se coupent pas, on dit qu'elles sont \textbf{parallèles}.
\end{definition}

\begin{exemple}
   \begin{pspicture}(-1.5,-0.5)(4,3)
      \psline(0,0)(4,2)
      \psline(-0.5,0.7)(3.5,2.7)
      \rput(3.8,1.4){$(d)$}
      \rput(3.4,2.3){$(d')$}
   \end{pspicture}
   \correction
      Les deux droites $(d)$ et $(d')$ sont parallèles. \\ [5mm]
      On note \fbox{$(d)\sslash(d')$}      
\end{exemple}