\vspace*{-5mm}
\begin{changemargin}{-10mm}{-10mm}
%pre-001
\begin{prerequis}[Connaisances \emoji{red-heart} et compétences \emoji{diamond-suit} du cycle 3]    
   \begin{itemize}        
       \item[\emoji{red-heart}] Vocabulaire associé à ces objets et à leurs propriétés : côté, sommet, angle, hauteur.
       \columnbreak
       \item[\emoji{diamond-suit}] Reconnaître, nommer, décrire des triangles, dont les triangles particuliers (triangle rectangle, triangle isocèle, triangle équilatéral).       
   \end{itemize}
\end{prerequis}
% \end{changemargin}
% \begin{changemargin}{-7mm}{-7mm}
\begin{debat}[Le premier tableur]
   \begin{changemargin}{-10mm}{-10mm}
   Des données brutes récoltées ont souvent peu de sens si elles sont utilisées ainsi, d'où la nécessité de les disposer d'une manière plus lisible à l'aide de tableaux et diagrammes. \\
   Avec l'avènement de l'informatique, les tableaux deviennent numériques grâce à l'apport des {\bf tableurs} : logiciels qui permettent de manipuler des données numériques, d'effectuer un certain nombre d'opérations de façon automatisée, de créer des représentations graphiques à partir des données : diagrammes , histogrammes, courbes\dots{} \\
   Le premier tableur fut créé en 1978 par {\it Daniel Bricklin}, étudiant à Harvard qui devait établir des tableaux comptables pour une étude de cas sur Pepsi-Cola sans pour autant établir tous les calculs \og à la main \fg. Son premier prototype, {\it VisiCalc} (pour Visible Calculator), pouvait manipuler un tableau de vingt lignes et cinq colonnes ! \\
   \end{changemargin}
   \begin{center}
      \vspace*{-5mm}
      \textsf{
      \begin{tabular}{|>{\columncolor{lightgray!30}}c|*{5}{>{\centering\arraybackslash}m{0.15\linewidth}|}}
         \hline
         \rowcolor{lightgray!30} & A & B & C & D & E \\
         \hline
         1 & & & & &  \\
         \hline
         2 & & & & & \\
         \hline
         3 & & & & & \\
         \hline
     \end{tabular}}
   \end{center}
   \begin{cadre}[B2][F4]
      \begin{center}
         \hrefVideo{https://www.youtube.com/watch?v=mmyjICMR7BM}{\bf La petite histoire du tableur}, {\it Communauté dynamique}.
      \end{center}
   \end{cadre}
   \vspace*{-10mm}
\end{debat}
\end{changemargin}

