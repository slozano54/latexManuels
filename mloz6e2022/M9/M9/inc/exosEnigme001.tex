% Les enigmes ne sont pas numérotées par défaut donc il faut ajouter manuellement la numérotation
% si on veut mettre plusieurs enigmes
%\refstepcounter{exercice}
%\numeroteEnigme
\vspace*{-10mm}
\begin{enigme}[Des pavés de toutes sortes]
    {\bf Objectifs :} calculer le volume d'un pavé ; différencier volume et capacité ; résoudre un problème dans le domaine des grandeurs et mesures. 
 
       \partie[observations]
          Quelle action est matérialisée par le schéma suivant ? 
          \par\dotfill 
          \par\dotfill 
          \begin{center}
             \includegraphics[width=0.6\linewidth]{\currentpath/images/piscine}
          \end{center}
       \partie[questions]
          \begin{enumerate}
             \item Quel est le volume d'eau en \Vol[cm]{} contenu dans la boite transparente ?  
                \par\medskip\dotfill 
             \item Quelle est la capacité d'eau en L contenue dans la boite transparente ?  
                \par\medskip\dotfill 
             \item Dans la bassine, on plonge trois cubes et quatre pavés. Quelle doit être la hauteur des pavés pour que l'eau monte de \Lg[cm]{4} ? 
                \par\medskip\dotfill  
                \par\medskip\dotfill  
                \par\medskip\dotfill
          \end{enumerate}
 
       \creditLibre{d'après l'activité \href{http://www-irem.univ-paris13.fr/site_spip/IMG/pdf/des_paves_dans_la_mare_2_noir.pdf}{Des pavés dans la mare}, IREM Paris Nord.} 
\end{enigme}  
% % Pour le corrigé, il faut décrémenter le compteur, sinon il est incrémenté deux fois
% \addtocounter{exercice}{-1}
% \begin{corrige}
%     Correction enigme de la fin de la partie cours.  
%     
% \end{corrige}