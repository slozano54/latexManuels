\begin{exercice}
    Une boîte a la forme d'un pavé droit de dimensions \Lg[cm]{12}, \Lg[cm]{8} et \Lg[cm]{5}.
    \begin{center}
       \begin{pspicture}(0,0.25)(3.6,3)
          \pspolygon(0,0)(3,0)(3.6,0.6)(3.6,2.6)(0.6,2.6)(0,2)
          \psline(0,2)(3,2)(3,0)
          \psline(3,2)(3.6,2.6)
          \psline(0,1.6)(3,1.6)(3.6,2.2)
          \psline(2.25,0)(2.25,0.25)(3,0.25)(3.25,0.5)(3.25,0.25)
          \psline(2.5,0)(2.5,0.25)
          \psline(2.75,0)(2.75,0.5)(3,0.5)(3.125,0.625)(3.125,0.125)
       \end{pspicture}
    \end{center}
    \begin{enumerate}
       \item Calculer le nombre de cubes de côté \Lg[cm]{1} que l'on peut ranger dans cette boîte.
       \item Déterminer le nombre de cubes de côté \Lg[mm]{1} que l'on peut ranger dans cette boîte.
       \item Exprimer son volume en \Vol[cm]{} puis en \Vol[mm]{}.
    \end{enumerate}
 \end{exercice}
