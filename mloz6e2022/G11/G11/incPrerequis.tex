%pre-001
\begin{prerequis}[Connaisances \emoji{red-heart} et compétences \emoji{diamond-suit} du cycle 3]    
   \begin{itemize}        
       \item[\emoji{red-heart}] Vocabulaire associé à ces objets et à leurs propriétés : côté, sommet, angle, hauteur.
       \columnbreak
       \item[\emoji{diamond-suit}] Reconnaître, nommer, décrire des triangles, dont les triangles particuliers (triangle rectangle, triangle isocèle, triangle équilatéral).       
   \end{itemize}
\end{prerequis}
\vfill
\begin{debat}[La symétrie au natruel] 
   Dans la nature, on trouve de nombreux objets ayant une {\bf symétrie axiale} : animaux, paysages, monuments\dots{} regardez et observez autour de vous !
   \begin{center}
      \begin{Geometrie}
         pair A[],B[],C[],SymA[],SymB[];
         % Maison
         A0=u*(1,1);
         A1-A0=u*(2,0);
         A2-A1=u*(0,1);
         A3-A0=u*(0,1);
         A4=iso(A2,A3) shifted (0,0.5u);
         % Porte
         B0-A0=u*(1.2,0);
         B1-B0=u*(0.4,0);
         B2-B1=u*(0,0.4);
         B3-B0=u*(0,0.4);
         % Axe
         C0-A0=u*(3,-1);
         C1-C0=u*(0,3);
         % Tracés
         trace polygone(A0,A1,A2,A4,A3);
         trace segment(A3,A2);
         trace polygone(B0,B1,B2,B3);
         trace segment(C0,C1) dashed dashpattern(on6bp off3bp on1.5bp off3bp) withcolor red;
         % Symétriques
         SymA0=symetrie(A0,C0,C1);
         SymA1=symetrie(A1,C0,C1);
         SymA2=symetrie(A2,C0,C1);
         SymA3=symetrie(A3,C0,C1);
         SymA4=symetrie(A4,C0,C1);
         SymB0=symetrie(B0,C0,C1);
         SymB1=symetrie(B1,C0,C1);
         SymB2=symetrie(B2,C0,C1);
         SymB3=symetrie(B3,C0,C1);
         trace polygone(SymA0,SymA1,SymA2,SymA4,SymA3);
         trace segment(SymA3,SymA2);
         trace polygone(SymB0,SymB1,SymB2,SymB3);
      \end{Geometrie}
   \end{center}
   \vfill
   \begin{cadre}[B2][J4]
      \begin{center}
         \hrefVideo{https://www.yout-ube.com/watch?v=6LfXWSvtKM4}{\bf La symétrie ou les maths au clair de lune}, chaîne {\it Le Blob}
      \end{center}
   \end{cadre}
\end{debat}