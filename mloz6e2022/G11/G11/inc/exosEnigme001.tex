% Les enigmes ne sont pas numérotées par défaut donc il faut ajouter manuellement la numérotation
% si on veut mettre plusieurs enigmes
% \refstepcounter{exercice}
% \numeroteEnigme
\begin{enigme}[Multi-symétries]
    \begin{changemargin}{-10mm}{-10mm}
        \vspace*{-15mm}
        Dans les deux quadrillages suivants, construire le symétrique de la figure par rapport à tous les axes tracés.
        \begin{center}
           {\psset{unit=0.5}
           \begin{pspicture}(-11,-8)(11,8)
              \psgrid[subgriddiv=0,gridlabels=0,gridcolor=lightgray](-11,-8)(11,8)
              \psline(-11,0)(11,0)
              \psline(0,-8)(0,8)
              \psset{linewidth=1mm}
              \psarc(3,0){3}{0}{180}
              \psarc(7,0){1}{0}{180}
              \psarc(0,3){3}{-90}{90}
              \psline(8,0)(8,2)(10,4)(10,0)
           \end{pspicture} \\ [10mm]
           \begin{pspicture}(-11,-11)(11,11)
              \psgrid[subgriddiv=0,gridlabels=0,gridcolor=lightgray](-11,-11)(11,11)
              \psline(-11,0)(11,0)
              \psline(0,-11)(0,11)
              \psline(-11,-11)(11,11)
              \psline(-11,11)(11,-11)
              \psset{linewidth=1mm}
              \psline(0,0)(2,0)(2,1)(3,1)(3,2)(4,2)(4,3)(5,3)(5,4)(6,4)(6,5)(7,5)(7,6)(8,6)(10,0)(7,0)(5,2)(3,0)
           \end{pspicture} } 
        \end{center}
    \end{changemargin}
\end{enigme}

% Pour le corrigé, il faut décrémenter le compteur, sinon il est incrémenté deux fois
% \addtocounter{exercice}{-1}
\begin{corrige}
    Pas de corrigé.
\end{corrige} 
