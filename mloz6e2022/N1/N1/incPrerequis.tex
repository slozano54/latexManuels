\vspace*{-2mm}
%pre-001
\begin{prerequis}[Connaisances \emoji{red-heart} et compétences \emoji{diamond-suit} du cycle 3]    
   \begin{itemize}        
       \item[\emoji{red-heart}] Vocabulaire associé à ces objets et à leurs propriétés : côté, sommet, angle, hauteur.
       \columnbreak
       \item[\emoji{diamond-suit}] Reconnaître, nommer, décrire des triangles, dont les triangles particuliers (triangle rectangle, triangle isocèle, triangle équilatéral).       
   \end{itemize}
\end{prerequis}
\begin{debat}[Nombre ou chiffre ?]
    On croit souvent que les chiffres sont 0, 1, 2, 3, 4, 5, 6, 7, 8, 9 et que les nombres sont 10, 11, 12, 13, 14, 15, 16, 17, 18, 19, 20, 21\dots{} La vérité n'est pas si simple : un chiffre est un signe, un symbole, une forme pour représenter une quantité alors qu'un nombre désigne une quantité qui dépend d'une unité. Par exemple : \\ [5mm]
    \centerline{\textcolor{B1}{\fontsize{30}{30}\selectfont 8 \; - \; \fontsize{60}{60}\selectfont 5}} \\ [5mm]
    Le plus grand nombre est \og 8 \fg{} car il désigne la plus grande quantité alors que le plus grand chiffre est \og 5 \fg{} puisque c'est lui qui prend \og plus de place \fg.
    \bigskip
    \begin{cadre}[B2][F4]
       \begin{center}
          Vidéo : \href{https://www.yout-ube.com/watch?v=WRrLnktqUmE&feature=emb_logo}{\bf Raconte-moi une histoire : les petits cailloux},
          
          chaîne YouTube {\it m@ths et tiques}, Yvan Monka.
       \end{center}
    \end{cadre}
 \end{debat}