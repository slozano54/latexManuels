\section{La droite graduée}

\begin{definition}
    Pour graduer une droite, il faut choisir : 
    \begin{itemize}
        \item une {\bf origine} qui correspond au \og 0 \fg{}
        \item une {\bf unité} qui sera reportée de manière régulière
        \item un {\bf sens croissant}.
    \end{itemize}
   Un point est repéré par son {\bf abscisse}. A a pour abscisse 3 se note A(3).\\
   \begin{pspicture}(-2,-1)(5.5,1)
   \psset{xunit=2}
      \psaxes[yAxis=false]{->}(0,0)(5.2,1)
      \psline[linecolor=violet]{<-}(-0.02,0.04)(-0.3,0.5)
      \rput(-0.6,0.5){\textcolor{violet}{origine}}
      \psline[linecolor=A1]{<->}(0,0.3)(1,0.3)
      \rput(0.5,0.6){\textcolor{A1}{unité}}
      \rput(3,0.4){\textcolor{B1}{A}}
      \psline[linecolor=B1]{<-}(3.02,0.04)(3.3,0.5)
      \rput(4.1,0.5){\textcolor{B1}{l'abscisse de A est 3}}
      \psline[linecolor=mygreen]{<-}(5.15,0)(5.5,1.5)
      \rput(5.3,1.6){\textcolor{mygreen}{sens croissant}}
   \end{pspicture}
\end{definition}
      
\begin{exemple*1}   
   \ \\
   \begin{pspicture}(-1,-0.5)(10,0.5)
      \psaxes[yAxis=false,labels=none]{->}(0,0)(10.5,0)
      \rput(0,-0.4){0}
      \rput(1,-0.4){100}
      \rput(3,0.4){A}
      \rput(7,0.4){R}
      \rput(10,0.4){C}
   \end{pspicture} \\
   Ici, l'unité vaut 100, donc les points A, R et C ont pour abscisses 300; 700 et \num{1000}.\\
   On note  A(300), R(700) et C(\num{1000}).
\end{exemple*1}