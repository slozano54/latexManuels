% Les enigmes ne sont pas numérotées par défaut donc il faut ajouter manuellement la numérotation
% si on veut mettre plusieurs enigmes
% \refstepcounter{exercice}
%\phantom{\numeroteEnigme}
\begin{enigme}[La numération égyptienne]
   \partie[introduction]
      L'Égypte antique est une ancienne civilisation qui a vécu un peu plus de $3000$ ans entre $3150$ av. J.-C. et $30$ av. J.-C., elle s'est développée le long du Nil. Les Égyptiens maitrisent l'écriture, ils utilisent des symboles qui représentent des choses réelles ou non grâce à des hiéroglyphes. \\ [3mm]
      \begin{minipage}{0.4\linewidth}
         \includegraphics[scale=1.5]{\currentpath/images/rosette}
      \end{minipage}
      \qquad
      \begin{minipage}{0.58\linewidth}
         En 1821, Jean-François Champollion réussit à décrypter les caractères égyptiens. Comment a-t-il pu réussir cet exploit ? On pourra s'aider de l'image ci-contre. \\ [3mm]
         \makebox[\linewidth]{\dotfill} \\ [3mm]
         \makebox[\linewidth]{\dotfill} \\ [3mm]
         \makebox[\linewidth]{\dotfill}
      \end{minipage}

      \bigskip
      
   \partie[la numération égyptienne]
      \begin{enumerate}
         \item Dans la numération égyptienne, voilà comment on écrit certains nombres indo-arabes : \smallskip
         \begin{center} 
            \begin{tabular}{|c|c|c|c|c|}
               \hline
               $18$ & $70$ & $235$ & $\num{3 018}$ & $\num{1 230 012}$ \\
               \hline
               & & & & \\ [-3mm]
               \Large\textpmhg{\Hten\Hone\Hone\Hone\Hone\Hone\Hone\Hone\Hone}
               &
               \Large\textpmhg{\Hten\Hten\Hten\Hten\Hten\Hten\Hten}
               & 
               \Large\textpmhg{\Hhundred\Hhundred\Hten\Hten\Hten\Hone\Hone\Hone\Hone\Hone} & \Large\textpmhg{\Hthousand\Hthousand\Hthousand\Hten\Hone\Hone\Hone\Hone\Hone\Hone\Hone\Hone}
               &
               \Large\textpmhg{\Hmillion\HCthousand\HCthousand\HXthousand\HXthousand\HXthousand\Hten\Hone\Hone} \\
               \hline
            \end{tabular}
         \end{center} \medskip
      À la manière de Champollion, retrouver la valeur de ces symboles égyptiens. \smallskip
      \begin{center}
         \begin{tabular}{|*{7}{>{\centering\arraybackslash}p{1.8cm}|}}
            \hline
            & & & & & & \\ [-3mm]
            \Large\textpmhg{\Hone}
            &
            \Large\textpmhg{\Hten}
            &
            \Large\textpmhg{\Hhundred}
            &
            \Large\textpmhg{\Hthousand}
            &
            \Large\textpmhg{\HXthousand}
            &
            \Large\textpmhg{\HCthousand}
            &
            \Large\textpmhg{\Hmillion} \\
            \hline
            & & & & & & \\ [3mm]
           \hline
         \end{tabular}
      \end{center}
 
      \bigskip
   
      \item Écrire les nombres égyptiens suivants en nombres indo-arabes : \\ [2mm]
      {\Large\textpmhg{\HXthousand\Hthousand\Hthousand\Hten\Hten\Hten\Hten\Hten\Hten}} : \dotfill \\ [3mm]
      {\Large\textpmhg{\HCthousand\HXthousand\HXthousand\HXthousand\Hthousand\Hthousand\Hthousand\Hhundred\Hhundred\Hhundred\Hten\Hten\Hten\Hten\Hone\Hone\Hone\Hone\Hone\Hone\Hone\Hone}} : \dotfill \\  [3mm]
      {\Large\textpmhg{\Hmillion\Hmillion\Hmillion\HCthousand\Hone\Hone}} : \dotfill \\
      \item Écrire les nombres indo-arabes suivants en nombres égyptiens : \\ [4mm]
      $\num{8 032}$ : \dotfill \\ [5mm]
      $\num{3 000 100}$ :  \dotfill \\
      \item Quel nombre présent dans notre numération écrite n'existe pas dans la numération égyptienne ? Pourquoi ? \\ [3mm]
      \makebox[\linewidth]{\dotfill}
      \end{enumerate}
\end{enigme}
% Pour le corrigé, il faut décrémenter le compteur, sinon il est incrémenté deux fois
% \addtocounter{exercice}{-1}
\begin{corrige}
    \setcounter{partie}{0} % Pour s'assurer que le compteur de \partie est à zéro dans les corrigés
    \partie[introduction]
       En 1821, Jean-François Champollion réussit à décrypter les caractères égyptiens. Comment a-t-il pu réussir cet exploit ?\\
       {\red Sur la pierre de rosette, figurait trois fois le même texte dans des langages différents dont les caractères égyptien et le grec.
       Ceci a permis de faire su lien avec le sens des caractères égyptiens}

    \partie[la numération égyptienne]

    \begin{enumerate}
        \item À la manière de Champollion, retrouver la valeur de ces symboles égyptiens. 
        \begin{tabular}{|*{4}{>{\centering\arraybackslash}p{0.14\linewidth}|}}
           \hline
           & & & \\ [-3mm]
           \Large\textpmhg{\Hone}
           &
           \Large\textpmhg{\Hten}
           &
           \Large\textpmhg{\Hhundred}
           &
           \Large\textpmhg{\Hthousand}\\
           \hline
           {\red $1$} & {\red $10$} & {\red $100$} & {\red $\num{1 000}$} \\ [3mm]
          \hline
        \end{tabular}

        \begin{tabular}{|*{3}{>{\centering\arraybackslash}p{0.2\linewidth}|}}
            \hline
            & & \\ [-3mm]
            \Large\textpmhg{\HXthousand}
            &
            \Large\textpmhg{\HCthousand}
            &
            \Large\textpmhg{\Hmillion} \\
            \hline
            {\red $\num{10 000}$} & {\red $\num{100 000}$} & {\red $\num{1 000 000}$}\\ [3mm]
           \hline
         \end{tabular}
 
     \item Écrire les nombres égyptiens suivants en nombres indo-arabes : \\ [2mm]
     {\Large\textpmhg{\HXthousand\Hthousand\Hthousand\Hten\Hten\Hten\Hten\Hten\Hten}} : {\red $\num{12 060}$} \\ [3mm]
     {\Large\textpmhg{\HCthousand\HXthousand\HXthousand\HXthousand\Hthousand\Hthousand\Hthousand\Hhundred\Hhundred\Hhundred\Hten\Hten\Hten\Hten\Hone\Hone\Hone\Hone\Hone\Hone\Hone\Hone}} : {\red $\num{133 348}$} \\  [3mm]
     {\Large\textpmhg{\Hmillion\Hmillion\Hmillion\HCthousand\Hone\Hone}} : {\red $\num{3 100 002}$} \\
     \item Écrire les nombres indo-arabes suivants en nombres égyptiens : \\ [4mm]
     $\num{8 032}$ : {\red {\Large\textpmhg{\HXthousand\Hthousand\Hthousand\Hthousand\Hthousand\Hthousand\Hthousand\Hthousand\Hten\Hten\Hten\Hone\Hone}}} \\ [5mm]
     $\num{3 000 100}$ :  {\red {\Large\textpmhg{\Hmillion\Hmillion\Hmillion\Hhundred}}} \\
     \item Quel nombre présent dans notre numération écrite n'existe pas dans la numération égyptienne ? Pourquoi ? \\ [3mm]
     
     {\red Le zéro n'existait pas. Il n'était pas nécessaire. Les symboles étaient répétés jusqu'à $9$ fois.}
     \end{enumerate}
\end{corrige}
