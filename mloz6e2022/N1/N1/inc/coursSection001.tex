\section{Écriture des nombres entiers}
\begin{definition}
    Dans notre numération, il y a dix chiffres : 0, 1, 2, 3, 4, 5, 6, 7, 8 et 9.
    
    Avec ces chiffres, on peut écrire des nombres : il y en a une infinité et on les utilise avec des unités pour qu'ils aient un sens.
\end{definition}

\begin{exemple*1}
   Une classe de 28 {\it élèves}, une voiture à 5 {\it places}, un sac de 20 {\it kg}, une pièce de 2 {\it euros}\dots
\end{exemple*1}


\begin{definition}[Tableau des nombres entiers]
   Les nombres sont regroupés en classes composées de trois rangs : unités, dizaines, centaines.
   \begin{center}
    \scalebox{0.91}{
    \begin{tabular}{*{3}{|c}|*{3}{|c}|*{3}{|c}|*{3}{|c}|}
        \hline
        \multicolumn{3}{|c||}{\textbf{Classe des Milliards}} & \multicolumn{3}{c||}{\textbf{Classe des Millions}}&\multicolumn{3}{c||}{\textbf{Classe des Milliers}}&\multicolumn{3}{c|}{\textbf{Classe des Unités}}\\
        \hline
        \rotatebox{90}{\parbox{3.5cm}{\textbf{Centaines}\\ de milliards }}& 
        \rotatebox{90}{\parbox{3.5cm}{\textbf{Dizaines}\\ de milliards }}& 
        \rotatebox{90}{\parbox{3.5cm}{\textbf{Unités}\\ de milliards }}& 
        \rotatebox{90}{\parbox{3.5cm}{\textbf{Centaines}\\ de millions }}& 
        \rotatebox{90}{\parbox{3.5cm}{\textbf{Dizaines}\\ de millions }}& 
        \rotatebox{90}{\parbox{3.5cm}{\textbf{Unités}\\ de millions }}& 
        \rotatebox{90}{\parbox{3.5cm}{\textbf{Centaines}\\ de milliers }}& 
        \rotatebox{90}{\parbox{3.5cm}{\textbf{Dizaines}\\ de milliers }}& 
        \rotatebox{90}{\parbox{3.5cm}{\textbf{Unités}\\ de milliers }}& 
        \rotatebox{90}{\parbox{3.5cm}{\textbf{Centaines}\\ d'unités }}& 
        \rotatebox{90}{\parbox{3.5cm}{\textbf{Dizaines}\\ d'unités }}&
        \rotatebox{90}{\parbox{3.5cm}{\textbf{Unités}\\ d'unités}} 
        \\
        \hline
        & & & & 1 & 2 & 0 & 4 & 5 & 9 & 7 & 6 \\
        \hline
    \end{tabular} 
    }
   \end{center}
\end{definition}

\begin{exemple*1}
   Dans le nombre \num{12045976} :
   \begin{colitemize}{2}
      \item 9 est le chiffre des centaines ;
      \item 5 est le chiffre des milliers ;
      \item 1 est le chiffre des dizaines de millions ;
      \item il y a \num{120459} centaines ;
      \item il y a \num{1204} dizaines de milliers ;
      \item il y a 12 millions.
   \end{colitemize}
\end{exemple*1}

\begin{remarques}
    On peut décomposer tout nombre sous sa forme canonique : \\
    $\begin{array}{*{8}{l}}
    \num{12045976} & =\num{10000000} & + \num{2000000} & + \num{40000} & + \num{5000} & + 900 & + 70 & + 6 \\
        & =(1\times\num{10000000}) & + (2\times\num{1000000}) & + (4\times\num{10000}) & +(5\times\num{1000}) & +(9\times100) & +(7\times10) & +(6\times1) \\
        & =1\times10^7 & +2\times10^6 & +4\times10^4 & +5\times10^3 & +9\times10^2 & +7\times10^1 & +6\times10^0 \\
    \end{array}$
    La dernière écriture n'est pas au programme de la sixième : il s'agit de la décomposition du nombre dans la base 10 qui permet juste d'observer comment on écrit un nombre dans une base.
\end{remarques}

\begin{propriete}[Écriture des grands nombres en chiffres]
    Pour pouvoir lire les grands nombres plus facilement, on regroupe les chiffres par tranches de trois en partant du chiffre des unités de la classe des Unités.
\end{propriete}

\begin{exemple*1}
    \begin{itemize}
        \item $12345678910111213$ s'écrira plutôt $\num{12345678910111213}$.
        \item $9123456789$ s'écrira plutôt $\num{9123456789}$, et se lit 
    \end{itemize}
    \begin{center}
        {\footnotesize \og neuf \underline{milliards} cent vingt-trois \underline{millions} quatre cent cinquante-six \underline{mille} sept cent quatre-vingt-neuf \underline{unités} \fg}
    \end{center}
\end{exemple*1}

\begin{propriete}[Règles orthographiques]   
   \begin{itemize}
      \item Deux mots d'un même nombre sont séparés par un trait d'union ;
      \item \og mille \fg est invariable ;
      \item Multipliés, \og cent \fg ou \og vingt \fg prennent la marque du pluriel, \og s \fg, mais ils la perdent quand ils sont suivis d’un autre adjectif numéral (\og quatre \fg, \og douze \fg, \og quarante \fg, etc.)\\
      Toutefois, devant \og millier \fg, \og million \fg, \og milliard \fg, qui sont des noms, le \og s \fg du pluriel subsiste.
      \item \og million \fg et \og milliard \fg prennent toujours un \og s \fg{} quand il sont multipliés.
   \end{itemize}
\end{propriete}

\begin{exemple*1}
   \begin{itemize}
        \item \num{4000} : quatre-mille ;        
        \item \num{12045976} : douze-millions-quarante-cinq-mille-neuf-cent-soixante-seize.
        \item $80$ s'écrit "quatre-vingt\underline{s}" mais $83$ s'écrit "quatre-vingt-trois"
        \item $200$ s'écrit "deux-cent\underline{s}" mais $237$ s'écrit "deux-cent trente-sept"
        \item Deux-cents personnes sont attendues, mais \\
        établissez un chèque de cinq-cent quarante euros.
        \item Ce film a rapporté deux-cents-millions de dollars.
        \item La société compte quatre-vingts salariés, mais\\
        tous les manteaux sont soldés à quatre-vingt-dix-neuf euros.
        \item Ce club de foot réputé a un budget de quatre-vingts-millions d’euros.
    \end{itemize}
\end{exemple*1}