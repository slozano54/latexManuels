% \section{Différentes manière de penser les fractions}
\section{Fraction et partages}
\begin{myBox}{\emoji{light-bulb} Idée}
   Au départ l'idée de fraction correspond à une situation de partage.\\
   Mémé Carmela fait ses succulentes pizzas et les partage pour ses enfants et petits enfants adorés.\\
   Cela peut mener aux représentations simplifiées qui suivent. 
\end{myBox}

\begin{center}
   \begin{tabular}{|>{\centering}m{5cm}|>{\centering}m{4cm}|>{\centering}m{2cm}|c|}
      \hline 
      Descriptif & Figure & Fraction & En français on lit \\ 
      \hline 
      Giovani est gourmand il veut la moitié d'une pizza 
      & 
      \begin{pspicture}(-1,-1.1)(1,1.1)
         \rput(0,0){\includegraphics[width=2cm]{\currentpath/images/pizza}}
         \pscircle(0,0){1}
        \psset{linecolor=white,fillstyle=solid,fillcolor=white}
         \pswedge(0,0){0.98}{180}{0}
         \psline(-1,0)(1,0)
      \end{pspicture}
      & $\dfrac12$ & un \textbf{demi} \\ 
      \hline 
      Liliane aussi elle veut cette partie là! 
      & 
      \begin{pspicture}(-1,-1.1)(1,1.1)
         \rput(0,0){\includegraphics[width=2cm]{\currentpath/images/pizza}}
         \pscircle(0,0){1}
        \psset{linecolor=white,fillstyle=solid,fillcolor=white}
         \pswedge(0,0){0.98}{240}{0}
         \pnode(0,0){O}
         \SpecialCoor
         \psline(O)([angle=120,nodesep=1]O)         
      \end{pspicture}     
      & $\dfrac23$ & deux \textbf{tiers} \\ 
      \hline 
      Angelina n'a pas trop faim! 
      &
      \begin{pspicture}(-1,-1.1)(1,1.1)
         \rput(0,0){\includegraphics[width=2cm]{\currentpath/images/pizza}}
         \pscircle(0,0){1}
        \psset{linecolor=white,fillstyle=solid,fillcolor=white}
         \pswedge(0,0){0.98}{90}{0}
      \end{pspicture}     
      & $\dfrac14$ & un \textbf{quart} \\ 		
      \hline 
      Tonio est un gourmand! 
      &
      \begin{pspicture}(-1,-1.1)(1,1.1)
         \rput(0,0){\includegraphics[width=2cm]{\currentpath/images/pizza}}
         \pscircle(0,0){1}
        \psset{linecolor=white,fillstyle=solid,fillcolor=white}
         \pswedge(0,0){0.98}{288}{0}
         \pnode(0,0){O}
         \SpecialCoor
         \psline(O)([angle=72,nodesep=1]O)
         \psline(O)([angle=144,nodesep=1]O)
         \psline(O)([angle=216,nodesep=1]O)
      \end{pspicture}     
      & $\dfrac45$ & quatre \textbf{cinquièmes} \\ 
      \hline 
      Adélaïde aussi est gourmande 
      &
      \begin{pspicture}(-1,-1.1)(1,1.1)
         \rput(0,0){\includegraphics[width=2cm]{\currentpath/images/pizza}}
         \pscircle(0,0){1}
        \psset{linecolor=white,fillstyle=solid,fillcolor=white}
         \pswedge(0,0){0.98}{257.14}{0}
         \pnode(0,0){O}
         \SpecialCoor
         \psline(O)([angle=51.43,nodesep=1]O)
         \psline(O)([angle=102.85,nodesep=1]O)
         \psline(O)([angle=154.28,nodesep=1]O)
         \psline(O)([angle=205.71,nodesep=1]O)
         \psset{linecolor=black}
         \psline(O)([angle=308.57,nodesep=1]O)
      \end{pspicture}          
      & $\dfrac57$ & cinq \textbf{septièmes} \\ 
      \hline 
      C'est tellement bon tout ça hein Francesco! 
      & 
      \begin{pspicture}(-1,-1.1)(1,1.1)
         \rput(0,0){\includegraphics[width=2cm]{\currentpath/images/pizza}}
         \pscircle(0,0){1}
        \psset{linecolor=white,fillstyle=solid,fillcolor=white}
         \pswedge(0,0){0.98}{252}{0}
         \pnode(0,0){O}
         \SpecialCoor
         \psline(O)([angle=36,nodesep=1]O)
         \psline(O)([angle=72,nodesep=1]O)
         \psline(O)([angle=108,nodesep=1]O)
         \psline(O)([angle=144,nodesep=1]O)
         \psline(O)([angle=180,nodesep=1]O)
         \psline(O)([angle=216,nodesep=1]O)
         \psset{linecolor=black}
         \psline(O)([angle=288,nodesep=1]O)
         \psline(O)([angle=324,nodesep=1]O)
      \end{pspicture}          
      & $\dfrac7{10}$ & sept \textbf{dixièmes} \\ 
      \hline 
      Alors pas trop faim Jeanina? 
      &
      \begin{pspicture}(-1,-1.1)(1,1.1)
         \rput(0,0){\includegraphics[width=2cm]{\currentpath/images/pizza}}
         \pscircle(0,0){1}
        \psset{linecolor=white,fillstyle=solid,fillcolor=white}
         \pswedge(0,0){0.98}{150}{0}
         \pnode(0,0){O}
         \SpecialCoor
         \psline(O)([angle=30,nodesep=1]O)
         \psline(O)([angle=60,nodesep=1]O)
         \psline(O)([angle=90,nodesep=1]O)
         \psline(O)([angle=120,nodesep=1]O)
         \psset{linecolor=black}
         \psline(O)([angle=180,nodesep=1]O)
         \psline(O)([angle=210,nodesep=1]O)
         \psline(O)([angle=240,nodesep=1]O)
         \psline(O)([angle=270,nodesep=1]O)
         \psline(O)([angle=300,nodesep=1]O)
         \psline(O)([angle=330,nodesep=1]O)
      \end{pspicture}                
      & $\dfrac5{12}$ & cinq \textbf{douzièmes} \\ 
      \hline 
   \end{tabular} 
\end{center}

\begin{definition}
   Une \textbf{fraction} indique quelle partie d'un tout on doit prendre, elle peut représenter un {\bf partage}, une {\bf proportion}.
\end{definition}

\bigskip

On partage une pizza en quatre parts égales.

\begin{center}
   {\small
   \begin{tabular}{>{\centering\arraybackslash}p{3cm}>{\centering\arraybackslash}p{3cm}>{\centering\arraybackslash}p{3cm}>{\centering\arraybackslash}p{3.5cm}>{\centering\arraybackslash}p{4cm}}
      \begin{pspicture}(-1,-1.1)(1,1.1)
         \rput(0,0){\includegraphics[width=2cm]{\currentpath/images/pizza}}
         \pscircle(0,0){1}
         \psset{linecolor=white,fillstyle=solid,fillcolor=white}
         \pswedge(0,0){0.98}{90}{0}
         \psset{linecolor=black}
         \psline(-1,0)(1,0)
         \psline(0,-1)(0,1)  
      \end{pspicture}
      &
      \begin{pspicture}(-1,-1.1)(1,1.1)
         \rput(0,0){\includegraphics[width=2cm]{\currentpath/images/pizza}}
         \pscircle(0,0){1}
        \psset{linecolor=white,fillstyle=solid,fillcolor=white}
         \pswedge(0,0){0.98}{180}{0}
         \psline[linewidth=0.7mm](0,0)(0,1) 
         \psset{linecolor=black}
         \psline(-1,0)(1,0)
         \psline(0,-1)(0,0)  
      \end{pspicture}
      &
      \begin{pspicture}(-1,-1.1)(1,1.1)
         \rput(0,0){\includegraphics[width=2cm]{\currentpath/images/pizza}}
         \pscircle(0,0){1}
         \psset{linecolor=white,fillstyle=solid,fillcolor=white}
         \pswedge(0,0){0.98}{-90}{0}
         \psline[linewidth=0.7mm](-1,0)(1,0)
         \psline[linewidth=0.7mm](0,0)(0,1)
      \end{pspicture}
      &
      \begin{pspicture}(-1,-1.1)(1,1.1)
         \rput(0,0){\includegraphics[width=2cm]{\currentpath/images/pizza}}
        \pscircle(0,0){1}
         \psset{linewidth=0.7mm,linecolor=white}
         \psline(-1,0)(1,0)
         \psline(0,-1)(0,1) 
      \end{pspicture} \\
      {\bf une} part & \textbf{deux} parts & \textbf{trois} parts & \textbf{quatre} parts \\   
      un {\bf quart} de pizza & deux {\bf quarts} de pizza & trois {\bf quarts} de pizza & quatre {\bf quarts} de pizza \\ [2mm]
      $\dfrac14$ & $\dfrac24 =\dfrac14+\dfrac14 =2\times\dfrac14$ & $\dfrac34 =\dfrac14+\dfrac14+\dfrac14 =3\times\dfrac14$ & $\dfrac44 =\dfrac14+\dfrac14+\dfrac14+\dfrac14 =4\times\dfrac14$ \\ 
   \end{tabular}}
\end{center}

\smallskip

\begin{minipage}{10cm}
   Et si on mangeait plus d'une pizza ? \\
   Dans ce cas, on obtient une fraction supérieure à 1. \\
   On a pris {\bf sept} parts de pizza, soit $\dfrac74$ de pizza. \\
   On peut aussi dire {\bf une} pizza et {\bf trois} quarts de pizza, soit $1+\dfrac34$.
\end{minipage}
\qquad
\begin{minipage}{6cm}
   \begin{pspicture}(-1,-1.2)(1,1.2)
      \rput(0,0){\includegraphics[width=2cm]{\currentpath/images/pizza}}
      \pscircle(0,0){1}
      \psset{linewidth=0.7mm,linecolor=white}
      \psline(-1,0)(1,0)
      \psline(0,-1)(0,1) 
   \end{pspicture}
   \qquad 
   \begin{pspicture}(-1,-1.2)(1,1.2)
      \rput(0,0){\includegraphics[width=2cm]{\currentpath/images/pizza}}
      \pscircle(0,0){1}
      \psset{linecolor=white,fillstyle=solid,fillcolor=white}
      \pswedge(0,0){0.98}{-90}{0}
      \psline[linewidth=0.7mm](-1,0)(1,0)
      \psline[linewidth=0.7mm](0,0)(0,1)
   \end{pspicture}
\end{minipage}
