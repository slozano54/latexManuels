\section{Axes gradués}

% \definNum{On repère un point sur un axe gradué grâce à un nombre que l'on appelle son ABSCISSE}
% \begin{center}
% \includegraphics[scale=0.8]{figures/coursecriturefractionnaire.1}
% \end{center}

% L'extrémité de la flèche indique le sens positif.

% %\newpage
% \definNum{
% \begin{mylist}
% \item $O$ est l'Origine de cet axe gradué.
% \item $I$ est le point repéré par le nombre $1$. On dit que $1$ est l'abscisse de $I$.
% \item $K$ est le point d'abscisse $2,5$.
% \item La distance entre $O$ et $I$, les points d'abscisses $0$ et $1$, est \underline{l'unité de longueur}
% \item Parmi tous les axes gradués, l'axe gradué horizontal s'appelle AXE des ABSCISSES.
% \end{mylist}
% }

% \includegraphics[scale=1]{figures/coursecriturefractionnaire.2}
% \par 
% ici, on a pris 2 cm comme unité de longueur.

% \subsection{Placer un quotient sur un axe gradué}
% \Exemples{}{
% Place le quotient $\dfrac{5}{3}$ sur un axe gradué.
% \begin{enumerate}
% \item Choisis une unité de longueur qui soit un multiple du dénominateur ( ici 3 ).
% \par
% \includegraphics[scale=0.7]{figures/coursecriturefractionnaire.3}
% \item Reporte la fraction unitaire autant de fois qu'il le faut (ici 5 fois) ie un nombre de fois égal au numérateur.
% \par
% \includegraphics[scale=0.7]{figures/coursecriturefractionnaire.4}
% \end{enumerate}
% }

% \CadreLampe{Remarques}{
% \begin{mylist}
% \item $5\times \dfrac13=\dfrac53$
% \item $\dfrac53>1$, on remarque que son numérateur est plus grand que son dénominateur.
% \end{mylist}
% }

% \Exemples[Exemple]{}{
% Place le quotient $\dfrac{4}{7}$ sur un axe gradué, il est plus grand ou plus petit que 1?
% }