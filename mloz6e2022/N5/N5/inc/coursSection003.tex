\section{Nombre fraction}

% \definNum{$a$ et $b$ sont deux nombres entiers, $b$ n'est pas nul.\\
% Le  nombre qui, lorsqu'il est multiplié par $b$ donne $a$:
% \begin{mylist}
% \item se note $\dfrac{a}{b}$
% \item s'appelle "FRACTION $a$ sur $b$"
% \end{mylist}
% }

% \CadreLampe{Pourquoi d'abord?}{
% $\dfrac{a}{b}$ est donc, par définition, le quotient de a par b alors pourquoi cettte notation?\par\vspace{0.25cm}
% Pour déterminer le nombre qui multiplié par $3$ donne $7$, on pose la division $7\div 3$ mais elle est interminable!!!\\
% $$\opdiv[maxdivstep=4]{7}{3}$$\\
% Le quotient vaut donc seulement \textbf{environ} $2,333$ mais \textbf{exactement} $\dfrac73$ avec cette notation!
% }

% \CadreLampe{Remarque}{
% tout nombre décimal peut s'écrire sous forme fractionnaire, $0,4=\dfrac{4}{10}$; $0,75=\dfrac{75}{100}$ alors que toute écriture fractionnaire n'a pas d'écriture décimale, les quotients interminables, $\dfrac{7}{3}$; ...
% }

% \CadreLampe{"Justification"}{
% \begin{tabular}{|p{0.5cm}p{0.5cm}p{0.5cm}p{0.5cm}|}
% \hline 
% \cellcolor{red}&\cellcolor{red}&\cellcolor{red}&\cellcolor{red}\\ 
% \hline 
% \end{tabular} 
% représente 1 unité
% \par\vspace{0.5cm}
% \begin{tabular}{|p{0.5cm}p{0.5cm}p{0.5cm}p{0.5cm}|p{0.5cm}p{0.5cm}p{0.5cm}p{0.5cm}|p{0.5cm}p{0.5cm}p{0.5cm}p{0.5cm}|}
% \hline 
% \cellcolor{red}&\cellcolor{red}&\cellcolor{red}&\cellcolor{red}&\cellcolor{red}&\cellcolor{red}&\cellcolor{red}&\cellcolor{red}&\cellcolor{red}&\cellcolor{red}&\cellcolor{red}&\cellcolor{red}\\ 
% \hline 
% \end{tabular}
% représente 3 unités 
% \par\vspace{0.5cm}
% \begin{tabular}{|p{0,5cm}|p{0,5cm}|p{0,5cm}|}
% \hline
% \cellcolor{green}&\cellcolor{green}&\cellcolor{green}\\
% \hline
% \end{tabular}
% représente $\dfrac34$ d'unité
% \par\vspace{0.5cm}
% \begin{tabular}{|p{0,5cm}|p{0,5cm}|p{0,5cm}|p{0,5cm}|p{0,5cm}|p{0,5cm}|p{0,5cm}|p{0,5cm}|p{0,5cm}|p{0,5cm}|p{0,5cm}|p{0,5cm}|}
% \hline
% \cellcolor{green}&\cellcolor{green}&\cellcolor{green}&
% \cellcolor{green}&\cellcolor{green}&\cellcolor{green}&
% \cellcolor{green}&\cellcolor{green}&\cellcolor{green}&
% \cellcolor{green}&\cellcolor{green}&\cellcolor{green}\\
% \hline
% \end{tabular}
% représente $4\times \dfrac34$ d'unité \\qui équivaut bien à :
% \par\vspace{0.5cm}
% \begin{tabular}{|p{0.5cm}p{0.5cm}p{0.5cm}p{0.5cm}|p{0.5cm}p{0.5cm}p{0.5cm}p{0.5cm}|p{0.5cm}p{0.5cm}p{0.5cm}p{0.5cm}|}
% \hline 
% \cellcolor{red}&\cellcolor{red}&\cellcolor{red}&\cellcolor{red}&\cellcolor{red}&\cellcolor{red}&\cellcolor{red}&\cellcolor{red}&\cellcolor{red}&\cellcolor{red}&\cellcolor{red}&\cellcolor{red}\\ 
% \hline 
% \end{tabular}
% représente 3 unités 
% \par\vspace{0.5cm}
% $\dfrac34$ est donc bien le nombre qui lorsqu'il est multiplié par 4 donne 3.
% Donc $\dfrac34\times4 = 3$ et $4\times\dfrac34 = 3$
% }

 \begin{definition}
    Soit $a$ et $b$ deux nombres ($b\neq0$). Le {\bf quotient} $\dfrac{a}{b}$ est le nombre qui, multiplié par $b$, donne $a$. \\
    Ce quotient écrit sous forme d'une fraction est le résultat d'une division : $\dfrac{a}{b} =a\div b$.
 \end{definition}
 
 \begin{exemple}
    La fraction $\dfrac73$ peut être interprétée comme :
    \correction
    \ \\ [-10mm]
    \begin{itemize}
       \item $7\div3$, dont une valeur approchée est 2,33 ;
       \item sept tiers ;
       \item le nombre qui, multiplié par $3$, donne $7$.
    \end{itemize}
 \end{exemple}
 
 \begin{remarque} 
    quand on partage une quantité en 
    \begin{itemize}
       \item  \textbf{deux} parts égales, on obtient des \textbf{demis}, le \textbf{double} d'un demi fait un.
       \item \textbf{trois} parts égales, on obtient des \textbf{tiers}, le \textbf{triple} d'un tiers fait un.
       \item \textbf{quatre} parts égales, on obtient des \textbf{quarts}, le \textbf{quadruple} d'un quart fait un.
    \end{itemize}
 \end{remarque}