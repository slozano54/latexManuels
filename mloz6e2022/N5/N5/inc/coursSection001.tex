\opset{voperator=bottom,decimalsepsymbol={,},strikedecimalsepsymbol=\rlap{,}\rule[-1pt]{3pt}{0.4pt}}
\section{Écriture fractionnaire}
Le résultat de l'opération $7\div 5$ est appelé quotient de $7$ par $5$.\vspace*{-7mm}
\begin{remarques}
    \begin{itemize}
        \item on peut le calculer et obtenir son {\bfseries écriture décimale} : $1,4$.
        \item on peut ne pas le calculer et garder son {\bfseries écriture fractionnaire} : $\dfrac{7}{5}$
    \end{itemize}
\end{remarques}
\vspace*{-15mm}
\begin{minipage}[c]{0.2\linewidth}
	\pnode(2.7,0.2em){C}{\colorbox{red!30}{NUMÉRATEUR}}\\	
	\pnode(3.1,0.2em){D}{\colorbox{blue!30}{DÉNOMINATEUR}}
\end{minipage}
$\dfrac{\OPoval{A}{0,1}{\colorbox{red!30}{7}}}{\OPoval{B}{0,1}{\colorbox{blue!30}{5}}}$\qquad
\ncarc[arcangle=-15]{->}{A}{C}
\ncarc{->}{B}{D}
\begin{minipage}[c]{0.4\linewidth}
se lit "sept cinquièmes"\\
c'est une écriture fractionnaire du quotient
\end{minipage}
$\OPoval{G}{0,1}{\colorbox{red!30}{7}} \div \OPoval{H}{0,1}{\colorbox{blue!30}{5}}$\qquad
\begin{minipage}[c]{0.3\linewidth}
	\pnode(0,0.2em){E}{\colorbox{red!30}{DIVIDENDE}}
	\par\vspace{2.5cm}
	\pnode(0,0.2em){F}{\colorbox{blue!30}{DIVISEUR}}
\end{minipage}
\ncarc[arcangle=-15]{<-}{E}{G}
\ncarc{<-}{F}{H}
\vspace*{-10mm}
\begin{definition}
    Une écriture fractionnaire s'appelle {\bfseries fraction} lorsque le numérateur ET le dénominateur sont tous les deux des nombres entiers.
\end{definition}

\begin{exemples*1}
    \vspace*{2mm}
    \begin{itemize}
        \item $\dfrac{17}{100}$; $\dfrac{10}{18}$; $\dfrac{1}{1}$; $\dfrac{10175245}{9999999}$ sont des fractions. \vspace*{2mm}
        \item $\dfrac{17,100}{100}$; $\dfrac{18,200}{2,10}$; $\dfrac{200}{1,82}$ ne sont que des écritures fractionnaires.
    \end{itemize}
\end{exemples*1}

\begin{definition}
    Les {\bfseries fractions décimales} sont des fractions dont le dénominateur peut être égal à $10$, $100$, $\num{1000}$, $\num{10 000}$ \dots
\end{definition}

\begin{exemples*1}
    \vspace*{2mm}
    \begin{itemize}
        \item $\num{0.1}$ ($1$ dixième) est égal au quotient $\dfrac{1}{10}$; $\num{0.2}$ (2 dixièmes) est égal au quotient $\dfrac{2}{10}$;
        \item $\num{0.01}$ (1 centième) est égal au quotient $\dfrac{1}{100}$; \vspace*{2mm}
        \item Le nombre $\num{17.163}$ est égal à 1 dizaine et 7 unités et 1dixième et 6 centièmes et 3 millièmes.
        \par\vspace*{2mm}
        c'est à dire $\num{17.163}=1\times 10+7\times 1+1\times \dfrac{1}{10}+6\times \dfrac{1}{100} + 3\times \dfrac{1}{\num{1000}}$
        \par
        Compléter : $\num{357.246}=\ldots \times 100+\ldots \times 10+\ldots \times 1+\ldots \times \dfrac{1}{10}+\ldots \times \dfrac{1}{100} + \ldots \times \dfrac{1}{\num{1000}}$
    \end{itemize}
\end{exemples*1}

\begin{myBox}{\emoji{light-bulb} Aïe Aïe Aïe}
    Un nombre entier peut toujours s'écrire comme une fraction.
    $$45=\dfrac{45}1$$
    Mais $45$ peut aussi s'écrire $\dfrac{90}{2}$ ou $\dfrac{135}{3}$ ou $\dfrac{180}{4}$ ou \ldots
\end{myBox}
