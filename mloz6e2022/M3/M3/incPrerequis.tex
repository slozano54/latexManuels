\vspace*{-8mm}
%pre-001
\begin{prerequis}[Connaisances \emoji{red-heart} et compétences \emoji{diamond-suit} du cycle 3]    
   \begin{itemize}        
       \item[\emoji{red-heart}] Vocabulaire associé à ces objets et à leurs propriétés : côté, sommet, angle, hauteur.
       \columnbreak
       \item[\emoji{diamond-suit}] Reconnaître, nommer, décrire des triangles, dont les triangles particuliers (triangle rectangle, triangle isocèle, triangle équilatéral).       
   \end{itemize}
\end{prerequis}
\vspace*{-5mm}
% \begin{debat}[le SI (Système International)]
%     \vspace*{-7mm}
%     En 1795, il existe en France plus de 700 {\bf unités de mesures différentes} qui varient d'une ville à l'autre. Source d'erreurs et de fraudes lors des transactions commerciales, politiques et scientifiques vont tenter de réformer cet état de fait : leur idée est d'assurer l'invariabilité des mesures en les rapportant à un étalon universel emprunté à un phénomène naturel. Le 26 mars 1791 nait le mètre (du grec {\it metron}, mesure), dont la longueur est établie comme égale à la dix-millionième partie du quart du méridien terrestre. L'unité de mesure de base étant déterminée, il suffit désormais d'établir toutes les autres unités de mesure qui en découlent : le mètre carré et le mètre cube, le litre, le gramme\dots{} Le système international des unités (SI) est né en 1960. En 2018, les unités de base sont redéfinies à partir de sept constantes physiques. \\
%     \begin{center}
%        {\psset{unit=0.75}
%        \begin{pspicture}(-2.5,-2)(2.5,1.8)
%           \pscircle*[linecolor=gray](0,0){2.5}
%           \pswedge*[linecolor=orange](0,0){2}{0}{51}
%           \rput(1.25;25){\large\white m}
%           \pswedge*[linecolor=red](0,0){2}{51}{103}
%           \rput(1.25;75){\large\white kg}
%           \pswedge*[linecolor=magenta](0,0){2}{103}{154}
%           \rput(1.25;126){\large\white cd}
%           \pswedge*[linecolor=violet](0,0){2}{154}{205}
%           \rput(1.25;176){\large\white mol}
%           \pswedge*[linecolor=blue](0,0){2}{205}{257}
%           \rput(1.25;228){\large\white K}
%           \pswedge*[linecolor=cyan](0,0){2}{257}{308}
%           \rput(1.25;280){\large\white A}
%           \pswedge*[linecolor=green](0,0){2}{308}{360}
%           \rput(1.25;334){\large\white s}
%        \end{pspicture}}
%     \end{center}
%     \smallskip    
%     \begin{cadre}[B2][F4]
%        \begin{center}
%           \hrefVideo{https://www.youtube.com/watch?time_continue=2&v=bInHclEN6zQ&feature=emb_logo}{\bf Système International d'unités. L'épopée}
          
%           \hfill{\it Laboratoire national de métrologie et d'essais}. 
%        \end{center}
%     \end{cadre}
%  \end{debat}