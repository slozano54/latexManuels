
\section{Les angles}
\begin{definition}
    Un \textbf{angle} est une portion du plan délimitée par deux droites.
 \end{definition}
 
 \begin{pspicture}(-6,-0.3)(4,2.7)
    {\psset{yunit=0.6}
    \small
    \rput(0,-0.3){A}
    \rput(4,3.6){B}
    \rput(0,2.35){C}
    \rput(4,0.4){D}
    \rput(1.4,1){O}
    \psline(0,0)(4,4)
    \psline(0,2)(4,0)
    \psarc[linecolor=A1](1.4,1.4){1}{-20}{30}
    \psline[linecolor=B2]{->}(-1,1.1)(1.3,1.3)
    \rput(-1.8,1.3){\textcolor{B2}{sommet}}
    \rput(3.5,1.7){\textcolor{A1}{angle $\widehat{BOD}$}}}
 \end{pspicture}
 
 En général, on marque l'angle considéré par un arc de cercle. Pour nommer un angle, il suffit de connaître son sommet et un point de chacun de ses côtés.
 
 \begin{exemple}
    \begin{pspicture}(-1,-0.5)(3,2)
       \psset{unit=1.2}
       \pspolygon(0,0)(3,0)(3,1.5)
       \rput(0,-0.3){A}
       \rput(3,-0.3){B}
       \rput(3,1.8){C}
       \psarc[linecolor=A1,doubleline=true](0,0){0.7}{0}{27}
       \psarc[linecolor=J1,linestyle=dashed](3,0){0.5}{90}{180}
       \psarc[linecolor=B1](3,1.5){0.5}{205}{270}
    \end{pspicture}
    \correction
    Dans le triangle (trois angles) ABC, on a les angles :
    \begin{itemize}
       \item \textcolor{A1}{$\widehat{BAC} =\widehat{CAB}$} ;
       \item \textcolor{J1}{$\widehat{ABC} =\widehat{CBA}$} ;
       \item \textcolor{B1}{$\widehat{BCA} =\widehat{ACB}$}.
    \end{itemize}
 \end{exemple}