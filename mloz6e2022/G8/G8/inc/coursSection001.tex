\section{Axe de symétrie d'une figure : Définition}
\begin{definition}[Axe de symétrie]
    Un \textbf{axe de symétrie} pour une figure est une \textbf{droite} qui partage cette figure en deux parties superposables.
\end{definition}
\begin{exemples*1}
    \phantom{rrr}

    \begin{minipage}{0.35\linewidth}
        Sur la figure ci-contre, la droite en gras est un \textbf{axe de symétrie} alors que les autres droites n'en sont pas.
    \end{minipage}
    \hspace{1cm}
    \begin{minipage}{0.6\linewidth}
        \includegraphics[scale=0.5]{\currentpath/images/6eme-G5-def-axe-sym} 
    \end{minipage}
    \hrule
    \begin{minipage}{0.45\linewidth}
        \begin{center}
            \includegraphics[scale=0.5]{\currentpath/images/6eme-G5-def-axe-sym-1}\\
            Cette figure a \textbf{un} axe de symétrie.
        \end{center}
    \end{minipage}
    \vrule
    \begin{minipage}{0.45\linewidth}
        \begin{center}
            \includegraphics[scale=0.5]{\currentpath/images/6eme-G5-def-axe-sym-2}\\ 
            Cette figure n'a pas d'axe de symétrie.
        \end{center}
    \end{minipage}
\end{exemples*1}

