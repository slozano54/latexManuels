\section{Calcul en ligne et en colonnes}

Pour effectuer une addition ou une soustraction, on peut : calculer mentalement ; calculer en ligne en posant l'opération ; calculer en colonnes en posant l'opération ; calculer grâce à une calculatrice.

\begin{methode}[Priorités opératoires dans un calcul en ligne]
   Dans un calcul comportant des parenthèses, additions et soustractions, on effectue en priorité les calculs dans les parenthèses les plus intérieures, puis les additions/soustractions de gauche à droite. S'il n'y a que des additions, on peut faire les calculs dans l'importe quel {\small ordre}
   \exercice
   Calculer la valeur de $A$ :
   $A =5+3+(15-(9+2))$ \\
   \correction
      $A =5+3+(15-\underline{(9+2)}) =5+3+\underline{(15-\psframebox*[fillcolor=yellow]{11})}$ \\
      $\phantom{A} =\underline{5+3}+\psframebox*[fillcolor=yellow]{4} =\underline{\psframebox*[fillcolor=yellow]{8}+4} =\psframebox*[fillcolor=yellow]{12}$
\end{methode}
 
\begin{propriete}
   Lorsqu'on calcule une somme ou une différence en colonnes, on aligne les chiffres de même rang les uns au-dessus des autres.
\end{propriete}

\begin{exemple}
   Calculer  : \\
   $27,89+1\,298,7$ \\
   $85,8-34,54$.
   \correction
   \opadd[voperation=center,resultstyle=\red,carrystyle=\footnotesize\blue,decimalsepsymbol={,}]{1298,7}{27,89}
   \qquad
   \opsub[voperation=center,resultstyle=\red,carrystyle=\footnotesize\blue,carrysub,lastcarry,columnwidth=2.5ex,offsetcarry=-0.4,decimalsepsymbol={,}]{85,8}{34,54}
   \; ou \;
   \opsub[voperation=center,resultstyle=\red,columnwidth=2.5ex,offsetcarry=-0.4,decimalsepsymbol={,}]{85,8}{34,54} 
   \rput(-0.65,0.7){\textcolor{blue}{/}}
   \rput(-0.65,1.1){\textcolor{blue}{\small 7}}
   \rput(-0.42,0.71){\textcolor{blue}{\small 1}}
\end{exemple}
