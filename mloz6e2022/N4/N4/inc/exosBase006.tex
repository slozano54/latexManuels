\begin{exercice}
    Pour chaque problème, écrire en ligne la (ou les) opération(s) à faire pour le résoudre puis effectuer le calcul à la calculatrice.
    \begin{enumerate}
       \item La somme de deux nombres vaut $\num{78.92}$. Un des deux nombres est $\num{29.6}$. Quel est le second nombre ?
       \item La différence de deux nombres est $\num{68.72}$. Un des deux nombres est $\num{70.35}$. Quel est le second nombre ?
       \item Philippe fait une randonnée de $\Lg[km]{13.7}$. Il a parcouru $\Lg[km]{8.6}$ le matin.
        Combien lui reste-t-il à parcourir ?
       \item Un manteau coûte $\Prix{56.80}$. Le commerçant me fait une remise de $\Prix{12.40}$. Combien vais-je payer ?
       \item Noé veut acheter un livre. Il a $\Prix{12.42}$ mais il lui manque $\Prix{3,45}$. Quel est le prix du livre ?   
    \end{enumerate}
\end{exercice}
\begin{corrige}
    Pour chaque problème, écrire en ligne la (ou les) opération(s) à faire pour le résoudre puis effectuer le calcul à la calculatrice.
    \begin{enumerate}
       \item La somme de deux nombres vaut $\num{78.92}$. Un des deux nombres est $\num{29.6}$. Quel est le second nombre ?
       
       {\red $\num{78.92} - \num{29.6}$ le second nombre vaut donc $\num{49.32}$}
       \item La différence de deux nombres est $\num{68.72}$. Un des deux nombres est $\num{70.35}$. Quel est le second nombre ?
       
       {\red Attention ici, il y a deux possibilités.
       \begin{itemize}
        \item $\num{70.35} + \num{68.72}$ donc le second nombre vaut $\num{139.07}$.
        \item $\num{70.35} - \num{68.72}$ donc le second nombre vaut $\num{1.63}$.
       \end{itemize}       
       }
       \item Philippe fait une randonnée de $\Lg[km]{13.7}$. Il a parcouru $\Lg[km]{8.6}$ le matin.
        Combien lui reste-t-il à parcourir ?

        {\red $\Lg[km]{13.7} - \Lg[km]{8.6}$ donc il reste $\Lg[km]{5.1}$}
       \item Un manteau coûte $\Prix{56.80}$. Le commerçant me fait une remise de $\Prix{12.40}$. Combien vais-je payer ?
       
       {\red $\Prix{56.80} - \Prix{12.40}$ le prix sera donc de $\Prix{44.40}$}
       \item Noé veut acheter un livre. Il a $\Prix{12.42}$ mais il lui manque $\Prix{3,45}$. Quel est le prix du livre ?   
       
       {\red $\Prix{12.42} + \Prix{3.45}$ le livre coûte donc $\Prix{15.87}$}
    \end{enumerate}
\end{corrige}
