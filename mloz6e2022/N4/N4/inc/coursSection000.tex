% \section{Section 1}
% \subsection{Sous-section 1.1}
% \begin{definition}[Titre optionnel]
%     Dans le cours, on utilise assez souvent des cadres du type
%     définition (comme ici par exemple).    
% \end{definition}
% \begin{remarque}
%     Ceci est une remarque utilisant une commande du paquet profcollege.
    
%     \begin{center}
%       Truc centré
%     \end{center}


% \end{remarque}
% \begin{propriete}[Titre optionnel]
%   Dans le cours, on utilise assez souvent des cadres du type
%   définition, comme ici par exemple pour une propriete.
% \end{propriete}
% \begin{remarques}
%   \begin{itemize}
%     \item remarque.
%     \item remarque.
%   \end{itemize}
% \end{remarques}

% \subsection{Sous-section 1.2}
% \begin{theoreme}[Titre optionnel]
%   Dans le cours, on utilise assez souvent des cadres du type
%   définition, comme ici par exemple pour un théorème.
% \end{theoreme}
% \begin{notation}
%   notation
% \end{notation}
% \begin{notations}
%   \begin{itemize}
%     \item notation.
%     \item notation.
%   \end{itemize}
% \end{notations}
% \begin{preuve}
%   Ceci est une preuve\par Deuxième ligne de la preuve
% \end{preuve}
% \begin{exemple}
%   Texte de l’exemple
%   \correction
  
% \end{exemple}

% \begin{exemple*1}
%   \phantom{rrr}
%   Texte

%   \correction
%   \phantom{rrr}
%   Texte  
  
% \end{exemple*1}

% \begin{exemple}[0.6]
%   Texte de l’exemple très long sur une ligne, très très très long.
%   On peut modifier la répartition horizontale  à l'aide d'un argument optionnel valant par défaut 0,4, valant ici 0,6.
%   \correction
%   Texte de la correction en vis à vis
% \end{exemple}
% \section{Section 2}
% \subsection{Sous-section 2.1}
% Quatre affichages prévus pour les méthodes.

% \begin{methode}[Titre de la méthode]
%     \Papiers[Largeur=10,Hauteur=5,Couleur=Olive]
    
%     Texte introductif
%     \exercice
%     Texte de l’exercice
%     \correction
%     Texte de la correction sur un minimum de trois lignes pour faire la
%     différence entre vis-à-vis et double colonne. C’est l’endroit de la
%     coupure qui va différer.
% \end{methode}

% \begin{methode*1}[Titre de la méthode*1]
%     Texte introductif
%     \exercice
%     Texte de l’exercice
%     \correction
%     Texte de la correction sur un minimum de trois lignes pour faire la
%     différence entre vis-à-vis et double colonne. C’est l’endroit de la
%     coupure qui va différer.
% \end{methode*1}

% \subsection{Sous-section 2.2}
% \begin{methode*2}[Titre de la méthode*2]
%     Texte introductif
%     \exercice
%     Texte de l’exercice
%     \correction
%     Texte de la correction sur un minimum de trois lignes pour faire la
%     différence entre vis-à-vis et double colonne. C’est l’endroit de la
%     coupure qui va différer.
% \end{methode*2}

% \begin{methode*2*2}[Dernière méthode  \MethodeRefExercice{exoN1-exemple1} \MethodeRefExercice{exoN1-exemple2}]
%     \exercice
%     \label{methodeN1-exemple}
%     Texte du premier exercice
%     \correction
%     Correction du premier exercice
%     \exercice
%     Texte du deuxième exercice
%     \correction
%     Texte de la correction du deuxième exercice sur un minimum de trois
%     lignes pour faire la différence entre vis-à-vis et double
%     colonne. C’est l’endroit de la coupure qui va différer.
% \end{methode*2*2}