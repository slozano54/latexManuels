\section{Ordre de grandeur}

\begin{definition}
   Un {\bf ordre de grandeur} d'un nombre est une valeur approchée de ce nombre.
\end{definition}

\begin{exemple*1}
   Ordre de grandeur de $392,5+703,56$ : \\
   une valeur approchée de 392,5 est 400 et une valeur approchée de 703,56 est 700 donc, un ordre de grandeur de $392,5+703,56$ est $400+700 =1\,100$.
\end{exemple*1}
