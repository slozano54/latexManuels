\vspace*{-17mm}
\begin{activite}[Propriétés de la symétries axiale]
    \begin{changemargin}{-10mm}{-15mm}
        \vspace*{-8mm}
        {\bf Objectifs :} observer l'effet d'une symétrie axiale sur les longueurs, les angles, le parallélisme, l'alignement, les aires ; utiliser un logiciel de géométrie dynamique.
        \partie[construction de la figure]
            \vspace*{-3mm}
            Ouvrir {\bf Geogebra} et choisir l'onglet \textbf{Géométrie}.
            \begin{center}
                {\newcolumntype{M}{>{\itshape\footnotesize}p{4.5cm}}
                \small
                \renewcommand{\arraystretch}{1.3}
                \begin{tabular}{|c|p{5.5cm}|p{3cm}|M|}
                \hline
                & Objet à tracer & Outils & Instructions \\
                \hline
                1 & \multicolumn{3}{l|}{Construction de la {\bf figure}} \\
                & Tracer un segment $[AB]$ & Segment & cliquer en deux points du plan \\
                & Tracer le cercle de centre $C$ passant par $A$ & Cercle (centre-point) & cliquer en un point quelconque puis sur $A$ \\
                & Tracer la parallèle à $(AB)$ passant par $C$ & Parallèle & cliquer sur le segment $[AB]$ puis sur $C$ \\
                & Tracer la perpendiculaire à $(AB)$ passant par $B$ & Perpendiculaire & cliquer sur le segment $[AB]$ puis sur $B$ \\
                & Les deux droites se coupent en $D$ & Intersection & cliquer sur les deux droites \\
                & Construire le quadrilatère $ABDC$ & Polygone & cliquer successivement sur $A, B, D$ et $C$ \\
                & Effacer les droites $(BD)$ et $(CD)$ & Clic droit sur la droite & décocher \og Afficher objet \fg \\
                \hline
                2 & \multicolumn{3}{l|}{Construction de la figure {\bf symétrique} par rapport à la droite $(EF)$} \\
                & Tracer en rouge la droite $(EF)$ & Droite & cliquer en deux points puis propriétés \\
                & Tracer le symétrique du cercle & Symétrie axiale & sélectionner le cercle puis la droite $(EF)$ \\
                & Tracer le symétrique du quadrilatère & Symétrie axiale & sélectionner le quadrilatère puis $(EF)$ \\
                \hline
                3 & \multicolumn{3}{l|}{Faire apparaître différentes {\bf mesures}} \\
                & Mesurer la longueur de $[AB]$ et de $[A'B']$ & Distance & cliquer sur les segments $[AB]$ et $[A'B']$ \\
                & Mesurer les angles $\widehat{ACD}$ et $\widehat{A'C'D'}$ & Angle & cliquer sur les trois points de l'angle \\
                & Mesurer l'aire de $ABDC$ et de $A'B'C'D'$ & Aire & cliquer à l'intérieur des quadrilatères \\
                \hline
                \end{tabular}}
            \end{center}   
            \vspace*{-2mm}             
            \partie[observations]
                \vspace*{-7mm}
                \begin{enumerate}
                    \item Observer ce qu'il se passe lorsque l'on déplace l'un des points du quadrilatère $ABCD$ ou de l'axe de symétrie. \smallskip
                    \item En quelle figure géométrique est transformé un segment ? un cercle ? un quadrilatère ? \pointilles \smallskip
                    \item Comparer la longueur des segments $[AB]$ et $[A'B']$ : \pointilles \smallskip
                    \item Comparer la mesure des angles $\widehat{ACD}$ et $\widehat{A'C'D'}$ : \pointilles \smallskip
                    \item Comparer l'aire des deux quadrilatères : \pointilles \smallskip
                    \item Les deux droites parallèles / perpendiculaires dans la figure d'origine le restent-elles dans la figure symétrique ? \smallskip
                \end{enumerate}                
            \partie[conjectures]
                \vspace*{-3mm}
                Conjecturer des propriétés de symétrie centrale : \pointilles

                \pointilles

                \pointilles
    \end{changemargin}
    \vspace*{-10mm}
\end{activite}
