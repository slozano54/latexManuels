\begin{exercice*}
   Un quadrilatère $ABCD$ est appelé isocerfvolant en $A$ si l'angle $\widehat{BAD}$ est droit et si la droite $(AC)$ est un axe de symétrie.
   \begin{center}
      \begin{Geometrie}
         pair A,B,C,D;
         A=u*(1,3);
         B-A=u*(2,-2);
         C-A=u*(6,0);
         D=symetrie(B,A,C);
         trace polygone(A,B,C,D);
         trace segment(A+0.3(A-C),C+0.2(C-A)) dashed dashpattern(on6bp off3bp on1.5bp off3bp) withpen pencircle scaled 1bp withcolor red;
         trace codeperp(B,A,D,8);
         label.ulft(TEX("$A$"),A);
         label.bot(TEX("$B$"),B);
         label.urt(TEX("$C$"),C);
         label.top(TEX("$D$"),D);
      \end{Geometrie}
   \end{center}
\begin{enumerate}
   \item
   \begin{enumerate}
      \item Démontrer que $\widehat{DAC} =\widehat{BAC}$
      \item En déduire la mesure de l'angle $\widehat{DAC}$
      \item Quelle est la position relative des droites $(BD)$ et $(AC)$ ?
   \end{enumerate}
   \item
   \begin{enumerate}
      \item Construire un quadrilatère $ABCD$ qui soit un isocerfvolant en $A$.
      \item Construire un quadrilatère qui admette un axe de symétrie et qui ne soit pas un isocerfvolant.
   \end{enumerate}
   \item Les affirmations suivantes sont-elles vraies ou fausses ? Justifier les réponses.
   \begin{enumerate}
      \item Tous les carrés sont des isocerfvolants.
      \item Tous les rectangles sont des isocerfvolants.
   \end{enumerate}
\end{enumerate}
\end{exercice*}
\begin{corrige}
   Pas de corrigé.
\end{corrige}
