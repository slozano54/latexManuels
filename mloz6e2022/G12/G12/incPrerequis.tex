\vspace*{-7mm}
\begin{changemargin}{-10mm}{-10mm}
%pre-001
\begin{prerequis}[Connaisances \emoji{red-heart} et compétences \emoji{diamond-suit} du cycle 3]    
   \begin{itemize}        
       \item[\emoji{red-heart}] Vocabulaire associé à ces objets et à leurs propriétés : côté, sommet, angle, hauteur.
       \columnbreak
       \item[\emoji{diamond-suit}] Reconnaître, nommer, décrire des triangles, dont les triangles particuliers (triangle rectangle, triangle isocèle, triangle équilatéral).       
   \end{itemize}
\end{prerequis}
\end{changemargin}
\vspace*{-13mm}
\begin{debat}[Des films pour les vacances...] 
   Pour clore cette année, voici une liste de films à base mathématique fondés sur des histoires vraies.
   \begin{itemize}
       \item {\bf Les figures de l'ombre} (2017) : le destin extraordinaire des trois femmes scientifiques afro-américaines qui ont permis aux États-Unis de prendre la tête de la conquête spatiale.
       \item {\bf Imitation game} (2015) : l'histoire d'Alan Turing, mathématicien, qui contribua à changer le cours de la Seconde Guerre mondiale et de l’Histoire en étant chargé par le gouvernement britannique de percer le secret de la célèbre machine de cryptage allemande Enigma, réputée inviolable.
           \item {\bf L'homme qui défiait l'infini} (2015) : la vie de Srinivasa Ramanujan, un des plus grands mathématiciens de notre temps. Élevé à Madras en Inde, il intègre la prestigieuse université de Cambridge en Angleterre pendant la Première Guerre mondiale et y développe de nombreuses théories mathématiques.
      \item {\bf Le monde de Nathan} (2014) : les aventures de Nathan, un adolescent souffrant de troubles autistiques et prodige en mathématiques. Brillant mais asocial, il tisse une amitié étonnante avec son professeur anticonformiste qui le pousse à intégrer l’équipe britannique et participer aux Olympiades Internationales de Mathématiques.
   \end{itemize}
   %  \bigskip
   %  \begin{cadre}[B2][J4]
   %     \begin{center}
   %        \hrefVideo{https://www.yout-ube.com/watch?v=eawBr43xWf8}{\bf ...}
   %     \end{center}
   %  \end{cadre}
 \end{debat}