\section{Surface et aire}

\begin{definition}
   La \textbf{surface} d'une figure est la partie située à l'intérieur de son contour. \\
   Sa mesure s'appelle l'\textbf{aire}, qui est le nombre d'unités d'aire que la figure contient.
\end{definition}

\begin{remarque}
   attention à ne pas confondre avec le périmètre qui est une mesure de longueur !
\end{remarque}

Pour déterminer l'aire d'une surface, on peut découper la figure en figures simples ou utiliser un pavage simple.

\begin{exemple*1}
\ \\
   {\psset{unit=0.7}
   \begin{pspicture}(-3,-0.5)(10,6.6)
      \put(1,1){\pspolygon[fillstyle=solid,fillcolor=B2,linewidth=0.1](0,0)(2,0)(2,3)(0,3)(0,2)(1,2)(1,1)(0,1)(0,0)}
      \put(4,1){\pspolygon[fillstyle=solid,fillcolor=A2,linewidth=0.1](1,0)(2,0)(2,2)(1,2)(1,3)(0,3)(0,1)}
      \put(7,1){\pspolygon[fillstyle=solid,fillcolor=J2,linewidth=0.1](1,0)(1.5,0.5)(2,0)(2,1)(3,1)(2.5,1.5)(3,2)(2,2)(2,3)(1.5,2.5)(1,3)(1,2)(0,2)(0.5,1.5)(0,1)(1,1)(1,0)}
      \rput(2.5,2.4){\textbf{A}}
      \rput(5,2.4){\textbf{B}}
      \rput(8.5,2.4){\textbf{C}}
      \rput(1.5,5.5){{$u_1$}} 
      \psframe[fillstyle=solid,fillcolor=darkgray,linewidth=0.1](2,5)(3,6)
      \rput(4.5,5.5){{$u_2$}}
      \pspolygon[fillstyle=solid,fillcolor=darkgray,linewidth=0.1](6,5)(6,6)(5.5,5.5)
      \rput(7.5,5.5){{$u_3$}}\pspolygon[fillstyle=solid,fillcolor=darkgray,linewidth=0.1](8,5)(9,5)(8,6)
      \psgrid[subgriddiv=0,gridlabels=0pt,gridwidth=0.02,gridcolor=darkgray](11,7)
      \multido{\i=0+1}{5}{%
	\FPeval{a}{\i+7}
	\psline[linecolor=darkgray,linewidth=0.02](\i,0)(\a,7)
	\psline[linecolor=darkgray,linewidth=0.02](\i,7)(\a,0)
	}
      \multido{\i=1+1}{6}{%
	\FPeval{b}{7-\i}
	\FPeval{c}{4+\i}
	\psline[linecolor=darkgray,linewidth=0.02](0,\i)(\b,7)
	\psline[linecolor=darkgray,linewidth=0.02](\i,0)(0,\i)
	\psline[linecolor=darkgray,linewidth=0.02](\c,0)(11,\b)
	\psline[linecolor=darkgray,linewidth=0.02](\c,7)(11,\i)
	}
   \end{pspicture}}
   \correction   
   Lorsque l'on n'a pas une unité d'aire entière $u_1$, on prend une partie de l'unité : 
   \begin{itemize}
      \item $u_2$ correspond à la moitié d'un carré $=\dfrac12 =0,5$ ;
      \item $u_3$ correspond au quart d'un carré $=\dfrac14 =0,25$. \smallskip
   \end{itemize}
   On peut aussi \og découper \fg{} une partie de la figure afin de la déplacer ailleurs pour former une unité d'aire.
   \smallskip
   \begin{center}
      \begin{cltableau}{0.9\linewidth}{4}
         \hline
         Unité & fig. A & fig. B & fig. C \\
         \hline
            $u_1$ & $5$ & $4,5$ & $4$ \\
         \hline
         $u_2$ & $20$ & $18$ & $16$ \\
         \hline
         $u_3$ & $10$ & $9$ & $8$ \\
         \hline
      \end{cltableau}
   \end{center}
\end{exemple*1}
