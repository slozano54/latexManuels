% Les enigmes ne sont pas numérotées par défaut donc il faut ajouter manuellement la numérotation
% si on veut mettre plusieurs enigmes
%\refstepcounter{exercice}
%\numeroteEnigme
\begin{enigme}[Le retour du Curvica !]

   \partie[les 24 formes du Curvica]

   \curvica{\rput(1,1){\textcolor{B1}{a}} \psline(0,0)(0,2) \psline(0,2)(2,2) \psline(2,2)(2,0) \psarc(1,-2){2.24}{63.4}{116.6}}
   \curvica{\rput(1,1){\textcolor{B1}{b}} \psline(0,0)(0,2) \psline(0,2)(2,2) \psarc(4,1){2.24}{153.4}{-153.4} \psarc(1,2){2.24}{-116.6}{-63.4}}
   \curvica{\rput(1,1){\textcolor{B1}{c}} \psarc(2,1){2.24}{153.4}{-153.4} \psline(2,0)(2,2) \psline(0,2)(2,2) \psarc(1,2){2.24}{-116.6}{-63.4}}
   \curvica{\rput(1,1){\textcolor{B1}{d}} \psline(0,0)(0,2) \psline(2,0)(2,2) \psline(0,2)(2,2) \psline(0,0)(2,0)} \\ [1mm]
   
   \curvica{\rput(1,1){\textcolor{B1}{e}} \psline(0,0)(0,2) \psline(2,0)(2,2) \psarc(1,0){2.24}{63.4}{116.6} \psarc(1,2){2.24}{-116.6}{-63.4}}
   \curvica{\rput(1,1){\textcolor{B1}{f}} \psline(0,0)(0,2) \psline(2,0)(2,2) \psarc(1,4){2.24}{-116.6}{-63.4} \psarc(1,2){2.24}{-116.6}{-63.4}}
   \curvica{\rput(1,1){\textcolor{B1}{g}} \psline(0,0)(0,2) \psline(2,0)(2,2) \psarc(1,4){2.24}{-116.6}{-63.4} \psarc(1,-2){2.24}{63.4}{116.6}}
   \curvica{\rput(1,1){\textcolor{B1}{h}} \psline(0,0)(0,2) \psline(2,0)(2,2) \psline(0,2)(2,2) \psarc(1,2){2.24}{-116.6}{-63.4}} \\ [1mm]
   
   \curvica{\rput(1,1){\textcolor{B1}{i}} \psline(0,0)(0,2) \psarc(1,4){2.24}{-116.6}{-63.4} \psarc(4,1){2.24}{153.4}{-153.4} \psarc(1,2){2.24}{-116.6}{-63.4}} 
   \curvica{\rput(1,1){\textcolor{B1}{j}} \psarc(2,1){2.24}{153.4}{-153.4} \psarc(1,4){2.24}{-116.6}{-63.4} \psarc(4,1){2.24}{153.4}{-153.4} \psarc(1,-2){2.24}{63.4}{116.6}}
   \curvica{\rput(1,1){\textcolor{B1}{k}} \psarc(2,1){2.24}{153.4}{-153.4} \psarc(0,1){2.24}{-26.6}{26.6} \psarc(1,0){2.24}{63.4}{116.6} \psarc(1,2){2.24}{-116.6}{-63.4}}
   \curvica{\rput(1,1){\textcolor{B1}{l}} \psarc(-2,1){2.24}{-26.6}{26.6} \psline(2,0)(2,2) \psarc(1,4){2.24}{-116.6}{-63.4} \psarc(1,-2){2.24}{63.4}{116.6}} \\ [1mm]
   
   \curvica{\rput(1,1){\textcolor{B1}{m}} \psline(0,0)(0,2) \psarc(0,1){2.24}{-26.6}{26.6} \psarc(1,4){2.24}{-116.6}{-63.4} \psarc(1,-2){2.24}{63.4}{116.6}} 
   \curvica{\rput(1,1){\textcolor{B1}{n}} \psarc(-2,1){2.24}{-26.6}{26.6} \psarc(4,1){2.24}{153.4}{-153.4} \psarc(1,0){2.24}{63.4}{116.6} \psarc(1,2){2.24}{-116.6}{-63.4}}
   \curvica{\rput(1,1){\textcolor{B1}{o}} \psarc(2,1){2.24}{153.4}{-153.4} \psarc(1,4){2.24}{-116.6}{-63.4} \psarc(4,1){2.24}{153.4}{-153.4} \psarc(1,2){2.24}{-116.6}{-63.4}}
   \curvica{\rput(1,1){\textcolor{B1}{p}} \psarc(2,1){2.24}{153.4}{-153.4} \psline(2,0)(2,2) \psarc(1,0){2.24}{63.4}{116.6} \psarc(1,-2){2.24}{63.4}{116.6}} \\ [1mm]
   
      \curvica{\rput(1,1){\textcolor{B1}{q}} \psline(0,0)(0,2) \psarc(4,1){2.24}{153.4}{-153.4} \psarc(1,0){2.24}{63.4}{116.6} \psarc(1,2){2.24}{-116.6}{-63.4}}
      \curvica{\rput(1,1){\textcolor{B1}{r}} \psarc(2,1){2.24}{153.4}{-153.4} \psarc(0,1){2.24}{-26.6}{26.6} \psarc(1,4){2.24}{-116.6}{-63.4} \psarc(1,2){2.24}{-116.6}{-63.4}}
      \curvica{\rput(1,1){\textcolor{B1}{s}} \psarc(-2,1){2.24}{-26.6}{26.6} \psarc(1,4){2.24}{-116.6}{-63.4} \psarc(4,1){2.24}{153.4}{-153.4} \psarc(1,-2){2.24}{63.4}{116.6}}
      \curvica{\rput(1,1){\textcolor{B1}{t}} \psarc(2,1){2.24}{153.4}{-153.4} \psline(2,0)(2,2) \psarc(1,0){2.24}{63.4}{116.6} \psarc(1,2){2.24}{-116.6}{-63.4}} \\ [1mm]
      
      \curvica{\rput(1,1){\textcolor{B1}{u}} \psline(0,0)(0,2) \psarc(1,4){2.24}{-116.6}{-63.4} \psarc(4,1){2.24}{153.4}{-153.4} \psline(0,0)(2,0)} 
      \curvica{\rput(1,1){\textcolor{B1}{v}} \psarc(2,1){2.24}{153.4}{-153.4} \psarc(1,4){2.24}{-116.6}{-63.4} \psarc(4,1){2.24}{153.4}{-153.4} \psline(0,0)(2,0)}
      \curvica{\rput(1,1){\textcolor{B1}{w}} \psarc(2,1){2.24}{153.4}{-153.4} \psarc(1,0){2.24}{63.4}{116.6} \psarc(4,1){2.24}{153.4}{-153.4} \psline(0,0)(2,0)}
      \curvica{\rput(1,1){\textcolor{B1}{x}} \psarc(2,1){2.24}{153.4}{-153.4} \psarc(1,4){2.24}{-116.6}{-63.4} \psline(2,0)(2,2) \psline(0,0)(2,0)}

\pagebreak

    \partie[la grille réponse] \bigskip

    {\renewcommand{\arraystretch}{1.75}
    \begin{tabular}{|p{11cm}|>{\centering\arraybackslash}p{3cm}|>{\centering\arraybackslash}p{1cm}|}
    \hline
    Défis & Réponses & Points \\
    \hline
    \multicolumn{2}{|c|}{Niveau facile} & 1 pt \\
    \hline
    1. Trouver la pièce dont l'aire est la plus grande. & & \\
    \hline
    2. Trouver la pièce dont le périmètre est le plus petit. & & \\
    \hline
    3. Assembler deux pièces pour obtenir un rectangle. & & \\
    \hline
    4. Trouver la pièce de plus grand périmètre et de plus petite aire. & & \\
    \hline
    5. Trouver une pièce ayant un seul axe de symétrie. & & \\
    \hline
    6. Trouver une pièce ayant exactement deux axes de symétrie. & & \\
    \hline
    7. Trouver deux pièces de même périmètre mais d'aires différentes. & & \\
    \hline
    \hline
    \multicolumn{2}{|c|}{Niveau moyen} & 1,5 pt \\
    \hline
    8. Trouver deux pièces différentes de même aire mais de périmètres différents. & & \\
    \hline
    9. Trouver deux pièces différentes de même aire et de même périmètre. & & \\
    \hline
    10. Assembler quatre pièces pour obtenir un carré. & & \\
    \hline
    11. Trouver deux pièces différentes ayant même aire, même périmètre et au moins un axe de symétrie chacune. & & \\
    \hline
    12. Trouver deux pièces dont l'une a un périmètre plus grand que l'autre mais une aire plus petite. & & \\
    \hline
    13. Assembler deux pièces pour obtenir une figure dont l'aire et le périmètres sont les plus grands possibles. & & \\
    \hline
    \hline
    \multicolumn{2}{|c|}{Niveau difficile} & 2 pts \\
    \hline
    14. Trouver deux pièces ayant ni axe de symétrie, ni même périmètre, ni même aire. & & \\
    \hline
    15. Assembler cinq ou six pièces pour obtenir un rectangle. & & \\
    \hline
    \hline
    \multicolumn{2}{|r|}{Total sur 20 points} & \\
    \hline
    \end{tabular}}
\end{enigme}  
% % Pour le corrigé, il faut décrémenter le compteur, sinon il est incrémenté deux fois
% \addtocounter{exercice}{-1}
% \begin{corrige}
%     Correction enigme de la fin de la partie cours.  
%     
% \end{corrige}