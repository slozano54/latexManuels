\section{Calculer avec des formules}
\subsection{Quadilatères}

\scalebox{0.8}{
\begin{Geometrie}[CoinBG={(0,-.5u)},CoinHD={(5u,5.5u)}]    
    pair A,B,C,D;
    A=u*(0.75,0.75);
    B=u*(4.25,0.75);
    C=rotation(A,B,-90);
    D-A=C-B;
    trace A--B--C--D--cycle;
    trace cotationmil(D,C,5mm,5,btex $c$ etex) withcolor red;
    trace codeperp(A,B,C,5);
    trace codeperp(B,C,D,5);
    trace codeperp(C,D,A,5);
    trace codesegments(A,B,B,C,2);
    trace codesegments(C,D,D,A,2);
\end{Geometrie}
}
\begin{Geometrie}[CoinBG={(0,-.5u)},CoinHD={(5u,5.5u)}]    
    pair A,B,C,D;
    A=u*(0.75,0.75);
    B=u*(4.25,0.75);
    C=rotation(A,B,-90);
    D-A=C-B;
    trace A--B--C--D--cycle;
    trace cotationmil(D,C,5mm,5,btex $c$ etex) withcolor red;
    trace codeperp(A,B,C,5);
    trace codeperp(B,C,D,5);
    trace codeperp(C,D,A,5);
    trace codesegments(A,B,B,C,2);
    trace codesegments(C,D,D,A,2);
\end{Geometrie}

% \begin{center}
% \begin{tabular}{|m{8cm}|m{4.25cm}|c|}
% \hline
% \multicolumn{1}{|c|}{\bf Nom de la figure}&\multicolumn{1}{c|}{\bf Représentation}&\multicolumn{1}{c|}{\bf P\'erimètre et aire}\\
% \hline
% \textbf{ Trapèze} de petite base $b$, de grande base $B$ et de hauteur $h$&\includegraphics{aires_vol.3}&$\Eqalign{
% {\cal P}&=\mbox{somme des côtés}\cr
% {\cal A}&=\frac{(B+b)\times h}{2}\cr
% }$\\
% \hline
% \textbf{ Parallélogramme} de côté $c$ et de hauteur relative à ce côté $h$&\includegraphics{aires_vol.4}&$\Eqalign{
% {\cal P}&=\mbox{somme des côtés}\cr
% {\cal A}&=c\times h\cr
% }$\\
% \hline
% \textbf{ Losange} de côté $c$, de grande diagonale $D$ et de petite diagonale $d$&\includegraphics{aires_vol.5}&$\Eqalign{
% {\cal P}&=4c\cr
% {\cal A}&=\dfrac{d\times D}{2}\cr
% }$\\
% \hline
% \textbf{ Rectangle} de longueur $L$ et de largeur $l$&\includegraphics{aires_vol.2}&$\Eqalign{
% {\cal P}&=2(l+L)\cr
% {\cal A}&=L\times l\cr
% }$\\
% \hline
% \textbf{ Carré} de côté $c$&\includegraphics{aires_vol.1}&$\Eqalign{
% {\cal P}&=4c\cr
% {\cal A}&=c^2\cr
% }$\\
% \hline
% \end{tabular}
% \end{center}

\subsection{Triangles}
% \begin{center}
% \begin{tabular}{|m{8cm}|m{4.25cm}|c|}
% \hline
% \multicolumn{1}{|c|}{\bf Nom de la figure}&\multicolumn{1}{c|}{\bf Représentation}&\multicolumn{1}{c|}{\bf P\'erimètre et aire}\\
% \hline
% \textbf{ Triangle} de côté $c$ et de hauteur relative à ce côté $h$&\includegraphics{aires_vol.6}&$\Eqalign{
% {\cal P}&=\mbox{somme des côtés}\cr
% {\cal A}&=\dfrac{c\times h}{2}\cr
% }$\\
% \hline
% \end{tabular}
% \end{center}

\subsection{Disques}
% \begin{center}
% \begin{tabular}{|m{8cm}|m{4.25cm}|c|}
% \hline
% \multicolumn{1}{|c|}{\bf Nom de la figure}&\multicolumn{1}{c|}{\bf Représentation}&\multicolumn{1}{c|}{\bf P\'erimètre et aire}\\
% \hline
% \textbf{ Cercle et disque} de rayon $r$&\includegraphics{aires_vol.7}&$\Eqalign{
% {\cal P}&=2\pi r\cr
% {\cal A}&=\pi r^2\cr
% }$\\
% \hline
% \end{tabular}
% \end{center}