\begin{activite}[Curvica]
    {\bf Objectif :} différencier aire et périmètre ; comparer des périmètres ; comparer des aires.
       \partie[présentation]
          À partir d’un carré, on obtient une pièce du puzzle Curvica en \og creusant \fg, en \og bombant \fg{} ou en \og laissant droit \fg{} les côtés. Par exemple, voici une pièce de Curvica : \\
          \begin{center}
            \hspace*{-10mm}
            \curvica{}
             \begin{pspicture}(0,-0.25)(2,2.25)
                \rput(1,1){$\Longrightarrow$}
             \end{pspicture}
             \curvica{
                \psline(0,0)(2,0)(2,2)
                \psarc(1,4){2.24}{-116.6}{-63.4}
                \psarc(2,1){2.24}{153.4}{-153.4}}
             \begin{pspicture}(-1.5,-0.25)(2.5,2.25)
                \rput(1,1.75){\it\small tous les arcs de cercles}
                \rput(1,1.25){\it\small (creusés et bombés) reliant}
                \rput(1,0.75){\it\small deux sommets du carré}
                \rput(1,0.25){\it\small sont superposables.} 
             \end{pspicture}
          \end{center}
          \smallskip
       \partie[défis]
          Construire deux pièces différentes,\\          
          non superposables par rotation, déplacement ou retournement, telles que : \\ [5mm]
          \parbox{5cm}{les deux pièces ont la même aire mais des périmètres différents}\parbox{1.5cm}{\phantom{}}\parbox{5cm}{\curvica{}}\parbox{5cm}{\curvica{}} \\ [10mm]
          \parbox{5cm}{les deux pièces ont le même périmètre mais des aires différentes.}\parbox{1.5cm}{\phantom{}}\parbox{5cm}{\curvica{}}\parbox{5cm}{\curvica{}} \\ [10mm]
          \parbox{5cm}{les deux pièces ont le même périmètre et la même aire} \parbox{1.5cm}{\phantom{}}\parbox{5cm}{\curvica{}}\parbox{5cm}{\curvica{}} \\ [5mm] 
    \vfill\hfill {\footnotesize\it Source : Yves Martin. Curvica - activités mathématiques ludiques. 2015, pp.75. hal-01502901}
 \end{activite}