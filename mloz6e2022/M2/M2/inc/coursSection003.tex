\section{Unités de mesure d'aires}

\begin{methode}[Convertir des longueurs]
    Pour convertir des longueurs, on divise ou on multiple par \num{100},\num{10 000} \dots{}\\
    Pour un terrain, des terres agricoles ou forestières, on utilise des \textbf{unités de mesure agraires}.
    \begin{center}
        \textbf{Tableau de conversion}
        \begin{small}
            {\renewcommand{\arraystretch}{1.3}        
            \begin{longtable}{|*{2}{>{\centering\arraybackslash}m{0.07\textwidth}|}>{\centering\arraybackslash}m{0.1\textwidth}|*{2}{>{\centering\arraybackslash}m{0.11\textwidth}||}*{3}{>{\centering\arraybackslash}m{0.0966\textwidth}|}}
                \hline
                \multicolumn{1}{|c|}{}&\multicolumn{3}{c||}{\textbf{Multiples de l'unité}} & \multicolumn{1}{c||}{\textbf{Unité}} & \multicolumn{3}{c|}{\textbf{Sous-multiples de l'unité}} \\ \hline
                Unités d'aire& \Aire[km]{} & \Aire[hm]{} & \Aire[dam]{} & \Aire[m]{} & \Aire[dm]{} & \Aire[cm]{} & \Aire[mm]{} \\ \hline
                Unités agraires&$\bigotimes$ & \Aire[ha]{1} & \Aire[a]{1} & \Aire[ca]{1} & $\bigotimes$ &$\bigotimes$  &$\bigotimes$  \\  \hline
                Valeur en \Aire[m]{}& $\bigotimes$ & \Aire[m]{10000} & \Aire[m]{100} & \Aire[m]{1} & \Aire[km]{0.01} & $\bigotimes$& $\bigotimes$\\ \hline
                \rule[-1.2em]{0pt}{3em}$\bigotimes$ & \rule[-1.2em]{0pt}{3em}$\bigotimes$ & \rule[-1.2em]{0pt}{3em}$\dfrac{1}{100}$ \Aire[km]{} & \rule[-1.2em]{0pt}{3em} $\dfrac{1}{100}$ \Aire[hm]{} & \rule[-1.2em]{0pt}{3em} $\dfrac{1}{100}$ \Aire[dam]{} & \rule[-1.2em]{0pt}{3em} $\dfrac{1}{100}$ \Aire[m]{} & \rule[-1.2em]{0pt}{3em} $\dfrac{1}{100}$ \Aire[dm]{} & \rule[-1.2em]{0pt}{3em} $\dfrac{1}{100}$ \Aire[cm]{}\\ \hline
            \end{longtable}
            }
        \end{small}
    \end{center}
    \exercice
    \begin{enumerate}
        \item Convertir \Aire[dam]{53} en \Aire[m]{}.
        \item Convertir \Aire[dm]{5} en \Aire[m]{}.
        \item Convertir \Aire[ha]{7.81} en \Aire[m]{}.
        \item Convertir \Aire[a]{8.5} en \Aire[m]{}.
        \item Convertir \Aire[ha]{17} en \Aire[a]{}.
    \end{enumerate}
    \correction
    \begin{enumerate}
        \item \Aire[dam]{1} = \Aire[m]{100} donc \Aire[dam]{53} = $53 \times 100$ \Aire[m]{} = \Aire[m]{5300}
        \item \Aire[dm]{1}  = \Aire[m]{0.01} donc \Aire[dm]{5} = $5 \div 100$ \Aire[m]{} = \Aire[m]{0.05}
        \item \Aire[ha]{1}  = \Aire[m]{10000} donc \Aire[ha]{7.81} = $7,81 \times \num{10000}$ \Aire[m]{} \\ soit \Aire[ha]{7.81} = $\num{78100}$ \Aire[m]{}
        \item \Aire[a]{1}   = \Aire[m]{100} donc \Aire[a]{8.5} = $\num{8.5} \times 100$ \Aire[m]{} = $850$ \Aire[m]{}
        \item \Aire[ha]{1}  = \Aire[a]{100} donc \Aire[ha]{17} = $17 \times 100$ \Aire[a]{} = \Aire[a]{17000}
    \end{enumerate}
\end{methode}

On peut utiliser un tableau plus simple selon la situation.\\
Pour désigner une aire, on utilise le mètre carré (\Aire[m]{}) et ses multiples et sous-multiples.\\
Pour les mesures agraires, on utilise l'are (a) qui vaut \Aire[m]{100}. L'hectare (ha) vaut 100 ares, soit \Aire[m]{10000}.

\begin{center}
   \renewcommand{\arraystretch}{1}
   \begin{tabularx}{0.8\linewidth}{|*{14}{X|}}
      \hline
      \rowcolor{FondTableaux} \multicolumn{2}{|c|}{\Aire[km]{}} & \multicolumn{2}{c|}{\Aire[hm]{}} & \multicolumn{2}{c|}{\Aire[dam]{}} & \multicolumn{2}{|c||}{\Aire[m]{}} & \multicolumn{2}{c|}{\Aire[dm]{}} & \multicolumn{2}{c|}{\Aire[cm]{}} & \multicolumn{2}{c|}{\Aire[mm]{}} \\
      \hline
      & & & & & $3$ & $7$ & \multicolumn{1}{>{\centering\arraybackslash}p{0.5cm}||}{$0$} & $1$ & $5$ & $0$ & $4$ & & \\
      \hline
   \end{tabularx}
\end{center}

\begin{exemple*1}
    \Aire[m]{370.1504} = \Aire[dm]{37015.04} = \Aire[mm]{370150400} = \Aire[dam]{3.701504}.
\end{exemple*1}
 