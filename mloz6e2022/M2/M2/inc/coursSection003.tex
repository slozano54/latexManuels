\section{Unités de mesure d'aires}

Pour désigner une aire, on utilise le mètre carré (\umq{}) et ses multiples et sous-multiples. Pour les mesures agraires, on utilise l'are (a) qui équivaut à \umq{100} et l'hectare (ha) qui vaut 100 ares, c'est-à-dire \umq{10000}.

\begin{center}
   \renewcommand{\arraystretch}{1}
   \begin{tabularx}{0.8\linewidth}{|*{14}{X|}}
      \hline
      \rowcolor{FondTableaux} \multicolumn{2}{|c|}{\ukmq{}} & \multicolumn{2}{c|}{\uhmq{}} & \multicolumn{2}{c|}{\udamq{}} & \multicolumn{2}{c||}{\umq{}} & \multicolumn{2}{c|}{\udmq{}} & \multicolumn{2}{c|}{\ucmq{}} & \multicolumn{2}{c|}{\ummq{}} \\
      \hline
      & & & & & $3$ & $7$ & \multicolumn{1}{>{\centering\arraybackslash}p{0.5cm}||}{$0$} & $1$ & $5$ & $0$ & $4$ & & \\
      \hline
   \end{tabularx}
\end{center}

\begin{exemple*1}
   \umq{370,1504} = \udmq{37015,04} = \ummq{370150400} = \udamq{3,701504}.
\end{exemple*1}



\begin{methode*1}[Convertir des longueurs]
Pour convertir des longueurs, on effectue des multiplications ou des divisions par\\ \num{100},\num{10 000} \dots{}\\
Pour la surface d'un terrain, de terres agricoles ou forestières, on utilise des \textbf{unités de mesure agraires}.

\begin{center}
    \textbf{Tableau de conversion}\par\vspace{5mm}
    \begin{small}
        \begin{tabular}{|>{\centering}p{1.5cm}|>{\centering}p{1cm}|>{\centering}p{2cm}|>{\centering}p{2cm}||>{\centering}p{2cm}||>{\centering}p{1.5cm}|>{\centering}p{1.5cm}|>{\centering}p{1.5cm}|} \hline
            \multicolumn{1}{|c|}{}&\multicolumn{3}{c||}{\textbf{Multiples de l'unité}} & \multicolumn{1}{c||}{\textbf{Unité}} & \multicolumn{3}{c|}{\textbf{Sous-multiples de l'unité}} \\ \hline
            &&&&&&&\cr
            Unités d'aire& $km^2$ & $hm^2$ & $dam^2$ & $m^2$ & $dm^2$ & $cm^2$ & $mm^2$ \cr \hline
            Unités agraires&$\bigotimes$ & 1 ha & 1 a & 1 ca & $\bigotimes$ &$\bigotimes$  &$\bigotimes$  \cr  \hline
            &&&&&&&\cr
            Valeur en $m^2$& $\bigotimes$ & 10 000 $m^2$ & 100 $m^2$ & 1 $m^2$ & 0,01 $m^2$ & $\bigotimes$& $\bigotimes$\cr \hline
            &&&&&&&\cr
            $\bigotimes$&$\bigotimes$& $\dfrac{1}{100}$ $km^2$ & $\dfrac{1}{100}$ $hm^2$ & $\dfrac{1}{100}$ $dam^2$ & $\dfrac{1}{100}$ $m^2$ & $\dfrac{1}{100}$ $dm^2$ & $\dfrac{1}{100}$ $cm^2$ \cr 
            &&&&&&&\cr \hline
        \end{tabular}
    \end{small}
\end{center}
\exercice
\begin{enumerate}
    \item Convertir \Aire[dam]{53} en \Aire[m]{}.
    \item Convertir \Aire[dm]{5} en \Aire[m]{}.
    \item Convertir \Aire[ha]{7.81} en \Aire[m]{}.
    \item Convertir \Aire[a]{8.5} en \Aire[m]{}.
    \item Convertir \Aire[ha]{17} en \Aire[a]{}.
\end{enumerate}
\correction
\begin{enumerate}
    \item \Aire[dam]{1} = \Aire[m]{100} donc \Aire[dam]{53} = $53 \times 100$ \Aire[m]{} = \Aire[m]{5300}
    \item \Aire[dm]{1}  = \Aire[m]{0.01} donc \Aire[dm]{5} = $5 \div 100$ \Aire[m]{} = \Aire[m]{0.05}
    \item \Aire[ha]{1}  = \Aire[m]{10000} donc \Aire[ha]{7.81} = $7,81 \times \num{10000}$ \Aire[m]{} = $\num{78100}$ \Aire[m]{}
    \item \Aire[a]{1}   = \Aire[m]{100} donc \Aire[a]{8.5} = $\num{8.5} \times 100$ \Aire[m]{} = $850$ \Aire[m]{}
    \item \Aire[ha]{1}  = \Aire[a]{100} donc \Aire[ha]{17} = $17 \times 100$ \Aire[a]{} = \Aire[a]{17000}
\end{enumerate}

\end{methode*1}