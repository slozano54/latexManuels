\vspace*{-20mm}
\begin{activite}[Quelle opération ?]
    \vspace*{-5mm}
    \begin{changemargin}{-10mm}{-15mm}
        {\bf Objectifs :} poser une addition, une soustraction, une multiplication, une division euclidienne et une division décimale ; résoudre un problème en choisissant la bonne opération. 
        Pour chacun des problèmes suivants, indiquer quelle opération est en jeu, effectuer l'opération puis conclure. \medskip
        \begin{center}
           {\small
           \begin{ltableau}{\linewidth}{3}
              \hline
              Énoncé & Opération & Conclusion \\ [2mm]
              \hline
              Harry Potter a acheté 8 paquets de chocogrenouilles à 180 noises le paquet. 
              \newline Combien a-t-il payé ? & & \\ [3.1cm]
              \hline
              Severus Rogue dispose de \Capa[L]{180} de philtre de confusion.
              \newline Combien de chaudrons de \Capa[L]{8} chacun peut-il faire ? & & \\ [3.1cm]
              \hline
              Voldemore mesure \Lg[cm]{180} soit \Lg[cm]{8} de plus que Dumbledore. 
              \newline Quelle est la taille de Dumbledore ? & & \\ [3.1cm]
              \hline
              Un bâton de cerisier mesure \Lg[cm]{180}. Gilderoy Lockhart découpe 8 morceaux de ce bâton pour se faire des baguettes.
              \newline Combien mesure chaque baguette ? & & \\ [3.1cm]
              \hline
              La maison Gryffondor vient de perdre 8 points suite à une mauvaise réponse en divination et elle en a maintenant 180.
              \newline  Combien avait-elle de points avant ce cours ? & & \\ [3.1cm]
              \hline
           \end{ltableau}}
           \bigskip
        \end{center}
    \end{changemargin}
\end{activite}
\vspace*{-50mm}