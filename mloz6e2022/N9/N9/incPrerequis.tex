\vspace*{-7mm}
\begin{changemargin}{-10mm}{-10mm}
%pre-001
\begin{prerequis}[Connaisances \emoji{red-heart} et compétences \emoji{diamond-suit} du cycle 3]    
   \begin{itemize}        
       \item[\emoji{red-heart}] Vocabulaire associé à ces objets et à leurs propriétés : côté, sommet, angle, hauteur.
       \columnbreak
       \item[\emoji{diamond-suit}] Reconnaître, nommer, décrire des triangles, dont les triangles particuliers (triangle rectangle, triangle isocèle, triangle équilatéral).       
   \end{itemize}
\end{prerequis}
\end{changemargin}
\vspace*{-13mm}
 \begin{debat}[La division euclidienne] 
   Le nom de {\bf division euclidienne} est un hommage rendu à {\it Euclide} (300 av. J.-C.), mathématicien grec qui en explique le principe par soustractions successives dans son \oe uvre {\it Les éléments}. Mais elle apparait très tôt dans l'histoire des mathématiques, par exemple dans les mathématiques égyptiennes, babyloniennes et chinoises.
   \begin{center}
      \begin{pspicture}(0,1)(4,4.5)
         \psline[linewidth=1mm](2,1)(2,4)
         \psline[linewidth=1mm](2,3)(4,3)
         \textcolor{B1}{\it\large
         \rput(0.8,3.5){dividende}
         \rput(3,3.5){diviseur}
         \rput(3,2.5){quotient}
         \rput(1,1.5){reste}}
      \end{pspicture}
   \end{center}
   \bigskip
   \begin{changemargin}{-15mm}{-15mm}
   \begin{cadre}[B2][F4]
      \begin{center}
         \hrefVideo{https://www.yout-ube.com/watch?v=VWS9NyXbEyY&t=18s}{\bf Division euclidienne avec matériel multibase}, chaîne YouTube {\it Méthode Heuristique}.
      \end{center}
   \end{cadre}
   \end{changemargin}
\end{debat}