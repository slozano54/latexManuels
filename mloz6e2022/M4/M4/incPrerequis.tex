\vspace*{-8mm}
%pre-001
\begin{prerequis}[Connaisances \emoji{red-heart} et compétences \emoji{diamond-suit} du cycle 3]    
   \begin{itemize}        
       \item[\emoji{red-heart}] Vocabulaire associé à ces objets et à leurs propriétés : côté, sommet, angle, hauteur.
       \columnbreak
       \item[\emoji{diamond-suit}] Reconnaître, nommer, décrire des triangles, dont les triangles particuliers (triangle rectangle, triangle isocèle, triangle équilatéral).       
   \end{itemize}
\end{prerequis}
\vspace*{-5mm}
\begin{debat}[Débat : ce fabuleux nombre $\pi$] 
    $\pi$ est un nombre, au même titre que 2 ou 100, ou 6,538. Il désigne le rapport du périmètre d'un cercle par son diamètre. Il a fasciné de nombreux savants depuis l'antiquité et il existe des livres entiers qui lui sont consacrés !
    \begin{center}
        \spiraleArchimedePi[2]
    \end{center}
    \begin{cadre}[B2][F4]
          \hrefVideo{https://leblob.fr/fondamental/le-nombre-pi}{\bf Le nombre Pi}, site {\it Le Blob}, {\it Petits contes mathématiques}.
          
          \smallskip
          \hrefVideo{https://www.mypiday.com/}{Sa date de naissance dans les décimales de $\pi$.}
    \end{cadre}
 \end{debat}