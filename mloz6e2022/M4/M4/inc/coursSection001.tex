\section{Périmètre}
\begin{definition}
    Le \textbf{périmètre} d'une figure, c'est la \textbf{longueur} de son contour.
\end{definition}

\begin{remarque}
    Un périmètre s'exprime avec une \textbf{unité de longueur}.
\end{remarque}

\begin{methode}[Convertir des longueurs]
    \begin{changemargin}{0mm}{-15mm}
        Pour convertir des longueurs, on effectue des multiplications ou des divisions par 10,100,1 000 ...\\
        \begin{center}
            \textbf{Tableau de conversion}\par\vspace{5mm}
            \begin{tabular}{|>{\centering}p{0.1\linewidth}|>{\centering}p{0.1\linewidth}|>{\centering}p{0.1\linewidth}||>{\centering}p{0.1\linewidth}||>{\centering}p{0.1\linewidth}|>{\centering}p{0.1\linewidth}|>{\centering}p{0.1\linewidth}|} \hline
                \multicolumn{3}{|c||}{\textbf{Multiples de l'unité}} & \multicolumn{1}{c||}{\textbf{Unité}} & \multicolumn{3}{c|}{\textbf{Sous-multiples de l'unité}} \\ \hline
                \Lg[km]{}   & \Lg[hm]{}             & \Lg[dam]{}           & \Lg[m]{}            & \Lg[dm]{}            & \Lg[cm]{}             & \Lg[mm]{}     \cr \hline
                \Lg[km]{1}  & \Lg[hm]{1}            & \Lg[dam]{1}          & \Lg[m]{1}           & \Lg[dm]{1}           & \Lg[cm]{1}            & \Lg[mm]{1}    \cr
                &&&&&& \cr
                \Lg[hm]{10} & \Lg[dam]{10}          & \Lg[m]{10}           & \Lg[dm]{10}         & \Lg[cm]{10}          & \Lg[mm]{10}           &  $\bigotimes$ \cr 
                &&&&&& \cr
                $\bigotimes$& $\dfrac{1}{10}$ km    & $\dfrac{1}{10}$ hm   & $\dfrac{1}{10}$ dam & $\dfrac{1}{10}$ m    & $\dfrac{1}{10}$ dm    & $\dfrac{1}{10}$ cm \cr 
                &&&&&& \cr \hline
            \end{tabular}
        \end{center}
    \end{changemargin}
\exercice
    Convertir :
    \begin{itemize}
        \item \Lg{43.5} en \Lg[mm]{};
        \item \Lg{21500} en \Lg[m]{}.
    \end{itemize}
\correction
    \begin{itemize}
        \item $1$ cm = $10$ mm donc $43,5$ cm = $43,5 \times 10$ mm = $435$ mm
        \item $1$ cm = $\dfrac{1}{10}$ m donc $21\:500$ cm = $\dfrac{21\:500}{100}$ m = $215$ m
    \end{itemize}
\end{methode}


