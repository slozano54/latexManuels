\begin{exercice*}
    Calculer le périmètre des figures suivantes. \\ [1mm]
    {\psset{unit=0.5}
    \small
    \begin{pspicture}(-3,-0.5)(5,4)
       \psframe(0,0)(4,3)
       \psframe(0,0)(0.3,0.3)
       \rput(2,0){x}
       \rput(2,3){x}
       \rput(0,1.5){=}
       \rput(4,1.5){=}
       \rput(2,-0.5){\Lg[m]{6}}
       \rput{90}(-0.5,1.5){\Lg[m]{4}}
    \end{pspicture}
    \begin{pspicture}(-1,0)(4,4.5)
       \psframe(0,0)(4,4)
       \psframe(0,0)(0.3,0.3)
       \rput(2,0){x}
       \rput(2,4){x}
       \rput(0,2){x}
       \rput(4,2){x}
       \rput(2,-0.5){\Lg[cm]{27}}
    \end{pspicture}
    
    \dotfill

    \dotfill

    \dotfill

    \dotfill

    \begin{pspicture}(-1,0)(4,6)
       \pspolygon(1,1)(7,1)(2,5)
       \psline(2,1)(2,5)
       \psframe(2,1)(2.3,1.3)
       \rput{-40}(5,3.3){\Lg[cm]{6.9}}
       \rput{78}(1,3){\Lg[cm]{4.6}}
       \rput(3.5,0.5){\Lg[cm]{6.6}}
       \rput{90}(2.5,2.7){\Lg[cm]{4.4}}
    \end{pspicture}
    \begin{pspicture}(4,1)(14,6)
       \pspolygon(8,2)(14,2)(8,5)
       \rput(11,1.5){\Lg[mm]{83}}
       \rput{90}(7.5,3.5){\Lg[mm]{25}}
       \rput{-30}(11,4){\Lg[mm]{86.68}}
       \psframe(8,2)(8.3,2.3)
    \end{pspicture}}

    \dotfill

    \dotfill

    \dotfill

    \dotfill    
 \end{exercice*}