\section{Circonférence d'un cercle}
\begin{propriete}
    Le \textbf{périmètre} d'un cercle est proportionnel à la longueur de son rayon.

    Il est égal au produit du double de son rayon par $\pi$.

    Il revient au même de calculer le produit de son diamètre par $\pi$.

    $${\mathcal P} = 2\times \text{rayon} \times \pi = \text{diamètre}\times \pi$$
\end{propriete}

\begin{myvocabulaire}
    On parle indistinctement de {\bfseries périmètre}, de \textbf{longueur} ou de \textbf{circonférence} du cercle.
\end{myvocabulaire}

\begin{remarques}
    \begin{itemize}
        \item Le nombre $\pi$ n'est pas un nombre décimal, il a une infinité de chiffres après la virgule mais aucune séquence de chiffres n'est périodique.
        \item On utilise souvent \num{3.14} comme valeur approchée pour le nombre $\pi$.
        \item On peut aussi utiliser la touche \Calculatrice{/$\pi$} de la calculatrice.
    \end{itemize}
\end{remarques}

\begin{myBox}{\emoji{light-bulb} Moyen mnémotechnique \emoji{light-bulb}}
    Si on note $R$ le rayon du cercle, que l'on simplifie l'expression littérale en réordonnant ses facteurs :
    $$2\times R \times \pi = 2 \times \pi \times R = 2\pi R$$
    L'expression $2\pi R$ se lit alors \og{} Deux pierres \fg{}.
\end{myBox}

{\renewcommand*{\arraystretch}{1.5}
\begin{longtable}{|m{0.2\textwidth}|>{\centering\arraybackslash}m{0.25\textwidth}|m{0.15\textwidth}|>{\centering\arraybackslash}m{0.325\textwidth}|}%
  \hline
  \rowcolor{gray!20}\multicolumn{1}{|c|}{\bf Nom de la figure}&{\bf Représentation}&\multicolumn{1}{|c|}{\bf Périmètre}&{\bf Exemple}\\
  \hline
  \textbf{ Cercle} de rayon $R$&        
  \begin{Geometrie}[CoinHD={u*(4,4)}]
      trace feuillet withcolor white;
      pair O,A;
      path co;
      O=u*(2,2);
      co=cercles(O,2*u);
      trace co;
      A=pointarc(co,60);
      trace cotationmil(O,A,0mm,5,btex $R$ etex) withcolor red;
  \end{Geometrie}        
  &$\Eqalign{
  {\mathcal P}&=2\times\pi\times R\cr
  {\mathcal P}&=2\pi R\cr
  }$&
  \begin{Geometrie}[CoinHD={u*(4,4.5)}]
    trace feuillet withcolor white;
    pair O,A;
    path co;
    O=u*(2,2);
    co=cercles(O,2*u);
    trace co;
    A=pointarc(co,60);
    trace cotationmil(O,A,0mm,20,btex $\Lg[km]{6400}$ etex) withcolor red;
\end{Geometrie}        
$\Eqalign{
    {\mathcal P}&=2\times\pi\times \Lg[km]{6400}\cr
    {\mathcal P}&\simeq2\times\num{3.14}\times \Lg[km]{6400}\cr
    {\mathcal P}&\simeq\Lg[km]{40192}\cr
}$\\
  \hline
\end{longtable}
}
